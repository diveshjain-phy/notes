\documentclass[main.tex]{subfiles}
\begin{document}

\section{Coordinate astronomiche}

\marginpar{Monday\\ 2020-5-11, \\ compiled \\ \today}

Vogliamo definire un sistema di coordinate sulla sfera celeste. Innanzitutto, specie se vogliamo adottare un sistema di tipo sferico ma anche in linea generale dobbiamo scegliere un centro.
Una possibile scelta è di adottare l'osservatore (\emph{sfera topocentrica}), ma spesso risulta più utile scegliere il centro della Terra (\emph{sfera geocentrica}), in modo tale che sia più semplice confrontare le misure.

Alternativamente, possiamo utilizzare il centro del Sole oppure il baricentro del Sistema Solare. 
Questo tipo di traslazioni creano grandi differenze nella trattazione dei fenomeni del Sistema Solare, mentre sono poco influenti per quanto riguarda i fenomeni al di fuori di esso. 

Inoltre, dobbiamo scegliere un \textbf{asse}. Da ogni località la volta celeste ci pare ruotare attorno ad un asse, le stelle sorgono a Est e tramontano a Ovest ruotando per cerchi paralleli, che si possono descrivere come intersezioni fra una sfera e un piano. 

\subsection{Sistema altazimutale}

Il piano di riferimento è quello dell'orizzonte; la direzione di riferimento è la verticale, lo \emph{zenit} è il ``punto'' ove questa retta interseca la sfera celeste, \emph{nadir} è il suo opposto.
Si chiamano \emph{almucantarat} i paralleli di altezza --- cerchi di ``alzata'' costante. 

Le coordinate che usiamo sono l'\emph{azimut}, l'angolo misurato da un punto fissato (ad esempio Nord o Sud) in senso orario lungo l'orizzonte, e l'\emph{altezza} \(h\), distanza angolare orizzonte-stella, oppure la \emph{distanza zenitale} \(z = \pi/2 - h\).

Entrambe le coordinate di un oggetto celeste sono variabili nel tempo e nello spazio. 

\subsection{Sistema orario}

Consideriamo come piano quello definito dall'equatore terrestre e come asse quello di rotazione della Terra. 
Guardando verso Sud possiamo definire il ``mezzocielo'': il punto in cui il meridiano passante per lo Zenit taglia l'equatore celeste. 

Le nostre coordinate ora sono: l'\emph{angolo orario} \(HA\), misurato generalmente verso ovest lungo l'equatore celeste a partire dal mezzocielo fino alla proiezione della stella sull'equatore celeste; la \emph{declinazione} \(\delta \), ovvero l'alzata rispetto all'equatore celeste. 

L'angolo orario si misura in ore, minuti e secondi.

Nel suo moto annuale, il Sole percorre un cerchio massimo detto \emph{eclittica}, che è inclinato di \(\epsilon \approx \SI{23}{\degree} \SI{27}{\prime}\) rispetto al piano equatoriale. 
Se ci mettiamo in un riferimento \emph{equatoriale}, ovvero concorde con le stelle fisse, vediamo proprio questo moto. 
Le intersezioni fra eclittica e piano equatoriale sono gli \emph{equinozi}, mentre i punti di massima declinazione sono i \emph{solstizi}.

Possiamo individuare l'eclittica a partire dalla posizione dei pianeti, oltre che del Sole. 

Definiamo \emph{coluri} i cerchi massimi che passano per i poli della sfera celeste e per gli equinozi o per i solstizi.

Il \emph{polo eclitticale} corrisponde alla normale all'eclittica passante per il centro della Terra. Cade nella costellazione del Draco. 

\subsection{Sistema equatoriale}

Le coordinate sono le stesse del sistema orario, ma facciamo girare il riferimento come riferimento delle stelle. 
Scegliamo un punto sul cerchio, corrispondente al coluro vernale (equinozio d'autunno), e da lì misuriamo la \emph{ascensione retta} \(\alpha \) verso Est (inverso rispetto alla HA).

\subsection{Sistema eclitticale}

Simile al sistema equatoriale, ma si utilizza come asse e piano quelli dell'eclittica: le coordinate sono longitudine e latitudine eclitticali. 
Queste coordinate sono quasi fissate, ma per la precessione degli equinozi il punto di riferimento precede nel tempo.

\subsection{Tempo siderale}

È l'\emph{angolo orario} HA del punto \(\gamma \), ovvero dell'equinozio vernale (d'autunno).
Varia nel tempo, in quanto la Terra gira. Possiamo scrivere: 
%
\begin{align}
\alpha = ST - HA
\,,
\end{align}
%
dove \(ST\) è il tempo siderale.

Il passaggio del tempo siderale non è uniforme, a causa delle disuniformità della rotazione della Terra. 
Il punto \(\gamma \) inoltre cambia l'orientazione del piano equatoriale rispetto a quello dell'eclittica. 

Dobbiamo applicare delle correzioni alle coordinate equatoriali: partiamo da un'epoca fissata, ad esempio J2000, e tenendo conto dei varii moti di precessione a lungo periodo otteniamo le \emph{coordinate medie}. Successivamente, se includiamo anche i moti a corto periodo otteniamo le \emph{coordinate vere}. 
Con ulteriori correzioni per altri piccoli fenomeni otteniamo le \emph{coordinate apparenti}.

I cataloghi \emph{fondmentali} danno le posizioni delle stelle rispetto ad equatore ed eclittica (definita dal Sole).
Ora si utilizza l'FK6. Altri cataloghi possono essere \emph{differenziali}. 

Altri sistemi come l'ICRF utilizzano la radioastronomia. Questa è l'\textbf{astrometria}.

Sistema \emph{galattico}: le coordinate sono longitudine galattica \(l\) e latitudine galattica \(b\). Non si utilizza queste coordinate per misure di precisione.

Si può definire meglio il sistema galattico utilizzando la riga a \SI{21}{cm} dell'idrogeno che caratterizza le nubi di polvere.

\todo[inline]{Non possiamo definire il sistema galattico con shift fissato arbitrariamente da quello equatoriale? In questo modo non ci sarebbero problemi di precisione.}

Ci sono nuovi sistemi di riferimento in GR: Baricentric e Geocentric Celestial Reference System, orientati sempre come ICRF e centrati sul baricentro del Sistema Solare o nella Terra.
La cosa bella è che ci sono le espressioni del tensore metrico nei riferimenti. 

\todo[inline]{Ma cioè, c'è il tensore metrico numerico? È uno Schwarzschild naïf o come? Di cosa tiene conto?}

Nell'emisfero Nord, se la latitudine è \(\phi \), una stella di declinazione \(\delta \) è circumpolare se \(\delta \geq \pi /2 - \phi \).

Con un po' di conti si può mostrare che \(\sin(\alpha _{\odot}) = \tan(\alpha _{\odot}) / \tan( \epsilon )\), e poi \(ST = \alpha _{\odot} + HA _{\odot}\).

\end{document}