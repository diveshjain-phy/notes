\documentclass[main.tex]{subfiles}
\begin{document}

This course is in italian. 

\marginpar{Tuesday\\ 2020-4-7, \\ compiled \\ \today}

\section{Lezione congiunta Cremonesi--Ghilardi} 

Ora l'astronomia è uniformata nel mondo, ma la sua storia dipende dalla regione. 

Ogni descrizione del mondo dipende dal modo in cui noi ci poniamo le domande, da che tipo di ricerca svolgiamo. 

Rapporto \emph{mythos}-\emph{logos}. 
Scienza contemporanea come \emph{mythos} contemporaneo. 

L'incontro interculturale permette di uscire dal \emph{mythos} della nostra società. 

Osservazione della CMB e del Big Bang: ``retrocessione dell'osservatore''.
Sto condizionando l'origine\dots?

La scienza moderna è più efficace del mito antico ma non più vera. 

Verità: costruzione generale di senso, distinta dall'aspetto predittivo. 

Foucault: \emph{L'ermeneutica del soggetto} \cite[]{foucaultErmeneuticaSoggetto1982}. 
Verità.

Galileo è alla soglia fra l'approccio osservativo moderno e l'astrologia.

Idea dello ``sfatare i miti''.
% \todo[inline]{Se vogliamo date un altro significato alla parola ``mito'', ha senso discutere su utilizzi comuni di questo termine?}

Sempre verità vs mito, sfatare i miti con ``non ci sono prove'' e affermazioni scientifiche. 
Scientifiche ora, storicamente anche religiose con il Cristianesimo. 

Bufale e falsi miti: ``falsi'' qui è utilizzato come rafforzativo. 

Mito, qui, non è semplicemente ``storia'', è invece un approccio intero alla realtà, un insieme di idee che forniscono un quadro interpretativo.

Le narrazioni ``mitologiche'' si pongono le stesse domande della scienza oppure no? Questo è cruciale. 
È \emph{efficace} utilizzare una narrazione scientifica in contrapposizione al complottismo? 
Il mito-complottismo si pone su un piano diverso? 
``Il mito è parola della crisi''.  

De Martino (chi è?) sul mito. 

La narrazione complottista crea un ``noi'' nel quale rifugiarsi. 
Il mito è ``narrazione fondante''. 

Il mito è vivo nel momento in cui vive in qualcuno, se è oggetto di uno studio storico-religioso è morto, una storia falsa.

Marcel Detienne: il mito è un sistema di pensiero che ingloba l'insieme dei racconti essenziali della società. 
Il mito \emph{mobilita}. 

J.-P. Vernant: il mito deve essere inserito in una cultura ben definita.

Tre domande:
\begin{enumerate}
    \item A cosa si conferisce l'etichetta di mito? Chi definisce ``mito'' le narrazioni di altri? Com'è che qualcosa viene definito ``sapere'' e qualcos'altro ``credenza''?
    \item Il mito esiste ``di per sè'' o è un oggetto culturale che si è prodotto storicamente sulla base della comparazione?
    \item Come si è costituito il mito come oggetto del pensiero scientifico? 
\end{enumerate}

``Superstizione'' scientifica nel senso di sovrastruttura. 

Bruno Latour. 

Retrocessione dell'osservatore.
Carlo Sini: ``Transito Verità'', si menziona questo, cosmologia dal Timeo di Platone. 
Idea già nel pensiero di Nietzsche: credendo di investigare le cose nella loro inseità proiettiamo le nostre categorie.

Come faccio a studiare l'altro se turbo il sistema che osservo nell'osservarlo?
L'evoluzione è parte di un processo evolutivo?

C'è una porosità fra mitologie e gruppi sociali.

\end{document}
