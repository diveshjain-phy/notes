\documentclass[main.tex]{subfiles}
\begin{document}

\marginpar{Thursday\\ 2020-4-9, \\ compiled \\ \today}

\textbf{(BSPNDG=basato su poche note di Giorgio)}

In generale alle stelle vengono associate figure su un piano superiore: dei-stelle in età arcaica e sovrani-stelle in età ellenistica.
Difficile demarcare la separazione astronomia/astrologia e tralaltro "Gli dei erano un modo di pensare il mondo per i Greci". Forti legami con politica (forza persuasiva delle narrazioni mitologiche), arte, letteratura, etc.
Nella trattazione mostreremo gli \textbf{aspetti ricorrenti} dei miti legati all'astronomia.
Innanzitutto identifichiamo due accezioni di "mito" per i Greci
1) Teodoreto di Cirro, \textit{Historia religiosa}: racconto degli asceti orientali, es. sullo stilita (un tipo che viveva in cima ad una colonna) dice che "quello che racconterò sarà veritiero". Questo è emblematico dell'importanza della comparazione mito=fabula(=mucchio di balle) con narrazione veritiera: dialettica vero-falso (anche in Agostino etc.).

2) Accezione religiosa successiva: mito=storia sacra. Infatti il mito come discorso può avere efficacia, ossia in grado di muovere "forze sacrali" negli uomini. Legame "culto"-"coltivare": gli dei nascono contestualmente al mondo fisico. Pertanto in questa accezione mito=parola inspirata (dalle Muse): il poeta così impara ciò che è stato, ciò che è e ciò che sarà grazie al mito. La parola poetica è scandita, così come il contenuto scandisce le attività umani, tenendo il tempo del susseguirsi delle stagioni etc.

Nei Veda dall'uomo primordiale nasce, diremmo noi, il cosmo. Il mito racconta di come le cose \textit{si sono stabilizzate}: mito è quindi "storia fondante", dove si racconta di un tempo in cui degli \textit{altri} protagonisti muovono eventi straordinari rispetto ad oggi, ma che conducono al presente per "come dev'essere".

Altra ambivalenza:
\begin{itemize}
    \item Da un lato la poliedricità di narrazioni, tante narrazioni, locali, dei miti
    \item Legami (parentele fra dei, citazioni di altri miti) fra i racconti
\end{itemize}
A volte il primo aspetto implica una "lotta" fra le varianti dei miti per la verità, che si concretizza, per esmpio, nelle rivendicazioni della paternità dei poeti: infatti non esiste un testo sacro come riferimento.

Si definisce l'Atlante magico (\textit{Zauberatlas}), che "acquista leggibilità soltanto nella letteratura". E' l'insieme di tutti i luoghi fantastici, il risultato di tutto il nostro patrimonio poetico-letterario personale.

Inoltre c'è un'interrelazione fra i miti di diverse culture (esempio quando si spiega l'antropomorfismo delle divinità egizie)
Il mito è perlopiù intressato a spiegare le \textit{origini} di fenomeni, attività culturali: l'esigenza di dar nome alle cose ed ordine.

\subsection{Esiodo ed Iliade}
Nella Teogonia il proemio è alle Muse, in cui il poeta prega loro di dargli un canto \textit{seducente}: altro connotativo del lavoro poetico.
Di nuovo la richiesta di mito come narrazione delle origini.
Due esempi: Pleiadi e Orione, ma nulla di concettuale.

Iliade: sguardo di chi osserva il cielo come un pastore e poi la descrizione dello scudo. L'Orsa invece, come vedremo domani, sarà molto importante in quanto una delle costellazioni più antiche.
\end{document}