\documentclass[main.tex]{subfiles}
\begin{document}

\section{Spherical top-hat model}

We start off with the discussion of curved models, which is the same as in the FAC course.

We consider an Einstein-de Sitter universe, and a sphere with \(\Omega_{p} (t_i) = 1 + \delta_{i}\).

The density profile is considered to be in the shape of a top-hat, which is flat in the center and at the boundaries, shaped like a plateau.
We want to have zero peculiar velocity at the beginning; pure Hubble flow.
Initially we have \(\delta = \delta_{+} + \delta_{-}\), and we must also have \( \frac{2}{3} \delta_{+} = \delta_{-}\).

We find 
%
\begin{align}
\rho_{P} (t_m) = \rho_{c} (t_i) \Omega_{P} (t_i) \qty[ \frac{\Omega_{P}(t_i) - 1}{\Omega_{P}(t_i)}]^3
\,,
\end{align}
%
where \(t_m\) is the turnaround time.

With these hypotheses, we find that the ratio of the inner to the outer density is around \(5.6\): 
%
\begin{align}
\frac{\rho_{P}}{\rho_{C}} = \qty( \frac{3 \pi }{4})^2
\,.
\end{align}

If we used linear theory instead of the correct theory we would have found \(2.07\) for the ratio: the result is very different. 

The total energy of our top-hat is given by 
%
\begin{align}
E _{\text{eq}} = - \frac{1}{2} \frac{3}{5} \frac{GM^2}{R _{\text{eq}}}
\,.
\end{align}

At turn-around \(T = 0\), so at that point 
%
\begin{align}
E_m = - \frac{3}{5} \frac{GM^2}{R_m}
\,,
\end{align}
%
and as long as the energy is conserved, we find that \(R_m = 2 R _{\text{eq}}\). 
This means that the volume at turnaround is 8 times that of equilibrium.

Therefore, we can find what the ratio of the perturbation density to the background one is of 180 at collapse.
If we still were to use linear theory at this time we would get \num{1.686}.

Now we introduce the \emph{mass function of cosmic structure}, which is defined as 
%
\begin{align}
n(M) = \dv{N}{M}
\,,
\end{align}
%
the number of objects per unit volume with mass between \(M \) and \(M + \dd{M}\). 

Then we take a filter, we consider the probability that at a certain point the perturbation's magnitude is larger than \num{1.686}, which is equivalent to saying that it has virialized. 

Then, the ansatz is 
%
\begin{align}
n(M) M \dd{M} = \rho_{m} \qty(\mathbb{P}_{> \delta_{c}} - \mathbb{P}_{ > \delta_{c}} (M + \dd{M}))
= \rho_{m} \abs{ \dv{\mathbb{P}_{> \delta_{c}}}{M}} \dd{M}
\,,
\end{align}
%
but the probability depends on the mass only through the variance of the distribution of density: then we write 
%
\begin{align}
n(M) M \dd{M} = \rho_{m} 
\abs{ \dv{\mathbb{P}_{> \delta_{c}}}{\sigma_{M}}}
\abs{ \dv{\sigma_{M}}{M}}
\dd{M}
\,,
\end{align}
%
and finally we get 
%
\begin{align}
n(M) = \frac{2}{\sqrt{\pi }} \frac{\rho_{m}}{M_*^2}  \alpha \qty( \frac{M}{M_{*}})^{\alpha -2} \exp( - \qty( \frac{M}{M_{*}})^{2\alpha } )
\,,
\end{align}
%
which is equivalent to the Schecter luminosity function, where \(\alpha = 1/2\) --- that is, with white noise.

\end{document}
