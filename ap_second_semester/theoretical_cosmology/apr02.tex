\documentclass[main.tex]{subfiles}
\begin{document}

\marginpar{Thursday \\ 2020-4-02, \\ compiled \\ \today}

On the Right Hand Side, instead, since the stress energy tensor is given by 
%
\begin{align}
T^{i}_{j} = \sum _{\text{all species \(a\)}} g_{a}
\int  \frac{ \dd[3]{p}}{(2 \pi )^3} 
\frac{p^{i}p_{j}}{E_{a}(p)} f_a (\vec{x}, \vec{p}, t)
\,
\end{align}
%
we obtain 
%
\begin{align}
\qty(\hat{k}_{i} \hat{k}^{j} - \frac{1}{3} \delta^{i}_{j})
\widetilde{T}^{i}_{j} 
= \sum _{a} g_{a} \int  \frac{\dd[3]{p}}{(2\pi )^3} \frac{p^2\mu^2 - p^2/3}{E_a (p)} f_a (\vec{p})
\,,
\end{align}
%
but we know that 
%
\begin{align}
\mu^2 - \frac{1}{3} = \frac{2}{3} P_2 (\mu )
\,,
\end{align}
%
where \(P_2\) is the second Legendre polynomial, so we get 
%
\begin{align}
\qty(\hat{k}_{i} \hat{k}^{j} - \frac{1}{3} \delta^{i}_{j})
\widetilde{T}^{i}_{j} &=
-2 \int_{0}^{ \infty } \frac{ \dd{p} p^2}{2\pi^2} \pdv{f^{(0)}}{p}p^2
\int_{-1}^{1} \frac{ \dd{\mu }}{2} \frac{2}{3} P_2 (\mu ) \theta (\mu )  \\
&= 2 \frac{2}{3} \widetilde{\theta}_{2} \int_{0}^{ \infty } \frac{ \dd{p}p^2}{2 \pi^2} p^2 \pdv{f^{(0)}}{p} = - \frac{8}{3} p_{\gamma } \widetilde{\theta}_{2}
\,,
\end{align}
%
so if we collect terms we find 
%
\begin{align} \label{eq:radiation-temperature-perturbation-term}
k^2 \qty(\Phi - \Psi ) &= - 32 \pi G a^2 
\underbrace{\qty[\rho_{\gamma } \widetilde{\theta}_{2} + \rho_{\nu } \widetilde{w}_{2}] }_{\rho_{r} \theta_{r, 2}} 
\,.
\end{align}

Since we have added the neutrino contribution, we are accounting for all of radiation. 

The Einstein 00 equation \eqref{eq:einstein-00-perturbed-equation} can be written in Fourier space as 
%
\begin{align}
k^2 \Phi + 3 \frac{\dot{a}}{a} \qty(\dot{\Psi} + \frac{\dot{a}}{a} \Phi )
= -4 \pi G a^2 \qty(\rho_{m } \delta_{m} + 4 \rho_{r} \widetilde{\theta}_{r, 0}) 
\,,
\end{align}
%
where 
%
\begin{align}
\rho_{m} \delta_{m} = \rho_{m} \delta + \rho_{b} \delta_{b} 
\qquad \text{and} \qquad
\rho_{r} \theta_{r, 0} = \rho_{\gamma } \theta + \rho_{\nu } w
\,.
\end{align}

Dark Matter and baryons do not contribute to the quadrupole (at first order). 

Recall that the EFE are not independent: we need another one, and we choose the \(^0_i\) one: those components of the stress-energy tensor are given by 
%
\begin{align}
T^{0}_{i} = \sum _{\alpha } a \rho_{\alpha } 
\int \frac{ \dd[3]{p}}{(2\pi )^3} p_{i} f_\alpha  (\vec{x}, \vec{p}, t)
\,,
\end{align}
%
so with reasoning similar to what was done before we find that the \(^0_{i}\) equation reads
%
\begin{align}
\dot{\Psi} + a H \Phi = - \frac{4 \pi G a^2}{ik}
\qty[\rho_{m} v_{m} - 4 i \rho_{r} \theta_{r, 1}]
\,,
\end{align}
%
where 
%
\begin{align}
\rho_{m} v_{m} = \rho _{\text{dm}} v + \rho_{b} v_{b}
\qquad \text{and} \qquad
\rho_{r} \theta_{r, 1} = \rho_{g} \theta_1 + \rho_{\nu } w_{1}
\,.
\end{align}

Combining the \(00\) and \(0i\) equations we find 
%
\begin{align}
k^2 \Psi = -4\pi G a^2
\qty[\rho_{m} \delta_{m} + 4 \rho_{r} \theta_{r, 0}
+ \frac{32H}{k} \qty(i \rho_{m} v_m + 4 \rho_{r} \theta_{r, 1})]
\,.
\end{align}

\subsubsection{Tensor models}

Now that we have found the equation of motion of the scalar field, let us discuss tensor modes: we can find the equation of motion for these starting from the transverse traceless part of the \(^i_{j}\) equation. We set \(\chi_{ij} = 2 h_{ij}\), and with this we can write the traceless Christoffel symbols:
%
\begin{align}
\Gamma^{0}_{ij} &= \frac{1}{2} \dot{h}_{ij} + \frac{\dot{a}}{a} h_{ij}  \\
\Gamma^{i}_{0j} &= \frac{1}{2} \dot{h}  \\
\Gamma^{i}_{jk} &= h^{i}_{(j,k)} - \frac{1}{2} \tensor{h}{_{jk}^{,i}}  \\
\,,
\end{align}
%
which yield the traceless Einstein tensor 
%
\begin{align}
G^{i}_{j} &= \frac{\dot{a}}{a} \dot{h}^{i}_{j} 
- \frac{1}{2} \nabla^2 h^{i}_{j} + \frac{1}{2} \ddot{h}^{i}_{j}
\,.
\end{align}

In order to get the correct contribution on the RHS we need to project it: we apply the projection operator 
%
\begin{align}
\tensor{\mathcal{P}}{^{k}_{i}^{j}_{l}} =
\mathcal{P}^{k}_{i} \mathcal{P}^{j}_{l}
- \frac{1}{2} \mathcal{P}^{k}_{l} \mathcal{P}^{j}_{i}
\,,
\end{align}
%
where 
%
\begin{align}
\mathcal{P}^{i}_{j} = \delta^{i}_{j} - \hat{k}^{i}\hat{k}_{j}
\,.
\end{align}

\todo[inline]{The work \cite[]{carboneUnifiedTreatmentCosmological2005} is mentioned, but the work only the two-index \(\mathcal{P}\) is defined (in position space and not in momentum space, but it's the same) (eq. 13); but the four-index combination is not explicitly written\dots}

What is usually done is to completely neglect the right hand side of the EFE (since it is second order in \(v/c\)?), to get the homogeneous expression 
%
\begin{align}
\ddot{h}_{ij} + 2 \frac{\dot{a}}{a} h_{ij} - \nabla^2 h_{ij} = 0
\,.
\end{align}

We can express the tensor perturbation in Fourier space as 
%
\begin{align}
h_{ij} = \frac{1}{(2\pi )^3} \int \dd[3]{k} e^{- \vec{k} \cdot \vec{x}}
h_{ij} (\vec{k}, t)
\,,
\end{align}
%
where \(h_{ij} (\vec{k}, t)\) can be decomposed into the two basis polarizations \(+\) and \(\times \)
so that our equation will read 
%
\begin{align}
\ddot{h}_{\epsilon } + 2 \frac{\dot{a}}{a} \dot{h}_{\epsilon } + k^2 h_{\epsilon } = 16 \pi G \pi_{\epsilon }
\,,
\end{align}
%
where \(\epsilon = +, \times \) (see Pritchard \& Kamionkowski \cite[]{pritchardCosmicMicrowaveBackground2005}). 



\end{document}