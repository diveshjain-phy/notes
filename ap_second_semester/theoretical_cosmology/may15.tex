\documentclass[main.tex]{subfiles}
\begin{document}

\marginpar{Friday\\ 2020-5-15, \\ compiled \\ \today}

Yesterday we defined correlation functions starting from probabilities. 

We want to connect them to the density perturbations.

We define the mean density \(\expval{\rho } = n\). To order two then we can define 
%
\begin{align}
\xi (r) = \expval{\qty(\rho (x+r) - \expval{\rho }) \qty(\rho - \expval{\rho })} / n^2
\,,
\end{align}
%
which is defined so that 
%
\begin{align}
\expval{\rho (x +r) \rho (x)} = n^2 \qty(1 + \xi (r))
\,.
\end{align}

Then we define 
%
\begin{align}
\delta P = n^3 \delta V_1 \delta V_2 \delta V_3 \qty(1 + \xi (r_a)+ \xi (r_b) + \xi (r_c) + \zeta (r_a, r_b, r_c))
\,.
\end{align}

We assume the galaxy distribution to be connected to the (dark) matter density by  
%
\begin{align}
\delta_{g} (\vec{x}) = b \delta (\vec{x}) 
\,,
\end{align}
%
which is ok to large enough scales, typically tens of \SI{}{Mpc}.
The function \(\xi (s)\) has a peak around \SI{100}{Mpc}, due to \emph{baryon acoustic oscillations}.

The power spectrum is easier to work with experimentally, since its values are independent for different \(k\), while the correlation function values at different radii are correlated.

\section{Path integrals in cosmology}

The approach here is very different from the one in field theory. 
We do not start from an action principle here. 

Consider the space of square-integrable 3D functions. 
We consider them at fixed time, even though it does not really make sense since we can only observe our past light-cone. 

We require the existence of a basis \(\phi_{n}\) such that 
%
\begin{align}
\int \dd[3]{x} \phi_{m } (x) \phi_{n} (x) = \delta_{mn}
\,,
\end{align}
%
and we also ask that they are complete: 
%
\begin{align}
\sum _{n} \phi_{n} (x) \phi_{m} (y) = \delta^{(3)} (x-y)
\,.
\end{align}

A functional \(F\) takes a function and returns a real or complex number. If we write it with respect to the basis components of a function, we can express \(F\) as a (real, or complex number valued) function of an infinite number of variables. 

For instance, we could have 
%
\begin{align}
F_1 [q] &= \sum _{n} q_n f_n  \\
F_2 [q] &= \sum _{mn} k_{mn} q_m q_n
\,,
\end{align}
%
where \(q_n\) are the basis components of a function \(q\). 

We can discretize the space, so that the functional is given by its contribution to each small space box. 

We can write a functional as a series in the form 
%
\begin{align}
F[q] = \sum _{n=0}^{ \infty } \frac{1}{n!} \prod_i \dd{x_{i}} q(x_i)
\,,
\end{align}
%
for instance we can use the exponential. In that case the expression is 
%
\begin{align}
e^{\int f(x)q(x) \dd{x}}
= \sum _{n} \frac{1}{n!} \qty[\int f(x) q(x) \dd{x}]^{n}
\,,
\end{align}
%
which can be extended replacing \(\int f(x) q(x) \dd{x}\) with anything else. 

\end{document}
