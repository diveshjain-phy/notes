\documentclass[main.tex]{subfiles}
\begin{document}

\marginpar{Thursday\\ 2020-4-23, \\ compiled \\ \today}

We are going to continue discussing cosmic structure formation.

We treat the dynamics of self-gravitating collisionless particles. 

We will neglect pressure gradients since the pressure is generally very small. 

We have a Eulerian approach: we take a LIF with respect to the background field. 

We sometimes do not write the mass of the fluid particle: by the equivalence principle the inertial and gravitational masses are equal.

We have the relations \(\vec{r} = a \vec{x}\) and 
%
\begin{align}
\ddot{r} = - \nabla_{r} \Phi 
\,,
\end{align}
%
where the gravitational field is defined by 
%
\begin{align}
\nabla^2_{r} \Phi  = 4 \pi G \rho 
\,.
\end{align}

If we take the background potential 
%
\begin{align}
\Phi = \Phi_{b} = \frac{2}{3} \pi G \rho_{b} (t) r^2
\,,
\end{align}
%
we recover the Friedmann equation 
%
\begin{align}
\ddot{a} = - \frac{4}{3} \pi G \rho_{b} (t) a
\,.
\end{align}

The derivative of \(\vec{r}\) is given by 
%
\begin{align}
\dot{r} = \dot{a} x + a \dot{x} = Hr + v
\,,
\end{align}
%
which is the Hubble law in a fully Newtonian setting. 

We can write the particle Lagrangian: it looks like 
%
\begin{align}
\mathscr{L}' = \frac{m}{2} \qty(a \dot{x} + \dot{a} x)^2 - m \Phi (\vec{x}, t)
\,,
\end{align}
%
so we have three terms: \(a^2 \dot{x}^2\), \(\dot{a}^2 x^2 \) and \(2 a x \dot{a} \dot{x}\).

We use the canonical transformation 
%
\begin{align}
\mathscr{L} = \mathscr{L}' - \dv{\psi }{t}
\qquad \text{where} \qquad
\psi = \frac{1}{2} m a \dot{a} x^2
\,,
\end{align}
%
using which we remove the mixed terms, so we get 
%
\begin{align}
\mathscr{L} = \frac{m}{2} a^2 \dot{x}^2 - m \phi 
\,,
\end{align}
%
where \(\phi = \Phi - \Phi_{b} = \Phi + a \ddot{a} x^2 / 2\).

We have different kinds of problems: we have problems in which we give initial positions and velocities, and ones in which we have initial and final positions. 

We can also have mixed problems, where we give the initial velocity and the final position.

\todo[inline]{What are they called?}

Then we get the equation for \(\phi \): 
%
\begin{align}
\nabla^2_{x} = 4 \pi G a^2 \rho + 3 a \ddot{a} = 4 \pi G a^2\rho_{b} \delta 
\,.
\end{align}

In order to write Hamilton's equations we need the conjugate momentum: 
%
\begin{align}
\vec{p} = \pdv{\mathscr{L}}{\vec{\dot{x}}} = m a^2 \dot{x}
\,,
\end{align}
%
so 
%
\begin{align}
\dot{p} = \pdv{\mathscr{L}}{\vec{x}} = - m \nabla_{x} \phi 
\,,
\end{align}
%
and if we define \(\vec{v} = a \dot{\vec{x}}\) then we have 
%
\begin{align}
\vec{\dot{p}} = \pdv{}{t} \qty(m a \vec{v}) = m a \vec{\dot{v}} + m \dot{a} \vec{v}
\,,
\end{align}
%
which yields our Euler-Lagrange equations: 
%
\begin{align}
\dv{\vec{v}}{t} + H \vec{v} = - \frac{1}{a} \nabla \phi 
\,.
\end{align}

This equation is always true, it does not assume that the element of mass behaves like a fluid. 

Hamilton's equations are written by writing the Hamiltonian: 
%
\begin{align}
\mathscr{H}(\vec{x}, \vec{p}, t) &= \vec{p} \cdot \dot{\vec{x}} - \mathscr{L}   \\
&= \frac{p^2}{2 m a^2} + m \phi (\vec{x}, t)
\,,
\end{align}
%
and Hamilton's equations read 
%
\begin{align}
\vec{p} &= m a^2 \dot{\vec{x}}  \\
\dot{\vec{p}} &= - m \nabla_{x} \phi 
\,.
\end{align}

By Liouville's theorem for collisionless matter the phase-space density is conserved: 
%
\begin{align}
\dd{f} = 0 \implies 
\pdv{f}{t} + \dot{\vec{x}} \cdot \pdv{f}{\vec{x}} + \dot{\vec{p}} \cdot \pdv{f}{\vec{p}} = 0
\,.
\end{align}

This yields the Vlasov equation: 
%
\begin{align}
\pdv{f}{t} + \frac{\vec{p}}{m a^2} \cdot \nabla f - m \nabla \phi \cdot \pdv{f}{\vec{t}}= 0
\,,
\end{align}

which must be solved together with the Poisson equation for \(\phi \).

Nobody has managed to solve it yet: it is very hard. 

We cannot use the linear regime: stars and structures are highly nonlinear. 

Jim Peebles for ``reconstruction problem''.
There is an ambiguity in the trajectories. 
We can find information about these things in a dropbox folder by the professor. 

We take \emph{moments}: 
%
\begin{align}
\rho (\vec{x}, t) = \frac{m}{a^3} \int \dd[3]{p} 
f(\vec{x}, \vec{p}, t)
\,,
\end{align}
%
the first one is
%
\begin{align}
\vec{v}(\vec{x}, t) = \frac{1}{ma} \frac{\int \dd[3]{p} \vec{p} f (\vec{x}, \vec{p}, t)}{\int \dd[3]{p} f(\vec{x}, \vec{p}, t)}
\,,
\end{align}
%
the second one is 
%
\begin{align}
\Pi^{ij} &= \frac{\expval{p^{i} p^{j}}}{m^2a^2} - v^{i} v^{j}  \\
&= \frac{1}{m^2a^2} \qty[\dots]
\,,
\end{align}
%
which represents a velocity dispersion. In general we are not able to drop this term. 

We integrate the Vlasov equation over \(\vec{p}\) and manipulate. 

Collecting all the terms we get 
%
\boxalign{
\begin{align}
\pdv{v^{i}}{t} + H v^{i} + \frac{1}{a} \qty(v_{j} \partial^{j}) v^{i}
= - \frac{1}{a} \partial^{i} \phi - \frac{1}{a \rho } \partial_{j} \qty(\rho \Pi^{ij})
\,,
\end{align}}
%
so the Euler equation changed: we have an additional velocity dispersion term.

What do we do then? Kolmogorov's approach is through a hierarchy of moments, and it works well in the turbolent regime. 

We can have an ansatz for \(\Pi^{ij}\), for example we can say that it is diagonal: this specifically does not really work here. 

People say that \(f\) can be split in two: a \emph{coarse-grain} distribution and a \emph{fine-grain} distribution. The coarse distribution is an average, the fine grain captures the details. 

A system behaves like a fluid if there is scattering between the particles. 

How do we define the scale? This is difficult, since CDM does not have a cross-section.
The ``physical truth'' is provided by N-body simulations, we make approximations until they match the simulations.

This is also why we do not use multiple-scale models: they are complicated and difficult to tune.

\end{document}
