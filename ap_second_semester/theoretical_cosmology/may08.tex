\documentclass[main.tex]{subfiles}
\begin{document}

\marginpar{Friday\\ 2020-5-8, \\ compiled \\ \today}

Our variables are \(\vec{u}\), \(\eta \) and \(\varphi \), to be considered as \(a\) varies. 

Notice that \(\nu \) from now is not the viscosity!

We have the equations we wrote last time using 
%
\begin{align}
\eta = \frac{\rho}{\rho_{b}} = 1+ \delta 
\,,
\end{align}
%
and \(\vec{u} = \dv*{\vec{x}}{\tau }\), and \(\tau = a(t)\).

So, we define the wavefunction: 
%
\begin{align}
\Psi = R e^{i \Phi  / \nu } = \qty(1 + \delta )^{1/2} e^{i \Phi / \nu }
\,.
\end{align}

With the Madelung procedure we get a Schrödinger equation with a correction to the potential: 
%
\begin{align}
i \nu \pdv{\Psi }{t} = - \frac{\nu^2}{2}  \nabla^2 \Psi 
+ \qty(V + \frac{\nu^2}{2} \frac{\nabla^2 R}{R})\Psi 
\,.
\end{align}

Then, we take 
%
\begin{align}
V = \frac{3}{2 \tau } \qty(\Phi + \varphi ) 
\,,
\end{align}
%
so that we get the Poisson equation in the form 
%
\begin{align}
\nabla^2 \qty(V + \frac{3i\nu}{4 \tau } \log \qty( \frac{\Psi}{\Psi^{*}} )) = - \frac{3}{2 \tau^2} \qty(\abs{\Psi  }^2 - 1)
\,.
\end{align}

Can we drop the quantum term in the Schrödinger equation? kind of, we can say that it is quadratic in \(\nu \)\dots
This is what is done in the literature, anyway.

Then, we get the quantum term in the regular Poisson equation:
%
\begin{align}
\pdv{\Phi }{\tau } + \frac{1}{2} \qty(\nabla \Phi )^2 = - V - \frac{\nu^2}{2} \frac{\nabla^2 R}{R }
\,.
\end{align}

Also, let us set the potential \(V = 0\): then we will have the quantum mechanics of free particles. 

So then, we get 
%
\begin{align}
i \nu \pdv{\Psi }{ \tau } + \frac{\nu^2}{2} \nabla^2 \Psi  = 0 
\,.
\end{align}

We solve this equation by performing a Wick rotation and using the results we obtained in the adhesion approximation. 

The similarity is to the Fokker-Planck equation: the only difference is that we have to map \(\tau \to i \tau \), that is, we perform a Wick rotation. 
So, we take everything we discussed for the solution to the Fokker-Planck equation and adapt it here. 

So, we consider a Green's function as 
%
\begin{align}
\Psi (\vec{x}, \tau ) = \int G(\vec{x}, \tau, \vec{q}, 0) \Psi_{i} (\vec{q}) \dd[3]{q}
\,,
\end{align}
%
so we can use the Feynman formula: 
%
\begin{align}
G(\vec{x}, \tau , \vec{q}, 0) = \qty(2 \pi i \nu \tau )^{-3/2} 
\exp( \frac{i}{\nu } \frac{(\vec{x} - \vec{q})^2}{2 \tau })
\,,
\end{align}
%
and then from the Madelung transformation we get 
%
\begin{align}
\Psi (\vec{x}, \tau ) = \qty(2 \pi i \nu \tau )^{ - 3/ 2}
\int \dd[3]{q} \qty(1 + \delta_{i} (\vec{q}))^{1/2} \exp( \frac{i}{\nu } \qty[ \frac{(\vec{x} - \vec{q})^{2}}{2 \tau } + \Phi_{i} (\vec{q})])
\,.
\end{align}

Now, we apply the saddle-point approximation: this integral, kind of like Faynman's path-integral, is dominated by the regions in which the exponentials interfere constructively, which are defined by 
%
\begin{align}
\nabla_{q} \qty[ \frac{(\vec{x} - \vec{q})^2}{2 \nu \tau } + \frac{\Phi_{i} (\vec{q})}{\nu }] = 0
\,.
\end{align}

Now we are going to directly get both velocity and density. 
We assume that \(\delta_{i}=0\). Then we find 
%
\begin{align}
\vec{x} = \vec{q}_{s} - \tau \nabla_{q } \varphi_0 (q_{s})
\,.
\end{align}

As \(\nu \to 0\), we find 
%
\begin{align}
\Psi (\vec{x}, \tau ) \approx \eta^{1/2} (\vec{q}_{s}) e^{ \frac{i}{\nu } \chi (\vec{q}_{s})} \qty[\det \qty(\delta_{ij} - \tau \pdv[2]{\varphi_0 }{q_{i}}{q_{j}})]^{1/2}
\,,
\end{align}

where 
%
\begin{align}
\chi = \frac{(\vec{x} - \vec{q})^2}{2 \tau } - \varphi_0 (\vec{q})
\,.
\end{align}

This looks similar to Zeldovich, but suppose that we had different stationary points: then, we get interference! 

In the laminar regime we get 
%
\begin{align}
\eta = \Psi^{*} \Psi \sim \det^{-1} \qty(\delta_{ij} - \tau \pdv[2]{\varphi_0 }{q_{i}}{q_{j}})
\,.
\end{align}

Sandro Winberger is in Parma, he knows how to do this stuff. 

\section{Statistics of the Large-Scale Structure}

We want to understand the statistics of the distribution in space of structures.
We can compare the models of LSS formation in terms of statistical properties of the distribution of galaxies, not in terms of geography. 

We only have one universe, and we see only a part of it. 

The person who introduced the fair sample hypothesis is Birkhoff. 

\end{document}
