\documentclass[main.tex]{subfiles}
\begin{document}

\marginpar{Thursday\\ 2020-5-7, \\ compiled \\ \today}

Using the decaying mode is a mathematical trick. 

\todo[inline]{Understand this better.}

As we saw last time, the length in \(\nu \) can be interpreted as the thickness of the pancakes. 

We have gotten an equation with no nonlinear terms: the Fokker-Planck equation, 
%
\begin{align}
\pdv{U}{\tau } = \nu \nabla^2 U
\,.
\end{align}

If we change \(\tau \) to \(i \tau \), this is the Schrödinger equation. 
We solve it by separation of variables. 

In Fourier space, we get 
%
\begin{align}
E_{k} g_{\vec{k}} = - \nu k^2 g_{\vec{k}}
\,.
\end{align}

This is a linear ODE, a general solution will be a superposition of eigenstates: 
%
\begin{align}
U(\vec{x}, \tau ) = \int \frac{\dd[3]{k}}{(2 \pi )^3} e^{- \nu k^2 \tau } g_{\vec{k}} e^{i \vec{k} \cdot \vec{x}}
\,,
\end{align}
%
and we just need to impose boundary conditions. 

We look at the kernel: the ``impulse response'' of our function. 
We call it 
%
\begin{align}
K ( \hat{x}, \tau | \hat{q}, 0)
\,,
\end{align}
%
which is indeed a conditional probability amplitude: as \(\tau \to 0\) it approaches \(\delta (\hat{x} - \hat{q})\). 

This is a probability, not a probability amplitude. 

We can integrate to find it: it is 
%
\begin{align}
K (\hat{x}, \tau | \hat{q}, 0) = \qty(4 \pi \nu \tau )^{-3/2} \exp( -\frac{(\hat{x} - \hat{q})}{4 \nu \tau })
\,.
\end{align}

We find this by marginalizing. 

This yields an equation which is basically the Hamilton-Jacobi one, 
%
\begin{align}
\pdv{S}{\tau } + \frac{1}{2} \qty(\nabla S)^2 = 0
\,.
\end{align}

This yields an expression for the full (Eulerian) velocity. 
We are interested in the limit \(\nu \to 0 \). Only the minima of the action matter, and we can perform Gaussian integrals. We expand \(S\) around the absolute minima: we find 
%
\begin{align}
U(\hat{x}, \tau ) &= (4 \pi \nu \tau )^{-3/2} e^{-S / 2 \nu }
\int \dd[3]{ \delta q} \exp(- \frac{1}{4 \nu } \pdv[2]{S}{q_{i} q^{j} \delta q^{i} \delta q_{j}})  \\
&= \det \dots
\,.
\end{align}

So, we are saying that the velocity is given by 
%
\begin{align}
\vec{u} (\vec{x}, \tau ) = \sum _{s} \frac{\vec{x} - \vec{q}_{s}}{\tau } w_{s} (\vec{x}, \vec{q}, \tau )
\,.
\end{align}

It is a superposition of different solutions, weighted by the \(w_{s}\).

\section{Schrödinger equation approach for LSS formation}

Madelung proposed a hydrodynamical approach to QM: a decomposition 
%
\begin{align}
\psi (\vec{x}, t) = R (\vec{x}, t) e^{- \frac{S(\vec{x},t)}{i \hbar}}
\,,
\end{align}
%
which we can substitute into the Schrödinger equation 
%
\begin{align}
i \pdv{\psi }{t} = - \frac{\nabla^2}{2m} \psi + V \psi 
\,,
\end{align}
%
which yields two separate equations for the real and imaginary parts: we get 
%
\begin{align}
- \nabla \qty( \frac{R^2}{2m} \nabla S) = \pdv{}{t} R^2
\,,
\end{align}
%
so if we define the usual probability current 
%
\begin{align}
j = \frac{\hbar}{2 m i} \qty(\psi^{*} \nabla \psi - \psi \nabla \psi^{*}) = \frac{R^2}{2m} \nabla S
\,,
\end{align}
%
so we get the equation 
%
\begin{align}
\pdv{\rho }{t} + \nabla \cdot \qty(\rho \vec{v}) = 0
\,
\end{align}
%
for the imaginary part of the equation. 

For the real part of the equation, instead, we find 
%
\begin{align}
\pdv{S}{t} + \frac{1}{2m} \qty(\nabla S)^2 = - (V + Q ) 
\,,
\end{align}
%
where 
%
\begin{align}
Q ( \vec{x}, t) = - \frac{\hbar^2}{2m} \frac{1}{R} \nabla^2 R 
\,
\end{align}
%
a part we must include in the potential to account for quantum effects. 
This is the effectively the Bernoulli equation! 

After the Madelung transformation, then, we get hydrodynamic equations. 
We cna also do this backward: we start from the Bernoulli equation and continuity: 
%
\begin{align}
\pdv{\rho }{t} + \nabla \cdot \qty(\rho \vec{v}) &= 0  \\
\pdv{\phi }{t} + \frac{1}{2} \qty(\nabla \phi )^2 &= -V 
\,,
\end{align}
%
and associate it with Schrödinger, using \(\nu \) instead of \(\hbar\). 
We then get Schrödinger, with a different potential: 
%
\begin{align}
i \nu \pdv{\Psi }{t} = - \frac{\nu^2}{2 } \nabla^2 \Psi + \qty(V + \frac{\nu^2}{2} \frac{\nabla^2 R}{R}) \Psi 
\,.
\end{align}

We apply this procedure in cosmology: we get 
%
\begin{align}
\pdv{\eta }{\tau } + \nabla \cdot \qty(\eta \vec{u}) = 0   \\
&
\,,
\end{align}


\end{document}
