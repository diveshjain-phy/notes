\documentclass[main.tex]{subfiles}
\begin{document}

\marginpar{Friday\\ 2020-3-27, \\ compiled \\ \today}

In the perturbed Einstein Equations we will need the stress-energy tensor: it does not require the full phase space distribution, but only certain kinds of integrated information. 

With our approach, we did not assume that DE, DM and such are fluids: instead, we have shown it. 

We can combine the equations for electrons and protons into a unique equation for the baryons: 
%
\begin{align}
m_p \pdv{}{t} \qty(n_b v_b^{j}) + 4H m_p n_b v_b^{j} \dots
\,.
\end{align}
%

The quantity \(- \expval{c_{e \gamma } p \mu } / \rho_{b}\) is similar to what we had for photons: a monopole term, the temperature anisotropy, and the Doppler term, which is called that since it involves a velocity. Then we get 
%
\begin{subequations}
\begin{align}
- \expval{c_{e \gamma } p \mu } / \rho_{b}
&= \frac{n_e \sigma_{T}}{\rho_{b}}
\int \frac{ \dd[3]{p}}{(2 \pi )^3} p^2 \pdv{f^{(0)}}{p} \mu \qty[ \widetilde{\theta}_{0} - \widetilde{\theta} (\mu ) + \widetilde{v}_{b} \mu ]  \\
&= \frac{n_e \sigma_{T}}{\rho_{b}}
\int_{0}^{ \infty } \frac{ \dd{p}}{2 \pi^2} p^{4} \pdv{f^{(0)}}{p} \int_{-1}^{1} \frac{ \dd{\mu }}{2} \mu \qty[\widetilde{\theta}_{0} - \widetilde{\theta} (\mu ) + \widetilde{v}_{b} \mu ] 
\,,
\end{align}
\end{subequations}
%
notice that the terms inside the bracket are independent of the modulus of \(p\). 
Now we define the \emph{dipole}: 
%
\begin{align}
\theta_{1} = i \int_{-1}^{1} \frac{ \dd{\mu }}{2 } \mu \theta(\mu )
\,,
\end{align}
%
so we finally get 
%
\begin{align}
\widetilde{\dot{v}}_{b} + \frac{\dot{a}}{a} \widetilde{v}_{b}
+ i k \widetilde{\Phi} =
\dot{\tau} R \qty(3 i \widetilde{\theta}_{1} + \widetilde{v}_{b})
\,,
\end{align}
%
where we defined 
%
\begin{align}
R = \frac{3 \rho_b^{(0)}}{4 \rho_{\gamma }^{(0)}}
\,.
\end{align}

In general we define the multipole moments as 
%
\begin{align}
\theta_{\ell} = \frac{1}{(-i)^{\ell}} \int_{-1}^{1} \frac{ \dd{\mu }}{2} P_\ell (\mu ) \theta(\mu )
\,,
\end{align}
%
where \(P_e (\mu )\) are the Legendre polynomials. 

Then we get 
%
\begin{align}
\widetilde{\dot{\theta}} + ik \mu \widetilde{\theta} - \widetilde{\dot{\Psi}}
+ ik \mu \widetilde{\Phi} =
-\dot{\tau} \qty[\widetilde{\theta}_{0} - \widetilde{\theta} 
+ \mu v_b - \frac{1}{2} P_2 (\mu ) \widetilde{\theta}_{2}] 
\,,
\end{align}
%
where the addition of the quadrupole term accounts for (???), see the notes by Natale \cite[]{nataleNoteCorsoDi2017}. 

Dropping the tildes, we get 
%
\begin{subequations}
\begin{align}
\dot{\theta} + ik \mu \theta &= \dot{\Psi} - ik \mu \Phi - \dot{\tau} \qty[\theta_0 - \theta + \mu v_b - \frac{1}{2} P_2 (\mu ) \Pi ]  \\
\Pi &= \theta_2 - \theta_{P_2 } + \theta_{P_0 }  \\
\dot{\theta}_{p} + i k \mu \theta_{p} &= - \dot{\tau} \qty[- \theta_{p} + \frac{1}{2} \qty(1 - P_2 (\mu ) \Pi )] \dots
\,,
\end{align}
\end{subequations}
%
anisotropy creates polarization, polarization does not affect polarization much. 

Why do we not have a quadrupole term? Its integral is zero since it is multiplied by \(\mu \). 

What about neutrinos? 
We can model the anisotropies in the cosmic neutrino background using Fermi-Dirac statistics, with 
%
\begin{align}
f = \frac{1}{\exp(d)} \dots
\,,
\end{align}
%
so now repeating the considerations we made for the photons, we get 
%
\begin{align}
\dot{w} \dots
\,.
\end{align}

\section{Perturbing the Einstein Equations}

Why do we consider only the scalar perturbations? Vectors do not obey dynamical equation but only conservation equations, and they die away quickly in the expansion. 
The Lense-Thirring effect is about the \(g_{0i}\) terms, and it is not governed by the (???) theorem, but we neglect it as well. 

Tensor perturbations are gravitational waves. 
They evolve separately from the scalar perturbations: we decompose the anisotropy into 
%
\begin{align}
\theta^{T} (k, \mu, \phi ) = \theta^{T}_{+} (k, \mu ) \qty(1 - \mu^2) \cos( 2 \phi ) + \theta^{t}_{ \times } (k, \mu ) (1-\mu^2)
\,,
\end{align}
%
and both of the polarizations (\(\epsilon = +, \times \)) satisfy 
%
\begin{align}
\dot{\theta}^{T}_{\epsilon } + ik \mu \theta^{T}_{\epsilon } + \frac{1}{2} \dot{h}_{\epsilon } = \dot{\tau} \qty[
\theta^{T}_{\epsilon } - \frac{1}{10} \theta^{T}_{\epsilon, 0} - \frac{1}{7} \theta^{T}_{ \epsilon , 2} - \frac{3}{7} \theta^{T}_{\epsilon , 4}]
\,.
\end{align}

Let us then perturb Einstein's equations: we use conformal time, so the metric (in Euclidean flat 3-space) is 
%
\begin{align}
\dd{s^2} = a^2 \qty[- (1 + 2 \Phi ) \dd{\eta^2} + (1- 2 \Psi ) \dd{\ell^2}]
\,.
\end{align}

We compute the Christoffel symbols, and from these we get the Ricci tensor, the Ricci scalar and the Einstein tensor. 
It is generally much more convenient to write the equations with mixed indices 
%
\begin{align}
G^{\mu }_{\nu } = 8 \pi G T^{\mu }_{\nu }
\,.
\end{align}

To first order we get 
%
\begin{subequations}
\begin{align}
\Gamma^{0}_{00} &= \frac{ \dot{a} }{a } + \dot{\Phi}  \\
\Gamma^{i}_{0j }&= \qty(\frac{\dot{a}}{a} - \dot{\Psi}) \delta^{i}_{j}  \\
\Gamma^{0}_{ij} &= \qty[\frac{\dot{a}}{a } \qty(1 - 2 \Phi - 2 \Psi ) - \dot{\Psi}] \delta_{ij}  \\
\Gamma^{0}_{0i} &= \partial_{i} \Phi   \\
\Gamma^{i}_{00} &= \partial^{i} \Phi   \\
\Gamma^{i}_{jk} &= - \partial_{j} \Psi \delta^{i}_{k} - \partial_{k} \Psi \delta^{i}_{j} + \partial^{i} \Psi 
\delta_{jk}
\,,
\end{align}
\end{subequations}
%
so the Einstein tensor reads: 
%
\begin{align}
G^{0}_{0} = \frac{1}{a^2} \qty[\dots]
\,,
\end{align}
%

Often people make the quasi-static approximation, in which we neglect the time-derivatives of the potentials. We do not do that here. 
The stress-energy tensor has the following 00 component: 
%
\begin{align}
T^{0}_{0} = - \sum _{i} g_i \int  \frac{ \dd[3]{p}}{(2 \pi )^3} E_i (p) f_i (\vec{x}, \vec{p} ,t)
\,.
\end{align}

For nonrelativistic matter, \(E_i \sim m_i\): then we get 
%
\begin{align}
T^{0, \text{dm}}_{0} = - \rho_{\text{dm}} (1 + \delta )
\,.
\end{align}

For the photons we have 
%
\begin{align}
T^{0,(\gamma )}_{0} = 
-2 \int \frac{\dd[3]{p}}{(2 \pi )^3} p \qty[f^{(0)} - p \pdv{f^{(0)}}{p} \theta ] 
= - \rho_{\gamma } \qty[ 1 + 4 \Theta_0 ]
\,,
\end{align}
%
and for the neutrinos we get the exact same result, substituting the variables: 
%
\begin{align}
T_{0}^{0, (\nu )} = - \rho_{\nu } \qty[1 + 4 w_0 ]
\,.
\end{align}

Dark Energy is usually considered to be smooth. Then we have, for the 00 EFE: 
%
\begin{align} \label{eq:einstein-00-perturbed-equation}
\nabla^2 \Psi - 3 \frac{\dot{a}}{a} \qty(\dot{\Psi} + \Phi \frac{\dot{a}}{a }) = 4 \pi G a^2 \qty(
\rho_{\text{dm}} \delta + \rho_{b} \delta_{b} + 4 \rho_{\gamma } \Theta_0 + 4 \rho_{\nu } w_0 
)
\,.
\end{align}

Recall that we want to solve for \(\Psi \) and \(\Phi \): we have a lot of redundancy in the EFE. 
Another convenient independent equation is the traceless part of the \(ij\) components. 
The simplest way to do these calculations is to use projectors in Fourier space.

On the left hand side we get 
%
\begin{subequations}
\begin{align}
\qty(\hat{k}_{i} \hat{k}^{j} - \frac{1}{3} \delta^{i}_{j}) \hat{G}^{i}_{j} &= \qty(\hat{k}_{i} \hat{k}^{j} - \frac{1}{3} \delta_{i}^{j}) 
\frac{k^{i}k_{j}}{a^2} \qty(\Phi - \Psi )  \\
&= \frac{2}{3} \frac{k^2}{a^2} \qty(\Phi- \Psi )
\,.
\end{align}
\end{subequations}


\end{document}
