\documentclass[main.tex]{subfiles}
\begin{document}

\marginpar{Friday\\ 2020-5-22, \\ compiled \\ \today}

We have found that the functional \(Z\), which is the partition function, can be expressed as the functional integral of the exponential of the double integral of the bilinear form \(K^{-1}\) applied to the linear term. 

The average of a functional is given by the expression 
%
\begin{align}
\expval{ F [q]} = \int \mathcal{D}q \qty(P [q] F[q])
\,.
\end{align}

We can calculate this for any functional if we know its functional derivatives and the \(n\)-point correlation function. 

Plus, we can express the correlation function as a functional derivative of the partition function.
This can be expressed compactly as 
%
\begin{align}
\expval{F[q]} = \eval{F[-i \fdv{}{J}] Z[J] }_{J=0}
\,.
\end{align}

We define the \emph{connected} correlation function  by 
%
\begin{align}
W[J] = \sum _{n} \frac{i^{n}}{n!} \int \dd{x_1 } \dots \dd{x_{n}} C^{(n)}_{C} (x_1 , \dots, x_{n}) J(x_1) \dots J(x_{n})
\,,
\end{align}

where \(W[J] = \log Z[J]\). 
From the classical field \(q _{\text{cl}} \) we define \(\Gamma \), and then this gives us the classical equation of motion: 
%
\begin{align}
J(x) = - \fdv{\Gamma [q _{\text{cl}}]}{q _{\text{cl}}}
\,.
\end{align}

Now, we want to make the discussion concrete: from a probability functional to a probability density function. 

We want to compute the probability density \(\dd{P}_{\alpha }\) that the field \(q\) at a point \(\overline{x}\) takes a value between \(\alpha \) and \(\alpha + \dd{\alpha }\). We can compute it as the average of \(\expval{q (\vec{x}) - \alpha }_{q}\).



\end{document}
