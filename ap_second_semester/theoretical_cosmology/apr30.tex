\documentclass[main.tex]{subfiles}
\begin{document}

\marginpar{Thursday\\ 2020-4-30, \\ compiled \\ \today}

Today we will improve on the Zel'dovich approximation with the adhesion approximation.

We see plots from a 1992 paper from the professor. 

We are taking collisionless particles in 1D. The Zel'dovich approximation breaks down since the particles going through each other does not make sense: they will have gravitational interactions in that case.

We will discuss the difference, in structure formation, between ``pancakes'' and ``filaments''.

The issue seems to be that, after the trajectories of the particles cross the matrix of transformation between Eulerian and Lagrangian coordinates becomes singular.

How do we get extended structures? By continuity of the eigenvalues. 

We cannot really estimate the global size of the structures here, they are transient anyway.

\section{The adhesion approximation}

We add a fictitious term for the gravitational ``sticking'' of particles into pancakes or filaments. 

We start from the equations
%
\begin{align}
\bigd{\vec{u}}{\tau } &= \nu \nabla^2 \vec{u}  \\
\bigd{\eta }{\tau } &= - \eta \nabla \cdot \vec{u}
\,.
\end{align}

We have basically added a kinematic viscosity term to the Navier-Stokes equations.
The coefficient \(\nu \) is called the kinematic viscosity coefficient, which has dimensions \(\qty[\nu ] = L^2 /T\).

This is completely artificial, but we do it to model the effect of gravitational interactions. 

We are basically modelling gravity as a local effect. 

What we obtain is called the 3D Burgers' equation. 
We can obtain turbulence here, but the system is solvable.

\todo[inline]{Is the system solvable because at small scales the vortices die out?}

Let us assume that the velocity is irrotational: \(\vec{u} = \nabla \Phi \). The quantity \(\Phi \) is called a velocity potential. 

Substituting in, we get 
%
\begin{align}
\pdv{\Phi }{\tau } + \frac{1}{2} \qty(\nabla \Phi )^2 = \nu \nabla^2 \Phi 
\,,
\end{align}
%
which is called the Bernoulli equation.

To see that these are equivalent: if we take the gradient, we get 
%
\begin{align}
\frac{1}{2} \partial^{i}  \qty(\partial_{j} \Phi \partial^{j} \Phi )
&= \partial^{j} \Phi \partial_{i } \partial_{j} \Phi  
= u^{j} \partial_{j} u^{i}
\,.
\end{align}

The other derivatives commute, so if we take the gradient of the equation we get back Euler's equation.

The issue is the fact that this is a nonlinear equation.

We do the Hopf-Cole transformation to solve the Burgers equation. So, we define 
%
\begin{align}
\Phi = - 2 \nu \log U
\,,
\end{align}
%
where \(U\) is called the ``expotential''.
This simplifies the equation into the Fokker-Planck equation: 
%
\begin{align}
\pdv{U}{\tau } = \nu \nabla^2 U
\,.
\end{align}

This is a linear diffusion equation. It models heat transfer, but it is also the Schrödinger equation after a Wick\footnote{Pronounced ``Vick'', not ``Uick''. } rotation.

This is a parabolic equation. 
The Poisson equation, for instance, is elliptic: it only requires spatial initial conditions. Hyperbolic equations include the wave equation: they require initial conditions. 

Parabolic equations require both: they imply infinite-speed transmission of signal. 

We use the ansatz 
%
\begin{align}
U(\vec{x}, \tau ) = f(\tau ) g(\vec{x})
\,,
\end{align}
%
and we substitute 
%
\begin{align}
f(\tau ) = e^{E \tau }
\,,
\end{align}
%
so that we need to solve 
%
\begin{align}
E g(\vec{x}) = \nu \nabla^2 g(\vec{x})
\,.
\end{align}

In Fourier space, then, 
%
\begin{align}
E_k g_{\vec{k}} = - \nu k^2 g_{\vec{k}}
\,.
\end{align}
%
So, a general solution will be a superposition: 
%
\begin{align}
U(\vec{x}, \tau ) = \int \frac{ \dd[3]{k}}{(2 \pi )^3}
e^{- \nu k \tau^2} g_{\vec{k}} e^{i \vec{k} \cdot \vec{x}}
\,. 
\end{align}

The standard thing to do to get some boundary conditions is to look at a kernel of the function, its impulse response.

We get 
%
\begin{align}
K(\vec{x}, \tau | \hat{q}, 0) \overset{\tau \to 0}{ \to } \delta (\vec{x} - \vec{q})
\,.
\end{align}

In order to get this, we need to set \(g_{\vec{k}} = e^{- i \vec{k} \cdot \vec{q}}\). 
This yields 
%
\begin{align}
K (\vec{x}, \tau | \vec{q}, 0) &=
\int \frac{ \dd[3]{k}}{(2\pi )^3} e^{- \nu k^2\tau } e^{i \vec{k} \cdot \qty(\vec{x} - \vec{q})}  \\
&= (4 \pi \nu \tau )^{-3/2} \exp(- \frac{(\vec{x} - \vec{q})^2}{4 \nu \tau })
\,.
\end{align}

To get the initial conditions \(U_0 (\vec{q})\) we use the Chapman-Kolmogorov equation: 
%
\begin{align}
U(\vec{x}, \tau ) = \int \dd[3]{q} U_0 (\vec{q}) K(\vec{x}, \tau | \vec{q}, 0)
\,.
\end{align}

This is a consequence of Bayes's theorem: we can marginalize over a parameter like
%
\begin{align}
\mathbb{P}(A) = \int \dd{B} \mathbb{P}(B) \mathbb{P}(A | B)
\,.
\end{align}

Inverting the definition of \(U\), we have 
%
\begin{align}
U_0 (\vec{q}) = e^{- \Phi_0 (\vec{q})  / 2 \nu } = e^{ \phi_0 / 2 \nu }
\,,
\end{align}
%
and then we get 
%
\begin{align}
U(\vec{x}, \tau ) = \int \frac{ \dd[3]{q}}{(4 \pi \nu \tau )^{3/2}}
e^{- S(\vec{x}, \vec{q}, \tau )  / 2 \nu }
\,.
\end{align}

The Bernoulli equation is the same as the Hamilton-Jacobi equation for the action.

Here, 
%
\begin{align}
S(\vec{x}, \vec{q}, \tau ) = \frac{(\vec{x} - \vec{q})^2}{2 \tau } - \phi_0 (\vec{q})
\,,
\end{align}
%
and then we only need to recover the velocity: we need to differentiate and get 
%
\begin{align}
\vec{u} (\vec{x}, \tau ) = - 2 \nu \frac{\nabla U }{U }
= \frac{ \int \dd[3]{q}\frac{\vec{x} - \vec{q}}{\tau } e^{- S(\vec{x}, \vec{q}, \tau ) / 2 \nu }}{ \int \dd[3]{q} e^{- S (\vec{x}, \vec{q}, \tau ) / 2 nn}}
\,.
\end{align}

We are interested in small values for \(\nu \). This parameter regulates the thickness of our pancakes, since it gives us a length scale for the derivative in the ``Navier-Stokes'' equation.

In this limit, the exponential is very peaked, and we can consider only the absolute minima of the action, just like in the path integral. 

If we perform this approximation, we get 
%
\begin{align}
U(\vec{x}, \tau ) = e^{ - S(\vec{x}, \vec{q}, \tau )  / 2 \nu } \sum _{s} j_s (\vec{x}, \vec{q}, \tau )
\,,
\end{align}
%
where we expanded the action around the minimum to get 
%
\begin{align}
j_S = \qty[\det \pdv[2]{S}{q_{i}}{q_{j}}]^{- 1/2}
\,.
\end{align}

\end{document}
