\documentclass[main.tex]{subfiles}
\begin{document}

\marginpar{Friday\\ 2020-4-24, \\ compiled \\ \today}

\section{The Zel'dovich approximation}

So far we have discussed perturbation in two ways: initially, we had a contribution on the RHS of the EFE given by the perturbations.

During radiation dominance there is not much need to go beyond linear theory. 

Now we want to go full nonlinear.

We need to use a mixture of techniques: both the study of the 2-body problem and a continuous distribution of matter.
This is what we do if we use numerical techniques.

On the other hand, we can make some smart approximations.

Last semester we mentioned the ``spherical top-hat solution'': we assume spherical symmetry, and work from there. 

The Zel'dovich approximation gives an exact solution in the case of planar symmetry.
We can also get exact solutions with cylindrical symmetry. 

We come back to the fluid equations: continuity, Euler, Poisson on a FRLW background. 

When you do linear perturbation theory, you perturb around the background value.

However, in doing this we are assuming that all the perturbations have the same weight: this is not necessarily true. We should define an internal hierarchy. 

What we are doing is formal in the context of Lagrangian perturbation theory. 

We define 
%
\begin{subequations}
\begin{align}
\eta &= \frac{\rho }{ \rho_{b}}  = 1 + \delta  \\
\vec{u} &= \dv{\vec{x}}{a} = \frac{\vec{v}}{a \dot{a} }  \\
\varphi &= \frac{3 t_{*}^2}{2 a_{*}^3} \phi 
\,.
\end{align}
\end{subequations}

This is because he wanted dependence on the growth factor. In EDS this is the scale factor, in general this could be a different thing.

We are referring to a more general background: there can be a growth suppression factor, such that the density perturbation \(\delta \) does not scale with \(a\) but instead based on some function of \(a\). 
We should use this function and not the scale factor in general.
In the matter-filled universe we are considering right now this is the same.

The new equations are 
%
\begin{subequations}
\begin{align}
\frac{ \DD \vec{u}}{\DD a} + \frac{3}{2a} \vec{u} &= - \frac{3}{2a} \nabla \varphi  \\
\frac{ \DD \eta }{\DD a} + \eta \nabla \cdot \vec{u} &= 0  \\
\nabla^2 \varphi &= \frac{\delta}{a}
\,,
\end{align}
\end{subequations}
%
where 
%
\begin{align}
\frac{\DD }{\DD a} = \pdv{}{a} + \vec{u} \cdot \nabla
\,
\end{align}
%
is the convective derivative.

We have chosen this variable \(\vec{u}\) precisely because we want something which is almost constant in linear theory: we have \(\delta \sim t^{2/3}\) and \(v \sim t^{1/3}\) and \(\phi \sim \const\). 

So, we get 
%
\begin{align}
\pdv{\vec{u}}{a} = 0  
\,,
\end{align}
%
so then we can approximate:
%
\begin{align}
\bigd{\vec{u}}{a} = 0 
\,.
\end{align}

At the linear level, \(\vec{u} = - \nabla \varphi \). 
The argument is that we can \emph{extend} the linear result beyond the linear level. This is because the velocity goes like \(k\), while the potential goes like \(\delta / k^2\).

The system becomes more and more linear as we increase the scale.

Why don't we try to solve the equation by neglecting the nonlinearity in the velocity? 

Then the equation system reads: 
%
\begin{subequations}
\begin{align}
\bigd{\vec{u}}{a} &= 0  \\
\bigd{\eta }{a} + \eta \nabla \cdot \vec{u} &= 0 
\,.
\end{align}
\end{subequations}

This can be solved exactly, as we will now see.

This is describing the inertial motion of particles with no external force acting on them. 

It is important to say that this is a total derivative: the variation is zero along the trajectories of the particles.

The velocity of a particle which has a velocity \(\vec{u}_{0}\) at a position \(\vec{q}\) is preserevd. We can integrate the position straightforwardly, since the relation we get is linear.

This, however, is in our weird variables: the real motion is more complicated, but we can map the new variables to the new ones.

We can solve the continuity equation in different ways: we can divide by \(\eta \) to get logarithmic derivatives, so that 
%
\begin{align}
\bigd{\log \eta }{a} =  -\nabla \cdot \vec{u}
\,.
\end{align}

when we integrate, though, we should follow the trajectory: 
%
\begin{align}
\eta (\vec{x}, a) = \eta_0 (\vec{q}) \exp{ - \int_{a_0 }^{a} \dd{a'} \nabla \cdot \vec{u} \qty[\vec{x}(\vec{q}, a',), a']}
\,.
\end{align}

This calculation is done in `The large scale structure of the universe' by Peebles \cite[]{peeblesLargescaleStructureUniverse1980}.

Also, we can use an approach in which we use the Jacobian of the change of coordinates between Lagrangian and Eulerian.

The tensor 
%
\begin{align}
D_{ij} = \pdv[2]{\varphi_0}{q_{i}}{q^{j}}
\,
\end{align}
%
is called the deformation tensor.
The matrix 
%
\begin{align}
\pdv{x^{i}}{q_{j}} = \delta^{i}_{j} - a \pdv[2]{\varphi_0}{q_{i}}{q^{j}}
\,
\end{align}
%
defines the transformation between the old and new coordinates.
The eigenvalues of the transformation tensor are local. 

It can be shown (Doroshevick 1970) that there is a \SI{92}{\percent} probability that at least one eigenvalue is positive.

This is computed using the assumption that \(\varphi_0 \) has a gaussian distribution.

This is important since we can write 
%
\begin{align}
\eta (\vec{x}, a) = \frac{1}{\prod_{i=1}^{3} (1 - \lambda_{i}(\vec{q}) a)}
\,.
\end{align}

Pancakes in terms of shapes, of course, not in terms of ingredients.

The jacobian of the change of coordinates is ill defined if two particles collide. 

At that point, we get nonlinearities, which \emph{are} structure formation.
This is a sort of non-gaussianity we get from this approximation.

\end{document}
