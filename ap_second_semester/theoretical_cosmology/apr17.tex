\documentclass[main.tex]{subfiles}
\begin{document}

\marginpar{Friday\\ 2020-4-17, \\ compiled \\ \today}

There are some additional notes on gravitational dynamics.

We have effects due to curvature: normalization, tilt, tensor modes. 
Tensor modes describe primordial gravitational waves, a stochastic GW background, which as opposed to the CMB has no reason to be thermal. 

These GWs are radiation, so they dilute like \(a^{-4}\). 
The scalar perturbation and primordial gravitational waves mix. There is no way to tell them apart.

We can characterize them by an integrated Sachs-Wolfe effect. 

An interesting place to find topics for a discussion for the exam: Hu's website.

\section{Gravitational instability}

We are going to describe the full nonlinearity in the Newtonian approximation. 

We need to choose coordinates which absorb the expansion of the universe. These are given by \(\vec{r} = a \vec{x}\). 
These are inertial with respect to the background FRW evolution (not with respect to the matter, which is not uniform).
So, we have 
%
\begin{align}
\vec{w} = \vec{\dot{r}} = \frac{\dot{a}}{a} \vec{r}
+ a \dv{\vec{x}}{t} = H \vec{r} + \vec{v}
\,,
\end{align}
%
where \(\vec{v}\) is the peculiar velocity. We also define the gradient 
%
\begin{align}
\nabla_{\vec{r}} = \pdv{}{\vec{r}} = \frac{1}{a} \pdv{}{\vec{x}} = \frac{1}{a} \nabla_{\vec{x}} 
\,.
\end{align}

We can also take the total derivative of a generic function: 
%
\begin{align}
\frac{\DD f(\vec{r}, t)}{\DD t}
= \eval{\pdv{f}{t}}_{\vec{r}} + \pdv{f}{\vec{r}} \dot{\vec{r}} 
\,,
\end{align}
%
[stuff]
which means that we have 
%
\begin{align}
\eval{\pdv{f}{t}}_{\vec{x}} = \eval{\pdv{f}{t}}_{\vec{r}}
+ H \qty(\hat{r} \cdot \nabla_{r}) f
\,.
\end{align}

The equations describing the fluid then are 
%
\begin{subequations}
\begin{align}
\eval{\pdv{f}{t}}_{\vec{r}} + \nabla_{\vec{r}} \qty(\rho \vec{w}) &= 0  \\
\eval{\pdv{\vec{w}}{t}}_{\vec{r}} + \qty(\vec{w} \cdot \nabla_{\vec{r}}) \vec{w} &= - \frac{1}{\rho } \nabla_{\vec{r}} p - \nabla_{\vec{r}} \Phi  \\
\nabla^2_{\vec{r}} \Phi &= 4 \pi G \rho 
\,,
\end{align}
\end{subequations}
%
which are continuity, Euler and gravitational field evolution. 
We perturb: \(\rho = \rho_{b} + \delta \rho \) and \(\Phi = \Phi_{b} + \phi  \). 
We do this so that 
%
\begin{align}
\nabla^2_{r} \Phi_{b} = 4 \pi G \rho_{b} (t)
\,,
\end{align}
%
and 
%
\begin{align}
\nabla^2_{r} \phi = 4 \pi G \delta \rho 
\,.
\end{align}

Let us open the continuity equation. We get 
%
\begin{align}
\pdv{\rho }{t}_{\vec{x}} + 3 H \rho + \frac{1}{a} \nabla_{\vec{x}} \qty(\rho \vec{v}) &= 0
\,.
\end{align}

The Euler equation, instead, gives 
%
\begin{align}
\eval{\pdv{\vec{v}}{t}}_{\vec{r}} + 
H \vec{v}  + \frac{1}{a} \vec{v} \cdot \nabla_{x} \vec{v}
= - \frac{1}{a \rho } \nabla_{x} p
- \frac{1}{a} \nabla_{x} \phi 
\,,
\end{align}
%
since we are neglecting the spatial variation of the background.

Now we Fourier transform \(\delta \), \(\vec{v}\) and \(\phi \).

If we linearize, we get 
%
\begin{subequations}
\begin{align}
k^2 \phi_{\vec{k}} &= - 4 \pi G a^2 \rho_{b} (t) \delta_{\vec{k}}  \\
\dot{\delta}_{\vec{k}} + \frac{i \vec{k} \cdot \hat{v}_{\vec{k}}}{a} &=0  \\
\dot{\vec{v}}_{\vec{k}} + H \vec{v}_{\vec{k}} &= - \frac{i \vec{k}}{a} 
\qty(c_s^2 \delta_{\vec{k}} + \phi_{\vec{k}})
\,,
\end{align}
\end{subequations}
%
where \(c_s^2 = \pdv*{p}{\rho }\).
In the absence of dissipation, vorticity is conserved along stream lines: this is a theorem from Newtonian mechanics. 
This means that 
%
\begin{align}
\dot{\vec{v}}_{\perp} + H \vec{v}_{\perp} = 0
\,,
\end{align}
%
so \(\vec{v}_{\perp} \propto 1 / a\). 
It is not possible a priori to generate vorticity in QFT (?)

We can combine our equations to get 
%
\begin{align}
\ddot{\delta}_{\vec{k}} + 2 H \dot{\delta}_{\vec{k}} + \qty[\frac{c_s^2 k^2}{a^2} - 4 \pi G \rho_{b}] \delta_{\vec{k}} = 0
\,,
\end{align}
%
so the Hubble radius is crucial since it appears in the second term.
We can define a comoving Jeans wavenumber by setting the bracket to zero: 
%
\begin{align}
k_J = \frac{a}{c_s} \sqrt{4 \pi G \rho_{b}}
\,.
\end{align}

For CDM the Jeans wavenumber goes to infinity. 
Even if the DM is no longer thermal, it still has memory of its initial condition through velocity dispersion. 

We can define a mass from the wavenumber.

Take a sphere of radius half the wavelength: then we have 
%
\begin{align}
\frac{4\pi}{3} ( \frac{\lambda}{2} )^3 \rho  = m 
\,.
\end{align}

We get two solutions: \(\delta \propto t^{\alpha }\), with \(\alpha = 2/3\) or \(\alpha = -1\). Therefore, 
%
\begin{align}
v_{\vec{k}} \propto t^{\beta }
\,,
\end{align}
%
where \(\beta = 1/3\) or \(\beta = -4/3\). 

Do galaxies have vorticity? No, that's fake news.

\end{document}
