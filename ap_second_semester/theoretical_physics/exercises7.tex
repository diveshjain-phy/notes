\documentclass[main.tex]{subfiles}
\begin{document}

\section{Chiral interactions}

\marginpar{Saturday\\ 2020-6-20, \\ compiled \\ \today}

(Coming from an exercise, 9.1.e)

Consider the following interaction Lagrangian 
%
\begin{align}
\mathscr{L} _{\text{int}} =\varphi \overline{\psi}_{\ell} \qty(a_{\ell} + i b_\ell \gamma_{5}) \psi_{\ell}
\,,
\end{align}
%
which describes the interaction between a real scalar field and a fermion. \(a_\ell\) and \(b_\ell\) are small real parameters.

It is self-adjoint: its conjugate reads 
%
\begin{align}
\mathscr{L} _{\text{int}} ^\dag &= \varphi \psi_{\ell} ^\dag \qty(a_{\ell} - i b_\ell \gamma_{5} ^\dag) \gamma^{0 \dag} \psi_{\ell} ^\dag  \\
&= \varphi \psi_{\ell} ^\dag \qty(a_{\ell} - i b_\ell \gamma_{5}) \gamma_0 \psi_{\ell}   \marginnote{\(\gamma_5 \) and \(\gamma^0 \) are self-adjoint.} \\
&= \varphi \psi_{\ell} ^\dag \gamma^{0} \qty(a_{\ell} + i b_\ell \gamma_{5}) \psi_{\ell}
\,,
\end{align}
%
where we switched the sign when anticommuting \(\gamma_5\) and \(\gamma^{0}\).

The Feynman rule for this interaction is given from the interaction Hamiltonian; in our case there are no derivative couplings so that is just minus the interaction Lagrangian. The vertex contribution to the diagram is given by the first-order term in the expansion of the \(S\)-matrix: in momentum space we do not have to include the integral over \(\dd[4]{x}\) so the vertex contribution is simply \(-i \mathscr{H} _{\text{int}} = i \mathscr{L} _{\text{int}}\), with all the fields removed: so it is 
%
\begin{align}
i \qty(a_{\ell} + i b_\ell \gamma_{5})
\,.
\end{align}



\end{document}