\documentclass[main.tex]{subfiles}
\begin{document}

\subsection{Classical system with infinite DoF}

\marginpar{Sunday\\ 2020-5-3, \\ compiled \\ \today}

We move from a system of particles to a field, basically. We make the following substitutions: 
\begin{enumerate}
    \item the coordinates \(q_{i}(t)\) become a field, a function of the coordinates \(\varphi (\vec{x}, t)\);
    \item sums over the coordinates, \(\sum _{i}\), become integrals in space: \(\int \dd[3]{x}\);
    \item the Lagrangian \(L(q_{i}, \dot{q}_{i})\) becomes a Lagrangian density \(\mathscr{L} (\varphi , \partial_{\mu } \varphi )\).  
\end{enumerate}

The Lagrangian density is a function of position in space, we can recover the full Lagrangian by integrating: 
%
\begin{align}
L(t) = \int \dd[3]{x} \mathscr{L}(\varphi (\vec{x}, t), \partial_{\mu } \varphi (\vec{x}, t)) 
\,.
\end{align}

The action is now a functional of the fields: 
%
\begin{align}
S [ \varphi ] = \int \dd[4]{x} \mathscr{L} (\varphi, \partial_{\mu } \varphi )
\,,
\end{align}
%
where we integrate in spacetime since we must integrate in \(\dd[3]{x}\) to recover the Lagrangian, and then in time to recover the action from the Lagrangian. 
Note that this yields a number as an output, and takes the whole field in spacetime as an input. As such, it is not a function of the spacetime coordinates.

The domain of integration is usually Minkowski spacetime with some boundary conditions at infinity: we assume that the fields and their derivatives vanish at infinity. 

As in the finite DoF case, we stationarize the action and set \(\delta S =0\). 
What does it mean to vary the field? We assume the variation to be ``synchronous'', that is, 
%
\begin{align}
\varphi '(x) = \varphi (x) + \delta_0 \varphi (x)
\,,
\end{align}
%
so the point in spacetime is not affected: only the field changes.

The variation of the action is given by 
%
\begin{align}
\delta_0 S [\varphi ] &= S [ \varphi + \delta_0 \varphi ] - S[\varphi ]  \\
&= \int \dd[4]{x} \qty{ \mathscr{L} (\varphi + \delta_0 \varphi , \partial_{\mu } \varphi + \delta_0 \partial_{\mu } \varphi ) - \mathscr{L} (\varphi, \partial_{\mu } \varphi )}  \\
&= \int \dd[4]{x } \delta_0 \mathscr{L} (\varphi , \partial_{\mu }\varphi )  \\
&= \int \dd[4]{x} 
\qty{ \pdv{\mathscr{L}}{\varphi } \delta_0 \varphi  + \pdv{\mathscr{L}}{\partial_{\mu } \varphi } \delta_0\partial_{\mu }  \varphi  }  \\
&= \int \dd[4]{x} \qty{\pdv{\mathscr{L}}{\varphi } - \partial_{\mu }\pdv{\mathscr{L}}{\partial_{\mu } \varphi }} \delta_0 \varphi 
\,.
\end{align}

We integrated by parts, and the boundary term vanished because of our boundary conditions on the fields. 
Also, we assumed that \(\delta_0 \qty(\dd[4]{x}) =0 \) in our second step.

In this case, like in the finite-DoF one, \(\qty[\partial_{\mu }, \delta_0 ] =0\). 

Since \(\delta_0 S = 0\) must hold for all \(\delta_0 \varphi \), the rest of the integrand must be equal to zero: these are the Euler-Lagrange equations of the theory, 
%
\begin{align}
\pdv{\mathscr{L}}{\varphi } - \partial_{\mu }\pdv{\mathscr{L}}{\partial_{\mu } \varphi } =0
\,.
\end{align}


\begin{claim}
Two Lagrangians which are equal up to the divergence of a function of the field, \(\partial_{\mu } k^{\mu }(\varphi )\), are equivalent (they have the same EL equations).
\end{claim} 

\begin{proof}
We make this assumption, we can calculate the variation of the action we get after adding this term:
%
\begin{align}
\Delta S = \int \dd[4]{x} \partial_{\mu } k^{\mu } 
\,,
\end{align}
%
so we can calculate this in a region whose radius we send to infinity. Properly speaking, our region should be in the form \(I \times B_R\), the product of an interval in time an a spere in space. We then apply the divergence theorem: the field \(\varphi \) goes to zero at infinity, so as long as \(k^{\mu }\) is regular it will tend to a constant value \(k^{\mu } (0)\), of which we must take the flux on a symmetric region: it will be zero.
\end{proof}

\subsection{Interlude: functional derivatives}

We define a space \(\varphi \) of smooth functions: 
%
\begin{align}
\varphi = \qty{
 f:
\left|
  \begin{array}{rcl}
    M_{4} & \longrightarrow & \mathbb{R} \text{ or } \mathbb{C} \\
    x & \longmapsto & f(x) \\
  \end{array}
\right.
}
\,,
\end{align}
%
and define \textbf{functionals} to be maps from \(\varphi \) to \(\mathbb{R}\) (or \(\mathbb{C}\)), so the space of functionals is the dual of \(\varphi \): a functional \(F\) is in the form
%
\begin{align}
 F:
\left|
  \begin{array}{rcl}
    \varphi  & \longrightarrow & \mathbb{R} \text{ or } \mathbb{C}   \\
    f & \longmapsto & F[f] \\
  \end{array}
\right.
\,.
\end{align}

We are giving general definitions, but the case we will be interested in is usually the one of real-valued functionals.

We encountered two functionals so far: the Lagrangian is a functional of the field \(\varphi (\vec{x}, \overline{t})\) for a fixed \(\overline{t}\); the action is a functional of the field \(\varphi (x^{\mu })\).

We can vary a functional: if we are given a function \(g \in \varphi \), we can select a small perturbation \(\delta g\) so that we can define 
%
\begin{align}
\delta S[g] =  \eval{\dv{}{\epsilon } S[g + \epsilon (\delta g )]}_{\epsilon =0}
= \eval{S[g + \delta g] - S[g]}_{\text{first order}}
\,.
\end{align}

The functional derivative of \(F[g]\), which is denoted by \(\fdv*{f}{g}\), is defined by the relation 
%
\begin{align}
\delta F = \int \fdv{F}{g} \delta g \dd{x}
\,,
\end{align}
%
which should be interpreted as the functional analogue of the formula for the differential 
%
\begin{align}
\dd{F} (x_{i}) = \sum _{i} \pdv{F}{x_{i}} \dd{x_{i}} 
\,.
\end{align}

\begin{claim}
The functional derivative is indeed a derivative: it is linear 
%
\begin{align}
\fdv{}{g} \qty(\alpha F + \beta  G) = \alpha \fdv{F}{g} + \beta \fdv{G}{g}
\,;
\end{align}
%
zero for constants: \(\fdv*{C}{g} = 0 \) if \(C[g] \equiv C \in \mathbb{R}\); and finally it satisfies the Leibniz rule: 
%
\begin{align}
\fdv{}{g} \qty(FG)= \fdv{F}{g} G + F \fdv{G}{g}
\,.
\end{align}
\end{claim}

\begin{proof}
First of all, what to we mean by sum and product of functionals? 
The intuitive definition to give is that the sum of two functionals \(F[g]\) and \(G[g]\) is the functional \(F+G\) such that \((F+G)[g] = F[g] + G[g]\).
For the product we adopt a similar definition. 

Then, we get 
%
\begin{align}
\delta (\alpha F + \beta G)[g] &= \eval{\qty(\alpha F[g+ \delta g] + \beta G[g + \delta g]) - \qty(\alpha F[g] + \beta G[g])}_{\text{lin}}
\\
&= \alpha \eval{\qty(F[g + \delta g] - F[g])}_{\text{lin}} + \beta \eval{\qty(G [g + \delta g] - G[g])}_{\text{lin}}  \\
&= \alpha \delta F [g] + \beta \delta G[g]
\,.
\end{align}

For a constant we get \(C[g + \delta g] = C[g]\), so the differential is zero. 

For the Leibniz rule, we have 
%
\begin{align}
\delta (FG)[g] &= F[g + \delta g] G[g + \delta g] - F[g] G[g]  \\
&= 
\,,
\end{align}
%
\todo[inline]{To review\dots the definitions do not convince me}

A reference: \cite[]{engelDensityFunctionalTheory2011}.
\end{proof}

A possible functional we can apply to smooth functions is the \emph{evaluation} functional: \(F_{\mathbb{1}}[g] = g(y )\); we can represent it as an integral by introducing the delta function, 
%
\begin{align}
F_{\mathbb{1}} [g]  = \int \delta (x - y) g(x) \dd{x}
\,.
\end{align}
%
This is sometimes referred to as the \emph{identity functional}, since it does not alter the function beyond changing the name of the variable.

By the definition of the functional derivative, we find 
%
\begin{align}
F_{\mathbb{1}} [g] = \int \delta (x-y) g(x) \dd{x} = \int \fdv{F_{\mathbb{1}}}{g} g(x) \dd{x}
\,,
\end{align}
%
so we must identify 
%
\begin{align}
\fdv{F_{\mathbb{1}}}{g(x)} = \fdv{g(y)}{g(x)} = \delta (x -y )
\,.
\end{align}

In the 4D case we have, analogously: 
%
\begin{align}
\fdv{\varphi (x^{\mu })}{ \varphi (y^{\mu })} = \delta^{(4)} (x^{\mu } - y^{\mu })
\,,
\end{align}
%
and if we fix one or more of the coordinates the dimension of the \(\delta \) function diminishes accordingly. 

\begin{claim}
We can derive the Euler-Lagrange equations using functional derivatives.
\end{claim}

\todo[inline]{Add discussion of the formal stuff}

\subsection{Hamiltonian formulation}

We define the \textbf{canonically conjugate momentum} as 
%
\begin{align}
\pi (\vec{x}, t) = \fdv{L(t)}{\dot{\varphi}_{t} (\vec{x})} = \pdv{\mathscr{L}}{\partial_0 \varphi (\vec{x})}
\,,
\end{align}
%
so that we can introduce the Hamiltonian density via the Legendre transform: 
%
\begin{align}
\mathscr{H} = \pi (\vec{x}, t) \partial_0 \varphi (\vec{x}, t) - \mathscr{L} (\varphi , \partial_{\mu } \varphi )
\,,
\end{align}
%
and the Hamiltonian as its integral 
%
\begin{align}
H(t) = \int \dd[3]{x} \mathscr{H}
\,.
\end{align}

The Hamiltonian is a functional of the fields \(\pi \) and \(\varphi \): 
%
\begin{align}
H(t) = H[\pi_{t}, \varphi_{t}]
\,.
\end{align}

\begin{claim}
We can calculate Hamilton's equations from the functional variation of \(H\): they read 
%
\begin{align}
\dot{\varphi} (\vec{x}, t) = \fdv{H}{\pi_{t}(\vec{x})}
\qquad \text{and} \qquad
\dot{\pi}(\vec{x}, t) = - \fdv{H}{\varphi_{t}(\vec{x})}
\,.
\end{align}
\end{claim}

We can also write the HE with the Poisson brackets: the definition must however be adapted; it will read 
%
\begin{align}
\qty{F, G}_{t} = \int \dd[3]{x} \qty(\fdv{F}{\varphi_{t}(\vec{x})} \fdv{G}{\pi_{t}(\vec{x})} - \fdv{F}{\pi_{t}(\vec{x})} \fdv{G}{\varphi_{t}(\vec{x})})
\,,
\end{align}
%
\todo[inline]{which yields a function of the coordinates \(\vec{y}\), since the functional derivatives are functions of \(\vec{y}\) as well as of \(\vec{x}\), and we marginalize over \(\vec{x}\)?}

Then, Hamilton's equations will read 
%
\begin{align}
\dot{\varphi}_{t} = \qty{\varphi, H}_{t}
\qquad \text{and} \qquad
\dot{\pi}_{t} = -\qty{\pi , H}(t)
\,.
\end{align}

The brackets of the fields \(\varphi \) and \(\pi \) are the same as in the finite-DoF case, substituting \(\delta_{j}^{i}\) with \(\delta^{(3)} (\vec{x}- \vec{y})\).

\end{document}
