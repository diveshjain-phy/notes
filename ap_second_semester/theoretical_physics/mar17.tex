\documentclass[main.tex]{subfiles}
\begin{document}

\section{The Klein paradox}

\marginpar{Thursday\\ 2020-3-19, \\ compiled \\ \today}

The fact that it was not possible to derive a conserved charge for the KG equation is not just a mathematical inconvenience: we will use a gedankenexperiment to show the physical consequences of this. 

Consider the scattering of a particle, which is described by a pure plane wave, on a an electromagnetic potential step.
Suppose we are in a frame in which \(qA^{\mu } = (V(x), \vec{0})\); if we are not in such a frame we can always perform a boost so that we are.
Also suppose that the potential looks like a step: \(V(x) = V_0 [x \geq 0]\). Here ``\(x\)'' refers to a 1d spatial coordinate. 

Now, suppose that we have an incoming wave with energy \(\omega \) and momentum \(k_{x}\) in the \(x<0\) region. 
Upon impact, there will in general be a reflected wave in the \(x<0\) region and a transmitted wave in the \(x>0\) region. We will call the former \(\varphi_{1}\) and the latter \(\varphi_{2}\). 

We can decompose both of these into a time and space exponential: \(\varphi_{i} (t, x) = e^{-i \omega t} \chi_{i}(x)\), for \(i = 1, 2\). 

Now, the incoming wave contributes to the global wavefunction with an exponential \(e^{ik_x x}\) in the \(x<0\) region. 
In the same region we can find the reflected wave, which has the opposite momentum; also, its amplitude is less than that of the incoming wave. We write its contribution to \(\chi_{1}\) as \(r e^{-i k_x x}\), where \(r\) is the reflection coefficient. 

By a similar reasoning, in the \(x>0\) region we will have a contribution \(t e^{i k_x^{\prime } x}\): the direction of propagation is the same as the incoming wave, the momentum might be different. In the end, our wavefunction looks like 
%
\begin{align}
\varphi (t, x) &= \varphi_1 [x\leq 0] + \varphi_2 [x \geq 0] \\
&= e^{-i \omega t} \qty[\chi_1 [x\leq 0] + \chi_2 [x \geq 0]] \\
&=
 e^{-i \omega t} \qty[ \qty(e^{i k_x x} + r e^{-i k_x x})[x\leq 0] + t e^{i k_x^{\prime }x } [x\geq 0]]
\,.
\end{align}

This wavefunction will need to satisfy the KG equation everywhere, however the equation has a different form in the two regions: if \(\DD^{\mu } = \partial^{\mu } + iqA^{\mu }\), then we can say that the two equations are 
%
\begin{align}
  \qty[\partial^{\mu } \partial_{\mu } + M^2] \varphi_1 &= 0 \\
  \qty[\DD^{\mu } \DD_{\mu } + M^2] \varphi_2 &= 0 
\,,
\end{align}
%
so we can insert the solutions into the equations: for the region without potential we get
%
\begin{align}
\qty[\partial^{\mu } \partial_{\mu } + M^2] \varphi_1 &= \qty[+ \partial_0^2 - \partial_{x}^2 + M^2 ] \qty[e^{-i \omega t} \qty(e^{ik_{x}x} + r e^{-ik_x x})]   \\
&= (-i\omega)^2 \varphi_1 + M^2 \varphi_1 - e^{-i \omega t}\partial_{x}^2  \qty(e^{ik_{x}x} + r e^{-ik_x x}) \\
&= -\omega^2 \varphi_1 + M^2 \varphi_1 - e^{-i \omega t}  \qty((ik_x)^2 e^{ik_{x}x} + r (-ik_x)^2 e^{-ik_x x}) \\
&= \qty[-\omega^2 + M^2 + k_x^2 ] \varphi_1  
\,,
\end{align}
%
notice that even though the function has components with momentum in either direction it still is an eigenfunction of the operator \(\square\)! 
For the other region we can use the fact we discussed earlier: with the minimal substitution, the momentum becomes \(p^{\mu } = i \partial^{\mu } \rightarrow i \DD^{\mu } = p^{\mu } - qA^{\mu }\), so in our case we will have \(p^{0} \rightarrow \omega_{k} - q A^{0} = \omega_{k} - V_0 \), while \(p^{i} \rightarrow \vec{k}'\), since \(\vec{A} = 0\). We can substitute the energy directly since the potential is constant, so there is no \(\partial_0 V_0 \) term. 
Then, the rest of the calculation is perfectly analogous, with \(k_{x}'\) instead of \(k_x\).

So, we have the two relations 
%
\begin{align}
- \omega_k^2 + k_x^2 + M^2 = 0
\qquad \text{and} \qquad
- \qty(\omega_{k} - V_0 )^2 + k_x^{\prime 2}  + M^2 = 0
\,.
\end{align}

We choose the solutions in which the wave is propagating towards increasing \(x\): so we get 
%
\begin{align}
k_x = \sqrt{\omega_{k}^2  - M^2} 
\qquad \text{and} \qquad
k_x^{\prime 2} = \sqrt{\qty(\omega_{k} - V_0)^2 - M^2}
\,.
\end{align}

Now we must make these consistent with each other, by imposing that at the border the function be \(\mathcal{C}^{1}\). 

Why do we fix only the first derivative? Probably something to do with the fact that the KG eq is second order, so we would need \(\square = \DD^{\mu } \DD_{\mu }\) at the boundary, which cannot be the case.

We can work with the spatial components \(\chi_{i}\), since the time evolution factors. So, for continuity we get that \(\chi_1 (0) = \chi_2 (0)\) implies:
%
\begin{align}
1+r = t
\,,
\end{align}
%
while for differentiability we get that \(\partial_{x} \chi_1(0) = \partial_{x} \chi_2 (0)\) implies:
%
\begin{align}
i k_x  - r i k_x = i t k_x'
\implies
\frac{k_x'}{k_x} = \frac{1- r}{t}
\,.
\end{align}

Now, what do we know of these variables? Well, from the dispersion relations if we fix the mass \(M\), the energy \(\omega \) and the potential \(V_0 \) we have \(k_x\) and  \(k_x^{\prime }\). 
Then, from the two equations we have found we can calculate \(r\) and \(t\); their expressions are exactly those found in the nonrelativistic case. Since \(1-r = 2-t\) we have 
%
\begin{align}
\frac{k_{x}'}{k_x} = \frac{2}{t} - 1 \implies 
\frac{k_x^{\prime } + k_x}{k_x} = \frac{2}{t}
\implies t = \frac{2 k_x}{k_x' + k_x}
\qquad \text{and} \qquad
r = t-1 = \frac{k_x - k_x'}{k_x' + k_x}
\,.
\end{align}

Now, we shall discuss the probability currents for the two regions: the formula is always 
%
\begin{align}
j^{\mu } = \frac{i}{2} \varphi^{*} \partial^{\mu } \varphi  + \frac{q}{2} A^{\mu } \varphi^{*} \varphi  + \text{c. c.}
\,,
\end{align}
%6
but in the first region the potential is identically zero. 

One non-obvious thing to consider is that, in general, we could have \(k_x'\) be either real or imaginary, since while \( \omega^2 - M^2\) must be \(>0\) there is no such constraint on \((\omega- V_0 )^2 - M^2\). The calculation is as follows:
%
\begin{align}
j^{0}_{(1)} &= \frac{i}{2} e^{i \omega t} \partial^{0} e^{-i \omega t} \abs{\chi_1 }^2  + \text{c. c.} = \frac{2}{2} \omega \abs{\chi_1 }^2 
\marginnote{The \(\chi_1 \) is not time-dependent, we get two terms by adding the conjugate}  \\
j_1 \overset{\text{def}}{=} j^{x}_{(1)} &= - \frac{i}{2} \qty(e^{ik_x x} + r e^{-ik_x x}) \partial_{x} \qty(e^{ik_x x} + r e^{-ik_x x}) + \text{c. c. }  \marginnote{Minus sign from lowerig the index}\\
&= - \frac{2}{2} i (i k_x ) \qty(e^{ik_x x} + r e^{-ik_x x}) \qty(e^{ik_x x} - r e^{-ik_x x}) = k_x \qty(1 - \abs{r}^2)  \\
j^{0}_{(2)} &= \frac{i}{2} t^{*} e^{i \omega t} e^{-i k_x^{\prime *}x} t (-i \omega ) e^{-i \omega t} e^{i k_x^{\prime } x} + \frac{1}{2} V_0 \varphi^{*} \varphi  + \text{c. c.}  \\
&= (\omega - V_0 ) \abs{t}^2 e^{i \qty(k_x^{\prime }- k_x^{\prime *})x}  \\
j_2 \overset{\text{def}}{=} j^{x}_{(2)} &= -\frac{i}{2} t^{*} e^{i \omega t} e^{-i k_x^{\prime *} x} 
(i k_x^{\prime }) e^{-i \omega t} e^{i k_x^{\prime }x } + \text{c. c. }  \\
&= \frac{1}{2} \abs{t}^2 e^{i \qty(k_x^{\prime } - k_x^{\prime *})x} k_x^{\prime } + \text{c. c. }  \\
&= \frac{k_x^{\prime } + k_x^{\prime *}}{2} \abs{t}^2 e^{i \qty(k_x^{\prime } - k_x^{\prime *})x}
\,.
\end{align}

We can define the reflection and transmission coefficients, which are probabilities:
%
\begin{align}
\mathcal{R} = \frac{j _{\text{in}} - j_1}{j _{\text{in}}} = \abs{r}^2
\qquad \text{and} \qquad
\mathcal{T} = \frac{j_2}{j _{\text{in}}}
= \frac{k_x^{\prime } - k_x^{\prime *}}{2 k_x} \abs{t}^2 e^{i (k_x^{\prime } -k_x^{\prime *})x}
\,,
\end{align}
%
where \(j _{\text{in}} = k_x\). 
\todo[inline]{What is the motivation behind these definitions?}



\end{document}