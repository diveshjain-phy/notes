\documentclass[main.tex]{subfiles}
\begin{document}

\chapter{Interacting field theories}

We have studied free field theories for spin-0, spin-\(1/2\) and spin-1 particles. Now we introduce interaction terms between these, starting from the classical description of interaction. 

\section{Interactions in classical field theories}

\marginpar{Thursday\\ 2020-6-11, \\ compiled \\ \today}

\subsection{Scalar field self-interactions}

Scalar fields have acted like systems of decoupled harmonic oscillators up to now. So, we introduce the complication of a self-interaction between the field: we write a Lagrangian density like 
%
\begin{align}
\mathscr{L} = \underbrace{\frac{1}{2} \qty(\partial_{\mu }\varphi ) \qty(\partial^{\mu} \varphi ) - \frac{1}{2} m^2 \varphi^2}_{\mathscr{L}_{0}} \underbrace{- V(\varphi )}_{\mathscr{L} _{\text{int}}}
\,.
\end{align}

The added interaction term is a potential in the form 
%
\begin{align}
V(\varphi ) = k \varphi^3 + \lambda \varphi^{4} + \mu \varphi^{5} \dots
\,,
\end{align}
%
with an arbitrary number of added positive powers of \(\varphi \). 

We require \(V(\varphi )\) to have a stable minimum, which constrains the coefficients. 
We will discuss this aspect in more detail later, when we treat Spontaneous Symmetry Breaking \ref{sec:ssb}. 

Since the dimensions of the field are those of a mass and the Lagrangian density must be dimensionally a mass to the fourth, the coefficients must have dimensions of decreasing integer powers of \(M\): for the term \(\alpha_{i} \varphi^{i}\), the dimension of \(\alpha_{i}\) will be \(M^{4-i} = M^{D}\). 

Terms whose mass dimension has an exponent \(D \geq 0 \) are said to be \textbf{renormalizable}.
So, in our case the only renormalizable terms are the first two, \(k \varphi^{3 } + \lambda \varphi^{4}\). 

For a complex field the potential needs to be expressed in terms of \(\varphi^{*} \varphi \) in order to be \(U(1)\) gauge invariant, so the only renormalizable term we can add is \(\lambda (\varphi^{*} \varphi )^2\). 

\subsection{Dirac field self-interactions}

The interaction term is added to the Dirac Lagrangian as follows: 
%
\begin{align}
\mathscr{L} 
= \underbrace{\overline{\psi} \qty(i \slashed{\partial} - m) \psi }_{\mathscr{L}_{0}}
+ 
\underbrace{G \qty(\overline{\psi} \Gamma \psi ) \qty(\overline{\psi} \Gamma \psi )}_{\mathscr{L} _{\text{int}}}
\,,
\end{align}
%
where \(\Gamma \) is a (1, 1) tensor in spinorial space. 

Fermi theory, for example, is given by 
%
\begin{align}
\mathscr{L} _{\text{int}} = \frac{G_F}{\sqrt{2}}
\qty(\overline{\psi}_{p} \gamma^{\mu } \psi_{n}) \qty(\overline{\psi}_{e} \gamma_{\mu } \psi_{\nu })
\,,
\end{align}
%
and it gives an effective description of beta decay, 
%
\begin{align}
n \to p + e^{-} + \overline{\nu}_{e}
\,.
\end{align}

Since \(\psi \) has dimension \(M^{3/2}\), we find that the coupling constant \(G_F\) must have dimension \(M^{-2}\): so, this interaction is \textbf{not renormalizable}.

\subsection{Scalar - Dirac field interaction (Yukawa)}

We want to describe an interaction between a real scalar field \(\varphi \) and a spinor field \(\psi \). The interaction Lagrangian is: 
%
\begin{align}
\mathscr{L} = \frac{1}{2} \qty(\partial_{\mu } \varphi ) \qty(\partial^{\mu} \varphi )
- \frac{1}{2} m^2\varphi^2
+ \overline{\psi} \qty(i \slashed{\partial} - m)\psi 
+ y_{s}  \varphi \overline{\psi} \psi 
+ y_{p} \varphi \overline{\psi} \gamma_{5} \psi 
+ \omega_{1} \varphi^2 \overline{\psi} \psi 
\,,
\end{align}
%
where \(y_s\) and \(y_p\) are known as the Yukawa couplings; one is a scalar, the other is a pseudoscalar (because of the presence of \(\gamma_{5}\)); both are renormalizable. 

We always require Poincaré invariance, sometimes we do not require invariance under discrete symmetries like parity. 

\subsection{Massless vector - Dirac field interaction}

The base Lagrangian is the sum of those describing the fields: 
%
\begin{align}
\mathscr{L}_{0}
= - \frac{1}{4} F^{\mu \nu } F_{\mu \nu }
+ \overline{\psi} \qty(i \slashed{\partial} - m ) \psi 
\,,
\end{align}
%
while the leading terms in the interaction term will be: 
%
\begin{subequations}
\begin{align}
\begin{split}
\mathscr{L} _{\text{int}}
&= g_{v } \qty(\overline{\psi} \gamma^{\mu } \psi ) A_{\mu }
+ g_{A} \qty(\overline{\psi} \gamma^{\mu } \gamma_{5} \psi )A_{\mu }
+ c_{v } \qty(\overline{\psi} \psi ) A^{\mu } A_{\mu } 
\\
&\phantom{=}\ 
+ i c_A \overline{\psi} \gamma_{5} \psi A_{\mu }A^{\mu }
+ d_{v } \qty(\overline{\psi} \sigma^{\mu \nu }\psi )F_{\mu \nu }
+ i d_A  \qty(\overline{\psi} \sigma^{\mu \nu } \gamma_5 \psi) F_{\mu\nu  } + \dots
\,.
\end{split}
\end{align}
\end{subequations}

To find the dimensionality, recall that \([A_{\mu }] = M^{1}\), so we get that \(g_v \) and \(g_A\) have dimension \(M^{0}\), while the other coefficients have dimension \(M^{-1}\). 
If we include only the renormalizable terms, we have 
%
\begin{subequations}
\begin{align}
\mathscr{L} _{\text{int}}^{\text{ren}}
&= g_{v } \qty(\overline{\psi} \gamma^{\mu } \psi ) A_{\mu }
+ g_{A} \qty(\overline{\psi} \gamma^{\mu } \gamma_{5} \psi )A_{\mu }  \\
&= g_{L } \qty(\overline{\psi} \gamma^{\mu }_{L} \psi ) A_{\mu }+
 g_{R } \qty(\overline{\psi} \gamma^{\mu }_{R} \psi ) A_{\mu }
\,,
\end{align}
\end{subequations}
%
where we defined 
%
\begin{align}
\gamma^{\mu }_{L} = \gamma^{\mu } \frac{1 - \gamma_5 }{2} 
\qquad \text{and} \qquad
\gamma^{\mu }_{R} = \gamma^{\mu } \frac{1 + \gamma_5 }{2} 
\,,
\end{align}
%
so the new coefficients are 
%
\begin{align}
g_R = g_{v } + g_A \qquad \text{and} \qquad
g_L = g_{\nu } - g_A
\,.
\end{align}

\(g_v\) is known as the vector coupling, while \(g_A\) is the axial coupling. \(g_L\) and \(g_R\) are the left and right-handed couplings. We distinguish:
\begin{enumerate}
    \item the vector-spinor interaction is \textbf{vector-like} if \(g_A = 0\), or equivalently if \(g_R = g_L\);
    \item the vector-spinor interaction is \textbf{chiral} if \(g_A \neq 0\), or \(g_R \neq g_L\).
\end{enumerate}

In QED, the interactions between the photon and the charged fermions are vector-like. 
On the other hand, the weak interaction between the \(W\), \(Z\) bosons and fermions are chiral. 

The QED Lagrangian is a vector-spinor interaction Lagrangian:
%
\begin{align}
\mathscr{L} _{\text{QED}} = - \frac{1}{4} F^{\mu \nu } F_{\mu \nu }
+ \overline{\psi} \qty(i \slashed{\partial} - m) \psi 
- q \overline{\psi} \gamma^{\mu } \psi A_{\mu } 
\,.
\end{align}

\begin{claim}
The QED Lagrangian is invariant under the transformation: 
%
\begin{subequations}
\begin{align}
A^{\prime \mu } (x)  &= A_{\mu } (x) + \partial_{\mu } \alpha (x)  \\
\psi^{\prime }(x) &= e^{-iq \alpha (x)} \psi (x)
\,.
\end{align}
\end{subequations}

We can rewrite this Lagrangian with a covariant derivative \(\DD_{\mu } = \partial_{\mu } + i q A_{\mu }\): it then becomes 
%
\begin{align}
\mathscr{L} _{\text{QED}} = - \frac{1}{4} F^{\mu \nu } F_{\mu \nu }
+ \overline{\psi} \qty(i \slashed{\DD} - m) \psi 
\,.
\end{align}
\end{claim}

\begin{proof}
The kinetic term of the vector and the mass term of the fermion are unaffected by the interaction, and are manifestly gauge invariant. 
Let us instead consider the interesting bits: 
%
\begin{subequations}
\begin{align}
\overline{\psi} i \slashed{\partial} \psi &\to e^{iq \alpha } \overline{\psi} i \slashed{\partial} \qty(e^{-iq \alpha } \psi )  \\
&= \overline{\psi} i \slashed{\partial} \psi 
+ \overline{\psi} i \gamma^{\mu } \qty(-iq \partial_{\mu } \alpha ) \psi  \\
&= \overline{\psi} i \slashed{\partial} \psi
+ \hlc{blue}{q \overline{\psi} \gamma^{\mu } \psi   \partial_{\mu } \alpha }
\,.
\end{align}
\end{subequations}

On the other hand, the interaction term transforms like 
%
\begin{subequations}
\begin{align}
- q \overline{\psi} \gamma^{\mu } \psi A_{\mu } &\to -q \overline{\psi} \gamma^{\mu } \psi \qty(A_{\mu } + \partial_{\mu } \alpha )  \\
&= -q \overline{\psi} \gamma^{\mu } \psi A_{\mu } - 
\hlc{blue}{q \overline{\psi} \gamma^{\mu } \psi \partial_{\mu } \alpha }
\,,
\end{align}
\end{subequations}
%
which precisely balances the kinetic term of the fermion. 
\end{proof}

\subsubsection{Complex scalar - massless vector interaction}

In a similar way we can write the interaction between a complex scalar field and a photon (massless vector): the Lagrangian is 
%
\begin{align}
\mathscr{L} _{\text{scalar QED}}
= \frac{1}{4} F^{\mu \nu } F_{\mu \nu } 
+ \qty(\DD^{\mu } \varphi )^{*} \qty(\DD_{\mu } \varphi )
\,.
\end{align}

\begin{claim}
This Lagrangian is \(U(1)\) gauge symmetric, and 
its Euler-Lagrange equations read: 
%
\begin{subequations}
\begin{align}
\DD_{\mu } \DD^{\mu } \varphi &= 0   \\
\square A^{\sigma } + \partial^{\sigma } \qty(\partial_{\rho } A^{\rho })
-iq \qty(\varphi^{*} \partial_{\rho } \varphi - \varphi \partial_{\rho } \varphi^{*})
+ 2 q^2 A_{\rho } \varphi^{*} \varphi 
&=0
\,.
\end{align}
\end{subequations}
\end{claim}

\todo[inline]{Might be able to clean up the equation for the EM field?}

\begin{proof}
Let us see how the term \(\DD_{\mu } \varphi \) changes with a \(U(1)\) transformation: 
%
\begin{subequations}
\begin{align}
\partial_{\mu } \varphi + i q A_{\mu } \varphi &\to 
\partial_{\mu } \qty(e^{-iq \alpha } \varphi )
+ i q \qty(A_{\mu } + \partial_{\mu }\alpha ) e^{-iq \alpha } \varphi  \\
&= e^{-iq \alpha } \partial_{\mu } \varphi 
- \hlc{teal}{e^{-iq \alpha } \varphi (iq \partial_{\mu } \alpha )} 
+e^{-iq \alpha } i q A_{\mu } \varphi 
+ \hlc{teal}{(iq \partial_{\mu }\alpha ) e^{-iq \alpha } \varphi}  \\
&= e^{-iq \alpha } \qty(\partial_{\mu } \varphi + i q A_{\mu } \varphi) 
\,.
\end{align}
\end{subequations}

So, this term changes by a global phase: since it is contracted with its conjugate, its appearance in the Lagrangian is invariant. 
The kinetic EM term is invariant as usual. 
    
If we write out the Lagrangian explicitly we get: 
%
\begin{align}
\mathscr{L}_{\text{scalar - QED}}
= - \frac{1}{4} F^{\mu \nu } F_{\mu \nu } 
+ \partial^{\mu } \varphi^{*} \partial_{\mu } \varphi 
- iq A_{\mu } \qty(\varphi^{*}\partial^{\mu } \varphi - \varphi \partial^{\mu } \varphi^{*})
+ q^2 A^{\mu }A_{\mu } \varphi^{*} \varphi 
\,.
\end{align}

We can write the EL equations differentiating both with respect to \(A_{\sigma }\) and \(\varphi^{*}\). We find 
%
\begin{subequations}
\begin{align}
-iq \qty(\varphi^{*} \partial_{\rho } \varphi - \varphi \partial_{\rho } \varphi^{*})
+ 2 q^2 A_{\rho } \varphi^{*} \varphi 
+ \partial_{\sigma } \qty(\partial^{\sigma }A_{\rho } - \partial_\rho A^{\sigma })&= 0   \\
-iq A_{\mu } \partial^{\mu } \varphi 
+ q^2 A^{\mu } A_{\mu } \varphi 
- \partial_{\sigma } \qty(
\partial^{\sigma } \varphi 
+ i q A^{ \sigma } \varphi 
)
&=0 
\,,
\end{align}
\end{subequations}
%
where we have made use of the expression \eqref{eq:derivative-kinetic-term-vector}. 
The second equation can we written in a more compact way if we substitute back in the expression for \(\DD_{\mu }\): we get 
%
\begin{subequations}
\begin{align}
0 &=
-i q A_{\mu } \qty(\partial^{\mu } + i q A^{\mu } ) \varphi 
- \partial_{\sigma } \qty(\partial^{ \sigma } + i q A^{\sigma }) \varphi   \\
&= -iq A_{\mu } \DD^{\mu } \varphi 
- \partial_{\sigma } \DD^{\sigma } \varphi  \\
&= - \DD_{\mu } \DD^{\mu } \varphi =0
\,.
\end{align}
\end{subequations}
\end{proof}

\section{Quantizing interactions}

Let us now try to quantize these classical interacting theories. 
We start with the self-interacting real scalar field, with \(V(\varphi ) = \lambda \varphi^{4} / 4! \). Its EOM read: 
%
\begin{align}
\qty(\square + m^2) \varphi = - \frac{\lambda}{3!} \varphi^3
\,.
\end{align}

The Hamiltonian description is derived as usual, with \(\pi = \partial_0 \varphi \) and 
%
\begin{align}
\mathscr{H} =  \pi \partial_0 \varphi 
- \mathscr{L} =  \mathscr{H}_{0} + \mathscr{H} _{\text{int}}
= \mathscr{H}_{0} + V(\varphi )
\,.
\end{align}

We also have the usual Poisson brackets and Hamilton equations. 
So, since the field is spin-0 we quantize with commutators, imposing 
%
\begin{align}
\qty[\varphi (\vec{x}, t) , \pi (\vec{y}, t)] = i \delta^{(3)} (\vec{x}-\vec{y})
\,.
\end{align}

The other commutators vanish. 
This is all fine, we encounter no issues.

The problem arises because of the fact that there is \textbf{no analytic solution to the EOM}. 
So, we have no way to relate the operators \(\varphi \) and \(\pi \) with \(a\) and \(a ^\dag\). We cannot directly extend the reasoning we did before (finding the harmonic oscillator algebra, defining the number operator\dots) to this case, at least not straightforwardly. 

So, we \textbf{perturb}. 

\subsection{The interaction picture}

The interaction picture is an alternative to the Schrödinger and Heisenberg ones which is useful when dealing with interacting theories, for which the Hamiltonian can be written as 
%
\begin{align}
H = H_0 + H _{\text{int}}
\,,
\end{align}
%
so that we can define a free evolution operator 
%
\begin{align}
U_0 = e^{-i H_0 \Delta t}
\,.
\end{align}

Then, in the interaction picture we take the time-evolving Schrödinger state kets, \(\ket{\psi (t)}_{S}\), and evolve them \textbf{back} with the free Hamiltonian: 
%
\begin{align}
\ket{\psi (t)}_{I} = U_0 ^\dag \ket{\psi (t)}_{S}
\,,
\end{align}
%
so that the states only evolve with the interaction term. 
On the other hand, we have operators (which are stationary in the Schrödinger picture) evolve with the free Hamiltonian:
%
\begin{align}
A_{I}(t) = U_0 ^\dag (t) A U_0 (t)
\,,
\end{align}
%
so that expectation values are preserves: 
%
\begin{subequations}
\begin{align}
\expval{A} &= \bra{\psi (t)}_{I} A_I (t) \ket{\psi (t)}_{I}   \\
&= \bra{\psi (t)}_{S} U_0 (t) U_0 ^\dag(t) A U_0 (t) U_0 ^\dag \ket{\psi (t)}_{S}  \\
&=  \bra{\psi (t)}_{S} A \ket{\psi (t)}_{S}
\,.
\end{align}
\end{subequations}

We usually write everything with respect to base kets which are eigenstates of the free Hamiltonian. 
This picture allows us to ``factor out'' the uninteresting free evolution, and focus on the interesting interaction part. 

Operators obey the evolution law 
%
\begin{align}
i \dv{}{t} A_I (t) = \qty[X_I(t), H_0 ]
\,.
\end{align}

Since field operators are operators, they evolve with the free Hamiltonian, we can write them with the free solution: 
%
\begin{align}
\varphi_{I} (X) \sim \int a(k) e^{-ikx} + a ^\dag(k) e^{ikx}
\,.
\end{align}

The total evolution of a state can be written as 
%
\begin{subequations}
\begin{align}
i \dv{}{t} \ket{\psi (t)}_{I} &= i \dv{}{t} \qty(U_0 U) \ket{\psi (t_0 )} \\
&= U_0 ^\dag H _{\text{int}} U_0 \ket{\psi (t_0 )}_{I}
\,,
\end{align}
\end{subequations}
%
so if we evolve the interaction Hamiltonian as \(H _{\text{int}} (t) = U_0 ^\dag H _{\text{int}} U_0 \) we recover a Schrödinger-like equation: 
%
\begin{align}
i \dv{}{t} \ket{\psi (t)}_{I} = H _{\text{int}}^{I} \ket{\psi (t)}_{I}
\,.
\end{align}

\begin{claim}
The evolution operator for states, \(U_I (t)\), can be written as 
%
\begin{align}
U_I (t, t_0 ) = U_0^\dag(t,0)  U (t, t_0 ) U_0 (t_0, 0) 
\,,
\end{align}
%
and it is unitary, satisfies \(U_{I}(t_2, t_0 ) = U_I (t_2 , t_1 ) U_I (t_1 , t_0 )\) and \(U_I (t, t_0 ) = U ^\dag_I (t_0, t)\).  
\end{claim}

% \todo[inline]{The last one looks weird\dots}

\begin{proof}
First of all, we derive the expression for the evolution operator: a state \(\ket{\psi (t)}_{I}\) evolves as
%
\begin{subequations}
\begin{align}
\ket{\psi (t)}_{I} &= U_0 ^\dag (t, 0) \ket{\psi (t)}_{S}  \\
&= U_0 ^\dag (t, 0) U (t, t_0 ) \ket{\psi (t_0 )}_{S}  \\
&= U_0 ^\dag (t, 0) U(t, t_0 ) U_0 (t_0, 0) \ket{\psi (t_0 )}_{I}  \\
&= U_{I} (t, t_0 ) \ket{\psi (t_0 )}_{I}
\,.
\end{align}
\end{subequations}

To show unitarity: 
%
\begin{subequations}
\begin{align}
U_I (t, t_0 ) ^\dag U_I (t, t_0 )
&= 
U_0 ^\dag(t_0 , 0)  
U ^\dag (t, t_0 )  
U_0 (t, 0) 
U_0 ^\dag (t, 0)  
U (t, t_0 ) 
U_0 (t_0 , 0)  \\
&= 
U_0 ^\dag(t_0 , 0)  
U ^\dag (t, t_0 )  
U (t, t_0 ) 
U_0 (t_0 , 0)  \\
&= 
U_0 ^\dag(t_0 , 0)  
U_0 (t_0 , 0)  
= \mathbb{1}
\,.
\end{align}
\end{subequations}

To show that it works well with successive times: 
%
\begin{subequations}
\begin{align}
U_I (t_2, t_1 ) U_I (t_1 , t_0 )
&= U_0 ^\dag (t_2 , 0) U (t_2, t_1 ) U_0 (t_1, 0)
U_0 ^\dag (t_1, 0) U(t_1, t_0 ) U_0 (t_0, 0)  \\
&= U_0 ^\dag (t_2 , 0) U (t_2, t_1 ) U(t_1, t_0 ) U_0 (t_0, 0)  \\
&= U_0 ^\dag (t_2 , 0) U (t_2, t_0 ) U_0 (t_0, 0)  \\
&= U_I (t_2, t_0 )
\,.
\end{align}
\end{subequations}

The last property can be shown as follows: we have
%
\begin{align}
U_I (t_0, t) = U_0 ^\dag (t_0 , 0) U(t_0, t) U_0 (t ,0 )
\,,
\end{align}
%
so 
%
\begin{align}
U_I ^\dag (t_0, t) 
= U_0 ^\dag (t, 0) U ^\dag (t_0 , t) U_0 (t_0, 0) 
= U_0 ^\dag (t, 0) U (t , t_0 ) U_0 (t_0, 0) 
= U_I (t, t_0 )
\,,
\end{align}
%
since \(U(t, t_0 ) = \exp(-i H (t - t_0) )\). 
\end{proof}

\subsubsection{Differential equation for the evolution operator}

We know that 
%
\begin{align}
i \dv{}{t} U (t, t_0 ) = H U (t, t_0 )
\,,
\end{align}
%
so we can take its conjugate: 
%
\begin{align}
- i \dv{}{t} U ^\dag (t, t_0 ) = U ^\dag (t, t_0 ) H 
\,.
\end{align}

This allows us to compute the differential equation for the evolution of the evolution operator: 
%
\begin{align}
i \dv{}{t} U_I (t, t_0 ) 
= H _{\text{int}}^{I} (t) U_I (t, t_0 )
\,,
\end{align}
%
where as usual \(H_{\text{int}}^{I} = H - H_0 \). 

Now, since the Hamiltonian is not constant this differential equation cannot be solved directly. However, we can cast it into an integral form: 
%
\begin{align} \label{eq:integral-expression-evolution-operator-interaction}
U_I (t, t_0 ) = \mathbb{1} - i \int_{t_0 }^{t} \dd{\tau }
H _{\text{int}}^{I} (\tau ) U(\tau , t_0 )
\,.
\end{align}

Crucially, in the interaction picture operators evolve with the free Hamiltonian (which we know well), while states evolve in a complicated way involving the interaction Hamiltonian. 

\subsection{Perturbative solution for the evolution operator}

We assume that the interaction term in the Hamiltonian is proportional to some small parameter \(\lambda \ll 1 \), and then we look for solutions of the form 
%
\begin{align}
U_I (t, t_0 ) = \sum _{n=0}^{\infty } C_n (t, t_0 )
\,,
\end{align}
%
such that the term \(C_n\) is of order \(\lambda^{n }\).

As long as \(\lambda \) is actually small, this series converges to the true result.

This perturbative approach can be made more explicit with the integral formulation \eqref{eq:integral-expression-evolution-operator-interaction}.
We know that \(H _{\text{int}} \sim \lambda \), so if we take the integral expression and plug the perturbative expression \(\sum _{n} C_n\) for the evolution operator into both sides, we get 
%
\begin{align}
\sum _{n} C_n(t, t_0 ) = \mathbb{1} - i \int_{t_0 }^{t} \dd{\tau }
H _{\text{int}}^{I} (\tau ) \qty(\sum _{n} C_n(\tau , t_0 ))
\,,
\end{align}
%
and now, if we set these two expressions to be equal at any order in \(\lambda \), we find that 
%
\begin{subequations}
\begin{align}
C_0 (t, t_0 ) &= \mathbb{1}   \\
C_1 (t, t_0 ) &=  - i \int_{t_0 }^{t} H _{\text{int}}^{I} (\tau) \label{eq:first-order-perturbative-evolution} \\
C_2 (t, t_0 ) &=  - i \int_{t_0 }^{t} H _{\text{int}}^{I} (\tau) C_1 (\tau , t_0 ) \label{eq:second-order-perturbative-evolution}
\,
\end{align}
\end{subequations}
%
and so on, in general if we sum these up to an order \(N\) to get the evolution operator to \(N\)-th order we find the recursive expression for \(U^{(N)} = \sum _{n=0}^{N} C_n\):
%
\begin{align}
U^{(N)} (t,  t_0  )
\approx \mathbb{1} - i \int_{t_0 }^{t} \dd{\tau } H _{\text{int}}^{I} U^{(N-1)} (\tau , t_0 ) + \mathcal{O} (\lambda^{N+1})
\,.
\end{align}

So, we can work to ever higher orders in \(\lambda \) to obtain better approximations as we go! 
The zeroth order is the \textbf{free theory}, since it is equivalent to setting \(\lambda = 0\). At this order, the interaction picture is equivalent to the Heisenberg picture: states are constant, while operators evolve according to 
%
\begin{align}
i \dv{}{t} X_I (t ) = \qty[X_I (t), H_0 ]
\,.
\end{align}

The first order is given by equation \eqref{eq:first-order-perturbative-evolution}, which we can substitute into the next equation \eqref{eq:second-order-perturbative-evolution} to get the explicit integral: 
%
\begin{subequations}
\begin{align}
C_2 (t, t_0 ) &=  
- i \int_{t_0 }^{t} \dd{\tau } H _{\text{int}}^{I} (\tau ) C_1 (\tau, t_0 )  \\ 
&= (-i)^2 \int_{t_0}^{t} \dd{\tau_1 } \int_{t_0 }^{\tau_1 } \dd{\tau_2 }
H _{\text{int}}^{I} (\tau_1 ) 
H _{\text{int}}^{I} (\tau_2 ) 
\label{eq:second-order-correction-no-time-ordering}
\,,
\end{align}
\end{subequations}
%
which generalizes to 
%
\begin{align}
C_N (t, t_0 ) &=
(-i)^{n}
\int_{t_0 }^{t} \dd{\tau_1 }
\prod_{n=1}^{N-1} \qty(\int_{t_0 }^{\tau_{n}} \dd{\tau_{n+1}})
\prod_{n=1}^{N} H _{\text{int}}^{I} (\tau_{n})
\,.
\end{align}

This is a complicated but explicit expression: as \(N\) increases, we are basically evolving linearly with the Hamiltonian for ever decreasing amounts of time, which approaches the actual evolution. 

Formally, we can recover the \textbf{exact solution} by summing the series \(\sum _{n} C_n\) (as long as it converges). 
This is often impossible to do in practice, but we can compute to any required precision, so we can compute to below the experimental precision and we can make the required prediction. 

\begin{claim}
With some manipulations, we can rewrite the general expression for \(C_n\) as: 
%
\begin{align}
C_n (t, t_0 ) = \frac{(-i)^{n}}{n!} 
\int_{t_0 }^{t} \dd{\tau_1} \dots \dd{\tau_n} T [H _{\text{int}}^{I} (\tau_1 ) \dots H _{\text{int}}^{I}(\tau_{n})] 
\,,
\end{align}
%
where \(T\) is the time ordered product, defined by 
%
\begin{subequations}
\begin{align}
T[A(t_1 ) B(t_2 )]
= \begin{cases}
    A(t_1 ) B(t_2 ) \qquad \text{if } t_1 > t_2 \\
    \pm B(t_2 ) A(t_1 ) \qquad \text{if } t_1 < t_2 \\
\end{cases}
\,,
\end{align}
\end{subequations}
%
where \(+\) refers to bosonic operators, \(-\) to fermionic operators.
\end{claim}

The \(\pm\) allows the time-ordered product to be consistent with both bosonic and fermionic quantizations. 
We can also write this as 
%
\begin{align}
T[A(t_1 ) B(t_2 )]
= \theta (t_1 - t_2 ) A(t_1 ) B(t_2 )
\pm \theta (t_2 - t_1 ) B(t_2 ) A(t_1 )
\,.
\end{align}

Note that when time-ordering Hamiltonians we always have a \(+\) sign, even if they describe fermions: fermionic operators always appear in pairs in the Hamiltonian, so we will always make an even number of swaps when we commute two Hamiltonians. 

The definition of time-ordering can be generalized to any number of operators recursively: we write them in decreasing order of time, and add a sign \((-)^{\text{\# of fermion operator swaps}}\). 

Then, we can write a full expression for the evolution operator: 
%
\boxalign{
\begin{subequations}
\begin{align} \label{eq:dyson-series}
U_I (t, t_0 ) &= \sum _{n=0}^{\infty }
\frac{(-i)^{n}}{n!}
\int_{t_0 }^{t} \dd{\tau_1 } \dots \dd{\tau_{n}} T [H _{\text{int}}^{I} (\tau_1 ) \dots H _{\text{int}}^{I}(\tau_{n})]  \\
&= T \qty[\exp(-i \int_{t_0 }^{t} \dd{ \tau } H _{\text{int}}^{I}(\tau ))]
\,.
\end{align}
\end{subequations}}

%

What we have done is basically to bring the expression for the time evolution into a form which is close to the one we have in the free case, in which we can just exponentiate the Hamiltonian: 
%
\begin{align}
U_0 (t, t_0 ) = \exp( - i H_0 (t - t_0 ))
\,.
\end{align}

\begin{claim}
The interaction Hamiltonian 
%
\begin{align}
H _{\text{int}}^{I} = - q \overline{\psi}_{I} \slashed{A}_{I} \psi_{I}
\,,
\end{align}
%
is a bosonic operator. 
\end{claim}

\begin{proof}
Suppose we had to compute the time-ordered product 
%
\begin{align}
T \qty[
(\overline{\psi}\slashed{A} \psi )_{t_1 }
(\overline{\psi}\slashed{A} \psi )_{t_2 }
]
\,,
\end{align}
%
where \(t_1 < t_2\): then we would need to bring the \(t_2 \) operators to the left; each of them would need to ``hop over'' two fermionic operators and a bosonic one. So, the total number of fermion-over-fermion hops will be 4: an even number.
\end{proof}

\todo[inline]{While swapping a fermionic operator and a bosonic one is always fine, right?}

We can show explicitly (and graphically) that 
%
\begin{align}
C_2 (t, t_0 ) = \frac{(-i)^2}{2!} 
\int_{t_0 }^{t} \dd{\tau_1} \dd{\tau_2 } T \qty[H _{\text{int}}^{I}(\tau_1 ) H _{\text{int}}^{I} (\tau_2 )]
\,.
\end{align}

\begin{proof}
Our integration region in \eqref{eq:second-order-correction-no-time-ordering} is a triangle, the set of \((\tau_1 , \tau_2 )\) such that \(\tau_1 \in [t_0 , t]\) and \(\tau_2 \in [t_0, \tau_1 ]\). 

We can also write this region as the set of \((\tau_1 , \tau_2 )\) such that \(\tau_2 \in [t_0 , t]\) and \(\tau_1 \in [\tau_2, t]\). They are the same pairs, so we can split the integral in two equal rewritings (and dividing by two, since we are doubling the integral): 
%
\begin{align}
C_2 (t, t_0 )
&= \frac{(-i)^2}{2} \qty(\int_{t_0}^{t} \dd{\tau_1 } \int_{t_0 }^{\tau_1 } \dd{\tau_2 }
+ \int_{t_0 }^{t} \dd{\tau_2 } \int_{\tau_2}^{t} \dd{\tau_1 }
)
H _{\text{int}}^{I} (\tau_1 ) 
H _{\text{int}}^{I} (\tau_2 ) 
\,,
\end{align}
%
but now we can change the names of the integration variables in the second integration: then we get 
%
\begin{align}
C_2 (t, t_0 )
&= \frac{(-i)^2}{2} \int_{t_0}^{t} \dd{\tau_1 } \int_{t_0 }^{\tau_1 } \dd{\tau_2 }
H _{\text{int}}^{I} (\tau_1 )
H _{\text{int}}^{I} (\tau_2 )
+ \frac{(-i)^2}{2} \int_{t_0 }^{t} \dd{\tau_1 } \int_{\tau_1}^{t} \dd{\tau_2 }
H _{\text{int}}^{I} (\tau_2 )
H _{\text{int}}^{I} (\tau_1 )
\,,
\end{align}
%
which we can write in a simpler way using theta functions (written as Iverson brackets \cite[]{knuthTwoNotesNotation1992}) and enlarging the domain of integration: 
%
\begin{subequations}
\begin{align}
C_2 (t, t_0 ) &=
\frac{(-i)^2}{2} \int_{t_0 }^{t} \dd{\tau_1 } \int_{t_0 }^{t} \dd{\tau_2 
}
\qty([t_1 > t_2  ] H _{\text{int}}^{I} (\tau_1 )
H _{\text{int}}^{I} (\tau_2 ) 
+ [t_1 < t_2 ] H _{\text{int}}^{I} (\tau_2 )
H _{\text{int}}^{I} (\tau_1 ) )  \\
&= \frac{(-i)^2}{2} \int_{t_0 }^{t} \dd{\tau_1 } \int_{t_0 }^{t} \dd{\tau_2 }
T \qty[H _{\text{int}}^{I} (\tau_1 )
H _{\text{int}}^{I} (\tau_2 )]
\,.
\end{align}
\end{subequations}
\end{proof}

\end{document}