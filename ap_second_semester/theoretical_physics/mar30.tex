\documentclass[main.tex]{subfiles}
\begin{document}

\marginpar{Monday\\ 2020-4-6, \\ compiled \\ \today}

\subsection{General solution of the Dirac equation}

We have imposed that solutions of the Dirac equation are solutions of the KG equation and vice-versa, so a general solution of the Dirac equation should look like those we found when discussing the Klein-Gordon equation: 
%
\begin{align}
\psi (x) = \psi_{+ } (x) + \psi_{-} (x) 
\sim \eval{e^{-ikx} u(k) + v(k) e^{ikx}}_{k_0 = \omega_{k}}
\,,
\end{align}
%
where the subscript \(+\) or \(-\) refers to the sign of the energy of the solution, and now \(u(k)\) and \(v(k)\) are spinors in momentum space. 
We use the \(\sim\) sign since these are only plane wave solutions, a general solution will be an integral of these over momentum space.

If we apply the Dirac operator \(i \slashed{\partial} - M\) to these we find 
%
\begin{subequations}
\begin{align}
\qty( i \slashed{\partial} - M) \psi_{+} \sim e^{-ikx} \qty(\slashed{k} - M) u(k) &= 0 \\
\qty( i \slashed{\partial} - M) \psi_{-} \sim -e^{ikx} \qty(\slashed{k} + M) v(k) &= 0
\,,
\end{align}
\end{subequations}
%
so, since the exponentials are nonzero we can write these two equations as 
%
\boxalign{
\begin{align}
\qty(\slashed{k}- M) u(k) = 0 
\qquad \text{and} \qquad
\qty(\slashed{k} + M) v(k) = 0
\,.
\end{align}}

Let us now assume that the particle we are considering is not massless, so we can go in its rest frame. 
If we were to consider a massless particle, we could work with the \(N=2\) Weyl spinors. Instead, we will need the \(N=4\) Dirac spinors. 

In the rest frame \(k^{\mu } = (M, \vec{0})\), so the two equations read 
%
\begin{subequations}
\begin{align}
0&= \qty(\slashed{k} - M) u(k)  \\
&= \qty(\gamma^{\mu }k_{\mu } - M) u(k)  \\
&= \qty(\gamma^{0} M - M ) u(k) = M \qty(\gamma^{0} - \mathbb{1}) u(k)  \\
&=M\qty( \left[\begin{array}{cc}
\mathbb{1} & 0 \\ 
0 & -\mathbb{1}
\end{array}\right] 
-
\left[\begin{array}{cc}
\mathbb{1} & 0 \\ 
0 & \mathbb{1}
\end{array}\right]
) u(k)  \\
&= \left[\begin{array}{cc}
0 & 0 \\ 
0 & -2 \mathbb{1}
\end{array}\right] M u(k) =0
\,,
\end{align}
\end{subequations}
%
so a generic solution looks like 
%
\begin{subequations}
\begin{align}
u(M) = c \left[\begin{array}{c}
\xi  \\ 
0
\end{array}\right]
\,.
\end{align}
\end{subequations}

Note that we are using the Dirac representation, but if we were to choose a different one the spinors would look different. 
For \(v(k)\) the equation looks like 
%
\begin{subequations}
\begin{align}
M \qty(\gamma^{0} + \mathbb{1}) v(M) =
\left[\begin{array}{cc}
2 \mathbb{1} & 0 \\ 
0 & 0
\end{array}\right] M v(M) = 0
\,,
\end{align}
\end{subequations}
%
so 
%
\begin{subequations}
\begin{align}
v(M) = c \left[\begin{array}{c}
0 \\ 
\xi 
\end{array}\right]
\,.
\end{align}
\end{subequations}

Here \(\xi \) is a two-dimensional vector, while \(c\) is a normalization constant. 
We have two independent solutions for each case, so in total there are four independent ones. 
A basis we can choose is 
%
\begin{subequations}
\begin{align}
u_{r} (M) = \sqrt{2M} \left[\begin{array}{c}
\xi_{r} \\ 
0
\end{array}\right]
\qquad \text{and} \qquad
v_{r} (M) = \sqrt{2M} \left[\begin{array}{c}
0 \\
\xi_{r} 
\end{array}\right]
\,,
\end{align}
\end{subequations}
%
where \(r=1,2\) and \((\xi_{r})^{i} = \delta^{i}_{r}\) are unit vectors, a basis for the 2D space. 

If we choose this normalization, then we will have \(\overline{u}_{r}(M) u_s(M) = 2 M \delta_{rs}\) and \(\overline{v}_{r}(M) v_s(M) =- 2 M \delta_{rs}\), while all the \(u\)s and the \(v\)s are respectively orthogonal: \(\overline{u}_{r} v_s =  \overline{v}_{r} u_s = 0\). 

This choice corresponds to having chosen to have the spin of the particle along the third axis, as we shall see shortly. 

The fact that we were able to find a basis of four independent vectors for the solution means that the solution of the Dirac equation has four independent degrees of freedom, two of which have positive energy and two of which have negative energy. 

We put ourselves in the rest frame: if we wish to compute the solutions \(u(k)\) and \(v(k)\) in a generic frame we need to perform a Lorentz boost from the rest frame. 
In order to study how this boost affects the spinor, we need to study the spinorial representation of the Lorentz boost, \(S(\Lambda)\). 

Instead of explicitly writing out the full representation of the Lorentz matrix we are interested in, which is long and complicated, we can use the following trick: 
%
\begin{subequations}
\begin{align} \label{eq:on-shell-equality}
\qty(\slashed{k} - M) \qty(\slashed{k} + M)
&=
\qty(\slashed{k} + M) \qty(\slashed{k} - M) \\
&= \gamma^{\mu } \gamma^{\nu } k_{\mu }k_{\nu } - M^2   \\
&= \frac{1}{2} \qty{\gamma^{\mu }, \gamma^{\nu }} k_{\mu } k_{\nu } - M^2 = k^2-M^2 = 0
\,,
\end{align}
\end{subequations}
%
since \( \frac{1}{2} \qty{\gamma^{\mu }, \gamma^{\nu }} = \eta^{\mu \nu }\), and the square of the 4-momentum always corresponds to the mass of the particle under our assumptions (of working on-shell).

This allows us to quickly prove that the ansatz 
%
\begin{align}
u(k) = C (\slashed{k} + M) u(M)
\,
\end{align}
%
satisfies the KG equation (with positive energy), since 
%
\begin{align}
(\slashed{k} - M) u(k) = C \qty(k^2 - M^2) u(M) = 0
\,,
\end{align}
%
which follows from the identity \eqref{eq:on-shell-equality} which we just proved.
So, we can get a solution in a generic frame by applying a known operator onto the res

% \todo[inline]{But then we could use something which is not \(u(M)\) for our rest-frame solution, right?}

Similarly, for the negative-energy case we have the solution 
%
\begin{align}
v(k) = C (\slashed{k} - M )v(M)
\,,
\end{align}
%
which will satisfy \((\slashed{k} + M) v(k) = 0\).

The constant \(C\) is for normalization, and we choose it such that the normalization is the same as in the rest frame: so, the identities to be satisfied are 
%
\begin{subequations}
\begin{align} \label{eq:normalization-spinor}
\overline{u}_{r} (k) u_{s}(k) &= 2 M \delta_{rs}  \\
\overline{v}_{r} (k) v_{s}(k) &= - 2 M \delta_{rs}  \\
\overline{u}_{r} (k) v_{s}(k) &=   
\overline{v}_{r} (k) u_{s}(k) = 0  
\,.
\end{align}
\end{subequations}

\begin{claim}
The final result we get from this manipulation is 
%
\begin{subequations} \label{eq:dirac-equation-solutions}
\begin{align}
u_r (k) &= \frac{(\slashed{k} + M)}{\sqrt{2 M \qty(\omega_{k} + M )}} u_r (M)
= \left[\begin{array}{c}
\xi_{r}\sqrt{\omega_{k} + M} \\ 
\displaystyle
\frac{\vec{k} \cdot \vec{\sigma}}{\sqrt{\omega_{k} + M}} 
\xi_{r}
\end{array}\right] \label{eq:positive-energy-dirac-solution-momentum-space}\\
v_r (k) &= \frac{(-\slashed{k} + M)}{\sqrt{2 M \qty(\omega_{k} + M )}} v_r (M)
= \left[\begin{array}{c}
\displaystyle
\frac{\vec{k} \cdot \vec{\sigma}}{\sqrt{\omega_{k} + M}} \xi_{r}\\
\xi_{r}\sqrt{\omega_{k} + M} 
\end{array}\right]
\,.
\end{align}
\end{subequations}
\end{claim}

The first solution is the positive-energy one, the second is the negative-energy one. 
Notice that the expression \(\vec{k} \cdot \vec{\sigma}\) yields a \(2 \times 2\) complex matrix, which is applied to the vector \(\xi_{r}\).

We prove the second equality for the positive energy solution. 

\begin{proof}
We begin by writing out the operator 
%
\begin{subequations}
\begin{align}
\slashed{k}  = \gamma^{\mu } \eta_{\mu \nu } k^{\nu }
= \left[\begin{array}{cc}
\mathbb{1} \omega_{k} & 0 \\ 
0 & -\mathbb{1} \omega_{k}
\end{array}\right]
- 
\left[\begin{array}{cc}
0 & \vec{k} \cdot \vec{\sigma} \\ 
-\vec{k} \cdot \vec{\sigma} & 0
\end{array}\right]
= \left[\begin{array}{cc}
\mathbb{1} \omega_{k} & -\vec{k} \cdot \vec{\sigma} \\ 
\vec{k} \cdot \vec{\sigma} & - \mathbb{1} \omega_{k}
\end{array}\right]
\,.
\end{align}
\end{subequations}

So, when we apply this (plus \(M\) times the identity) to the solution 
%
\begin{subequations}
\begin{align}
u_r(M) = \sqrt{2M} \left[\begin{array}{c}
\xi_{r} \\ 
0
\end{array}\right]
\,,
\end{align}
\end{subequations}
%
we get: 
%
\begin{subequations}
\begin{align}
\qty(\slashed{k} + M) u_{r}(M) = 
\sqrt{2M} \left[\begin{array}{c}
\qty(\omega_{k} + M)  \xi_{r} \\ 
\qty(\vec{k} \cdot \vec{\sigma}) \xi_{r}
\end{array}\right]
\,,
\end{align}
\end{subequations}
%
which we can divide by \(\sqrt{2M (\omega_{k} + M)}\) to find the desired expression, equation \eqref{eq:positive-energy-dirac-solution-momentum-space}.
\end{proof}

\begin{claim}
The conjugate spinors \(\overline{u}\) and \(\overline{v}\) in momentum space read respectively: 
%
\begin{subequations}
\begin{align}
\overline{u}_{r} (k) = \overline{u}_{r}(M) \frac{\qty(\slashed{k} + M)}{\sqrt{2 (\omega_{k} + M)}}
= \left[\begin{array}{cc}
\xi_{r}^{\top} \sqrt{\omega_{k} + M}, & 
\displaystyle
- \xi_{r}^{\top} \frac{\vec{k} \cdot \vec{\sigma}}{\sqrt{\omega_{k} + M}}
\end{array}\right] \\
\overline{v}_{r} (k) = \overline{v}_{r}(M) \frac{\qty(-\slashed{k} + M)}{\sqrt{2 (\omega_{k} + M)}}
= \left[\begin{array}{cc}
\displaystyle
\xi_{r}^{\top} \frac{\vec{k} \cdot \vec{\sigma}}{\sqrt{\omega_{k} + M}}, &
-\xi_{r}^{\top} \sqrt{\omega_{k} + M}
\end{array}\right]
\,.
\end{align}
\end{subequations}
\end{claim}

\begin{proof}
Recall that \(\overline{u} = u ^\dag \gamma^{0}\), and that in our representation 
%
\begin{subequations}
\begin{align}
\gamma^{0} = \left[\begin{array}{cc}
\mathbb{1} & 0 \\ 
0 & -\mathbb{1}
\end{array}\right]
\,.
\end{align}
\end{subequations}

We can then compute 
%
\begin{subequations}
\begin{align}
\overline{u}_{r} (k) &= \left[\begin{array}{c}
\xi_{r}\sqrt{\omega_{k} + M} \\ 
\displaystyle
\frac{\vec{k} \cdot \vec{\sigma}}{\sqrt{\omega_{k} + M}} 
\xi_{r}
\end{array}\right] ^\dag
\gamma_0  \\
&= \left[\begin{array}{cc}
\xi_{r}^{\top} \sqrt{\omega_{k} + M}, & 
\xi_{r}^{\top} \displaystyle
\frac{(\vec{k} \cdot \vec{\sigma})}{\sqrt{\omega_{k} + M}}
\end{array}\right] \gamma_0   \marginnote{\(\vec{k}\) and \(\xi_{r}\) are real, \(\sigma = \sigma ^\dag\).}\\
&=  \left[\begin{array}{cc}
\xi_{r}^{\top} \sqrt{\omega_{k} + M}, & 
-\xi_{r}^{\top} \displaystyle
\frac{(\vec{k} \cdot \vec{\sigma})}{\sqrt{\omega_{k} + M}}
\end{array}\right]
\,.
\end{align}
\end{subequations}

The computation for the negative energy solution is analogous.
\end{proof}

\begin{claim}
We can derive the normalization conditions \eqref{eq:normalization-spinor} from the explicit expressions of the solutions.
\end{claim}

\begin{proof}
Writing out the multiplication explicitly for the real solutions we have: 
%
\begin{subequations}
\begin{align}
\overline{u}_{r}(k) u_s(k) &=
\left[\begin{array}{cc}
\xi_{r}^{\top} \sqrt{\omega_{k} + M}, & 
\displaystyle
- \xi_{r}^{\top} \frac{\vec{k} \cdot \vec{\sigma}}{\sqrt{\omega_{k} + M}}
\end{array}\right]
\left[\begin{array}{c}
\xi_{s}\sqrt{\omega_{k} + M} \\ 
\displaystyle
\frac{\vec{k} \cdot \vec{\sigma}}{\sqrt{\omega_{k} + M}} 
\xi_{s}
\end{array}\right]  \\
&= \xi_{r}^{\top} \sqrt{\omega_{k} +M }
\xi_{s} \sqrt{\omega_{k} + M}
- \xi_{r}^{\top} 
\frac{(\vec{k} \cdot \vec{\sigma})}{\sqrt{\omega_{k} + M}}
\frac{(\vec{k} \cdot \vec{\sigma})}{\sqrt{\omega_{k} + M}}
\xi_{s}  \label{eq:normalization-condition-step-positive-energy}\\
&= \delta_{rs} (\omega_{k} + M) - \delta_{rs} \frac{\abs{k}^2}{\omega_{k} + M}  \\
&= \delta_{rs} \frac{(\omega_{k} + M)^2 - \abs{k}^2}{\omega_{k} + M}  \\
&= \delta_{rs} \frac{M^2 + 2 \omega_{k} M + \omega_{k}^2 - \abs{k}^2}{\omega_{k} + M} = 2M \delta_{rs}  \frac{\omega_{k} + M}{\omega_{k} + M}
\marginnote{Used \(\omega_{k}^2 - \abs{k}^2 = M^2\)}
\,,
\end{align}
\end{subequations}
%
where we applied the identity 
%
\begin{align}
(\vec{a} \cdot \vec{\sigma}) (\vec{b} \cdot \vec{\sigma}) = \qty(\vec{a} \cdot \vec{b}) \mathbb{1} + i \qty(\vec{a} \times \vec{b}) \cdot \vec{\sigma} 
\,,
\end{align}
%
which follows from the commutation and anticommutation relations 
%
\begin{subequations}
\begin{align}
\qty[\sigma_{a}, \sigma_{b}] &= 2i \epsilon_{abc} \sigma_{c}  \\
\qty{\sigma_{a}, \sigma_{b}} &= 2\delta_{ab} \mathbb{1}
\,.
\end{align}
\end{subequations}

Now we can replicate the calculation for the negative energy solution: 
%
\begin{subequations}
\begin{align}
\overline{v}_{r} v_s &=
\left[\begin{array}{cc}
\displaystyle
\xi_{r}^{\top} \frac{\vec{k} \cdot \vec{\sigma}}{\sqrt{\omega_{k} + M}}, &
-\xi_{r}^{\top} \sqrt{\omega_{k} + M}
\end{array}\right]
\left[\begin{array}{c}
\displaystyle
\frac{\vec{k} \cdot \vec{\sigma}}{\sqrt{\omega_{k} + M}} \xi_{s}\\
\xi_{s}\sqrt{\omega_{k} + M} 
\end{array}\right]  \\
&= \delta_{rs} \qty( \frac{\abs{k}^2}{\omega_{k} + M}  - \omega_{k} - M) \\
&= -2M \delta_{rs}
\,,
\end{align}
\end{subequations}
%
where we skipped some steps since we can recognize the opposite of the expression we found earlier, equation \eqref{eq:normalization-condition-step-positive-energy}. 

For the mixed terms, instead, we get 
%
\begin{subequations}
\begin{align}
\overline{v}_{r} u_{s} &= 
\left[\begin{array}{cc}
\displaystyle
\xi_{r}^{\top} \frac{\vec{k} \cdot \vec{\sigma}}{\sqrt{\omega_{k} + M}}, &
-\xi_{r}^{\top} \sqrt{\omega_{k} + M}
\end{array}\right]
\left[\begin{array}{c}
\xi_{s}\sqrt{\omega_{k} + M} \\ 
\displaystyle
\frac{\vec{k} \cdot \vec{\sigma}}{\sqrt{\omega_{k} + M}} 
\xi_{s}
\end{array}\right]  \\
&= \xi_{r}^{\top} \frac{\vec{k} \cdot \vec{\sigma}}{\sqrt{\omega_{k} + M}}
\xi_{s} \sqrt{\omega_{k} +M} - 
\xi_{r}^{\top} \frac{\vec{k} \cdot \vec{\sigma}}{\sqrt{\omega_{k} + M}}
\xi_{s} \sqrt{\omega_{k} +M} = 0 = \overline{u}_{r} v_s
\label{eq:normalization-orthogonality-positive-negative-solutions}
\,.
\end{align}
\end{subequations}
\end{proof}

\begin{claim}
The following identities hold:
%
\begin{subequations}
\begin{align} 
u_{r}^\dag (k) u_{s}(k) &= 2 \omega_{k}\delta_{rs}  \\
v_{r}^\dag (k) v_{s}(k) &= 2 \omega_{k} \delta_{rs}  \\
u_{r}^\dag (k) v_{s}(-k) &=   
v_{r}^\dag (k) u_{s}(-k) = 0  
\,.
\end{align}
\end{subequations}

Notice that now we have a dagger instead of a bar.
\end{claim}

\begin{proof}
In order to see what these solutions are we need to compute \(u_{r} ^\dag\): the difference between it and \(\overline{u}_{r}\) is the lack of multiplication by \(\gamma^{0}\), which in our representation means that the sign of the second component is not flipped. 
So, in the calculation at step \eqref{eq:normalization-condition-step-positive-energy} we have instead 
%
\begin{subequations}
\begin{align}
\xi_{r}^{\top} \sqrt{\omega_{k} +M }
\xi_{s} \sqrt{\omega_{k} + M}
- \xi_{r}^{\top} 
\frac{(\vec{k} \cdot \vec{\sigma})}{\sqrt{\omega_{k} + M}}
\frac{(\vec{k} \cdot \vec{\sigma})}{\sqrt{\omega_{k} + M}}
\xi_{s}
&= \delta_{rs} \frac{\omega_{k}^2 + M^2 + 2 \omega_{k} M + \abs{k}^2}{\omega_{k} + M}  \\
&=2 \delta_{rs} \omega_{k}
\,,
\end{align}
\end{subequations}
%
where we applied a similar line of reasoning to the other proof. The negative sign makes it so instead of cancelling the \(\omega_{k}^2\) term we cancel the \(M^2\) term.

For the negative energy solution we have basically the same thing.
Let us consider the product of the negative and positive solutions: if we swap the sign we find that the result is nonzero since the terms in \eqref{eq:normalization-orthogonality-positive-negative-solutions} do not cancel anymore. 

However, if we flip the sign of one of the two momenta the terms cancel.
\end{proof}

\end{document}
