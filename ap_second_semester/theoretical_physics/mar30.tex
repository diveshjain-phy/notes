\documentclass[main.tex]{subfiles}
\begin{document}

\marginpar{Monday\\ 2020-4-6, \\ compiled \\ \today}

\subsection{General solution of the Dirac equation}

We have imposed that solutions of the Dirac equation are solutions of the KG equation and vice-versa, so a general solution of the Dirac equation should look like those we found when discussing the Klein-Gordon equation: 
%
\begin{align}
\psi (x) = \psi_{+ } (x) + \psi_{-} (x) 
\sim \eval{e^{-ikx} u(k) + v(k) e^{ikx}}_{k_0 = \omega_{k}}
\,,
\end{align}
%
where the subscript \(+\) or \(-\) refers to the sign of the energy of the solution, and now \(u(k)\) and \(v(k)\) are spinors in momentum space. 
We use the \(\sim\) sign since these are only plane wave solutions, a general solution will be an integral of these over momentum space.

If we apply the Dirac operator \(i \slashed{\partial} - M\) to these we find 
%
\begin{align}
\qty( i \slashed{\partial} - M) \psi_{+} \sim e^{-ikx} \qty(\slashed{k} - M) u(k) &= 0 \\
\qty( i \slashed{\partial} - M) \psi_{-} \sim -e^{ikx} \qty(\slashed{k} + M) v(k) &= 0
\,,
\end{align}
%
so, since the exponentials are nonzero we can write these two equations as 
%
\boxalign{
\begin{align}
\qty(\slashed{k}- M) u(k) = 0 
\qquad \text{and} \qquad
\qty(\slashed{k} + M) v(k) = 0
\,.
\end{align}}

Let us now assume that the particle we are considering is not massless, so we can go in its rest frame. 
If we were to consider a massless particle, we could work with the \(N=2\) Weyl spinors. Instead, we will need the \(N=4\) ones. 

In the rest frame \(k^{\mu } = (M, \vec{0})\), so the two equations read 
%
\begin{align}
0&= \qty(\slashed{k} - M) u(k)  \\
&= \qty(\gamma^{\mu }k_{\mu } - M) u(k)  \\
&= \qty(\gamma^{0} M - M ) u(k) = M \qty(\gamma^{0} - \mathbb{1}) u(k)  \\
&=\qty( \left[\begin{array}{cc}
\mathbb{1} & 0 \\ 
0 & -\mathbb{1}
\end{array}\right] 
-
\left[\begin{array}{cc}
\mathbb{1} & 0 \\ 
0 & \mathbb{1}
\end{array}\right]
) u(k)  \\
&= \left[\begin{array}{cc}
0 & 0 \\ 
0 & -2 \mathbb{1}
\end{array}\right] M u(k) =0
\,,
\end{align}
%
so a generic solution looks like 
%
\begin{align}
u(M) = c \left[\begin{array}{c}
\xi  \\ 
0
\end{array}\right]
\,.
\end{align}

Note that we are using the Dirac representation, but if we were to choose a different one the spinors would look different. 
For \(v(k)\) the equation looks like 
%
\begin{align}
M \qty(\gamma^{0} + \mathbb{1}) v(M) =
\left[\begin{array}{cc}
2 \mathbb{1} & 0 \\ 
0 & 0
\end{array}\right] M v(M) = 0
\,,
\end{align}
%
\todo[inline]{Reported with a wrong sign in the notes!}
so 
%
\begin{align}
v(M) = c \left[\begin{array}{c}
0 \\ 
\xi 
\end{array}\right]
\,.
\end{align}

Here \(\xi \) is a two-dimensional vector, while \(c\) is a normalization constant. 
We have two independent solutions for each case, so in total there are four independent ones. 
A basis we can choose is 
%
\begin{align}
u_{r} (M) = \sqrt{2M} \left[\begin{array}{c}
\xi_{r} \\ 
0
\end{array}\right]
\qquad \text{and} \qquad
v_{r} (M) = \sqrt{2M} \left[\begin{array}{c}
0 \\
\xi_{r} 
\end{array}\right]
\,,
\end{align}
%
where \(r=1,2\) and \((\xi_{r})^{i} = \delta^{i}_{r}\) are unit vectors, a basis for the 2d space. 

\end{document}