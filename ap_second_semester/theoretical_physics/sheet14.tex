\documentclass[main.tex]{subfiles}
\begin{document}

\chapter{Quantum ElectroDynamics}

\section{QED \(S\)-matrix expansion}

\marginpar{Monday\\ 2020-6-15, \\ compiled \\ \today}

As an explicit example of the theory we developed for an interacting QFT, we consider Quantum ElectroDynamics.

The Lagrangian is written as 
%
\begin{align}
\mathscr{L} = 
- \frac{1}{4} F^{\mu \nu } F_{\mu \nu }
+ \mathscr{L} _{\text{gauge-fixing}}
+ i q \overline{\psi} \qty(i\slashed{\DD} - M) \psi  
\,,
\end{align}
%
where \(q\) is the charge of the fermion --- for instance, electrons have \(q = - \abs{e}\). 
This can be decomposed as \(\mathscr{L} = \mathscr{L}_{0} + \mathscr{L} _{\text{int}}\), with 
%
\begin{align}
\mathscr{L} _{\text{int}} = - q \overline{\psi} \gamma^{\mu } \psi A_{\mu } = - \mathscr{H} _{\text{int}}
\,.
\end{align}

Notice that the interaction Lagrangian and Hamiltonian are opposite of each other: this holds in general, as long as there are no derivative terms --- basically, we have an interaction \emph{potential}. 

We are working in the \textbf{interaction picture}. 

The \(S\)-matrix operator \eqref{eq:s-matrix} for QED reads: 
%
\begin{align}
S _{\text{QED}}
&= T \qty[\exp(-iq \int \dd[4]{x} N [\overline{\psi}(x) \slashed{A} (x) \psi (x)])]  \\
&= \sum _{n=0}^{ \infty }
\frac{(-iq)^{n}}{n!}
\int \dd[4]{x_1 } \dots \dd[4]{x_{n}}
T \qty[N[\dots]_{x_1} \dots N[\dots]_{x_n}]
\,.
\end{align}

\subsection{Expansion in position space}

\subsubsection{0th order term}

It is just the identity, since it corresponds to no interaction:
%
\begin{align}
S^{(0)} = \mathbb{1}
\,.
\end{align}

\subsubsection{1st order term}

It is given by 
%
\begin{align}
S^{(1)} &= - iq \int \dd[4]{x} T \qty[N[\overline{\psi}(x) \slashed{A} (x) \psi (x)]]  \\
&= - iq \int \dd[4]{x} N[\overline{\psi}(x) \slashed{A} (x) \psi (x)]
\,.
\end{align}

In order to remove the time-ordering we have used the corollary of Wick's theorem: we know that 
%
\begin{align}
T[N[A(x) B(x)]] = T[A(x) B(x)]_{\text{NCET}}
\,,
\end{align}
%
so we can apply Wick's theorem to the last expression: we only have the first term, since all the possible contractions would be equal-time ones, therefore we get 
%
\begin{align}
T[A(x) B(x)]_{\text{NCET}} = N[A(x)B(x)]
\,.
\end{align}

For three operators the reasoning is exactly the same.

We can split \(\overline{\psi} \), \(\slashed{A}\) and \(\psi \) into their \(+\) and \(-\) components: we get eight contributions, 
%
\begin{align}
N \qty[
    \qty(\overline{\psi}_{+} + \overline{\psi}_{-})
    \qty(\slashed{A}_{+} + \slashed{A}_{-})
    \qty(\psi_{+} + \psi_{-})
]
= N \qty[
    \sum \overline{\psi}_{\pm} \slashed{A}_{\pm} \psi_{\pm}
]
\,,
\end{align}
%
which we can represent using Feynman diagrams.
All of these diagram only have one vertex at \(x\), and they have a fermion, an antifermion and a photon being created/annihilated there (all the 8 possible combinations). As an example, we show: 
%
\begin{align} \label{eq:fermion-fermion-photon-vertex-1}
\overline{\psi}_{+} \slashed{A}_{-} \psi_{+}
= \feynmandiagram[inline=(c.base), horizontal = c to p]{
    c [particle=\(x\)] -- [photon] p [particle=\(\gamma \)],
    a [particle=\(e^{-}\)] -- [fermion] c -- [fermion] b [particle=\(e^{+}\)]
};
\end{align}
%
here we have an electron and a positron annihilating to form a photon.

\todo[inline]{I think this vertex has a wrong sign in the professor's notes.}

In general, particles which are created (so, with subscript \(-\)) go rightward, particles which are annihilated (with subscript \(+\)) come from the left.

These diagrams are \emph{vertices}: they represent a certain interaction in QED, between 2 fermions and a photon. 

Notice that the flow of the fermion arrows is continuous in the diagram we wrote: this generalizes to all of them, and it is a manifestation of the \(U(1)\) charge conservation. 

Now, all these 8 diagrams represent \textbf{unphysical processes}: due to the fact that the photon is massless, relativistic kinematics forbids them. 

Let us show it for the process \eqref{eq:fermion-fermion-photon-vertex-1}, which can be written in more usual notation as 
%
\begin{align}
e^{+}(p) + e^{-} (p') \to \gamma (k)
\,.
\end{align}

Relativistic kinematics tells us that \(p + p' = k\). 
The fermions are massive, so we can go to their center-of-mass frame before the annihilation: in this frame, we will have \((p + p')_{i} = 0 \), which implies \(k_{i} = 0 \), which is not possible since \(k^{0} = \abs{\vec{k}}\) (and the photon must have positive energy). 

Consider another process: 
%
\begin{align}
e^{-}(p) \to e^{-} (p') + \gamma (k)
\,,
\end{align}
%
which is represented diagrammatically as 
\begin{align} \label{eq:fermion-fermion-photon-vertex-2}
\overline{\psi}_{-} \slashed{A}_{-} \psi_{+}
= \feynmandiagram[inline=(c.base), horizontal = a to c]{   
    c [particle=\(x\)] -- [photon] p [particle=\(\gamma \)],
    a [particle=\(e^{-}\)] -- [fermion] c,
    c -- [fermion] b [particle=\(e^{-}\)]
};
\end{align}

This is also forbidden (all of them are). To show this, let us go to the rest frame of the incoming electron: starting from the relation \(p = p' + k\) we get 
%
\begin{align}
m_e &= p^{\prime 0} + k^{0}  \\
0 &= \vec{p}' + \vec{k}
\,,
\end{align}
%
so, since for the photon \(k^{0} = \abs{k}\), we can write the first relation as 
%
\begin{align}
m_e &= p^{\prime 0} - \abs{\vec{p}'}  \\
m_e^2 &= \qty(p^{\prime 0} - \abs{\vec{p}'})^2  \\
\qty(p^{\prime 0})^2 - \abs{\vec{p}'}^2 &= \qty(p^{\prime 0})^2
- 2 p^{\prime 0} \abs{\vec{p}'} + \abs{\vec{p}'}^2  \\
0 &= - 2 p^{\prime 0} \abs{\vec{p}'} + 2 \abs{\vec{p}'}^2
\,,
\end{align}
%
which means that either \(\vec{p}' = 0\) (unacceptable, it would mean the photon has zero energy) or \(p^{\prime 0} = \abs{\vec{p}'}\), which tells us that the electron is massless, which it is not. 

Some of these processes are \textbf{allowed} if the boson is \textbf{massive}: for example, we can have 
%
\begin{align}
Z(k) \to e^{+}(p) + e^{-}(p')
\,,
\end{align}
%
where the mass of the \(Z\) boson satisfies \(M^2_Z > (2 m_e)^2\). 
This works since we can go to the rest frame of the \(Z\) boson, and if the inequality is satisfied the electron and positron momenta can be real. 

\subsubsection{2nd order term}

Since the first order processes are unphysical, we have to go to the second order. The second term in the \(S\)-matrix expansion reads: 
%
\begin{align}
S^{(2)} = \frac{(-iq)^2 }{2!}
\int \dd[4]{x} \dd[4]{y}
T \qty[
    N[\overline{\psi} \slashed{A} \psi ]_{x}
    N[\overline{\psi} \slashed{A} \psi ]_{y}
]
\,.
\end{align}

By the corollary of Wick's theorem, we can write 
%
\begin{align}
T \qty[
    N[\overline{\psi} \slashed{A} \psi ]_{x}
    N[\overline{\psi} \slashed{A} \psi ]_{y}
]
= 
T \qty[
    \qty(\overline{\psi} \slashed{A} \psi) _{x}
    \qty(\overline{\psi} \slashed{A} \psi )_{y}
] _{\text{NCET}}
\,,
\end{align}
%
so when we compute the contractions we can only connect fields computed at \(x\) with ones computed at \(y\). Also, contractions between different types of fields vanish, as do contractions between a complex-valued field and itself. So, we have: 
\begin{enumerate}
    \item the 0 contraction term 
    %
    \begin{align}
        \underbrace{N \qty[
            \qty(\overline{\psi} \slashed{A} \psi) _{x}
            \qty(\overline{\psi} \slashed{A} \psi )_{y}
        ]}_{\Circled{A}} 
    \,,
    \end{align}
    %
    \item the 1 contraction term 
    %
    \begin{align}
        \underbrace{N \left[
            \contraction{}{(\overline{\psi}}{ \slashed{A} \psi ) _{x} (\overline{\psi} \slashed{A} }{\psi} % psibar-psi
            (\overline{\psi} \slashed{A} \psi ) _{x} (\overline{\psi} \slashed{A} \psi ) _{y}
        \right] +
        N \left[
            \contraction{(\overline{\psi} \slashed{A} }{\psi}{ ) _{x} (}{\overline{\psi}} % psi-psibar
            (\overline{\psi} \slashed{A} \psi ) _{x} (\overline{\psi} \slashed{A} \psi ) _{y}
        \right]}_{\Circled{B_1}} +
        \underbrace{N \left[
            \contraction{(\overline{\psi} }{\slashed{A}}{ \psi ) _{x} (\overline{\psi} }{\slashed{A}} % ph-ph
            (\overline{\psi} \slashed{A} \psi ) _{x} (\overline{\psi} \slashed{A} \psi ) _{y}
        \right]}_{\Circled{B_2}}
    \,,
    \end{align}
    %   
    \item the 2 contraction term 
    %
    \begin{align}
        \underbrace{N \left[
            \bcontraction{(\overline{\psi} }{\slashed{A}}{ \psi ) _{x} (\overline{\psi} }{\slashed{A}} % ph-ph
            \contraction{}{(\overline{\psi}}{ \slashed{A} \psi ) _{x} (\overline{\psi} \slashed{A} }{\psi} % psibar-psi
            (\overline{\psi} \slashed{A} \psi ) _{x} (\overline{\psi} \slashed{A} \psi ) _{y}
        \right]+ 
        N \left[
            \bcontraction{(\overline{\psi} }{\slashed{A}}{ \psi ) _{x} (\overline{\psi} }{\slashed{A}} % ph-ph
            \contraction{(\overline{\psi} \slashed{A} }{\psi}{ ) _{x} (}{\overline{\psi}} % psi-psibar
            (\overline{\psi} \slashed{A} \psi ) _{x} (\overline{\psi} \slashed{A} \psi ) _{y}
        \right]}_{\Circled{C_1}}+ 
        \underbrace{N \left[
            \bcontraction{}{(\overline{\psi}}{ \slashed{A} \psi ) _{x} (\overline{\psi} \slashed{A} }{\psi} % psibar-psi
            \contraction{(\overline{\psi} \slashed{A} }{\psi}{ ) _{x} (}{\overline{\psi}} % psi-psibar
            (\overline{\psi} \slashed{A} \psi ) _{x} (\overline{\psi} \slashed{A} \psi ) _{y}
        \right]}_{\Circled{C_2}} 
    \,,
    \end{align}
    %
    \item and the 3 contraction term 
    %
    \begin{align}
        \underbrace{N \left[
            \contraction{(\overline{\psi} }{\slashed{A}}{ \psi ) _{x} (\overline{\psi} }{\slashed{A}} % ph-ph
            \bcontraction[2ex]{}{(\overline{\psi}}{ \slashed{A} \psi ) _{x} (\overline{\psi} \slashed{A} }{\psi} % psibar-psi
            \bcontraction{(\overline{\psi} \slashed{A} }{\psi}{ ) _{x} (}{\overline{\psi}} % psi-psibar
            (\overline{\psi} \slashed{A} \psi ) _{x} (\overline{\psi} \slashed{A} \psi ) _{y}
        \right]}_{\Circled{D}}
    \,.
    \end{align}
\end{enumerate}

For any of these we can write a specific propagator, whose expression is 
%
\begin{align}
S_X = \frac{(-iq)^2}{2!} \int \dd[4]{x} \dd[4]{y} \Circled{X}
\,,
\end{align}
%
where \(\Circled{X}\) is any of the terms we have just written.

\subsubsection{No propagators: \(\Circled{A}\)}

The diagram looks like the product of two disconnected first-order diagrams. Since they were not physically allowed, this is not allowed either. 

If our vector were massive, like the \(Z\) boson, this would describe two separate decay processes. 

\subsubsection{One fermion propagator: \(\Circled{B_1}\)}

This term looks like 
%
\begin{align}
S_{B_1 } &= \frac{(-iq)^2}{2!} \int \dd[4]{x} \dd[4]{y} \Circled{B_1}  \\
&= \frac{- q^2}{2!} \int \dd[4]{x} \dd[4]{y}
N \left[
    \contraction{}{(\overline{\psi}}{ \slashed{A} \psi ) _{x} (\overline{\psi} \slashed{A} }{\psi} % psibar-psi
    (\overline{\psi} \slashed{A} \psi ) _{x} (\overline{\psi} \slashed{A} \psi ) _{y}
\right] +
N \left[
    \contraction{(\overline{\psi} \slashed{A} }{\psi}{ ) _{x} (}{\overline{\psi}} % psi-psibar
    (\overline{\psi} \slashed{A} \psi ) _{x} (\overline{\psi} \slashed{A} \psi ) _{y}
\right]  \\
&= - q^2 \int \dd[4]{x} \dd[4]{y} 
N \qty[
    \overline{\psi} (x) \slashed{A}(x)
    S_F(x-y) 
    \slashed{A}(y) \psi (y)
]
\,,
\end{align}
%
since 
%
\begin{align}
S_F(x-y) = \bcontraction{}{\psi}{ (x) }{\overline{\psi}}
\psi (x) \overline{\psi} (y)
\,,
\end{align}
%
and the two contributions are equal, since inside of the normal ordering the fermion operators anticommute, we need to swap an even number of them, and changing \(x \leftrightarrow y\) is not a problem since they are both integrated away.  

Now, we have 4 free fields, which correspond to \(2^{4}= 16\) terms in the \(\pm\) fields. 

\begin{claim}
The only physical processes are those in which 2 particles are created and 2 are annihilated.
\end{claim}

\begin{proof}
4-momentum conservation must hold: its \(0\)-th component already disqualifies \(0 \leftrightarrow 4\) processes, since on one side of the process we have zero energy. 

As for \(1 \leftrightarrow 3\) processes: if the 1 particle is a fermion, we go to its rest frame, then the \(0\)-th component of the momentum conservation imposes that the sum of the energies of the 3 particles must equal the mass of the fermion. One of the three particles is that kind of fermion, so its energy is \(\geq \) the mass, which implies that the photons must have zero energy. 

If the 1 particle is a photon, we go to the center of mass of the three particles, where the three-momentum is zero, and find a contradiction since the photon does not have a rest frame. 
\end{proof}

The possible diagrams are then only six. They are: 
%
\begin{align}
(\overline{\psi}_{-} \slashed{A}_{-})_{x} S_F (x-y) (\psi_{+} \slashed{A}_+)_{y} =
\feynmandiagram[baseline=(y.base), large, horizontal = y to x]{
    g1 [particle = \(\gamma \)] -- [photon, edge label = {\(\slashed{A}_+(y)\)}] y,
    e1 [particle = \(e^{-}\)] -- [fermion, edge label = {\(\psi_{+} (y)\)}] y,
    y -- [fermion, edge label = {\(S_F(x-y)\)}] x ,
    x -- [photon, edge label = {\(\slashed{A}_{-}(x)\)}] g2 [particle = \(\gamma \)],
    x -- [fermion, edge label = {\(\overline{\psi}_{-}(x)\)}] e2 [particle = {\(e^{-}\)}]
};
\,,
\end{align}
%
which contributes to Compton scattering, 
%
\begin{align}
(\overline{\psi}_{-} \slashed{A}_{+})_{x} S_F (x-y) (\psi_{+} \slashed{A}_{-})_{y} =
\begin{tikzpicture}[baseline=(y)]
    \begin{feynman}
        \vertex (y);
        \vertex [below left = of y] (e1) {\(e^{-}\)};
        \vertex [right = 2cm of y] (x);
        \vertex [below right = of x] (e2) {\(e^{-}\)};
        \vertex [above left = of y] (g1) {\(\gamma \)};
        \vertex [above right = of x] (g2) {\(\gamma \)};
        \diagram*{
        (e1) -- [fermion] (y)  -- [fermion] (x) -- [fermion] (e2),
        (y) -- [photon] (g2),
        (g1) -- [photon] (x), 
        };
    \end{feynman}
\end{tikzpicture}
\,,
\end{align}
%
which also contributes to Compton scattering, and 
%
\begin{align}
(\overline{\psi}_{+} \slashed{A}_{-})_{x} S_F (x-y) (\psi_{+} \slashed{A}_-)_{y} =
\begin{tikzpicture}[baseline=(center)]
    \begin{feynman}
        \vertex (y);
        \vertex [above left = of y] (e1) {\(e^{-}\)};
        \vertex [below = 2cm of y] (x);
        \vertex [below left = of x] (e2) {\(e^{-}\)};
        \vertex [above right = of y] (g1) {\(\gamma \)};
        \vertex [below right = of x] (g2) {\(\gamma \)};
        \node (center) at ($(y)!0.5!(x)$) {};
        \diagram*{
        (e1) -- [fermion] (y)  -- [fermion] (x) -- [fermion] (e2),
        (y) -- [photon] (g1),
        (g2) -- [photon] (x), 
        };
    \end{feynman}
\end{tikzpicture}
\,,
\end{align}
which contributes to electron-positron annihilation into two photons. 

The other three are the left-to-right mirror symmetric of these: two of them contribute to Compton scattering off a positron, one contributes to pair production from two photons. Their expressions are derived from the ones already written, mapping \(+ \leftrightarrow -\). They are: 
%
\begin{align}
&\qty(\overline{\psi}_{+} \slashed{A}_{+})_{x} 
  S_F (x-y) \qty(\psi_{-} \slashed{A}_{-})_{y} \\
&\qty(\overline{\psi}_{+} \slashed{A}_{-})_{x} 
  S_F (x-y) \qty(\psi_{-} \slashed{A}_{+})_{y} \\
&\qty(\overline{\psi}_{-} \slashed{A}_{+})_{x}
  S_F (x-y) \qty(\psi_{+} \slashed{A}_{-})_{y} 
\,.
\end{align}

When we draw these diagrams, particles become antiparticles and creation becomes annihilation: so, we take the \(x\) and \(y\) points and swap their position, but in doing so we must keep the arrows rigidly pointing in the same direction, so the arrow in the propagator switches direction. 

As we have seen, different diagrams can contribute to the same physical process. We can have either \emph{equivalent} operators, which are the same if we exchange \(x \leftrightarrow y\), and ones which are not equivalent. 

The terms \((\overline{\psi}_{-} \slashed{A}_{-})_x (\psi_{+} \slashed{A}_+)_{y}\) and \((\overline{\psi}_{-} \slashed{A}_{+})_x (\psi_{+} \slashed{A})_{y}\), for example, are different and both contribute to Compton scattering. 

\subsubsection{One photon propagator: \(\Circled{B_2 }\)}

The term looks like: 
%
\begin{align}
S_{\Circled{B_2 }} &= - \frac{q^2}{2!}
\int \dd[4]{x} \dd[4]{y} N \qty[
    \contraction{(\overline{\psi} }{\slashed{A}}{ \psi ) _{x} (\overline{\psi} }{\slashed{A}} % ph-ph
    (\overline{\psi} \slashed{A} \psi ) _{x} (\overline{\psi} \slashed{A} \psi ) _{y}
]  \\
&=- \frac{q^2}{2!}
\int \dd[4]{x} \dd[4]{y}
D^{\mu \nu}_{F} (x-y) N \qty[
    \qty(\overline{\psi} \gamma_{\mu } \psi )_x
    \qty(\overline{\psi} \gamma_{\nu } \psi )_y
]
\,.
\end{align}

As before, we have 16 possible combinations of \(+\) and \(-\) when we open the normal-ordered product, but only the ones with two destructions and two creations are kinematically allowed. 

When drawing these, notice that the bare field always has the arrow pointing towards the vertex, while for the conjugate field the arrow always points away from it. 

% The \(++--\) and \(--++\) terms contribute to \(e^{+}e^{-} \to e^{+}e^{-}\), the \(+-+-\) term contributes to \(e^{-} e^{-} \to e^{-} e^{-}\), the \(-+-+\) term contributes to \(e^{+} e^{+} \to e^{+} e^{+}\).

What are the diagrams contributing to \(e^{+} e^{-} \to e^{+}e^{-}\)? 
There are exactly 4, the signs of the fermions being 
\begin{enumerate}
    \item \(++--\) and \(--++\);
    \item \(-++-\) and \(+--+\).
\end{enumerate}

The first two are equivalent, as are the second two (since the photon propagator is symmetric): so, we count only one of them twice. 
Now, we know that \(D^{\mu \nu } (x-y) = - \eta^{\mu \nu } D_F\) as long as we select \(\xi = 1\). Then, we can write 
%
\begin{align}
S_{e^{-}e^{+}} = - q^2 \int \dd[4]{x} \dd[4]{y} D_F^{\mu \nu } (x-y)
\qty(\underbrace{N \qty[
    \qty(\overline{\psi}_{-} \gamma_{\mu } \psi_{-} )_x
    \qty(\overline{\psi_{+}} \gamma_{\nu } \psi_{+} )_y
]}_{\Circled{A}}+
\underbrace{N \qty[
    \qty(\overline{\psi_{+}} \gamma_{\mu } \psi_{-} )_x
    \qty(\overline{\psi}_{-} \gamma_{\nu } \psi_{+} )_y
]}_{\Circled{B}}
)
\,.
\end{align}

The two diagrams are topologically distinct: \Circled{A} is called the \textbf{S-channel contribution}, in which the fermions annihilate into a virtual photon which propagates and then annihilates into fermions again.

On the other hand, \Circled{B} is called the \textbf{T-channel contribution}, in which the electron and positron exchange a virtual photon but are never annihilated. 

\subsubsection{One photon and one fermion propagator: \(\Circled{C_1 }\)}

We now consider the term 
%
\begin{align}
S_{\Circled{C_1 }} &= - \frac{q^2}{2!} \int \dd[4]{x} \dd[4]{y} 
N \left[
    \bcontraction{(\overline{\psi} }{\slashed{A}}{ \psi ) _{x} (\overline{\psi} }{\slashed{A}} % ph-ph
    \contraction{}{(\overline{\psi}}{ \slashed{A} \psi ) _{x} (\overline{\psi} \slashed{A} }{\psi} % psibar-psi
    (\overline{\psi} \slashed{A} \psi ) _{x} (\overline{\psi} \slashed{A} \psi ) _{y}
\right]+ 
N \left[
    \bcontraction{(\overline{\psi} }{\slashed{A}}{ \psi ) _{x} (\overline{\psi} }{\slashed{A}} % ph-ph
    \contraction{(\overline{\psi} \slashed{A} }{\psi}{ ) _{x} (}{\overline{\psi}} % psi-psibar
    (\overline{\psi} \slashed{A} \psi ) _{x} (\overline{\psi} \slashed{A} \psi ) _{y}
\right]  \\
&= - q^2 \int \dd[4]{x} \dd[4]{y} D^{\mu \nu }_{F} (x-y) 
N \qty[\overline{\psi}(x) \gamma_{\mu } S_F (x-y) \gamma_{\nu } \psi (x)]
\label{eq:fermion-self-energy}
\,,
\end{align}
%
since the two components can be cast into each other by renaming \(x \leftrightarrow y\) (which can always be done since they are integrated away), and permuting 4 (\(\equiv 0 \mod 2\)) fermion operators.

The \(0 \to 2\) and \(2 \to 0\) processes are not allowed (they enforce zero energy for all the particles). So, we have two possibilities: either an electron having a loop and coming back to itself, or a positron doing the same. 

\begin{figure}[ht]
\centering
\feynmandiagram[layered layout, horizontal = y to x]{
e1 -- [fermion] y -- [fermion] x -- [fermion] e2,
y -- [photon, half left] x
};
\quad
\feynmandiagram[layered layout, horizontal = y to x]{
e1 -- [anti fermion] y -- [anti fermion] x -- [anti fermion] e2,
y -- [photon, half left] x
};
\caption{Electron loop, positron loop.}
\label{fig:electron-loop-positron-loop}
\end{figure}

This is known as the electron (positron) self-energy. 

\subsubsection{Two fermion propagators: \(\Circled{C_2}\)}

This term looks like 
%
\begin{align}
S_{\Circled{C_2 }}
&= - \frac{q^2}{2} \int \dd[4]{x} \dd[4]{y}
N \left[
    \bcontraction{}{(\overline{\psi}}{ \slashed{A} \psi ) _{x} (\overline{\psi} \slashed{A} }{\psi} % psibar-psi
    \contraction{(\overline{\psi} \slashed{A} }{\psi}{ ) _{x} (}{\overline{\psi}} % psi-psibar
    (\overline{\psi} \slashed{A} \psi ) _{x} (\overline{\psi} \slashed{A} \psi ) _{y}
\right]  \\
&= \frac{q^2}{2} \int \dd[4]{x} \dd[4]{y} 
\Tr \qty[S_F (y-x) \gamma^{\mu } S_F(x-y) \gamma^{ \nu }] N \qty[A_{\mu } A_{\nu }]
\,.
\end{align}

Why can we write this as a trace? This comes from a permutation of the spinors; in order to make it explicit we restore the spinorial indices of the \(\gamma^{\mu }\) matrices, using the first letters of the Greek alphabet: 
%
\begin{align}
&N \left[
    \bcontraction{}{(\overline{\psi}_{\alpha }}{ \gamma^{\mu }_{\alpha \beta } \psi_{\beta } ) _{x} (\overline{\psi}_{\gamma } \gamma^{\nu }_{\gamma \delta }}{ \psi_{ \delta }} % 1-4
    \acontraction{(\overline{\psi}_{\alpha } \gamma^{\mu }_{\alpha \beta } }{\psi_{\beta }}{ ) _{x} (}{\overline{\psi}_{\gamma }}  % 2-3
    (\overline{\psi}_{\alpha } \gamma^{\mu }_{\alpha \beta } \psi_{\beta } ) _{x} (\overline{\psi}_{\gamma } \gamma^{\nu }_{\gamma \delta } \psi_{ \delta } ) _{y}
\right]  \\
&= 
(-)^3 
\bcontraction{}{\psi_{ \delta }}{}{\overline{\psi}_{\alpha }}
\psi_{ \delta } \overline{\psi}_{\alpha }
\bcontraction{}{\psi_{ \beta }}{}{\overline{\psi}_{\gamma }}
\psi_{ \beta } \overline{\psi}_{\gamma }
\gamma^{\mu }_{\alpha \beta } \gamma^{\nu }_{\gamma \delta }  \\
&= - S_F(y-x)_{\delta \alpha } \gamma^{\mu }_{ \alpha \beta }
S_{F}(x-y) _{\beta \gamma }
\gamma^{\nu }_{\gamma \delta } \\
&= - \Tr(S_F(y-x) \gamma^{\mu }
S_F(x-y)
\gamma^{\nu })
\,, 
\end{align}
%
where we are taking the trace since all the indices are contracted, including the first and last. 

Now, we only have one diagram: it looks like 

\begin{figure}[ht]
\centering
\feynmandiagram[layered layout, horizontal = y to x]{
    a -- [photon] y,
    y -- [half left, fermion] x,
    y -- [half right, anti fermion] x,
    x -- [photon] b
};
\caption{Photon self-energy.}
\label{fig:photon-self-energy}
\end{figure}

This corresponds to \(A^{\mu }_{+}(y) A^{\nu }_{-} (x)\), the only other valid one is the same except for \(x \leftrightarrow y\).  

\subsubsection{Three propagators: \(\Circled{D}\)}

There are no uncontracted fields: this kind of process cannot contribute to the physical properties of already-existing particles, but it does alter the vacuum. This becomes relevant when QED is coupled to gravity. 

\begin{figure}[ht]
\centering
\feynmandiagram[layered layout, horizontal = y to x]{
    y -- [photon] x,
    y -- [fermion, half left] x,
    y -- [anti fermion, half right] x,
};
\caption{Vacuum diagram.}
\label{fig:vacuum-QED}
\end{figure}

\subsubsection{Third order QED diagrams}

The \(S\)-matrix element looks like: 
%
\begin{align}
S = \frac{(-iq)^3}{3!} 
\int \dd[4]{x} \dd[4]{y} \dd[4]{z}
T \qty[ 
    (\overline{\psi} \slashed{A} \psi )_{x}
    (\overline{\psi} \slashed{A} \psi )_{y}
    (\overline{\psi} \slashed{A} \psi )_{z}
]
\,.
\end{align}

None of the propagators with zero, one 

\subsection{Expansion in momentum space}

We have seen that even terms which are not allowed kinematically are included in the position-space \(S\)-matrix expansion. 

In order to determine which states describe physical processes we need to take the matrix element with the initial and final states: 
%
\begin{align}
S_{fi} = \bra{f} S \ket{i}
\,,
\end{align}
%
where \(\ket{f} = \ket{\psi (t \to + \infty )}\) and \(\ket{i} = \ket{\psi (t \to - \infty )}\).
This matrix element is the transition amplitude; the transition probability is given by \(\abs{S_{fi}}^2\). 

This works well and is normalized, since if the states \(\ket{f}\) are a basis we have 
%
\begin{align}
\sum _{f} \abs{S_{fi}}^2
&= \sum_f \bra{i} S ^\dag \ket{f}\bra{f} S \ket{i}  \\
&= \bra{i} S ^\dag \mathbb{1} S \ket{i}  \\
&= 1
\,,
\end{align}
%
since \(S ^\dag S  = \mathbb{1}\). 

Now, when we compute the \(T\)-product we get propagators and uncontracted fields, the latter will act on the initial states.
Let us see how they do so. 

\subsubsection{Contractions with \(\ket{i}\)}

As a common QED example, consider 
%
\begin{align}
\ket{i} = \ket{e^{-}_{s} (p)}
= (2 \pi )^{3/2}
\sqrt{2 \omega_{p}}
c_s ^\dag (p) \ket{0}
\,.
\end{align}

Let us calculate the applications 
\todo[inline]{So ``contractions'' has a different meaning here as opposed to the contractions of field operators, right?}
of the annihilation operators \(\psi_{+}\) and \(\overline{\psi}_{+}\) on this state: we get 
%
\begin{align}
\psi_{+} (x) \ket{e^{-}_{s} (p)}
&= \int \dd[3]{k} \sqrt{\frac{2 \omega_{p}}{2 \omega_{k}}}
e^{-ikx} \sum_{r} u_r (k) c_r (k) c_s ^\dag (p) \ket{0}  \\
&= \int \dd[3]{k} \sqrt{\frac{2 \omega_{p}}{2 \omega_{k}}}
e^{-ikx} \sum_{r} u_r (k) \qty(\qty{ c_r (k), c_s ^\dag (p)} - c_s ^\dag (p) c_r (k)) \ket{0}  \\
&= \int \dd[3]{k} \sqrt{\frac{2 \omega_{p}}{2 \omega_{k}}}
e^{-ikx} \sum_{r} u_r (k) \delta_{rs} \delta^{(3)}(\vec{p} - \vec{k}) \ket{0}  \\
&= e^{-ipx} u_s (p) \ket{0}
\,.
\end{align}

The same reasoning applied to \(\overline{\psi}_{+}(x) \ket{e^{-}_{s}(p)}\) yields 0, since \(\qty{d_r, c_s ^\dag }= 0\). 
The same holds for any field operator which is unrelated to the state (like a photon annihilation operator). 

The useful relations in this sense are: 
%
\begin{align}
\psi_{+} (x) \ket{e^{-}_{s} (p)} &= e^{-ipx} u_s (p) \ket{0} \\
\overline{\psi}_{+} (x) \ket{e^{+}_{s} (p)} &= e^{-ipx} \overline{v}_s (p) \ket{0}  \\
A^{\mu }_{+} (x) \ket{\gamma_{\lambda } (p)} &= e^{-ipx} \epsilon^{\mu }_{\lambda } (p)\ket{0}  \\
\varphi ^\dag_{+} (x) \ket{s(p)} &= e^{-ipx} \ket{0} 
\,.
\end{align}

\subsubsection{Contractions with \(\bra{f}\)}

We take the adjoint of the example from before: 
%
\begin{align}
\bra{e^{-}_{s} (p)} = (2 \pi)^{3/2} \sqrt{2 \omega_{p}}
\bra{0} c_s (p)
\,,
\end{align}
%
and we apply from the right a creation operator \(\overline{\psi}_{-}(x)\): 
%
\begin{align}
\bra{e^{-}_{s}(p)}
\overline{\psi}_{-}(x)
&= \bra{0} \int \dd[3]{k}
\sqrt{ \frac{2 \omega_{k}}{2 \omega_{p}}}
e^{ikx} \sum _{r} \overline{u}_{r} (k) c_s (k) c_r ^\dag (p)  \\
&= \bra{0} e^{ipx} \overline{u}_{s} (p)
\,,
\end{align}
%
since \(\qty{c_s (k), c_r ^\dag (p)} = \delta_{rs} \delta^{(3)}(\vec{k} - \vec{p})\). Like before, we have 
%
\begin{align}
\bra{e^{-}_{s}(p)} \psi_{-} (x) = 0
\,.
\end{align}

A summary of useful relations: 
%
\begin{align}
\bra{e^{-}_{s} (p)} \overline{\psi}_{-} (x) &= \bra{0} e^{ipx} \overline{u}_{s} (p) \\
\bra{e^{+}_{s} (p)} \psi_{-} (x) &= \bra{0} e^{ipx} v_{s} (p) \\
\bra{\gamma_{\lambda } (p)} A^{\mu }_{-} (x) &= \bra{0} e^{ipx} e^{\mu *}_{\lambda }(p) \\
\bra{s(p)} \varphi ^\dag_{-} (x) &= \bra{0} e^{ipx} 
\,.
\end{align}

\subsubsection{\(S\)-matrix expansion in momentum space}

At \textbf{0th order} we get \(S^{(0)} = \mathbb{1}\), so the transition amplitude is \(\braket{f}{i}= \delta_{fi} \). 

At \textbf{1st order} things get more interesting. Let us consider the process \(e^{-} \to e^{-} \gamma \): so, we are setting 
%
\begin{align}
\ket{i} &= \ket{e^{-}_{s}(p)}  \\
\ket{f} &= \ket{e^{-}_{s'} (p') \gamma_{\lambda'} (k')}
\,,
\end{align}
%
so the only term which will contribute to the \(S\)-matrix expansion will be 
%
\begin{align}
S^{(1)}_{e^{-} \to e^{-}\gamma } 
= -iq \int \dd[4]{x} N \qty[ \overline{\psi}_{-} (x) \slashed{A}_{-} \psi_{+} (x)]
\,,
\end{align}
%
so the matrix element is 
%
\begin{align}
S^{(1)}_{fi} = -iq \int \dd[4]{x} \bra{e^{-}_{s'} (p') \gamma_{\lambda'} (k')} \overline{\psi}_{-} (x) \slashed{A}_{-} \psi_{+} (x) \ket{e^{-}_{s}(p)}
\,,
\end{align}
%
where any operator has a specific state it is acting on, if it acts on a state which is not its own the contribution vanishes. 
So, using the relations which were written before we have 
%
\begin{align}
S^{(1)}_{fi} &= -iq \epsilon^{\mu }_{\lambda' } (k') \overline{u}_{s'} (p') \gamma_{\mu } u_s (p) \int \dd[4]{x} e^{-i(p - p' k')x} \braket{0}{0}  \\
&= (2 \pi )^{4} \delta^{(4)} (p - p' - k') \mathcal{M}^{(1)}_{e^{-} \to e^{-}\gamma }
\,,
\end{align}
%
where the \textbf{Feynman amplitude} \(\mathcal{M}\) is given by 
%
\begin{align}
\mathcal{M}^{(1)}_{e^{-} \to e^{-}\gamma } =   
-iq \slashed{\epsilon}_{\lambda' } (k') \overline{u}_{s'} (p') \gamma_{\mu } u_s (p)
\,.
\end{align}

Let us make some observations on this result.

\begin{enumerate}
    \item The term \(\delta^{(4)}(p _{i} - p _{f})\) enforces momentum conservation in the matrix element: it removes all kinematically forbidden processes.\footnote{This is related to the fact that we are considering \emph{asymptotic} initial and final states: since they are at a diverging time difference, we are allowed to completely remove the energy uncertainty. }
    \item All the interaction physics is contained in the Feynman amplitude: we have the terms \(\overline{u}_{s'} (p')\) and \(\epsilon^{\mu }_{\lambda '} (k ') u_s (p)\) accounting for the spins and polarizations of the incoming particles, while \(-i q \gamma_{\mu }\) describes the structure of the QED interaction. 
    \item With the same conventions used in position space we can draw a diagram for the process in momentum space: 
    \begin{figure}[ht]
    \centering
    \feynmandiagram[layered layout, large, horizontal = e to x]{
    e [particle = {\(e^{-}_{s}\)}] -- [fermion, momentum = \(p\)] x [particle = {\(- i q \gamma^{\mu } \)}] -- [fermion, momentum = \(p'\)] e2 [particle = {\(e^{-}_{s'}\)}],
    x -- [photon, momentum = \(k'\)] p [particle = \(\gamma_{\lambda '} \)]
    };
    \caption{Momentum space diagram.}   
    \label{fig:eegamma-interaction}
    \end{figure}
    \item We can develop an automatic procedure to derive the Feynman amplitude directly from this diagram. 
\end{enumerate}

The rules for this procedure are as follows: 
\begin{enumerate}
    \item go to the end of the fermionic arrow and start moving backward;
    \item insert a term \(\overline{u}_{s'} (p')\) based on the momentum and spin of the outgoing electron; 
    \item encounter the vertex: insert a term \(-iq \gamma^{\mu }\) since we deal with QED;
    \item encounter the initial electron: insert a term \(u_s (p)\) based on its spin and momentum;
    \item include the polarization of the photon  with a term \(\epsilon^{\mu }_{\lambda '} (k')\).
\end{enumerate}

In order to develop these rules fully, we go to the second order. 

\subsubsection{Second order term: \(e^{-} \gamma \to e^{-} \gamma \)}

We consider the process of \textbf{Compton scattering}, so our initial and final states are: 
%
\begin{align}
\ket{i} &= \ket{e^{-}_{s} (p) \gamma_\lambda (k)}  \\
\ket{f} &= \ket{e^{-}_{s'} (p') \gamma_{\lambda'} (k')} 
\,,
\end{align}
%
and as we saw in the previous section the only diagrams which contribute are: 
%
\begin{align}
\begin{split}
S _{\text{Compton}} &= (-iq)^2
\int \dd[4]{x} \dd[4]{y} \\
& \underbrace{N [(\overline{\psi}_{-} \slashed{A}_{-})_{x} S_F (x-y) (\overline{\psi}_{+} \slashed{A}_{+})_{y}]}_{\Circled{A}}
+ \underbrace{N [(\overline{\psi}_{-} \slashed{A}_{+})_{x} S_F (x-y) (\overline{\psi}_{+} \slashed{A}_{-})_{y}]}_{\Circled{B}}
\end{split}
\,.
\end{align}

Let us start with the term \Circled{A}: 
%
\begin{align}
S^{A}_{fi} &= \bra{f} S^{A} \ket{i}  \\
&= - q^2 \int \dd[4]{x} \dd[4]{y} 
\bra{e^{-}_{s'} (p') \gamma_{\lambda'} (k')} 
N [(\overline{\psi}_{-} \slashed{A}_{-})_{x} S_F (x-y) (\overline{\psi}_{+} \slashed{A}_{+})_{y}]
\ket{e^{-}_{s} (p) \gamma_\lambda (k)}  \\
\begin{split}
&= -q^2 \int \frac{ \dd[4]{q}}{(2 \pi )^{4}} \overline{u}_{s'} (p') \slashed{\epsilon }_{\lambda'} (k') \widetilde{S}_{F} (q) \slashed{\epsilon }_{\lambda } (k) u_s (p) \times \\
&\phantom{=}\  \times \int \dd[4]{x} e^{-i (q - p' - k' )x}
\int \dd[4]{y} e^{i (q-p -k) y} \braket{0}{0} 
\end{split}  \\
&= \mathcal{M}_{A} (2 \pi )^{4} \delta^{4} (p + k - p' - k') 
\,,
\end{align}
%
where 
%
\begin{align}
\mathcal{M}_{A} = - q^2 \overline{u}_{s'} (p') \slashed{\epsilon }_{\lambda'} (k') \widetilde{S}_{F} (p+k) \slashed{\epsilon }_{\lambda } (k) u_s (p)
\,.
\end{align}

Note that we have substituted in \(\widetilde{S}_F(q)\), the fermion propagator in momentum space \eqref{eq:fermion-propagator-momentum-space}.
The exponential its definition contained allowed for the connection between the momentum conservation between the initial and final states. 

This can be derived from the momentum-space diagram, like before! The diagram is 

\begin{figure}[ht]
\centering
\feynmandiagram[layered layout, extra large, horizontal = ia to ib]{
e1 [particle = {\(e^{-}_{s}\)}] -- [fermion, momentum = \(p\)] ia [label = -45: \(-i q \gamma_{\mu }\)] -- [fermion, momentum = {\(q= p+k\)}] ib [label = 235 : {\(-iq \gamma_{\nu }\)}] -- [fermion, momentum = {\(p'\)}] e2 [particle = {\(e^{-}_{s'}\)}],
p1 [particle = {\(\gamma_{\lambda }\)}] -- [photon, momentum' = \(k\)] ia,
ib -- [photon, momentum' = \(k'\)] p2 [particle = {\(\gamma_{\lambda' }\)}]
};
\caption{Momentum  space, first contribution to the Compton effect. }
\label{fig:momentum-space-compton}
\end{figure}

The algorithm to get the amplitude goes as follows: 
\begin{enumerate}
    \item Start from the end of the fermion line and move backward;
    \item at the final fermion insert a term \(\overline{u}_{s'} (p')\);
    \item at the QED vertex insert a term \(-iq \gamma_{\nu }\);
    \item for the propagator insert a term \(\widetilde{S}_{F} (q)\);
    \item at the QED vertex insert a term \(-iq \gamma_{\mu }\);
    \item for the initial electron insert a term \(u_s (p)\);
    \item for the initial and final photons insert terms \(\epsilon^{\mu }_{\lambda }(k)\) and \(\epsilon^{\nu }_{\lambda '} (k')\), the index should match that of the QED vertex at which the photon is attached. 
\end{enumerate}

This yields: 
%
\begin{align}
\mathcal{M}_{A} = \overline{u}_{s'} (p')
\qty(-iq \gamma_{\nu })
\widetilde{S}_{F} (q)
\qty(-iq \gamma_{\mu })
u_s (p)
\epsilon^{\mu }_{\lambda }(k)
\epsilon^{\nu }_{\lambda '} (k')
\,.
\end{align}

Since this will be multiplied by the Dirac delta in the end, we are allowed to use momentum conservation when doing the computation. 
Following the fermion arrows ensures we preserve the spinorial structure. 

Now, we can do the same for the term \Circled{B}. This reads 
\begin{align}
S^{B}_{fi} &= \bra{f} S^{B} \ket{i}  \\
&= - q^2 \int \dd[4]{x} \dd[4]{y} 
\bra{e^{-}_{s'} (p') \gamma_{\lambda'} (k')} 
N [(\overline{\psi}_{-} \slashed{A}_{+})_{x} S_F (x-y) (\overline{\psi}_{+} \slashed{A}_{-})_{y}]
\ket{e^{-}_{s} (p) \gamma_\lambda (k)}  \\
\begin{split}
&= -q^2 \int \frac{ \dd[4]{q}}{(2 \pi )^{4}} \overline{u}_{s'} (p') \slashed{\epsilon }_{\lambda} (k) \widetilde{S}_{F} (q) \slashed{\epsilon }_{\lambda' } (k') u_s (p) \times \\
&\phantom{=}\  \times \int \dd[4]{x} e^{-i (q - p' + k )x}
\int \dd[4]{y} e^{i (q-p +k') y} \braket{0}{0} 
\end{split}  \\
&= \mathcal{M}_{A} (2 \pi )^{4} \delta^{4} (p + k - p' - k') 
\,,
\end{align}
%
so now we have momentum conservation like before, but \(q = p - k' = p' - k\). 

This diagram is topologically different from the one before: now the outgoing photon is created in the \emph{first} vertex, the incoming photon is destroyed in the \emph{second} vertex.

Applying the same procedure as before, we find 
%
\begin{align}
\mathcal{M}_{B} = \overline{u}_{s'} (p') \qty(-iq \gamma_{\mu })
\widetilde{S}_{F} (p - k') \qty(-i q \gamma_{\nu }) u_s (p) 
\epsilon^{\lambda }_{\mu } (k) \epsilon^{\lambda '}_{\nu } (k')
\,.
\end{align}

% Note that this is \textbf{different} from \(\mathcal{M}_A\)! 
The total probability amplitude is then given by 
%
\begin{align}
S^{(2)} (e^{-} \gamma ) = (2\pi )^{4} \delta^{(4)}
(p + k - p' - k') \qty(\mathcal{M_A} + \mathcal{M}_{B}) 
\,.
\end{align}

\subsubsection{Second order term: photon self-energy}

We have seen that there is a second-order QED term in which both the initial and final states are photons, the \textbf{photon self-energy}: 
%
\begin{align}
\ket{i} &= \ket{\gamma_{\lambda } (p)} \\
\ket{f} &= \ket{\gamma_{\lambda' } (p')}
\,,
\end{align}
%
so that we have both a fermion and an antifermion propagator: 
%
\begin{align}
S^{(2)}_{\gamma \gamma } = 
- \frac{q^2}{2} \int \dd[4]{x} \dd[4]{y}
(-) \Tr \qty(S_F(y-x) \gamma^{\mu } S_F(x-y) \gamma^{\nu })
N[A_{\mu } (x) A_{\nu } (y)]
\,.
\end{align}

As we calculate the matrix element we get a factor two, since the signs of the two \(A_{\mu }\), \(A_{\nu }\) can be either \(+ -\) or \(- +\) equivalently: we have 
%
\begin{align}
N[A_{\mu }(x) A_{\nu } (y)] = N \qty[
    A_{\mu +} (x) A_{\nu -} (y) +
    A_{\mu -} (x) A_{\nu +} (y) 
]
\,,
\end{align}
%
since there is \emph{a priori} no causal constraint between \(x\) and \(y\). These two terms give an equal contribution (they are symmetric if we substitute \(\mu \leftrightarrow \nu \) ad \(x \leftrightarrow y\)), so we find 
%
\begin{align}
\begin{split}
S^{(2)}_{fi} &=
q^2 \int \frac{ \dd[4]{k_1 }}{(2 \pi )^{4}}
\int \frac{ \dd[4]{k_2 }}{(2 \pi )^{4}}
\epsilon^{\mu}_{\lambda }(p) \epsilon^{\nu }_{\lambda '} (p') 
\Tr \qty[\widetilde{S}_{F}(k_2 ) \gamma_{\mu } \widetilde{S}_{F} (k_1 )\gamma_{\nu }] \times \\
&\phantom{=}\ \times 
\int \dd[4]{x} e^{-i(p - k_2 + k_1 )x}  
\int \dd[4]{y} e^{-i (p' - k_2 + k_1 ) x}
\end{split}
\,.
\end{align}

The diagram for this interaction is that shown in figure \ref{fig:photon-self-energy}. 
Now, we can do one of the integrals in momentum space, but the other remains: we have 
%
\begin{align}
S^{(2)}_{fi} &= q^2 \int \dd[4]{k_1 } \int \dd[4]{k_2 } \delta^{(4)}(p - k_1 + k_2 )
\delta^{(4)} (p' - k_2 + k_1 )
\epsilon^{\nu }_{\lambda }( p) \epsilon^{\mu }_{\lambda '} (p') \Tr[\dots]  \\
&= q^2 \int \dd[4]{k_1 } \delta^{(4)} (p' - (p + k_1 ) + k_1 ) \epsilon^{\nu }_{\lambda }( p) \epsilon^{\mu }_{\lambda '} (p') \Tr[\dots]  \\
&= ( 2 \pi )^{4} \delta^{(4)} (p - p') \underbrace{q^2\int \frac{ \dd[4]{k_1 }}{(2 \pi )^{4}}
\epsilon^{\nu }_{\lambda }( p) \epsilon^{\mu }_{\lambda '} (p') \Tr[\widetilde{S}_{F}(p' + k_1  ) \gamma_{\mu } \widetilde{S}_{F} (k_1 )\gamma_{\nu }]}_{\mathcal{M}}
\,.
\end{align}

Note that inside of \(\mathcal{M}\) we have a factor \(-1\) because of the fact that it is a fermionic loop, so we need to swap an odd number of fields. 

Momentum conservation is enforced by the delta, but there is a degree of arbitrariness in the momentum of the loop: \(k_1 \) is free to vary, as long as \(k_2 \) varies with it. We integrate over all of the possible configurations in which this loop can occur.  

In general, the number of loops \(L\) is given in terms of the number of fermion propagators \(P_e\), of photon propagators \(P_{\gamma }\), of vertices \(V\) as: 
%
\begin{align}
L = P_e - P_\gamma - V + 1
\,.
\end{align}

In this case, we have \(L =1 \).

\subsubsection{Divergences}

Let us give an estimate of this self-energy, assuming \(k \gg m, p\) (so, in the ``ultraviolet'': the loop is extremely energetic). 

We then have 
%
\begin{align}
S^{(2)}_{fi} \sim \int \dd[4]{k} \frac{\Tr[(\slashed{p+k} +m) \slashed{\epsilon } \qty(\slashed{k} + m) \slashed{\epsilon }]}{(k^2-m^2) \qty((k+p)^2- m^2)} \sim \int \dd[4]{k} \frac{k^2}{k^{4}} \to \infty 
\,,
\end{align}
%
where we have used the expression \eqref{eq:fermion-propagator-momentum-space-explicit} for the propagator; in terms of orders of \(k\) we can switch the integral to \(\int \dd[4]{k} \sim \int k^3 \dd{k} \). So, we are left with \(\int k \dd{k}\), which diverges \textbf{quadratically}. 

\begin{claim}
    A similar argument can be applied to show that the electron self-energy is divergent as well.
\end{claim}

\begin{proof}    
The matrix element comes from the \(S\)-matrix in equation \eqref{eq:fermion-self-energy}; expressing everything in momentum space we find: 
%
\begin{align}
\begin{split}
S^{(2)}_{fi} &= - q^2 \int \frac{ \dd[4]{k_{\gamma }}}{(2 \pi )^{4}}
\int \frac{ \dd[4]{k_{F}}}{(2 \pi )^{4}}
\widetilde{D}^{\mu \nu }(k_\gamma )
\overline{u}_{s'} (p') \gamma_{\mu } \widetilde{S}_{F}(k_F) \gamma_{\nu } u_s (p) \\
&\phantom{=}\ 
\underbrace{\int \dd[4]{x} e^{i (p' -k_\gamma - k_F) x}  }_{(2 \pi )^{4} \delta^{(4)} (p' - k_\gamma - k_F)}
\underbrace{\int \dd[4]{y} e^{-i (p - k_{\gamma } - k_F) y}}_{(2 \pi )^{4} \delta^{(4)} (p - k_\gamma - k_F)}
\end{split}  \\
&= - q^2 
( 2\pi )^{4} \delta^{(4)} (p - p')
\int \frac{ \dd[4]{k_{\gamma }}}{(2 \pi )^{4}}
\widetilde{D}^{\mu \nu }(k_\gamma )
\overline{u}_{s'} (p') \gamma_{\mu } \widetilde{S}_{F}(p' - k_{\gamma }) \gamma_{\nu } u_s (p) 
\,,
\end{align}
%
so like before we can do power-counting to the integral in \(\dd[4]{k_\gamma }\): we have 
%
\begin{align}
\mathcal{M} &= - q^2 \int \frac{ \dd[4]{k_{\gamma }}}{(2 \pi )^{4}}
\widetilde{D}^{\mu \nu }(k_\gamma )
\overline{u}_{s'} (p') \gamma_{\mu } \widetilde{S}_{F}(p' - k_{\gamma }) \gamma_{\nu } u_s (p)   \\
&= - q^2 \int \frac{ \dd[4]{k_{\gamma }}}{(2 \pi )^{4}}
\frac{-i \eta^{\mu \nu }}{k^2+ i \epsilon }
\overline{u}_{s'} (p') \gamma_{\mu } 
% \frac{\slashed{p' - k_\gamma } + m}{\qty(p' - k_\gamma )^2 - m}
\frac{(p' - k_\gamma )_{\alpha}\gamma^{\alpha} + m}{\qty(p' - k_\gamma )^2 - m}
\gamma_{\nu } u_s (p)  \\
&\sim \int \dd[4]{k_\gamma } \frac{1}{k^2} \frac{k}{k^2} \sim \int \dd{k_\gamma } k_{\gamma }^3 k_\gamma^{-2} k_\gamma^{-1} \sim \int \dd{k_\gamma }
\,,
\end{align}
%
which still diverges, but this time only linearly. 
\end{proof}

\end{document}
