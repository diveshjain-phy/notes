\documentclass[main.tex]{subfiles}
\begin{document}

\marginpar{Thursday\\ 2020-3-12, \\ compiled \\ \today}

\section*{General information}

Written \& oral exam. 

As opposed to what was done in the physics curriculum, here there will be no grade truncation: we can grade a grade up to a 30 with the written exam only.

The suggested book is D'Auria \& Trigiante \cite{dauriaSpecialRelativityFeynman2011}. 
For the second part it is also useful to have a look at Mandl \& Shaw \cite{mandlQuantumFieldTheory2010}.


Live question time in Zoom at half past 11 on Mondays. 

Things which will be taken for granted: four-vectors, Lorenz and Poincaré groups, basics of QM, basics of linear operators. 

\subsection*{Contents}

This course will deal with the basics of Relativistic Quantum Field Theory. 

We will discuss the Lagrangian formalism for a Classical Field Theory. 
We will quantize these theories using canonical quantization, specifically for a scalar, a Dirac fermion, and a vector boson. 

Then, we will introduce interactions in our Lagrangian: we will use the \(S\)-matrix expansion, and Feynman diagrams. 

\chapter{Relativistic Quantum Field Theory}

\section{The nonrelativistic wave equation}

We will review the derivation of the nonrelativistic Schrödinger equation. 
We find it starting from the correspondence principle: we start from the expression of the energy 
%
\begin{align}
E = \frac{p^2}{2m} + V(x)
\,,
\end{align}
%
and substitute the energy with \(E \rightarrow i \partial_{t}\), the momentum with \(\vec{p} \rightarrow - i \vec{\nabla}_{x}\) and the position with the position operator \(\vec{x}\), all acting on the wavefunction.
With this we get 
%
\begin{align}
i \pdv{\psi }{t} (\vec{x}, t)
= \qty(\frac{-\nabla^2 }{2m}+  V(\vec{x})) \psi (\vec{x}, t) 
\,.
\end{align}
%

We still need to assign a meaning to the wavefunction: this is given by the Bohr condition, which tells us that the probability density of finding the particle in a specific  region is
%
\begin{align}
\rho (\vec{x}, t) = \abs{\psi (\vec{x}, t)}^2 \geq 0 
\,.
\end{align}

This probability density must be normalized as an initial condition:
%
\begin{align}
\mathbb{P} (t_0 ) = \int_{\mathbb{R}^3} \dd[3]{x} \rho (\vec{x}, t_0 ) = 1
\,,
\end{align}
%
and we wish to show that it will also be normalized at later times: 
%
\begin{align}
\dv{\mathbb{P}}{t} &= 
\int_{\mathbb{R}^3} \dd[3]{x} \pdv{}{t} \abs{\psi (\vec{x}, t)}^2  \\
&= \int_{\mathbb{R}^3} \dd[3]{x} \qty(\psi^{*} \pdv{\psi }{t} + \pdv{\psi^{*}}{t} \psi )
\,.
\end{align}

Using the Schrödinger equation we can substitute in the expression for the derivative of the wavefunction: 
%
\begin{align}
\dv{\mathbb{P}}{t} &= \int_{\mathbb{R}^3} \dd[3]{x} \qty{
\psi^{*}  \frac{1}{i}\qty(-\frac{\nabla^2}{2m}+V) \psi - \frac{1}{i} \psi \qty(- \frac{\nabla^2}{2m} + V) \psi^{*}   
}  \\
&= \frac{i}{2m} \int_{\mathbb{R}^3} \dd[3]{x} \qty{
\psi^{*} \nabla^2 \psi - 2m \psi^{*} V \psi 
- \psi \nabla^2 \psi^{*}
+ 2m \psi V \psi^{*}
}
\,,
\end{align}
%
and we use the fact that 
%
\begin{align}
\psi^{*} V \psi  = \psi V \psi^{*} = \qty(\psi^{*} V \psi )^{*}
\,,
\end{align}
%
which is true since \(V\) is a symmetric operator: it has real eigenvalues. This allows us to simplify the terms which include \(V\), and we find:
% \todo[inline]{There seem to be some \(2m\) factors missing in the formula in the notes.}
%
\begin{align}
\dv{\mathbb{P}}{t} &= \frac{i}{2m} \int_{\mathbb{R}^3 } \dd[3]{x} \qty{
\psi^{*} \nabla^2 \psi 
- \psi \nabla^2 \psi^{*}  
}  \\
&= \frac{i}{2m} \int_{\mathbb{R}^3} \dd[3]{x}
\nabla_{\vec{x}} \cdot \qty[
\psi^{*} \vec{\nabla} \psi - \psi\qty(\vec{\nabla} \psi^{*})]
\,,
\end{align}
%
where we integrated by parts\footnote{The calculation, expressed using index notation (and the Einstein summation convention) for clarity, is as follows: 
%
\begin{align}
\psi^{*} \partial_{i} \partial^{i} \psi  = \partial_{i} \qty(\psi^{*} \partial^{i} \psi)  - (\partial_{i} \psi^{*})(\partial^{i} \psi)
\,
\end{align}
%
and similarly for the other term. The terms which come out as the products of two gradients, \(\qty(\partial_{i} \psi^{*}) (\partial^{i} \psi)\), are equal for both the terms, so they simplify. Then, we are left with 
%
\begin{align}
\psi^{*} \partial_{i} \partial^{i} \psi - \psi \partial_{i} \partial^{i} \psi^{*} = \partial_{i} \qty(\psi^{*} \partial^{i} \psi - \psi \partial^{i} \psi^{*})
\,.
\end{align}
%
} so we can define 
%
\begin{align}
\vec{j} (\vec{x}, t) = - \frac{i}{2m} \qty(\psi^{*} \vec{\nabla} \psi - \psi \vec{\nabla} \psi^{*}  )
\,,
\end{align}
%
so that our equation now reads 
%
\begin{align}
\dv{\mathbb{P}}{t} = - \int_{\mathbb{R}^3} \dd[3]{x} \vec{\nabla}_{x} \cdot \vec{j} 
= \int_{\partial \mathbb{R}^3} \vec{j} \cdot \hat{n} \dd[2]{x} = 0
\,,
\end{align}
%
since the wavefunction is integrable: that is, it goes to zero \emph{quickly} as \(\abs{\vec{x}} \rightarrow \infty \). 
Therefore, \(\abs{\vec{j}} \rightarrow 0\) as \(\abs{\vec{x}} \rightarrow \infty \). For a more detailed explanation, see the Quantum Mechanics notes by Manzali \cite[page 147]{manzaliAppuntiDiFisica2019}.

So, if the probability is equal to one at a certain time than it keeps being equal to one. 

We can express this as a differential equation for the integrand: the \emph{continuity equation}, 
%
\begin{align}
\pdv{}{t} \abs{\psi  (\vec{x}, t)}^2 + \vec{\nabla} \cdot \vec{j} = 0
\,.
\end{align}

Let us now consider the way to solve the free Schrödinger equation: 
%
\begin{align}
i \pdv{\psi }{t} = - \frac{\nabla^2 \psi }{2m } 
\,.
\end{align}

We start from an ansatz of the equation being factorizable: \(\psi (\vec{x}, t) = \chi (t) \varphi (\vec{x})\). So, we get 
%
\begin{align}
i \pdv{\psi_{0}}{t} = \varphi (\vec{x}) i \pdv{\chi }{t}
\,
\end{align}
%
on the LHS, and 
%
\begin{align}
H_0 (\psi ) = - \chi (t) \frac{\vec{\nabla}^2}{2m} \varphi (\vec{x})
\,
\end{align}
%
on the RHS. Dividing both by \(\psi = \chi \varphi \) we get 
%
\begin{align}
i \frac{1}{\chi } \pdv{\chi }{t} = - \frac{1}{\varphi } \frac{\vec{\nabla}^2}{2m} \varphi 
\,,
\end{align}
%
and since these are dependent only on time (for the LHS) and only on position (for the RHS) they must be separately constant: let us call their value \(E\). Therefore, we can integrate them to get 
%
\begin{align}
\pdv{\chi }{t} = - i E \chi  \implies \chi (t) = \chi (0) \exp(-iEt)
\,
\end{align}
%
and 
%
\begin{align}
\nabla^2 \varphi = -2 m E \varphi  \implies \varphi (\vec{x}) = \varphi (0) \exp(i \vec{k} \cdot \vec{x})
\,.
\end{align}
%
Here , \(\vec{k}\) is a 3D vector such that \(\abs{\vec{k}}^2 = 2 m E\). 

This is called the \emph{dispersion relation}. So, the full solution, which is called a \emph{monochromatic} solution, is 
%
\begin{align}
\psi (\vec{x}, t) = \exp(-i \qty(Et - \vec{k} \cdot \vec{x}))
\,,
\end{align}
%
where \(\abs{\vec{k}}^2 = 2mE\).

The general solution will be a continuous superposition of solutions of this form: 
%
\begin{align}
\psi (\vec{x}, t) = \frac{1}{(2\pi )^{3/2}} \int \dd[3]{x} \widetilde{\varphi} (\vec{k}) \eval{\exp(-i \qty(\omega_{k} - \vec{k} \cdot \vec{x}))}_{\omega_{k} = \frac{\abs{\vec{k}}^2}{2m}}
\,.
\end{align}

Our conventions for the Fourier transform are:
%
\begin{align}
\varphi (\vec{x}) &= \frac{1}{(2\pi )^{3/2}} \int \dd[3]{x} \widetilde{\varphi} (\vec{k}) \exp(-i \vec{k}\cdot \vec{x})
\widetilde{\varphi} (\vec{x}) &= \frac{1}{(2\pi )^{3/2}} \int \dd[3]{x} \varphi (\vec{k}) \exp(i \vec{k}\cdot \vec{x})
\,,
\end{align}
%
so we use the symmetric definition. Other conventions have factors \((2\pi )^{-3}\) on one side and nothing on the other; it is the same but the we must be consistent. 

It is a theorem that \(\abs{\varphi }^2 = \abs{\widetilde{\varphi}}^2\), where the square norm of \(\varphi \), \(\abs{\varphi}^2\), is just the integral of \(\varphi^{*} \varphi \) over all 3D space. 

The 3D dirac delta function is defined as 
%
\begin{align}
\delta^{3 } (\vec{x} - \vec{y}) = \frac{1}{(2\pi )^3}
\int \dd[3]{k} \exp(- i \vec{k} \cdot (\vec{x} - \vec{y}))
\,,
\end{align}
%
and the 3D delta in the momentum space is perfectly analogous. 

The Schrödinger equation is manifestly \emph{non relativistic}: we started from the nonrelativistic expression \(E = p^2/2m + V\), so we should expect so. 
In the differential equation we have a second spatial derivative and a first temporal derivative: there is no way to write such an equation  covariantly. 

This kind of law of physics is only invariant under \emph{galilean transformations}, which do not change time. 

\end{document}