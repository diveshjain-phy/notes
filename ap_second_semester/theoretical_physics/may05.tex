\documentclass[main.tex]{subfiles}
\begin{document}

\section{Relativistic classical vector field}

\marginpar{Wednesday\\ 2020-6-10, \\ compiled \\ \today}

We want to write a relativistic field theory of particles such as the photon, which are vectors.
So, we require the vector transformation law under Lorentz transformations, 
%
\begin{align}
V^{\prime \mu } (x') = \tensor{\Lambda }{^{\mu }_{\nu }} V^{\nu } (x)
\,,
\end{align}
%
where \(\Lambda \in SO(1, 3)\) is a Lorentz transformation.
This theory will describe a spin-1 particle. 

Now, we know that massive spin-1 particles have three possible \(s_z\) values while massless ones only have two. Will this be an issue? Stay tuned to find out. We shall start with the massive theory. 

\subsection{Massive vector field theory}

We need to make an ansatz for our Lagrangian --- we require it to be of second order in the derivatives and Lorentz invariant. We define the field strength as in the electromagnetic case: \(V_{\mu \nu } = 2 \partial_{[\mu} V_{\nu ]}\). With it we write the Lagrangian:
%
\begin{align}
\mathscr{L} = - \frac{1}{4}  V^{\mu \nu } V_{\mu \nu } + \frac{1}{2} M^2 V^{\mu }V_{\mu }
\,.
\end{align}

Let us compute its EL equations: they read 
%
\begin{align}
\pdv{\mathscr{L}}{V_{\sigma }} - \partial_{\rho } \pdv{\mathscr{L}}{\partial_{ \rho } V_{\sigma }} = 0
\,;
\end{align}
%
so we need to compute these two derivatives. In the derivative with respect to \(\partial_{\rho } V_{\sigma }\) we get: 
%
\begin{subequations}
\begin{align} \label{eq:derivative-kinetic-term-vector}
\pdv{V_{\mu \nu } V^{\mu \nu }}{\partial_{\rho }V_{\sigma }}
&= 2 V_{\mu \nu } \pdv{V^{\mu \nu }}{\partial_{\rho }V_{\sigma }}  \\
&= 2 V_{\mu \nu }\qty(\eta^{\mu \rho } \eta^{\nu \sigma } - \eta^{\nu \rho } \eta^{\mu \sigma })  \\
&= 2 V^{\rho \sigma } - 2 V^{\sigma \rho } = 4 V^{\rho \sigma }
\,.
\end{align}
\end{subequations}

Finally then, we find 
%
\begin{align}
M^2 V^{\sigma } + \partial_{\rho } \qty(V^{\rho \sigma }) = \qty(M^2 + \square )V^{\sigma } + \partial^{\sigma }\qty(\partial_{\rho }V^{\rho }) = 0
\,.
\end{align}

This is the \textbf{Proca equation}. If we take its divergence, we find the condition \( \partial_{\sigma } V^{\sigma } = 0\): we can then impose this constraint to simplify the equation to 
%
\begin{subequations}
\begin{align}
\qty(\square + M^2 ) V^{\sigma } &= 0   \\
\partial_{\sigma } V^{\sigma } &= 0
\,.
\end{align}
\end{subequations}

Note that this works \emph{only because} \(M \neq 0\): if the mass vanishes the divergence of the Proca equation vanishes identically, and the condition \(\partial_{\sigma } V^{\sigma }\) is not imposed. 

\subsection{Solving the massive Proca equation}

The procedure we will follow is to write a generic vector solution to the Klein-Gordon equation, and then to impose the vanishing divergence constraint to it. 
In the real-valued case, it will look like 
%
\begin{align}
V^{\mu } (x) &= \int \frac{ \dd[4]{k}}{(2 \pi )^{4}} \qty(f^{\mu }(k) e^{-ikx} + f^{\mu *} (k) e^{ikx}) 
\,.
\end{align}

If we impose the constraint that it be a solution of the KG equation \(\square V^{\sigma } + M^2 V^{\sigma } = 0\), we get 
%
\begin{align}
0= \int \frac{ \dd[4]{k}}{(2 \pi )^{4}} \qty(- k^2 + M^2) \qty(f^{\mu }(k) e^{-ikx} + f^{\mu *} (k) e^{ikx}) 
\,,
\end{align}
%
and like we did before we can force the particle to be on-shell with a delta-function, so we can remove a degree of freedom:
%
\begin{align}
f^{\mu }(k) = (2 \pi )^{5/2} \sqrt{2 \omega_{k}} \delta (k^2 - M^2)
\sum _{\lambda=0}^{3}
\epsilon^{\mu }_{\lambda }(k) a_\lambda (k)
\,.
\end{align}

The four vectors \(\epsilon^{\mu }_{\lambda }\) are called the \textbf{polarization vectors}; they describe the independent degrees of freedom in momentum space. Note that the index \(\mu \) is a four-vector index, while \(\lambda \) is a label for the position of the vector in the tuple. 

With this definition we can write 
%
\begin{align}
V^{\mu } (x) = \frac{1}{(2 \pi )^{3/2}}
\int 
\frac{ \dd[3]{k}}{\sqrt{2 \omega_{k}}}
\sum _{\lambda } \qty[
\underbrace{\epsilon^{\mu }_{\lambda }(k) a_\lambda e^{-ikx}}_{V_{+}^{\mu }(x)}
+
\underbrace{\epsilon^{\mu *}_{\lambda } (k) a_\lambda^{*} e^{ikx}}_{V_{-}^{\mu }(x)}
]_{k_0 = \omega_{k}}
\,,
\end{align}
%
and now we can impose the condition of vanishing divergence. This brings down a \(k_{\mu }\) from the exponentials, so we find 
%
\begin{align}
\epsilon^{\mu }_{\lambda } a_{\lambda } (-i k_\mu ) + \epsilon^{\mu *}_{\lambda } a^{*}_{\lambda } i k_\mu = 0 \implies \Re \qty[\epsilon^{\mu }_{\lambda } a_{\lambda } k_\mu ] = 0
\,.
\end{align}

\todo[inline]{The professor's notes do not mention the fact that we need to take the real part\dots}

This constraint removes one of the degrees of freedom for the polarization vector, leaving three, which is consistent with the fact that we have a massive spin-1 particle. 

\begin{claim}
If \(k^{\mu } = (\omega_{k}, 0, 0, k)^{\top}\) we can form a basis for the polarization vectors with 
%
\begin{subequations}
\begin{align}
\epsilon^{\mu }_{1} &= (0, 1, 0, 0)^{\top}  \\
\epsilon^{\mu }_{2} &= (0, 0, 1, 0)^{\top}  \\
\epsilon^{\mu }_{3} &= (k/M, 0, 0 , \omega_{k} / M)^{\top}  
\,.
\end{align}
\end{subequations}

They satisfy the following completeness and orthogonality relations: 
%
\begin{subequations}
\begin{align}
\epsilon^{\mu }_{ (\lambda )} \epsilon^{(\lambda' )}_{\mu } &= - \delta_{\lambda \lambda '} = \eta_{\lambda \lambda '} \\
\epsilon^{\mu }_{(\lambda )} \epsilon^{\nu }_{(\lambda )} &= -\eta^{\mu \nu } + \frac{1}{M^2} k^{\mu } k^{\nu }
\,.
\end{align}
\end{subequations}

So, they form an orthonormal basis for the spacelike subspace orthogonal to the timelike vector \(k^{\mu }\). 
\end{claim}

\begin{proof}
The conditions to be checked are that these are orthogonal to \(k_\mu \) and independent. 

For the orthogonality relation the interesting component to show is 
%
\begin{align}
\epsilon^{\mu }_{3} \epsilon^{3}_{\mu } = \frac{k^2}{M^2} - \frac{\omega_{k}^2}{M^2} = -1
\,.
\end{align}

The completeness relation is long to show directly but fast to see intuitively. The outer product of a basis vector with itself, \(\epsilon^{\mu }_{(\lambda )} \epsilon^{\nu }_{(\lambda )}\) (not summed) gives a rank-1 projection tensor onto the vector's subspace; adding all of these together yields the projection matrix onto the subspace \(k^{\perp}\). 
\end{proof}

The Nöther energy-momentum tensor reads: 
%
\begin{subequations}
\begin{align}
\widetilde{T}^{\mu }_{\nu } &= \pdv{\mathscr{L}}{\partial_{\mu } V_{\rho }} \partial_{\nu } V_{\rho } - \mathscr{L} \delta^{\mu }_{\nu }  \\
&= - V^{\mu \rho } \partial_{\nu } V_{\rho } + \frac{1}{4} V^{\alpha \beta } V_{\alpha \beta } \delta^{\mu }_{\nu } - \frac{1}{2} M^2 V^{\alpha} V_{\alpha } \delta^{\mu }_{\nu }
\,,
\end{align}
\end{subequations}
%
and its corresponding charges are:
%
\begin{subequations}
\begin{align}
P_{\mu } &= \int \dd[3]{x} \widetilde{T}^{0}_{\mu }  \\
&= \int \dd[3]{x} \qty(- V^{0 \rho } V_{\nu \rho } + \frac{1}{4} V^{\alpha \beta } V_{\alpha \beta } \delta^{0}_{\nu } - \frac{1}{2} M^2V^{\alpha }V_{\alpha } \delta^{0}_{\nu }) 
\,,
\end{align}
\end{subequations}
%
while the conserved currents corresponding to Lorentz transformations are: 
%
\begin{align}
J^{\mu }_{(\rho \sigma )} = 
2 x_{[\rho } \widetilde{T}^{\mu }_{\sigma ]}
+ (\mathscr{J}_{\mu \nu })_{\rho \sigma } V^{\mu \nu } V^{\lambda }
\,,
\end{align}
%
where 
%
\begin{align}
(\mathscr{J}_{\mu \nu })_{\rho \sigma } = 2 i \eta_{\mu [\rho |} \eta_{\nu |\sigma ]}
\,.
\end{align}

\begin{proof}
Under an infinitesimal Lorentz transformation \(\omega_{\mu \nu }\) the vector \(V^{\mu } \) transforms like: 
%
\begin{align}
V^{\prime \mu }(x') = V^{\mu } (x) + \tensor{\omega }{^{\mu }_{\nu }} V^{\nu }(x)
\,,
\end{align}
%
where the variation can be written as 
%
\begin{align}
\delta V^{\mu } = \frac{1}{2} \omega^{\rho \sigma } X^{\mu }_{\rho \sigma }
\,,
\end{align}
%
with 
%
\begin{align}
X^{\mu }_{\rho \sigma } = -i (\mathscr{J}^{\mu \nu })_{\rho \sigma } V_{\nu }
\,.
\end{align}

This holds, since it reduces to 
%
\begin{align}
\tensor{\omega }{^{\mu }_{\nu }} V^{\nu } = \frac{2}{2} \omega^{\rho \sigma } \delta^{\mu }_{[\rho } \delta^{\nu }_{\sigma ]} V_{\nu }
\,.
\end{align}

The same applies for the position vector, for which we have 
%
\begin{align}
Y^{\mu }_{\rho \sigma } = -i (\mathscr{J}^{\mu \nu })_{ \rho \sigma } x_{\nu }
\,.
\end{align}

Now we can apply the general formula: 
%
\begin{subequations}
\begin{align}
J^{\mu }_{(a)} &= \widetilde{T}^{\mu }_{\nu } Y^{\nu }_{(a)} - \pdv{\mathscr{L}}{\partial_{\mu } V^{\nu }} X_{(a)}^{\nu }  \\
J^{\mu }_{\rho \sigma }&= 2\widetilde{T}^{\mu }_{\nu  } \delta^{\nu  }_{[\rho  } \delta^{\alpha }_{\sigma ]} x_{\alpha } 
+ 2 V^{\mu}_{\nu } \delta^{\nu}_{[\rho } \delta^{\alpha }_{\sigma ]} V_{\alpha }  \\
&= 2\widetilde{T}^{\mu }_{[\rho } x_{\sigma ]}
+ V^{\mu }_{[\rho } V_{\sigma ]}
\,.
\end{align}
\end{subequations}
\end{proof}

The corresponding conserved charges can then be decomposed into \(L_{\rho \sigma } \) and \(S_{\rho \sigma }\). 

\begin{claim}
The Pauli-Lubanski pseudovector's modulus square in the rest frame is given by 
%
\begin{subequations}
\begin{align}
W^2 = -2 M^2 \left[\begin{array}{cccc}
0 & 0 & 0 & 0 \\ 
0 & 1 & 0 & 0 \\ 
0 & 0 & 1 & 0 \\ 
0 & 0 & 0 & 1
\end{array}\right]
\,.
\end{align}
\end{subequations}
\end{claim}

\todo[inline]{Kinda confusing statement in the professor's notes\dots}

\subsection{Hamiltonian formalism}

The conjugate field is given by 
%
\begin{align}
\pi^{\mu } = \pdv{\mathscr{L}}{\partial_0 V_{\mu }} = - V^{0 \mu }
\,,
\end{align}
%
so \(\pi^{0} = 0\), while \(\pi^{i} = V^{i0}\). 
Because one of Hamilton's equations is \(\partial_0 V_0 = \qty{\pi_0, H}\) we have that \(V_0 = \const\).
So, without loss of generality we set it to zero --- this essentially amounts to a constant shift of the Hamiltonian, as we will see in a moment. 

\begin{claim}
The Hamiltonian density is given by \(\mathscr{H} = \pi^{\mu } \partial_0 V_{\mu } - \mathscr{L}\), or equivalently \(\widetilde{T}^{00}\). Its explicit expression is 
%
\begin{align}
    \mathscr{H} = \frac{1}{2} \pi_{i}^2 + \frac{1}{4} V_{ij}^2 + \frac{M^2}{2} V_{i}^2
    \,.
\end{align}
\end{claim}

\begin{proof}

%
\begin{subequations}
\begin{align}
\mathscr{H} &= \pi^{\mu }\partial_{0} V_{\mu } - \mathscr{L} \\
&= \cancelto{}{\pi^{0} \partial_0 V_0} + \pi^{i} \partial_0 V_{i} 
+ \frac{1}{2} V^{0i} V_{0i} + \frac{1}{4} V^{ij} V_{ij} - \frac{1}{2} M^2 \qty(\cancelto{}{V^{0} V_0} + V^{i} V_i )  \marginnote{There is a factor \(2\) multiplying the \(V^{0i}\) term since we also account for \(V^{i0}\).}\\
&= \pi^{i}\partial_0 V_{i} - \frac{1}{2} \pi_{i}^2 + \frac{1}{4} V_{ij}^2 + \frac{1}{2} M^2 V_i^2  \\
&= \pi^{i} \underbrace{\qty(\partial_0 V_{i} - \partial_{i} V_{0})}_{-\pi_{i}} + \pi^{i} \partial_{i} V_0 - \frac{1}{2} \pi_{i}^2 + \frac{1}{4} V_{ij}^2 + \frac{1}{2} M^2 V_i^2  \\
&= \frac{1}{2} \pi_{i}^2 + \frac{1}{4} V_{ij}^2 + \frac{1}{2} M^2 V_i^2  
+ \pi^{i} \partial_{i} V_0 
\\
&= \frac{1}{2} \pi_{i}^2 + \frac{1}{4} V_{ij}^2 + \frac{1}{2} M^2 V_i^2  
\,,
\end{align}
\end{subequations}
%
where we have neglected all the terms containing \(V_0 \). 
What we have neglected can be written as \(- \partial_{i} \qty(\pi_{i} V_0 ) + M^2 V_0^2 /2\), by integrating by parts and using the equations of motion, which tell us that \(\partial_{i} \pi_{i} = M^2 V_0 \). 
\end{proof}

We have the following Poisson brackets: 
%
\begin{align}
\qty{V^{i} (\vec{x}, t), \pi^{j}(\vec{y}, t)} = \delta_{ij} \delta^{(3)} (\vec{x} - \vec{y})
\,,
\end{align}
%
while the \(\qty{V, V}\) and \(\qty{\pi, \pi }\) brackets vanish. 

\subsection{Canonical quantization}

We can quantize this theory using commutators (since it has integer spin).
The quantization procedure seems to break the covariance of the theory, since we impose the condition \(V_0 = 0\), which is not covariant. 

In fact, however, it is possible to quantize the theory if the massive vector field in a covariant way without changing much of our approach: the covariant commutator reads \cite[eq.\ 6.98]{greinerFieldQuantization1996}: 
%
\begin{align}
\qty[V^{\mu }(x), V^{\nu }(y)] = -i \qty(\eta^{\mu \nu } + \frac{p^{\mu }p^{\nu }}{M^2}) D_F(x-y)
\,,
\end{align}
%


\subsection{Massless classical vector field}

The massless case of the Proca Lagrangian is the usual Maxwell Lagrangian: 
%
\begin{align}
\mathscr{L} = - \frac{1}{4} F^{\mu \nu } F_{\mu \nu }
\,,
\end{align}
%
where \(F_{\mu \nu } = 2 \partial_{[\mu } A_{\nu ]}\). 
The Euler-Lagrange equations read \(\partial_{\mu } F^{\mu \nu } = 0\), or 
%
\begin{align}
\square A^{\mu } + \partial^{\mu } \partial_{\nu } A^{\nu } = 0
\,.
\end{align}

As we mentioned before, differentiating this equation yields an identity, not an additional constraint. So, we must say that this Lagrangian describes four degrees of freedom!?

\subsubsection{Gauge invariance}

An element which was not present in the massive case is gauge invariance. A gauge transformation is an internal transformation of the field \(A^{\mu } \) which is a symmetry of the Lagrangian. In the electromagnetic case, the gauge is \(A^{\prime \mu } (x) = A^{\mu } (x) + \partial^{\mu } \alpha (x)\), where \(\alpha (x)\) is a generic scalar function of spacetime. 

This symmetry is known as a \(U(1)\) symmetry, since in the minimal coupling approach it corresponds to a phase shift of the wave function by \(e^{i \alpha }\). 

In this case, the field strength is invariant under this transformation as well as the Lagrangian density. 

The massive Proca Lagrangian does not have this symmetry: the term \(A^{\mu } A_{\mu }\) changes if we change the gauge. 

So, in order to have a one-to-one mapping between a physical field configuration and a mathematical field we need to make a gauge choice. 

A commonly adopted one is the Coulomb gauge, \(\nabla \cdot \vec{A} = 0\); however this is not covariant. A covariant choice is 
%
\begin{align}
\partial_{\mu } A^{\mu } = 0
\,,
\end{align}
%
the \textbf{Lorentz gauge}. It can be always be reached by choosing \(\alpha \) such that \(\square \alpha = - \partial_{\mu } A^{\mu }\). 

This does not in fact fix the gauge, since we can still perform transformations which preserve the divergence of \(A\) --- specifically, the residual gauge \(\beta \) must satisfy \(\square \beta = 0\). 

If we forget this fact for a minute and only consider the Lorentz gauge, the EOM look like a \(M=0\) version of the  Proca 3-dof ones: 
%
\begin{align}
\square A^{\mu } = 0 \qquad \text{and} \qquad \partial_{\mu } A^{\mu } = 0 
\,.
\end{align}

\subsubsection{Hamiltonian description}

Even though the Lorentz gauge condition is covariant, the Poisson brackets we write are not; the conjugate fields as in the Proca case are \(\pi^{\mu} = - F^{0\mu}\), and the fact that \(\pi^{0} = 0\) means that they force \(A^{0} = \const\), which is not covariant. 

So, this theory \textbf{cannot be quantized in a covariant way}. 

Now we can either quantize it ignoring this problem, or we can change the Lagrangian and quantize a different theory. We choose the latter. 

\subsection{Gauge fixing Lagrangian}

What we can do is then to write the following Lagrangian: 
%
\begin{align}
\mathscr{L} = - \frac{1}{4} F^{\mu \nu } F_{\mu \nu } - \frac{1}{2 \xi } 
\qty(\partial_{\mu } A^{\mu })^2
\,,
\end{align}
%
where \(\xi \) is an arbitrary real parameter. This is known as a \textbf{gauge fixing Lagrangian}. This still has all the good properties the old one had, except for gauge invariance --- it explicitly depends on the divergence of \(A^{\mu }\), which as we saw can be arbitrarily determined by the gauge. 

\begin{claim}
This Lagrangian is equivalent to 
%
\begin{align}
\mathscr{L}' = - \frac{1}{2} \qty(\partial_{\mu } A_{\nu })\qty(\partial^{\mu }A^{\nu }) + \frac{1}{2} \frac{\xi-1}{\xi } \qty(\partial_{\mu } A^{\mu })^2
\,.
\end{align}
\end{claim}

\begin{proof}
The term \(1/(2\xi) (\partial A)^2 \) is the same in both Lagrangians. So, we are asking whether 
%
\begin{align}
- \frac{1}{4} F^{\mu \nu} F_{\mu \nu } \sim -\frac{1}{2} (\partial_{\mu } A_{\nu }) \qty(\partial^{\mu } A^{\nu }) + \frac{1}{2} \qty(\partial_{\mu } A^{\mu })^2
\,.
\end{align}

Let us open the field strength: 
%
\begin{align}
F^{\mu \nu } F_{\mu \nu } =
\qty(\partial_{\mu } A_{\nu } - \partial_{\nu } A_{\mu })
\qty(\partial^{\mu } A^{\nu } - \partial^{\nu } A^{\mu })
&= 2 \qty(\partial_{\mu } A_{\nu }) \partial^{\mu} A^{\nu }
- 2 \qty(\partial_{\mu } A_{\nu }) \partial^{\nu }A^{\mu }  
\,,
\end{align}
%
so we have recovered the term \((\partial_{\mu } A_{\nu })(\partial^{\mu } A^{\nu })\) in \(\mathscr{L}'\), now what is left to show is that 
%
\begin{align}
\frac{1}{2} \qty(\partial_{\mu } A_{\nu }) \partial^{\nu }A^{\mu }  
\sim \frac{1}{2} \qty(\partial_{\mu } A^{\mu })^2
\,.
\end{align}

This is accomplished by two integrations by parts: 
%
\begin{subequations}
\begin{align}
\qty(\partial_{\mu } A_{\nu }) \partial^{\nu }A^{\mu }
&\sim \cancelto{}{\partial_{\mu } \qty(A_{\nu } \partial^{\nu } A^{\mu })}
- A_{\nu} \partial^{\nu }\partial_{\mu } A^{\mu }   \\
&\sim
-\cancelto{}{\partial^{\nu } \qty(A_{\nu } \partial_{\mu }A^{\mu })}
+ \qty(\partial^{\nu } A_{\nu }) \qty(\partial_{\mu }A^{\mu })
\sim
 \qty(\partial_{\mu }A^{\mu })^2
\,.
\end{align}
\end{subequations}
\end{proof}

\begin{claim}
The Euler-Lagrange equations now read 
%
\begin{align}
\square A^{\sigma } - \qty(\frac{\xi -1}{\xi }) \partial^{\sigma }
\qty(\partial_{\mu } A^{ \mu })^2 = 0
\,.
\end{align}
\end{claim}

\begin{proof}
We start from the Lagrangian \(\mathscr{L}'\), which is easier. 

The term \(\pdv*{\mathscr{L}'}{A_{\sigma }}\) vanishes, so we only get 
%
\begin{subequations}
\begin{align}
0=\partial_{\rho } \qty(\pdv{\mathscr{L}'}{\partial_{\rho } A_{\sigma }}) 
&= \partial_{\rho } \partial^{\rho } A^{\sigma } - \partial_{\rho } 
\frac{\xi - 1}{\xi } \qty(\partial_{\mu } A^{\mu }) \pdv{}{\partial_{\rho } A_{\sigma }} \qty(\partial_{\mu } A_{\nu } \eta^{\mu \nu })  \\
&= \partial_{\rho } \partial^{\rho } A^{\sigma } - \partial_{\rho } 
\frac{\xi - 1}{\xi } \qty(\partial_{\mu } A^{\mu }) \eta^{\rho \sigma }  \\
&= \square A^{\sigma } - \partial^{\sigma } 
\frac{\xi - 1}{\xi } \qty(\partial_{\mu } A^{\mu }) =0
\,. 
\end{align}
\end{subequations}
\end{proof}

Now comes the kicker: since \(\xi \) is arbitrary, we can set it to \(\xi = 1\), so that the equations of motion become \(\square A^{\sigma } = 0\). This is known as the \textbf{Feynman gauge fixing term}.

\subsubsection{General solution}

With the condition \(\xi = 1\), the general solution reads 
%
\begin{align}
A^{\mu } (x) = \frac{1}{(2\pi )^{3/2}} \int \frac{ \dd[3]{k}}{\sqrt{2 \omega_{k}}} 
\sum _{\lambda } \epsilon^{\mu }_{(\lambda )}(k)
\qty(a_{(\lambda )}(k) e^{-ikx} + a^{*}_{(\lambda )} (l) e^{ikx})_{k_0 = \abs{k}}
\,.
\end{align}

This solution has no constraints on the polarization vectors \(\epsilon^{\mu }_{\lambda }\): there are four independent ones, so this is not electromagnetism. 
The four polarization vectors can be chosen to be an orthonormal basis. 

\subsubsection{Choices of polarization}

If we are given the photon wavevector \(k^{\mu }\), which is null, we can set the \(0-\)th polarization vector to 
%
\begin{align}
\epsilon^{\mu }_{(0)} = \left[\begin{array}{cccc}
1 & 0 & 0 & 0
\end{array}\right]^{\top}
= n^{\mu }
\,,
\end{align}
%
the third one to 
%
\begin{align}
\epsilon^{\mu }_{(3)} = \left[\begin{array}{cccc}
0 & 0 & 0 & 1
\end{array}\right]^{\top}
= \frac{k^{\mu } - \qty(n^{\nu }k_{\nu })n^{\mu }}{n^{\nu }k_{\nu }}
\,.
\end{align}

The other two can be chosen as an arbitrary orthonormal basis of the spacelike two-dimensional space which is left.
These will be the \textbf{transverse} polarizations. 

From what we know about photons, they are the only ones which are actually physical. 
We will need a way to remove the other two ones to recover electromagnetic theory; for now we keep them since they are needed in order to quantize the spin-1 theory in a covariant way. 

\subsubsection{Hamiltonian description}

We use the Lagrangian \(\mathscr{L}'\), with \(\xi =1 \): 
%
\begin{align}
\mathscr{L}' = - \frac{1}{2} \qty(\partial_{\mu } A_{\nu })\qty(\partial^{\mu } A^{\nu } )
\,.
\end{align}

So, we find 
%
\begin{align}
\pi^{\mu } = \pdv{\mathscr{L}}{\partial_0 A_{\mu }}
= - \partial^{0} A^{\mu }
\,,
\end{align}
%
and 
%
\begin{subequations}
\begin{align}
\mathscr{H} &= \pi^{\mu } \partial_0 A_{\mu }
- \mathscr{L}   \\
&= - \pi^{\mu } \pi_{\mu }
+ \frac{1}{2} \qty(\partial_{\mu } A_{\nu })\qty(\partial^{\mu } A^{\nu })  \\
&= - \pi^{\mu } \pi_{\mu } + \frac{1}{2} \qty(+ \pi_{\mu }\pi^{\mu } + \qty( \partial_{i } A_{0}) \qty(\partial^{i }A^{0})
+ \qty(\partial_{i} A_{j}) \qty(\partial^{i} A^{j}) )  \\
&= - \frac{1}{2} \pi^{\mu } \pi_{\mu }
+ \frac{1}{2}\qty(\partial_{i } A_{0})\qty(\partial^{i }A^{0})
+ \frac{1}{2} \qty(\partial_{i } A_{j})^2  \\
&=-\frac{1}{2} \pi_{0}^2
+ \frac{1}{2} \pi_{i}^2
- \frac{1}{2}\qty(\partial_{i} A_{0})^2
+ \frac{1}{2} \qty(\partial_{i } A_{j})^2 
\,.
\end{align}
\end{subequations}

This is not positive definite, due to the unphysical degree of freedom \(A_0 \).
However, we can still proceed. The total Hamiltonian is 
%
\begin{align}
H = \int \dd[3]{x} \mathscr{H} = \frac{1}{2} \dd[3]{x}
\qty{ \pi_{i}^2  + \qty(\partial_{i} A_{j})^2 - \pi_0^2 - \qty(\partial_{i} A_{0})^2}
\,.
\end{align}

With it, we also have the functional version of Hamilton's equations written for \(A^{\mu }\) and \(\pi^{\mu }\). 

\begin{claim}
We also have the Poisson brackets: 
%
\begin{align}
    \qty{A^{\mu } (\vec{x}, t), \pi^{\nu }(\vec{y}, t)} = \eta^{\mu \nu }
    \delta^{(3)} (\vec{x} - \vec{y})
    \,,
\end{align}
%
while the same-field brackets vanish.
\end{claim} 

\begin{proof}
The bracket reads: 
%
\begin{subequations}
\begin{align}
\qty{A^{\mu } (\vec{x}, t), \pi^{\nu }(\vec{y}, t)}
&= \int \dd[3]{z} \qty(
\fdv{A^{\mu }(\vec{x}, t)}{A^{\rho }(\vec{z}, t)}
\fdv{\pi ^{\nu }(\vec{y}, t)}{\pi _{\rho }(\vec{z}, t)}
-
\fdv{A^{\mu }(\vec{x}, t)}{\pi _{\rho }(\vec{z}, t)}
\fdv{\pi ^{\nu }(\vec{y}, t)}{A^{\rho }(\vec{z}, t)}
)  \\
&= \int \dd[3]{z} \delta^{(3)} (\vec{x} - \vec{z}) \delta^{\mu }_{\rho }
\delta^{(3)}(y-z) \eta^{\nu \rho }  \\
&= \delta^{(3)} (\vec{x} - \vec{y}) \eta^{\mu \nu }
\,.
\end{align}
\end{subequations}
\end{proof}

\begin{claim}
Starting from the original gauge-fixing Lagrangian with \(\xi = 1\) we obtain the same results. 
\end{claim}

\todo[inline]{Still to do exercise.}

\end{document}
