\documentclass[main.tex]{subfiles}
\begin{document}

\section{Relativistic classical vector field}

\marginpar{Wednesday\\ 2020-6-10, \\ compiled \\ \today}

We want to write a relativistic field theory of particles such as the photon, which are vectors.
So, we require the vector transformation law under Lorentz transformations, 
%
\begin{align}
V^{\prime \mu } (x') = \tensor{\Lambda }{^{\mu }_{\nu }} V^{\nu } (x)
\,,
\end{align}
%
where \(\Lambda \in SO(1, 3)\) is a Lorentz transformation.
This theory will describe a spin-1 particle. 

Now, we know that massive spin-1 particles have three possible \(s_z\) values while massless ones only have two. Will this be an issue? Stay tuned to find out. We shall start with the massive theory. 

\subsection{Massive vector field theory}

We need to make an ansatz for our Lagrangian --- we require it to be of second order in the derivatives and Lorentz invariant. We define the field strength as in the electromagnetic case: \(V_{\mu \nu } = 2 \partial_{[\mu} V_{\nu ]}\). With it we write the Lagrangian:
%
\begin{align}
\mathscr{L} = - \frac{1}{4}  V^{\mu \nu } V_{\mu \nu } + \frac{1}{2} M^2 V^{\mu }V_{\mu }
\,.
\end{align}

Let us compute its EL equations: they read 
%
\begin{align}
\pdv{\mathscr{L}}{V_{\sigma }} - \partial_{\rho } \pdv{\mathscr{L}}{\partial_{ \rho } V_{\sigma }} = 0
\,;
\end{align}
%
so we need to compute these two derivatives. In the derivative with respect to \(\partial_{\rho } V_{\sigma }\) we get: 
%
\begin{align}
\pdv{V_{\mu \nu } V^{\mu \nu }}{\partial_{\rho }V_{\sigma }}
&= 2 V_{\mu \nu } \pdv{V^{\mu \nu }}{\partial_{\rho }V_{\sigma }}  \\
&= 2 V_{\mu \nu }\qty(\eta^{\mu \rho } \eta^{\nu \sigma } - \eta^{\nu \rho } \eta^{\mu \sigma })  \\
&= 2 V^{\rho \sigma } - 2 V^{\sigma \rho } = 4 V^{\rho \sigma }
\,.
\end{align}

Finally then, we find 
%
\begin{align}
M^2 V^{\sigma } + \partial_{\rho } \qty(V^{\rho \sigma }) = \qty(M^2 + \square )V^{\sigma } + \partial^{\sigma }\qty(\partial_{\rho }V^{\rho }) = 0
\,.
\end{align}

This is the \textbf{Proca equation}. If we take its divergence, we find the condition \( \partial_{\sigma } V^{\sigma } = 0\): we can then impose this constraint to simplify the equation to 
%
\begin{align}
\qty(\square + M^2 ) V^{\sigma } &= 0   \\
\partial_{\sigma } V^{\sigma } &= 0
\,.
\end{align}

Note that this works \emph{only because} \(M \neq 0\): if the mass vanishes the divergence of the Proca equation vanishes identically, and the condition \(\partial_{\sigma } V^{\sigma }\) is not imposed. 

\subsection{Solving the massive Proca equation}

The procedure we will follow is to write a generic vector solution to the Klein-Gordon equation, and then to impose the vanishing divergence constraint to it. 
In the real-valued case, it will look like 
%
\begin{align}
V^{\mu } (x) &= \int \frac{ \dd[4]{k}}{(2 \pi )^{4}} \qty(f^{\mu }(k) e^{-ikx} + f^{\mu *} (k) e^{ikx}) 
\,.
\end{align}

If we impose the constraint that it be a solution of the KG equation \(\square V^{\sigma } + M^2 V^{\sigma } = 0\), we get 
%
\begin{align}
0= \int \frac{ \dd[4]{k}}{(2 \pi )^{4}} \qty(- k^2 + M^2) \qty(f^{\mu }(k) e^{-ikx} + f^{\mu *} (k) e^{ikx}) 
\,,
\end{align}
%
and like we did before we can force the particle to be on-shell with a delta-function, so we can remove a degree of freedom:
%
\begin{align}
f^{\mu }(k) = (2 \pi )^{5/2} \sqrt{2 \omega_{k}} \delta (k^2 - M^2)
\sum _{\lambda=0}^{3}
\epsilon^{\mu }_{\lambda }(k) a_\lambda (k)
\,.
\end{align}

The four vectors \(\epsilon^{\mu }_{\lambda }\) are called the \textbf{polarization vectors}; they describe the independent degrees of freedom in momentum space. Note that the index \(\mu \) is a four-vector index, while \(\lambda \) is a label for the position of the vector in the tuple. 

With this definition we can write 
%
\begin{align}
V^{\mu } (x) = \frac{1}{(2 \pi )^{3/2}}
\int 
\frac{ \dd[3]{k}}{\sqrt{2 \omega_{k}}}
\sum _{\lambda } \qty[
\underbrace{\epsilon^{\mu }_{\lambda }(k) a_\lambda e^{-ikx}}_{V_{+}^{\mu }(x)}
+
\underbrace{\epsilon^{\mu *}_{\lambda } (k) a_\lambda^{*} e^{ikx}}_{V_{-}^{\mu }(x)}
]_{k_0 = \omega_{k}}
\,,
\end{align}
%
and now we can impose the condition of vanishing divergence. This brings down a \(k_{\mu }\) from the exponentials, so we find 
%
\begin{align}
\epsilon^{\mu }_{\lambda } a_{\lambda } (-i k_\mu ) + \epsilon^{\mu *}_{\lambda } a^{*}_{\lambda } i k_\mu = 0 \implies \Re \qty[\epsilon^{\mu }_{\lambda } a_{\lambda } k_\mu ] = 0
\,.
\end{align}

\todo[inline]{The professor's notes do not mention the fact that we need to take the real part\dots}

This constraint removes one of the degrees of freedom for the polarization vector, leaving three, which is consistent with the fact that we have a massive spin-1 particle. 

\begin{claim}
If \(k^{\mu } = (\omega_{k}, 0, 0, k)^{\top}\) we can form a basis for the polarization vectors with 
%
\begin{align}
\epsilon^{\mu }_{1} &= (0, 1, 0, 0)^{\top}  \\
\epsilon^{\mu }_{2} &= (0, 0, 1, 0)^{\top}  \\
\epsilon^{\mu }_{3} &= (k/M, 0, 0 , \omega_{k} / M)^{\top}  \\
\,.
\end{align}

They satisfy the following completeness and orthogonality relations: 
%
\begin{align}
\epsilon^{\mu }_{ (\lambda )} \epsilon^{(\lambda' )}_{\mu } &= - \delta_{\lambda \lambda '} = \eta_{\lambda \lambda '} \\
\epsilon^{\mu }_{(\lambda )} \epsilon^{\nu }_{(\lambda )} &= -\eta^{\mu \nu } + \frac{1}{M^2} k^{\mu } k^{\nu }
\,.
\end{align}

So, they form an orthonormal basis for the spacelike subspace orthogonal to the timelike vector \(k^{\mu }\). 
\end{claim}

\begin{proof}
The conditions to be checked are that these are orthogonal to \(k_\mu \) and independent. 

For the orthogonality relation the interesting component to show is 
%
\begin{align}
\epsilon^{\mu }_{3} \epsilon^{3}_{\mu } = \frac{k^2}{M^2} - \frac{\omega_{k}^2}{M^2} = -1
\,.
\end{align}

The completeness relation is long to show directly but fast to see intuitively. The outer product of a basis vector with itself, \(\epsilon^{\mu }_{(\lambda )} \epsilon^{\nu }_{(\lambda )}\) (not summed) gives a rank-1 projection tensor onto the vector's subspace; adding all of these together yields the projection matrix onto the subspace \(k^{\perp}\). 
\end{proof}

The Nöther energy-momentum tensor reads: 
%
\begin{align}
\widetilde{T}^{\mu }_{\nu } &= \pdv{\mathscr{L}}{\partial_{\mu } V_{\rho }} \partial_{\nu } V_{\rho } - \mathscr{L} \delta^{\mu }_{\nu }  \\
&= - V^{\mu \rho } \partial_{\nu } V_{\rho } + \frac{1}{4} V^{\alpha \beta } V_{\alpha \beta } \delta^{\mu }_{\nu } - \frac{1}{2} M^2 V^{\alpha} V_{\alpha } \delta^{\mu }_{\nu }
\,,
\end{align}
%
and its corresponding charges are:
%
\begin{align}
P_{\mu } &= \int \dd[3]{x} \widetilde{T}^{0}_{\mu }  \\
&= \int \dd[3]{x} \qty(- V^{0 \rho } V_{\nu \rho } + \frac{1}{4} V^{\alpha \beta } V_{\alpha \beta } \delta^{0}_{\nu } - \frac{1}{2} M^2V^{\alpha }V_{\alpha } \delta^{0}_{\nu }) 
\,,
\end{align}
%
while the conserved currents corresponding to Lorentz transformations are: 
%
\begin{align}
J^{\mu }_{(\rho \sigma )} = 
2 x_{[\rho } \widetilde{T}^{\mu }_{\sigma ]}
+ (\mathscr{J}_{\mu \nu })_{\rho \sigma } V^{\mu \nu } V^{\lambda }
\,,
\end{align}
%
where 
%
\begin{align}
(\mathscr{J}_{\mu \nu })_{\rho \sigma } = 2 i \eta_{\mu [\rho |} \eta_{\nu |\sigma ]}
\,.
\end{align}

\begin{proof}
Under an infinitesimal Lorentz transformation \(\omega_{\mu \nu }\) the vector \(V^{\mu } \) transforms like: 
%
\begin{align}
V^{\prime \mu }(x') = V^{\mu } (x) + \tensor{\omega }{^{\mu }_{\nu }} V^{\nu }(x)
\,,
\end{align}
%
where the variation can be written as 
%
\begin{align}
\delta V^{\mu } = \frac{1}{2} \omega^{\rho \sigma } X^{\mu }_{\rho \sigma }
\,,
\end{align}
%
with 
%
\begin{align}
X^{\mu }_{\rho \sigma } = -i (\mathscr{J}^{\mu \nu })_{\rho \sigma } V_{\nu }
\,.
\end{align}

This holds, since it reduces to 
%
\begin{align}
\tensor{\omega }{^{\mu }_{\nu }} V^{\nu } = \frac{2}{2} \omega^{\rho \sigma } \delta^{\mu }_{[\rho } \delta^{\nu }_{\sigma ]} V_{\nu }
\,.
\end{align}

The same applies for the position vector, for which we have 
%
\begin{align}
Y^{\mu }_{\rho \sigma } = -i (\mathscr{J}^{\mu \nu })_{ \rho \sigma } x_{\nu }
\,.
\end{align}

Now we can apply the general formula: 
%
\begin{align}
J^{\mu }_{(a)} &= \widetilde{T}^{\mu }_{\nu } Y^{\nu }_{(a)} - \pdv{\mathscr{L}}{\partial_{\mu } V^{\nu }} X_{(a)}^{\nu }  \\
J^{\mu }_{\rho \sigma }&= 2\widetilde{T}^{\mu }_{\nu  } \delta^{\nu  }_{[\rho  } \delta^{\alpha }_{\sigma ]} x_{\alpha } 
+ 2 V^{\mu}_{\nu } \delta^{\nu}_{[\rho } \delta^{\alpha }_{\sigma ]} V_{\alpha }  \\
&= 2\widetilde{T}^{\mu }_{[\rho } x_{\sigma ]}
+ V^{\mu }_{[\rho } V_{\sigma ]}
\,.
\end{align}
\end{proof}

The corresponding conserved charges can then be decomposed into \(L_{\rho \sigma } \) and \(S_{\rho \sigma }\). 

\begin{claim}
The Pauli-Lubanski pseudovector's modulus square in the rest frame is given by 
%
\begin{align}
W^2 = -2 M^2 \left[\begin{array}{cccc}
0 & 0 & 0 & 0 \\ 
0 & 1 & 0 & 0 \\ 
0 & 0 & 1 & 0 \\ 
0 & 0 & 0 & 1
\end{array}\right]
\,.
\end{align}
\end{claim}

\todo[inline]{Kinda confusing statement in the professor's notes\dots}

\subsection{Hamiltonian formalism}

The conjugate field is given by 
%
\begin{align}
\pi^{\mu } = \pdv{\mathscr{L}}{\partial_0 V_{\mu }} = - V^{0 \mu }
\,,
\end{align}
%
so \(\pi^{0} = 0\), while \(\pi^{i} = V^{i0}\). 
Because one of Hamilton's equations is \(\partial_0 V_0 = \qty{\pi_0, H}\) we have that \(V_0 = \const\).
So, WLOG we set it to zero (with a gauge choice). 

\begin{claim}
The Hamiltonian density is given by \(\mathscr{H} = \pi^{\mu } \partial_0 V_{\mu } - \mathscr{L}\), or equivalently \(\widetilde{T}^{00}\). Its explicit expression is 
%
\begin{align}
    \mathscr{H} = \frac{1}{2} \pi_{i}^2 + \frac{1}{4} V_{ij}^2 + \frac{M^2}{2} V_{i}^2
    \,.
\end{align}
\end{claim}

\begin{proof}
\todo[inline]{Yeah}
\end{proof}


\end{document}
