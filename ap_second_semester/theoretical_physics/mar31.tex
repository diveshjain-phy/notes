\documentclass[main.tex]{subfiles}
\begin{document}

\marginpar{Wednesday\\ 2020-4-29, \\ compiled \\ \today}

So, we can write out the \textbf{general solution} for the Dirac equation: the spinor and its conjugate are
%
\begin{subequations}
\begin{align}
\psi (x) &=
\frac{1}{(2 \pi )^{3/2}}
\int \frac{ \dd[3]{k}}{\sqrt{2 \omega_{k}}}
\sum _{r=1}^{2} \qty[
    c_r(k) u_r(k) e^{-ikx}
    d^{*}_{r} (k) v_r(k) e^{ikx}
]_{\omega_{k} = k^{0}}
\\
\overline{\psi} (x) &=
\frac{1}{(2 \pi )^{3/2}}
\int \frac{ \dd[3]{k}}{\sqrt{2 \omega_{k}}}
\sum _{r=1}^{2} \qty[
    d_r(k) \overline{v}_r(k) e^{-ikx}
    c^{*}_{r} (k) \overline{u}_r(k) e^{ikx}
]_{\omega_{k} = k^{0}}
\,.
\end{align}
\end{subequations}

The functions \(c_r\) and \(d_r\) represent the two degrees of freedom of the positive and negative energy solution respectively.

The important thing to recall is 
%
\begin{align}
\psi_{+} \sim cu \qquad \text{and} \qquad
\psi_{-} \sim d ^* v
\,.
\end{align}

Note that \(c_r\) and \(d_r\) are scalar functions, while \(u_r\) and \(v_r\) are 4D spinors.

\subsection{Energy projectors}

We wish to distinguish positive and negative energy solutions: so, we define the projector onto the positive and negative energy subspaces, 
%
\begin{align}
\Lambda_{\pm } (k) = \frac{\pm \slashed{k} + M}{2M}
\,.
\end{align}

These two are an incomplete set of orthogonal projectors.

\begin{claim}
They satisfy 
%
\begin{subequations}
\begin{align}
\Lambda_{\pm}^2 (k) &= \Lambda_{\pm} (k)  \\
\Lambda_{+} (k) + \Lambda_{-} (k) &= \mathbb{1}  \\
\Lambda_{+} (k) \Lambda_{-} (k) &= 0  \\
\Tr[\Lambda_{\pm} (k)] &= 2
\,.
\end{align}
\end{subequations}
\end{claim}

\begin{proof}
Let us first establish what \(\slashed{k}^2\) is equal to: 
%
\begin{align}
\slashed{k}^2 =
\gamma^{\mu } k_{\mu } \gamma^{\nu } k_{\nu }
= \frac{1}{2} \qty{\gamma^{\mu }, \gamma^{\nu }}k_{\mu } k_{\nu }
= \eta^{\mu \nu } k_{\mu } k_{\nu }
= k^2 \mathbb{1}_{4}
\,.
\end{align}

So, we can compute 
%
\begin{subequations}
\begin{align}
\Lambda_{\pm}^2 (k) &= \frac{1}{4 M^2}\qty[\slashed{k}^2 + M^2 \mathbb{1} \pm 2 \slashed{k} M]  \\
&= \frac{1}{4 M^2} \qty[2 M \mathbb{1} \pm 2 \slashed{k}M] = \frac{\pm \slashed{k} + M}{2M}
\,.
\end{align}
\end{subequations}

For the second property, we get 
%
\begin{align}
\Lambda_{+} + \Lambda_{-} = \frac{+ \slashed{k} + M - \slashed{k} + M}{2M} = \frac{2M}{2M} = \mathbb{1} 
\,.
\end{align}

For the third expression, we find 
%
\begin{align}
\Lambda_{+} \Lambda_{-} \propto \qty(\slashed{k} + M) \qty(\slashed{k} - M) = \slashed{k}^2 - M^2 = 0
\,.
\end{align}

For the fourth expression, instead, we find that since \(\slashed{k}\) is traceless the trace is equal to that of \((M / 2 M)  \mathbb{1}\), which is equal to \(2\) since spinorial matrices are 4-dimensional.
\end{proof}

We can apply these projectors to the solutions we know how to write from the general expression \eqref{eq:dirac-equation-solutions}: 
%
\begin{subequations}
\begin{align}
\Lambda_{+} u_r (k) &= \frac{C}{2M} (\slashed{k} + M)^2 u_r(M) = C(\slashed{k} + M) u_r (M) = u_r(k) \\
\Lambda_{-} u_r (k) &= \frac{C}{2M} (-\slashed{k} + M) (\slashed{k} + M) u_r(M) = 0 \\
\Lambda_{+} v_r (k) &= \frac{C}{2M} (\slashed{k} + M) (\slashed{k} - M) u_r(M) = 0 \\
\Lambda_{-} v_r (k) &= \frac{C}{2M} (-\slashed{k} + M) (-\slashed{k} + M) u_r(M) = v_r(k)
\,.
\end{align}
\end{subequations}

These projectors can also be obtained from the vectors themselves: if a vector \(v\) is normalized to 1, the matrix \(v v ^{\top}\) is a projector onto the subspace of the vector.

We can add projectors together in order to get projectors onto larger subspaces.

\begin{claim}
We can recover the projectors \(\Lambda_{\pm }\) by the expressions: 
%
\begin{subequations} \label{eq:energy-projectors-spinor}
\begin{align}
\Lambda_{+} (k) &= \sum_{r} \frac{u_r(k) \overline{u}_{r} (k)}{2M} \\
\Lambda_{-} (k) &= - \sum_{r} \frac{v_r(k) \overline{v}_{r} (k)}{2M} 
\,.
\end{align}
\end{subequations}
\end{claim}

\begin{proof}
The calculation yields 
%
\begin{subequations}
\begin{align}
u_r(k) \overline{u}_{r}(k) &=
\left[\begin{array}{c}
\xi_{r}\sqrt{\omega_{k} + M} \\ 
\displaystyle
\frac{\vec{k} \cdot \vec{\sigma}}{\sqrt{\omega_{k} + M}} 
\xi_{r}
\end{array}\right]
\left[\begin{array}{cc}
\xi_{r}^{\top} \sqrt{\omega_{k} + M}, & 
\displaystyle
- \xi_{r}^{\top} \frac{\vec{k} \cdot \vec{\sigma}}{\sqrt{\omega_{k} + M}}
\end{array}\right]  \\
&= \left[\begin{array}{cc}
\xi_{r} \xi_{r}^{\top} (\omega_{k} + M) & - \xi_{r} \xi_{r}^{\top} (\vec{k} \cdot \vec{\sigma}) \\ 
(\vec{k} \cdot \vec{\sigma}) \xi_{r} \xi_{r}^{\top} & 
- \frac{ (\vec{k}\cdot \vec{\sigma}) \xi_{r} \xi_{r}^{\top} (\vec{k} \cdot \vec{\sigma})}{\omega_{k} + M}
\end{array}\right]
\,,
\end{align}
\end{subequations}
%
and now we notice the fact that \(\sum _{r} \xi_{r} \xi_{r} ^{\top} = \mathbb{1}_{2}\), so if we sum over \(r\) we get 
%
\begin{subequations}
\begin{align}
\sum _{r} u_r (k) \overline{u}_{r}(k) &=
\left[\begin{array}{cc}
\omega_{k} + M & - \vec{k} \cdot \vec{\sigma} \\ 
\vec{k} \cdot \vec{\sigma} & - \frac{\abs{k}^2}{\omega_{k} + M}
\end{array}\right]  \\
&= \left[\begin{array}{cc}
\omega_{k} + M & - \vec{k} \cdot \vec{\sigma} \\ 
\vec{k} \cdot \vec{\sigma} & - \omega_{k} + M
\end{array}\right]  \\
&= \slashed{k} + M
\,.
\end{align}
\end{subequations}

If we divide by \(2M\) we get our result. 
The computation for \(v_{r}\) is similar. 
\end{proof}

\subsection{Spin operators}

Recall, the spinorial representation of the Lorentz group is generated by the matrices \(\Sigma^{\mu \nu }\), which can be calculated by 
%
\begin{align}
\Sigma^{\mu \nu } = \frac{i}{4} \qty[\gamma^{\mu }, \gamma^{\nu }]
\,.
\end{align}

We can use these to define, in the Dirac representation, the generators of boosts: 
%
\begin{subequations}
\begin{align}
K_{i} = \Sigma^{i0} = -\frac{i}{2} \left[\begin{array}{cc}
0 & \sigma_{i} \\ 
\sigma_{i} & 0
\end{array}\right]
\,,
\end{align}
\end{subequations}
%
and of rotations: 
%
\begin{subequations}
\begin{align} \label{eq:rotation-spin-generator-definition}
\Sigma_{i} = \frac{1}{2} \epsilon_{ijk} \Sigma^{jk} 
= \frac{1}{2} \left[\begin{array}{cc}
\sigma_{i} & 0 \\ 
0 & \sigma_{i}
\end{array}\right]
\,.
\end{align}
\end{subequations}

We can exponentiate these to get the representations of general boosts and rotations: a boost will be written as 
%
\begin{align}
S(\Lambda) = \exp( - \frac{i}{2} \omega_{0i} K_{i} )
\,,
\end{align}
%
while a rotation will be written as 
%
\begin{align}
S(\Lambda ) = \exp( - \frac{i}{2} \theta_{k} \Sigma_{k})
\,,
\end{align}
%
where we used the angle \(\theta_{k} = \epsilon_{ijk} \omega^{ij}\).
These \(\Sigma_{k}\) are called the spin operators. They result in an \emph{internal rotation} of the spinor corresponding to a rotation of our coordinates.

In the rest frame of the particle, the basis states we chose are the eigenstates of \(\Sigma_{3}\): We have 
%
\begin{subequations}
\begin{align}
\Sigma_{3} u_r (M) = \frac{1}{2} \left[\begin{array}{cc}
\sigma_3  & 0 \\ 
0 & \sigma_3 
\end{array}\right]
\sqrt{2M}
\left[\begin{array}{c}
\xi_{r} \\ 
0
\end{array}\right]
= \pm \frac{1}{2} u_R (M)
\,,
\end{align}
\end{subequations}
%
where we have either \(+\) or \(-\) depending on whether \(r=1\) or \(2\), respectively.

Also, the negative energy solutions are basis states: we get 
%
\begin{subequations}
\begin{align}
\Sigma_{3} v_r (M) = \frac{1}{2}
\left[\begin{array}{cc}
\sigma_3  & 0 \\ 
0 & \sigma_3 
\end{array}\right]
\sqrt{2M}
\left[\begin{array}{c}
0   \\
\xi_{r} 
\end{array}\right]
= \pm \frac{1}{2} v_r(M)
\,.
\end{align}
\end{subequations}

So, our particles have spin \(1/2\). We can have the \(z\) component of the spin to be \(\pm 1 /2 \), and positive and negative energy: in the end, these are 4 degrees of freedom. 

Note that we have eigenstates of \(\Sigma_3 \) specifically because of the way we chose \(\xi_{r}\): we can have other equivalent choices. 

\subsection{The helicity operator}

\begin{claim}
The spin projection along a given axis is not a good quantum number: it does not commute with the Hamiltonian, \(\qty[H , \Sigma_3 ] \neq 0\).
\end{claim}

\begin{proof}
What is the Dirac Hamiltonian? 
Recall that we started the whole discussion by requiring that the Dirac equation should have the form 
%
\begin{align}
i \partial_{t} \psi = H_D \psi 
\,,
\end{align}
%
and the ansatz we used was 
%
\begin{align}
H_D = \alpha \cdot \vec{p} + \beta M
\,,
\end{align}
%
and we found that for a massive particle we can write \(\alpha_{i}\) and \(\beta \) as 4-dimensional matrices, which in the Dirac representation are 
%
\begin{subequations}
\begin{align}
\alpha_{i} = \left[\begin{array}{cc}
0 & \sigma_{i} \\ 
\sigma_{i} & 0
\end{array}\right]
\qquad \text{and} \qquad
\beta = \left[\begin{array}{cc}
\mathbb{1} & 0 \\ 
0 & -\mathbb{1}
\end{array}\right]
\,.
\end{align}
\end{subequations}

So, the Hamiltonian is a 4-dimensional matrix, which looks like 
%
\begin{subequations}
\begin{align}
H_D = \left[\begin{array}{cc}
M & \vec{\sigma} \cdot \vec{p}\\ 
 \vec{\sigma} \cdot \vec{p} & -M
\end{array}\right]
\,.
\end{align}
\end{subequations}

So, let us compute the commutator: we get 
%
\begin{align}
\qty[\vec{\alpha} \cdot \vec{p} + \beta M, \Sigma_3 ]
= \qty[\vec{\alpha}, \Sigma_{3}] \cdot \vec{p} + M \qty[\beta, \Sigma_3 ]
\,.
\end{align}

We can just compute the two matrix commutators, since the momentum and the mass are fixed. We find, for each component:
%
\begin{subequations}
\begin{align}
\qty[\alpha_{i}, \Sigma_{3}] = 
\frac{1}{2}
\left[\begin{array}{cc}
0 & \sigma_{i} \\ 
\sigma_{i} & 0
\end{array}\right]
\left[\begin{array}{cc}
\sigma_3  & 0 \\ 
0 & \sigma_3 
\end{array}\right] 
- \frac{1}{2} 
\left[\begin{array}{cc}
\sigma_3  & 0 \\ 
0 & \sigma_3 
\end{array}\right]
\left[\begin{array}{cc}
0 & \sigma_{i} \\ 
\sigma_{i} & 0
\end{array}\right]
=  \frac{1}{2}
\left[\begin{array}{cc}
0 & \qty[\sigma_3, \sigma_{i}] \\ 
\qty[\sigma_3, \sigma_{i}] & 0
\end{array}\right]
\,,
\end{align}
\end{subequations}
%
and we know that this is given by\footnote{A more elegent way to deal with these computations is to intrpret them in terms of tensor products of Pauli matrices: we have, for instance, \(\alpha_{i} = \sigma_{x} \otimes \sigma_{i}\), and there are explicit formulas for products in the form \((\sigma_{a } \otimes \sigma_{b})(\sigma_{c } \otimes \sigma_{d}) \) \cite[]{agartthaCommutatorsTensorProduct2018}.}
%
\begin{align}
\qty[\sigma_{3}, \sigma_{i}] = 2i \epsilon_{3ik} \sigma_{k} 
\,,
\end{align}
%
so in the full expression we find 
%
\begin{align}(\sigma_{a } \otimes \sigma_{b})
\qty[\alpha_{i}, \Sigma_3]
=
i \epsilon_{3ik} \alpha_{k}
\,,
\end{align}
%
while \(\beta \) and \(\Sigma_3 \) are both diagonal, and therefore they commute. So, we can finally say that 
%
\begin{align} \label{eq:spin-conservation-dirac}
\qty[H_D, \Sigma_3 ] = i \epsilon_{3jk} p_{j} \alpha_{k} 
\,,
\end{align}
%
which is not zero \emph{unless} \(\vec{p}\) is oriented along the \(z\) axis.
This, however, is not a covariant condition. 
\end{proof}

So, we define the \textbf{helicity} as the projection of the spin along the direction of motion: 
%
\begin{align}
\Sigma_{p} = \frac{\vec{\Sigma} \cdot \vec{p}}{\abs{p}} = \vec{\Sigma} \cdot \hat{p}
\,.
\end{align}

The helicity is a good quantum number. It commutes with the Dirac Hamiltonian, whereas \(\Sigma_3 \) does not.
So, the eigenvalue of \(\Sigma_{p}\) is conserved: this makes it a good quantum number, since it provides a reliable description of the state.

From the result we found before we can see that what we are basically doing here is calculating \(\Sigma_{p}\) along the momentum, which is equivalent (after a rotation of the axes) to setting \(\vec{p} = \abs{p} \hat{z}\).
Then, from equation \eqref{eq:spin-conservation-dirac} we can see that \(\Sigma_{p}\) will commute with the Hamiltonian.

\todo[inline]{This is much simpler than the solution proposed in the professor's notes, but just as valid, I think: after all, we can choose the coordinates in which we perform the calculation as we wish since in the end we get a covariant result.}

\subsection{Pauli-Lubanski vector and helicity}

We define the following pseudovector:\footnote{Pseudovector means that it is even under parity transformations \(P : \vec{x} \to - \vec{x}\).

More formally, we can say that it transforms under spatial rotations like a vector density: if \(R \in O(3)\), a regular vector transforms like \(v' = R v\), but a pseudovector transforms like \(w' = (\det R) Rw \). So, while a vector is flipped by a parity: \(Pv = - v\) a pseudovector is not: \(P w = w\).

We can obtain pseudovectors by taking the cross product of regular vectors. The magnetic field \(\vec{B}\) is an example of a pseudovector.
}
%
\begin{align}
\omega^{\mu } = \frac{1}{2} \epsilon^{\mu \nu \rho \sigma } J_{\nu \rho } p_{\sigma }
\,,
\end{align}
%
where \(J_{\nu \rho }\) is the total angular momentum tensor, which includes both an angular and spin component: 
%
\begin{align}
J_{\nu \rho } = L_{\nu \rho } + \Sigma_{\nu \rho } 
\qquad \text{where} \qquad
L_{\nu \rho } = 2 x_{[\nu } p_{\rho  ]}
\,.
\end{align}

There are two terms in the expression for \(\omega^{\mu }\), but the angular momentum one vanishes: we have 
%
\begin{align}
\frac{1}{2} \epsilon^{\mu \nu \rho \sigma }L_{\nu \rho } p_{\sigma }
= 
\frac{1}{2} \epsilon^{\mu \nu \rho \sigma } \qty(x_{\nu }p_{\rho } - x_{\rho } p_{\nu }) p_{\sigma } =0
\,,
\end{align}
%
since in both cases we have the contraction of two copies of the momentum, which are symmetric in their indices, with the antisymmetric tensor. 

Now, let us move to the rest frame of the particle, so that \(p^{\mu } = (M, \vec{0})\).
Then, we will have 
%
\begin{align}
\omega^{\mu } = \frac{M}{2} \epsilon^{\mu \nu \rho 0} \Sigma_{\nu \rho }
\,.
\end{align}

Therefore, this vector is purely spatial since setting \(\mu =0\) means the RHS vanishes: so, \(\omega^{0} = 0\). On the other hand, we have 
%
\begin{align}
\omega^{i} = - \frac{M}{2} \epsilon^{0i \nu \rho } \Sigma_{\nu \rho } = - \frac{M}{2} \epsilon^{i j k } \Sigma_{j k } = - M \Sigma^{i}
\,,
\end{align}
%
where we applied the definition of the spin generator \(\Sigma^{i}\) \eqref{eq:rotation-spin-generator-definition}, and we moved from the four-dimensional Kronecker symbol to the three-dimensional one: we are allowed to move to three-vector indices since the case in which they are equal to zero is excluded by the first index being zero.

\todo[inline]{There is an extra minus sign due to the odd permutation of indices in the Levi-Civita symbol (thanks Andrea!): this does not change the observable \(\abs{\Sigma }^2\), but it is reported incorrectly in the professor's notes. See \cite[eqs 9.121 to 9.124]{dauriaSpecialRelativityFeynman2011}.}

So, we can take the square modulus of \(\omega^{\mu }\) in the rest frame, and normalize it by \(M^2\). Since this is a 4-vector this will be a covariant quantity: 
%
\begin{subequations}
\begin{align}
\eval{\frac{\omega_{\mu } \omega^{\mu }}{M^2}}_{\text{rest frame}}
= \Sigma^{i} \Sigma_{i} = - \Sigma^{i} \Sigma^{i} = -\frac{1}{4} \left[\begin{array}{cc}
\sigma_{i} \sigma_{i} & 0 \\ 
0 & \sigma_{i } \sigma_{i}
\end{array}\right]
= - \frac{3}{4} \mathbb{1}
\,. \marginnote{Factor \(3\) since there is a sum over \(i\).}
\end{align}
\end{subequations}

This is consistent with the expression we know from quantum mechanics, that is, the eigenvalue of \(s^2\) being \(s (s+1)\) with \(s = 1/2\).
The negative sign has no particular physical meaning: it is due to the fact that our metric signature gives spacelike vectors negative norms. 

So, the PL vector gives us a way to define the spin of a particle in a relativistic manner. 

We can use a unit vector \(n^{\mu }\) which is orthogonal to the momentum of the particle, that is, such that \(n^{\mu } p_{\mu }=0\) and \(n^{\mu } n_{\mu }=-1\), to define a spin in a generic direction: 
%
\begin{align}
\eval{\frac{\omega^{\mu }n_{\mu }}{M}}_{\text{rest frame}} = - \vec{\Sigma} \cdot \vec{n}
\,.
\end{align}

We can also define the helicity in this frame work, using the vector 
%
\begin{subequations}
\begin{align}
n^{\mu }_{p} = \frac{1}{M} \left[\begin{array}{c}
\abs{p} \\ 
\omega_{p} \hat{p}
\end{array}\right]
\,,
\end{align}
\end{subequations}
%
which has square modulus \((\abs{p}^2 - \omega_{p}^2) / M^2 = -1\), and is orthogonal to the momentum: their product is \(\propto \abs{p} \omega_{p}  - \omega_{p} \abs{p} = 0\).

For some geometrical intuition: if we trace out a 2D spacetime diagram with the directions of time and \(\hat{p}\), the momentum will be some timelike vector, while this \(n^{\mu }_{p}\) will be its reflection with respect to the light-like axis.

With this, we recover 
%
\begin{align}
\eval{\frac{\omega^{\mu }n_{\mu , p}}{M}}_{\text{rest frame}}
= - \Sigma_{p} = - \vec{\Sigma} \cdot \hat{p}
\,.
\end{align}
%
\subsection{Chirality}

We can define another matrix beyond the \(\gamma^{\mu }\): it is 
%
\begin{align} \label{eq:gamma5-definition}
\gamma_{5} = - \frac{i}{4} \epsilon_{\mu \nu \rho \sigma } \gamma^{\mu } \gamma^{\nu } \gamma^{\rho } \gamma^{\sigma } = + i \gamma^{0} \gamma^{1} \gamma^{2} \gamma^{3}
\,.
\end{align}

Note the sign convention: we choose \(\epsilon^{0123} = +1 \), which means that we get \(\epsilon_{0123} = -1\), since we must lower an odd amount of spacelike indices.

\begin{claim}
The \(\gamma^{5}\) matrix is self adjoint: 
%
\begin{align}
\gamma_{5} = \qty(\gamma_{5})^\dag
\,,
\end{align}
%
it squares to the identity: 
%
\begin{align}
\qty(\gamma_5 )^2 = \mathbb{1}
\,,
\end{align}
%
and it anticommutes with the other gamma matrices: 
%
\begin{align}
\qty{\gamma^{\mu } , \gamma_5 } = 0
\,.
\end{align}
\end{claim}

\begin{proof}
We will need to move a few matrices around, so first let us establish that, since 
%
\begin{align}
\frac{1}{2} \qty{\gamma^{\mu }, \gamma^{\nu }} = \eta^{\mu \nu }
\,,
\end{align}
%
we have that the square of \(\gamma^{0} \) is \(\mathbb{1}\), while the square of \(\gamma^{i} \) is \(- \mathbb{1}\). 

On the other hand, we can switch the places of two different \(\gamma \) matrices as long as we switch the global sign. So, first of all let us show that the square of the \(\gamma_5 \) matrix is the identity: we write a number in place of the \(\gamma \), for clarity.  
%
\begin{subequations}
\begin{align}
\qty(\gamma_{5})^2 &= i^2 01230123 = - 01230123  \\
&= (-)^{4} 00123123 = (-)^{6} 00112323 = (-)^{7} 00112233  \\
&= - \mathbb{1} (-\mathbb{1})^3 = \mathbb{1}
\,.
\end{align}
\end{subequations}

For the self-adjointness, we have 
%
\begin{subequations}
\begin{align}
\gamma_5 ^\dag &= + \frac{i}{4} \epsilon_{\mu \nu \rho \sigma } 
\gamma^{\sigma \dag} \gamma^{\rho \dag} \gamma^{\nu \dag} \gamma^{\mu \dag} \\
&= \frac{i}{4} \epsilon_{\mu \nu \rho \sigma } \gamma^{\sigma } \gamma^{\rho } \gamma^{\nu } \gamma^{\mu }  \\
&= \frac{i}{4} \epsilon_{\mu \nu \rho \sigma }
\gamma^{\mu } \gamma^{\nu } \gamma^{\rho } \gamma^{\sigma }
\,.
\end{align}
\end{subequations}

Here we used the fact that the Dirac matrices are self-adjoint, and that we can get from the configuration \(\mu \nu \rho \sigma \) to \(\sigma \rho \nu \mu \) in an even number of ``hops''. At each hop we can exchange the matrices by changing sign, like before, so there is no issue. The matrices we are exchanging are surely different since they are multiplied by the Kronecker symbol.

For the anticommutation: say we have the expression \(\gamma^{\mu } \gamma_5 \) for some \(\mu \) and we want to bring the \(\gamma^{\mu } \) to the other side. We can commute it with the Kronecker symbol and itself, and we need to anticommute it with three matrices different from itself. This gives us three minus signs, so the result has an opposite sign to what we started with: \(\gamma^{\mu } \gamma_5 = - \gamma_5 \gamma^{\mu }\), what we wanted to prove. 
\end{proof}

In the Dirac representation, 
%
\begin{subequations}
\begin{align}
\gamma_5 = \left[\begin{array}{cc}
0 & \mathbb{1} \\ 
\mathbb{1} & 0
\end{array}\right] 
\,.
\end{align}
\end{subequations}
%

\begin{claim}
The \(\gamma_{5}\) matrix commutes with the spin tensor \(\Sigma^{\mu \nu }\).
\end{claim}

\begin{proof}
Recall that \(\Sigma^{\mu \nu} \propto \qty[\gamma^{\mu}, \gamma^{\nu }]\). Then, as before writing the indices without the \(\gamma \)s: 
%
\begin{subequations}
\begin{align}
\qty[\gamma_5, \Sigma^{\mu \nu }] &\propto 5 \mu \nu - 5 \nu \mu - \mu \nu 5 + \nu \mu 5   \\
&= -\mu 5 \nu + \nu 5 \mu + \mu 5 \nu - \nu 5 \mu  =0
\,,
\end{align}
\end{subequations}
%
where we anticommuted \(\gamma_5 \) with \(\gamma^{\mu }\) and \(\gamma^{\nu }\).
\end{proof}

The fact that these commute means that they can be simultaneously diagonalized.

This allows us to classify the eigenvectors by the eigenvalue f \(\gamma_5 \) instead of the spin.

The \(N=4\) spinorial representation is reducible: it is the sum of 2 2-dimensional representations.

The \(\gamma_5 \) matrix is called the \textbf{chirality} operator, using it we can define the projectors 
%
\begin{align} \label{eq:chirality-projection-operators}
P_{R, L} = \frac{\mathbb{1} \pm \gamma_5 }{2}
\,,
\end{align}
%
where the minus corresponds to the left projector, while the plus corresponds to the right.

\begin{claim}
These are indeed a complete set of orthogonal projectors: they are idempotent, orthogonal and they sum to the identity.

Also, they are self adjoint.
\end{claim}

\begin{proof}
First, let us show idempotence; we denote either of \(L\) or \(R\) by \(x\).

\begin{align}
P_x^2 = \frac{1}{4} \qty(1 \pm 2 \gamma_5  + \gamma_5^2)
= \frac{1}{4} \qty(2 \mathbb{1} \pm 2 \gamma_5 ) = P_x
\,.
\end{align}

Then, to see orthogonality: 
%
\begin{align}
P_L P_R = \frac{1}{4} \qty(\mathbb{1} - \gamma_5 ) \qty(\mathbb{1} + \gamma_5 ) = \frac{1}{4} \qty(\mathbb{1}^2 - \gamma_5^2) = 0
\,.
\end{align}

Finally, they sum to the identity: 
%
\begin{align}
P_L + P_R = \frac{1}{2} \qty(\mathbb{1} - \gamma_5 ) + \frac{1}{2} \qty(1 + \gamma_5) = \mathbb{1}
\,.
\end{align}

They are self-adjoint because \(\mathbb{1}\) and \(\gamma_5 \) are.
\end{proof}

We can then project any Dirac spinor into the left or right chiral subspaces by applying these matrices. 

The projections onto the right-chiral subspace will have a \(+1 \) eigenvalue for \(\gamma_5 \), the projections onto the left-chiral subspace will have eigenvalue \(-1\).

These chiral spinors are the irreducible \(N=2\) representations which constitute the full spinor.

For the massless case we could also have used a \(N=2\) representation for the full space, the Weyl spinor. This would have behaved like a left or right chiral spinor.

Finally, we can define the \textbf{conjugate chiral spinors}: 
%
\begin{subequations}
\begin{align}
\overline{\psi}_{L} &= \qty(\psi_{L} ^\dag) \gamma^{0} 
= \psi ^\dag P_L \gamma^{0} = \overline{\psi} P_R \\
\overline{\psi}_{R} &= \qty(\psi_{R} ^\dag) \gamma^{0} 
= \psi ^\dag P_R \gamma^{0} = \overline{\psi} P_L 
\,,
\end{align}
\end{subequations}
%
where we used the fact that 
%
\begin{align}
\gamma^{0} \qty(1 \pm \gamma_5 ) 
= \qty(1 \mp \gamma_5 ) \gamma^{0}
\,,
\end{align}
%
since the identity commutes with \(\gamma^{0}\), while \(\gamma^{5}\) anticommutes with it.
\end{document}
