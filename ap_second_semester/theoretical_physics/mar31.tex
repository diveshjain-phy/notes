\documentclass[main.tex]{subfiles}
\begin{document}

\marginpar{Wednesday\\ 2020-4-29, \\ compiled \\ \today}

So, we can write out the \textbf{general solution} for the Dirac equation: the spinor and its conjugate are
%
\begin{subequations}
\begin{align}
\psi (x) &=
\frac{1}{(2 \pi )^{3/2}}
\int \frac{ \dd[3]{k}}{\sqrt{2 \omega_{k}}}
\sum _{r=1}^{2} \qty[
    c_r(k) u_r(k) e^{-ikx}
    d^{*}_{r} (k) v_r(k) e^{ikx}
]_{\omega_{k} = k^{0}}
\\
\overline{\psi} (x) &=
\frac{1}{(2 \pi )^{3/2}}
\int \frac{ \dd[3]{k}}{\sqrt{2 \omega_{k}}}
\sum _{r=1}^{2} \qty[
    d_r(k) \overline{v}_r(k) e^{-ikx}
    c^{*}_{r} (k) \overline{u}_r(k) e^{ikx}
]_{\omega_{k} = k^{0}}
\,.
\end{align}
\end{subequations}

The functions \(c_r\) and \(d_r\) represent the two degrees of freedom of the positive and negative energy solution respectively.

The important thing to recall is 
%
\begin{align}
\psi_{+} \sim cu \qquad \text{and} \qquad
\psi_{-} \sim d ^* v
\,.
\end{align}

Note that \(c_r\) and \(d_r\) are scalar functions, while \(u_r\) and \(v_r\) are 4D spinors.

\subsection{Energy projectors}

We wish to distinguish positive and negative energy solutions: so, we define the projector onto the positive and negative energy subspaces, 
%
\begin{align}
\Lambda_{\pm } (k) = \frac{\pm \slashed{k} + M}{2M}
\,.
\end{align}

These two are an incomplete set of orthogonal projectors.

\begin{claim}
They satisfy 
%
\begin{subequations}
\begin{align}
\Lambda_{\pm}^2 (k) &= \Lambda_{\pm} (k)  \\
\Lambda_{+} (k) + \Lambda_{-} (k) &= \mathbb{1}  \\
\Lambda_{+} (k) \Lambda_{-} (k) &= 0  \\
\Tr[\Lambda_{\pm} (k)] &= 2
\,.
\end{align}
\end{subequations}
\end{claim}

\begin{proof}
Let us first establish what \(\slashed{k}^2\) is equal to: 
%
\begin{align}
\slashed{k}^2 =
\gamma^{\mu } k_{\mu } \gamma^{\nu } k_{\nu }
= \frac{1}{2} \qty{\gamma^{\mu }, \gamma^{\nu }}k_{\mu } k_{\nu }
= \eta^{\mu \nu } k_{\mu } k_{\nu }
= k^2 \mathbb{1}_{4}
\,.
\end{align}

So, we can compute 
%
\begin{align}
\Lambda_{\pm}^2 (k) &= \frac{1}{4 M^2}\qty[\slashed{k}^2 + M^2 \mathbb{1} \pm 2 \slashed{k} M]  \\
&= \frac{1}{4 M^2} \qty[2 M \mathbb{1} \pm 2 \slashed{k}M] = \frac{\pm \slashed{k} + M}{2M}
\,.
\end{align}

For the second property, we get 
%
\begin{align}
\Lambda_{+} + \Lambda_{-} = \frac{+ \slashed{k} + M - \slashed{k} + M}{2M} = \frac{2M}{2M} = \mathbb{1} 
\,.
\end{align}

For the third expression, we find 
%
\begin{align}
\Lambda_{+} \Lambda_{-} \propto \qty(\slashed{k} + M) \qty(\slashed{k} - M) = \slashed{k}^2 - M^2 = 0
\,.
\end{align}

For the fourth expression, instead, we find that since \(\slashed{k}\) is traceless the trace is equal to that of \((M / 2 M)  \mathbb{1}\), which is equal to \(2\) since spinorial matrices are 4-dimensional.
\end{proof}

We can apply these projectors to the solutions we know how to write from the general expression \eqref{eq:dirac-equation-solutions}: 
%
\begin{align}
\Lambda_{+} u_r (k) &= \frac{C}{2M} (\slashed{k} + M)^2 u_r(M) = C(\slashed{k} + M) u_r (M) = u_r(k) \\
\Lambda_{-} u_r (k) &= \frac{C}{2M} (-\slashed{k} + M) (\slashed{k} + M) u_r(M) = 0 \\
\Lambda_{+} v_r (k) &= \frac{C}{2M} (\slashed{k} + M) (\slashed{k} - M) u_r(M) = 0 \\
\Lambda_{-} v_r (k) &= \frac{C}{2M} (-\slashed{k} + M) (-\slashed{k} + M) u_r(M) = v_r(k)
\,.
\end{align}

These projectors can also be obtained from the vectors themselves: if a vector \(v\) is normalized to 1, the matrix \(v v ^{\top}\) is a projector onto the subspace of the vector.

We can add projectors together in order to get projectors onto larger subspaces.

\begin{claim}
We can recover the projectors \(\Lambda_{\pm }\) by the expressions: 
%
\begin{align}
\Lambda_{+} (k) &= \sum_{r} \frac{u_r(k) \overline{u}_{r} (k)}{2M} \\
\Lambda_{-} (k) &= - \sum_{r} \frac{v_r(k) \overline{v}_{r} (k)}{2M} 
\,.
\end{align}
\end{claim}

\begin{proof}
The calculation yields 
%
\begin{align}
u_r(k) \overline{u}_{r}(k) &=
\left[\begin{array}{c}
\xi_{r}\sqrt{\omega_{k} + M} \\ 
\displaystyle
\frac{\vec{k} \cdot \vec{\sigma}}{\sqrt{\omega_{k} + M}} 
\xi_{r}
\end{array}\right]
\left[\begin{array}{cc}
\xi_{r}^{\top} \sqrt{\omega_{k} + M}, & 
\displaystyle
- \xi_{r}^{\top} \frac{\vec{k} \cdot \vec{\sigma}}{\sqrt{\omega_{k} + M}}
\end{array}\right]  \\
&= \left[\begin{array}{cc}
\xi_{r} \xi_{r}^{\top} (\omega_{k} + M) & - \xi_{r} \xi_{r}^{\top} (\vec{k} \cdot \vec{\sigma}) \\ 
(\vec{k} \cdot \vec{\sigma}) \xi_{r} \xi_{r}^{\top} & 
- \frac{ (\vec{k}\cdot \vec{\sigma}) \xi_{r} \xi_{r}^{\top} (\vec{k} \cdot \vec{\sigma})}{\omega_{k} + M}
\end{array}\right]
\,,
\end{align}
%
and now we notice the fact that \(\sum _{r} \xi_{r} \xi_{r} ^{\top} = \mathbb{1}_{2}\), so if we sum over \(r\) we get 
%
\begin{align}
\sum _{r} u_r (k) \overline{u}_{r}(k) &=
\left[\begin{array}{cc}
\omega_{k} + M & - \vec{k} \cdot \vec{\sigma} \\ 
\vec{k} \cdot \vec{\sigma} & - \frac{\abs{k}^2}{\omega_{k} + M}
\end{array}\right]  \\
&= \left[\begin{array}{cc}
\omega_{k} + M & - \vec{k} \cdot \vec{\sigma} \\ 
\vec{k} \cdot \vec{\sigma} & - \omega_{k} + M
\end{array}\right]  \\
&= \slashed{k} + M
\,.
\end{align}

If we divide by \(2M\) we get our result. 
The computation for \(v_{r}\) is similar. 
\end{proof}

\subsection{Spin operators}

Recall, the spinorial representation of the Lorentz group is generated by the matrices \(\Sigma^{\mu \nu }\), which can be calculated by 
%
\begin{align}
\Sigma^{\mu \nu } = \frac{i}{4} \qty[\gamma^{\mu }, \gamma^{\nu }]
\,.
\end{align}

We can use these to define, in the Dirac representation, the generators of boosts: 
%
\begin{align}
K_{i} = \Sigma^{i0} = -\frac{i}{2} \left[\begin{array}{cc}
0 & \sigma_{i} \\ 
\sigma_{i} & 0
\end{array}\right]
\,,
\end{align}
%
and of rotations: 
%
\begin{align}
\Sigma_{i} = \frac{1}{2} \epsilon_{ijk} \Sigma^{jk} 
= \frac{1}{2} \left[\begin{array}{cc}
\sigma_{i} & 0 \\ 
0 & \sigma_{i}
\end{array}\right]
\,.
\end{align}

We can exponentiate these to get the representations of general boosts and rotations: a boost will be written as 
%
\begin{align}
S(\Lambda) = \exp( - \frac{i}{2} \omega_{0i} K_{i} )
\,,
\end{align}
%
while a rotation will be written as 
%
\begin{align}
S(\Lambda ) = \exp( - \frac{i}{2} \theta_{k} \Sigma_{k})
\,,
\end{align}
%
where we used the angle \(\theta_{k} = \epsilon_{ijk} \omega^{ij}\).



\end{document}
