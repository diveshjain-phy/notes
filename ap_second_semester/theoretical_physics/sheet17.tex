\documentclass[main.tex]{subfiles}
\begin{document}

\chapter{Beyond QED}

\section{Decay of a heavy vector boson}

\marginpar{Friday\\ 2020-6-19, \\ compiled \\ \today}

We consider a \textbf{massive real vector field}. The generalization to a complex one is simple. 
The Lagrangian, describing the massive vector \(V^{\mu }\) and the fermion \(\psi \), is: 
%
\begin{align}
\mathscr{L} &= - \frac{1}{4} V^{\mu \nu } V_{\mu \nu } 
+ \frac{1}{2} M^2 V^{\mu } V_{\mu } 
+ \overline{\psi} \qty(i \slashed{\partial} - m) \psi 
+ \mathscr{L} _{\text{int}} \\ 
\mathscr{L} _{\text{int}} &= 
\frac{g}{c_{w}} \overline{\psi} \qty(
    c_L \gamma^{\mu } p_L +
    c_R \gamma^{\mu } p_R 
) \psi V_{\mu }
\,,
\end{align}
%
where as usual the field strength is defined as \(V^{\mu \nu } = 2 \partial^{[\mu } V^{\nu ]}\), while \(\theta_{w}\) is the Weinberg angle, whose cosine is denoted \(c_w = \cos \theta_{w} \approx \num{.88}\). 

It is experimentally measured that \(\sin^2\theta \approx \num{.22}\). 
On the other hand, the coupling \(g\) is \(g = e / \sin \theta \approx \num{.64} \) in natural units. 

The couplings \(c_{L, R}\) depend on the specific particle considered. 

The terms \(P_L\) and \(P_R\) are the projection operators defined in \eqref{eq:chirality-projection-operators}: they project onto the left- or right-handed subspaces. 

\subsection{Massive vector diagrams}

If we have a free field term for the massive vector in which it is incoming we get a factor \(\epsilon^{\mu }_{\lambda } (p)\), if the massive vector is outgoing we have its conjugate, \(\epsilon^{\mu *}_{\lambda } (p)\). 

The propagator (in momentum space) is given by 
%
\begin{align}
\widetilde{D}^{\mu \nu }_F (q) &=- 
\frac{i}{q^2 - M^2} 
\qty(
    \eta^{\mu \nu } 
    - \frac{p^{\mu }p^{\nu }}{M^2}
)
\,,
\end{align}
%
while for a vertex in which the massive vector interacts with a fermion with \(\mathscr{L} _{\text{int}}\) we get a term: 
%
\begin{align}
-i \frac{g}{c_w} \qty(c_L \gamma^{\mu }_{L} + c_R \gamma^{\mu }_{R})
\,,
\end{align}
%
where the matrices \(\gamma^{\mu }_{L, R}\) are defined as \(\gamma^{\mu} P_{L, R}\). 

A useful relation is this pseudo-completeness one:
%
\begin{align}
\sum _{\lambda = 1}^{3} \epsilon^{\mu }_{\lambda }(p) \epsilon^{\nu }_{\lambda } (p) &= - \eta^{\mu \nu } + \frac{p^{\mu } p^{\nu }}{M^2}
\,.
\end{align}

Notice the fact that we have three independent polarizations now, as opposed to the massless vector which has two; the polarizations of the massive vector span the whole spacelike section of spacetime orthogonal to \(p^{\mu }\), whereas the polarizations of the massless vector only span the two-dimensional region which is spacelike \emph{and} orthogonal to the 3D wavevector \(\vec{k}\). 
This is also relevant when averaging over polarizations: we will need to divide by \(3\). 

Now, since we have the Feynman rules we can quickly calculate the Feynman amplitude for the decay \(V \to \overline{f} f\), which now is permitted: 
%
\begin{align}
\mathcal{M}_{V \to \overline{f}f}
= -i \frac{g}{c_w} \overline{u}_{r} (k)  
\qty(c_L \gamma^{\mu }_{L} + c_R \gamma^{\mu }_{R})
u_{r'} (k') \epsilon_{\mu }^{\lambda} (p)
\,,
\end{align}
%
and from it we can sum and average to get the unpolarized square amplitude: 
%
\begin{align}
\abs{\mathcal{\overline{M}}}^2 &=
\frac{g^2}{c_w^2} \frac{1}{3} 
\sum _{\lambda } \epsilon_{\mu }^{\lambda }(p) \epsilon_{\nu }^{\lambda *}(p) 
\sum _{r r'} (\overline{u}_r (\dots)^{\mu } v_{r'}  )
( \overline{v}_{r'} (\dots)^{\nu } u_r )  \\
&= \frac{1}{3} \frac{g^2}{c_w^2} 
\qty(- \eta_{\mu\nu } \frac{p_{\mu }p_{\nu}}{M^2})
\Tr[(\slashed{k} - m)\qty(c_L \gamma^{\mu }_{L} + c_R \gamma^{\mu }_{R})
 (\slashed{k}' + m)\qty(c_L \gamma^{\nu }_{L} + c_R \gamma^{\nu }_{R})]
\,.
\end{align}

\begin{claim}
The trace can be explicitly calculated to give 
%
\begin{align}
\begin{split}
&\Tr[(\slashed{k} - m)\qty(c_L \gamma^{\mu }_{L} + c_R \gamma^{\mu }_{R})
(\slashed{k}' + m)\qty(c_L \gamma^{\nu }_{L} + c_R \gamma^{\nu }_{R})] = \\
&\phantom{=}\ = 2 (c_L^2 + c_R^2) \qty(k^{\prime \nu } k^{\mu } - \eta^{\mu \nu } \qty(k \cdot k') + k^{\prime \mu } k^{\nu })
- 4 m^2 c_L c_R \eta^{\mu \nu }
\end{split}
\,.
\end{align}
\end{claim}

\begin{proof}
\todo[inline]{To do.}
\end{proof}

Using this result, we find 
%
\begin{align}
\abs{\mathcal{\overline{M}}}^2 &= 
\frac{1}{3} \frac{g^2}{c_w^2} \qty(2 (c_L^2 + c_R^2)
\qty(k \cdot k' + 2 \frac{(p \cdot k) (p \cdot k')}{M^2}) 
+ 12 m^2 c_L c_R
)
\,.
\end{align}

Now, we can specify the kinematics to the rest frame of the massive vector. Then, we can see that the ``Mandelstam variables'' are given by \(s = p^2 = M^2\), \(t = u = (p-k)^2 = (k')^2 = m^2\), and that 
%
\begin{align}
k \cdot k' = \frac{M^2}{2} - m^2
&&
p \cdot k = p \cdot k' = \frac{M^2}{2}
\,.
\end{align}

So the amplitude reads: 
%
\begin{align}
\abs{\mathcal{\overline{M}}}^2 &= 
\frac{1}{3} \frac{g^2}{c_w^2} \qty(2 (c_L^2 + c_R^2)
\qty( \frac{M^2}{2} - m^2 + 2 \frac{M^2}{4}) + 12 m^2 c_L c_R )  \\
&= \frac{g^2}{c_w^2} \frac{M^2}{3} 
\qty(2 (c_L^2 + c_R^2) \qty(1 - \frac{m^2}{M^2}) + 12 c_L c_R \frac{m^2}{M^2})
\,.
\end{align}

If the vector is very massive compared to the fermion (\(m \ll M\)) then this simplifies heavily: 
%
\begin{align}
\abs{\mathcal{\overline{M}}}^2
\overset{m \ll M}{\approx}
\frac{g^2}{c_w^2} \frac{M^2}{3} 2 (c_L^2 + c_R^2)
\,,
\end{align}
%
however we will not make this assumption.

We can apply the general formula for the \textbf{differential decay rate} (in the lab frame): 
%
\begin{align}
\eval{\dv{\overline{\Gamma} }{\Omega }}_{\text{lab}}
= \frac{1}{64 \pi^2} \frac{1}{M} 
\sqrt{1- \frac{4m^2}{M^2}} \abs{\mathcal{\overline{M}}}^2_{\text{lab}}
\,,
\end{align}
%
but this does not depend on the angles (as it should: this is the decay of a particle in its frame, it better be isotropic!) we can integrate over the angles, which gives a factor \(4 \pi \), so we get 
%
\begin{align}
\eval{\overline{\Gamma}}_{\text{lab}}=
\frac{g^2}{c_w^2}
\frac{M}{48 \pi } \sqrt{1 - \frac{4m^2}{M^2}}
\qty(2 (c_L^2 + c_R^2) \qty(1 - \frac{m^2}{M^2}) + 12 c_L c_R \frac{m^2}{M^2})
\,,
\end{align}
%
which, in the limit \(m \ll M\), becomes  
%
\begin{align}
\eval{\overline{\Gamma}}_{\text{lab}}
\overset{m \ll M}{\approx}
\frac{g^2}{c_w^2}
\frac{M}{24 \pi }
(c_L^2 + c_R^2)
\,.
\end{align}

For example, in the decay \(Z \to \nu \overline{\nu}\) we have \(c_L = 1/2\), \(c_R = 0\) so the result is 
%
\begin{align}
\overline{\Gamma}_{\text{lab}}
= \frac{g^2}{c_w^2} \frac{M}{96 \pi }
\,.
\end{align}

This is known as the \emph{invisible decay width} of the \(Z\) boson, since it is extremely difficult to detect the neutrino pair generated by the decay, so at a collider one would see a fraction of the \(Z\) bosons effectively disappear without a trace.

\section{Spontaneous Symmetry Breaking}

Consider the following Lagrangian for a real scalar field \(\varphi \): 
%
\begin{align}
\mathscr{L} &= \underbrace{\frac{1}{2} \qty( \partial_{\mu } \varphi ) \qty(\partial^{\mu } \varphi )}_{\text{kinetic term}}
\underbrace{- \frac{1}{2} m^2 \varphi^2 - \frac{\lambda}{4!} \varphi^{4} }_{V(\varphi )}
\,.
\end{align}

The potential \(V(\varphi )\) can also be written as 
%
\begin{align}
V(\varphi ) = \frac{\lambda}{4!} \qty(\varphi^2 + \mu^2)^2 + C
\,,
\end{align}
%
where \(\mu^2 = 6 m^2 / \lambda \) and \(C\) is an inessential constant (which does not affect the EOM). 

The Lagrangian has a discrete symmetry: \(\varphi \to -\varphi \). 

The Hamiltonian for the theory is 
%
\begin{align}
\mathscr{H} = \frac{1}{2} \qty(\partial_0 \varphi )^2 + \frac{1}{2} \qty(\nabla \varphi )^2 + V(\varphi )
\,,
\end{align}
%
and it is positive definite as long as \(\lambda > 0\), while \(\lambda = 0\) returns us to the free theory. The parameter \(\mu \), instead, is unconstrained.

We are interested in the study of the \textbf{ground state} \(\varphi_0 \) of the theory: the field configuration which has the minimum energy. 
What are the requirements we must ask in order for it to actually be the ground state? 

\begin{enumerate}
    \item It must be stationary: \(\partial_{\mu } \varphi_0 = 0\). This corresponds to the kinetic part of the Lagrangian vanishing, so that the kinetic energy of the configuration is zero. 
    \item The value of the constant \(\varphi_0 \) is determined by minimizing the potential \(V (\varphi_0 )\). 
\end{enumerate}

\todo[inline]{Why do we ask \(\partial_{\mu } \varphi =0\) and not just the time derivative? In the expression for the momentum it appears as though \(\partial_0 \varphi_0 =0\) would be enough to make everything vanish\dots}

Let us consider two separate cases. 

\subsubsection{Positive mass: \(\mu^2> 0\)} 

In this case the state \(\varphi_0  = 0\) is a stable minimum: we have \(V' (\varphi  = 0) = 0\) and \(V'' (\varphi = 0) > 0\). 

This is the same vacuum we have in the free theory. 

\subsubsection{Negative mass: \(\mu^2 <0\)}

Now we have \(V'(\varphi =0) = 0\) and \(V'' (\varphi=0) < 0\): the configuration \(\varphi_0 = 0\) is an \emph{unstable} maximum. 

Now, however, we have two new minima at \(\varphi = \pm \abs{\mu }^2\): these are stable.

Now, a perturbation of the ground state will look like 
%
\begin{align}
\varphi (x) = \sigma (x) + \abs{\mu }
\,,
\end{align}
%
where we have arbitrarily chosen the right vacuum: in order to perturb around the vacuum we lose the parity symmetry. 

Then, we can express the Lagrangian with respect to this fluctuation: 
%
\begin{align}
\mathscr{L} = \frac{1}{2} \qty(\partial_{\mu }\sigma ) \qty(\partial^{\mu }\sigma ) - \frac{1}{2} \qty(\frac{\lambda }{3} \abs{\mu }^2)\sigma^2 - \frac{\lambda}{3!} \abs{\mu } \sigma^3 + \frac{\lambda}{4!} \sigma^{4}
\,,
\end{align}
%
which appears to have no parity symmetry anymore! 
In fact the symmetry is still there, but it will have to be expressed in a complicated way in terms of the powers of \(\sigma \). 

\end{document}