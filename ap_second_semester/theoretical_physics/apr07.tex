\documentclass[main.tex]{subfiles}
\begin{document}

\chapter{Field theories}

\section{Lagrangian and Hamiltonian formalisms}

\marginpar{Saturday\\ 2020-5-2, \\ compiled \\ \today}

The Lagrangian and Hamiltonian formalism can aid in the description of systems with either a finite of infinite number of degree of freedom. 

\subsection{Classical system with finite DoF: Lagrangian formalism}

The usual example is a system of particles labelled by the index \(i\) with masses \(m_i\), positions \(q_i (t)\) and velocities \(\dot{q}_{i}(t)\).

If the forces acting on the particles are \textbf{conservative}, we can express them in terms of a potential, which we assume not to depend on the velocities nor on time: \(V(q_{i})\).
From now on when we write \(q\) or \(\dot{q}\) we will mean the full vector of the positions or velocities.

If this is the case, the motion of the particles is described by Newton's equation: 
%
\begin{align}
m_i \ddot{q}_{i} = - \pdv{V}{q_{i}}
\,.
\end{align}

This can alternatively be described in terms of a function called the Lagrangian: 
%
\begin{align}
L(q, \dot{q}, t)
\,,
\end{align}
%
which could depend on time, but we usually assume it to be independent of time explicitly, that is, \(\partial_{t} L = 0\) (although the \emph{total} derivative of the Lagrangian may be nonzero!).

The condition of the Lagrangian being explicitly time-independent is equivalent to the system of forces being conservative.

The Lagrangian can be a generic function of the positions and velocities of the particles, but in order to reproduce Newton's law we need it to be in the form 
%
\begin{align}
L = T-V = \frac{1}{2} m_i \dot{q}_i^2 - V(q_i)
\,,
\end{align}
%
where a sum over the particles is implied in the kinetic energy term.

Typically problems in Lagrangian mechanics are given by fixing the boundary conditions as the initial and final position, as opposed to writing the initial values of position and velocity. 

We want to find the physical trajectory the particle(s) will take corresponding to those initial and final conditions.\footnote{There can be issues arising from this approach: for instance, the process for finding trajectories may find many equivalent ones, for example in situations with some symmetry or nontrivial topology.
In the gravitational two-body problem we can ask what is the stationary action path needed in order to reach the antipodal point to the current one: there are infinite ones, corresponding to the choice of azimuthal angle of the initial velocity.
However, this is not really a problem, since the approach we are about to introduce --- Hamilton's principle of stationary action --- just provides a formulation which eventually yields the same differential equations as the initial value problem; physically the velocity as well as the position of the particles is well determined in the initial moment.}

This can be accomplished using Hamilton's principle of stationary action. We start by introducing the action: 
%
\begin{align}
S[q(t), t _{\text{in}}, t _{\text{fin}}] = \int_{t _{\text{in}}}^{t _{\text{fin}}} L(q(t), \dot{q}(t)) \dd{t}
\,,
\end{align}
%
which depends on the full \emph{path} \(q(t)\): the integral is computed by evaluating the Lagrangian along it.
Because of this the action \(S\) is called a \emph{functional}, since it is a function of a function. The Lagrangian, on the other hand, is  a regular function.

We then state the following: 
\begin{claim}
The path taken physically by the system satisfies the stationary action principle: 
%
\begin{align}
\delta S = 0
\,.
\end{align}

The variation is the functional derivative of \(S\), meant to be a variation in the infinite-dimensional space of possible curves.
\end{claim}

This is not a proven principle, but rather an axiom to be taken, analyzed and confronted with experiment. 

\todo[inline]{The notes by the professor state that the action must be minimized by the physical path, but this is not the case! The physical path can be a maximum of the action.}

Let us calculate what the variation of the action means physically. We start by considering two nearby paths, \(\gamma \) and \(\gamma '\), such that 
%
\begin{align}
\gamma ' = \gamma + \delta_0 \gamma 
\,,
\end{align}
%
so that the coordinates \(q\) at a time \(t\) are given by 
%
\begin{align}
q_{\gamma } (t) 
\qquad \text{and} \qquad
q_{\gamma '} (t) = q_{\gamma }(t) + \delta_0 q_{\gamma }(t)
\,,
\end{align}
%
and we impose that the variation of the path, \(\delta_0 q_{\gamma } (t)\), is zero at the initial and final time, so that the boundary conditions are satisfied: \(\delta_0 q_{\gamma }(t _{\text{in}}) = 0 = \delta_0 q_{\gamma }(t _{\text{fin}})\).

The reason for the subscript \(0\) for the variation \(\delta \) is that we consider \emph{synchronous} variations: the difference between the perturbed and unperturbed trajectory, at a \emph{fixed time}. 
Also, we take the synchronous variation of the action:
%
\begin{align}
\delta_0 S [q_{\gamma }] &= S [q_{\gamma } + \delta_0 q_{\gamma }] - S[q_{\gamma }]  \\
&= \int_{t _{\text{in}}}^{t _{\text{fin}}} \delta_0 L (q_{\gamma }, \dot{q}_{\gamma }) \dd{t}  \\
&= \int_{t _{\text{in}}}^{t _{\text{fin}}} \qty[ \pdv{L}{q} \cdot \delta_0 q + \pdv{L}{\dot{q}} \cdot \dot{q}] \dd{t}  \\
&= \int_{t _{\text{in}}}^{t _{\text{fin}}} \qty[\pdv{L}{q}  - \dv{}{t} \pdv{L}{\dot{q}}] \cdot \delta_0 q \dd{t}
\,,
\end{align}
%
since, as we are considering synchronous variations, \(\delta_0 \) and \(\partial_{t}\) commute; also we neglected boundary terms in the integration by parts as they are set to zero by the fact that the variation of the path,  \(\delta_0 q_{\gamma }\), is zero at the boundaries.

Since this must hold for any variation of the path, by the fundamental lemma of the calculus of variation the integrand must vanish: this yields the Euler-Lagrange equations 
%
\begin{align}
\pdv{L}{q} - \dv[]{}{t} \pdv{L}{\dot{q}} = 0
\,.
\end{align}

Note that there is \textbf{gauge freedom} in the choice of Lagrangian: an easy symmetry to see is the scaling one; the Lagrange equations of \(L\) and \(cL\) for \(c \in \mathbb{R}\) are the same. 
Also, the equations for \(L\) and \(L + \dv*{F}{t}\) are the same: the action is the integral in time of the Lagrangian, therefore adding a total derivative to it shifts the action by a constant \(\eval{\Delta F}_{\text{in}}^{\text{fin}}\), which vanishes when taking the variation.

\subsection{Classical system with finite DoF: Hamiltonian formalism}



\end{document}