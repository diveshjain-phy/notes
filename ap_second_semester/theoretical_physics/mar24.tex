\documentclass[main.tex]{subfiles}
\begin{document}

\subsection{Covariance of the Dirac Equation}

\marginpar{Sunday\\ 2020-3-29, \\ compiled \\ \today}

How do these new spinors transform under Lorentz transformations? 

We suppose that for a general Lorentz transformation \(\Lambda \) we will have something in the form 
%
\begin{align}
\psi^{\prime } (x') = S(\Lambda ) \psi (x)
\,,
\end{align}
%
where \(S(\Lambda )\) is a \(4 \times 4\) matrix, belonging to the \emph{spinorial representation of the Lorentz group}. 
We do this because if we were to simply impose \(\psi^{\prime }(x^{\prime }) = \psi (x)\) we get a contradiction, so the equation is not covariant. 

Now, let us see what we must require of the matrices \(S(\Lambda )\) so that the Dirac equation holds in the new frame as well as in the old. 
When we do the computation, recall that the \(\gamma^{\mu }\) matrices are \emph{not} a 4-vector, despite the position of the index: they need not be transformed using a \(\Lambda \) matrix; better put, they are a set of 4 Lorentz scalars

So, we must have 
%
\begin{subequations}
\begin{align}
\qty(i \slashed{\partial}' - M) \psi^{\prime } (x') &=
\qty(i \gamma^{\mu } \tensor{\Lambda }{_{\mu }^{\nu }} \partial_{\nu } -M) S(\Lambda ) \psi (x)  
\\
&= S(\Lambda ) \qty[i \tensor{\Lambda }{_{\mu }^{\nu }} S^{-1}(\Lambda ) 
\gamma^{\mu } S(\Lambda ) \partial_{\nu } - M ] \psi (x)  \\
&=S(\Lambda ) \qty[i \slashed{\partial} - M] \psi (x)
\,,
\end{align}
\end{subequations}
%
where we used the following:
\begin{enumerate}
  \item the matrix \(S(\Lambda )\) is constant with respect to the spatial coordinates, since the Lorentz transformation is fixed, so we can bring it outside of the derivative \(\partial_{\nu }\);
  \item the matrices \(\gamma^{\mu }  \) and \(S(\Lambda )\) do not commute \emph{a priori}, so if we wish to bring the equation in the form \(S(\Lambda ) \times \qty[\text{old-coordinates Dirac eq}]\) we cannot simply commute them, instead we multiply by \(\mathbb{1} = S S^{-1}\) on the left;
  \item we impose the condition 
  %
  \begin{align}
  \tensor{\Lambda }{_{\mu }^{\nu }} S^{-1}(\Lambda ) \gamma^{\mu } S(\Lambda ) \overset{!}{=} \gamma^{\nu }
  \,
  \end{align}
  %
  in order to find the expression we need in order for the equation to be covariant, since as long as the matrix \(S(\Lambda )\) is nondegenerate the equation \(S(\Lambda ) \qty[i \slashed{\partial} - M] \psi =0\) has the same solutions as the Dirac equation.
\end{enumerate}

If we multiply by an inverse Lorentz matrix on both sides we can bring this equation into the form 
% 
\begin{align} \label{eq:spinorial-representation-definition}
S^{-1}(\Lambda ) \gamma^{\mu } S(\Lambda ) = \tensor{\Lambda }{^{\mu }_{\nu }} \gamma^{\nu }
\,,
\end{align}
%
so we can say that the Dirac equation is covariant as long as we find some matrices \(S(\Lambda )\) satisfying these conditions. 
Do note that while this looks like a vector transformation law, the \(\gamma^{\mu }\) matrices do not transform under Lorentz transformations: what we are stating is that it is equivalent to apply a Lorentz matrix to them and to transform them as spinorial matrices. 

In a way, we are asking the transformation laws for spinors and vectors to be compatible. 

\subsubsection{Explicit realization of the spinorial representation}

In order to do this, we consider infinitesimal Lorentz transformations, 
%
\begin{align}
\tensor{\Lambda }{^{\mu }_{\nu }} = \tensor{ \delta }{^{\mu }_{\nu }} + \tensor{\omega}{^{\mu}_{\nu }}
\,,
\end{align}
%
where \(\omega_{\mu \nu } = \omega_{[\mu \nu ]}\) is an antisymmetric tensor.\footnote{The fact that \(\omega \) must be antisymmetric may be derived by imposing the condition \(\eta_{\mu \nu } = \tensor{\Lambda }{_{\mu }^{\alpha }} \tensor{\Lambda }{_{\nu }^{\beta }} \eta_{\alpha \beta }\), with the perturbed \(\Lambda \) we wrote above.}

A rank-2 antisymmetric tensor in 4 dimensions has 6 degrees of freedom: these physically correspond to three rotations and three boosts. 

A rather general ansatz looks like:
%
\begin{align}
S(\Lambda ) = \mathbb{1} - \frac{i}{2} \omega_{\mu \nu } \Sigma^{\mu \nu }
\,,
\end{align}
%
where \(\Sigma^{\mu \nu }\) is the set of the generators of the spinorial representation of the Lorentz group. For fixed \(\mu \) and \(\nu \), these are matrices in the spinor space: if we write all the indices explicitly, they look like \(\Sigma^{\mu \nu }_{\aleph \beth}\). 

This means that to each possible basis Lorentz transformation of spacetime (think boost or rotation) we are associating a \(4 \times 4\) basis spinor transformation matrix. 
This is what finding a representation of the group means: for each element of the Lorentz group we are finding a transformation matrix. 

Do note, however, that we are only working at linear order in \(\omega_{\mu \nu }\), so we are not looking yet at a representation of the whole group, instead we are only considering elements which are close to the identity. 
So, we can insert this expression for \(S(\Lambda )\) into the relation between the \(S(\Lambda )\) and \(\Lambda \), equation \eqref{eq:spinorial-representation-definition}, using the fact that to first order \(S = \mathbb{1} + \epsilon \) is the inverse of \(S^{-1} = \mathbb{1} - \epsilon \) we have, to first order in \(\omega \): 
%
\begin{subequations}
\begin{align}
\qty(\mathbb{1} + \frac{i}{2} \omega_{\rho \sigma } \Sigma^{\rho \sigma }) \gamma^{\mu } \qty(\mathbb{1} - \frac{i}{2} \omega_{\alpha \beta } \Sigma^{\alpha \beta }) &= \gamma^{ \mu} +\tensor{\omega}{^{\mu }_{\nu }} \gamma^{\nu } \\
\gamma^{\mu } + \frac{i}{2} \omega_{\rho \sigma } \Sigma^{\rho \sigma } \gamma^{ \mu } - \frac{i}{2} \gamma^{\mu }\omega_{\alpha \beta } \Sigma^{\alpha \beta } &= \gamma^{\mu } + \omega_{\sigma \nu } \eta^{\mu \sigma } \gamma^{\nu }  \\
-\frac{i}{2} \omega_{\rho \sigma } \qty[\gamma^{\mu }, \Sigma^{\rho \sigma }] &= \omega_{\sigma \nu } \eta^{\mu [\sigma } \gamma^{\nu ]}  \\
\qty[\gamma^{\mu }, \Sigma^{\rho \sigma }] &= i \qty(\eta^{\mu \rho } \gamma^{\sigma } - \eta^{\mu \sigma } \gamma^{\rho })
\,,
\end{align}
\end{subequations}
%
where we antisymmetrized the indices in \(\eta^{\mu \rho } \gamma^{\sigma }\) since they are multiplied by the antisymmetric tensor \(\omega_{\rho \sigma }\), so any symmetric part would not contribute to the equation. 
We also used the fact that \(\omega_{\rho \sigma } \) is proportional to the identity in the spinor space. 

\begin{claim}
This is satisfied by 
%
\begin{align}
\Sigma^{\mu \nu } = \frac{i}{4} \qty[\gamma^{\mu }, \gamma^{\nu }] = \frac{1}{2} \sigma^{\mu \nu }
\,,
\end{align}
\end{claim}

\begin{proof}
We plug the definitions in: 
%
\begin{subequations}
\begin{align}
\qty[\gamma^{\mu }, \Sigma^{\rho \sigma }] &= \frac{i}{4} \qty(\gamma^{\mu } \gamma^{\rho} \gamma^{\sigma } - \gamma^{\mu } \gamma^{\sigma } \gamma^{\rho } - \gamma^{ \rho } \gamma^{\sigma } \gamma^{\mu } + \gamma^{ \sigma } \gamma^{ \rho } \gamma^{ \mu })  \\
i \qty(\eta^{\mu \rho } \gamma^{\sigma } - \eta^{\mu \sigma } \gamma^{\rho }) &= \frac{i}{2} \qty(\gamma^{\mu} \gamma^{\rho } \gamma^{\sigma } + \gamma^{\rho } \gamma^{\mu } \gamma^{\sigma } -\gamma^{\mu } \gamma^{\sigma }\gamma^{\rho } - \gamma^{\sigma } \gamma^{\mu } \gamma^{ \rho } )
\,,
\end{align}
\end{subequations}
%
where we used the anticommutation relations \(\qty{\gamma^{\mu }, \gamma^{\nu }} = 2 \eta^{\mu \nu }\). 
Then, we must check whether these two expression are equal, that is, whether 
%
\begin{subequations}
\begin{align}
\frac{1}{2} \qty( \mu  \rho \sigma  - \mu  \sigma  \rho  -  \rho  \sigma  \mu  +  \sigma   \rho   \mu ) &\overset{?}{=} 
\mu \rho  \sigma  + \rho  \mu  \sigma  -\mu  \sigma \rho  - \sigma  \mu   \rho   \\
&=  \qty{\mu, \rho }\sigma - \qty{\mu , \sigma } \rho 
\,,
\end{align}
\end{subequations}
%
where we only write the indices of the \(\gamma \)s for clarity. 

We can now use the identity: 
%
\begin{subequations}
\begin{align}
[\mu , \rho \sigma ] &= \mu \rho \sigma - \rho \sigma \mu  \\
&= \mu \rho \sigma + \rho \mu \sigma - \rho \mu \sigma - \rho \sigma \mu \marginnote{Added and subtracted}  \\
&= \qty{\mu, \rho } \sigma - \rho \qty{\mu, \sigma }
\,.
\end{align}
\end{subequations}

Note that in our convention there is no division by 2 in the commutator and anticommutator. 
Then, we can recognize these in the initial claim: we find 
%
\begin{subequations}
\begin{align}
\frac{1}{2} \qty( \mu  \rho \sigma  - \mu  \sigma  \rho  -  \rho  \sigma  \mu  +  \sigma   \rho   \mu )
&= \frac{1}{2} \qty(\qty[\mu, \rho \sigma ] - \qty[\mu, \sigma \rho ])  \\
&= \frac{1}{2} \qty(\qty{\mu, \rho } \sigma - \rho \qty{\mu, \sigma } -
\qty{\mu, \sigma } \rho + \sigma \qty{\mu, \rho })  \\
&= \qty{\mu, \rho } \sigma - \qty{\mu, \sigma } \rho 
\,,
\end{align}
\end{subequations}
%
since the anticommutators \(\qty{\mu, \sigma }\) are proportional to the metric \(\eta^{\mu \sigma }\), which is proportional to the identity in the spinorial space, which commutes with the gamma matrices, so \(\qty{\mu , \sigma }\) commutes with any gamma matrix. 
\end{proof}

Up until now we have worked ``near the identity'' of our transformation group; if we want to extrapolate these results to general transformations we may use the exponential map, which gives us relations in the form 
%
\begin{subequations}
\begin{align}
S(\Lambda ) &= \exp( -\frac{i}{2} \omega_{\mu \nu } \Sigma^{\mu \nu })  \\
S^{-1} (\Lambda ) &= \exp(\frac{i}{2} \omega_{\mu \nu } \Sigma^{\mu \nu })
\,.
\end{align}
\end{subequations}
%

\subsection{Dirac conjugate spinor}

We want to define a notion of a conjugate spinor, starting from what we know about hermitian conjugate operators: we start by taking the Hermitian conjugate of the Dirac equation, 
%
\begin{align}
\qty[\qty(i \slashed{\partial} -M) \psi ]^\dag = - i \psi ^\dag \qty(i \gamma^{\mu \dag} \overleftarrow{\partial}_{\mu } + M )
\,,
\end{align}
%
where the result follows from the fact that the adjoint of a product is the product of the reverse-ordered adjoints, while the notation \(\overleftarrow{\partial}_{\mu }\) means that the derivative operator is acting on what is on its left. 

In order to simplify this, we employ the following facts: \((\gamma^{0})^2 = \mathbb{1}\), and 
%
\begin{align}
\gamma^{0} \gamma^{\mu \dag} \gamma^{0} = \gamma^{\mu }
\,.
\end{align}

\begin{proof}
For \(\gamma^{0}\) the result is immediate, since it is self adjoint: \(\gamma^{0} = \gamma^{0, \dag}\), so the expression reduces to \(\qty(\gamma^{0})^3 = \qty(\gamma^{0})^2 \qty(\gamma^{0}) = \gamma^{0}\).

Then let us consider \(\gamma^{i}\): since they are block matrices of the Pauli matrices, which are self-adjoint and which have determinant \(-1\), we have:
%
\begin{subequations}
\begin{align}
\gamma^{i \dag} = \left[\begin{array}{cc}
0 & \sigma_{i} \\ 
- \sigma_{i} & 0
\end{array}\right] ^\dag = 
\left[\begin{array}{cc}
0 & - \sigma_{i} \\ 
\sigma_{i} & 0
\end{array}\right] = - \gamma^{i}
\,,
\end{align}
\end{subequations}
%
therefore we need to prove that \(\gamma^{0} \gamma^{i} \gamma^{0} = -\gamma^{i}\): 
%
\begin{subequations}
\begin{align}
\left[\begin{array}{cc}
1 & 0 \\ 
0 & -1
\end{array}\right]
\left[\begin{array}{cc}
0 & \sigma_{i} \\ 
-\sigma_{i} & 0
\end{array}\right]
\left[\begin{array}{cc}
1 & 0 \\ 
0 & -1
\end{array}\right] &= 
\left[\begin{array}{cc}
1 & 0 \\ 
0 & -1
\end{array}\right]
\left[\begin{array}{cc}
0 & -\sigma_{i} \\ 
- \sigma_{i} & 0
\end{array}\right]  \\
&= 
\left[\begin{array}{cc}
0 & - \sigma_{i} \\ 
\sigma_{i} & 0
\end{array}\right]
\,.
\end{align}
\end{subequations}
\end{proof}

With these results, we can use the manipulation 
%
\begin{subequations}
\begin{align}
\psi ^\dag \gamma^{\mu \dag} \overleftarrow{\partial}_{\mu } &= 
\psi ^\dag \gamma^{0} \gamma^{0} \gamma^{\mu \dag} \gamma^{0} \gamma^{0} \overleftarrow{\partial}_{\mu }  \\
&= \psi ^\dag \gamma^{0} \gamma^{\mu } \overleftarrow{\partial}_{\mu } \gamma^{0}
\,,
\end{align}
\end{subequations}
%
which holds since the matrix \(\gamma^{0}\) is constant.
We also define \(\overline{\psi} = \psi ^\dag \gamma^{0}\), so that the whole equation reads 
%
\begin{align}
\overline{\psi} \qty(i \slashed{\partial} + M ) =0 
\,,
\end{align}
%
where we removed the \(\gamma^{0}\) at the right, since the equation multiplied by it is equivalent to the one which is not. 

Now, in order to see how \(\overline{\psi}\) transforms under a Lorentz transformation we use the following facts: 
\begin{enumerate}
  \item \(\psi' (x' ) = S(\Lambda ) \psi (x)\) and 
  \item \(\gamma^{0} S ^\dag (\Lambda ) \gamma^{0} = S^{-1}(\Lambda )\).
\end{enumerate}

\begin{proof}
Let us first work to first order: 
we know that \(S(\Lambda )\) is written as 
%
\begin{align}
S(\Lambda )=   \exp( -\frac{i}{2} \omega_{\mu \nu } \Sigma^{\mu \nu })
\approx \mathbb{1} - \frac{i}{2} \omega_{\mu \nu } \Sigma^{\mu \nu }
\,,
\end{align}
%
so its adjunct is 
%
\begin{align}
S ^\dag (\Lambda ) = \exp( \frac{i}{2} \omega_{\mu \nu } \Sigma^{\mu \nu \dag}) \approx \mathbb{1} + \frac{i}{2} \omega_{\mu \nu } \Sigma^{\mu \nu \dag}
\,,
\end{align}
%
so when we multiply it from the left and right by \(\gamma^{0}\) it goes through everything, and we are left with 
%
\begin{align}
\gamma^{0} \Sigma^{\mu \nu \dag} \gamma^{0} \overset{?}{=} 
\Sigma^{\mu \nu }
\,,
\end{align}
%
since the expression for \(S^{-1}(\Lambda )\) differs from that of \(S(\Lambda )\) only for a sign in the exponent, but we picked up a sign when taking the adjunct. 
Recall that the explicit expression of the \(\Sigma \) matrices is 
%
\begin{align}
\Sigma^{\mu \nu } = \frac{i}{4} \qty[\gamma^{\mu }, \gamma^{\nu }]
\,,
\end{align}
%
so we can show directly that 
%
\begin{subequations}
\begin{align}
\gamma^{0} \Sigma^{\mu \nu \dag} \gamma^{0} &=
\frac{i}{4} \gamma^{0} \gamma^{\mu \dag}  \gamma^{\nu\dag } \gamma^{0 }
-\frac{i}{4} \gamma^{0} \gamma^{\nu\dag }  \gamma^{\mu \dag} \gamma^{0 } \\
&= 
\frac{i}{4} \gamma^{0} \gamma^{\mu \dag} \gamma^{0} \gamma^{0} \gamma^{\nu\dag } \gamma^{0 }
-\frac{i}{4} \gamma^{0} \gamma^{\nu\dag }  \gamma^{0} \gamma^{0} \gamma^{\mu \dag} \gamma^{0 }  \\
&= \frac{i}{4} \qty[\gamma^{\mu } \gamma^{\nu }] = \Sigma^{\mu \nu }
\,.
\end{align}
\end{subequations}

We have worked to first order until now, but the result we found actually works up to any order, since in the full expression of the exponential we will have terms proportional to 
%
\begin{align}
\frac{1}{n!} \qty(\omega_{\mu \nu } \Sigma^{\mu \nu \dag})^{n}
\,,
\end{align}
%
so if we multiply on both sides by \(\gamma^{0}\) we get something like (we omit the indices for simplicity) 
%
\begin{align}
\frac{1}{n!} \omega^{n} \gamma^{0} \Sigma^{\dag n} \gamma^{0} = 
\frac{1}{n!} \omega^{n} \gamma^{0} \Sigma^\dag \gamma^{0} \gamma^{0} \Sigma^\dag  \dots \gamma^{0} \gamma^{0} \Sigma^\dag \gamma^{0} =
\frac{\omega^{n}}{n!}\prod_{n} \qty(\gamma^{0} \Sigma^{\dag} \gamma^{0}) 
= \frac{\omega^{n}}{n!} \Sigma^{n}
\,,
\end{align}
%
so the reasoning extends to arbitrary orders in the expansion of the exponential. 
\end{proof}

Using these results, we can show that the conjugate spinor \(\overline{\psi} = \psi ^\dag \gamma^{0}\) transforms like: 
%
\begin{subequations}
\begin{align}
\overline{\psi}' (x') &= \psi^{\prime \dag} \gamma^{0}  \\
&= \qty(S(\Lambda ) \psi ) ^\dag \gamma^{0}  \\
&= \psi ^\dag S ^\dag (\Lambda ) \gamma^{0} \\
&= \psi ^\dag \gamma^{0} \gamma^{0} S ^\dag (\Lambda ) \gamma^{0} \\
&= \overline{\psi} S^{-1}(\Lambda ) \\
\,.
\end{align}
\end{subequations}

\subsection{Continuity equation}

We can apply the same strategy we used in the KG case: if we multiply the Dirac equation by \(\overline{\psi}\) on the left and the conjugate equation by \(\psi \) on the right, we get 
%
\begin{align}
\overline{\psi} \qty(i \slashed{\partial} - M ) \psi &= 0 
\overline{\psi} \qty(i \overleftarrow{\slashed{\partial}} + M ) \psi &= 0 
\,.
\end{align}

if we sum them, the \(M \overline{\psi} \psi  \) terms cancel, so we have 
%
\begin{subequations}
\begin{align}
0 &= \overline{\psi} \qty(i \overleftarrow{\slashed{\partial}} + i \slashed{\partial}) \psi   \\
&= i \qty( (\partial_{\mu }\overline{\psi}) \gamma^{\mu } \psi   + \overline{\psi} \gamma^{\mu }\partial_{\mu } \psi)  \\
&= \partial_{\mu } \qty(\overline{\psi} \gamma^{\mu } \psi )
\,,
\end{align}
\end{subequations}
%
so we have the conserved current \(\overline{\psi} \gamma^{ \mu } \psi   = J^{\mu }\).
Its corresponding charge density is \(\rho = J^{0}= \overline{\Psi} \gamma^{0} \psi = \psi ^\dag \psi \). 

This looks like a positive definite conserved charge, so it could be interpreted as a probability. 
In fact, this is not the case, as we shall see in the future. 

\subsection{Bilinear forms and Lorentz transformations}

Our observables can not still carry spinorial indices: all the spinorial indices must be saturated in the expression for an observable, since we can only observe scalars in the spin tensor algebra. 

For instance, \(\overline{\psi} \psi \) is the dot product of two spinors, so it carries no indices.
It is a scalar under Lorentz transformations, since \(\overline{\psi}\) transforms with a \(S^{-1}\) while \(\psi \) transforms with an \(S\).  

\(\overline{\psi} \gamma^{\mu } \psi \) is instead a vector under Lorentz transformations: it transforms like 
%
\begin{align}
\overline{\psi} \gamma^{\mu } \psi \rightarrow
\overline{\psi} S^{-1}(\Lambda ) \gamma^{\mu } S(\Lambda  ) \psi 
= \overline{\psi} \tensor{\Lambda }{^{\mu }_{\nu }} \gamma^{\nu } \psi 
\,,
\end{align}
%
where we used the property shown in equation \eqref{eq:spinorial-representation-definition}.

So, everything we are doing is covariant when we are discussing observables. 

\end{document}