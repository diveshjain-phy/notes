\documentclass[main.tex]{subfiles}
\begin{document}

\marginpar{Saturday\\ 2020-3-14, \\ compiled \\ \today}

\section{Conventions}

\subsection{Natural units}

Two constants which often come up in theoretical physics are Planck's constant \(\hbar = h /2 \pi  \approx \SI{6.582e-22}{MeV / Hz}\) and the speed of light \(c \approx \SI{2.997e8}{m/s}\). 
They are used to convert quantities which are \emph{equivalent}: a length is equivalent to a time interval if light passes through that length in that time interval in a vacuum; an energy is equivalent to an angular velocity if a photon with that angular velocity has that energy.

So, we can express lengths in ``seconds \( \times c\)'', energies in ``kilograms times \(c^2\)'' or ``Hertz times \(\hbar\)'', and so on.
Since this is very convenient, we go one step further and do not write the \(c\) and the \(\hbar\). 
This allows us to not worry about the number of times these should appear in a formula. 

Formally, we do this by imposing the conditions \(\hbar = c = 1\), where \(1\) is adimensional. 
Then, we can select a common unit and use it for everything: a common choice is the electronVolt (or its multiples).

Some examples of physical quantities in natural units:
\begin{enumerate}
  \item masses and energies are both measured in \SI{}{eV} (for masses, we should multiply by \(c^2\));
  \item linear momenta \(p = mv\) are measured in \SI{}{eV} (times \(c\));
  \item angular momenta \(L = r \wedge p\) are adimensional (they could be expressed in units of \(\hbar\));
  \item times and lengths are both measured in \SI{}{eV^{-1}}.
\end{enumerate}

This shortens formulas, but it obscures their dimensionality.
Thankfully we can always reinsert the necessary \(c\)s and \(\hbar\)s by dimensional analysis. 

\subsection{Relativistic notation}

A contravariant vector is denoted by writing its components, 
%
\begin{align}
v^{\mu } = \left[\begin{array}{c}
v^{0} \\ 
v^{1} \\ 
v^{2} \\ 
v^{3}
\end{array}\right]
\,,
\end{align}
%
and examples of these include the position 4-vector \(x^{\mu } = (t, \vec{x})\), and the 4-momentum \(p^{\mu } = (E, \vec{p})\). 

We shall use the worse convention for the metric, that is, the mostly negative \((+,-,-,-)\) one. 
This allows us to obtain covariant vectors as 
%
\begin{align}
x_{\mu } = \eta_{\mu \nu } x^{\nu } = (t,-\vec{x})
\,.
\end{align}

The derivative operator, instead, is naturally covariant: 
%
\begin{align}
\partial_{\mu } = \pdv{}{x^{\mu }} = (\partial_{t}, \vec{\nabla})
\,.
\end{align}

\section{The Klein Gordon equation}

We shall use the correspondence principle, as we did for the Schrödinger equation; however this time we will apply it to a relativistic particle. Its 4-momentum has a modulus square equal to the square of its mass: \(M^2\), since it is a relativistic invariant and we may compute it in any reference frame we like. 
In a generic frame, it is
%
\begin{align}
M^2 = p^{\mu } p_{\mu } =  p^{\mu } \eta_{\mu \nu } p^{\nu } = E^2 - \abs{\vec{p}}^2
\,,
\end{align}
%
which allows us to write the \emph{dispersion relation} 
%
\begin{align}
E^2 = \vec{p}^2 + M^2
\,.
\end{align}

This is \emph{quadratic} in the energy, as opposed to the nonrelativistic expression \(E = m + mv^2 /2\). 

Applying the correspondence principle, we find 
%
\begin{align}
- \partial_{t}^2 \varphi (\vec{x}, t) &= \qty(- \nabla^2 + M^2) \varphi (\vec{x}, t)  \\
0 &= \qty[\qty(\partial_{t}^2 - \nabla^2) + M^2] \varphi (\vec{x}, t)   \\
&= \qty[\square + M^2] \varphi (\vec{x}, t)
\,,
\end{align}
%
where we defined \(\square = \partial^{\mu } \partial_{\mu } = \partial_{t}^2 - \nabla^2\). 
Another way to see this is that, compactly stated, the correspondence principle is \(p_{\mu } = i  \partial_{\mu }\),\footnote{So we get \(E = p_0 = p^{0} = i \partial_{t}\), while \(\vec{p} = p^{i} = - p_{i} = - i \vec{\nabla} = - i \partial_{i}\).} so from \(p^2= M^2\) we get 
%
\begin{align}
\qty[\square + M^2] \varphi (\vec{x}, t) = 0
\,.
\end{align}

This is the \emph{free Klein-Gordon relativistic equation}. 

Let us check its covariance \emph{a posteriori}, even though it is guaranteed since we started from a covariant relation. 
Both \(M^2 = p^{\mu } p_{\mu }\) and \(\square = \partial^{\mu } \partial_{\mu }\) are scalars.

What about the wavefunction \(\varphi \)?
It should transform under a Lorentz transformation \(x \rightarrow x' = \Lambda x\) as 
%
\begin{align}
\varphi' (x')= \varphi (x)
\,.
\end{align}

This amounts to saying that it is a \emph{scalar} function. 
With these constraints, we can say that the Klein-Gordon equation is \emph{covariant}; moreover, since it is a scalar equation it is actually \emph{invariant}. 
If we apply a Lorentz transformation, we find that indeed 
%
\begin{align}
\qty[\square' + M^{\prime 2} ] \varphi^{\prime } (x^{\prime }) = \qty[\square + M^2] \varphi (x)
\,,
\end{align}
%
since \(\square\), \(M^2\) and \(\varphi (x)\) are scalars; therefore covariance is proven. 
If this is zero in an inertial reference frame, it is also zero in any other inertial reference frame. 

\subsection{Continuity equation}

Now, let us seek a probability current for the KG equation as we did with the Schrödinger equation. 
Let us multiply on the left the KG equation by the conjugate of the wavefunction, \(\varphi^{*}\). We shall use this equation and its conjugate: 
%
\begin{align}
\varphi^{*} \qty[\square + M^2] \varphi &= 0 \\
\varphi \qty[\square + M^2] \varphi^{*} &= 0
\,,
\end{align}
%
these both hold since \(\square\) and \(M^2\) are real. 
If we subtract one of these from the other, the mass terms simplify and we get 
%
\begin{align}
0 &= \varphi^{*} \square \varphi - \varphi \square \varphi^{*}  \\
&= \varphi^{*} \partial^{\mu } \partial_{\mu } \varphi  - (\varphi^{*} \leftrightarrow \varphi )
\,,
\end{align}
%
where the notation  \(- (\varphi^{*} \leftrightarrow \varphi )\) means that we are subtracting the same thing which appears before, but written swapping \(\varphi \) and \(\varphi^{*}\). We can expand the derivatives: 
%
\begin{align}
0 &= \partial^{\mu } \qty(\varphi^{*} \partial_{\mu } \varphi ) - \qty(\partial^{\mu } \varphi^{*}) \qty(\partial_{\mu } \varphi ) - (\varphi^{*} \leftrightarrow \varphi )  \\
&= \partial^{\mu } \qty(\varphi^{*} \partial_{\mu } \varphi ) - (\varphi^{*} \leftrightarrow \varphi )
\marginnote{The product of the gradients is symmetric under interchange of \(\varphi \) and \(\varphi^{*}\), since the metric is constant.}  \\
&= \partial^{\mu } \qty(\varphi^{*} \partial_{\mu } \varphi - \varphi \partial_{\mu } \varphi^{*})
\overset{\text{def}}{=} -2i \partial^{\mu } j_{\mu }
\,,
\end{align}
%
where we defined the 4-current 
%
\begin{align}
j_{\mu } = \frac{i}{2}\varphi^{*} \partial_{\mu } \varphi 
- (\varphi^{*} \leftrightarrow \varphi )
\,, 
\end{align}
%
so we can write the conservation equation \(\partial^{\mu } j_{\mu } = \partial_{\mu } j^{\mu } = 0\) in 3-vector form as: 
%
\begin{align}
\partial_{t} \rho + \vec{\nabla} \cdot \vec{j}
\,,
\end{align}
%
where 
%
\begin{align}
\rho = \frac{i}{2} \varphi^{*} \partial_{t} \varphi - (\varphi^{*} \leftrightarrow \varphi )
\qquad \text{and} \qquad
\vec{j} = - \frac{i}{2} \varphi^{*} \partial^{i} \varphi 
-(\varphi^{*} \leftrightarrow \varphi ) \marginnote{Minus sign since we want the contravariant components.}
\,.
\end{align}

We can integrate the continuity equation over all 3D space to obtain a conserved quantity: 
%
\begin{align}
\int_{\mathbb{R}^{3}} \dd[3]{x} \rho  \overset{\text{def}}{=} Q
\,,
\end{align}
%
which is actually constant since 
%
\begin{align}
\dv{Q}{t} = \int_{\mathbb{R}^{3}} \dd[3]{x}  \partial_{t} \rho 
= - \int_{\mathbb{R}^{3}} \dd[3]{x} \vec{\nabla} \cdot \vec{j} 
= - \int_{S^{2}_{ \infty }} \vec{j} \cdot \hat{n} \dd{A} \rightarrow 0 \marginnote{Used the divergence theorem, the surface \(S^{2}_{ \infty }\) is a sphere with diverging radius.}
\,.
\end{align}

We call this conserved quantity a ``charge''. 
There is an issue: in the Schrödinger case the quantity called \(\rho \) was positive by definition; now instead \(\rho \) and \(Q\) are not necessarily positive.

This can be proven as follows: suppose \(\rho \) was positive for a certain wavefunction \(\varphi \). The conjugate wavefunction \(\varphi^{*} \) is also a solution to the KG equation, and for it the density will be negative, since permuting \(\varphi \) and \(\varphi^{*}\) is equivalent to changing the sign of \(j_{\mu }\). 

So, we cannot use the Bohr ansatz, interpreting \(Q\) as a probability. 

\subsection{Solutions to the free KG equation}

Let us forget about the physical interpretation for a while, and discuss the general solutions of the KG equation. 
We can decompose the wavefunction \(\varphi(x) \) in terms of its Fourier transform \(\widetilde{\varphi}(k)\): 
%
\begin{align}
\varphi (x) = \frac{1}{(2 \pi )^2} \int \dd[4]{k} e^{-i k^{\mu } x_{\mu }} \widetilde{\varphi}(k)
\,.
\end{align}

In order for this to be covariant, if \(\varphi \) is a scalar then \(\widetilde{\varphi}\) also must be. 
The argument of the exponential is also a scalar. 

The volume form in the momentum space \(\dd[4]{k}\) is a scalar: under a Lorentz transformation it transforms as 
%
\begin{align}
\dd[4]{k'} = \abs{\det \Lambda } \dd[4]{k}
\,,
\end{align}
%
so it does not change since \(\abs{\det \Lambda } = 1\) for Lorentz transformations.

\begin{claim}
The inverse Fourier transform reads 
%
\begin{align}
\widetilde{\varphi} (k) = \frac{1}{(2\pi )^2}
\int \dd[4]{x} e^{i k^{\mu } x_{\mu } } \varphi (x)
\,.
\end{align}
\end{claim}

\begin{proof}
We take the transform of the antitransform: 
%
\begin{align}
\widetilde{\varphi} (k) &=  \frac{1}{(2\pi )^{2}}
\int \dd[4]{x} \varphi (x) e^{i k^{\mu } x_{\mu }}  \\
&=  \frac{1}{(2\pi )^{4}}
\int \dd[4]{x} e^{i k^{\mu } x_{\mu }}
\int \dd[4]{k'} e^{-i k^{\prime \mu } x_{\mu }} \varphi (k')  \\
&= \int \dd[4]{k'} \qty[\frac{1}{(2\pi )^{4}} 
\int \dd[4]{x} e^{-i (k^{\prime \mu } - k^{\mu }) x_{\mu }} ] \varphi (k^{\prime })  \\
&= \int \dd[4]{k'} \delta^{(4)} (k - k') \varphi (k') = \varphi (k)
\,,
\end{align}
%
where we used the definition of the 4D Dirac \(\delta \) function (here in position space, the definition in momentum space is perfectly analogous): 
%
\begin{align}
\delta^{(4)} (x^{\mu }) = \frac{1}{(2\pi )^{4}}
\int \dd[4]{k} e^{-i k_{\mu } x^{\mu }}
\,,
\end{align}
%
and its main property: 
%
\begin{align}
\int \dd[4]{x} \delta^{(4)} (x) f(x) = f(0)
\,.
\end{align}
\end{proof}

Now, to solve the KG equation we insert the Fourier expression of the wavefunction into it: 
%
\begin{align}
0= \qty[\square + M^2] \varphi (x) &= \int \frac{ \dd[4]{k}}{(2\pi )^2} \qty[- k^2+M^2] e^{-i k^{\mu }x_{\mu }} \widetilde{\varphi}(k) 
\marginnote{The \(\square\) operator acts in position space, so it has no effect on \(\widetilde{\varphi}(k)\): it applies only to the exponential, yielding \(-k^2=(-ik^{\mu })(-ik_{\mu })\).}
\,,
\end{align}
%
and since the integral must be zero the integrand must be zero as well. 
% \todo[inline]{What? this is not true in general!}
This is because the monochromatic waves are a basis for the Hilbert space, and the Fourier transform is an isomorphism of Hilbert spaces, which maps the zero function to the zero function. 

So, we can use the ansatz \(\widetilde{\varphi} (k)  = \delta (k^2 -M^2) \widetilde{f}(k)\), where \(\widetilde{f}(k)\) is a generic scalar function. 
If \(\widetilde{\varphi} (k)\) is written in this way, it automatically satisfies the KG equation. 

Now, recall that for a Dirac delta function applied to a generic function \(f(x)\), whose zeroes are enumerated by the index \(i\) (that is, \(f(x_{i}) = 0\) for all \(i\) between \(1\) and \(N\)) the following property holds: 
%
\begin{align}
\delta (f(x)) =  \sum _{i=1}^{N} \frac{ \delta (x - x_{i})}{\abs{f'(x_i)}}
\,.
\end{align}

We can apply this property to \(\delta (k^2-M^2)\).
First of all, since \(k^{2} = k_0^2 - \abs{\vec{k}}^2 \), we can write this expression as \(\delta (k_0^2 - \omega_{k}^2)\), by defining \(\omega_{k}\) as the positive root:
%
\begin{align}
\omega_{k} = + \sqrt{\abs{k}^2 + M^2}
\,.
\end{align}

Now, we apply the \(\delta \) function property: 
%
\begin{align} 
\delta (k^2 - M^2) = 
\delta (k_0^2 - \omega_{k}^2) &= \frac{ \delta (k_0 - \omega_{k})}{ \abs{2 k_0 }} 
+ 
\frac{ \delta (k_0 + \omega_{k})}{ \abs{2 k_0} }  \\
&= \frac{ \delta (k_0 - \omega_{k}) + \delta (k_0 + \omega_{k})}{2 \omega_{k}}
\,,
\end{align}
%
where we substituted \(\abs{k_0} = \omega_{k}\), which holds both if \(k_0 = \omega_{k}\) and if \(k_0 = - \omega_{k}\). 
This finally gives us  
%
\begin{align}
\widetilde{\varphi} (k) = \frac{1}{2 \omega_{k}} \qty(\delta (k_0 - \omega_{k}) + \delta (k_0 + \omega_{k}))
\widetilde{f}(k)
\,,
\end{align}
%
which we can insert this into the Fourier transform of \(\varphi (x)\): 
%
\begin{align}
\varphi (x) &= 
\frac{1}{(2\pi )^2}  
\int \dd[4]{k} \widetilde{\varphi} (k) e^{-ik^{\mu }x_{\mu }} \\
&=
\frac{1}{(2\pi )^2} 
\int \frac{ \dd[4]{k}}{2 \omega_{k}} 
\qty(\delta (k_0 - \omega_{k}) + \delta (k_0 + \omega_{k}))
e^{-ik^{\mu } x_{\mu }} \widetilde{f}(k)  \\
&= \frac{1}{(2\pi)^2} 
\int \frac{ \dd[3]{k}}{2 \omega_{k}}
\qty[
e^{- i \omega_{k} x_0 } e^{i \vec{k} \cdot \vec{x}} \widetilde{f} (\omega_{k}, \vec{k})  + 
e^{ i \omega_{k} x_0 } e^{i \vec{k} \cdot \vec{x}} \widetilde{f} (-\omega_{k}, \vec{k})  
]  \marginnote{We integrate in \(\dd{k^{0}}\) to get rid of the \(\delta \) functions.}\\
&= \frac{1}{(2\pi )^2}
\int \frac{ \dd[3]{k}}{2 \omega_{k}}
\qty[e^{-i k^{\mu} x_{\mu }} \widetilde{f}(k^{\mu })
+ e^{i k^{\mu }x_{\mu }} \widetilde{f}(-k^{\mu })]_{k_0 = \omega_{k}} 
\,,
\end{align}
%
where we indicate \(\eval{k^{\mu }}_{k_0 = \omega_{k}} = (\omega_{k}, \vec{k})\).

We used the fact that, in the Fourier transform integral, the terms \(e^{i \vec{k} \cdot \vec{x}} \widetilde{f} (\vec{x})\) and \(e^{-i \vec{k} \cdot \vec{x}} \widetilde{f}(- \vec{x})\) are equivalent: this is because, since we are integrating over all of 3D space, any contributions which are \emph{odd} in \(\vec{k}\) will not affect the total integral, therefore we can only consider the even part of the integrand. 

Now, in order to simplify the notation we define 
%
\begin{align}
a(k) = \frac{\widetilde{f}(k)}{\sqrt{2\pi } \sqrt{2 \omega_{k}}}
\qquad \text{and} \qquad
b(k) = \frac{\widetilde{f} (-k)}{\sqrt{2 \pi } \sqrt{2 \omega_{k}}}
\,,
\end{align}
%
which are arbitrary like the initial function \(\widetilde{f}\), however they are connected since \(a(k) = b^{*} (-k)\). 
So, the final solution reads: 
%
\begin{align}
\varphi (x) &= \frac{1}{(2 \pi )^{3/2}} \int \frac{ \dd[3]{k}}{\sqrt{2 \omega_{k}}} \qty[a(k) e^{-ik \cdot x} + b^{*}(k) e^{i k \cdot x}]_{k_0 = \omega_{k}}  \\
&= \varphi_{+}(x) + \varphi_{-}(x)
\,.
\end{align}

Recall that all \(k\)s appearing in the expression are to be interpreted as \((\omega_{k}, \vec{k})\). 
The part dependent on \(a(k)\) is conventionally called the \emph{positive energy} solution \(\varphi_{+}\), while the part depending on \(b^{*}(k)\) is the \emph{negative energy} solution \(\varphi_{-}\).

This is because, as the energy of a wavefunction \(\varphi \) is computed by \(E = i \partial_{t} \varphi \), we have 
%
\begin{align}
E (\varphi_{+}) &= i \partial_{0} \qty{
  \frac{1}{(2 \pi)^{3/2}} 
  \int \frac{ \dd[3]{k}}{\sqrt{2\omega_{k}}} a(k) e^{-i k^{\mu } x_{\mu }} 
} = i (-i) \omega_{k} \varphi_{+} = \omega_{k} \varphi_{+}  \\
E ( \varphi_{-}) &= i \partial_{0} \qty{
  \frac{1}{(2 \pi)^{3/2}} 
  \int \frac{ \dd[3]{k}}{\sqrt{2\omega_{k}}} b^{*}(k) e^{i k^{\mu } x_{\mu }} 
} = i (i) \omega_{k} \varphi_{-} =- \omega_{k} \varphi_{-}  
\,,
\end{align}
%
so \(\varphi_{+}\) has a positive energy while \(\varphi_{-}\) has a negative one. 

This is the main difference between the Schrödinger and KG equations. 

The solution to the KG equation is not explicitly covariant, but all the steps preserved covariance so the final solution is still covariant. 

The KG equation is real, since \(\square\) is a real operator and \(M^2\) is real, so it will admit real solutions. 
In order to find these we impose \(\varphi = \varphi^{*}\). 

\begin{claim}
\(\varphi = \varphi^{*}\) implies \(a = b\).
\end{claim}
\begin{proof}
We write only the argument of the integrals for simplicity: 
%
\begin{align}
\varphi &\sim a e^{-ikx} + b^{*} e^{ikx}  \\
\varphi^{*} &\sim a^{*} e^{ikx} + b e^{-ikx}
\,,
\end{align}
%
so if \(\varphi = \varphi^{*}\) we must identify these component by component, so we must have \(a= b\), and \(a^{*} = b^{*}\).
\end{proof}

Then, the most general real solution to the KG equation is 
%
\begin{align}
\varphi_{\mathbb{R}} (x) = \frac{1}{(2 \pi)^{3/2}} 
\int \frac{ \dd[3]{k}}{\sqrt{2 \omega_{k}}}
\qty[a(k) e^{-ik \cdot x} + a^{*} (k) e^{i k \cdot x}]_{k_0 = \omega_{k}}
\,.
\end{align}

\begin{claim}
For a real KG solution, the function \(a(k)\) can be written as 
%
\begin{align}
a(k) = \frac{1}{(2 \pi  )^{3/2}} \int \frac{ \dd[3]{x}}{\sqrt{2 \omega_{k}}} \qty(\omega_{k} \varphi(x) +i \partial_{0} \varphi (x)) \eval{e^{i k \cdot x} }_{k_0 = \omega_{k}} 
\,.
\end{align}
\end{claim}

\begin{proof}
The solution is found by direct substitution of \(\varphi_{\mathbb{R}}\) into the expression for \(a\) in order to verify it; the operations are all reversible so we can use the derivation backwards or forwards equivalently. We find 
%
\begin{align}
a &\overset{?}{=} 
\frac{1}{(2 \pi )^3} \int \frac{ \dd[3]{k} \dd[3]{x}}{2 \omega_{k}}
\qty[\omega_{k} \qty(a e^{-i k \cdot x} + a^{*} e^{i k \cdot x}) + i \partial_{0} \qty(a e^{-i k \cdot x} + a^{*} e^{i k \cdot x})]\eval{e^{i k \cdot x}}_{k_0 = \omega_{k}}  \\
&= \frac{1}{(2 \pi )^3} \int \frac{ \dd[3]{k} \dd[3]{x}}{2 \omega_{k}} \qty[\qty(\omega_{k} +i (-i k_0 )) a^{-ik \cdot x} + \qty(\omega_{k} +i (i k_0 )) a^{*} e^{i k \cdot x}]\eval{e^{i k \cdot x}}_{k_0 = \omega_{k}} \\
&= \frac{1}{(2 \pi )^3} \int \frac{ \dd[3]{k} \dd[3]{x}}{2 \omega_{k}} \qty[2\omega_{k} a^{-ik \cdot x}]\eval{e^{i k \cdot x}}_{k_0 = \omega_{k}} \marginnote{Used the fact that \(\omega_{k} = k_0 \), and \(i^2=-1\).}  \\ 
&= \frac{1}{(2\pi )^{3/2}} \int \dd[3]{x} e^{i \vec{k} \cdot \vec{x}} \qty[
\frac{1}{(2\pi )^{3/2}} \int \dd[3]{k} e^{-i \vec{k} \cdot \vec{x}} a
] = a \marginnote{3D Fourier inverse and direct transform. We simplified two factors of \(e^{ik_0 x^{0}}\) and \(e^{-ik_0 x^{0}}\) since they are equal everywhere (\(k_0 = \omega_{k}\)).}
\,.
\end{align}
\end{proof}

\begin{claim}
For a complex KG solution, the functions \(a(k)\) and \(b^{*}(k)\) can be written as 
%
\begin{align}
a(k) &= \frac{1}{(2 \pi  )^{3/2}} \int \frac{ \dd[3]{x}}{\sqrt{2 \omega_{k}}} \qty(\omega_{k} \varphi(x) +i \partial_{0} \varphi (x)) \eval{e^{i k \cdot x} }_{k_0 = \omega_{k}} \\ 
b^{*}(k) &= \frac{1}{(2 \pi  )^{3/2}} \int \frac{ \dd[3]{x}}{\sqrt{2 \omega_{k}}} \qty(\omega_{k} \varphi(x) - i \partial_{0} \varphi (x)) \eval{e^{-i k \cdot x} }_{k_0 = \omega_{k}} 
\,.
\end{align}
\end{claim}

\begin{proof}
The derivation is the same as the real-valued solution case. The \(b^{*}\) terms simplify if there is a plus in front of the \(i \partial_{0}\) term, if instead we have a minus the \(a\) terms simplify; everything else is precisely the same.  
\end{proof}

\begin{claim}
Given two real solutions to the KG equation, \(\varphi_{1}\) and \(\varphi_{2}\), one can always write a complex solution \(\varphi = (\varphi_{1} + i \varphi_{2}) / \sqrt{2}\). 

Then, the functions \(a\) and \(b^{*}\) for the complex solution can be written in terms of the \(a_{1}\) and \(a_2 \) for the real solution as: 
%
\begin{align}
a &=  \frac{ a_1 + i a_2 }{\sqrt{2}} \\
% b^{*} &= \frac{a_1^{*} + i a_2^{*} }{\sqrt{2}}  \\
b &= \frac{a_1 - i a_2 }{\sqrt{2}}
\,.
\end{align}
\end{claim}

\begin{proof}
We write out the complex function: 
%
\begin{align}
\frac{\varphi_1 + i \varphi_2 }{\sqrt{2}} = 
\frac{1}{(2 \pi )^{3/2}} 
\int \frac{ \dd[3]{k}}{\sqrt{2 \omega_{k }}} \qty[
\frac{a_1 + i a_2 }{\sqrt{2}} e^{-i k \cdot x}
+  \frac{a_1^{*} + i a_2^{*}}{\sqrt{2}}
e^{i k \cdot x}]
\,,
\end{align}
%
so, since \(b^{*} = \qty(a_1^{*} + i a_2^{*}) / \sqrt{2}\), we can get \(b\) by conjugating, 
%
\begin{align}
b = \frac{a_1 - i a_2 }{\sqrt{2}}
\,,
\end{align}
%
while \(a\) can be directly read off the expression.
\end{proof}

\end{document}
