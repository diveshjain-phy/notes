\documentclass[main.tex]{subfiles}
\begin{document}

\marginpar{Monday\\ 2020-3-16, \\ compiled \\ \today}

\section{The Klein-Gordon equation in the presence of an external electromagnetic field}

A complex scalar solution to the Klein-Gordon equation can represent a charged relativistic spin-0 particle. 
Since the particle is charged, it is interesting to study its interaction with the electromagnetic field.

We will discuss only the interaction of the particle with an externally determined field, that is, we will not discuss how the particle influences the field around it.

In order to describe the electric and magnetic fields in a covariant way we use the four-vector \(A_{\mu } = (A_0 , \vec{A})\), such that \(\vec{E} = - \vec{\nabla} A_0 - \partial_{0} \vec{A}\) and \(\vec{B} = \vec{\nabla} \times \vec{A}\), or, in covariant terms, 
%
\begin{align}
F^{ \mu \nu } = 2\partial^{[\mu } A^{\nu ]}
\,,
\end{align}
%
where the antisymmetric field-strength tensor \(F^{\mu \nu }\) encodes both the electric and magnetic fields. 

\begin{definition}
The \textbf{minimal coupling} to an external electromagnetic field is obtained by substituting in the wave equation the partial derivative \(\partial_{\mu }\) with the \emph{covariant derivative}
%
\begin{align}
\mathrm{D}_{\mu } \overset{\text{def}}{=} \partial_{\mu } + i q A_{\mu }
\,,
\end{align}
%
where \(q\) is the electric charge of the particle. 
\end{definition}

Inserting this, the coupled Klein-Gordon equation reads 
%
\begin{align}
\qty[\DD^{\mu } \DD_{\mu } + M^2 ] \varphi (x) = 0 
\,.
\end{align}

Recall that the momentum is defined as \(p_{\mu } = i \partial_{\mu }\), so this substitution is equivalent to changing the momentum to \(i (\partial_{\mu } +iqA_{\mu }) =  p_{\mu } - q A_{\mu }\).

This is the same minimal coupling ansatz which is used in nonrelativistic quantum mechanics: there, the minimally-coupled Schrödinger equation reads 
%
\begin{align}
E \psi = \qty[\frac{\qty(\vec{p} - q \vec{A})^2}{2M}  + q A_0 ] \psi 
\,,
\end{align}
%
in which we can see the two contributions: the three-momentum \(\vec{p}\) is exchanged for \(\vec{p} - q \vec{A}\), while the energy \(E\) is exchanged for \(E - q A_0 \) (and the term is brought on the other side of the equation for convenience). 

Let us make the terms in the minimally coupled Klein-Gordon equation explicit: 
%
\begin{subequations}
\begin{align}
\qty[\qty(\partial^{\mu } +iqA^{\mu }) \qty(\partial_{\mu } +i q A_{\mu }) + M^2] \varphi &= 0  \\
\qty[\square + iq A^{\mu } \partial_{\mu } + iq \partial_{\mu } A^{\mu } - q^2A^2 + M^2] \varphi &=0  \\
\qty[\square + 2iq A^{\mu } \partial_{\mu } + iq \qty(\partial_{\mu } A^{\mu }) - q^2A^2 + M^2] \varphi &=0  
\,,
\end{align}
\end{subequations}
%

Do note that anything on the right acts on anything on the left: the application of the derivatives looks like \(
\DD^{\mu } \qty(\DD_{\mu } \qty(\varphi ))\). So, the term 
\(\partial_{\mu } A^{\mu }\) should be read as \(\partial_{\mu } \qty(A^{\mu } \varphi )\), which we expanded into \(\qty(\partial_{\mu } A^{\mu }) + A^{\mu } \partial_{\mu }\) --- all acting on \(\varphi \), but the term \(\qty(\partial_{\mu }A^{\mu })\) is just multiplied by \(\varphi \), it does not perform a differentiation.

\subsubsection{The Coulomb potential}

The 4-potential is not physical: if we change it by \(A_{\mu } \rightarrow A^{\prime }_{\mu } = A_{\mu } + \partial_{\mu } \Lambda \) for some scalar function \(\Lambda \) it does not change the resulting measurable electric and magnetic fields. 
Therefore, while still retaining full generality in our description of physical systems we can impose certain conditions, such as \(\partial_{\mu } A^{\mu } = 0\). 
This condition, known as the \emph{Coulomb gauge}, does not actually fix all of the gauge freedom --- that is, even by imposing this we still do not have a one-to-one correspondence between the physical fields and the 4-potentials: after imposing it, we can still perform gauge transformations where the function \(\Lambda \) is harmonic, that is, it satisfies \(\square \Lambda = 0\). 

This condition is convenient since it allows us to get rid of a term in the KG equation, so we impose it. 

\subsubsection{Antimatter: an interpretation for the negative-energy solution}

Let us consider a simple case: the monochromatic solution to the KG equation, \(\varphi_{\pm} (x) = e^{\mp i k \cdot x}\), coupled to the EM field.
Let us suppose that the 4-momentum \(k^{\mu} = (\omega_{k}, \vec{k})\) corresponding to the two solutions is fixed, that is, the positive energy solution has \(E \varphi^{+} = \omega_{k}\) while the negative energy solution has \(E \varphi^{-} = - \omega_{k}\), with the same \(\omega_{k}\) and the same \(\vec{k}\). 

Splitting the energy and momentum terms, the equation reads: 
%
\begin{align}
\qty[\qty(E - q A_0 )^2 + \qty(\vec{p} - q \vec{A}) + M^2] \varphi^{\pm}(x) = 0
\,,
\end{align}
%
where \(E = i \partial_{t} \) while \(\vec{p} = - i \vec{\nabla}\). 

Then, we can write the two equations like 
%
\begin{subequations}
\begin{align}
\qty[\qty(\omega_{k} - q A_0 )^2 + \qty(\vec{k} - q \vec{A}) + M^2] \varphi^{+}(x) &= 0 \\
\qty[\qty(\omega_{k} + q A_0 )^2 + \qty(\vec{k} + q \vec{A}) + M^2] \varphi^{-}(x) &= 0 
\,,
\end{align}
\end{subequations}
%
where we used the fact that \((-\omega_{k} - q A_0 )^2 = (\omega_{k} + q A_0 )^2\), and similarly for the other term.  

So, we can see that the two equations can be transformed into each other also by swapping the charge, \(q \rightarrow -q\). 

The interpretation for this is that the KG equation describes two degrees of freedom, which we call a \emph{particle} (with \(E>0\)) and an \emph{antiparticle} (with \(E<0\)).

The particle and antiparticle have the same mass, but their charge is opposite. 
In a certain sense, the existence of antimatter is a prediction of relativistic quantum mechanics. 

\subsection{The nonrelativistic limit}

As a consistency check for our relativistic equation, we wish to verify that the nonrelativistic limit of the KG equation is the Schrödinger equation.

In the nonrelativistic limit the mass of the particle \(M\) is much larger than its momentum \(p\), so we can expand: 
%
\begin{align}
E = \sqrt{\abs{p}^2 + M^2}
= M \sqrt{1 + \frac{\abs{p}^2}{M^2}}
= M + \underbrace{\frac{\abs{p}^2}{2M}}_{E_k} + \order{\frac{\abs{p}^{4}}{M^{3}}}
\,.
\end{align}

In quantum mechanics the time evolution of a particle depends on its energy, so if \(M \gg E_k\) the evolution due to the mass will be much faster than that due to the kinetic energy: so, we factor it out by 
%
\begin{align}
\varphi (t, \vec{x}) = e^{-iMt} \varphi^{\prime } (t, \vec{x})
\,,
\end{align}
%
where the evolution of \(\varphi^{\prime }\) will look like \(e^{-iE_k t} \varphi'\). 

Now, let us expand the derivative terms in the KG equation (in the Coulomb gauge, so we do not worry about \(\partial_{\mu }A^{\mu }\)): 
%
\begin{subequations}
\begin{align}
\DD^{\mu } \DD_{\mu } &= \qty(\partial^{0} + i q A^{0})
\qty(\partial_{0} + i q A_{0})
+ \qty(\partial^{i} + i q A^{i})
\qty(\partial_{i} + i q A_{i}) \\
&= \qty(\partial_{0} + i q A_{0})^2
- \qty(\partial_{i} + i q A_{i})^2  \\
&= \qty(\partial_{0} + i q A_{0})^2
- \qty(\vec{\nabla} - i q \vec{A})^2
\,,
\end{align}
\end{subequations}
%
where we used the fact that \(\vec{A} = A^{i}\), while \(\vec{\nabla} = \partial_{i} = - \partial^{i}\). 

In the full KG equation applied to the decomposed field \(\varphi = e^{-iMt} \varphi^{\prime }\) we move the momentum terms to the right and keep the mass and energy on the left: 
%
\begin{subequations}
\begin{align}
\qty[\qty(\partial_{0} + i q A_0 )^2 + M^2] e^{-iMt} \varphi^{\prime }
&=  
\qty(\nabla- iq \vec{A})^2 e^{-iMt} \varphi^{\prime }
\marginnote{The term \(e^{-iMt}\) is constant with respect to the spatial coordinates}
\\ 
\qty[\partial_0^2 + 2iq A_0 \partial_0 - q^2 A_0^2 + M^2]
e^{-iMt} \varphi^{\prime }
&=  
e^{-iMt} \qty(\nabla- iq \vec{A})^2 \varphi^{\prime }
\,.
\end{align}
\end{subequations}

The time-derivative on the right acts both on the mass term and on the wavefunction \(\varphi^{\prime }\). So, we get\footnote{It is really easy when doing these computations to forget some terms. A way I've found helpful is to write down the full expression, 
%
\begin{align}
\qty(\partial_0 + i q A_0 )
\qty(\partial_0 + i q A_0 )
\qty(e^{-iMt} \varphi')
\,,
\end{align}
%
give binary labels (000, 001\dots until 111) to each term and work through all \(2^3\) of them. Notice the expansion of the term 
%
\begin{align}
\partial_0 \partial_0 \qty(e^{-iMt} \varphi')
= \partial_0 \qty(-iM e^{-iMt} \varphi' + e^{-iMt} \partial_0 \varphi')
= -iM e^{-iMt} \partial_0 \varphi' 
- M^2 e^{-iMt} \varphi' 
+ e^{-iMt} \partial_0^2 \varphi 
- iM e^{iMt} \varphi' 
\,.
\end{align}
}
%
\begin{subequations}
\begin{align}
\begin{split}
e^{-iMt} \qty(\nabla- iq \vec{A})^2 &\varphi^{\prime } = \\
=e^{-iMt}
\qty[\partial_0^2 - M^2 - 2iM \partial_0  + 2iq A_0 (-iM) + 2 iq \qty(\partial_0 A_0 ) + 2iq A_0 \partial_0 - q^2 A_0^2 + M^2] & \varphi^{\prime }  
\end{split}
\\ 
\qty[\partial_0^2 - 2iM \partial_0  + 2q A_0 M + 2 iq \qty(\partial_0 A_0 ) + 2iq A_0 \partial_0 - q^2 A_0^2 ]
\varphi^{\prime }
=  
\qty(\nabla- iq \vec{A})^2  &\varphi^{\prime }
\,,
\end{align}
\end{subequations}
%
and now we must discuss which terms we can discard. 

On the right hand side we have terms of the order of the mass and ones of the order of the energy: we wish to keep only the former, since as we assumed they dominate the latter. Specifically, terms without explicit dependence on the mass are of the order of the kinetic energy:\footnote{This is imprecise, we can do better: we distinguish the terms with time derivatives and those without. 
Those with time derivatives are applied to \(\varphi'\), which by construction evolves with \(\omega = E_k\), therefore they are negligible. 
The other terms are \(q^2 A_0^2\), which is the electric potential energy, and  \(2iq \partial_0 A_0 \), which is proportional to the electric field: we assume the electric field to be nonrelativistic, that is, of intensity comparable to the kinetic energy of the particle, and much smaller than its mass.  
}
then, we remove and get precisely the minimally coupled Schrödinger equation:
%
\begin{subequations}
\begin{align}
\qty[- 2 iM \partial_0 + 2 q_0 A_0 M ] \varphi' &= \qty(\vec{\nabla} - iq \vec{A})^2 \varphi^{\prime }  \\
i \partial_0 \varphi' &= \qty[- \frac{ \qty(\vec{\nabla} - i q \vec{A})^2}{2M} + q_0 A_0 ] \varphi'
\,.
\end{align}
\end{subequations}
%

This is precisely the equation we wrote above: the nonrelativistic Schrödinger equation for a charged spin-0 particle in an electromagnetic field. 

\section{Summary}

\begin{enumerate}
  \item The dispersion relation \(E^2 = p^2+m^2\) is a quadratic relation, the positive root is called \(\omega_{p} = \sqrt{p^2+m^2}\) and the energy can be \(E = \pm \omega_{p}\). 
  \item A solution to the KG equation has two independent terms \(\varphi^{+}\) and \(\varphi^{-}\), which can be interpreted as a particle-antiparticle pair, where \(\varphi^{+}\) has charge \(q\) while \(\varphi^{-}\) has charge \(-q\). 
  
  An antiparticle has the same quantum numbers, but the opposite charge. 
  \item There is no way to define a conserved (positive definite) probability for the KG equation, but the nonrelativistic limit of the KG equation is consistent with nonrelativistic quantum mechanics. 
\end{enumerate}

\begin{claim}
The continuity equation for the coupled KG equation is given by 
%
\begin{align}
\partial_{\mu } j^{\mu }_{EM} = 0
\,,
\end{align}
%
where 
%
\begin{align}
j^{\mu }_{EM} = \frac{1}{2} \qty(i \varphi^{*} \partial^{\mu } \varphi   - q A^{\mu } \varphi^{*} \varphi ) - \text{c. c.}
\,.
\end{align}
\end{claim}

\begin{proof}
Similarly to what was done before, we notice that 
%
\begin{align}
\varphi^{*} \qty(\DD^{\mu } \DD_{\mu } + M^2) \varphi = 0
\,
\end{align}
%
can be added to the negative of its complex conjugate to yield 
%
\begin{subequations}
\begin{align}
0&= \varphi^{*} \qty(\DD^{\mu } \DD_{\mu }) \varphi - \text{c. c. } \marginnote{Simplified mass terms, which were symmetric in \(\varphi^{*} \leftrightarrow \varphi \)}
\\
0&= \DD^{\mu } \qty(\varphi^{*} \DD_{\mu } \varphi) - \qty(\DD^{\mu } \varphi^{*}) \qty(\DD_{\mu } \varphi ) - \text{c. c. } \marginnote{Removed symmetric term in \(\varphi \leftrightarrow \varphi^{*}\)}\\ 
0&= \DD^{\mu } \qty(\varphi^{*} \DD_{\mu } \varphi) - \text{c. c. } \\ 
0&= \DD^{\mu } \qty( \varphi^{*} \DD_{\mu } \varphi - \text{c. c. }) \\ 
0&= \DD^{\mu } j_{\mu }
\,,
\end{align}
\end{subequations}
%
so our conserved current is 
%
\begin{subequations}
\begin{align}
j^{\mu } &= \varphi^{*} \DD^{\mu } \varphi - \text{c. c. } \\
j^{\mu } &= \varphi^{*} \partial^{\mu } \varphi 
+ iq A^{\mu } \varphi^{*} \varphi 
- \text{c. c. }  \\
&= \varphi^{*} \partial^{\mu} \varphi + i q A^{\mu } \varphi^{*} \varphi - \varphi \partial^{\mu }\varphi^{*} - i q A^{\mu } \varphi^{*} \varphi \\
&= \varphi^{*} \partial^{\mu} \varphi - \varphi \partial^{\mu }\varphi^{*} + 2i q A^{\mu } \varphi^{*} \varphi 
\,,
\end{align}
\end{subequations}
%
and conventionally we define \(j^{\mu }_{EM} = \frac{i}{2} j^{\mu }\), which is 
%
\begin{align}
j^{\mu }_{EM} = \frac{i}{2} \varphi^{*}\partial^{ \mu } \varphi 
-\frac{i}{2} \varphi\partial^{ \mu } \varphi^{*} + \frac{2}{2} i^2  q A^{\mu } \varphi^{*} \varphi 
\,.
\end{align} 
\end{proof}

\end{document}
