\documentclass[main.tex]{subfiles}
\begin{document}

\marginpar{Monday\\ 2020-3-16, \\ compiled \\ \today}

\section{The Klein-Gordon equation in the presence of an external electromagnetic field}

A complex scalar solution to the Klein-Gordon equation can represent a charged relativistic spin-0 particle. 
Since the particle is charged, it is interesting to study its interaction with the electromagnetic field.

We will discuss only the interaction of the particle with an externally determined field, that is, we will not discuss how the particle influences the field around it.

In order to describe the electric and magnetic fields in a covariant way we use the four-vector \(A_{\mu } = (A_0 , \vec{A})\), such that \(\vec{E} = - \vec{\nabla} A_0 - \partial_{0} \vec{A}\) and \(\vec{B} = \vec{\nabla} \times \vec{A}\), or, in covariant terms, 
%
\begin{align}
F^{ \mu \nu } = 2\partial^{[\mu } A^{\nu ]}
\,,
\end{align}
%
where the antisymmetric field-strength tensor \(F^{\mu \nu }\) encodes both the electric and magnetic fields. 

\begin{definition}
The \textbf{minimal coupling} to an external electromagnetic field is obtained by substituting in the wave equation the partial derivative \(\partial_{\mu }\) with the \emph{covariant derivative}
%
\begin{align}
\mathrm{D}_{\mu } \overset{\text{def}}{=} \partial_{\mu } + i q A_{\mu }
\,,
\end{align}
%
where \(q\) is the electric charge of the particle. 
\end{definition}

Inserting this, the coupled Klein-Gordon equation reads 
%
\begin{align}
\qty[\DD^{\mu } \DD_{\mu } + M^2 ] \varphi (x) = 0 
\,.
\end{align}

Recall that the momentum is defined as \(p_{\mu } = i \partial_{\mu }\), so this substitution is equivalent to changing the momentum to \(i (\partial_{\mu } +iqA_{\mu }) =  p_{\mu } - q A_{\mu }\).

This is the same minimal coupling ansatz which is used in nonrelativistic quantum mechanics: there, the minimally-coupled Schrödinger equation reads 
%
\begin{align}
E \psi = \qty[\frac{\qty(\vec{p} - q \vec{A})^2}{2M}  + q A_0 ] \psi 
\,,
\end{align}
%
in which we can see the two contributions: the three-momentum \(\vec{p}\) is exchanged for \(\vec{p} - q \vec{A}\), while the energy \(E\) is exchanged for \(E - q A_0 \) (and the term is brought on the other side of the equation for convenience). 

Let us make the terms in the minimally coupled Klein-Gordon equation explicit: 
%
\begin{align}
\qty[\qty(\partial^{\mu } +iqA^{\mu } \qty(\partial_{\mu } +i q A_{\mu }) + M^2)] \varphi &= 0  \\
\qty[\square + iq A^{\mu } \partial_{\mu } + iq \partial_{\mu } A^{\mu } - q^2A^2 + M^2] &=0
\,,
\end{align}
%


\end{document}
