\documentclass[main.tex]{subfiles}
\begin{document}

\marginpar{Monday\\ 2020-3-16, \\ compiled \\ \today}

Remember that the version of the slides which appears during the lesson might be updated. 

There might be some disruption in late March because the professor is having a child. 

\section{Linearized GR}

Let us assume that our metric tensor is almost flat: \(g_{\mu \nu } = \eta_{\mu \nu } + h_{\mu \nu }\), with \(\abs{h_{\mu \nu }} \ll 1\).

This is a coordinate dependent statement: however the physical situation is clear --- almost flat spacetime --- and the way we will proceed is to expand in \(h\).

Do note that \(g^{\mu \nu } = \eta^{\mu \nu } - h^{\mu \nu }\). 

Choosing a reference in which these components are small limits our gauge freedom: we will be able to do only transformations which preserve the condition. 

We will be able to do global Lorentz transformations: 
%
\begin{align}
g' = \Lambda \Lambda g
\,,
\end{align}
%
therefore 
%
\begin{align}
g' = \Lambda \Lambda \qty(\eta + h) = \eta + \Lambda \Lambda h
\,,
\end{align}
%
so the flat metric does not change, while \(h\) changes to  \(h' = \Lambda \Lambda h\). We are omitting indices, they are in the usual positions. 

A more general class of transformation is given by those which can be expressed as 
%
\begin{align}
x^{\mu } \rightarrow x^{\prime \mu } + \xi^{\mu }
\,,
\end{align}
%
so the Jacobian will look like 
%
\begin{align}
\delta^{ \mu }_{\nu } + \partial_{\nu } \xi^{ \mu }
\,,
\end{align}
%
and we ask that \(\partial \xi \) is small --- formally, first order in \(h_{\mu \nu }\). This yields 
%
\begin{align}
g^{\prime }_{\mu \nu } = \eta_{\mu \nu } + h_{\mu \nu } - 2 \partial_{(\mu  }\xi_{\nu )}
\,.
\end{align}
%
For an alternative reference on this derivation, see the notes on General Relativity in the course by Marco Peloso \cite[section 10]{tissinoGeneralRelativityNotes2020}.

Now, we wish to linearize the Riemann tensor: we start from the Christoffel symbols. We get 
%
\begin{subequations}
\begin{align}
\Gamma^{\sigma }_{\mu \nu }
&= \frac{1}{2} g^{\sigma \rho }\qty(2g_{\rho (\mu , \nu )} - g_{\mu \nu , \rho })  \\
&= \frac{1}{2} \qty(\partial_{\mu } h^{\sigma }_{\nu } + \partial_{\nu } h^{\sigma }_{\mu } + \partial^{\sigma } h_{\mu \nu })
\,,
\end{align}
\end{subequations}
%
and now in the Riemann tensor \(R = \partial \Gamma + \Gamma \Gamma \) the \(\Gamma \Gamma \) terms are second order in \(h\), so we ignore them. Then, we get the simplified expression 
%
\begin{align}
R^{\sigma }_{\mu \nu \rho } = \frac{1}{2} 
\qty(\partial_{\nu } \partial_{\mu } 
h^{\sigma }_{\rho } 
+ \partial_{\rho } \partial^{\sigma }h_{\mu \nu }
- \partial_{\nu } \partial^{\sigma } h_{\mu \rho }
- \partial_{\rho } \partial_{\mu } h^{\sigma }_{\nu })
\,,
\end{align}
%
so the Ricci tensor --- which we will  set to zero in the vacuum --- will be 
%
\begin{align}
R_{\mu \nu }
= R^{\sigma }_{\mu \nu \sigma }
= \frac{1}{2} 
\qty(\partial_{\nu } \partial_{\mu } h + \square h_{\mu \nu } 
- \partial_{\nu } \partial_{\sigma } h^{\sigma }_{\mu }
- \partial_{\sigma} \partial_{\mu } h^{\sigma }_{\nu })
\,,
\end{align}
%
and the Ricci scalar is 
%
\begin{align}
R = \eta^{\mu \nu } R_{\mu \nu } 
= \square h - \partial_{\nu }\partial_{\sigma } h^{\sigma \nu }
\,.
\end{align}

Now, the field equations read 
%
\begin{align}
R_{\mu \nu } - \frac{1}{2} g_{\mu \nu } R 
\approx \frac{1}{2} \qty[
  \partial_{\nu } \partial_{\mu } h + \square h_{\mu \nu } 
  - \partial_{\nu } \partial_{\sigma } h^{\sigma }_{\mu }
  - \partial_{\sigma} \partial_{\mu } h^{\sigma }_{\nu }
  - \eta_{\mu \nu }
  \qty(\square h - \partial_{\nu }\partial_{\sigma } h^{\sigma \nu })
] = - \frac{1}{M_P^2} T_{\mu \nu }
\,.
\end{align}

Now, we define the trace reverse perturbation as 
%
\begin{align}
\overline{h}_{\mu \nu } = h_{\mu \nu } - \frac{1}{2} \eta_{\mu \nu } h  
\,,
\end{align}
%
so the doubly-trace reverse is the perturbation itself, and \(\overline{h} = - h\). 

Now we shall use Lorenz gauge\footnote{This is not the same as Loren\emph{t}z, after whom Lorentz covariance is called.}

We shall use our gauge freedom. How does the trace-reverse perturbation transform? 
%
\begin{subequations}
\begin{align}
\overline{h}^{\prime  \mu \rho } &=  
h^{\mu \rho } - 2 \partial^{(\mu} \xi^{\rho )} - \frac{1}{2 } \eta^{\mu \rho } \qty(h - 2 \partial_{\sigma } \xi^{\sigma })
\\
&=
\overline{h}^{\mu \rho } - 2 \partial^{(\mu } \xi^{\rho )} + \eta^{\mu \rho } \partial_{\sigma } \xi^{\sigma }
\,.
\end{align}
\end{subequations}
%

The derivatives of the new and old perturbations differ by 
%
\begin{align}
\partial_{\rho } \overline{h}^{\prime \mu \rho }
- \partial_{\rho } \overline{h}^{\mu \rho } = - \square \xi^{\mu }
\,,
\end{align}
%
so we can choose a variable change \(\xi^{\mu } \) such that \(\partial_{\rho} \overline{h}^{\prime \mu \rho }\).

Then, the linearized field equations will read 
%
\begin{align}
\square \overline{h}_{\mu \nu } 
+ \eta_{\mu \nu } \partial_{\rho }\partial_{\sigma } h^{\rho \sigma } 
-  \partial_{\nu } \partial_{\rho } \overline{h}^{\rho  }_{\mu } - \partial_{\mu } \partial_{\rho } \overline{h}^{\rho }_{\nu } = -2 \frac{T_{\mu \nu }}{M_P^2}
\,,
\end{align}
%
which finally yields 
%
\begin{align}
\square \overline{h}_{\mu \nu } = - \frac{2 T_{\mu \nu }}{M_P^2}
\,.
\end{align}

\paragraph{Comments on the linearized equations}

Since we expanded in \(\eta_{\mu \nu }\), the quantities \(h_{\mu \nu }\) have a geometric meaning but we are treating them as 16 scalar fields. 

When we look at the geodesic equations, we get a prediction of the gravity having no effect on matter. 
We are treating gravity as a linear theory, so we have the superposition principle. 

We are ignoring the physical principle that ``gravity gravitates'': curvature of spacetime is associated to a SEMT in a nonlinear way. 

\paragraph{GWs in empty space}

We set \(T_{\mu  \nu }\) to zero, so we get 
%
\begin{align}
\square \overline{h}_{\mu \nu } = 0
\,,
\end{align}
%
which is the usual wave equation, with solutions \(\overline{h}_{\mu \nu } = A_{\mu \nu } e^{i k_{\lambda }x^{\lambda }}\). 

In general \(A_{\mu \nu } \) is symmetric, constant, complex. \(k_{\lambda }\) is constant and real, and we must have 
%
\begin{align}
\eta^{\rho \sigma } k_{\rho } k_{\sigma } A_{\mu \nu } e^{i k \cdot x } \implies k^2=0
\,.
\end{align}

In order to have these conditions, we must still have 
%
\begin{align}
\partial_{\mu } \overline{h}^{\mu \nu } 
= A^{\mu \nu } k_{\mu } e^{i k \cdot x } = 0 
\implies A^{\mu \nu }k_{\nu } = 0
\,.
\end{align}

The conjugate of the wave equation also holds, so after our manipulations we will always be able to take the real part. 

We still have gauge freedom: we can perform transformations if they satisfy \(\square \xi^{\mu }=0\).

We define 
%
\begin{align}
\xi^{\mu \nu } = \partial^{ \mu } \xi^{\rho } + \partial^{\rho } \xi^{\mu } - \eta^{\mu \rho } \partial_{\sigma } \xi^{\sigma }
\,,
\end{align}
%
and we will have \(\square \xi^{\mu \nu } =0 \)  since the Dalambertian commutes with the other derivatives.

So, if \(\overline{h}^{\mu \rho }\) satisties the Vacuum field equations, then \(\overline{h}^{\prime \mu \nu } = \overline{h}^{\mu \nu } - \xi^{\mu \nu }\) also does. 

So, we can use the 4 functions \(\xi^{\mu }\) to set 4 constraints on \(\overline{h}^{\mu \rho }\): 
%
\begin{subequations}
\begin{align}
\overline{h}^{0i}_{TT} &= 0  \\
 \overline{h}_{TT} &= 0
\,,
\end{align}
\end{subequations}
%
which conveniently means that \(\overline{h}_{\mu \nu } = h_{\mu \nu }\). This is called TT-gauge. 

Combined with the Lorenz gauge, which says 
%
\begin{align}
0= \partial_{\rho } \overline{h}^{0 \rho }_{TT} = \partial_{0} \overline{h}^{00}_{TT} = 0
\,,
\end{align}
%
which means that the metric element is constant, so we can rescale time in order to set it to zero. 

Also, the other Lorenz gauge conditions are 
%
\begin{align}
0= \partial_{\rho } \overline{h}^{j \rho }_{TT} = \partial_{i} \overline{h}^{ji}_{TT}
\,,
\end{align}
%
which means that, of the 6 potentially free components of the \(h^{ij}\), we actually have only 2 degrees of freedom.

Now, if we assume \(\vec{k} = k \hat{z}\) then we will have \(k^{\mu } = (k, 0, 0, k)\); also, in the matrix \(A_{\mu \nu }\) we will have \(A^{ij}k_{j} = 0\), so \(A^{i3}=0\). 

Then, in full generality under our gauge choices we shall have 
%
\begin{subequations}
\begin{align}
\overline{h}^{\mu \nu }_{TT} = \left[\begin{array}{cccc}
0 & 0 & 0 & 0 \\ 
0 & h_{+} & h_{ \times } & 0 \\ 
0 & h_{ \times } & - h_{+} & 0 \\ 
0 & 0 & 0 & 0
\end{array}\right]
e^{i k \qty(t -z)}
\,.
\end{align}
\end{subequations}

For a generic direction of propagation, we can define a projector onto the direction orthogonal to the direction of propagation \(k_{i}\): \(P_{ij} = \delta_{ij} - k_i k_j\), so we will have 
%
\begin{align}
A^{ij}_{TT} = \qty(P^{i}_{k} P^{j}_{l} - \frac{1}{2} P^{ij} P_{kl})A^{kl}
\,.
\end{align}

\end{document}
