\documentclass[main.tex]{subfiles}
\begin{document}

\marginpar{Friday\\ 2020-5-1, \\ compiled \\ \today}

We were discussing the basics of GW interferometry.

Moving to the TT gauge is a coordinate system in which the mirrors are free-falling. 
However, the light propagating through spacetime is affected by the fact that the spacetime is ``stretched'': for our photons,
%
\begin{align}
\dd{s^2} = - c^2 \dd{t^2} + \qty(1 + h_{+} (t)) \dd{x}^2 + \qty(1 - h_{+}(t))^2 \dd{y}^2 + \dd{z}^2 = 0
\,,
\end{align}
%
so in the \(x\) arm, depending on the direction of propagation we will have: 
%
\begin{align}
\dd{x} = \sqrt{ \frac{c^2 \dd{t^2}}{1 + h_{+}(t)}}
\approx \pm c \dd{t} \qty(1 - \frac{1}{2} h_{+}(t))
\,,
\end{align}
%
so if we integrate the whole round-trip of the photon we get, for the first leg: 
%
\begin{align}
L_{x} = \int_{0}^{L_x} \dd{x} = \int_{t_0 }^{t_1 } c \dd{t} = c (t_1 - t_0 ) 
- \frac{c}{2} \int_{t_0 }^{t_1 } \dd{t} h_{+}(t) 
\,,
\end{align}
%
and for the second leg: 
%
\begin{align}
- L_x = \int_{L_x}^{0} = - \int_{t_1 }^{t_2 } c \dd{t} \qty(1 - \frac{h_{+}(t)}{2}) = - c (t_2 - t_1 ) + \frac{c}{2} \int_{t_1 }^{t_2 } \dd{t} h_{+}(t) 
\,,
\end{align}
%
so the time difference is given by 
%
\begin{align}
t_2 - t_0 = \frac{2L_x}{c} + \frac{1}{2} \int_{t_0 }^{t_2} \dd{t} h_{+}(t) 
\,,
\end{align}
%
where we can consider \(t_2 \approx t_0  + 2 L_x  /c\) in the integration bound, since the correction would be second order. 

To first order, then, if the GW is sinusoidal we have for the \(x\) arm:
%
\begin{align}
t_2 - t_0 = \frac{2L_x}{c} + \frac{L_x}{c} h_{+} \qty(t_0 + \frac{L_x}{c}) \frac{\sin(\frac{\omega_{gw} L_x}{c})}{\frac{\omega_{gw}L_x}{c}}
\,,
\end{align}
%
while for the \(y\) arm the sign is inverted. 

The argument is basically the ratio of the length of our arm to the wavelength of the GW. We have a maximum of the effect, then, when the GW perturbation is effectively static during the time in which we are detecting. 

Keep in mind, though, that the effect also scales with \(L_x\): a very small detector wouldn't work. 

If the length, on the other hand, is very large the perturbation cancels out during the flight of the photon.  
If we fix the frequency, making the detector longer and longer does not help: further full oscillations of the path length do not have any effect.

The interesting thing is the phase difference: if in both arms the light reaches the detector at \(t_2 \) we have 
%
\begin{subequations}
\begin{align}
\Delta \phi 
&= \omega_{l} \qty(t_0^{x }- t_0^{y}) \\
&= \omega_{l} \qty(2 \frac{L_x -L_y}{c} + \frac{2L}{c} \operatorname{sinc}(\frac{\omega_{gw} L}{c} ) h_0 \cos(\omega_{gw} t + \alpha )) = \Delta \phi_0 + \Delta \phi_{gw}
\,.
\end{align}
\end{subequations}

So, we have a tradeoff: we want to stay before the first zero of the sinc, but we also want to have a relatively large detector.

So, our optimal length is of the order of a quarter of the wavelength.

This is the same as saying we want to keep the photon in-flight for a quarter of the period of the GW. 

This would mean, for a frequency of \SI{100}{Hz}, that we would need a detector of around \SI{750}{km}.

If we do the computation accounting for the oscillation of the laser light, we get that the field out of the BS is 
%
\begin{align}
E = \frac{E_0}{2} e^{-i \gamma } \qty(e^{-i \omega_{l} t}
+ \beta e^{-i \alpha } e^{-i (\omega_{l} - \omega_{gw})} + \beta e^{i \alpha } e^{-i (\omega_{l} + \omega_{gw}) t})
\,,
\end{align}
%
which is the same of a single field which is modulated in amplitude.

\subsection{Lasers and cavities}

The best mirrors in the world are built for GW detectors.
These are dielectric mirrors.
We stack dielectric interfaces on top of each other, so that the wave coming back has constructive interference while the wave being transmitted interferes destructively with the next layer.

A \textbf{cavity} is an arrangement of mirrors such that we have a closed path for light.

Mirrors are symmetric, if we can input some light then we are also losing the same amount.

We have an incident field \(E _{\text{in}}\), a circulating field \(E_{c}\) and a transmitted field \(E_{t}\). The ``in'' mirror is labelled 1, the ``out'' mirror is labelled 2. 

So, we have 
%
\begin{align}
E_{c} = t_1 E _{\text{in}} + r_1 r_2 E_{c} e^{-ik 2L} 
\,,
\end{align}
%
so 
%
\begin{align}
E_{c} = E _{\text{in}} \frac{t_1 }{1 - r_1 r_2 e^{-ik2L}}
\,.
\end{align}

Here \(k\) is the wavevector of the electric field. The exponential is called the \emph{round-trip gain}. 

The reflected field is given by 
%
\begin{subequations}
\begin{align}
E_{r} &= -r_1 E _{\text{in}} + r_2 t_1 E_{c} e^{-ik2L}  \\
&= -E _{\text{in}} \frac{r_1 - r_2 e^{-ik 2L}}{1 - r_1 r_2 e^{-ik2L}}
\,.
\end{align}
\end{subequations}

On the other hand, the transmitted field is 
%
\begin{align}
E_{t} = t_2 E_c 
\,.
\end{align}

The circulating intensity can be calculated by the square modulus of the field, we then get 
%
\begin{align}
I_{c} = E^2 _{\text{in}} \abs{\frac{t_1 }{1 - r_1 r_2 e^{-ik2L}}}^2
\,,
\end{align}
%
so we want to minimize the denominator: this means we want \(L = n \pi / k\). 

As we increase reflectivity the peaks get narrower and higher. 

The distance in frequency between the peaks, \(c / 2L\), is called the \emph{free spectral range}. 

The \emph{finesse} is defined as the free spectral range divided by the FWHM of the peaks, 
%
\begin{align}
\mathcal{F} = \frac{\pi \sqrt{r_1 r_2 }}{1 - r_1 r_2 }
\,.
\end{align}

This is \(2 \pi  / \text{losses}\).

Using this, we can estimate the storage time of a photon inside the cavity: 
%
\begin{align}
\tau_{s} \approx \frac{L \mathcal{F}}{c \pi }
\,.
\end{align}

\end{document}
