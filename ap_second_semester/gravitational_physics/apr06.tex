\documentclass[main.tex]{subfiles}
\begin{document}

\subsubsection{Rømer delay (observer)}

\marginpar{Monday\\ 2020-4-6, \\ compiled \\ \today}

% We are discussing the Hulse-Taylor pulsar, which is a very precise clock.
% There are many effects by which the time-of-arrival is shifted, if we take them all out we can get the effect from the source. 

This effect is denoted as \(\Delta_{R, {\odot}}\), it is due to the fact that if Earth is on the far side of the Sun the radio signal takes longer to get to it than it would if it were on the near side to the pulsar. The simplest way to model it is  
%
\begin{align}
\Delta_{R, {\odot}} = t_0 \cos(\Omega t - \lambda ) \cos(\beta )
\,,
\end{align}
%
where \(t_0 = R / c\), \(R\) being the radius of the Earth's orbit (here taken to be circular), while \(\lambda \) and \(\beta \) define the position of the source in the orbital plane. 

Proper modelling would require us to also take into account the ellipticity of the orbit and the Earth's rotation. 
In general, the effect can be written as 
%
\begin{align}
\Delta_{R, {\odot}} = - \frac{\vec{r} _{\text{ob}} \cdot \hat{n}}{c}
\,,
\end{align}
%
where \(\hat{n}\) denotes the direction of the source in barycentric coordinates for the solar system, while \(\vec{r} _{\text{ob}}\) is the vector connecting the barycenter of the Solar System to the observer, which can be decomposed as a sum of vectors to the Sun's center, Earth's center and finally to the observer:
%
\begin{align}
\vec{r} _{\text{ob}} = \vec{r}_{\text{ob}, \oplus} + \vec{r}_{\oplus, \odot} + \vec{r}_{\odot, \text{barycenter}}
\,.
\end{align}

Modeling this allows us get information about \(\lambda \) and \(\beta \), which define \(\hat{n}\). 
The \textbf{effect size} is dominated by the Earth's motion around the Sun, and it has a magnitude of about \(\pm \SI{e3}{s}\) in the span of six months.

\subsubsection{Shapiro delay (observer)}

% We will have the Shapiro delay from the Sun, the Einstein delay at the receiver from the curvature of the spacetime around the Earth, which is given by 

This effect is due to the curvature introduced by the gravitational potential of the Sun. 
The Schwarzschild metric in Cartesian coordinates reads: 
%
\begin{align}
\dd{s^2} &= - \qty(1 + 2 \phi (x)) c^2 \dd{t^2} + \frac{1}{\qty(1 + 2 \phi (x))} \dd{\vec{x}}^2  \\
&\approx - \qty(1 + 2 \phi (x)) c^2 \dd{t^2} + \qty(1 - 2 \phi (x)) \dd{\vec{x}}^2
\,,
\end{align}
%
where \(\phi (x) = -GM / \abs{x}c^2 \ll 1\) is the gravitational potential in the weak-field approximation.
Photons travel along null geodesics, so they have \(\dd{s}^2 = 0\): this means that 
%
\begin{align}
c\dd{t} = \pm\sqrt{\frac{1 - 2 \phi }{1 + 2 \phi }} \abs{\dd{x}}
\approx \pm \qty(1 - 2 \phi ) \abs{ \dd{x}} 
\,,
\end{align}
%
so we can compute the Shapiro delay as the correction to the total travel time:
%
\begin{align}
c \qty(t _{\text{obs}} - t _{\text{emit}}) 
&= \int_{r _{\text{obs}}}^{t _{\text{emit}}} \abs{ \dd{x}} \abs{1 - 2 \phi (r)} 
\,,
\end{align}
%
which automatically incorporates the Rømer delay, since we find 
%
\begin{align}
t _{\text{obs}} - t _{\text{emit}} =  \underbrace{\frac{\abs{\vec{r} _{\text{emit}} - \vec{r} _{\text{bary}}}}{c}}_{\text{simple propagation}} + \underbrace{\frac{\vec{r} _{\text{obs}} \cdot \hat{n}}{c}}_{\text{Rømer delay}}
\underbrace{- \frac{2}{c} \int_{r _{\text{obs}}}^{r _{\text{emit}}} \abs{ \dd{x}} \phi (x)}_{\text{Shapiro delay: } \Delta_{S, \odot}}
\,.
\end{align}
%
\todo[inline]{The sign of the Rømer effect is different here!}

Let us explicitly compute \(\Delta_{S, \odot}\): we consider a reference frame in which the Sun is at the center, the Earth is on the \(x\) axis at a distance \(r_{\oplus \odot}\), while the pulsar is at a distance \(\rho \) from the Earth and at an angle \(\theta \) with respect to the Sun-Earth axis: so, the Sun-pulsar distance \(r\) is given by 
%
\begin{align}
r^2 = \qty(r_{\oplus \odot} + \rho \cos \theta )^2 + \qty(\rho \sin \theta )^2
= r_{\oplus \odot }^2 \qty(u^2 + 2 u \cos \theta + 1)
\marginnote{Used \(\cos^2\theta+ \sin^2\theta = 1\).}
\,,
\end{align}
%
where \(u = \rho / r_{\oplus \odot}\).
The delay is given by
%
\begin{align}
\Delta_{S, \odot} &= - \frac{2}{c} \int_{r _{\text{obs}}}^{r _{\text{emit}}} \abs{ \dd{x}} \phi (x)
= \frac{2}{c} \int_{0}^{d} \dd{\rho } \frac{G M_{\odot}}{ c^2 r}  \\
&= \frac{2 G M_{\odot}}{c^3} \int_{0}^{d / r_{\oplus \odot} }
\dd{u} \frac{r_{\oplus \odot}}{r}   \\
&= \frac{R_{S, \odot}}{c} \int_{0}^{d / r_{\oplus \odot} }
\dd{u} \frac{1}{\sqrt{u^2 + 2 u \cos \theta  +1}}
\,,
\end{align}
%
where \(R_{S, \odot}\) is the Sun's Schwarzschild radius, while \(d\) is the distance from the Earth to the pulsar.
We used the fact that \(\abs{ \dd{x}} = \dd{ \rho }\): the differential \(\abs{\dd{x}}\) represents the \emph{coordinate distance} along the path from the Earth to the pulsar, which is linearly parametrized by \(\rho \); \(r\) on the other hand can be used to parametrize the curve but that would be a \emph{nonlinear} parametrization, not useful for our purposes.

We need to estimate this integral; it is a divergent one but the divergence is only logarithmic. It is reasonable to estimate it assuming \(d/r_{\oplus \odot} \gg 1\) --- realistically this is of the order of \num{e9}, since the distance to the pulsar is of the order \SI{5}{kpc} \cite[pag.\ L53]{hulseDiscoveryPulsarBinary1975}.


So, we add and subtract \(1/ \sqrt{u^2 + 1}\) in the integrand: then we find 
%
\begin{align}
\Delta_{S, \odot} = \frac{R_{S, \odot}}{c} \qty[ 
    \int_{0}^{d/r_{\oplus \odot}} \dd{u} \frac{1}{\sqrt{u^2+ 1}}
    +
    \int_{0}^{d/r_{\oplus \odot}} \dd{u} 
    \frac{1}{\sqrt{u^2 + 2 u \cos \theta +1}}
    -\frac{1}{\sqrt{u^2+ 1}}
]
\,,
\end{align}
%
where the first integral is a hyperbolic arcsine, which can be estimated by \(\log (2 d/ r_{\oplus \odot})\); the second term can be estimated with the following 
\begin{claim}
\begin{align}
\int_{0}^{ \infty } \dd{u} \qty[
\frac{1}{\sqrt{u^2 + 2 u \cos \theta +1}}
-\frac{1}{\sqrt{u^2+ 1}}]
= - \log \qty(1 + \cos \theta )
\,.
\end{align}
\end{claim}

\begin{proof}
Mathematica says so.
\end{proof}

Then finally our estimate is 
%
\boxalign{
\begin{align}
\Delta_{S, \odot} = \frac{R_{S, \odot}}{c} \qty(\log ( \frac{2d}{r_{\oplus \odot}}) - \log (1 + \cos \theta ) )
\,,
\end{align}}
%
where \(\theta \) varies seasonally, as the Earth moves around the Sun, while \(d\) is basically constant. 

It might seem that this can diverge as \(\theta \to \pi \), but this is not actually the case: that would correspond to the radiation coming through the center of the Sun, which it cannot. Actually, there is a range of values of \(\theta \) around \(\pi \) for which the radiation cannot reach us since the Sun is in the way. The maximum value of \(\theta \) that can be reached, as the radiation is tangent to the surface of the Sun, is called the grazing angle \(\theta_{g}\). 
This angle can be estimated by \(\theta_{g} \approx \pi - R_{\odot} / r_{\oplus \odot}\). 

The delay is maximal when the radiation grazes the Sun, and minimal when it is coming from the opposite direction as the Sun: \(\theta = 0\).
So, the \textbf{maximum size of the effect} is given by the difference 
%
\begin{align}
\Delta_{S, \odot} (\theta = \theta_{g}) - 
\Delta_{S, \odot} (\theta = 0) 
&= \frac{R_{S, \odot}}{c} \qty[
    \log( \frac{2d}{r_{\oplus \odot}} )
    - \log (1 + \theta_{g})
    -\log( \frac{2d - 2 r_{\oplus \odot}}{r_{\oplus \odot}})
] \marginnote{\(\log 1 = 0\).}  \\
&\approx - \frac{R_{S, \odot}}{c}
\log (1 + \cos(\pi - \frac{R_{\odot}}{r_{\oplus \odot}}))  \\
&\approx - \frac{R_{S, \odot}}{c}
\log (1 - 1 + \frac{1}{2}\qty(\frac{R_{\odot}}{r_{\oplus \odot}})^2)  \\
&\approx - \frac{R_{S, \odot}}{c}
\log (\frac{1}{2}\qty(\frac{R_{\odot}}{r_{\oplus \odot}})^2)
\approx + \SI{56}{\micro s}
\,,
\end{align}
%
where we discarded the difference of the two logarithms with \(d\) since the relative difference between their arguments is on the order \num{e-9}.
\todo[inline]{In the slides this is reported as being on the scale of \SI{100}{\micro s} with the logarithm's argument being \(2 r_{\oplus \odot} / d\): I do not see how that could come about, and a dependence on \(d\) seems not to make sense physically.}  

\todo[inline]{Also, the slide is repeated twice.}

\subsubsection{Einstein delay (observer)}

This effect is about the 

%
\begin{align}
\dv{\tau }{t} \approx 1 + \phi \qty(x _{\text{obs}}) - \frac{v^2 _{\text{obs}}}{2c^2}
\,,
\end{align}
%
so 
%
\begin{align}
\tau \approx t + \int^{t} \dd{\widetilde{t}} \qty(\phi \qty(x _{\text{obs}}) - \frac{v^2 _{\text{obs}}}{2c^2})
= t - \Delta_{\oplus, \odot}
\,.
\end{align}

If most of the velocity is due to the motion of the Earth in its elliptic orbit, we have from conservation of energy
%
\begin{align}
\frac{v^2 _{\text{obs}}}{2} = \frac{GM}{r} - \frac{GM}{2a}
\,,
\end{align}
%
so that we find 
%
\begin{align}
\dv{ \Delta}{t} \approx \frac{v^2}{2c^2} - \phi = \frac{GM_{\odot}}{c^2} \qty( \frac{1}{r} - \frac{1}{2a} - \frac{1}{r})
= R_{\text{earth-Sun}} \qty( \frac{1}{r} - \frac{1}{4a})
\,.
\end{align}

However, we must also consider the group velocity of the signal which travels through the ISM, which is ionized gas. Then, we get a delay which depends on the frequency: 
%
\begin{align}
t_L = \frac{L}{c} + \frac{e^2}{2 \pi m_e c} \frac{1}{\nu^2} DM
\,,
\end{align}
%
where \(DM\) is the Dispersion Measurement. 
For the HT pulsar, this spreads the time over a \SI{4}{MHz} bandwidth: 
however, we can measure precisely in the spectral domain, and we can ``connect the dots'' to find what corresponds to a single pulse. 

This allows us to reconstruct the original pulse.

Taking these effects out, we get 
%
\begin{align}
t _{\text{ssb}} = \tau - \frac{D}{\nu^2} + \Delta_{E, {\odot}} - \Delta_{S, {\odot}} + \Delta_{R, {\odot}}
\,.
\end{align}

This is ``time in solar-system barycenter''.
where
%
\begin{align}
D = \frac{e^2}{2 \pi m_e c} DM
\,.
\end{align}

We need to look at the gravitational time delay at the source: there is a contribution from the gravitational field of the NS itself, which is hard to calculate but constant, so we do not worry about it. 
The combined gravitational field, instead, is time-varying: so, we find 
%
\begin{subequations}
\begin{align}
\dv{T}{t} &= 1 - \frac{G m_c}{c^2 \abs{x_p - x_c}} - \frac{v_p^2}{2 c^2}  \\
 \dv{T}{u} &\approx \frac{P_b}{2 \pi } \qty(1 - \frac{G}{c^2} \frac{2 m_c m_p + 3 m_c^2}{2 a (m_p + m_c)})\qty(1 - e \cos u \qty(1 + \frac{G}{c^2})) \dots
\,,
\end{align}
\end{subequations}
%
[to finish]
where \(u\) is the angular parameter describing the orbit, while \(e\) is the eccentricity.

Also, we have the Romer delay: \(\Delta_{R} = \hat{z} \cdot x_{pb} / c\). 

The coordinates, for a Keplerian orbit, are 
%
\begin{align}
r_{pb} = r_1 = a_1 \qty(1 - e \cos u )
\qquad \text{and} \qquad
\cos \psi = \frac{\cos u - e}{1- e \cos u}
\,.
\end{align}

The Romer delay then is 
%
\begin{align}
\Delta_{R} = r_1 \sin \iota \sin(\omega+\psi ) 
= r_1 \sin \iota \qty(\cos \psi \sin \omega + \cos \omega \sin \psi )
\,,
\end{align}
%
where \(\iota \) is the observation angle, while \(\psi \) is the angle from the line of notes (see drawing). 

The relativistic effect, however, is large. 
We will not do the calculation, we find 
%
\begin{align}
\Delta_{R} = a_1 \sin \iota 
\qty((\cos u - e_r ) \sin \omega + \sqrt{1 - e^2_{\theta }} \sin u \cos \omega )
\,, 
\end{align}
%
where 
%
\begin{subequations}
\begin{align}
e_{r, \theta } &= e (1 + \delta_{r, \theta })  \\
\delta_{r} &= \frac{G}{c^2} \frac{3 m_p^2 + 6 m_p m_c + 2 m_c^2}{a (m_p + m_c)}  \\
\delta_{\theta } &= 
\,,
\end{align}
\end{subequations}
%
also here the advance of the periastron is much more significant than it is for Mercury. 

The Shapiro delay at the source must also be accounted for. 

If we get all the Keplerian parameters and two of the post-Newtonian ones then we should know everything.

We measure \(P_b, T_0, x= a \sin \iota /c, e, \omega \) and the post-Newtonian \(\dot{\omega}, \gamma \) and finally we make a prediction for \(\dot{P}\). This matches the data very well.

\section{GW from a rotating rigid body}

The moment of inertia tensor can be defined as 
%
\begin{align}
I^{ij} = \int \dd[3]{x} \rho (x) \qty(r^2 \delta^{ij} - x^{i} x^{j})
\,.
\end{align}

There exists a frame in which this tensor is diagonal, its eigenvalues are the moments of inertia, its eigenvectors are the axes of inertia.
They are then defined by equations like 
%
\begin{align}
I_1 = \int \dd[3]{x} \rho \qty(x_2^2 + x_3^2) 
\,.
\end{align}
 
For an ellipsoid with axes \(a, b, c\) and mass \(m\) we have 
%
\begin{align}
I_1 = \frac{m}{5} \qty(b^2 +c^2)
\,.
\end{align}

The rotational kinetic energy is given by 
%
\begin{subequations}
\begin{align}
E _{\text{rot}} &= \frac{1}{2} I_{ij} \omega_{i} \omega_{j} \\ 
&= \frac{1}{2} I_i \omega_{i}^2
\,,
\end{align}
\end{subequations}
%
where the last equality holds in the body frame. 

Suppose we have a body spinning around an axis, such that the position of any point shifts by a rotation matrix \(R_{ij}\). 

The inertia tensor shifts by \(I \rightarrow R^{\top} I R\).

The tensor we defined before, 
%
\begin{align}
M^{ij} = \frac{1}{c^2} \int \dd[3]{x} T^{00} (x) x^{i} x^{j} 
\approx - I^{ij} + \int \dd[3]{x} \rho (x) r^2 \delta^{ij}
\,,
\end{align}
%
which we can substitute in. This is the trace of the inertia tensor. 

Suppose we had a body whose angular momentum is not aligned with the moment of inertia: we can use Euler angles to express the rotation matrix.


\end{document}
