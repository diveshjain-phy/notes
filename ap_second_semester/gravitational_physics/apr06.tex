\documentclass[main.tex]{subfiles}
\begin{document}

\marginpar{Monday\\ 2020-4-6, \\ compiled \\ \today}

We are discussing the Hulse-Taylor pulsar, which is a very precise clock.
There are many effects by which the time-of-arrival is shifted, if we take them all out we can get the effect from the source. 

We will have the Shapiro delay from the Sun, the Einstein delay at the receiver from the curvature of the spacetime around the Earth, which is given by 
%
\begin{align}
\dv{\tau }{t} \approx 1 + \phi \qty(x _{\text{obs}}) - \frac{v^2 _{\text{obs}}}{2c^2}
\,,
\end{align}
%
so 
%
\begin{align}
\tau \approx t + \int^{t} \dd{\widetilde{t}} \qty(\phi \qty(x _{\text{obs}}) - \frac{v^2 _{\text{obs}}}{2c^2})
= t - \Delta_{\oplus, \odot}
\,.
\end{align}

If most of the velocity is due to the motion of the Earth in its elliptic orbit, we have from conservation of energy
%
\begin{align}
\frac{v^2 _{\text{obs}}}{2} = \frac{GM}{r} - \frac{GM}{2a}
\,,
\end{align}
%
so that we find 
%
\begin{align}
\dv{ \Delta}{t} \approx \frac{v^2}{2c^2} - \phi = \frac{GM_{\odot}}{c^2} \qty( \frac{1}{r} - \frac{1}{2a} - \frac{1}{r})
= R_{\text{earth-Sun}} \qty( \frac{1}{r} - \frac{1}{4a})
\,.
\end{align}

However, we must also consider the group velocity of the signal which travels through the ISM, which is ionized gas. Then, we get a delay which depends on the frequency: 
%
\begin{align}
t_L = \frac{L}{c} + \frac{e^2}{2 \pi m_e c} \frac{1}{\nu^2} DM
\,,
\end{align}
%
where \(DM\) is the Dispersion Measurement. 
For the HT pulsar, this spreads the time over a \SI{4}{MHz} bandwidth: 
however, we can measure precisely in the spectral domain, and we can ``connect the dots'' to find what corresponds to a single pulse. 

This allows us to reconstruct the original pulse.

Taking these effects out, we get 
%
\begin{align}
t _{\text{ssb}} = \tau - \frac{D}{\nu^2} + \Delta_{E, {\odot}} - \Delta_{S, {\odot}} + \Delta_{R, {\odot}}
\,.
\end{align}

This is ``time in solar-system barycenter''.
where
%
\begin{align}
D = \frac{e^2}{2 \pi m_e c} DM
\,.
\end{align}

We need to look at the gravitational time delay at the source: there is a contribution from the gravitational field of the NS itself, which is hard to calculate but constant, so we do not worry about it. 
The combined gravitational field, instead, is time-varying: so, we find 
%
\begin{subequations}
\begin{align}
\dv{T}{t} &= 1 - \frac{G m_c}{c^2 \abs{x_p - x_c}} - \frac{v_p^2}{2 c^2}  \\
 \dv{T}{u} &\approx \frac{P_b}{2 \pi } \qty(1 - \frac{G}{c^2} \frac{2 m_c m_p + 3 m_c^2}{2 a (m_p + m_c)})\qty(1 - e \cos u \qty(1 + \frac{G}{c^2})) \dots
\,,
\end{align}
\end{subequations}
%
[to finish]
where \(u\) is the angular parameter describing the orbit, while \(e\) is the eccentricity.

Also, we have the Romer delay: \(\Delta_{R} = \hat{z} \cdot x_{pb} / c\). 

The coordinates, for a Keplerian orbit, are 
%
\begin{align}
r_{pb} = r_1 = a_1 \qty(1 - e \cos u )
\qquad \text{and} \qquad
\cos \psi = \frac{\cos u - e}{1- e \cos u}
\,.
\end{align}

The Romer delay then is 
%
\begin{align}
\Delta_{R} = r_1 \sin \iota \sin(\omega+\psi ) 
= r_1 \sin \iota \qty(\cos \psi \sin \omega + \cos \omega \sin \psi )
\,,
\end{align}
%
where \(\iota \) is the observation angle, while \(\psi \) is the angle from the line of notes (see drawing). 

The relativistic effect, however, is large. 
We will not do the calculation, we find 
%
\begin{align}
\Delta_{R} = a_1 \sin \iota 
\qty((\cos u - e_r ) \sin \omega + \sqrt{1 - e^2_{\theta }} \sin u \cos \omega )
\,, 
\end{align}
%
where 
%
\begin{subequations}
\begin{align}
e_{r, \theta } &= e (1 + \delta_{r, \theta })  \\
\delta_{r} &= \frac{G}{c^2} \frac{3 m_p^2 + 6 m_p m_c + 2 m_c^2}{a (m_p + m_c)}  \\
\delta_{\theta } &= 
\,,
\end{align}
\end{subequations}
%
also here the advance of the periastron is much more significant than it is for Mercury. 

The Shapiro delay at the source must also be accounted for. 

If we get all the Keplerian parameters and two of the post-Newtonian ones then we should know everything.

We measure \(P_b, T_0, x= a \sin \iota /c, e, \omega \) and the post-Newtonian \(\dot{\omega}, \gamma \) and finally we make a prediction for \(\dot{P}\). This matches the data very well.

\section{GW from a rotating rigid body}

The moment of inertia tensor can be defined as 
%
\begin{align}
I^{ij} = \int \dd[3]{x} \rho (x) \qty(r^2 \delta^{ij} - x^{i} x^{j})
\,.
\end{align}

There exists a frame in which this tensor is diagonal, its eigenvalues are the moments of inertia, its eigenvectors are the axes of inertia.
They are then defined by equations like 
%
\begin{align}
I_1 = \int \dd[3]{x} \rho \qty(x_2^2 + x_3^2) 
\,.
\end{align}
 
For an ellipsoid with axes \(a, b, c\) and mass \(m\) we have 
%
\begin{align}
I_1 = \frac{m}{5} \qty(b^2 +c^2)
\,.
\end{align}

The rotational kinetic energy is given by 
%
\begin{subequations}
\begin{align}
E _{\text{rot}} &= \frac{1}{2} I_{ij} \omega_{i} \omega_{j} \\ 
&= \frac{1}{2} I_i \omega_{i}^2
\,,
\end{align}
\end{subequations}
%
where the last equality holds in the body frame. 

Suppose we have a body spinning around an axis, such that the position of any point shifts by a rotation matrix \(R_{ij}\). 

The inertia tensor shifts by \(I \rightarrow R^{\top} I R\).

The tensor we defined before, 
%
\begin{align}
M^{ij} = \frac{1}{c^2} \int \dd[3]{x} T^{00} (x) x^{i} x^{j} 
\approx - I^{ij} + \int \dd[3]{x} \rho (x) r^2 \delta^{ij}
\,,
\end{align}
%
which we can substitute in. This is the trace of the inertia tensor. 

Suppose we had a body whose angular momentum is not aligned with the moment of inertia: we can use Euler angles to express the rotation matrix.


\end{document}
