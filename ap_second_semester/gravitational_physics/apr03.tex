\documentclass[main.tex]{subfiles}
\begin{document}

\marginpar{Friday\\ 2020-4-3, \\ compiled \\ \today}

As we were discussing yesterday, we can approximate every orbit as a circular one. 

We finally get the relation 
%
\begin{align}
\dot{\omega}_{GW} = \frac{12}{5} 2^{1/3}
\qty(\frac{M_c G}{c^{3}})^{5/3} \omega_{GW}^{11/3}
\,,
\end{align}
%
where \(M_c\) is the chirp mass. 
So, we can integrate this: since 
%
\begin{align}
\dv{f_{GW}}{t} = k f^{11/3}_{GW} \implies - \frac{3}{8k} f^{-8/3} = t - t _{\text{coalescence}} 
\overset{\text{def}}{=} -\tau 
\,,
\end{align}
%
so that we can get the frequency as a function of the time until coalescence: 
%
\begin{align}
\tau = \frac{5}{256} \qty( \frac{1}{\pi f_{GW}})^{8/3} \dots
\,.
\end{align}

We can also get an expression for the radius at a given time from coalescence: 
%
\begin{align}
R(\tau ) = R_0 \qty(\frac{\tau }{\tau_0 })^{1/4}
\,,
\end{align}
%
where \(\tau_0 \) is the time to coalescence at \(t_0 \). 
If we plot this, it has a ``plunge'' phase, and up to it we can trust our plot. 

\subsection{Chirping waveform}

The \emph{phase} is the argument of the cosine, so we write \(\cos(\phi (t))\). 
The angular frequency is given by \(\omega_{GW} = \phi'\).

The chirping waveform cannot be trusted near the end. 
We know that if 
%
\begin{align}
R < R _{\text{ISCO}} = \frac{6GM}{c^2}
\,
\end{align}
%
then orbits are not stable. This formula only holds for extreme mass ratios (we actually could have these SMBHs merging with solar mass ones!). 
Anyway, we use it as a guideline to see when our approximations break down. 

The shape of the chirping waveform is basically correct; it goes out of phase with the numerical relativity calculation, but it works somewhat. 

\todo[inline]{Numerical relativity has a \emph{lower} frequency than the quadrupole approx!}

What about eccentric binaries? 
We can also analyze them. The formula is 
%
\begin{align}
\dv{E}{t} = \frac{32}{5} \frac{G \mu^2}{c^{5}} a^{4} \omega_0^{6} f(e)
\,,
\end{align}
%
where 
%
\begin{align}
f(e) = \frac{1}{(1 - e^2)^{7/2}} \qty(1 + \frac{73}{24} e^2 + \frac{37}{96} e^{4}) \geq 1 
\,.
\end{align}

We have emission at frequencies other than the orbital one; also, the GW emission has the effect of circularizing the orbit. 
So, we usually observe circular systems. 

\section{Hulse-Taylor binaries}

It is debatable whether the observation of this was the first observation of gravitational waves. 

This is a binary system in which one star is a pulsar. 

What is a pulsar? It's a kind of neutron star.
Not a moral judgement, but you are completely empty.

A pulsar has a large magnetic field; at a distance \(r_c = c/\omega \) the field lines cannot close so a radio beam escapes. 
This provides a clock!

``Taking the pulse'' of a pulsar: they usually have a certain well-defined shape, if we average over a few periods. 

The procedure is: take signal, FFT to get the fundamental, average over periods. 

The period can then be measured precisely, and we can observe its variations. 

Some relevant frequencies: the radio waves are on the order of \SI{e8}{Hz}, the pulsar's frequency is of the order \SI{10}{Hz}, the frequency of the binary period is \SI{e-5}{Hz}, the motion of the Earth around the Sun at \SI{e-8}{Hz} is also relevant. 

Since the pulsar frequency is very small, we can still average many pulses and still be measuring at what is basically ``a single point''. 



\end{document}
