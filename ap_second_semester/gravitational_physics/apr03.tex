\documentclass[main.tex]{subfiles}
\begin{document}

\subsection{Frequency evolution and time to coalesce}

\marginpar{Friday\\ 2020-4-3, \\ compiled \\ \today}

% As we were discussing yesterday, we can approximate every orbit as a circular one. 
From the equation for the evolution of the binary's radius \eqref{eq:radius-evolution-binary}, its explicit expression in terms of \(\omega_{s}\) \eqref{eq:derivative-radius-evolution-binary} and the expression for the gravitational wave emitted power \eqref{eq:gravitational-wave-total-power-emitted} we can write 
%
\begin{align}
- \frac{2}{3} \frac{\dot{\omega}_{s}}{\omega_{s}} R &= - \frac{R^2}{GM \mu } \dot{E}_{GW} \\
\dot{\omega}_{s} &= \frac{3}{\mu } \qty(GM)^{-2/3} \omega_{s}^{1/3} 
\frac{32}{5} \frac{c^{5}}{G} 
\qty( \frac{G M_c \omega_{GW}}{2 c^{3}})^{10/3} \marginnote{Substituted in \(\dot{E}\), used \(R = (GM)^{1/3} \omega_{s}^{-2/3} \).}  \\
&= \frac{6}{5} \sqrt[3]{2} \qty(\frac{M_c G}{c^{3}})^{5/3} \omega_{GW}^{11/3}
\,.
\end{align}

If we substitute in \(\omega_{GW}\) from \(\omega_{s}\)
we finally get the relation 
%
\begin{align}
\dot{\omega}_{GW} = \frac{12}{5} \sqrt[3]{2}
\qty(\frac{M_c G}{c^{3}})^{5/3} \omega_{GW}^{11/3}
\qquad \text{or} \qquad
\dot{f}_{GW} = \underbrace{\frac{96}{5} \pi^{8/3} \qty(\frac{M_c G}{c^{3}})^{5/3}}_{k}
f_{GW}^{11/3} 
\,.
\end{align}

This can be integrated directly, by separating the variables as \(f^{-11/3} \dd{f} = \dd{t}\): the result is 
%
\begin{align}
- \frac{3}{8k} f^{-8/3} = t - t _{\text{coal}} 
\overset{\text{def}}{=} -\tau 
\,,
\end{align}
%
where \(t _{\text{coal}}\) is an integration constant, chosen so that \(t = t _{\text{coal}}\) when the frequency diverges.
We can get the frequency as a function of the time until coalescence \(\tau \):
%
\begin{align} \label{eq:gw-frequency-evolution}
f_{GW} = \qty( \frac{3}{8 k \tau })^{3/8} 
= \frac{1}{\pi } \qty( \frac{5}{256 \tau })^{3/8} \qty(\frac{c^3}{G M_c})^{5/8}
\,.
\end{align}

\todo[inline]{Wrong sign in the slides at this point.}

Inverting this expression we can get the time until coalescence from the frequency: 
%
\begin{align}
\tau = \frac{5}{256} \qty( \frac{1}{\pi f_{GW}})^{8/3} \qty( \frac{c^3}{G M_c})^{5/3}
\,.
\end{align}

Now, we want to get an expression for the \textbf{evolution of the radius}. We know that \(\omega_{GW} \propto \tau^{-3/8}\), so we can calculate 
%
\begin{align}
\frac{1}{\omega_{GW}} \dv{\omega_{GW}}{t} = 
- \frac{1}{\omega_{GW}} \dv{\omega_{GW}}{\tau } = 
+ \frac{3}{8} \frac{\tau^{-11/8}}{\tau^{-3/8} }
= \frac{3}{8} \frac{1}{\tau }
\,,
\end{align}
%
which is also equal to \(\dot{\omega}_{s} / \omega_{s}\), since it is a constant multiple of \(\omega_{GW}\). Then, by using the relation between the logarithmic derivatives of \(R\) and \(\omega_{s}\) \eqref{eq:log-derivatives-radius-angular-velocity-binary} we can write 
%
\begin{align}
\frac{\dot{\omega}_{s}}{\omega_{s}} = \frac{3}{8} \frac{1}{\tau } = - \frac{3}{2} \frac{\dot{R}}{R} 
\implies \frac{\dot{R}}{R} = - \frac{1}{4 \tau }
\,,
\end{align}
%
which is solved by 
%
\begin{align}
R(\tau ) = R_0 \qty(\frac{\tau }{\tau_0 })^{1/4} = R_0 \qty(\frac{t _{\text{coal}} - t}{t _{\text{coal}} - t_0 })^{1/4}
\,,
\end{align}
%
where \(\tau_0 \) is the time to coalescence at \(t_0 \), and \(R_0 \) is the orbital radius at \(t_0 \). 

If we plot this, it has a ``plunge'' phase near \(\tau = 0\), near which our assumptions of a nonrelativistic system break down. 

\todo[inline]{Add plot maybe}

The radius and frequency at \(R = R_0 \) are related by Kepler's third law: 
%
\begin{align}
R_0 = \sqrt[3]{ \frac{GM}{\omega_{s}^2 (\tau_0 )}}
= \sqrt[3]{ \frac{GM}{\pi^2 f_{GW}^2 (\tau_0 )}}
\,,
\end{align}
%
which can be combined with the equation for the evolution of the GW frequency \eqref{eq:gw-frequency-evolution} to yield the following expression for the time until coalescence given the orbital parameters: 
%
\begin{align}
\tau_0 = \frac{5}{256}  \frac{c^{5} R_0^{4}}{G^3M^2 \mu }
\,.
\end{align}

\subsection{Chirping waveform}

The argument of the cosine is in general denoted as \(\phi (t)\) and called the \textbf{phase}, it is defined by the relation 
%
\begin{align}
\phi (t) = \int_{t_0 }^{t} \omega_{GW}(t') \dd{t'} &= -\int_{\tau_0 }^{\tau } \dd{\tau '} \frac{2 \pi }{\pi } \qty(\frac{5}{256} \frac{1}{\tau'})^{3/8} \qty(\frac{c^{3}}{G M_c})^{5/8} \\
&= - 2 \qty( \frac{c^3}{5 G M_c})^{5/8} \tau^{5/8} + \phi_0 
\,,
\end{align}
%
which for a circular orbit is instead just a linear function of \(t\); the angular frequency is given by its constant derivative \(\omega_{GW} = \phi'\).

For a quasi-circular orbit we need to generalize, and the mass moments will look like 
%
\begin{align}
M_{11} = \frac{\mu R^2(t) \qty(1 + \cos(\phi (t)))}{2}
\qquad \text{and} \qquad
M_{12} = \frac{\mu R^2(t) \sin(\phi (t))}{2}
\,,
\end{align}
%
however since \(\dot{\omega}_{GW} \ll \omega^2_{GW}\) and \(\dot{R} \ll R  \omega_{GW}\) we will be able to neglect the derivatives of \(R\) and \(\omega_{GW}\) when computing the second time derivatives of the mass moments.
So, our two polarizations' amplitudes will read 
%
\begin{align}
h_{+} (t) &= \frac{4}{r} \qty(\frac{GM_c}{c^2})^{5/3} \qty(\frac{\pi f_{GW} (t _{\text{ret}})}{c})^{2/3} \frac{1 + \cos^2\theta}{2} 
\cos(\phi (t _{\text{ret}})) \\
h_{ \times } (t) &= \frac{4}{r} \qty(\frac{GM_c}{c^2})^{5/3} \qty(\frac{\pi f_{GW} (t _{\text{ret}})}{c})^{2/3} \cos\theta  
\sin(\phi (t _{\text{ret}})) 
\,,
\end{align}
%
which can be expressed in terms of the time to coalescence \(\tau \) as 
%
\begin{align}
h_{+} (t) &= \frac{1}{r} \qty( \frac{G M_c}{c^2})^{5/4} \qty( \frac{5}{c \tau })^{1/4} \frac{1 + \cos^2\theta }{2} \cos(\phi (\tau )) \\
h_{ \times } (t) &= \frac{1}{r} \qty( \frac{G M_c}{c^2})^{5/4} \qty( \frac{5}{c \tau })^{1/4} \cos \theta  \sin(\phi (\tau )) 
\,,
\end{align}
%
since as long as we are measuring from a fixed position the time to coalescence is the same for any observer seeing the same part of the signal: we can compute \(\tau = t _{\text{coal}} -t = \qty(t _{\text{coal}}+ \Delta t) - (t + \Delta t)\) for any fixed shift \(\Delta t\).

This model predicts a signal with a \textbf{chirping waveform}, whose amplitude increases and frequency both increase.
The chirping waveform cannot be trusted near the end, when our quasi-circular orbit approximation breaks down. 

A result from the study of the orbit of a test particle around a Schwarzschild black hole is that when the radius of the orbit goes below the Innermost Stable Circular Orbit:
%
\begin{align}
R < R _{\text{ISCO}} = \frac{6GM}{c^2}
\,
\end{align}
%
the orbit becomes unstable.
This formula holds for extreme mass ratios which can approximate the scenario of a test mass in a large masses' gravitational field. 
We have not seen any mergers like this yet, although it is definitely possible that they might exist.
 
Anyway, we can use the Schwarzschild radius as a rough guideline to see when the system becomes very relativistic and our approximations break down.

The frequency corresponding to the ISCO can be also calculated: 
%
\begin{align}
f_{GW, ISCO} = \frac{1}{6 \sqrt{6} \pi } \frac{c^3}{GM}
\,.
\end{align}

From numerical full-GR simulations and the measured GW signals we can see that the shape of the chirping waveform is basically correct even after the approximations break down; it only slightly goes out of phase with the numerical relativity calculation. 

\todo[inline]{Numerical relativity has a \emph{lower} frequency than the quadrupole approx!}

\subsection{Eccentric binaries}

We only present the results of the rather long computations; radiated power is higher than the corresponding circular binary, it is given by
%
\begin{align}
\dv{E}{t} = \frac{32}{5} \frac{G \mu^2}{c^{5}} a^{4} \omega_0^{6} f(e)
\,,
\end{align}
%
where 
%
\begin{align}
f(e) = \frac{1}{(1 - e^2)^{7/2}} \qty(1 + \frac{73}{24} e^2 + \frac{37}{96} e^{4}) \geq 1 
\,.
\end{align}

Here the parameter \(e\) is the eccentricity of the elliptical orbit: the major semiaxis is given in terms of the minor one by \(a = b / \sqrt{1 - e^2}\).
This enhancement is due to the fact that the emission is large at the point of closest approach. 

This means that the time to coalescence is \emph{shorter} for elliptical binaries: for an initial eccentricity \(e_0 \sim 1\) we have \(\tau \sim 2 \qty(1- e_0^2)^{7/2} \tau _{\text{circ}}\).

We have emission at frequencies which are integer multiples of the circular-orbit one; the peak of this spectrum shifts upward as \(e\) increases. 

GW emission has the effect of circularizing the orbit rapidly --- by the time the merger arrives it is usually quite close to circular, so we usually detect circular systems. 

\section{Hulse-Taylor binaries}

The detection of gravitational waves in 2015 \cite[]{ligoscientificcollaborationandvirgocollaborationObservationGravitationalWaves2016} was the first \emph{direct} one, but the existence of gravitational waves was established before that through indirect observations. 
By what we have shown in the last section, GW emission causes the decay of the orbit of a binary. 


Weisberg and Taylor measured this effect \cite[fig.\ 6 especially]{taylorNewTestGeneral1982} on a binary pulsar discovered by Hulse and Taylor in the seventies \cite[]{hulseDiscoveryPulsarBinary1975}.
It is debatable whether this was the first observation of gravitational waves, but anyway it is an interesting case study for us.

In this section we will explore the nature of pulsar radioastronomy, the sources of delays in the signal and the compatibility of the period decay result with theoretical GR predictions. 

\subsection{Pulsars}

A \textbf{neutron star} is a kind of stellar remnant which is very dense, although not quite dense enough to form a black hole. 
It has the density of nuclear matter, \(\rho \sim \SI{e17}{kg/m^3}\), since in it electronic repulsion between atoms has been overcome.
Its mass is generally of the order of the mass of the Sun, so its radius is of the order of \SI{10}{km} --- on the scale of the Schwarzschild radius, but with a bit of a margin.

Neutron stars generally have very high angular velocities, which follows from conservation of angular momentum at the moment of collapse. 
Their moments of inertia, \(I = (2/5) M R^2 = (8 \pi /15) \rho R^{5} \sim \SI{e38}{kg m^2} \), are very high, so their angular velocities are very stable.

They also generally have very high magnetic field, on the order of \(B \sim \SI{e8}{T}\); the magnetic poles are often misaligned with the rotation axis of the NS, causing the magnetic field lines to spin with the star.

% This is a binary system in which one star is a pulsar. 

% What is a pulsar? It's a kind of neutron star.
% Not a moral judgement, but you are completely empty.

At a distance \(r_c = c/\omega \) the field lines cannot close since they would need to move faster than \(c\), so a radio beam escapes. 
% This provides a clock!
This beam sweeps around with the same frequency as the pulsar, and since the period is very stable it provides a stable clock. 

\subsubsection{Pulsar signal measurement}

% ``Taking the pulse'' of a pulsar: they usually have a certain well-defined shape, if we average over a few periods. 

Generally the radio signals coming from pulsars arrive buried in noise. 
However, we can still measure the signal, since it is characterized by a precise periodicity. 
We take a \emph{Fourier transform} of the radio signal and identify a peak corresponding to the fundamental frequency of the pulsar's emission. 
Then, we average the signal across periods corresponding to this frequency. 

The external noise will then average out, while the signal builds with each additional period.

% The procedure is: take signal, FFT to get the fundamental, average over periods. 

The period can then be measured precisely, and we can observe its variations. 

The pulsar which was discovered by Hulse and Taylor is called PSR 1913+16 \cite[]{hulseDiscoveryPulsarBinary1975}: it was discovered in 1974; its period (corrected for the effects which we will now mention) is \(T = \SI{59.030(1)}{ms}\).

\todo[inline]{The slides say that the \SI{20}{\micro s} measurement error is random and does not accumulate: we can identify the pulse number even after years of not observing the pulsar.
However, the paper reports a \SI{1}{\micro s} error: where is the \SI{20}{\micro s} figure coming from? Is it referring to the daily variations?}

We can identify fluctuations in the radio signal  period of around \SI{80}{\micro s} daily: this is very large compared to what is typically observed for pulsars, variations around \SI{10}{\micro s} \emph{yearly}.
This atypical behavior was understood by H\&T to correspond to a \textbf{binary} pulsar, where the mass of the pulsar is \(M_p = \num{1.4414} M_{\odot}\) (around the Chandrasekhar mass!), while the mass of the companion is  \(M_c = \num{1.3867} M_{\odot}\). 
As for the orbital parameters, the period is \(T \approx \SI{.323}{d}\), the eccentricity is \(e \approx \num{.6171338}\).

\todo[inline]{These are modern measurements, right? the paper has a much larger uncertainty on \(e\).}

These parameters are gathered by assuming that in the pulsar's frame the emission is stable, and that the delays come from effects pertaining, for example, to the orbit.

\subsection{All the delays}

\subsubsection{Frequencies' timescales and effect summary} 

Let us discuss the magnitudes of the frequencies at hand: the radio waves are on the order of \(f _{\text{radio}}\sim\SI{e8}{Hz}\), the pulsar's frequency is of the order of \(f _{\text{pulsar}}\sim\SI{10}{Hz}\), the frequency of the binary period is \(f _{\text{binary}} \sim\SI{e-5}{Hz}\), the motion of the Earth around the Sun at \(f _{\text{Earth}} \sim \SI{e-8}{Hz}\) is also relevant. 

It is advantageous for us that there are many orders of magnitude between these: since the pulsar frequency is very small, we can still average many pulses and still be measuring at what is basically ``a single point'' in the orbit of the binary.

The effects which we will account for are: 
\begin{enumerate}
    \item Rømer delay (observer): due to the position of the Earth with respect to the solar system barycenter;
    \item Einstein delay (observer): gravitational and Doppler shift at the observer due to the Earth's motion and gravitational field;
    \item Shapiro delay (observer): delay due to propagation in the Sun's gravitational field;
    \item propagation in the interstellar medium;
    \item Rømer delay and aberration (emitter): position of the source with respect to the barycenter of the binary, must account for a relativistic orbit;
    \item Einstein delay (emitter): gravitational and Doppler shift due to the pulsar's motion and gravitational field;
    \item Shapiro delay (emitter): delay due to the propagation in the companion's gravitational field;
    \item \textbf{secular changes}: reduction of the period due to GW emission!
\end{enumerate}

\end{document}
