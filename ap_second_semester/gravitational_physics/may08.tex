\documentclass[main.tex]{subfiles}
\begin{document}

% \subsection{Realistic interferometers}

\marginpar{Friday\\ 2020-5-8, \\ compiled \\ \today}

% \todo[inline]{About an hour of only listening}

With a FP cavity we increase the interaction time of a photon with the GW by a factor \(\mathcal{F} / 2 \pi \), yielding  
%
\begin{align}
\tau_{s} = \frac{L}{c} \frac{\mathcal{F}}{\pi }
\,.
\end{align}

The cavity, in a way, acts as a low-pass filter, removing any effects which happen on timescales shorter than \(\tau_{s}\).

Ideally, we would want to make the storage time as long as possible, but still smaller than the period of the GW: 
%
\begin{align}
\tau_{s} \sim \frac{1}{2 f_{GW}} 
\implies
\mathcal{F} \sim \frac{\pi c }{L 2 f_{GW}} \sim \num{e3} \qty(\frac{\SI{100}{Hz}}{f_{GW}})
\,.
\end{align}

This is an intuitive way to write it, let us make it more precise: we can model power loss at a mirror \(i\) as \(r_i^2 + t_i^2 = 1 - p_i\), where \(p_i\) is a small parameter quantifying the fraction of power lost at the mirror.  

We can define an \textbf{effective power loss} \(p\) by \((1-p_1 ) r_2^2 = 1 - p\).
Then, we will have 
%
\begin{align}
r_1^2 = 1 - p_1 - t_1^2 < 1 - p_1 \implies r_1^2 r_2^2 < 1 - p
\,,
\end{align}
%
\todo[inline]{This whole calculation is kind of unclear\dots}
so the finesse will satisfy the inequality 
%
\begin{align}
r_1 r_2 \approx 1 - \frac{\pi}{\mathcal{F}} \leq 1- \frac{p}{2}
\,.
\end{align}

Let us define the \textbf{coupling factor} \(\sigma = p \mathcal{F} / \pi \): it will be between 0 and 2, and its value will determine whether the cavity is over- or under-coupled; for \(\sigma = 1 \) it will be impedance-matched.

This is interesting since it allows us to study the response of the phase of the reflected field 
%
\begin{align}
E_r = - E _{\text{in}} \frac{r_1 - r_2 e^{-ik2L}}{1 - r_1 r_2 e^{-ik2L}}
\,,
\end{align}
%
to a small perturbation \(\epsilon = 2 k_l \delta L\): its phase \(\phi = \arg E_r\) responds as
%
\begin{align}
\phi  = \pi + \arctan (\frac{\mathcal{F} \epsilon }{\pi } \frac{1}{1 - \sigma })
+ \arctan (\frac{\mathcal{F}\epsilon }{\pi })
\,,
\end{align}
%
so for \(\sigma > 1\) we have partial cancellation of the reflected field, if \(\sigma = 1\)  there is no reflected light at all, while if \(\sigma < 1 \) gets small we get ever more light but less sensitivity. 

\todo[inline]{Why are we interested in the phase of the \emph{reflected} field?}

If we suppose \(r_2 = 1 \), \(r_1 \sim 1\) and \(p_1 = 0 \) we get \(p = 1 - r_2^2 = 0\), so \(\sigma = 0\), which means that the phase sensitivity is 
%
\begin{align}
\pdv{\phi }{\epsilon } = \frac{2 \mathcal{F}}{\pi  }
\,,
\end{align}
%
since we have two equal arctangent terms, both of which are linear near the origin.
This is much better than what we have with a simple Michelson interferometer, \(\pdv*{\phi }{\epsilon } = 1\). 

\subsubsection{GW response of the cavity in the detector frame}

Suppose we have a Michelson interferometer with a FP cavity, which is invested by a \(h_+\) polarized GW which propagates perpendicular to the interferometer plane: then, the phase response will look like 
%
\begin{align}
\Delta \phi_{FP} = \Delta \phi_{x} - \Delta \phi_{y} = 
2 \times 2 k_l \frac{1}{2} L h_0 \cos(\omega_{GW} t) \frac{2 \mathcal{F}}{\pi }
\,,
\end{align}
%
where the first factor of 2 is because the light goes back and forth, while the second factor of 2 is because the effects on the two arms are opposite in sign; the phase difference is computed as laser wavevector times GW length difference (which in the detector frame is due to the mirrors moving around), and finally the factor \(2 \mathcal{F} / \pi \) is the enhancement due to the FP cavity. The amplitude of this variation is then 
%
\begin{align}
\abs{\Delta \phi_{FP}} = \frac{4 \mathcal{F}}{\pi } k_l L h_0 
\,.
\end{align}

\subsubsection{GW response of the cavity in the TT gauge}

We treat the problem in the sidebands picture; the carrier angular frequency is \(\omega_{l}\), and we get some power in the sidebands \(\omega_{l} \pm \omega_{gw}\). 

At each round-trip of the light, some of the power is moved from the carrier to the sidebands. 
We express this as a vector equation: we define a vector containing the amplitudes of the fields, \(\vec{B} = (\text{carrier}, \text{upper sideband}, \text{lower sideband})\); if \(\vec{B}\) is calculated after a round trip while \(\vec{A}\) is calculated before it then we will have  
%
\begin{subequations}
\begin{align}
\left[\begin{array}{c}
B_1  \\ 
B_2  \\ 
B_3 
\end{array}\right]
=
\left[\begin{array}{ccc}
X_{00}  & 0 & 0 \\ 
X_{10}  & X_{11}  & 0 \\ 
X_{20}  & 0 & X_{22} 
\end{array}\right]
\left[\begin{array}{c}
A_1  \\ 
A_2  \\ 
A_3 
\end{array}\right]
\,.
\end{align}
\end{subequations}
%

The diagonal terms describe the in-cavity propagation of the three modes, while the \(X_{10} \) and \(X_{20} \) terms describe the GW-caused scattering of the carrier field into the sidebands. 

Going through the algebra, we find the following expression: 
%
\begin{align}
\abs{\Delta \phi_{x}} 
&= h_0 k_{l} L \operatorname{sinc} \qty( \frac{\omega_{gw}L}{c}) \frac{r_2 \qty(1 - r_1^2 - p )}{r_2 (1 - p) - r_1 }  \frac{1}{\abs{e^{2 i \omega_{gw} \frac{L}{c}} - r_1 r_2} } \\ 
&= h_0 k_{l} L \operatorname{sinc} \qty( \frac{\omega_{gw}L}{c}) \frac{r_2 \qty(1 - r_1^2 - p )}{r_2 (1 - p) - r_1 }  \frac{1}{\sqrt{1 + (r_1 r_2 )^2 - 2 r_1 r_2 \cos( 2 \omega_{GW} L / c)}}
\,.
\end{align}

We can make some approximations: a typical interferometer will work with \(p \approx 0 \), \(r_2 \approx 1\), \(r_1 \sim 1\). Applying these, we can write  
%
\begin{align}
\frac{r_2 \qty(1 - r_1^2 - p )}{r_2 (1 - p) - r_1 } \approx \frac{1 - r_1^2}{1 - r_1 } =  1 + r_1 \approx 2 \qty(1 + \epsilon (r1, r2, p))
\,,
\end{align}
%
where we defined the small parameter \(\epsilon \): what we are saying is that globally the term will be close to 2. 

Also, our finesse is large, of the order \(\mathcal{F} L / c \sim 1 / \omega_{GW}\), so \(\omega_{GW} L / c\) is very small: therefore the value of \(\operatorname{sinc} (\omega_{GW} L / c)\) is around 1, and 
%
\begin{align}
\cos(\frac{2 \omega_{GW} L}{c})\approx 1 - \frac{1}{2} \qty(\frac{2 \omega_{GW}L }{c})^2
\,.
\end{align}

With all of these approximations, we can write the phase response of teh cavity as 
%
\begin{align}
\abs{\Delta \phi_{x}} &\approx 2 h_0 k_l L \frac{1 + \epsilon (r_1, r_2, p)}{1 - r_1 r_2 } \frac{1}{\sqrt{1 + \frac{r_1 r_2 }{(1 - r_1 r_2 )^2} \qty( \frac{2 \omega_{GW} L}{c})^2}}  \\
&\approx 2 h_0 k_l L \frac{\mathcal{F}}{\pi } \frac{1}{\sqrt{1 + \qty(4 \pi f_{GW} \tau_{s})^2}}
\,,
\end{align}
%
where we have used the definition of the finesse: \(\mathcal{F} = \pi \sqrt{r_1 r_2 } / (1 - r_1 r_2 )\) and the storage time \(\tau_{s} = (L/c)(\mathcal{F} / \pi )\).  
\todo[inline]{The square should not include the 1! mistake in the slides.}

This is only for an arm: we recover the factor \(2\) which appears in the detector-frame expression if we consider both. 

So, we can see an additional effect which was hidden in the detector frame expression: the cavity acts as a low-pass filter, whose cutoff is called the \emph{cavity pole}, 
%
\begin{align}
f_p = \frac{1}{4 \pi \tau_{s}} \approx \frac{c}{4 \mathcal{F}L} 
\,,
\end{align}
%
using which we can write the response of the cavity as 
%
\boxalign{
\begin{align}
\abs{\Delta \phi_{FP}} = h_0 \frac{4 \mathcal{F}}{\pi } k_l L \frac{1}{\sqrt{1 + \qty( \frac{f_{gw}}{f_p})^2}}
\,.
\end{align}}

This is the \textbf{transfer function} of the cavity, from the GW signal \(h\) to the phase difference! We can also express it in terms of the laser wavelength \(\lambda_{l} = 2 \pi / k_l\) as 
%
\boxalign{
\begin{align}
T_{FP} (f_{GW}) = \frac{8 \mathcal{F} L}{\lambda_{l}}
\frac{1}{\sqrt{1 + (f_{GW} / f_p)^2}}
\,.
\end{align}}

This is of the order of \num{e13} radians per unit strain for typical values of the finesse \(\mathcal{F} \sim 500\), arm length \(L \sim \SI{3}{km}\) and laser wavelength \(\lambda_{l} \sim \SI{1000}{nm}\), if \(f_{GW} < f_p\). 
The cavity pole is of the order \(f_p \sim \SI{50}{Hz}\). 

For GW frequencies of the order of \SI{500}{Hz} the response drops to about \num{e12}. 

\subsubsection{Double-recycled Fabry-Perot interferometer}

This is the setup used by actual interferometers: we have 
\begin{enumerate}
    \item an input laser with a power of \SI{100}{W} and a wavelength of \(\lambda_{l} \approx \SI{1064}{nm}\); 
    \item an \textbf{input mode cleaner}, which filters out the high-order modes and stabilizes the frequency;
    \item a first \textbf{power-recycling cavity}, inside which the power is of the order \SI{5}{kW}, and which increases the effective power seen by the actual FP cavities;
    \item the beamsplitter;
    \item for each arm, the \SI{3}{km}-long \textbf{FP arm cavity}; the circulating power inside which is of the order \SI{750}{kW};
    \item the output of the beamsplitter, to which is connected a \textbf{signal extraction cavity}, which resonantly enhances the GW sidebands, and a \textbf{output mode cleaner}, which rejects any high-order modes which might have been generated.
\end{enumerate}
 
The FP arm cavities also act as mode filters and frequency stabilizers.
The signal extraction cavity can be tuned to change the response of the interferometer. 

\end{document}