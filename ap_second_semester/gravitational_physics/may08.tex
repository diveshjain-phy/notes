\documentclass[main.tex]{subfiles}
\begin{document}

% \subsection{Realistic interferometers}

\marginpar{Friday\\ 2020-5-8, \\ compiled \\ \today}

% \todo[inline]{About an hour of only listening}

With a FP cavity we increase the interaction time of a photon with the GW by a factor \(\mathcal{F} / 2 \pi \), yielding  
%
\begin{align}
\tau_{s} = \frac{L}{c} \frac{\mathcal{F}}{\pi }
\,.
\end{align}

The cavity, in a way, acts as a low-pass filter, removing any effects which happen on timescales shorter than \(\tau_{s}\).

Ideally, we would want to make the storage time as long as possible, but still smaller than the period of the GW: 
%
\begin{align}
\tau_{s} \sim \frac{1}{2 f_{GW}} 
\implies
\mathcal{F} \sim \frac{\pi c }{L 2 f_{GW}} \sim \num{e3} \qty(\frac{\SI{100}{Hz}}{f_{GW}})
\,.
\end{align}

This is an intuitive way to write it, let us make it more precise: we can model power loss at a mirror \(i\) as \(r_i^2 + t_i^2 = 1 - p_i\), where \(p_i\) is a small parameter quantifying the fraction of power lost at the mirror.  

We can define an \textbf{effective power loss} \(p\) by \((1-p_1 ) r_2^2 = 1 - p\).
Then, we will have 
%
\begin{align}
r_1^2 = 1 - p_1 - t_1^2 < 1 - p_1 \implies r_1^2 r_2^2 < 1 - p
\,,
\end{align}
%
\todo[inline]{This whole calculation is kind of unclear\dots}
so the finesse will satisfy the inequality 
%
\begin{align}
r_1 r_2 \approx 1 - \frac{\pi}{\mathcal{F}} \leq 1- \frac{p}{2}
\,.
\end{align}

Let us define the \textbf{coupling factor} \(\sigma = p \mathcal{F} / \pi \): it will be between 0 and 2, and its value will determine whether the cavity is over- or under-coupled; for \(\sigma = 1 \) it will be impedance-matched.

In the TT gauge, the carrier angular frequency is \(\omega_{l}\) and we get some power in the sidebands \(\omega_{l} \pm \omega_{gw}\). 
We express this as a vector equation: we define the vector \(\vec{B} = (\text{carrier}, \text{sidebands})\) after a round-trip, so we have  
%
\begin{subequations}
\begin{align}
\left[\begin{array}{c}
B_1  \\ 
B_2  \\ 
B_3 
\end{array}\right]
=
\left[\begin{array}{ccc}
X_{00}  & 0 & 0 \\ 
X_{10}  & X_{11}  & 0 \\ 
X_{20}  & 0 & X_{22} 
\end{array}\right]
\left[\begin{array}{c}
A_1  \\ 
A_2  \\ 
A_3 
\end{array}\right]
\,,
\end{align}
\end{subequations}
%
and in the end we get 
%
\begin{align}
\abs{\Delta \phi_{x}} = h_0 k_{l} L \operatorname{sinc} \qty( \frac{\omega_{gw}L}{c}) \frac{r_2 \qty(1 - r_1^2 - p )}{r_2 (1 - p) - r_1 }  \frac{1}{\abs{e^{2 i \omega_{gw} \frac{L}{c}}} - r_1 r_2 }
\,.
\end{align}

We can make some approximations: \(p \approx 0 \), \(r_2 \approx 1\), \(r_1 \sim 1\). We get 
%
\begin{align}
\frac{r_2 \qty(1 - r_1^2 - p )}{r_2 (1 - p) - r_1 } \approx 1 + r_1 \approx 2 \qty(1 + \epsilon (r1, r2, p))
\,.
\end{align}

Also, our finesse is large so that the value of the sinc is around 1. 

The cavity acts as a low-pass frequency: the cutoff is called the \emph{cavity pole}, 
%
\begin{align}
f_p = \frac{1}{4 \pi \tau_{s}} \approx \frac{c}{4 \mathcal{F}L} 
\,,
\end{align}
%
so the response looks like 
%
\begin{align}
\abs{\Delta \phi_{FP}} = h_0 \frac{4 \mathcal{F}}{\pi } k_l L \frac{1}{\sqrt{1 + \qty( \frac{f_{gw}}{f_p})^2}}
\,.
\end{align}

\end{document}