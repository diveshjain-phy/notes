\documentclass[main.tex]{subfiles}
\begin{document}

\marginpar{Monday\\ 2020-3-30, \\ compiled \\ \today}

% Last time we discussed: how can we distinguish what is a GW and what is not? 

% We do this by separating them by frequency. 
% We cannot really measure the nature of tensor perturbation of GWs, since we are only measuring integrated effects. 

% We can precisely map the effect of the GW in time, by sampling with a frequency which is much higher than the one of the GW.

So, let us do this formally: we consider \(g = \overline{g} + h\) where \(\overline{g}\) only contains low frequency components while \(h\) contains high frequency ones, we expand the Ricci tensor in powers of \(h_{\mu \nu }\) up to second order and rewrite the Einstein equations as 
%
\begin{align}
\frac{8 \pi G}{c^{4}} \qty(T_{\mu \nu } - \frac{1}{2} g_{\mu \nu }T)
= \overline{R}_{\mu \nu } 
+ R^{(1)}_{\mu \nu } 
+ R^{(2)}_{\mu \nu } 
\,,
\end{align}
%
where the second order in \(h_{\mu \nu }\) term, \(R^{(2)}_{\mu \nu }\), has both high and low frequency components, while the term \(\overline{R}_{\mu \nu }\) is exclusively low frequency and \(R_{\mu \nu }^{(1)}\) is exclusively high frequency.

Let us give a heuristic argument for this statement: the gravitational waves contained in \(h_{\mu \nu } \) will have several high frequencies; let us call two of them \(\omega_1  \) and \(\omega_2 \), so as we compute a term which is quadratic in \(h\) we will get
%
\begin{align}
\qty(\sin(\omega_{1} t) + \sin(\omega_2 t))^2 
= \\ =
\frac{1}{2} \left(-\cos \left(2 t \omega _1\right)-\cos \left(2 t \omega _2\right)-2 \cos \left(t \omega _1+t \omega _2\right)+2 \cos \left(t \omega _1-t \omega _2\right)+2\right)
\,,
\end{align}
%
where we used the prostapheresis formulas: we get terms oscillating with \(\omega_1 + \omega_2 \) as well as \(\omega_1 - \omega_2 \), one of which is high frequency while the other is low frequency, since we are considering a high-frequency wavepacket.

So, what we want to do is to separate the field equation into its low and high frequency parts: 
%
\begin{align} \label{eq:low-frequency-ricci-tensor-contribution}
\overline{R}_{\mu \nu } &= - \qty(R_{\mu \nu }^{(2)})^{\text{low}} + \frac{8 \pi G}{c^{4}} \qty(T_{\mu \nu } - \frac{1}{2} g_{\mu \nu } T)^{\text{low}} \\
R_{\mu \nu }^{(1)} &= - \qty(R_{\mu \nu }^{(2)})^{\text{high}} + \frac{8 \pi G}{c^{4}} \qty(T_{\mu \nu } - \frac{1}{2} g_{\mu \nu } T)^{\text{high}} 
\,.
\end{align}

How might we define these formally?
Recall, we expand in two parameters: \(h\) and \(\lambda / L_B \sim  f_B / f\). Then, in terms of orders of magnitude we have\footnote{The fact that the derivatives correspond to divisions by lengths is justified since we are looking at harmonic expansions, the metric will be a sum of sinusoids around a characteristic frequency.} 
%
\begin{align}
\overline{R}_{\mu \nu } \sim \partial^2 \overline{g}_{\mu \nu } \sim \frac{1}{L_B^2} 
\,,
\end{align}
%
while 
%
\begin{align}
\qty[R^{(2)}_{\mu \nu }]^{\text{low}} \sim \qty(\partial h)^2
 \sim \qty(\frac{h}{\lambda })^2
\,.
\end{align}

See Maggiore \cite[pag.\ 31]{maggioreGravitationalWavesVolume2007} for more details. 
So, inserting these order-of-magnitude estimates inside equation \eqref{eq:low-frequency-ricci-tensor-contribution} we get
%
\begin{align}
\frac{1}{L_B^2} \sim \frac{h^2}{\lambda^2} + \qty(T_{\mu \nu } \text{ contribution})
\,.
\end{align}

If we are in a vacuum (\(T_{\mu \nu }=0\)) then
%
\begin{align}
\frac{1}{L_B^2} \sim \qty(\frac{h}{\lambda })^2
\implies h \sim \frac{\lambda }{L_B}
\,,
\end{align}
%
while if the stress-energy tensor is nonvanishing we get
%
\begin{align}
\frac{1}{L_B^2} \sim \qty(\frac{h}{\lambda })^2 + (T_{\mu \nu } \text{ contribution}) \gg \qty(\frac{h}{\lambda })^2
\,,
\end{align}
%
which means \(h \ll \lambda / L_B\). This shows that \textbf{the linearized expansion cannot be extended beyond linear order} starting from a flat background: if we start from a flat metric then \(1/L_B = 0\), so any \(h > 0\) violates the condition \(h \lesssim \lambda / L_B\).
We can also see that the notion of a gravitational wave only makes sense as long as \(h\) is small: if \(h\) were close to 1, then we would have \(\lambda \sim L_B\), but our only way to distinguish GW from background is through their wavelength! 

\todo[inline]{Unclear what the problem is: we cannot use the flat metric, but the real world's metric is not flat anyways! Why can we not go to second order with a curved background?}

The solution to this problem is to take averages on a scale \(\ell\) such that \(\lambda_{GW} \ll \ell \ll L_B\), so that the gravitational wave is fully averaged out, while the background metric is approximately constant across the integration volume.
This low-frequency projection reads:
%
\begin{align}
\overline{R}_{\mu \nu } =
- \expval{R^{(2)}_{\mu \nu }} + \frac{8 \pi G}{c^{4}} \expval{T_{\mu \nu } - \frac{1}{2} g_{\mu \nu } T} 
\,.
\end{align}

After some math (see Maggiore \cite[eqs.\ 1.122-1.123]{maggioreGravitationalWavesVolume2007}) we can define a stress tensor of the GW, which looks like 
%
\begin{align}
t_{\mu \nu } = - \frac{c^{4}}{8 \pi G} \expval{R^{(2)}_{\mu \nu } - \frac{1}{2} \overline{g}_{\mu \nu } R^{(2)}}
\,,
\end{align}
%
where the Ricci scalar is computed using the smoothed metric: \(R^{(2)} = \overline{g}^{\mu \nu } R_{\mu \nu }^{(2)}\). 
Also, we define a  ``smoothed out'' stress-energy tensor of matter: 
%
\begin{align}
\expval{T_{\mu \nu } - \frac{1}{2} g_{\mu \nu } T}
\overset{\text{def}}{=}
\overline{T}_{\mu \nu } - \frac{1}{2} \overline{g}_{\mu \nu } \overline{T}
\,,
\end{align}
%
so the equation which will hold is the equivalence of the smoothed Einstein tensor with the sum of the smoothed and GW stress-energy tensors:
%
\boxalign{
\begin{align}
\overline{R}_{\mu \nu } - \frac{1}{2} \overline{g}_{\mu \nu } \overline{R} = 
\frac{8 \pi G}{c^{4}} \qty(\overline{T}_{\mu \nu } + t_{\mu \nu })
\,.
\end{align}}

Do note that if we work with these, the stress-energy tensor which is conserved is \(\overline{T}_{\mu \nu } + t_{\mu \nu }\): if we take the divergence of the equation we get, by the Bianchi identities,
%
\begin{align}
\nabla^{\mu } \qty(\overline{R}_{\mu \nu } - \frac{1}{2} \overline{g}_{\mu \nu }\overline{R}) = 0 = \nabla^{\mu } \qty(\overline{T}_{\mu \nu } + t_{\mu \nu })
\,.
\end{align}

This means that gravitational waves and matter can exchange energy and momentum. If the stress-energy tensor is slowly-varying enough, as often happens practically, we can just use \(\overline{T}^{\mu \nu } \approx T^{\mu \nu }\).

How does this look like far from the source?
There, we can approximate \(\overline{g} \approx \eta_{\mu \nu }\) and \(\nabla_{\mu } \approx \partial_{\mu }\). 
This \(t_{\mu \nu }\) only has 2 physical degrees of freedom (those of the gravitational wave): we need to gauge the others away.

The Lorentz gauge plus \(h=0\) eliminates \(5\) degrees of freedom. 

When we have terms like \(h \partial \partial h\), we can integrate by parts on a sufficiently large volume to turn them into \(\partial(h \partial ) - \partial h \partial h\). 
Using the facts \(\partial^{\mu }h_{\mu \nu } = h = \square h_{\mu \nu } = 0\) we can also simplify several terms: in the end we get 
%
\boxalign{
\begin{align}
t_{\mu \nu } = \frac{c^{4}}{32 \pi G} \expval{\partial_{\mu } h_{\alpha \beta } \partial_{\nu } h^{\alpha \beta }}
\,.
\end{align}}

This is invariant under the residual gauge transformations, and coordinate independent: we can compute it in any frame we like.

\subsubsection{Explicit TT-gauge expression}

Let us compute this in the simplest case: a gravitational wave travelling along the \(z\) axis, described in the TT-gauge.
The 00 component will read: 
%
\begin{align}
t^{00 } = \frac{c^{4}}{32 \pi G} \expval{\partial^{0} h_{\alpha \beta } \partial^{0} h^{\alpha \beta }} = \frac{c^2}{16 \pi G} \expval{\dot{h}^2_{+} + \dot{h}^2_{ \times }}
\,,
\end{align}
%
since: 
%
\begin{align}
\partial^{0} h_{\alpha \beta } \partial^{0} h^{\alpha \beta } =
\frac{\dot{h}^{TT}_{ij}}{c} \frac{\dot{h}^{TT}_{ij}}{c} 
= \frac{1}{c^2}\sum _{i, j =1}^{2} \qty(\dot{h}_{ij}^{TT})^2 
= \frac{2}{c^2} \qty(\dot{h}_{+}^2 + \dot{h}_{ \times }^2)
\,.
\end{align}

As for the other components, we will have \(t^{01} = t^{02} =0 \) by symmetry (they would represent momentum transfer in a direction orthogonal to the propagation, also formally \(\partial_{1,2} h_{ij}^{TT} = 0\) since it only depends on \(t\) and \(z\)), and also \(t^{03}= t^{00}\), since the perturbation is a function of \((t - z/c)\): so, we have 
%
\begin{align}
\partial_{3} h_{ij}^{TT} = - \partial_{0} h_{ij}^{TT} = + \partial^{0} h_{ij}^{TT}
\,.
\end{align}

\subsection{Energy and momentum flux far from the source}

If we are far enough away from the source, we can compute the energy crossing a surface \(\dd{A}\) in a time \(\dd{t}\) as the spacetime density contained in a volume \(\dd{A} c \dd{t}\): if we are considering a specific direction (say, the flux coming the way of the Earth) then we can say that the wave's propagation is aligned with the \(z\) axis and so we can write 
%
\begin{align}
\dd{E} &= \dd{A} c \dd{t} \frac{c^2}{32 \pi G} \expval{\dot{h}^{TT}_{ij}\dot{h}^{TT}_{ij}} \\
\frac{ \dd{E}}{ \dd{t} \dd{A}} &= \frac{c^3}{16 \pi G} \expval{\dot{h}^2_{+} + \dot{h}^2_{ \times }}
\,.
\end{align}

In order to get the total power \(\dv*{E}{t}\) which is emitted by the source we can integrate this expression in \(R^2 \dd{\Omega }\); however we will need to use the general expression \(\expval{\dot{h}^{TT}_{ij}\dot{h}^{TT}_{ij}}\) since we cannot have alignment with the \(z\) axis across the whole sphere. 

In order to do this calculation, we can use the explicit expression for the \(\Delta_{ij,kl}\) projection tensor. We find 
%
\begin{subequations}
\begin{align}
\dv{E}{t} &= \frac{r^2c^{3}}{32 \pi G} \int \dd{\Omega }
\expval{
\frac{1}{r} \frac{2G}{c^{4}} \Lambda_{ij,kl} \dot{\ddot{Q}}^{kl}
\frac{1}{r} \frac{2G}{c^{4}} \Lambda_{ij,kl} \dot{\ddot{Q}}^{kl}
}  \\
&=\frac{G}{8 \pi c^{5}} \int \dd{\Omega }
\Lambda_{ij, kl} \expval{\dot{\ddot{Q}}^{ij} \dot{\ddot{Q}}^{kl}}
\,,
\end{align}
\end{subequations}
%
where the only expression depending on the angle is \(\Lambda_{ij, kl} \): then we integrate and find 
%
\begin{align}
\dv{E}{t} = \dots
\,.
\end{align}

When we do the same from the momentum density, we get an integral in the form 
%
\begin{align}
\dv{P^{k}}{t} \propto \int \dd{\Omega } \ddot{Q} \partial^{k} \dot{\ddot{Q}}
\,,
\end{align}
%
but the integrand is odd under spatial inversion, so there is no contribution! This is not true if we go beyond the quadrupole, instead full GR calculations/simulations show that there are kicks at the merger. 

We can calculate the angular distribution in a relatively simple way, since it is easy to go to TT gauge at a point. 

The energy lost by the source at \(t _{\text{ret}} = t - r/c \) is the same as the energy measured in GW. 
\todo[inline]{Clarify: why are the two expressions calculated at the same time?}

If we model the back-reaction as a force, we have 
%
\begin{align}
\dv{E _{\text{source}}}{t} = 
\expval{\int  \dd[3]{x} \dv{F_i}{V} \dot{x}_{i}} = 
= - \frac{G}{5 c^{5}} \expval{\dv{Q_{ij}}{t} \dv[5]{Q_{ij}}{t}}
\,.
\end{align}

Equating terms and making some considerations, we get 
%
\begin{align}
\dv{F_{i}}{V} = - \frac{2G}{5c^{5}} \dv[5]{Q_{ij}}{t} \rho (t, \vec{x}) x_{j}
\,,
\end{align}
%
so finally 
%
\begin{align}
F_{i} = - \frac{2G}{5c^{5}} \dv[5]{Q_{ij}}{t} m \overline{x}_{j}
\,,
\end{align}
%
where \(\overline{x}_{j} \) is the center-of-mass coordinate. 

Then, we can calculate the torque explicitly: we get 
%
\begin{align}
T_{i} = - \frac{2G}{5c^{5}} \epsilon_{ijk} Q_{il} \dv[5]{Q_{kl}}{t}
\,,
\end{align}
%
so if we take the average we get 
%
\begin{align}
\expval{\dv{L_{i}}{t}} = -\frac{2G}{5c^{5}} \epsilon_{ijk} 
\expval{\dot{\ddot{Q}}_{jl} \dot{\ddot{Q}}_{kl}}
\,.
\end{align}

\section{Back-reaction and the evolution of binary systems}

We use the reduced mass formalism, the amplitude is 
%
\begin{align}
A = \frac{4 G^{5/3} \omega_{s}^{2/3} \mu M^{2/3}}{r c^{4}}
\,,
\end{align}
%
so we have the polarizations 
%
\begin{subequations}
\begin{align}
h_{+} &= A \frac{1 - \cos^2\theta   }{2} \cos(2 \omega_{s} t _{\text{ret}} + 2 \varphi ) \\ 
h_{\times } &= A \cos \theta  \sin(2 \omega_{s} t _{\text{ret}} + 2 \varphi )  
\,,
\end{align}
\end{subequations}
%
we define the chirp mass: 
%
\begin{align}
M_c = \mu ^{3/5} M^{2/5} = \frac{\qty(m_1 m_2 )^{3/5}}{(m_1 + m_2 )^{1/5}}
\,,
\end{align}
%
also the frequency of the GW is twice the frequency of the orbit. 

Coming back to the radiated power: it is 
%
\begin{align}
\dv{E}{t} = \frac{G}{5 c^{5}} \expval{\dot{\ddot{M}}_{ij} \dot{\ddot{M}}_{ij} - \frac{1}{3} \qty(\dot{\ddot{M}}_{kk})^2}
\,,
\end{align}
%
so we can calculate this explicitly for our binary: we need to average over a period, so we get a factor \(\expval{\sin^2\varphi } = 1/2\): 
%
\begin{align}
\dv{E}{t} = \frac{32}{5} \frac{c^{5}}{G} \qty(\frac{GM_c \omega_{GW}}{2 c^{3}})^{ 10 /3}
\,,
\end{align}
%
where we used the fact \(R^3 = GM / \omega_{s}^2\).
We can do a similar thing for the angular momentum: 
%
\begin{align}
\dv{L}{t} = \dots
\,.
\end{align}

Now, let us consider masses in quasi-circular orbit: by the virial theorem, 
%
\begin{align}
\dot{R} = - \frac{2R^2}{G m_1 m_2 } \dot{E}_{GW}
\,.
\end{align}

Therefore, as the GWs carry away energy the radius shrinks. 
But we assumed the orbits to be circular! This is fine: they are almost-circular usually. 
This is fine for most of the inspiral, until the merger phase. 

\end{document}
