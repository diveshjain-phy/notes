\documentclass[main.tex]{subfiles}
\begin{document}

\marginpar{Monday\\ 2020-3-30, \\ compiled \\ \today}

% Last time we discussed: how can we distinguish what is a GW and what is not? 

% We do this by separating them by frequency. 
% We cannot really measure the nature of tensor perturbation of GWs, since we are only measuring integrated effects. 

% We can precisely map the effect of the GW in time, by sampling with a frequency which is much higher than the one of the GW.

So, let us do this formally: we consider \(g = \overline{g} + h\) where \(\overline{g}\) only contains low frequency components while \(h\) contains high frequency ones, we expand the Ricci tensor in powers of \(h_{\mu \nu }\) up to second order and rewrite the Einstein equations as 
%
\begin{align}
\frac{8 \pi G}{c^{4}} \qty(T_{\mu \nu } - \frac{1}{2} g_{\mu \nu }T)
= \overline{R}_{\mu \nu } 
+ R^{(1)}_{\mu \nu } 
+ R^{(2)}_{\mu \nu } 
\,,
\end{align}
%
where the second order in \(h_{\mu \nu }\) term, \(R^{(2)}_{\mu \nu }\), has both high and low frequency components, while the term \(\overline{R}_{\mu \nu }\) is exclusively low frequency and \(R_{\mu \nu }^{(1)}\) is exclusively high frequency.

Let us give a heuristic argument for this statement: the gravitational waves contained in \(h_{\mu \nu } \) will have several high frequencies; let us call two of them \(\omega_1  \) and \(\omega_2 \), so as we compute a term which is quadratic in \(h\) we will get
%
\begin{align}
\qty(\sin(\omega_{1} t) + \sin(\omega_2 t))^2 
= \\ =
\frac{1}{2} \left(-\cos \left(2 t \omega _1\right)-\cos \left(2 t \omega _2\right)-2 \cos \left(t \omega _1+t \omega _2\right)+2 \cos \left(t \omega _1-t \omega _2\right)+2\right)
\,,
\end{align}
%
where we used the prostapheresis formulas: we get terms oscillating with \(\omega_1 + \omega_2 \) as well as \(\omega_1 - \omega_2 \), one of which is high frequency while the other is low frequency, since we are considering a high-frequency wavepacket.

So, what we want to do is to separate the field equation into its low and high frequency parts: 
%
\begin{align} \label{eq:low-frequency-ricci-tensor-contribution}
\overline{R}_{\mu \nu } &= - \qty(R_{\mu \nu }^{(2)})^{\text{low}} + \frac{8 \pi G}{c^{4}} \qty(T_{\mu \nu } - \frac{1}{2} g_{\mu \nu } T)^{\text{low}} \\
R_{\mu \nu }^{(1)} &= - \qty(R_{\mu \nu }^{(2)})^{\text{high}} + \frac{8 \pi G}{c^{4}} \qty(T_{\mu \nu } - \frac{1}{2} g_{\mu \nu } T)^{\text{high}} 
\,.
\end{align}

How might we define these formally?
Recall, we expand in two parameters: \(h\) and \(\lambda / L_B \sim  f_B / f\). Then, in terms of orders of magnitude we have\footnote{The fact that the derivatives correspond to divisions by lengths is justified since we are looking at harmonic expansions, the metric will be a sum of sinusoids around a characteristic frequency.} 
%
\begin{align}
\overline{R}_{\mu \nu } \sim \partial^2 \overline{g}_{\mu \nu } \sim \frac{1}{L_B^2} 
\,,
\end{align}
%
while 
%
\begin{align}
\qty[R^{(2)}_{\mu \nu }]^{\text{low}} \sim \qty(\partial h)^2
 \sim \qty(\frac{h}{\lambda })^2
\,.
\end{align}

See Maggiore \cite[pag.\ 31]{maggioreGravitationalWavesVolume2007} for more details. 
So, inserting these order-of-magnitude estimates inside equation \eqref{eq:low-frequency-ricci-tensor-contribution} we get
%
\begin{align}
\frac{1}{L_B^2} \sim \frac{h^2}{\lambda^2} + \qty(T_{\mu \nu } \text{ contribution})
\,.
\end{align}

If we are in a vacuum (\(T_{\mu \nu }=0\)) then
%
\begin{align}
\frac{1}{L_B^2} \sim \qty(\frac{h}{\lambda })^2
\implies h \sim \frac{\lambda }{L_B}
\,,
\end{align}
%
while if the stress-energy tensor is nonvanishing we get
%
\begin{align}
\frac{1}{L_B^2} \sim \qty(\frac{h}{\lambda })^2 + (T_{\mu \nu } \text{ contribution}) \gg \qty(\frac{h}{\lambda })^2
\,,
\end{align}
%
which means \(h \ll \lambda / L_B\). This shows that \textbf{the linearized expansion cannot be extended beyond linear order} starting from a flat background: if we start from a flat metric then \(1/L_B = 0\), so any \(h > 0\) violates the condition \(h \lesssim \lambda / L_B\).
We can also see that the notion of a gravitational wave only makes sense as long as \(h\) is small: if \(h\) were close to 1, then we would have \(\lambda \sim L_B\), but our only way to distinguish GW from background is through their wavelength! 

\todo[inline]{Unclear what the problem is: we cannot use the flat metric, but the real world's metric is not flat anyways! Why can we not go to second order with a curved background?}

The solution to this problem is to take averages on a scale \(\ell\) such that \(\lambda_{GW} \ll \ell \ll L_B\), so that the gravitational wave is fully averaged out, while the background metric is approximately constant across the integration volume.
This low-frequency projection reads:
%
\begin{align}
\overline{R}_{\mu \nu } =
- \expval{R^{(2)}_{\mu \nu }} + \frac{8 \pi G}{c^{4}} \expval{T_{\mu \nu } - \frac{1}{2} g_{\mu \nu } T} 
\,.
\end{align}

After some math (see Maggiore \cite[eqs.\ 1.122-1.123]{maggioreGravitationalWavesVolume2007}) we can define a stress tensor of the GW, which looks like 
%
\begin{align}
t_{\mu \nu } = - \frac{c^{4}}{8 \pi G} \expval{R^{(2)}_{\mu \nu } - \frac{1}{2} \overline{g}_{\mu \nu } R^{(2)}}
\,,
\end{align}
%
where the Ricci scalar is computed using the smoothed metric: \(R^{(2)} = \overline{g}^{\mu \nu } R_{\mu \nu }^{(2)}\). 
Also, we define a  ``smoothed out'' stress-energy tensor of matter: 
%
\begin{align}
\expval{T_{\mu \nu } - \frac{1}{2} g_{\mu \nu } T}
\overset{\text{def}}{=}
\overline{T}_{\mu \nu } - \frac{1}{2} \overline{g}_{\mu \nu } \overline{T}
\,,
\end{align}
%
so the equation which will hold is the equivalence of the smoothed Einstein tensor with the sum of the smoothed and GW stress-energy tensors:
%
\boxalign{
\begin{align}
\overline{R}_{\mu \nu } - \frac{1}{2} \overline{g}_{\mu \nu } \overline{R} = 
\frac{8 \pi G}{c^{4}} \qty(\overline{T}_{\mu \nu } + t_{\mu \nu })
\,.
\end{align}}

Do note that if we work with these, the stress-energy tensor which is conserved is \(\overline{T}_{\mu \nu } + t_{\mu \nu }\): if we take the divergence of the equation we get, by the Bianchi identities,
%
\begin{align}
\nabla^{\mu } \qty(\overline{R}_{\mu \nu } - \frac{1}{2} \overline{g}_{\mu \nu }\overline{R}) = 0 = \nabla^{\mu } \qty(\overline{T}_{\mu \nu } + t_{\mu \nu })
\,.
\end{align}

This means that gravitational waves and matter can exchange energy and momentum. If the stress-energy tensor is slowly-varying enough, as often happens practically, we can just use \(\overline{T}^{\mu \nu } \approx T^{\mu \nu }\).

How does this look like far from the source?
There, we can approximate \(\overline{g} \approx \eta_{\mu \nu }\) and \(\nabla_{\mu } \approx \partial_{\mu }\). 
This \(t_{\mu \nu }\) only has 2 physical degrees of freedom (those of the gravitational wave): we need to gauge the others away.

The Lorentz gauge plus \(h=0\) eliminates \(5\) degrees of freedom. 

When we have terms like \(h \partial \partial h\), we can integrate by parts on a sufficiently large volume to turn them into \(\partial(h \partial ) - \partial h \partial h\). 
Using the facts \(\partial^{\mu }h_{\mu \nu } = h = \square h_{\mu \nu } = 0\) we can also simplify several terms: in the end we get 
%
\boxalign{
\begin{align}
t_{\mu \nu } = \frac{c^{4}}{32 \pi G} \expval{\partial_{\mu } h_{\alpha \beta } \partial_{\nu } h^{\alpha \beta }}
\,.
\end{align}}

This is invariant under the residual gauge transformations, and coordinate independent: we can compute it in any frame we like.

\subsubsection{Explicit TT-gauge expression}

Let us compute this in the simplest case: a gravitational wave travelling along the \(z\) axis, described in the TT-gauge.
The 00 component will read: 
%
\begin{align}
t^{00 } = \frac{c^{4}}{32 \pi G} \expval{\partial^{0} h_{\alpha \beta } \partial^{0} h^{\alpha \beta }} = \frac{c^2}{16 \pi G} \expval{\dot{h}^2_{+} + \dot{h}^2_{ \times }}
\,,
\end{align}
%
since: 
%
\begin{align}
\partial^{0} h_{\alpha \beta } \partial^{0} h^{\alpha \beta } =
\frac{\dot{h}^{TT}_{ij}}{c} \frac{\dot{h}^{TT}_{ij}}{c} 
= \frac{1}{c^2}\sum _{i, j =1}^{2} \qty(\dot{h}_{ij}^{TT})^2 
= \frac{2}{c^2} \qty(\dot{h}_{+}^2 + \dot{h}_{ \times }^2)
\,.
\end{align}

As for the other components, we will have \(t^{01} = t^{02} =0 \) by symmetry (they would represent momentum transfer in a direction orthogonal to the propagation, also formally \(\partial_{1,2} h_{ij}^{TT} = 0\) since it only depends on \(t\) and \(z\)), and also \(t^{03}= t^{00}\), since the perturbation is a function of \((t - z/c)\): so, we have 
%
\begin{align}
\partial_{3} h_{ij}^{TT} = - \partial_{0} h_{ij}^{TT} = + \partial^{0} h_{ij}^{TT}
\,.
\end{align}

\subsection{Energy and momentum flux far from the source}

If we are far enough away from the source, we can compute the energy crossing a surface \(\dd{A}\) in a time \(\dd{t}\) as the spacetime density contained in a volume \(\dd{A} c \dd{t}\): if we are considering a specific direction (say, the flux coming the way of the Earth) then we can say that the wave's propagation is aligned with the \(z\) axis and so we can write 
%
\begin{align}
\dd{E} &= \dd{A} c \dd{t} \frac{c^2}{32 \pi G} \expval{\dot{h}^{TT}_{ij}\dot{h}^{TT}_{ij}} \\
\frac{ \dd{E}}{ \dd{t} \dd{A}} &= \frac{c^3}{16 \pi G} \expval{\dot{h}^2_{+} + \dot{h}^2_{ \times }}
\,.
\end{align}

In order to get the total power \(\dv*{E}{t}\) which is emitted by the source we can integrate this expression in \(R^2 \dd{\Omega }\); however we will need to use the general expression \(\expval{\dot{h}^{TT}_{ij}\dot{h}^{TT}_{ij}}\) since we cannot have alignment with the \(z\) axis across the whole sphere. 

To get the momentum density the reasoning is similar: we start from \(\dd{P^{k}} = \dd{A} c \dd{t} t^{0k}/c\) to get 
%
\begin{align}
\frac{ \dd{P^{k}}}{ \dd{A} \dd{t}} = \frac{c^{4}}{32 \pi G} \expval{ \partial^{0} h_{\alpha \beta } \partial^{ k} h^{\alpha \beta }} 
= - \frac{c^3}{32 \pi G} \expval{\dot{h}^{TT}_{ij} \partial^{k} h_{ij}^{TT}}
\,,
\end{align}
%
where we used the fact that \(\partial^{0} = -\partial_0  =-  \pdv*{}{(ct)}\). 

So, the general expressions for the energy and momentum density emitted at a distance will be 
%
\begin{align}
\dv{E}{t} = \frac{c^3}{32 \pi G} r^2 \int 
\expval{\dot{h}^{TT}_{ij}\dot{h}^{TT}_{ij}} \dd{\Omega }
\qquad \text{and} \qquad
\dv{P^{k}}{t} = - \frac{c^3}{32 \pi G} r^2 
\int \expval{\dot{h}^{TT}_{ij} \partial^{k} h_{ij}^{TT}} \dd{\Omega }
\,.
\end{align}

We want to express this explicitly in terms of the quadrupole moment. We start from the expression of the TT-gauge amplitude in terms of the quadrupole \eqref{eq:traceless-transverse-amplitude-from-quadrupole}: its derivative reads 
%
\begin{align}
\dot{h}_{ij}^{TT} (t, \vec{x}) = \frac{1}{r} \frac{2G}{c^{4}} \Lambda_{ij, kl} \dot{\ddot{Q}}^{kl} (t - r/c)
\,.
\end{align}

Inserting this into the expression we get 
%
\begin{subequations}
\begin{align}
\dv{E}{t} &= \frac{r^2c^{3}}{32 \pi G} \int \dd{\Omega }
\expval{
\frac{1}{r} \frac{2G}{c^{4}} \Lambda_{ij,kl} \dot{\ddot{Q}}^{kl}
\frac{1}{r} \frac{2G}{c^{4}} \Lambda_{ij,mn} \dot{\ddot{Q}}^{mn}
}  \\
&=\frac{G}{8 \pi c^{5}} \int \dd{\Omega }
\Lambda_{ij, kl} \expval{\dot{\ddot{Q}}^{ij} \dot{\ddot{Q}}^{kl}}
\marginnote{\(\Lambda \) is idempotent and symmetric under swaps of index pairs: so \(\Lambda_{ij, kl} \Lambda_{ij, mn} = \Lambda_{kl, mn}\).}
\,,
\end{align}
\end{subequations}
%
where the only expression depending on the angle is \(\Lambda_{ij, kl} \), since the quadrupole only depends on the source, not on an observer's position. 

\begin{claim}
The explicit expression for the projection tensor \(\Lambda_{ij,kl}(\hat{n})\) in terms of the unit vector \(\hat{n}\) is: 
%
\begin{align}
\Lambda_{ij, kl} (\hat{n}) = \delta_{ik} \delta_{jl} - \frac{1}{2} \delta_{ij} \delta_{kl} - n_j n_l \delta_{ik} - n_i n_k \delta_{jl}
+ \frac{1}{2} n_k n_l \delta_{ij} 
+ \frac{1}{2} n_i n_j \delta_{kl}
+ \frac{1}{2} n_i n_j n_k n_l
\,,
\end{align}
%
and its integral in \(\dd{\Omega }\) is given by 
%
\begin{align}
\int \dd{\Omega } \Lambda_{ij, kl} = \frac{4 \pi }{30} 
\qty(11 \delta_{ik} \delta_{jl} - 4 \delta_{ij} \delta_{kl} + \delta_{il} \delta_{jk} )
\,.
\end{align}
\end{claim}

With this result, we can write 
%
\begin{align}
\dv{E}{t} &= \frac{G}{8 \pi c^{5}} \frac{2\pi}{15} \qty(11 \delta_{ik} \delta_{jl} - 4 \delta_{ij} \delta_{kl} + \delta_{il} \delta_{jk} )
\expval{\dot{\ddot{Q}}^{ij} \dot{\ddot{Q}}^{kl}}
\,,
\end{align}
%
but since the quadrupole moment derivatives \(\dot{\ddot{Q}}^{ij}\) are both traceless and symmetric the first and third delta combinations are equal (so the factor multiplying them will be 12), while the second combination will vanish. So, we will get 
%
\begin{align}
\dv{E}{t} = \frac{G}{8 \pi c^{5}} \frac{2 \pi }{15} 12 
\expval{\dot{\ddot{Q}}^{ij} \dot{\ddot{Q}}^{ij}}
= \frac{G }{5 c^{5}}
\expval{\dot{\ddot{Q}}^{ij} \dot{\ddot{Q}}^{ij}}
= \frac{G}{5 c^{5}} 
\expval{\dot{\ddot{M}}_{ij} \dot{\ddot{M}}_{ij} - \frac{1}{3} \dot{\ddot{M}}_{kk}^{2} }
\,.
\end{align}

When we do the same from the \textbf{momentum loss}, we get an integral in the form 
%
\begin{align}
\dv{P^{k}}{t} \propto \int \dd{\Omega } \dot{\ddot{Q}}^{TT}_{ij} \partial^{k} \ddot{Q}^{TT}_{ij}
\,,
\end{align}
%
which is odd under spatial inversion because of the spatial derivative, so there is no contribution!

This is not true if we go beyond the quadrupole approximation: full GR calculations/simulations show that there can be kicks at the merger. 

In linearized gravity, we can calculate the power loss by energy conservation: 
%
\begin{align}
\dv{E _{\text{source}}}{t} = - \dv{E _{\text{far-field}}}{t} = - \frac{G}{5c^{5}} \expval{\dot{\ddot{Q}}^{ij} \dot{\ddot{Q}}^{ij}}
\,,
\end{align}
%
although this is not exact moment-by-moment in full GR, since there some of the emission can be delayed. 
In our approximation, the two sides of the equation should be calculated at the same retarded time: the power loss by the source at \(t - r/c\) will be the power detected a distance \(r\) away at a time \(t\). 

\subsection{Angular momentum loss from GW}

\subsubsection{Effective back-action force}

% We can calculate the angular distribution in a relatively simple way, since it is easy to go to TT gauge at a point. 

We can model the back-action from GW as a force applied on the source \cite[sec.\ 3.3.4]{maggioreGravitationalWavesVolume2007}, whose power is the average of \(\vec{F} \cdot \vec{v}\) as in classical mechanics; since we consider a continuous medium we have:
%
\begin{align}
- \frac{G}{5 c^{5}} \expval{\dv[3]{Q_{ij}}{t} \dv[3]{Q_{ij}}{t}}
= \dv{E _{\text{source}}}{t} = 
\expval{\int  \dd[3]{x} \dv{F_i}{V} \dot{x}_{i}} 
\,.
\end{align}

We can integrate by parts twice to write the two third derivatives as 
%
\begin{align}
\expval{\dv[3]{Q_{ij}}{t} \dv[3]{Q_{ij}}{t}}
= \expval{\dv{Q_{ij}}{t} \dv[5]{Q_{ij}}{t}}
\,.
\end{align}

Also, the 0th component of the conservation of the stress-energy tensor \(\partial_{\mu } T^{0 \mu }\) for a classical source reads \(\partial_{t} \rho + \partial_{i} \qty(\rho v_i) = 0 \). Using this fact, we can rewrite the first derivative of \(Q_{ij}\) as 
%
\begin{align}
\dv{Q_{ij}}{t} &= \dv{}{t} \int  \dd[3]{x} 
\rho \bigg( x_{i} x_{j} - \underbrace{\frac{1}{3} r^2 \delta_{ij}}_{\mathclap{\text{contracted with traceless \(\dot{\ddot{\ddot{Q}}}_{ij}\)}}}\bigg) 
= - \int \dd[3]{x} \partial_{k} \qty(\rho v_k) 
 x_{i} x_{j}  \\
&= + \int \dd[3]{x} \rho v_k \partial_{k} \qty(x_i x_j) = 
2 \int \dd[3]{x} \rho v_{(i } x_{j)}
\,.
\end{align}
%
\todo[inline]{Missing symmetrization in the slides! Although it is implicit when contracting with \(Q\)\dots}

Then our equation reads: 
%
\begin{align}
\expval{\int \dd[3]{x} \dv{F_i}{V} \dot{x}_{i}} = - \frac{2G}{5c^{5}}\expval{\int \dd[3]{x} \dv[5]{Q_{ij}}{t} \rho \dot{x}_{i} x_{j}}
\,,
\end{align}
%
so we can identify the terms to get
%
\begin{align}
\dv{F_{i}}{V} = - \frac{2G}{5c^{5}} \dv[5]{Q_{ij}}{t} \rho (t, \vec{x}) x_{j}
\,,
\end{align}
%
so finally, since the only position-dependent terms on the right-hand side are the density and the position we get that the effective force is: 
%
\boxalign{
\begin{align}
F_{i} = - \frac{2G}{5c^{5}} \dv[5]{Q_{ij}}{t} m \overline{x}_{j}
\,,
\end{align}}
%
where \(\overline{x}_{j} \) is the center-of-mass coordinate. 

\subsubsection{Angular momentum}

With this effective force we can calculate the torque explicitly: in general it is given by \(T_i = \epsilon_{ijk} x_j F_k\), which we can make into a local relation by substituting \(T_i\) and \(F_k\) with their densities. With this we can write 
%
\begin{align}
\dv{T_i}{V} = \epsilon_{ijk} x_j  \dv{F_k}{V} = -
\epsilon_{ijk} x_j  \frac{2G}{5 c^{5}} \dv[5]{Q_{kl}}{t} \rho (t, \vec{x}) x_l
\,,
\end{align}
%
so the total torque is given by
%
\begin{align}
T_{i} &= - \frac{2G}{5 c^{5}} \epsilon_{ijk} \dv[5]{Q_{kl}}{t}
\int \dd[3]{x} \rho (t, \vec{x}) x_l x_j \underbrace{- \frac{1}{3} r^2 \delta_{lj}}_{\mathclap{\text{does not contribute, since \(Q\) is traceless}}} \\
&= - \frac{2G}{5 c^{5}} \epsilon_{ijk} \dv[5]{Q_{kl}}{t} Q_{lj} 
\,,
\end{align}
%
so if we take the average, substituting the derivative of the angular momentum for the torque and integrating by parts twice to get a second and a third derivative, we find:
%
\begin{align}
\expval{\dv{L_{i}}{t}} = -\frac{2G}{5c^{5}} \epsilon_{ijk} 
\expval{\ddot{Q}_{jl} \dot{\ddot{Q}}_{kl}}
\,.
\end{align}

\section{Back-reaction and the evolution of binary systems}

\subsection{Compact circular inspiral}

We use the reduced mass formalism for a two-body problem: we define the total mass \(M = m_1 +m_2 \), the reduced mass \(\mu = m_1 m_2 / M\), the relative coordinate \(\vec{x} = \vec{x}_1 - \vec{x}_2\) whose modulus is \(R = \abs{\vec{x}}\). 
The angular velocity of the circular motion is \(\omega_{s}\), satisfying Kepler's law \(\omega_{s}^2 R^3 = GM\).

As we saw, the amplitude of the emitted gravitational waves is given by 
%
\begin{align}
A = \frac{4 G^{5/3} \omega_{s}^{2/3} \mu M^{2/3}}{r c^{4}}
\,,
\end{align}
%
and the amplitudes of the two polarizations in time are given in terms of \(A\) by
%
\begin{subequations}
\begin{align}
h_{+} &= A \frac{1 - \cos^2\theta   }{2} \cos(2 \omega_{s} t _{\text{ret}} + 2 \varphi ) \\ 
h_{\times } &= A \cos \theta  \sin(2 \omega_{s} t _{\text{ret}} + 2 \varphi )  
\,.
\end{align}
\end{subequations}

In order to make these expressions easier to interpret, we define the chirp mass: 
%
\begin{align}
M_c = \mu ^{3/5} M^{2/5} = \frac{\qty(m_1 m_2 )^{3/5}}{(m_1 + m_2 )^{1/5}}
\,.
\end{align}

Also, since the frequency of the emitted GW is double that of the binary we define \(f_{GW} = 2 f_s = 2 \omega_{s} / (2 \pi )\), and \(\omega_{GW} = 2 \pi f_{GW}\).
The reduced wavelength corresponding to this frequency is \(\lambdabar = c / \omega_{GW}\); also we can define a Schwarzschild radius corresponding to the chirp mass: \(R_C = 2GM_c / c^2\). 

With these definitions we can rewrite \(A\) as: 
%
\begin{align}
A = \frac{4}{r} \qty( \frac{G M_C}{c^2})^{5/3} \qty( \frac{\pi f_{GW}}{c})^{2/3}
= \frac{1}{\sqrt[3]{2}} \frac{R_C}{r} \qty( \frac{R_C}{\lambdabar})^{2/3}
\,,
\end{align}

As we have shown, in the quadrupole approximation the radiated power is: 
%
\begin{align}
\dv{E}{t} = \frac{G}{5 c^{5}} \expval{\dot{\ddot{M}}_{ij} \dot{\ddot{M}}_{ij} - \frac{1}{3} \qty(\dot{\ddot{M}}_{kk})^2}
\,,
\end{align}
%
which we can calculate explicitly for our binary.
We have already derived expressions for the second time derivatives of the second mass moment, \(\ddot{M}_{ij}\) for motion in the \(xy\) plane \eqref{eq:second-mass-moment-xy-components}; taking their derivative we get 
%
\begin{align}
\dot{\ddot{M}}_{11} = - \dot{\ddot{M}}_{22} = 
- 4 \mu R^2 \omega_{s}^3 \sin(2 \omega_{s}t )
\qquad \text{and} \qquad
\dot{\ddot{M}}_{12} = 4 \mu R^2 \omega_{s}^3 \cos(\omega_{s}t)
\,,
\end{align}
%
so the term \(\qty(\dot{\ddot{M}}_{ij})^2\) vanishes (\(\dot{\ddot{M}}_{ij}\) is traceless) and \(\dot{\ddot{M}}_{11}^2 = \dot{\ddot{M}}_{22}^2\), so the contribution we get is 
%
\begin{align}
\dv{E}{t} &= \frac{G}{5 c^{5}} \expval{2 \dot{\ddot{M}}_{11}^2 + 2 \dot{\ddot{M}}_{12}^2 }  \\
&= \frac{G}{5 c^{5}} 2 \qty(4 \mu R^2 \omega_{s}^3)^2 \underbrace{\qty(\expval{\sin^2(2 \omega_{s}t)} + \expval{\cos^2(2\omega_{s}t)})}_{= 1/2 + 1/2 = 1}
\,,
\end{align}
%
since the average of the squared sine or cosine is \(1/2\). 
Now, we can simplify this expression by using \(R^3 = GM / \omega_{s}^2\) and substituting in the GW angular velocity:
%
\begin{align}
\dv{E_{GW}}{t} &= \frac{32G}{5c^{5}} \mu^2 R^{4} \omega_{s}^{6}  \\
&= \frac{32G}{5c^{5}} \mu^2 \qty( \frac{GM}{\omega_{s}})^{4/3} \omega_{s}^{6}  \\
&= \frac{32}{5c^{5}} \mu^2 G^{7/3} M^{4/3} \omega_{s}^{10/3} \\
&= \frac{32}{5} \frac{c^{5}}{G} \qty( \frac{G M_c \omega_{GW}}{2 c^{3}})^{10/3} \label{eq:gravitational-wave-total-power-emitted}
\,.
\end{align}

We can apply a similar reasoning for the angular momentum loss (here \(L = \abs{\vec{L}}\)): 
%
\begin{align}
\dv{L}{t} = 
\frac{32}{5} \frac{c^{5}}{G} \qty( \frac{G M_c \omega_{GW}}{2 c^{3}})^{10/3} \frac{2}{\omega_{GW}}
= \frac{32}{5} \frac{c^{5}}{G} \qty(\frac{G M_c}{c^{3}})^{10/3} \qty(\frac{\omega_{GW}}{2})^{7/3}
\,.
\end{align}

If we have masses in circular orbit, then by the virial theorem their total energy is given by half of the potential energy: 
%
\begin{align}
E = - \frac{1}{2} \frac{G m_1 m_2  }{R}
\,,
\end{align}
%
which we can differentiate with respect to time to get \(\dot{E} = G m_1 m_2 / (2 R^2) \dot{R}\); this can be also written as
%
\begin{align} \label{eq:radius-evolution-binary}
\dot{R} = - \frac{2R^2}{G m_1 m_2 } \dot{E}_{GW}
\,,
\end{align}
%
as long as the orbital energy is only lost through gravitational wave emission. 

This means that, as the GW carry away energy, the radius shrinks; this corresponds to an increase in frequency, and a corresponding increase in gravitational wave emission. 

In all our calculations we assumed the orbits to be circular!
This is fine, as long as they are almost-circular: the condition to require is that the variation of the radius across a single orbit is very small: 
%
\begin{align}
\abs{\frac{\dot{R} T}{R}} \ll 1
\,,
\end{align}
%
where \(T\) is the period.
In order to express this conditions in terms of the orbital angular velocity \(\omega_{s}\) we use Kepler's law: if we differentiate \(R = \qty(GM / \omega_{s}^2)^{1/3}\) we get 
%
\begin{align} \label{eq:derivative-radius-evolution-binary}
\dot{R} = - \frac{2}{3} \omega_{s}^{-5/3} \qty(GM)^{1/3} \dot{\omega}_{s}
= - \frac{2}{3} \frac{R}{\omega_{s}} \dot{\omega}_{s}
\,.
\end{align}

This equality can also be written in terms of logarithmic derivatives as 
%
\begin{align} \label{eq:log-derivatives-radius-angular-velocity-binary}
\frac{\dot{R}}{R} + \frac{2}{3} \frac{\dot{\omega}_{s}}{\omega_{s}} = 0
\,.
\end{align}

Starting from this, and using \(T = 2 \pi / \omega_{s}\) we can write 
%
\begin{align}
\abs{\frac{\dot{R}T}{R}} = \frac{2}{3} \frac{R}{\omega_{s}} \dot{\omega}_{s} \frac{2\pi}{\omega_{s}} \frac{1}{R} = \frac{4 \pi }{3} \frac{\dot{\omega}_{s}}{\omega_{s}^2} 
\,,
\end{align}
% 
which means that we need to require \(\dot{\omega}_{s} \ll \omega_{s}^2\).

In practice, this condition is quite well satisfied for most of the inspiral, right up until the merger phase. 
Motion satisfying this condition is called \textbf{quasi-circular}. 

\end{document}
