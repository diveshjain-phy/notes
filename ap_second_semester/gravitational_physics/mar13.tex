\documentclass[main.tex]{subfiles}
\begin{document}

\marginpar{Friday\\ 2020-3-13, \\ compiled \\ \today}

What we will do in the first lessons: 
\begin{enumerate}
  \item A quick review of GR;
  \item linearization and GW in free space;
  \item the physical effect of GW: free falling reference frames, detector frame;
  \item GW sources : binary systems, multipole expansion and quadrupole approximation, GW back reaction: energy \& momentum loss, Hulse-Taylor pulsar;
  \item GW sources: a rotating rigid body.
\end{enumerate}

\section{A quick review of GR}

We start from special relativity. 
The ``old'' way to do transformations are galilean transformations: in 2D they are
%
\begin{align}
t' &= t  \\
x' &= x - vt
\,.
\end{align}

There are issues with these, such as the invariance of the speed of light. So, we use Lorentz transformations: 
%
\begin{align}
ct' &= \gamma \qty(ct - \beta x)  \\
x' &= \gamma \qty(x - \beta ct)
\,,
\end{align}
%
where \(\beta = v/c \leq 1\), \(c\) being the speed of light, and \(\gamma = 1/\sqrt{1 - \beta^2} \geq 1\). 

These preserve the spacetime interval 
%
\begin{align}
\Delta s^2 = - c^2 \Delta t^2 + \Delta x^2
\,.
\end{align}

The intervals between two events can be spacelike (\(\Delta s^2 > 0\)), null (\(\Delta s^2 = 0\)) or timelike (\(\Delta s^2 < 0\)). 

We can express this using an infinitesimal time interval  
%
\begin{align}
\dd{s^2} = \dd{x^{\mu }} \eta_{\mu \nu  } \dd{x^{\nu }}
\,,
\end{align}
%
where we use Einstein summation convention. 
We are going to use the mostly plus metric convention. 

We can define the differential \emph{proper time} along a curve, by 
%
\begin{align}
c^2 \dd{\tau^2}  = - \dd{s^2} = c^2 \dd{t^2} \qty(1 - \beta^2) = \frac{c^2}{\gamma^2} \dd{t^2}
\,,
\end{align}
%
which means that \(\dd{\tau } = \dd{t} / \gamma \). 
We can use this as a \emph{covariant} parametrization of a spacetime curve. 

For curved spacetime, we model it as a 4D semi-Riemannian manifold with signature \((1, 3)\). 
Since it is a manifold, the parametrization of points in spacetime must be a homeomorphism, and we ask for the \emph{transition maps} between two regions of spacetime to be infinitely differentiable. 
The set of local charts is called an atlas. 
The charts are maps from \(\mathbb{R}^{4}\) to the manifold. 

The metric is a function of the point at which we are, and (the way it changes) describes the local geometry of the manifold. 
Only the symmetric part of the  metric appears in the spacetime interval, therefore we say that the metric is always symmetric without losing any generality. 

The metric is a bilinear form at each point of the manifold, and it transforms as a \((0,2)\) tensor. 
The components of this tensor in our chosen reference frame are \(g_{\mu \nu }\). 
The choice of coordinates is arbitrary and tricky. 

In a neighborhood of a point we can always choose a reference frame (Riemann normal coordinates) such that \(g_{\mu \nu } = \eta_{\mu \nu }\), and \(g_{\mu \nu , \alpha } = 0\) (partial derivatives calculated \emph{at that point}), but the second derivatives \(g_{\mu \nu , \alpha \beta }\) cannot all be set to zero. 

Vectors in a manifold are defined in the tangent space \emph{at a point}. 
Formally, we define curves parametrically as \(X^{\mu }(\lambda )\). 

Then, we define the tangent vector to the curve as the \emph{directional derivative} operator along the curve: 
%
\begin{align}
\vec{v} (f) = \eval{\dv{f}{\lambda }}_{C} = \pdv{f}{x^{\mu }} \dv{X^{\mu }}{\lambda }
\,,
\end{align}
%
for any scalar field \(f\). The motivation for this definition, as opposed to just taking the tangent vector to the curve, is the fact that there is no \emph{intrinsic} way to to that. 

If we define a curve using a coordinate as a parameter, with the other coordinates staying constant along the curve, this is called a \emph{coordinate curve}. 

Vectors defined at different points are in different spaces, we cannot compare them directly. 

Tangent vectors to coordinate lines are called coordinate basis vectors \(e_{(\mu )}\), where \(\mu \) is not a vector index but instead it spans the basis vectors. So, any vector can be written as a linear combination as \(\vec{v} = v^{\mu } e_{\mu }\).
We also have \(e_{\mu } \cdot e_{\nu } = g_{\mu \nu }\), so, in order to find the components of the scalar product \(v \cdot w\) we need to do \(v^{\mu } w^{\nu } g_{\mu \nu  }\). 

This is because \(g_{\mu \nu } \dd{x^{\mu } } \dd{x^{\nu }} = \dd{s} \cdot \dd{s} = \qty(\dd{x^{\mu }} e_{\mu }) \cdot \qty(\dd{x^{\nu }} e_{\nu })\). 

An orthonormal basis in one for which \(e_{\mu } \cdot e_{\nu } = \eta_{\mu \nu }\). 

Dual basis vectors \(e^{\mu }\) are defined by \(e^{\mu } e_{\nu } = \delta^{\mu }_{\nu }\). 

We write a co-vector (or dual vector) as a linear combination of these: \(v = v_{\mu } e^{\mu }\). 

Then, we can raise and lower indices like 
%
\begin{align}
g_{\mu \nu } v^{\mu } w^{\nu } = v \cdot w = v_{\mu } e^{\mu } \cdot w^{\nu } e_{\nu } = v_{\mu } w^{\nu } \delta^{\mu }_{\nu } = v_{\mu } w^{\nu }
\,.
\end{align}

The inverse metric is defined as \(g^{\mu \nu } g_{\nu \rho } = \delta^{\mu }_{\rho }\). 

Tensors are geometrical objects which belong to the dual space to the cartesian product of \(n\) copies of the tangent space and \(m\) copies of the dual tangent space. 
The type of such a tensor is then said to be  \((n, m)\), and its rank is \(n+m\). 
This definition means that the tensor is a \emph{multilinear} transformation. 

Once we have a coordinate system, we can move to another via a coordinate transformation 
%
\begin{align}
x^{\prime \mu } = x^{\prime \mu } \qty(x^{\mu })
\implies 
\dd{x^{\prime \mu }} = \pdv{x^{\prime \mu }}{x^{\nu }} \dd{x^{\nu }}
\,.
\end{align}

A scalar does not transform: \(\phi (x) = \phi^{\prime } (x^{\prime })\).
A vector's component do transform: we find the transformation law by imposing \(v = v'\) in components. 
This works both for covariant and contravariant vectors, and we find that these transform using either the Jacobian of the transformation or its inverse. 

In order to compute derivatives we need to compare vectors in different tangent spaces: we need to ``connect'' infinitesimally close tangent spaces, and the tool to do so is indeed called a connection, or covariant derivative. 

For a scalar field \(S\) the covariant derivative is \(\nabla_{\alpha } S = \partial_{\alpha } S\). 

A torsionless manifold is one in which 
%
\begin{align}
[\nabla_{\mu }, \nabla_{\nu }] S = 0
\,,
\end{align}
%
for a scalar field \(S\). 
This means that 
%
\begin{align}
\nabla_{[\mu  } \nabla_{\nu ]} S = \nabla_{[\mu  } \partial_{\nu ]} S = \partial_{[\mu } \partial_{\nu ]} S - \Gamma^{\alpha }_{[\mu \nu ]} \partial_{\alpha } = 0
\implies \Gamma^{\alpha }_{[\mu \nu ]} = 0
\,.
\end{align}
%


Parallel transport: intuitively, we move along a curve and keep the angle with respect to the tangent vector constant. 
Formally, if \(u^{\mu }\) is the tangent vector to the curve and \(V^{\mu }\) is the vector we want to transport, we set \(u^{\mu } \nabla_{\mu } V^{\nu }= 0\). 

The Riemann tensor is defined as the commutator of the covariant derivatives:
%
\begin{align}
[\nabla_{\mu }, \nabla_{\nu }] V^{\alpha } = R^{\alpha }_{\beta \mu \nu } V^{\beta }
\,,
\end{align}
%
and it can be expressed in terms of the Christoffel symbols as 
%
\begin{align}
R^{\mu }_{\nu \rho \sigma } = -2 \qty(\Gamma^{\mu }_{\nu [\rho , \sigma ]} + \Gamma^{\beta }_{\nu [\rho } \Gamma^{\mu }_{\sigma ] \beta })
\,.
\end{align}

Geodesics: they are ``the straightest possible path between two points''.
They stationarize the proper length.
Formally, they are curves whose tangent vector is parallel-transported along the curve. 

We actually do not need to say that the derivative of the tangent vector with respect to the parameter is zero: it can be nonzero, as long as it is parallel to the tangent vector. 

So, we could say that 
%
\begin{align}
h_{\nu \rho } \qty(u^{\mu } \nabla_{\mu } u^{\nu }) =0
\,,
\end{align}
%
?


The path that a massive particle follows in the absence of external forces is a geodesic. 
We can describe the separation between two particles which follow geodesics: this is described by the equation of geodesic deviation. 
We take a geodesic \(x^{\mu }\) and another \(y^{\mu } = x^{\mu }+ \xi^{\mu }\), with \(\xi^{\mu }\) being (at least initially) small. 

We can choose a coordinate system in which \(\Gamma^{\mu }_{\nu \rho } =0\). So, 
%
\begin{align}
\eval{\dv[2]{x^{\mu }}{u}}_{P} = 0
\,,
\end{align} 
%
\begin{align}
\eval{\qty(\dv[2]{y^{\mu }}{u} + \Gamma^{\mu }_{\nu \rho } \dv{y^{\nu }}{u} \dv{y^{\rho }}{u})}_{P} = 0
\,,
\end{align}
%
where \(u\) is the tangent vector to the geodesics. We approximate the Christoffel symbols to first order as 
%
\begin{align}
\eval{\Gamma^{\mu }_{\nu \rho }}_{Q} = \xi^{\alpha } \partial_{\alpha } \Gamma^{\mu }_{\nu \rho }
\,.
\end{align}
%
If we subtract the two, we get 
%
\begin{align}
\ddot{\xi}^{\mu } + \qty(\partial_{\alpha } \Gamma^{\mu }_{\nu \rho }) \dot{x}^{\nu } \dot{x}^{\rho } \xi^{\alpha } = 0
\,,
\end{align}
%
but the first term is not an intrinsic derivative: that would be given by 
%
\begin{align}
\frac{\mathrm{D}^2 \xi^{\mu }}{\mathrm{D} u^2} 
= \dv{}{u} \qty(\dot{\xi}^{\mu } + \Gamma^{\mu }_{\nu \rho } \xi^{\nu } \dot{x}^{\rho }) = \ddot{\xi}^{\mu }+ \qty(\partial_{\alpha } \Gamma^{\mu }_{\nu \rho }) \xi^{\alpha } \dot{x}^{\nu } \dot{x}^{\rho }
\,,
\end{align}
%
which means that 
%
\begin{align}
0= \frac{\mathrm{D}^2 \xi^{\mu }}{\mathrm{D} u^2} 
+ \qty(\partial_{\alpha } \Gamma^{\mu }_{\nu \rho } - \partial_{\rho } \Gamma^{\mu }_{\nu \alpha }) \xi^{\alpha } \dot{x}^{\mu } \dot{x}^{\nu }
= \frac{\mathrm{D}^2 \xi^{\mu }}{\mathrm{D} u^2} 
+ R^{\mu }_{\nu \rho \sigma } u^{\nu } u^{\rho } \xi ^{\sigma }
\,.
\end{align}

The gravitational field is described by the Einstein Field Equations: 
%
\begin{align}
R_{\mu \nu } - \frac{1}{2} g_{\mu \nu } R = \frac{8 \pi G}{c^{4}} T_{\mu \nu } 
\,.
\end{align}

\end{document}
