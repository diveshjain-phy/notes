\documentclass[main.tex]{subfiles}
\begin{document}

\marginpar{Friday\\ 2020-4-10, \\ compiled \\ \today}

We define the \textbf{ellipticity} \(\epsilon = (I_1 - I_2 ) / I_3 \). Typical values of this parameter for astrophysical objects are at most of the order of \num{e-6}, which can be calculated as \(\epsilon \sim ( \delta R / R_0 )^2\), where \(\delta R\) is the scale of the radial anomaly while  \(R_0 \) is the scale of the radius of the object.
For a neutron star this corresponds to ``mountains'' of about \(\delta R \sim \SI{10}{m}\).

Then, we can define a typical amplitude \(h_0 \) as: 
%
\begin{align} \label{eq:typical-amplitude-rigid-nonprecessing-body}
h_0 = \frac{4 \pi^2G}{c^{4}} \frac{f_{GW}^2}{r} I_3 \epsilon 
\,,
\end{align}
%
where, as usual, \(f_{GW} = \omega_{r} / \pi = 2 f _{\text{rotation}}\).

In terms of typical orders of magnitude, this variable looks like
%
\begin{align}
h_0 \sim \num{e-25} \qty(\frac{\epsilon }{\num{e-6}})
\qty( \frac{I_3}{\SI{e38}{kg m^2}})
\qty( \frac{\SI{10}{kpc}}{r})
\qty( \frac{f_{GW}}{\SI{1}{kHz}})^2
\,.
\end{align}

With this, we can rewrite the amplitudes in the two polarizations as 
%
\boxalign{
\begin{align}
h_{+} &= h_0 \frac{1 + \cos^2 \iota }{2} \cos(2 \pi f_{GW} t) \\
h_{ \times } &= h_0 \cos \iota  \sin(2 \pi f_{GW} t) 
\,.
\end{align}}

To find the radiated power by this mechanism we can use the quadrupole formula \eqref{eq:radiated-power-GW}: 
%
\begin{align}
\dv{E_{GW}}{t} &= \frac{G}{5 c^{5}} \expval{
    \dot{\ddot{M}}_{ij} \dot{\ddot{M}}_{ij} - \frac{1}{3} \underbrace{\qty(\dot{\ddot{M}}_{kk})^2}_{= 0}
}  \\
&= \frac{G}{5c^{5}} 2 \expval{ \dot{\ddot{M}}_{11}^2 + \dot{\ddot{M}}_{12}^2}  \\
&= \frac{2G}{5c^{5}} \qty(4 \omega_{r}^3 (I_1 - I_2 ))^2 \underbrace{\expval{ \cos^2(2 \omega_{r}t) + \sin^2(2 \omega_{r}t)}}_{= 1/2 + 1/2}  \\
&= \frac{32 G}{5 c^{5}} \omega_{r}^{6} \epsilon^2 I_3^2
\,,
\end{align}
%
so by conservation of energy the neutron star will lose just as much energy. 
The rotational energy is given by \(E _{\text{rot}} = I_{3} \omega_{r}^2 / 2\), so we have 
%
\begin{align}
\dv{E _{\text{rot}}}{t} = - \dv{E_{GW}}{t} &= I_3 \omega_{r} \dot{\omega}_{r}  \\
- \frac{32 G}{5 c^{5}} \omega_{r}^{6} \epsilon^2 I_3^2 &= I_3 \omega_{r} \dot{\omega}_{r}  \\
\dot{\omega}_{r} &= - \frac{32G}{5 c^{5}} \omega_{r}^{5} \epsilon^2 I_3 <0
\,,
\end{align}
%
so, \emph{as opposed to binaries}, the orbit \textbf{slows down} because of GW emission.
Observations of binaries show \(\dot{\omega} \sim - \omega^{n} \) with \(n < 5\), meaning that there probably is another breaking mechanism contributing.   

\subsection{Precession}

Now, let us consider a body whose angular momentum \(\vec{J}\) is \emph{not aligned} with its axes of inertia \cite[sec.\ 4.2.2]{maggioreGravitationalWavesVolume2007}. 

We want to proceed like we did before, so we will need two reference frames. 
The first reference, \(S\), is a frame in which \(\vec{J} = J \hat{z}\); this will be at least approximately an inertial reference frame, so \(\vec{J}\) will be conserved.  
The second reference, \(S'\), is the \emph{body frame} of the object, in which it is stationary, and whose axes coincide with the principal axes of rotation. 

The transformation between these two frames will be a rotation matrix \(R\) (such that \(x' = R x\)), which we decompose as 
%
\begin{align}
R = R_{\gamma }^{(z)} R_{\alpha }^{(x)} R_{\beta }^{(z)} 
= \left[\begin{array}{ccc}
\cos \gamma  & \sin \gamma  & 0 \\ 
- \sin \gamma  & \cos \gamma  & 0 \\ 
0 & 0 & 1
\end{array}\right]
\left[\begin{array}{ccc}
1 & 0 & 0 \\ 
0 & \cos \alpha  & \sin \alpha  \\ 
0 & -\sin \alpha  & \cos \alpha 
\end{array}\right]
\left[\begin{array}{ccc}
\cos \beta  & \sin \beta  & 0 \\ 
- \sin \beta  & \cos \beta  & 0 \\ 
0 & 0 & 1
\end{array}\right]
\,.
\end{align}

We call the \emph{line of nodes} the intersection between the plane orthogonal to \(x_3 \) and that orthogonal to \(x_3'\).
The \(\beta \) rotation brings \(x_1 \) on the line of nodes, the \(\alpha \) rotation brings \(x_3 \) onto \(x_3'\), the \(\gamma \) rotation aligns \(x_1 \) with \(x_1'\). 
In order to understand this, it is customary to look at the figure \cite[fig.\ 4.15]{maggioreGravitationalWavesVolume2007} and fiddle around with your fingers in the ``right-hand-rule'' position. 

All three of these angles will in general be time-dependent, and their time evolution will completely determine the motion of the body.

We can recover the angular velocity vector \(\vec{\omega}\) by looking at the components of the three angular velocity vectors in the body frame:
%
\begin{align}
\dv{\vec{\alpha}}{t} &= \dot{\alpha} \left[\begin{array}{ccc}
\cos \gamma  &  - \sin \gamma  &  0 
\end{array}\right]^{\top}  \\
\dv{\vec{\beta}}{t} &= \dot{\beta} 
\left[\begin{array}{ccc}
\sin \alpha \sin \gamma  & \sin \alpha \cos \gamma  & \cos \alpha 
\end{array}\right]^{\top} \\
\dv{\vec{\gamma}}{t} &= \dot{\gamma} \left[\begin{array}{ccc}
0 & 0 & 1
\end{array}\right]^{\top}
\,,
\end{align}
%
so that then  \(\vec{\omega} = \vec{\dot{\alpha}} + \vec{\dot{\beta}} + \vec{\dot{\gamma}}\).
These expressions can be derived geometrically by looking at the figure. 
So, in the body frame the components of the angular velocity are 
%
\begin{align}
\vec{\omega} = \left[\begin{array}{c}
\dot{\alpha} \cos \gamma + \dot{\beta} \sin \alpha \sin \gamma  \\ 
- \dot{\alpha} \sin \gamma + \dot{\beta} \sin \alpha \cos \gamma  \\ 
\dot{\gamma} + \dot{\beta}\cos \alpha 
\end{array}\right]
\,.
\end{align}

In the body frame the angular momentum \(\vec{J}\) is \emph{not} conserved: we can recover its time-dependent expression in the body frame \(J'\) by applying a rotation, and then we can use \(J'_i = I_i \omega'_i\): this gives us 
%
\begin{align}
J'_1 &= I_1 \omega_1' &&\implies & J \sin \alpha \sin \gamma &= I_1 \qty(\dot{\alpha} \cos \gamma + \dot{\beta} \sin \alpha \sin \gamma ) \\
J'_2 &= I_2 \omega_2' &&\implies & J \sin \alpha \cos \gamma &= I_2 \qty(- \dot{\alpha} \sin \gamma + \dot{\beta} \sin \alpha \cos \gamma) \\
J'_3 &= I_3 \omega_3' &&\implies & J \cos \alpha  &= I_3 \qty(\dot{\gamma} + \dot{\beta} \cos \alpha ) 
\,.
\end{align}

Now we make the assumption that \(I_1 = I_2 \): we consider an \textbf{axisymmetric body}.
An astrophysical example of this will usually look like an ellipsoid.

Then, we perform the following manipulation (written in a formally peculiar way, which should make it easier to remember --- we are ``applying a rotation matrix to the system of equations''): 
%
\begin{align}
&\left[\begin{array}{cc}
\cos \gamma  & \sin \gamma  \\ 
- \sin \gamma  & \cos \gamma 
\end{array}\right]
\left[\begin{array}{c}
J \sin \alpha \sin \gamma = I_1 \qty(\dot{\alpha} \cos \gamma + \dot{\beta} \sin \alpha \sin \gamma ) \\ 
J \sin \alpha \cos \gamma = I_2 \qty(- \dot{\alpha} \sin \gamma + \dot{\beta} \sin \alpha \cos \gamma)
\end{array}\right] = \\
&= \left[\begin{array}{c}
I_1 \dot{\alpha} \qty(\cos^2 \gamma + \sin^2 \gamma ) = 0 \\ 
J \sin \alpha \qty(\cos^2\gamma + \sin^2 \gamma ) = \dot{\beta} I_1 \sin \alpha \qty(\sin^2\gamma + \cos^2\gamma ) 
\end{array}\right] = 
\left[\begin{array}{c}
\dot{\alpha} = 0 \\ 
\dot{\beta} = J / I_1 \overset{\text{def}}{=} \Omega 
\end{array}\right]
\,.
\end{align}

So, \(\alpha \) is constant while \(\beta \) changes linearly. We can substitute these relations into the third equation to get 
%
\begin{align}
J \cos \alpha &= I_3 \qty(\dot{\gamma} + \frac{J \cos \alpha }{I_1 }) \\
\dot{\gamma} &= \frac{J \cos \alpha }{I_3 } - \frac{J \cos \alpha }{I_1 }
= J \cos \alpha \frac{I_1 - I_3 }{I_1 I_3 } = \Omega \cos \alpha \frac{I_1 - I_3 }{I_3 } \overset{\text{def}}{=} - \omega_{p}
\,,
\end{align}
%
where the sign is a convention, such that when \(I_3 > I_1 \) (an oblate object, like a grapefruit or a coin) we have \(\omega_{p} > 0\).

Now, what do these represent? The fact that \(\dot{\alpha} = 0\) means that the angle between \(x_3 \) and \(x_3'\) stays the same. 
The rotation around \(\beta \) is the ``main'' one, as \(\vec{\beta} \) is aligned with \(\vec{J}\), and \(\dot{\beta}= \Omega \gg \abs{\omega_{p}} = \abs{\dot{\gamma}}\) typically. 

The rotation around \(\vec{\gamma}\) corresponds to a \emph{precession} of the angular velocity vector around the \(x_3'\) axis: the body's rotation axis precesses around its third principal axis.

Not that this is not the same as the precession of the body axis around the angular momentum. 
We should ``clean our minds'' from the idea of a spinning spintop precessing, this is not what is happening here.
This wobbling motion is similar to the one of a coin thrown on a table, although this is a \emph{free} wobble, happening without any external torque. 

% Here, the axis going around is the faster motion, the rotation of the body around its axis is slower. 

The time evolution of the inertia tensor reads 
%
\begin{align}
I (t) = R^{\top} I' R = 
R_{\beta }^{\top} R_{\alpha }^{\top} R_{\gamma }^{\top} I' 
R_{\gamma } R_{\alpha } R_{\beta }
\,,
\end{align}
%
but the matrices \(R_{\gamma }\) and \(R_{\gamma }^{\top}\) rotate the \(xy\) components of a matrix between each other: if \(I_1 = I_2 \) the components \(I'_{11} = I'_{22}\), so we have \(R_{\gamma }^{\top} I' = I' = I' R_{\gamma }\). So, we can write the expression as 
%
\begin{align}
I (t) = 
R_{\beta }^{\top} R_{\alpha }^{\top}  I' 
R_{\alpha } R_{\beta }
\,,
\end{align}
%
so the only time dependence which is left is inside \(\beta (t) = \Omega t\). 
We can expand the calculation, the result is given by Maggiore \cite[eq.\ 4.245]{maggioreGravitationalWavesVolume2007}. 
We are only interested in the projection of this variation onto the plane orthogonal to the direction of a propagation. 
The amplitudes in the two GW polarizations are also given by Maggiore \cite[eq.\ 4.246 -- 252]{maggioreGravitationalWavesVolume2007}.

% If we compute the evolution of the inertial tensor, we get terms both at \(\omega \) and at \(2 \omega \). 

What we find is both emission at \(\Omega \) and at \(2 \Omega \), while the frequency corresponding to the precession \(\omega_{p}\) does not appear: 
%
\boxalign{
\begin{align}
h_{+} &= h_0' \qty[ \sin (2 \alpha) \sin \iota \cos \iota \cos(\Omega t) 
+ 2 \sin^2\alpha \qty(1 + \cos^2\iota ) \cos(2\Omega t)]  \\
h_{ \times } &= h_0' \qty[\sin(2 \alpha  ) \sin \iota \sin(\Omega t)
+ 4 \sin^2 \alpha \cos \iota \sin(2 \Omega t)]  \\
h_0' &= - \frac{G}{c^{4}} \frac{I_3 - I_1 }{r} \Omega^2
\,,
\end{align}}
%
where this \(h_0'\) should be compared with \eqref{eq:typical-amplitude-rigid-nonprecessing-body}. 

\todo[inline]{Not sure about what comparison should be drawn: the formulas are the same with \(\omega_{s} \to \Omega \) and \(I_2 \to I_3 \)\dots}

We have four measurable amplitudes (corresponding to two polarizations and two frequencies), and we need to reconstruct the unknowns \(\alpha \), \(\iota \), \(r\) and \(I_3 - I_1 \).
This would in general be possible, however because of correlations we need to measure one more parameter externally (like the distance \(r\)).

To get an \textbf{intuition} for the biperiodicity: if we have a distribution which looks like a coin (\(I_1 \sim I_2 \ll I_3 \)) then it looks to us like a binary if we look at it from the top (in terms of periodicity at least), so we expect \(2 \omega \) emission, since the system looks the same to us after a rotation of \(\pi \). 

If, instead, we look at it from the side, the periodicity is the full period: after half a rotation the coin is edge-on (and this happens every \(\pi \)), but it will appear at two different angles with respect to the vertical direction, so the real periodicity is \(2 \pi \).
Therefore, we both have \(\omega \) and \(2 \omega \) emission. 

% If we were able to determine the amplitude at different inclinations, we would be able to determine the inclination \(\iota \). 

\subsubsection{Backreaction}

% [formula for back reaction is wrong!]

% In order to calculate the backreaction we assume that the motion is approximately constant during a single period. 
The radiated power is given by \cite[eq.\ 4.254]{maggioreGravitationalWavesVolume2007}: 
%
\begin{align}
\dv{E _{\text{rot}}}{t} &= - \frac{G}{5c^{5}} 
\expval{\dot{\ddot{M}}_{ij} \dot{\ddot{M}}_{ij}}  \\
&= - \frac{2G}{5c^{5}} 
(I_1 - I_3 )^2 \Omega^{6}
\sin^2\alpha  \qty(\underbrace{\cos^2\alpha}_{\mathclap{\text{at } \Omega }}  +\underbrace{16 \sin^2\alpha}_{\mathclap{\text{at }2 \Omega }} )
\,.
\end{align}
%
\todo[inline]{Wrong sign in the slides! }

So, we can see that the emission at \(\Omega \) is dominant for \(\alpha \sim 0\) (systems for which \(x_3\) and \(x_3'\) are almost aligned --- the ``coin seen head-on''), while the emission at \(2 \Omega \) is dominant for larger \(\alpha \) (the ``coin seen edge-on'').

The radiated angular momentum is instead given by 
%
\begin{align}
\dv{J}{t} &= - \frac{2G}{5c^{5}} \epsilon_{3jk}\expval{\ddot{Q}_{jl} \dot{\ddot{Q}}_{kl}}  \\
&= - \frac{4G}{5c^{5}} \expval{\ddot{M}_{1a} \dot{\ddot{M}}_{2a}}  \\
&= - \frac{2G}{5c^{5}} (I_1 - I_3 )^2 \Omega^{5} \sin^2\alpha \qty(\cos^2\alpha  + 16 \sin^2\alpha ) = \frac{1}{\Omega } \dv{E _{\text{rot}}}{t}
\,,
\end{align}
%
where we swapped \(Q\) for \(M\) since the terms \(\epsilon^{3kl} \delta_{ka} Q_{la} \) and \(\epsilon^{3kl} Q_{kl}\) do not contribute (by symmetry and tracelessness of \(Q\) respectively); also we integrated by parts.\footnote{To move from \(\expval{\ddot{M}_{1a} \dot{\ddot{M}}_{2a} - \ddot{M}_{2a} \dot{\ddot{M}}_{1a}}\) to \(2 \expval{\ddot{M}_{1a} \dot{\ddot{M}}_{2a}}\).} 

In order to understand how the rotation decays we need to express this in terms of the angles: recall the definition of \(\Omega = \dot{\beta} = J / I_1 \). This means that 
%
\begin{align}
\ddot{\beta} = \frac{1}{I_1 } \dv{J}{t} = 
- \frac{2G}{5c^{5}} \frac{(I_1-I_3)^2}{I_1 } \dot{\beta}^{5} \sin^2\alpha \qty(\cos^2\alpha  + 16 \sin^2\alpha )
\,,
\end{align}
%
which tells us that \(\dot{\beta} = \Omega \) is decreasing. 

To find the evolution of \(\alpha \) we need to write the rotational energy as 
%
\begin{align}
E _{\text{rot}} &= \frac{1}{2} I'_i \omega_{i}^{\prime 2} = \frac{1}{2} \frac{J_i^{\prime 2}}{I_i} 
= \frac{J^2}{2} \qty(\frac{\sin^2\alpha \sin^2 \gamma }{I_1 }
+ \frac{\sin^2 \alpha \sin^2 \gamma }{I_2 } + \frac{\cos^2 \alpha }{I_3 } ) \\
&= \frac{J^2}{2} \qty(\frac{\sin^2\alpha }{I_1 } + \frac{\cos^2 \alpha }{I_3 }) \marginnote{\(I_1 = I_2 \).}
\,,
\end{align}
%
which can be differentiated to yield 
%
\begin{align}
\dot{\alpha} = - \frac{2G}{5c^{5}} \frac{(I_1 - I_3 )^2}{I_1 } 
\dot{\beta}^{4} \sin \alpha \cos \alpha \qty(\cos^2\alpha + 16 \sin^2 \alpha )
\,,
\end{align}
%
so \(\alpha \) also decreases due to GW emission. 
This means that the wobble is decreasing, the rotation is aligning with the angular momentum. 

However, the combination \(J \cos \alpha \) is constant: 
%
\begin{align}
\dv{}{t} \qty(J \cos \alpha ) &= \dv{J}{t} \cos \alpha - J \dot{\alpha} \sin \alpha   \\
\begin{split}
&= - \frac{2G}{5c^{5}} (I_1-I_3)^2 \dot{\beta}^{5} \sin^2\alpha \qty(\cos^2\alpha  + 16 \sin^2\alpha ) \cos \alpha + \\
&\phantom{=}\ 
- \dot{\beta} I_1 \sin \alpha \qty(- \frac{2G}{5c^{5}} \frac{(I_1 - I_3 )^2}{I_1 } 
\dot{\beta}^{4} \sin \alpha \cos \alpha \qty(\cos^2\alpha + 16 \sin^2 \alpha ))
\end{split}  \\
&= 0
\,,
\end{align}
%
which means that \(\omega_{3}' = J \cos \alpha  / I_3 \) is a constant: this is the rotation speed of the body around its axis; \(J \cos \alpha \) is the projection of the angular momentum of the body around its axis \(x_3'\).

% We find differential equations telling us that \(\dot{\beta}
% \) and \(\alpha \) both decrease: the first means that the motion is slowing down; the second means that the wobbling is decreasing, as the rotation is aligning with the angular momentum.

\subsubsection{Backreaction}

In order to study the differential equations for \(\ddot{\beta}\) and \(\dot{\alpha}\) we can define the parameter \(u(t) = \dot{\beta} / \dot{\beta}_{0}\), and a characteristic time \(\tau_0 \): 
%
\begin{align}
\tau_0  = \qty( \frac{2G}{5 c^{5}} \frac{(I_1 - I_3 )^2}{I_1  } \dot{\beta}_{0}^{4})^{-1}
\,,
\end{align}
%
which has a typical value of 
%
\begin{align}
\tau_0 = \SI{1.8e6}{yr} \qty(\num{e-7} \frac{I_3 }{I_1 - I_3 })^2 \qty( \frac{ \SI{1}{kHz}}{f_0 })^{4} \qty(\frac{\SI{e38}{kg m^2}}{I_1 })
\,,
\end{align}
%
and we can write differential equations for \(\dot{u}\) and \(\dot{\alpha}\) as 
%
\begin{align}
\dot{u} &= - \frac{u^{5}}{\tau_0 } \sin^2\alpha  \qty(\cos^2 \alpha + 16 \sin^2\alpha ) \\
\dot{\alpha} &= - \frac{ u^{4}}{\tau_0 } \sin \alpha \cos \alpha 
\qty(\cos^2 \alpha + 16 \sin^2\alpha )
\,,
\end{align}
%
with initial conditions at the origin \(u(0) = 1\) (meaning \(\beta = \beta_0 \)) and \(\alpha(0) = \alpha_0 \). 

We have shown that \(J \cos \alpha \) is a constant: this means that we must have \(\dot{\beta} \cos(\alpha ) = \const\). 
This can aid us in the search of a steady state, by providing a constraint. The constant can be calculated at any time, so we compute it at \(t = 0\): then we get \(\dot{\beta}_{0} \cos(\alpha_0 ) = \cos(\alpha_0 )=  \const\).

This implies that the boundary condition at infinity must satisfy \(u_{ \infty } \cos(\alpha_{ \infty }) = \cos \alpha_0 \), which in general is different from 0 (unless \(\alpha_0 = \pi /2\), but it can be shown that this is an unstable equilibrium).

Having a steady state means that we require \(\dot{\alpha} = \dot{u} = 0\).
Since the factor \(\cos^2 + 16 \sin^2 \) is always positive for \(\alpha \in [0, \pi ]\) this can be either satisfied by \(u = 0\) or \(\sin \alpha = 0\), meaning \(\alpha = 0\) or \(\pi \).

The condition \(u = 0\) cannot in general obey the \(J \cos \alpha \) constraint, so we are left with \(\alpha = 0, \pi \).

\todo[inline]{Can we discard \(\alpha > \pi /2\) because otherwise we could just flip the axes until it became \(\alpha < \pi /2\)?}

So, we get the asymptotic state 
%
\begin{align}
\alpha_{ \infty } = 0 
\qquad \text{and} \qquad
u_{ \infty } = \cos \alpha_0 
\,,
\end{align}
%
and the way \(\alpha \) approaches this value asymptotically is 
%
\begin{align}
\dot{\alpha} \sim \alpha \frac{ u^{4}_{ \infty }  }{\tau_0 }
\qquad \text{as} \qquad
t \to \infty 
\,,
\end{align}
%
so asymptotically the decay is exponential, with a timescale \(\sim \tau_0 / u_{ \infty }^{4} \gtrsim \tau_0 \). 

\subsection{Observations}

The conditions we discussed do not apply in general: first of all, neutron stars are not truly rigid bodies since they have an internal structure. 
Even if they were, a generic rigid body's principal axes are all different. Maggiore discusses the triaxial case briefly \cite[pagg.\ 211--214]{maggioreGravitationalWavesVolume2007}.

In the triaxial case we will have emission at different frequencies, for example \(2 \omega_{r} \), \(2 \omega_{r} + \omega_{p} \), \(2 (\omega_{r} + \omega_{p})\).
There are even more: for each base frequency \(\omega \), radiation is emitted with decreasing amplitude for \(\omega + n \omega_{p}\), \(n \in \mathbb{N}\).

We have not seen pulsars yet in GW, but we can put upper bounds to the amplitude of their emission. 
What we do is to search for \textbf{quasi-stationary} GW signals close to known pulsars, accounting for the modulations due to the proper motion of the Earth, of the source etc.
There are about 400 pulsars in the LIGO-Virgo bandwidth which would be eligible for this kind of observation.

``Beating the spin-down limit'' means that we know that we would be able to see the GW emission in a certain case if the spin-down was only due to GW.
We have beaten it by a factor 10 for the Crab and Vela pulsars: a very tiny fraction \(\lesssim \SI{1}{\percent}\) of the rotational energy is lost to GW. 

\emph{Scorpius X-1} is low-mass X-ray binary: we see X-ray emission caused by accretion of the NS from the companion. 
This is a plausible mechanism for the deformation of the NS. 
We know its position and orbital period, not its spin frequency! 
The amplitude of its GW emission is expected to be of the order of \(h_0 \sim \num{5e-25}\).

\todo[inline]{What is this bit about? It does not seem really relevant.}

Could we differentiate a pulsar rotating and seen head-on and a binary system? 
Surely they are phenomena which happen in different frequency ranges, and last for different times. 
If the binary is spinning at those frequencies it's evolving very rapidly, instead a pulsar can give out a stable signal. 

Also, in full numerical relativity the waveform looks different. 

\end{document}
