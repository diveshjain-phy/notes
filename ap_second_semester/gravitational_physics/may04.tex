\documentclass[main.tex]{subfiles}
\begin{document}

\marginpar{Monday\\ 2020-5-4, \\ compiled \\ \today}

% We are discussing resonance and cavities in order to study the LASERs used in GW interferometry. 

We can have a lot of energy stored in the cavity, so that the power of the stored laser beam is much larger than the input power.

This does not violate conservation of energy: the energy is stored, but we cannot extract more power than what is coming in. 

% As the reflectivity increases, the peaks of intensity become sharper and sharper. 

We can treat the losses inside the cavity by introducing an intensity loss parameter, \(l_{I}\), which is the ratio of the lost power to the circulating power; then we can define
%
\begin{align}
r_{l} = 1 - l_{E} = \sqrt{1 - l_{I}}
\,,
\end{align}
% 
% which is derived from \(r_l^2 + l_I = r_l^2 + t_l^2 = 1\). 
and then each time \(r_2 \) appears we can substitute it by \(r_2 r_l\) --- this effectively amounts to attributing the power loss to the output coupler. 

Then, the fields inside and outside will be given by 
%
\begin{subequations}
\begin{align}
E_{c} &= E _{\text{in}} \frac{t_1 }{1 - r_1 r_2 r_{l} e^{-ik2L}}  \\
E_{r} &= - E _{\text{in}} \frac{r_1 - r_2 r_{l} e^{-ik2L}}{1 - r_1 r_2 r_{l} e^{-ik 2L}}
\,.
\end{align}
\end{subequations}

As we saw, the reflected electric field has a component which is due to the promptly reflected incoming field (\(E_{r, p}\)), and a component which is due to the loss of part of the circulating field (\(E_{r, l}\)): we can write this as \(E_r = E_{r, p} + E_{r, l}\). 

These are of the same order of magnitude: the mirrors are very reflective, but inside the cavity the power is much greater than outside. Since they are comparable, the interference of these two contributions is interesting. 

We can then distinguish three different behaviors by comparing the losses in the first mirror (described by \(r_1 \)) to the losses in the path and in the second mirror (described by \(r_2 r_l\)):
\begin{enumerate}
    \item the cavity is \textbf{overcoupled} when \(r_1 < r_2 r_{l}\) (mirror 1 does not reflect enough): then around resonance we will have \(E_{r, p } < E_{r, l}\);
    \item the cavity is \textbf{undercoupled} when \(r_1 >r_2 r_l\) (mirror 1 reflects too much): then around resonance we will have \(E_{r, p} > E_{r, l}\);
    \item the cavity is \textbf{impedence matched} when \(r_1 = r_2 r_{l}\), so that \(E_{r, p} = E_{r, l}\) around resonance.
\end{enumerate}

In this case the reflected intensity goes to zero, since by having the same intensity the two fields can destructively interfere --- they \emph{do} destructively interfere, since the phase difference is given by \(- e^{-ik2L} = -1\) if the cavity is tuned.
We also have a discontinuity in phase in the impedance matched case, but this is not an issue since the intensity goes to zero. 

\subsubsection{Beams}

Let us define a \textbf{beam} properly: its electric field at a fixed time will be in the form
%
\begin{align}
\vec{E}(x, y, z) = u (x, y, z) e^{-ikz}
\,,
\end{align}
%
where the function \(u(x, y, z)\) is a complex amplitude, and \(k = \omega_{l} / c\) is the wavenumber.
We are assuming that the beam propagates along the \(z\) axis.  

This must obey the wave equation \(\square E = 0\), which can be written as 
%
\begin{align}
\qty(\nabla^2 + k^2) E \propto
\pdv[2]{u}{x} +
\pdv[2]{u}{y} +
\pdv[2]{u}{z} - 2 ik \pdv{u}{z} - k^2 u + k^2 u = 0
\,.
\end{align}

The term \(\pdv*[2]{u}{z}\)
has magnitude much smaller than the other terms: so, we neglect it. 
This is called the \textbf{paraxial approximation} --- we are basically saying that as the wave propagates it diverges out in a fashion which is approximately linear in \(z\). 
This allows us to rewrite the wave equation as 
%
\begin{align}
\nabla^2_{t} u (\vec{s}, z) - 2ik \pdv{u(\vec{s}, z)}{z} = 0
\,,
\end{align}
%
where \(\nabla^2_t\) is the Laplacian operator only in the transverse directions, while \(\vec{s}\) are two coordinates such as \(x, y\) describing these directions. 

An approximate solution to this differential equation is given by the \textbf{Gaussian beam}\footnote{Notice that there is no propagation term \(\exp(i k z)\) since we factored it out earlier! The expression for \(E\) will include it.}
%
\begin{align}
u(x, y, z) = E_0 \frac{w_0 }{w(z)} \exp(-\frac{x^2 +y^2}{w^2(z)} -i \qty(kz + k\frac{x^2+y^2}{2 R(z)} - \psi (z)))
\,,
\end{align}
%
where we define the \textbf{beam waist} 
%
\begin{align}
w(z) = w_0 \sqrt{1 + \qty( \frac{z}{z_{r}})^2}
\,,
\end{align}
%
the \textbf{radius of curvature} of the wavefronts
%
\begin{align}
R(z) = z \qty(1 + \qty( \frac{z_{r}}{z})^2)
\,,
\end{align}
%
and the \textbf{Gouy phase}: 
%
\begin{align}
\psi (z) = \atan(z / z_r)
\,,
\end{align}
%
where the \textbf{Rayleigh range} \(z_r\) defines the scale of the distortions of the beam: it is the length after which the beam has doubled in area, and it is  given by 
%
\begin{align}
z_r = \frac{\pi w_0^2}{\lambda }
\,.
\end{align}

Qualitatively speaking, the beam is \emph{squeezed}: it is not a plane wave, its width can be made to be small in a certain region but the smaller it is there the faster it becomes wide outside it.

All the parameters can be determined if we give the width of the beam at the center \(w_0 \) and the wavelength of the light \(\lambda \). 

% The radius of the beam can only be small in a certain region, as we go further it widens. 

The Gouy phase means that passing through the narrow beam region gives   the beam a phase of \(\pi \). 

The divergence angle of the beam is given by \(\theta = w (z) / z \), and since we can approximate \(w(z) \approx w_0 z/z_r\) we can write \(\theta \approx w_0 / z_r = \lambda / \pi w_0 \), which is a sort of uncertainty principle: 
%
\begin{align}
w_0 \theta = \frac{\lambda}{\pi }
\,,
\end{align}
%
meaning that we have uncertainty between the width of the beam at its narrowest and its angle of dispersion.

Starting from the Gaussian beam, we can define orthonormal bases which describe any beam in the paraxial approximation: the Laguerre-Gauss (cylindrical, flower-looking) and Hermite-Gauss (rectangular, boxy-looking) bases.

These are eigenfunctions of the paraxial wave equation: they propagate without changing their shape (although they are scaled).
This is not the case for combinations of them, because of the Guoy phase.

For both bases, we have two ``quantum numbers'' \(l, m\) labelling the eigenfunctions.
We generally try to work with the 00 mode (which is a Gaussian in both bases), because it is easier. 

Resonance in a cavity means constructive interference of the beam with itself after a round-trip: we must have resonance both in phase and in the transverse profile. 

For the \textbf{phase} resonance, the round-trip length of the cavity must equal an integer times the wavelength, plus the Gouy phase. 
Because of the latter effect, different eigensolutions will resonate at different frequencies. 

For the \textbf{profile} resonance, the beam must come back from the round trip with the same spatial profile: this is selected through the geometry of the cavity. 
As the beam spreads the wavefronts curve, we need to account for this: the mirrors must be exactly parallel to the wavefronts. 
So, we need to make them curved. 

What is the length scale of this curvature?
For LIGO-VIRGO we have a \emph{confocal} cavity, so the sum of the curvatures of the mirrors is of the order of the length of the cavity. 
For the mirrors we use in the lab, their radii of curvature are from centimeters to meters. 

If a gaussian beam is reflected on the mirror and comes back to itself, one might think that all the higher orders modes should come back to themselves as well: they do not do so because of the Gouy phase difference.
The very high order modes are also suppressed since they are more affected by the imperfections of the mirror.
These are desired effects, we try to clean the beam until there is almost only the fundamental frequency.

\subsection{Realistic GW interferometers}

We can do a back-of-the-envelope calculation for the sensitivity of a Michelson interferometer.

We want to detect a GW amplitude whose amplitude spectral density is of the order 
%
\begin{align}
\frac{ \Delta L}{L} \approx \SI{e-21}{Hz^{-1/2}}
\,,
\end{align}
%
and since our interferometer is a few \SI{}{km} long this means that we must have 
%
\begin{align}
\Delta L \approx \SI{e-18}{m / \sqrt{Hz}}
\,,
\end{align}
%
so, since the light we use is in the visible to near-infrared range \(\lambda \sim \SI{e-6}{m}\), or equivalently  we want an interferometric sensitivity of around 
%
\begin{align}
\frac{\Delta L}{\lambda } \approx \SI{e-12}{Hz^{-1/2}}
\,.
\end{align}

The error in photon counting will be given by Poisson statistics, \(\Delta N \sim \sqrt{N}\); if we want it to be comparable to the signal then we must have 
%
\begin{align}
\frac{\Delta L}{\lambda } \approx \frac{\Delta N}{N} = \frac{1}{\sqrt{N}} = \SI{e-12}{Hz^{-1/2}}
\,,
\end{align}
%
which means we must get a number of photons per second the order of 
%
\begin{align}
N \approx \SI{e-24}{Hz}
\,.
\end{align}

This means we need a power of approximately 
%
\begin{align}
P = N \hbar \omega \approx \SI{200}{kW}
\,.
\end{align}

This is a very large requirement for a laser.
Fortunately, Fabry-Perot cavities are the solution: they increase the power inside the cavity by a factor \(\mathcal{F} / 2 \pi \), the average number of bounces a photon makes before being lost; they also useful to increase the effective length: we want \(L _{\text{arm}} \approx \SI{750}{km} \qty( \SI{100}{Hz}  / f _{\text{gw}})\), while the real arm length is of the order \SI{3}{km}. 

So, we have different reasons to introduce cavities: they increase power, they increase effective length. 

\end{document}
