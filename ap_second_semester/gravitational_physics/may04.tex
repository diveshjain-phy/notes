\documentclass[main.tex]{subfiles}
\begin{document}

\marginpar{Monday\\ 2020-5-4, \\ compiled \\ \today}

We are discussing resonance and cavities in order to study the LASERs used in GW interferometry. 

We can have a lot of energy stored in the cavity, so that the power of the stored laser beam is much larger than the input power.
This does not violate conservation of energy: the energy is stored, but we cannot extract more power than what is coming in. 

As the reflectivity increases, the peaks of intensity become sharper and sharper. 

We can treat the losses inside the cavity by introducing: 
%
\begin{align}
r_{l} = 1 - l_{E} = \sqrt{1 - l_{I}}
\,,
\end{align}
%
which is derived from \(r_l^2 + l_I = r_l^2 + t_l^2 = 1\). 

We have the fields inside and outside given by 
%
\begin{align}
E_{c} &= E _{\text{in}} \frac{t_1 }{1 - r_1 r_2 r_{l} e^{-ik2L}}  \\
E_{r} &= - E _{\text{in}} \frac{r_1 - r_2 r_{l} e^{-ik2L}}{1 - r_1 r_2 r_{l} e^{-ik 2L}}
\,.
\end{align}

The total reflected field inside the cavity can be overcoupled if \(r_1 < r_2 r_{l}\), so that the reflected field is less than the one outside; also we can have undercoupling and finally the \textbf{impedence matched} case is obtained when \(r_1 = r_2 r_{l}\). 

In this case the reflected intensity goes to zero. We also have discontinuity in phase, but this is not an issue. 

Let us define a \textbf{beam} properly: 
%
\begin{align}
\vec{E}(x, y, z) = u (x, y, z) e^{-ikz}
\,,
\end{align}
%
where we are neglecting the temporal dependence. 
The function \(u(x, y, z)\) is a complex amplitude. 

This obeys the wave equation, as long as 
%
\begin{align}
\qty(\nabla^2 + k^2) E = \nabla^2 u  - 2ik \pdv{u}{z}
\,.
\end{align}

The term 
%
\begin{align}
\pdv[2]{u}{z}
\,
\end{align}
%
has magnitude much smaller than the other terms. So, we neglect it. 
This is the paraxial approximation. The function is then given by 
%
\begin{align}
u(x, y, z) = E_0 \frac{w_0 }{w(z)} \exp(-\frac{x^2 +y^2}{w^2(z)} -i \qty(k \frac{x^2+y^2}{2 R(z)} - \psi (z)))
\,,
\end{align}
%
where we define 
%
\begin{align}
w(z) = w_0 \sqrt{1 + \qty( \frac{z}{z_{r}})^2}
\qquad \text{and} \qquad
R(z) = z \qty(1 + \qty( \frac{z_{r}}{z})^2)
\qquad \text{and} \qquad
\psi (z) = \atan(z / z_r)
\,,
\end{align}
%
so the beam is squeezed: it is not a plane wave. 

The factor \(z_{r} = \pi w_0^2 / \lambda \). 
The radius of the beam can only be small in a certain region, as we go further it widens. 

This also affects the phase, by the factor \(\psi (z)\).

The divergence angle is given by \(\theta \approx w (z) / z \approx \lambda / \pi w_0 \), which is a manifestation of the uncertainty principle: 
%
\begin{align}
w_0 \theta = \frac{\lambda}{\pi }
\,.
\end{align}

Starting from the gaussian beam, we can define orthonormal bases which describe any beam in the paraxial approximation: the Laguerre-Gauss (cylindrical) and Hermite-Gauss (rectangular) bases.

These are eigenfunctions of the paraxial wave equation: they propagate without changing their shape (although they are scaled). This is not the case for combination of them, because of the Guoy phase.

We generally try to work with the 00 mode, because it is easier. 

Resonance means constructive interference in the cavity: we must have resonance both in phase and in the transverse profile. 

As the beam spreads the wavefronts curve, we need to account for this: the mirrors must be exactly parallel to the wavefronts. 
So, we need to make them curved. 

For LIGO-VIRGO we have a \emph{confocal} cavity, so the sum of the curvatures of the mirrors is of the order of the length of the cavity. 

For the mirrors we use in the lab, their radii of curvature are from centimeters to meters. 

If a gaussian beam is reflected on the mirror and comes back to itself, all the higher orders modes come back to themselves as well. 
However, they do not do so because of the phase difference.

The very high order modes are suppressed, since they are more affected by the imperfections of the mirror.

We can do a back-of-the-envelope calculation for the sensitivity of a Michelson interferometer. We want to get 
%
\begin{align}
\frac{ \Delta }{L} \approx \SI{e-21}{Hz^{-1/2}}
\,,
\end{align}
%
so we want 
%
\begin{align}
\Delta L \approx \SI{e-18}{m Hz^{-1/2}}
\,,
\end{align}
%
so we want an interferometric sensitivity of around 
%
\begin{align}
\frac{\Delta L}{\lambda } = \frac{\Delta L}{\lambda } \approx \SI{e-12}{Hz^{-1/2}}
\,.
\end{align}

The relative error in photon counting is given by 
%
\begin{align}
\frac{\Delta L}{\lambda } \approx \frac{\Delta N}{N} = \frac{1}{\sqrt{N}}
\,,
\end{align}
%
which means we must get something on the order of 
%
\begin{align}
N \approx \SI{e-24}{Hz}
\,.
\end{align}

This means we need a power of approximately 
%
\begin{align}
P = N \hbar \omega \approx \SI{200}{kW}
\,.
\end{align}

Fabry-Perot cavities are also useful to increase the effective length: we want \(L _{\text{arm}} \approx \SI{750}{km} \qty( \SI{100}{Hz}  / f _{\text{gw}})\), while the arm length is of the order \SI{3}{km}. 

So, we have different reasons to introduce cavities: they increase power, they increase effective length. 

\end{document}
