\documentclass[main.tex]{subfiles}
\begin{document}

\subsubsection{Locking and alignment}

So far, we have always assumed that the cavity is at resonance. 

\marginpar{Monday\\ 2020-5-11, \\ compiled \\ \today}

We have an issue if noise is strong enough to move our mirrors out of lock, even if we do not care to observe GW at the frequency of that noise.

Our cavities have a finesse of \(\mathcal{F} \sim 500\), and the free spectral range will be of the order of \(c / 2L \sim \SI{50}{kHz}\). 
This means that the FWHM of the peaks will be of the order of \(\text{FWHM} \approx \text{FSR} / \mathcal{F} \approx \SI{100}{Hz} = \Delta f\).

If the variation of the frequency is due to a variation of the length of the cavity then the relative variations will be equal (at least to the linear level): \(\Delta L / L = \Delta f / f\), which means that, \emph{at the very least}, we will need to control the length of the cavity to the order of 
%
\begin{align}
\Delta L = L \frac{ \Delta f}{f} \approx \SI{e-9}{m}
\,.
\end{align}

In reality, we are able to control arm lengths to within \SI{e-15}{m} (root-mean-square of the position variation). 
There are many sources of noise in this respect: the seismic motion of the ground, the moon's pull, the intrinsic laser noise. 
Fortunately, there is a technique we can use to \textbf{keep} the cavity locked onto the wavelength of the laser. 

The first thing we might try is to measure the transmitted intensity of the laser light from the FP cavity to check whether the length is the correct one.
This has two issues: if the power decreases we cannot tell whether the cavity is slightly too \emph{long} or too \emph{short}; also, we cannot distinguish an intensity fluctuation due to a length imperfection from an intrinsic fluctuation of the laser. 

The solution is the \textbf{Pound-Drever-Hall} technique: an electro-optical modulator is used in order to insert sidebands at \(\omega_{l} \pm \Omega \) by doing phase modulation. 
These can be used as oscillators which detect any departure from resonance: they are \emph{not} at resonance in the cavity, so while a length fluctuation of the cavity affect the carrier frequency a lot, it leaves them basically unchanged. 
So, we see a term oscillating at \(\Omega \) whose amplitude is linear in \(\Delta \phi \), measuring it we can tell the sign of the length variation, and we can distinguish it from a laser power oscillation. 

If we know this, we can then actuate the cavity to follow the laser. 

Also, we need actuators to control the beam position: the angular control we need in order to prevent noise is of the order of \SI{e-9}{rad}.

\subsubsection{Antenna pattern}

The antenna pattern of the interferometer is described by the \textbf{detector tensor} \(D_{ij}\), which transforms the perturbation \(h_{ij}\) into the observed time-dependent scalar as \(h (t) = D_{ij} h_{ij}(t)\). It is given by 
%
\begin{align}
D_{ij} = \frac{1}{2} \qty(\hat{x}_{i} \hat{x}_{j} - \hat{y}_{i} \hat{y}_{j}) 
\,,
\end{align}
%
so the output of the detector will look like
%
\begin{align}
h(t) = \frac{1}{2} \qty(\ddot{h}_{xx} - \ddot{h}_{yy})
\,,
\end{align}
%
as long as the arms are aligned with the \(\hat{x}\) and \(\hat{y}\) axes. 
Note that this is similar to the antenna pattern of the resonant bar, but now we have two arms, as opposed to the single ``arm'' we had in that case. 
% The function \(D_{ij}\) describes the sensitivity of our detector in different directions. 

% These kinds of detectors return a scalar (a timeseries, yes, but a scalar with respect to 3D space). This scalar will be linear in the tensor \(h_{ij}\), so we can express the observation as 
% %
% \begin{align}
% h(t) = D_{ij} h_{ij} (t)
% \,.
% \end{align}

We must perform a rotation with two angles \(\phi \) and \(\theta \) to go from the orthogonal frame of the source and the orthogonal frame of the detector: it will look like 
%
\begin{align}
R = \left[\begin{array}{ccc}
\cos \phi  & \sin \phi  & 0 \\ 
- \sin \phi  & \cos \phi  & 0 \\ 
0 & 0 & 1
\end{array}\right]
\left[\begin{array}{ccc}
\cos \theta  & 0 & \sin \theta  \\ 
0 & 1 & 0 \\ 
-\sin \theta  & 0 & \cos \theta 
\end{array}\right]
\,,
\end{align}
%
\todo[inline]{Missing ones in the rotation matrices in the slides.}
and applying it (twice, once for each index of the perturbation tensor) we will find 
%
\begin{align}
h_{xx} &= h_{+} \qty(\cos^2\theta  \cos^2\phi - \sin^2 \phi ) + 2 h_{\times } \cos \theta \sin \phi \cos \phi   \\
h_{yy} &= h_{+} \qty(\cos^2\theta \cos^2\phi  - \cos^2\phi )
- 2 h_{\times}  \cos \theta \sin \phi \cos \phi 
\,,
\end{align}
%
so the output timeseries will look like 
%
\begin{align}
h(t) &=  F_{+} (\theta, \phi ) h_{+} + F_\times (\theta , \phi ) h_{\times }  \\
F_{+} (\theta, \phi ) &= \frac{1}{2} \qty(1 + \cos^2 \theta ) \cos 2 \phi  \\
F_{\times} (\theta, \phi ) &= \cos \theta \sin 2 \phi 
\,.
\end{align}

We have no way to distinguish these two components if we only have one detector; we must compare the outputs of different ones. 

\subsection{The interferometer's noise budget}

The main sources of noise in the interferometer are 
\begin{enumerate}
    \item \textbf{quantum noise}: it is not actually fundamental, we can decrease it by clever design;
    \item \textbf{seismic noise}: ground vibrations, this can be suppressed with better suspensions;
    \item \textbf{gravity gradients}: this is also a metric perturbation, so it is the hardest to work around, it is fundamental in a way;
    \item \textbf{thermal noise}, especially in the mirror coatings (which are exposed to hundreds of \SI{}{kW} of laser power!) is currently a big limiting factor.
\end{enumerate}

% The noise is dominated by the quantum noise, quantum fluctuations of the laser light. 
% The other source of noise giving us problems in the \SI{100}{Hz} region is the coating Brownian noise. 

At high frequencies, the problem is that it is hard to measure small displacements with small integration time. 
At low frequencies, the problem is that the mirrors move too much. 

\subsubsection{Quantum noise}

The \textbf{shot noise} is the error in the count of photons --- this is a Poisson process, since the photon arrivals are uncorrelated (the autocorrelation function is a \(\delta (t)\), the power spectrum is flat).
The power seen at the beamsplitter in an observation time \(T\) will look like 
%
\begin{align}
P_0  = \frac{N_\gamma \overline{h}\omega_{L}}{T} = \frac{\Delta E}{T}
\,.
\end{align}

Its square fluctuation will be given by
%
\begin{align}
\Delta P^2 = \frac{\Delta E^2}{T^2} = 
\frac{\Delta N^2 \hbar^2 \omega_{l}^2}{T^2}
= N \frac{\hbar^2   \omega_{l}^2}{T^2}
= \frac{P_0 \hbar \omega_{l}}{T}
= \frac{1}{2} \int_{0}^{1/T} S_P (\omega) \dd{\omega }
= \frac{1}{2} \frac{S_P(\omega )}{T}
\,,
\end{align}
%
so we get \(S_P (\omega ) = 2 P_0 \hbar \omega_{l}\). 
\todo[inline]{Wrong factor of 2 in the slides!}

Note that this is a power-PSD: it measures the average square \emph{power}, so it has the dimensions of a power squared over a frequency (\(\SI{}{W^2}/\SI{}{Hz} = \SI{}{W J}\)). 

The output of the detector, \(\Delta \phi \), is proportional to \(P_0\)..
It can be shown that the contribution to the noise PSD of the phase due to shot noise will be 
%
\begin{align}
\sqrt{S_{\Delta \phi , \text{ shot}} (\omega )}  = 
\frac{C}{P_0 } \sqrt{S_{P} (\omega )} = C \sqrt{\frac{2 \hbar \omega_{l}}{P_0 }}
\,,
\end{align}
%
where \(C\) is a dimensionless constant of order 1, accounting for the working point and the photodetector efficiency. 
We can refer this to the input by making use of the \(T_{FP}\) transfer function: we will then have 
%
\begin{align}
\sqrt{S_{h, \text{ shot}}} = \frac{\sqrt{S_{\Delta \phi, \text{ shot}}}}{T_{FP}} =
\frac{c}{8 \mathcal{F} L} \sqrt{\frac{4 \pi \hbar c \lambda_{l}}{P_0 }}
\sqrt{1 + \qty( \frac{f_{GW}}{f_{p}})^2}
\,.
\end{align}

This then diminishes as we increase the effective length of the cavity. 
So, one might say, why would we build a cavity which is several km long, instead of a tabletop experiment with a very high finesse? We shall answer shortly.

We also have \textbf{radiation pressure} noise, which scales differently: it is due to the fact that each photon impacting on the mirror gives it a bump of momentum \(2 \omega_{l} \hbar / c\). 
With a reasoning not unlike the previous one we find 
%
\begin{align}
\sqrt{S_{F, \text{ rp}}} = 2 \sqrt{\frac{2 P_0 \omega_{l}}{c^2}}
\,,
\end{align}
%
so the spectral density of the displacement of the mirror will be 
%
\begin{align}
\sqrt{S_{x, \text{ rp}}} = \frac{2}{M \omega^2} \sqrt{\frac{2 P_0 \hbar \omega_{l}}{c^2}} 
\,,
\end{align}
%
which means that the amplitude spectral density of the input will be 
%
\begin{align}
\sqrt{S_{h, \text{ rp}}} = \frac{16 \sqrt{2} \mathcal{F}}{ML (2 \pi f_{GW})^2} \sqrt{\frac{P_0 \hbar}{2 \pi c^2 \lambda_{l}}} \frac{1}{1 + (f_{GW} / f_{p})^2}
\,.
\end{align}

The interesting thing to note here is that this noise scales \emph{directly} with the finesse. 
This answers the question: if we try to raise the finesse too much, the power inside the laser increases by a lot, and this creates a huge amount of radiation pressure noise on the mirrors. 
The fact that we found a factor \(\mathcal{F}\) is due to the fact that a photon makes \(\mathcal{F} / 2 \pi \) bounces inside the cavity, creating noise for each. 

So, we must reach a compromise for the finesse (or, equivalently, for the circulating power).

The shot noise is flat in frequency, the RP noise decreases with frequency. 
At each frequency, we can define a Standard Quantum Limit, which is the lowest noise we could have at that frequency. 

We can go below this limit using Quantum Vacuum Squeezing. 

\subsubsection{Thermal Noise}

We have contribution from all dissipation sources. 
There is thermo-elastic noise: as a material bends, the side which compresses heats up a little. 

The limit is the internal dissipation: 
%
\begin{align}
S_{F, \text{th}} = 4 k_B T \Re[Z(\omega )]  
\,.
\end{align}

\subsubsection{Seismic noise}

\subsubsection{Newtonian noise}

\end{document}
