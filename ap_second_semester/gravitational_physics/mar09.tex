\documentclass[main.tex]{subfiles}
\begin{document}

\marginpar{Monday\\ 2020-3-9}

\section{Technical details}

Giacomo Ciani. Room 114 DFA
0498277036 or 0498068456
\url{giacomo.ciani@unipd.it}
Office hours: check by email.

Reading material: slides (to be used as an index of what is treated in the course), Hobson \cite[]{hobsonGeneralRelativityIntroduction2006a}, Michele Maggiore \cite[]{maggioreGravitationalWavesVolume2007,maggioreGravitationalWavesVolume2018}.

This is a general class on gravitational physics and GW, it does not really follow any textbook:
the field is young so there is no textbook covering all the necessary topics, really. 

The slides will be provided before lectures. There will be no home assignments. 

The idea for the exam is that formulas are important, detailed calculations and derivations are not. 

The target is to be able to read a research paper on GW and understand it. 
We will not go into very much detail on any topic:
the program of the class is very large. 

For the exam: it is a discussion of a GW paper (about \SI{25}{min}), plus theoretical questions --- focusing on the physical meaning, not on tedious derivations. It usually takes a bit less than an hour. 
The paper is optional. 

Off session exams are OK, if the exam is live then its is best if it is organized with several people (2-5 people). 

Please fill out the questionnaire on the course before taking the exam. 

\subsection{Topics}

Understanding \textbf{what gravitational waves are}: how they are described, how they are generated, what is their physical effect. 

% Some astrophysical and cosmological GW courses. 
% The professor's background is more in experimental physics than in astrophysics and cosmology. 

\textbf{Interactions} of GW with light and matter: ideas, techniques, experiments to detect GW, especially GW interferometers.

\textbf{Analysis} of GW signals. 

\textbf{What we can learn} from GW, overview of the most significant recent papers. 

\chapter{Gravitational waves}

\section{Introduction}

% What follows is a long, somewhat divulgative introduction.

Einstein thought the detection of GW impossible; at the time it was thought that they might be a coordinate artifact which could be ``gauged away''.

Now we can not only \emph{detect} them, we can actually \emph{observe} them, determining their position in the sky and their parameters.

They are a test of GR in \emph{extreme} conditions, where the weak-field approximation does not apply.
We can test the properties of matter in these extreme conditions, such as the equation of state for a neutron star.

GW are ``ripples'' in the metric of spacetime; their production is described by a quadrupole formula: the quadrupole is 
%
\begin{align}
Q_{jk} = \int \rho x_{j} x_{k} \dd[3]{x}
\,,
\end{align}
%
and then the perturbation propagates like 
%
\begin{align}
h_{jk} = \frac{2}{r} \dv[2]{Q_{jk}}{t}
\,.
\end{align}

What generates GW are non-spherically symmetric perturbations: by Jebsen-Birkhoff, if we have spherical symmetry there is no perturbation in the vacuum metric.
The simplest kind of object which can generate them is a binary system.

The effect of a GW is to  ``stretch'' space by squeezing one direction and stretching a perpendicular one, in an area-preserving way.
The typical relative scale of these perturbations is 
%
\begin{align}
\frac{\Delta L}{L} \sim \num{e-21}
\,,
\end{align}
%
which is \emph{really small}: if we multiply it by the radius of the Earth's orbit we get a length on the order of the size of an atom.

% An interesting thing which could emit in the \(\sim \SI{1}{Hz}\) range are extreme Mass Ratio inspirals: we have what is effectively a test particle in a strong gravitational field.

We have different kinds of interferometers for different GW frequency ranges: for now we have only used ground interferometers, but in the works there are also space detectors like LISA, Pulsar Timing Arrays at higher frequencies, and inflation probes.

In \textbf{binary systems}, we have different stages in the pulsation: an almost stationary one, the inspiral, the coalescence, and finally the ringdown.
The frequency and amplitude both increase up to the coalescence, after it the frequency is almost constant while the amplitude decreases.

In 1959, Joseph Weber proposed a ``\textbf{resonant bar}'' detector. These are based upon a sound principle, and this path was explored for several decades with, for example, AURIGA; the issue was that the sensitivity was insufficient, and these detectors would only be sensitive in a specific high frequency range.

GW were first detected indirectly using \textbf{Hulse-Taylor pulsars}: they measured the energy loss of a binary pulsar-NS system, which implied the loss of energy through gravitational wave emission.
The famous graph is not a fit line, it is the prediction based upon the measured orbital parameters.

Now we use ground-based laser \textbf{interferometers}: they are broad-band (a couple orders of magnitude, from \SI{10}{Hz} to \SI{1}{kHz}), they are inherently differential (as opposed to the single-mode excitation of a resonant bar). 

We can use Fabry-Pérot cavities in order to amplify effective length, by ``storing photons'' for several bounces in the interferometer. 
There is also a power recycling mirror in order for the light not to go back to the laser: with modern lasers and these systems we can get \SI{10}{kW} of power circulating in the cavities.

We can plot the sensitivity of the interferometers:
on the \(x\) axis we put the frequency of the incoming wave, and
on the \(y\) axis we put the amplitude spectral density \(h(f)\), which is measured in \SI{}{Hz^{-1/2}}. 

The curve describes where the noise dominates. 
We can plot both the theoretical sensitivity, with its various sources, and the measured one.

The signal comes out buried in noise, we must extract it in some way, like by correlating to a standard test signal.

A planned detector is \textbf{LISA}, which is a space interferometer but it works differently from the ground-based one, since it cannot reflect the beam back, and since the travel time for the beam is several seconds. 

Another way to detect low-frequency GW is by using \textbf{Pulsar Timing Arrays}: the idea is to monitor several millisecond pulsars, and compute the difference in the arrival times of their signals; distortions of spacetime will modify the relative distances between us and these. 

Also, we have \textbf{atomic interferometry} detectors, by having the wavefunctions of atoms do what light does in an interferometer.
This is not a mature technology as of now. 

We have seen several \textbf{BH-BH mergers} and some NS-NS ones, with masses between a few and about \(80 M_{\odot}\). With these, people are starting to do studies on the populations of these objects and their formation mechanisms. 
Unfortunately, with our frequency range we can only see a very short part of the signal, up to a few seconds at most. 

With \textbf{NS-NS mergers} we can investigate short gamma ray bursts, the formation of heavy elements, the mass of gravitational waves (which in GR is zero), the rate of expansion of the universe.

\end{document}
