\documentclass[main.tex]{subfiles}
\begin{document}

\marginpar{Monday\\ 2020-3-9}

\section{Introduction}

Giacomo Ciani. Room 114 DFA
0498277036 or 0498068456
\url{giacomo.ciani@unipd.it}
Office hour: check by email.

Reading material: slides (to be used as an index of what is treated in the course), Hobson, Michele Maggiore.

This is a general class on gravitational physics and GW, it does not really follow any textbook:
the field is young so there is not really a textbook. 

The slides will be provided before lectures. There will be no home assignments. 

The idea is that formulas are important, detailed calculations and derivations are not. 

The target is to be able to read a research paper on GW and understand it. 
We will not go into very much detail on any topic:
the program of the class is very large. 

For the exam: it is a discussion of a GW paper (about \SI{25}{min}), plus theoretical questions --- focusing on the physical meaning, not on tedious derivations. It usually takes a bit less than an hour. 
The paper is optional. 

Off session exams are OK, best if with several people (2-5 people). 

Please fill out the questionnaire on the course before taking the exam. 


\subsection{Topics}

Understanding what GW are: how they are described, how they are generated, what is their physical effect. 

Some astrophysical and cosmological GW courses. 
The professor's background is more in experimental physics than in astrophysics and cosmology. 

Interactions of GW with light and matter: ideas, techniques, experiments to detect GWs, especially GW interferometers.

Analysis of GW signals. 

What we can learn from GW, overview of the most significant recent papers. 

What follows is a long, somewhat divulgative introduction.

Einstein thought their detection impossible. 
Now we can not only \emph{detect} them, we can actually \emph{observe} them.

They are a test of GR in \emph{extreme} conditions, where the weak-field approximation does not apply.

We can derive properties of matter in these extreme conditions, such as the equation of state for a neutron star.

GWs are ``ripples'' in the metric of spacetime, described by a quadrupole formula: the quadrupole is 
%
\begin{align}
Q_{jk} = \int \rho x_{j} x_{k} \dd[3]{x}
\,,
\end{align}
%
and then the perturbation propagates like 
%
\begin{align}
h_{jk} = \frac{2}{r} \dv[2]{Q_{jk}}{t}
\,.
\end{align}

What generates GWs are non-spherically symmetric perturbations: by Jebsen-Birkhoff, if we have spherical symmetry there is no perturbation in the vacuum metric.

They ``stretch'' space by squeezing one direction and stretching a perpendicular one.

The typical relative scale of these perturbations is 
%
\begin{align}
\frac{\Delta L}{L} \sim \num{e-21}
\,,
\end{align}
%
which is \emph{really small}: if we multiply it by the radius of the Earth's orbit we get a length on the order of the size of an atom.

An interesting thing which could emit in the \(\sim \SI{1}{Hz}\) range are extreme Mass Ratio inspirals: we have what is effectively a test particle in a strong gravitational field.

We have different kinds of interferometers: for now we have used ground interferometers, there are also space detectors like LISA, Pulsar Timing Arrays at higher frequencies, and inflation probes (?).

In binary systems, we have different stages in the pulsation: an almost stationary one, the inspiral, the coalescence, and finally the ringdown.

In the early years, it was thought that GWs might be a coordinate artifact which could be ``gauged away''.

In 1959, Joseph Weber proposed a ``resonant bar'' detector. These are based upon a sound principle: one of the last ones was AURIGA, the issue was that the sensitivity was insufficient and they are only sensitive in a specific frequency range.

GWs were detected indirectly using Hulse-Taylor pulsars: they measured the energy loss of a binary pulsar-NS system, which implied the loss of energy through gravitational wave emission.

The famous graph is not a fit line, it is the prediction based upon the measured orbital parameters.

Now we use laser interferometers: they are broad-band (a couple orders of magnitude, from \SI{10}{Hz} to \SI{1}{kHz}), they are inherently differential (as opposed to the single-mode excitation of a resonant bar). 

We can use Fabry-Perrot cavities in order to amplify effective length. 
There is also a power recycling mirror in order for the light not to go back to the laser: modern lasers are on the order of \SI{100}{kW}, so there is a huge amount of power circulating in the cavities.

We can plot the sensitivity of the interferometers. 
On the \(x\) axis we put the frequency of the incoming wave. 
On the \(y\) axis we put the amplitude spectral density \(h(f)\), which is measured in \SI{}{Hz^{-1/2}}. 

\todo[inline]{Why?}

The curve describes where the noise dominates. 
We can plot both the theoretical sensitivity and the measured one.

The signal comes out buried in noise, we must extract it in some way, like by correlating to a standard test signal.

\end{document}
