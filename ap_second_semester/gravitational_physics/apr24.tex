\documentclass[main.tex]{subfiles}
\begin{document}

\marginpar{Friday\\ 2020-4-24, \\ compiled \\ \today}

% \todo[inline]{Recover ten minutes}
It is interesting to consider the fundamental mode for the oscillator, which is given by setting \(n =0\): this yields 
%
\begin{align}
\ddot{\xi}_{0} + \omega_0^2 \xi_{n} = \frac{2L}{\pi^2} \ddot{h}^{TT}_{xx}
\,,
\end{align}
%
where \(\omega_0 = \pi v_s /L\). 
The total energy in the bar can be recovered from this mode, by 
%
\begin{align}
E = \frac{M}{4} \qty(\dot{\xi}_{0}^2 + \omega_0^2 \xi_0^2 )
= \frac{1}{2} \int \dd{m} \qty(\dot{u}^2 + v_s^2 \qty(\pdv{u}{x})^2)
\,,
\end{align}
%
where \(M\) is the total mass of the bar. This basically amounts to integrating \(1/2 m v^2\), where \(v\) is calculated including both temporal and spatial variations. 

The equation for this fundamental mode has the same form as the one of a simple harmonic oscillator, whose mass is \(m_0 = M/2\), whose frequency is \(\omega_0 \) and which is subject to an effective force 
%
\begin{align}
F(t) = \frac{2  m_0  L}{\pi^2} \ddot{h}_{xx}^{TT} = \frac{LM}{\pi^2}
\ddot{h}^{TT}_{xx}
\,.
\end{align}

Like we did in the harmonic oscillator case we introduce a damping term to model the physical imperfections in the resonance: then the equation will read 
%
\begin{align} \label{eq:damped-zeroth-mode-resonant-bar}
\ddot{\xi}_{0} + \gamma_0 \dot{\xi}_{0} + \omega_0^2 \xi_0 = \frac{2L}{\pi^2} \ddot{h}^{TT}_{xx}
\,,
\end{align}
%
where \(\gamma_0 = \omega_0 / Q_0 \). 
After Fourier-transforming we find 
%
\begin{align}
\xi_0 (\omega ) 
&= \frac{2L}{\pi^2} \frac{\ddot{h}_{xx}^{TT} (\omega )}{\omega^2- \omega^2 - i \omega \gamma_0 }  \\
&= \underbrace{-\frac{2L}{\pi^2} \frac{\omega^2}{\omega^2- \omega^2 - i \omega \gamma_0 }}_{H_{h \to \xi_0 } (\omega )}  h_{xx}^{TT} (\omega )
\,,
\end{align}
%
so we have the expression from the transfer function, allowing us to quantify the response of the system to the oscillatory stimulus of the GW. 

Typical values for the parameters at play are a length of \(L \sim \SI{3}{m}\), a mass of \(M \sim \SI{2e3}{kg}\) and a speed of sound of the order \(v_s \sim \SI{5e3}{m/s}\), which means we have a characteristic frequency of the order of \(\omega_0 \sim \SI{5e3}{rad / s}\), so the long-wavelength approximation is valid. 

The value of the decay parameter \(Q_0 \) is of the order \num{e6} typically, which means that the characteristic time for an oscillation to die out is of the order \(\tau_0 \sim 1/ \gamma_0 = Q_0 / \omega_0 \sim \num{e2} \divisionsymbol \SI{e3}{s}\): oscillations need \textbf{several minutes to die out}. 

This only applies when \(\omega \) from the GW is \textbf{very close} to \(\omega_0 \), as the transfer function is very peaked. 
% Resonant bars are resonant: their transfer function is very much peaked. 

% The transfer function is the response to the strain as a function of signal frequency.

\subsubsection{Response to periodic signals}

If we have an incoming GW propagating along the \(z\) axis, while the bar is positioned along the \(x\) axis, then the GW amplitude will be described by 
%
\begin{align}
h_{xx}^{TT} (t, z) = h_0 \Re[\exp(-i \omega (t - z/c))]
\,,
\end{align}
%
so, fixing \(z =0 \) for simplicity (and without losing generality, it is only a phase shift) its second derivative is 
%
\begin{align}
\ddot{h}_{xx}^{TT} = - \omega^2 h_0 \Re[\exp(-i \omega t)]
\,.
\end{align}

A steady-state solution to the coupled differential equation \eqref{eq:damped-zeroth-mode-resonant-bar} is:
%
\begin{align}
\xi_0 (t) &= \frac{2L \omega^2 h_0 }{\pi^2}
\Re[\frac{\exp(-i \omega t)}{\omega^2 - \omega_0^2 + i \omega \gamma_0 }]
&= \frac{2 L \omega^2 h_0 }{\pi^2} \frac{(\omega^2 - \omega^2_{0}) \cos(\omega t) - \gamma_0 \sin(\omega t)}{(\omega^2-\omega_0^2)^2 + \omega^2 \gamma_0^2}
\,,
\end{align}
%
which has a simpler expression in Fourier space:
%
\begin{align}
\xi_0 (\omega ) = - \frac{2L}{\pi^2} \frac{\omega^2}{\omega_0^2 - \omega^2 - i \omega \gamma_0 } h_0 \delta (\omega )
\,.
\end{align}

\subsubsection{Response to bursts}

It is a fact from classical mechanics that the absorbed energy of an oscillator subject to an external impulsive force is 
%
\begin{align}
E 
= \frac{1}{2 m_0 } \abs{ \int_{ - \infty }^{ \infty  } F(t) e^{-i \omega_0 t} \dd{t}}^2
= \frac{ML^2}{\pi^{4} } \abs{ \int_{ - \infty }^{ \infty  } \ddot{h}_{xx}^{TT}(t) e^{-i \omega_0 t} \dd{t}}^2
\,,
\end{align}
%
and since the force is impulsive we can integrate by parts twice: so, we get 
%
\begin{align}
E = 16 M L^2 f_0^{4} \abs{h_{xx}^{TT} (f_0 )}^2
\,,
\end{align}
%
which is the energy of the GW component at frequency \(\omega_0 \): we can invert it as 
%
\begin{align}
\abs{h_{xx}^{TT} (f_0 )}^2 = \frac{1}{16L^2 f_0^{4}} \frac{E}{M}
\,.
\end{align}

The transfer function can be written as 
%
\begin{align}
\xi_0 (\omega ) = - \frac{2L}{\pi^2} \frac{\omega^2}{(\omega - \omega_{+}) (\omega-\omega_{-})} h^{TT}_{xx} (\omega )
\qquad \text{where} \qquad
\omega_{\pm} = \pm \sqrt{\omega_0 ^2 - \qty(\frac{\gamma_0 }{2})^2} - i \frac{\gamma_0 }{2}
\,.
\end{align}

We suppose that the incoming burst is a delta in time: then, it has components at all possible frequencies.

Then, we get 
%
\begin{align}
\xi_0 (t) = \frac{2L}{\pi^2} h_0 \tau_{GW} \int_{- \infty }^{ \infty } \frac{ \dd{\omega }}{2 \pi } \frac{\omega^2}{(\omega - \omega_{+}) (\omega-\omega_{-})} e^{-i \omega t} \approx \frac{2L }{\pi^2} h_0 \omega_0 \tau_{GW} e^{-\gamma_0 t / 2} \sin(\omega_0 t)
\,.
\end{align}

By saying that the signal is a delta, do we mean that \(\tau_{GW} \ll 1 / \omega_0 \)? YES, where \(\omega_0 \) is the frequency of the oscillator, which in our case is faster than the characteristic frequency of the dissipation. 

\subsection{Antenna pattern}

The sensitivity of the detector depends both on the direction from which che GW comes and the polarization.

Specifically, if \(\vec{J}\) is the orientation of the bar, then we have 
%
\begin{align}
h _{\text{out}} = \hat{J}^{i} \hat{J}^{j} h_{ij} (t)
\,.
\end{align}

If \(\theta \) is the angle between the bar direction and the source-Earth vector, we will have 
%
\begin{align}
h _{\text{out}} = h_{+} \sin^2\theta 
\,.
\end{align}

For a generic rotation, including a rotation of angle \(\varphi \) around the axis \(z'\) of the source system where the \(+\) and \(\times \) polarizations are defined, we get 
%
\begin{align}
h _{\text{out}} (t) = h_{+}' \sin^2\theta \cos 2 \varphi + h_{ \times }' \sin^2 \theta \sin 2 \varphi 
\,.
\end{align}

The response of the bar is quite small: we cannot really measure it; however we can transfer the energy to a lighter oscillator, which will move more. 

We need to tune the bar so that its resonant frequency is the same as the one of the other bar. 

The ratio of masses is \(\mu \), then we will have 
%
\begin{align}
A_t = \frac{A_0 }{\sqrt{\mu }}
\,,
\end{align}
%
and we typically can choose \(\mu \lesssim \num{e-4}\), if we go lower other noise sources dominate.

\subsection{Thermal and readout noise}

We saw that we can represent the stages of our processing by their transfer functions \(H_{i}(\omega )\). At each stage, though, we can get noise.

This noise is then processed by the later transfer functions. 

Say we have 
%
\begin{align}
y(\omega ) = H_1 (\omega ) H_2 (\omega ) H_3 (\omega ) x(\omega )
\,,
\end{align}
%
and we have noise after each stage. 

We can have thermal noise: noisy signals are described by a deterministic part plus a stochastic noisy part.

We can model this thermal noise using a Green's function: we get 
%
\begin{align}
x(t) = \dots
\,.
\end{align}

For thermal noise, we can assume that it is completely uncorrelated: it is due to billions of atomic interactions, so it loses memory quickly.
This means that the noise is white noise: the PSD is constant in freqency. Therefore, we can show that 
%
\begin{align}
\expval{x^2(t)} \approx \frac{A}{m_0^2 \omega_0^2} \qty(1 - e^{- \gamma_0 t})
\,.
\end{align}

Then, the average total energy of the mode is 
%
\begin{align}
\expval{E(t)} 
= \frac{1}{2} m_0 \omega_0^2 \expval{x^2(t)} + \frac{1}{2} m_0 \expval{\dot{x}^2(t)}
\approx \frac{A}{2 m_0 \gamma_0 } \qty(1 - e^{-\gamma_0 t})
\,.
\end{align}

So, we get asymptotically 
%
\begin{align}
\expval{E(t)} \to 2 k_B T m_0 \gamma_0 
\,,
\end{align}
%
which is the fluctuation-dissipation theorem: the PSD of the noise is 
%
\begin{align}
S_F(\omega ) = 4 k_B T m_0 \gamma_0 
\,.
\end{align}

We can always write 
%
\begin{align}
F(\omega ) = Z(\omega ) \dot{x}(\omega )
\,,
\end{align}
%
for any linear system.
So, the FDT gives us 
%
\begin{align}
S _{\text{F, th}} (\omega ) = 4 k_B T \Re{Z(\omega )}
\,.
\end{align}

This is in terms of the PSD of the force, which can be connected to that of the velocity.

If the energy of the DoF must be asymptotically constant, there needs to be a lossy mechanism giving energy to the DoF. 

\todo[inline]{Is there a problem with many transducers?}
\todo[inline]{Could we analyze different wavelengths by modifying the frequency}

The temperatures are of the order of milliKelvin, we cool them to close to zero by isolating them and making liquid helium go through everything. 

In this case the approximation of constant temperature is quite good.

\end{document}
