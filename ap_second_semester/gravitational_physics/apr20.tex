\documentclass[main.tex]{subfiles}
\begin{document}

\marginpar{Monday\\ 2020-4-20, \\ compiled \\ \today}

The Fourier transform is stochastic since the noise is stochastic: however, the PSD encompasses the statistical properties of the signal in a way that is stationary and well-defined. 

A physical system is a functional \(F\) which transforms input time series \(i_j (t)\) into output time series \(o_j (t) = F(i_j(t))\). 

Real systems are causal: there cannot be causality going backward in time. 

Also, often we can approximate systems as linear ones, as long as we work near a single point. 
Also, we can sometimes approximate them as stationary. 

Under all of these assumptions, we can express everything using an impulse response function: 
%
\begin{align}
o(t) = \int \dd{\widetilde{t}} i(\widetilde{t}) F( \delta (t - \widetilde{t})) 
= \int \dd{\widetilde{t}} h(t - \widetilde{t})
\,,
\end{align}
%
and this is a convolution: so we can express it as 
%
\begin{align}
o(\omega ) = i(\omega ) h(\omega )
\,.
\end{align}

The power spectral density then transforms as \(S_o (\omega ) = \abs{h(\omega )}^2 S_i (\omega )\), and if we have systems in series we can just multiply the impulse responses together. 

\subsection{Sampling}

Often we sample signals digitally.
Analogic systems are faster, but electronics are getting very fast as well and they are easier to use.

The signal is quantized in two ways: we quantize both in time by sampling at an interval \(t_s\) and in amplitude, by encoding it with a finite number of bits.

This introduces noise, which is however well-known and easy to calculate.

If we have a signal at a frequency \(f\) and we want to reconstruct it, we need to sample at a frequency \(2f\). 
If we sample at \SI{100}{Hz}, we can only accurately describe signals up to \SI{50}{Hz}. 
This is the Nyquist theorem.

This is true if we want to fit the data with the slowest sinusoid, if we know in which frequency range we should look.
If we work below the Nyquist frequency we can be sure of each frequency.

\section{Gravitational wave detection}

We can use an \textbf{elastic body} which resonates: we have the pro that this might enhance the effect of a GW through resonance and extend the duration of burst signals. 
However, it is only sensitive around its resonant frequency. 
Also, since it extends the signal it is hard to precisely reconstruct the time profile of the signal. 

These need to be isolated solid objects: they will fit in a lab, but the GW displacements are small. 

On the other side, we have \textbf{interferometers} which measure the distance between free falling masses. 

If we have a perfect harmonic oscillator with rest position \(x_0 \): then we will have 
%
\begin{align}
x(\omega ) = \frac{k x_0 + F _{\text{ext}}(\omega )}{k - m \omega^2}
\,,
\end{align}
%
while if there is damping we will have 
%
\begin{align}
m \ddot{x} = - k \qty(x(t) - x_0 (t)) - \beta \dot{x}(t) + F _{\text{ext}}(t)
\,,
\end{align}
%
which means we have 
%
\begin{align}
x(\omega ) = \frac{k x_0 + F _{\text{ext}}(t)}{k \qty(1 - \qty( \frac{\omega}{\omega_0 })^2 + \frac{i \omega \beta }{k})}
\,,
\end{align}
%
otherwise, we can have structural internal damping, which looks like 
%
\begin{align}
m \ddot{x} = - k (1 + i \delta ) \qty(x(t) - x_0 (t)) + F _{\text{ext}}(t)
\,.
\end{align}

The transfer function is more peaked for less damping. 

How do we see the effect of GW on an elastic body? Consider two masses, which are free falling, and connect them by a spring: they now will not move along geodesics.

We will have the equation 
%
\begin{align}
F _{\text{GW}} - k (L - \Delta x) = m \Delta \ddot{x}
\,.
\end{align}

This is given by 
%
\begin{align}
F = \frac{m}{2} \ddot{h}^{TT}_{xx} \Delta x \approx \frac{m}{2} L \ddot{h}^{TT}_{xx}
\,.
\end{align}

This, however, is only valid as long as \(L \ll \lambda _{\text{GW}}\). 

For a continuous bas, we will have 
%
\begin{align}
\dd{m} \qty(\pdv[2]{u}{t} - v_s^2 \pdv[2]{u}{x}) = \dd{F_x} = \dd{m} \frac{1}{2} x \ddot{h}_{xx}^{TT}
\,.
\end{align}
%

We further assume that 
%
\begin{align}
\eval{\pdv{u}{x}}_{x = \pm L/2} = 0
\,.
\end{align}

The general solution will be given by a sum of sines and cosines, but the cosines will move the center of the bar. 

We will have 
%
\begin{align}
u(t, x) = \sum _{n=0}^{ \infty } \xi_{n} \sin( \frac{\pi x}{L} (2 n + 1))
\,,
\end{align}
%
and we can take the scalar product in the \(L^{2}\) space with the basis sinusoids: so we find 
%
\begin{align}
\ddot{\xi}_{n} + \omega^2_{n} \xi_{n} = \frac{(-)^{n}}{(2n+1)^2} \frac{2L}{\pi^2} \ddot{h}^{TT}_{xx}
\,.
\end{align}

We have basically eliminated the spatial part. 
We can analyze the time evolution by itself. 

\end{document}
