\documentclass[main.tex]{subfiles}
\begin{document}

\marginpar{Tuesday\\ 2020-8-18, \\ compiled \\ \today}

This parametric equation is expressed with respect to the principal axes of the ellipse; if we want to write in a different coordinate system which is obtained from a rotation of the principal axes we must use the rotation matrix 
%
\begin{align}
\left[\begin{array}{c}
x \\ 
y
\end{array}\right]
=
\left[\begin{array}{cc}
\cos \chi  & -\sin \chi   \\ 
\sin \chi  &  \cos \chi 
\end{array}\right]
\left[\begin{array}{c}
x' \\ 
y'
\end{array}\right]
\,;
\end{align}
%
if we make \(x'\) and \(y'\) explicit we can find 
%
\begin{align}
x &= A \cos \beta \cos \chi \cos \omega t + A \sin \beta \sin \chi \sin \omega t \\
y &= A \cos \beta \sin \chi \cos \omega t - A \sin \beta \cos \chi \sin \omega t 
\,.
\end{align}

The expressions for \(x\) and \(y\) and the ones for \(E_x\) and \(E_y\) are quite similar: we must identify 
%
\begin{align}
\xi_1 \cos \phi_1 &= A \cos \beta \cos \chi \\
\xi_1 \sin \phi_1 &= A \sin \beta \sin \chi \\
\xi_2 \cos \phi_2 &= A \cos \beta \sin \chi \\
\xi_2 \sin \phi_2 &= -A \sin \beta \cos \chi 
\,,
\end{align}
%
which means that the electric field rotates in the shape of an ellipse, which is tilted about the \(x\), \(y\) axes by and angle \(\chi \). 

\subsubsection{Stokes parameters}

From these four equations, we find that (after working through the simplifications of the trigonometric functions)
%
\begin{align}
\xi_1 + \xi_2 &= A^2 \overset{\text{def}}{=} I \\
\xi_1 - \xi_2 &= A^2 \cos 2 \beta \cos 2 \chi  \overset{\text{def}}{=} Q \\
2 \xi_1 \xi_2 \cos(\phi_1 - \phi_2 ) &= A^2 \sin 2 \chi \cos 2 \beta \overset{\text{def}}{=} U \\
2 \xi_1 \xi_2 \sin(\phi_1 - \phi_2 ) &= A^2 \sin 2 \beta \overset{\text{def}}{=} V
\,,
\end{align}
%
where we defined the \textbf{Stokes parameters} \(I\), \(Q\), \(U\) and \(V\). They completely describe the state of an elliptically polarized monochromatic wave. 

We can reverse their definitions as 
%
\begin{align}
A &= \sqrt{I}  \\
\sin 2 \beta &= \frac{V}{I} \\
\tan 2 \chi &= \frac{U}{Q}
\,.
\end{align}

Now let us discuss the meaning of these parameters. Since \(A = \xi_0 \), \(I = \xi_0^2\), while the energy flux is given by 
%
\begin{align}
S = \frac{c}{4 \pi } \xi_0^2 = \frac{c}{4 \pi } I
\,.
\end{align}

So, the parameter \(I\) is directly proportional to the energy flux.
The major axes of the polarization ellipse are \(2A \cos \beta \) and \(2 A \sin \beta \) respectively; therefore their ratio \(\tan \beta \) measures the eccentricity of the ellipse, and the ratio \(V/I\) can be used to recover \(\beta \).

The ratio \(U/Q\) describes the orientation  of the ellipse with respect to the \(x\), \(y\) axes.
If the wave is linearly polarized then  \(V =0 \) (this is a degenerate case for our description); if \(V > 0\) the polarization  is left-handed, while if \(V< 0\) the polarization is right-handed.

If we have \(U = Q = 0\), this corresponds to circular polarization. 

The ellipse is fully determined by three numbers (\(\beta \), \(\chi \), \(A\)): why are there 4 Stokes parameters? They cannot all be independent if this is the case\dots  

In fact, they are connected by the relation 
%
\begin{align}
I^2= Q^2 + V^2+ U^2
\,.
\end{align}

The reason we use four parameters instead of just three is that this line of reasoning only holds for perfectly polarized monochromatic waves: in the general case, for partially polarized radiation, we have 
%
\begin{align}
I^2 \geq Q^2 + V^2 + U^2
\,.
\end{align}

For completely \emph{un}polarized light we have \(Q = V = U = 0\). 

A useful property is that these parameters are \textbf{additive}: the superposition of waves with different Stokes parameters can be the described by the sum of the Stokes parameters, as \(I _{\text{total}} = \sum _{k} I_k\) and so on. 

This suggests that we write them in a vector: 
%
\begin{align}
\vec{S} = \left[\begin{array}{c}
I \\ 
Q \\ 
U \\ 
V
\end{array}\right]
\,.
\end{align}

In general, we can decompose radiation which is partially polarized into a completely unpolarized part and a completely polarized part: 
%
\begin{align}
\vec{S} = \underbrace{\left[\begin{array}{c}
I - \sqrt{Q^2+V^2+U^2} \\ 
0 \\ 
0 \\ 
0
\end{array}\right]}_{\text{unpolarized}}
+ 
\underbrace{\left[\begin{array}{c}
\sqrt{Q^2+V^2+U^2} \\ 
Q \\ 
U \\ 
V
\end{array}\right]}_{\text{polarized}}
\,.
\end{align}

This allows us to define the \textbf{polarization degree}: the fraction of the radiation which is polarized, calculated as 
%
\begin{align}
\Pi_{L} = \frac{I _{\text{polarized}}}{I} = \frac{\sqrt{Q^2+U^2+V^2}}{I}
\,,
\end{align}
%
which is measurable with an instrument.

\subsubsection{Electromagnetic potentials}

We can express the electric and magnetic fields through the scalar and vector potentials \(\phi \) and \(\vec{A}\): 
%
\begin{align}
\vec{B} = \vec{\nabla} \times \vec{A} 
\qquad \text{and} \qquad
\vec{E} = - \vec{\nabla} \phi - \frac{1}{c} \pdv{\vec{A}}{t}
\,.
\end{align}

In terms of these potentials Maxwell's equations read 
%
\begin{align}
\nabla^2 \phi - \frac{1}{c^2} \pdv[2]{\phi }{t} &= - 4 \pi \rho \\
\nabla^2 \vec{A} - \frac{1}{c^2} \pdv[2]{\vec{A}}{t} &= - \frac{4 \pi }{c} \vec{j} 
\,.
\end{align}

The potentials are redundant, we can impose some conditions on them while still being able to describe any physical system. One we can impose is the \emph{Lorentz gauge}: 
%
\begin{align}
\vec{\nabla} \cdot \vec{A} + \frac{1}{c} \pdv{\phi }{t} = 0
\,,
\end{align}
%
under which we can get a formal solution for the potentials in terms of the charge and current densities: 
%
\begin{align}
\phi (\vec{r}, t) &= \int \frac{[ \rho ]}{\abs{\vec{r} - \vec{r}'}} \dd[3]{r'} \\
\vec{A} (\vec{r}, t) &= \frac{1}{c} \int \frac{[ \vec{j} ]}{\abs{\vec{r} - \vec{r}'}} \dd[3]{r'}
\,.
\end{align}

The square brackets are a notation meaning that the densities must be evaluated at the \emph{retarded time} \(t - \abs{\vec{r} - \vec{r}'} / c\). 
This accounts for the finite speed of the propagation of light. 

\subsubsection{Radiation from moving charges}

Consider a particle of mass \(m\) and charge \(q\) moving along a path described by the vector \(\vec{r}_0 (t)\). The charge and current densities will then be 
%
\begin{align}
\rho (\vec{r}, t) &= q \delta (\vec{r} - \vec{r}_0(t)) \\s
\vec{j} (\vec{r}, t) &= q \underbrace{\dv{\vec{r}_0}{t}}_{\vec{u}(t)} \delta (\vec{r} - \vec{r}_0(t))
\,.
\end{align}

From these expressions we can calculate the potentials generated by the moving charge. Let us introduce the relative radius \(\vec{R} = \vec{r} - \vec{r}_0\) and its corresponding unit vector \(\vec{n} = \vec{R} / \abs{\vec{R}}\). 

The potentials will then be 
%
\begin{align}
\phi = \qty[\frac{q}{kR}] \qquad \text{and} \qquad
\vec{A} = \qty[\frac{q \vec{n}}{ckR}]
\,,
\end{align}
%
where \(k = 1 - \vec{n} \cdot \vec{u} / c\), while \(R = \abs{\vec{R}}\).
These are called the \textbf{Liénard-Wiechert} potentials. 
From them we can perform the differentiation in order to recover the fields, which will be called the Liénard-Wiechert fields. 
The final expressions for the fields are 
%
\begin{align}
\vec{E} (\vec{r}, t) &= q \qty[\frac{(\vec{n} - \vec{\beta}) (1- \beta^2)}{ k^3 R^2}] + \frac{q}{c} \qty[\frac{\vec{n}}{k^3R} \times \qty((\vec{n} - \vec{\beta}) \times \dot{\vec{\beta}})] \\
\vec{B}(\vec{r}, t) &= \qty[\vec{n} \times \vec{E}]
\,,
\end{align}
%
where \(\vec{\beta} = \vec{u} / c\).
The important thing to notice here is that there are two contributions to the electric field: one goes as \(R^{-2}\) and is called the \textbf{Coulomb field} \(\vec{E}_c\) and the other goes as \(R^{-1}\) and is called the \textbf{radiation field} \(\vec{E}_r\).

Now we will discuss the radiation field mainly. Since the magnetic field can be found by taking a product with the electric field, we can make the same decomposition there.

The radiation field \(\vec{E}_r\) must be perpendicular to \(\vec{n}\), and so must be \(\vec{B}_r\). So, they form an orthogonal triple. 

Let us consider some simple cases. 

\subsubsection{Nonrelativistic charge}

Let us first suppose that the particle is moving slowly, \(\abs{u}/ c = \abs{\beta } \ll 1 \): this means that \(\abs{\vec{n} - \beta } \sim 1\) and also \(k \sim 1\). 

With these simplifications we find that the ratio of the magnitudes of the two components of the electric field is  
%
\begin{align}
\frac{E_r}{E_c} \sim \frac{R \dot{u}}{c^2}
\,.
\end{align}

If the characteristic time across which the velocity of the particle changes is \(\tau \), then we have \(\dot{u} \sim u / \tau\). 
This time \(\tau \) will be associated with a characteristic frequency \(\nu = 1 / \tau \) and a characteristic wavelength \(\lambda = c / \nu \). 
\todo[inline]{\dots and this is the wavelength of the emitted radiation? is this always the case? it does not seem that obvious}

Then we will get 
%
\begin{align}
\frac{E_r}{E_c} \sim \frac{R u \nu }{c^2} \sim \frac{R u}{\lambda c}
\,,
\end{align}
%
so we can see that if 
%
\begin{align}
\frac{R}{\lambda } \lesssim 1
\,
\end{align}
%
then we have \(E_r / E_c \lesssim u /c\). This defines an inner region around the emitter for which the Coulomb field dominates; outside it the radiation field dominates. 

Typically, we are far enough from the emitters of radiation so that the radiation field dominates. 
Then, under the assumption \(\beta \ll 1\) we have  
%
\begin{align}
\vec{E}_r = \frac{q}{Rc^2} \qty[\vec{n} \times \qty(\vec{n} \times \dot{\vec{u}})]
\qquad \text{and} \qquad
\vec{B}_r = \qty[\vec{n} \times \vec{E}_r]
\,.
\end{align}

\end{document}
