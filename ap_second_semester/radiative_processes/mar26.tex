\documentclass[main.tex]{subfiles}
\begin{document}

\marginpar{Tuesday\\ 2020-8-18, \\ compiled \\ \today}

Let us go into some more details regarding how the radiation field looks. 
Let us suppose that the angle between the acceleration \(\dot{\vec{u}}\) and the observation unit vector \(\vec{n}\) is \(\Theta \): then the absolute value of the electric field and magnetic fields will be 
%
\begin{align}
E_r = \frac{q}{Rc^2} \dot{u} \sin \Theta = B_r
\,.
\end{align}

\(\vec{E}\) and \(\vec{B}\) are orthogonal, so the magnitude of the Poynting vector will be 
%
\begin{align}
\abs{\vec{S}} = \frac{c}{4 \pi } \abs{\vec{E} \times \vec{B}} =
\frac{c}{4 \pi } E_r^2 = \frac{q^2}{4 \pi c^3 R^2} \dot{u}^2 \sin^2 \Theta 
\,,
\end{align}
%
which, as is expected, decays like \(R^{-2}\), since it is the power radiated per unit area. If we instead want the power radiated per unit solid angle then we must use \(\dd{A} = R^2 \dd{\Omega }\): 
%
\begin{align}
\frac{ \dd{w}}{ \dd{t} \dd{\Omega }} = SR^2 = \frac{q^2}{4 \pi c^3} \dot{u}^2 \sin^2 \Theta 
\,.
\end{align}

Integrating this over the solid angle allows us to calculate the total power emitted: 
%
\begin{align}
\dv{w}{t} &= \int \frac{q^2}{4 \pi c^3} \dot{u}^2 \sin^2 \Theta \dd{\Omega }  \\
&= \frac{q^2 \dot{u}^2}{2 c^3} \underbrace{\int_{-1}^{1} (1 - \mu^2) \dd{\mu }}_{= 2 - 2/3}
\marginnote{Reparametrized \(\mu = \cos \Theta \), integrated over \(\phi \).}  \\
&= \frac{2q^2 \dot{u}^2}{3 c^3}
\,,
\end{align}
%
which is the well-known \textbf{Larmor formula}. 

What we want to do now is to generalize this result to a system of \emph{many charges}. 
Each of these will have a charge \(q_i\), a distance \(R_i\) from the observer, an acceleration \(\dot{u}_i\).
In principle the generalization is simple: the electromagnetic field obeys the superposition principle, so we can just add the \(E_{i, r}\) from all the charges together.

The issue is that the retarded time for each of the particles is  slightly different from that of another particle. 
This is a treatable problem, but it makes the calculation more cumbersome. 

What we can do in order to mitigate it is to make the \textbf{dipole approximation}. 
Suppose that the system of \(N\) charges is contained within a volume whose characteristic length scale is \(L\). 
Also suppose that \(\tau \) is the typical scale across which the system evolves. 
If the evolution time \(\tau \) is slow compared to \(L / c\), then we can ignore the differences between the retarded times. 

Now, if the system changes significantly over a time \(\tau \) then the electric field also changes significantly over that time, so we can estimate the frequency of the emitted radiation as \(\nu \sim 1/ \tau \). 

What we are asking, \(\tau \gg L / c\), then becomes \(\nu \approx 1/\tau  \ll c/ L\), or equivalently \(\lambda \gg L\). 

We can also give an order of magnitude for the typical velocity of the particles: if \(\ell\) is the typical path length of a particle, then the typical velocity will be \(u \sim \ell / \tau \).  
This means that \(\ell / u = \tau\) must be larger than \(L / c\): rearranging, we have 
%
\begin{align}
\frac{\ell}{L} \gg \frac{u}{c}
\,,
\end{align}
%
but since \(\ell \lesssim L\) this means that \(u / c \ll 1\).

If these conditions are satisfied, then we can get the total electric field by adding all the electric field contributions from all the various particles: 
%
\begin{align}
\vec{E}_r &= \sum _{i} \vec{E}_{r, i} 
= \sum _{i} \frac{q_i}{R_i c^2} \vec{n}_i \times \qty(\vec{n}_i \times \dot{\vec{u}}_i)  \\ 
&= \sum _{i} \frac{1}{R_i c^2} \vec{n}_i \times \qty(\vec{n}_i \times q_i \dot{\vec{u}}_i) 
\,.
\end{align}

All of these will still need to be computed at the retarded time, although the retarded time will be the same for all the charges. 

If the length scale \(L\) of the system is smaller than all the distances \(R_i\), then we can approximate any \(R_i\) with a constant \(R_0\); also the unit vectors \(\vec{n}_i\) will be almost equal, and we can approximate them with a single \(\vec{n}\). 
\(R_0\) does not need to be the mean of the \(R_i\), the point is that all of them are quite similar so the distance to any point inside the source works. The reasoning is similar for \(\vec{n}\). 

If we make both of these approximations we find 
%
\begin{align}
\vec{E}_r &= \sum _{i} \frac{1}{R_0 c^2} \vec{n} \times \qty(\vec{n} \times q_i \dot{\vec{u}}_i)  \\
&= \frac{1}{R_0 c^2} \vec{n} \times \qty(\vec{n} \times \underbrace{\sum _{i} q_i \dot{\vec{u}}_i}_{ \ddot{d}})
\,,
\end{align}
%
where we define the dipole moment 
%
\begin{align}
\vec{d} = \sum _{i} q_i \vec{r}_i
\,.
\end{align}

Because the dipole moment appeared, this is called the \textbf{dipole approximation}. 
If \(\Theta \) is the angle between \(\vec{n}\) and \(\ddot{\vec{d}}\), then we have 
%
\begin{align}
E_r = \frac{ \ddot{d}   \sin \Theta }{R_0 c^2}
\,,
\end{align}
%
and with this we can calculate the power radiated per unit area 
%
\begin{align}
\frac{ \dd{w}}{ \dd{A} \dd{t}} = \frac{c}{4 \pi } E_r^2
\,,
\end{align}
%
and per unit solid angle: 
%
\begin{align}
\frac{ \dd{w}}{ \dd{t} \dd{\Omega }}= \frac{c}{4 \pi } E_r^2 R_0^2
= \frac{1}{c^3} \frac{ \ddot{d}^2}{4 \pi } \sin^2\Theta 
\,,
\end{align}
%
so if we integrate we can find the total emitted power, which generalizes the \textbf{Larmor formula}: 
%
\begin{align}
\dv{w}{t}= \frac{2 \ddot{d}^2}{3 c^3}
\,.
\end{align}

Now let us discuss the \textbf{spectral distribution} of the emitted energy. The energy per unit frequency and area will be 
%
\begin{align}
\frac{ \dd{w}}{ \dd{A} \dd{\omega }} = c \abs{\hat{E}_r(\omega )}^2
\,,
\end{align}
%
while the energy emitted per unit solid angle as usual is calculated by multiplying by \(R^2\): 
%
\begin{align}
\frac{ \dd{w}}{ \dd{\Omega } \dd{\omega }} = R^2 c \abs{\hat{E}_r (\omega )}^2 = \frac{1}{c^3} \sin^2\Theta \abs{ \hat{\ddot{d}} (\omega )}^2
\,,
\end{align}
%
where \(\hat{ \ddot{d} }\) is the Fourier transform of \(\ddot{d}\). 
This assumes that \(\Theta \) is fixed through time; we are allowing its magnitude but not its direction to change. 

Derivatives in the Fourier domain are multiplication by \(-i \omega \): so, we can recover the second derivative of the dipole moment in the time domain as 
%
\begin{align}
\ddot{d} (t) = - \int_{- \infty }^{\infty } \omega^2 e^{-i \omega t} \hat{d}(\omega ) \dd{\omega }
\,,
\end{align}
%
which means that we can express the electric field in the Fourier domain as 
%
\begin{align}
\hat{E} ( \omega ) = - \frac{\omega^2}{c^2 R} \hat{d}(\omega ) \sin \Theta 
\,,
\end{align}
%
so that the energy emitted per unit frequency and solid angle becomes 
%
\begin{align}
\frac{ \dd{w}}{ \dd{\Omega } \dd{\omega }}  = \frac{\omega^{4}}{c^3} \abs{\hat{d}(\omega  )}^2 \sin^2 \Theta 
\,.
\end{align}

If we integrate over the sphere we find 
%
\begin{align}
\dv{w}{\omega } = \frac{8 \pi }{3 c^2} \omega^{4} \abs{\hat{d}(\omega )}^2
\,.
\end{align}

Note that the spectrum of the radiation is heavily dependent on the frequency of the radiation. 

\subsection{Thompson scattering}

This is scattering of radiation by free electrons. 
Consider a plane, linearly polarized, electromagnetic wave impacting a free electron. The force onto the electron is the Lorentz force: 
%
\begin{align}
\vec{F} = q \qty(\vec{E} + \frac{\vec{v}}{c} \times \vec{B})
\,.
\end{align}

We know that \(\abs{E} = \abs{B}\) for the wave, so if the electron is nonrelativistic then the magnetic term is negligible: 
%
\begin{align}
\vec{F} \approx e \vec{E}
\,,
\end{align}
%


\end{document}
