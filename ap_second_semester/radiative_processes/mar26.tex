\documentclass[main.tex]{subfiles}
\begin{document}

\marginpar{Tuesday\\ 2020-8-18, \\ compiled \\ \today}

Let us go into some more details regarding how the radiation field looks. 
Let us suppose that the angle between the acceleration \(\dot{\vec{u}}\) and the observation unit vector \(\vec{n}\) is \(\Theta \): then the absolute value of the electric field and magnetic fields will be 
%
\begin{align}
E_r = \frac{q}{Rc^2} \dot{u} \sin \Theta = B_r
\,.
\end{align}

\(\vec{E}\) and \(\vec{B}\) are orthogonal, so the magnitude of the Poynting vector will be 
%
\begin{align}
\abs{\vec{S}} = \frac{c}{4 \pi } \abs{\vec{E} \times \vec{B}} =
\frac{c}{4 \pi } E_r^2 = \frac{q^2}{4 \pi c^3 R^2} \dot{u}^2 \sin^2 \Theta 
\,,
\end{align}
%
which, as is expected, decays like \(R^{-2}\), since it is the power radiated per unit area. If we instead want the power radiated per unit solid angle then we must use \(\dd{A} = R^2 \dd{\Omega }\): 
%
\begin{align}
\frac{ \dd{w}}{ \dd{t} \dd{\Omega }} = SR^2 = \frac{q^2}{4 \pi c^3} \dot{u}^2 \sin^2 \Theta 
\,.
\end{align}

Integrating this over the solid angle allows us to calculate the total power emitted: 
%
\begin{align}
\dv{w}{t} &= \int \frac{q^2}{4 \pi c^3} \dot{u}^2 \sin^2 \Theta \dd{\Omega }  \\
&= \frac{q^2 \dot{u}^2}{2 c^3} \underbrace{\int_{-1}^{1} (1 - \mu^2) \dd{\mu }}_{= 2 - 2/3}
\marginnote{Reparametrized \(\mu = \cos \Theta \), integrated over \(\phi \).}  \\
&= \frac{2q^2 \dot{u}^2}{3 c^3}
\,,
\end{align}
%
which is the well-known \textbf{Larmor formula}. 

What we want to do now is to generalize this result to a system of \emph{many charges}. 
Each of these will have a charge \(q_i\), a distance \(R_i\) from the observer, an acceleration \(\dot{u}_i\).
In principle the generalization is simple: the electromagnetic field obeys the superposition principle, so we can just add the \(E_{i, r}\) from all the charges together.

The issue is that the retarded time for each of the particles is  slightly different from that of another particle. 
This is a treatable problem, but it makes the calculation more cumbersome. 

What we can do in order to mitigate it is to make the \textbf{dipole approximation}. 
Suppose that the system of \(N\) charges is contained within a volume whose characteristic length scale is \(L\). 
Also suppose that \(\tau \) is the typical scale across which the system evolves. 
If the evolution time \(\tau \) is slow compared to \(L / c\), then we can ignore the differences between the retarded times. 

Now, if the system changes significantly over a time \(\tau \) then the electric field also changes significantly over that time, so we can estimate the frequency of the emitted radiation as \(\nu \sim 1/ \tau \). 

What we are asking, \(\tau \gg L / c\), then becomes \(\nu \approx 1/\tau  \ll c/ L\), or equivalently \(\lambda \gg L\). 

We can also give an order of magnitude for the typical velocity of the particles: if \(\ell\) is the typical path length of a particle, then the typical velocity will be \(u \sim \ell / \tau \).  
This means that \(\ell / u = \tau\) must be larger than \(L / c\): rearranging, we have 
%
\begin{align}
\frac{\ell}{L} \gg \frac{u}{c}
\,,
\end{align}
%
but since \(\ell \lesssim L\) this means that \(u / c \ll 1\).

If these conditions are satisfied, then we can get the total electric field by adding all the electric field contributions from all the various particles: 
%
\begin{align}
\vec{E}_r &= \sum _{i} \vec{E}_{r, i} 
= \sum _{i} \frac{q_i}{R_i c^2} \vec{n}_i \times \qty(\vec{n}_i \times \dot{\vec{u}}_i)  \\ 
&= \sum _{i} \frac{1}{R_i c^2} \vec{n}_i \times \qty(\vec{n}_i \times q_i \dot{\vec{u}}_i) 
\,,
\end{align}
%


\end{document}
