\documentclass[main.tex]{subfiles}
\begin{document}

\subsubsection{Angular distribution of emitted and received energy and power}

\marginpar{Friday\\ 2020-8-28, \\ compiled \\ \today}

We want to discuss how a relativistic particle emits radiation in the lab frame. 
We start off in the rest frame of the particle: there, it emits an energy \(\dd{w}'\) across a solid angle \(\dd{\Omega '} = \sin \theta ' \dd{\theta '} \dd{\varphi '} = \dd{\mu '} \dd{\phi '}\); in the lab frame we can say the same thing without the primes. 
We choose our reference frame for both lab and rest frame so that the particle is moving along the \(\hat{x}\) axis: \(\vec{v} = (v, 0, 0)\), while our observation direction is positioned at an angle \(\theta \) from the \(x\) axis. 

The energies \(\dd{w}\) and \(\dd{w}'\) are related by a Lorentz transformation, which in general reads 
%
\begin{align}
\dd{w} = \gamma \dd{w'} + \gamma v \dd{p'}_x
= \gamma \qty( \dd{w'} + \frac{v}{c} \cos \theta ' \dd{w'})
= \gamma \dd{w'} \qty(1 + \beta \mu')
\,.
\end{align}

In the second equality we have used the fact that the radiation four-momentum satisfies \(p^{\mu } p_{\mu } = 0\) since photons are massless, and therefore \((\dd{w'}, \dd{\vec{p}}')\) must be a null vector, meaning that \(\dd{w'} = \mu ' \dd{w'}/ c\).

We want to express this in terms of \(\mu \), the cosine of the angle in the lab frame. A geometric result we can derive is the \textbf{angular aberration formula}, 
%
\begin{align}
\mu = \frac{\mu ' + \beta }{1 + \beta \mu '}
\,,
\end{align}
%
which can be differentiated to yield 
%
\begin{align}
\dd{\mu } = \frac{ \dd{\mu '}}{\gamma^2 \qty(1 + \beta \mu ')^2}
\,.
\end{align}

The Lorentz boost does not affect the azimuthal direction since it is orthogonal to it, so \(\dd{\phi } = \dd{\phi '}\).
Therefore, we can transform the solid angle differential as 
%
\begin{align}
\dd{\Omega } = \dd{\mu } \dd{\phi } = \frac{ \dd{\mu '} \dd{\phi }' }{\gamma^2 ( 1 + \beta \mu ')^2} = \frac{ \dd{\Omega '}}{\gamma^2 ( 1 + \beta \mu ')^2}
\,.
\end{align}

The angular distribution of energy in the lab frame then can be expressed as 
%
\begin{align}
\dv{w}{\Omega } = \frac{\gamma (1+ \beta \mu ') \dd{w'}}{ \dd{\Omega '}} \gamma^2 (1 + \beta \mu ')^2
= \gamma^3 (1 + \beta \mu ')^3  \dv{w'}{\Omega '}
\,.
\end{align}

This is the distribution of the \emph{energy} emission: in order to find the \emph{power} emission we need to divide by time. However, we need to choose a time by which to divide: do we choose the one in the lab or rest frame? they are related by 
%
\begin{align}
\dd{t} = \gamma (1 - \beta \mu ) \dd{t}'
\,,
\end{align}
%
and it can be shown that the power emission in the two frames is related by 
%
\begin{align}
\frac{ \dd{w}}{ \dd{t} \dd{\Omega }}
= \frac{\gamma^3(1+ \beta \mu ')^3}{\gamma (1 - \beta \mu ) \dd{t'}} \dv{w'}{\Omega '}
= \frac{1}{\gamma^{4} (1- \beta \mu )^{4}} \frac{ \dd{w'}}{ \dd{\Omega '} \dd{t'}} 
\,.
\end{align}

Now, if radiation emission is isotropic in the particle frame then \(\dd{w'} / \dd{\Omega '}  \dd{t'}\) is independent of the angle. 
Under this assumption, the angular distribution of the received radiation is given by the prefactor. 

Let us analyze this for an ultra-relativistic particle, with \(\gamma \gg 1\).  Then, we can approximate 
%
\begin{align}
\beta = \sqrt{1 - \frac{1}{\gamma^2}} \approx 1 - \frac{1}{2 \gamma^2}
\,.
\end{align}

Inserting this into the prefactor yields 
%
\begin{align}
\frac{1}{\gamma^{4} (1 - \beta \mu )^{4}}
\approx \frac{1}{\gamma^{4} \qty[1 - \qty(1 - \frac{1}{2 \gamma^2}) \mu ]^{4}}
\,,
\end{align}
%
which in general will be small because of the \(\gamma^{-4}\) suppression. However, if \(\mu \sim 1\) then the quantity in square brackets is close to zero. We can expand \(\mu \sim 1 - \theta^2 /2\) since it is the cosine of \(\theta \): with this we can manipulate the expression into 
%
\begin{align}
\frac{1}{\gamma^{4} (1 - \beta \mu )^{4}} \sim \qty(\frac{2 \gamma }{1 + \gamma^2 \theta^2})^{4}
\,,
\end{align}
%
for small \(\theta \) at least. This is sharply peaked around \(\theta = 0\), with the FWHM of the peak of the order of \(1/\gamma \). 
This effect is \textbf{relativistic beaming}. 

So, how does this affect the emission from an accelerated particle? in its rest frame the emission follows Larmor's formula: 
%
\begin{align}
\frac{ \dd{w'}}{ \dd{t'} \dd{\Omega '}} = \frac{q^2 a^{\prime 2}}{4 \pi c^3} \sin^2 \Theta '
\,,
\end{align}
%
where \(\Theta' \) is the angle between the direction of emission and the direction of acceleration (in the rest frame). 

In the rest frame of the particle the acceleration \(\vec{a}'\) is orthogonal to the velocity \(\vec{v}\), since \(a^{\prime \mu } u_{\mu }\) vanishes identically and \(a^{\prime 0}\) is zero, therefore \(a' \cdot \vec{v} = 0\). 

The two polar angles \(\theta '\) and \(\phi '\) are chosen so that they define \(\vec{k}\), the direction of emission, while \(\theta'_a\) and \(\phi '_a\) define the direction of the acceleration, both with respect to the direction of motion \(\vec{v}\) which is chosen as the polar axis. 

It can then be shown by fiddling with the vectors that 
%
\begin{align}
\cos \Theta = \cos \theta ' \cos \theta '_a + \sin \theta' \sin \theta '_a \cos(\phi ' - \phi '_a)
\,,
\end{align}
%
however since \(a'\) is perpendicular to \(\vec{v}\) we have:
\todo[inline]{He says that \(a'\) being perpendicular to \(\vec{v}\) is a special case, but it can be proven by a general kinematic argument: is it not always true?}
%
\begin{align}
\cos \Theta = \sin \theta ' \cos \phi '
\implies \sin^2 \Theta = 1 - \cos^2 \Theta = 1 - \sin^2 \theta ' \cos^2 \phi '
\,,
\end{align}
%
and we can transform these into the lab frame as usual: 
%
\begin{align}
\sin^2 \theta ' &= 1 - \mu^{\prime 2 } = \frac{1 - \mu^2}{\gamma^2 ( 1- \beta\mu)^2} = \frac{\sin^2 \theta }{\gamma^2 (1 - \beta \mu )^2}  \\
\cos^2 \phi ' &= \cos^2 \phi 
\,.
\end{align}

Finally, we can substitute into the Larmor formula in order to find the angular distribution of the power which is seen by an external observer: 
%
\begin{align}
\frac{ \dd{w}}{ \dd{t} \dd{\Omega }}
&= \frac{1}{\gamma^{4} (1 - \beta \mu )^{4}}
\frac{ \dd{w'}}{ \dd{t}' \dd{\Omega }' }  \\
&= \frac{q^2 a^{\prime 2}_\perp}{4 \pi c^3}
\frac{1}{\gamma^{4} (1 - \beta \mu )^4}
\qty[1 - \frac{\sin^2 \theta  \cos^2\phi }{\gamma^2(1 - \beta \mu )^2}]  \\
&= \frac{q^2 a^{2}_\perp}{4 \pi c^3}
\frac{1}{ (1 - \beta \mu )^4}
\qty[1 - \frac{\sin^2 \theta  \cos^2\phi }{\gamma^2(1 - \beta \mu )^2}]  
\,,
\end{align}
%
note that we write \(a^{\prime 2}_\perp  \) just to specify that in that frame the acceleration is indeed perpendicular to the velocity.

\subsection{Synchrotron radiation}

This is the radiation emitted by a particle under the effect of a uniform magnetic field (with no electric field). 
Recall that the equation of motion will be 
%
\begin{align}
\dv{\vec{p}}{t} = \dv{[m \gamma \vec{v}]}{t} = \frac{q}{c} \vec{v} \times \vec{B}
\,,
\end{align}
%
while the time component of the equation tells us that 
%
\begin{align}
\dv{w}{t} = \dv{[m c^2 \gamma ]}{t} = q \vec{v} \cdot \vec{E}  =0 
\,.
\end{align}

This has an important consequence: \(\gamma \) is a constant, so the energy of the particle is conserved, the velocity of the particle has a constant modulus. Therefore, we can rewrite the equations of motion as 
%
\begin{align}
m \gamma \dv{\vec{v}}{t} = \frac{q}{c} \vec{v} \times \vec{B}
\,.
\end{align}

Let us decompose the velocity \(\vec{v}\) into two vectors, one orthogonal and one parallel to \(\vec{B}\): \(\vec{v} = \vec{v}_{\parallel} + \vec{v}_{\perp}\). 

Now, since the Lorentz force is always perpendicular to \(\vec{B}\) it cannot affect \(\vec{v}_\parallel\), which will then be a constant, while the perpendicular component will evolve as 
%
\begin{align}
\dv{\vec{v}_{\perp}}{t} = \frac{q}{mc \gamma } \vec{v}_{\perp} \times \vec{B}
\,.
\end{align}

This means that the motion \emph{along the field} is uniform, while the motion \emph{orthogonal to the field} is circular with constant angular velocity: the acceleration corresponding to \(\vec{v} _\perp\), \(\vec{a}_\perp\), is orthogonal to \(\vec{v}_\perp\) as well as to \(\vec{B}\).
The radius of the circle will then be defined by the equation 
%
\begin{align}
a_\perp = \frac{q}{mc \gamma } v_\perp B = \frac{v_\perp^2}{R}
\,,
\end{align}
%
and we can introduce the angular velocity as \(v_\perp = \omega _B R \): this is 
%
\begin{align}
\omega _B = \frac{v_\perp}{R} = \frac{qB}{m \gamma c}
\,.
\end{align}

From this we can calculate the radiated power from the relativistic Larmor formula: 
%
\begin{align}
P = \dv{w}{t} &= \frac{2 q^2}{3 c^3} \gamma^{4} a^2_\perp 
= \frac{2q^2}{3c^3} \gamma^{4} \frac{q^2B^2}{m^2c^2\gamma^2} v_\perp^2  \\
&= \frac{2}{3} \frac{r_0^2}{c} \gamma^2 B^2 v_\perp^2 
= \frac{2}{3} r_0^2 c \gamma^2 B^2 \beta_\perp^2
\,,
\end{align}
%
where \(r_0 \) is the classical particle radius. 

\end{document}
