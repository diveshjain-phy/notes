\documentclass[main.tex]{subfiles}
\begin{document}

\marginpar{Saturday\\ 2020-3-14, \\ compiled \\ \today}

\section{Radiative energy density}

We define the energy density per unit solid angle, \(u_{\nu }(\Omega )\) by: \(\dd{E} = u_{\nu } (\Omega ) \dd{V} \dd{\Omega } \dd{\nu }\).
This is the differential amount of energy in the volume \(\dd{V}\), carried by radiation coming from the solid angle \(\dd{\Omega }\) which has energies between  \(\nu \) and \(\nu + \dd{\nu }\). 

We consider a cylinder for our volume, its axis being aligned with the direction the radiation is coming from. 
Its volume can be expressed as \(\dd{V} =\dd{A} c \dd{t}\), where \(\dd{t}\) is the time taken by light to cross the height of the cylinder.

We can also express the differential energy using the definition of the specific intensity: then, we will have the following two equations: 
%
\begin{subequations}
\begin{align}
\dd{E} &= u_{\nu } (\Omega ) c \dd{A} \dd{t} \dd{\Omega } \dd{\nu }  \\
&= I_{\nu } \dd{A} \dd{t} \dd{\Omega } \dd{\nu }
\,,
\end{align}
\end{subequations}
%
therefore \(u_{\nu } = I_{ \nu } / c\).
Also, if we want to get the total energy density we just need to integrate over the volume of the whole sphere: 
%
\begin{align}
u_{\nu } = \frac{1}{c} \int I_{\nu }  (\Omega ) \dd{\Omega }
\overset{\text{def}}{=} \frac{4 \pi }{c} J_{\nu }
\,,
\end{align}
%
where \(J_{\nu  }\) is the \emph{mean intensity}: \(J_{\nu } = \expval{I_{\nu }}_{\Omega }\). 

We can also integrate over frequencies to get the total energy density: 
%
\begin{align}
u = \int u_{\nu } \dd{\nu } = \frac{4 \pi }{c} \int J_{\nu } \dd{\nu }
\,.
\end{align}

\subsection{Isotropic radiation field}

An isotropic radiation field is one for which the specific intensity does not depend on angles. 
Let us start from the definitions of \(u_{\nu }\) and \(P_{\nu }\): 
%
\begin{subequations}
\begin{align}
u_{\nu } &= \int \frac{I_{\nu }}{c} \dd{\Omega } = \frac{4 \pi J_{\nu }}{c}  \\
P_{\nu } &= \int  \frac{I_{\nu }}{c}  \cos^2\theta \dd{\Omega } =
\int \frac{I_{\nu }}{c} \cos^2\theta \sin \theta \dd{\theta } \dd{\varphi }
\,,
\end{align}
\end{subequations}
%
and let us use the assumption that \(I_\nu \) does not depend on \(\Omega \): so we can bring it out of the integrals, to find 
%
\begin{subequations}
\begin{align}
u_{\nu } &= 4 \pi  \frac{I_{\nu }}{c}  \\
P_{\nu } &= - 2 \pi  \frac{I_{\nu }}{c} \int \cos^2\theta \dd{\cos \theta } = 2 \pi \frac{I_{\nu }}{c} \int_{-1}^{1} x^2 \dd{x}  \marginnote{The differential is negative, but we swap the integration bounds.}\\
&= \frac{4 \pi }{3} \frac{I_{\nu }}{c}
\,,
\end{align}
\end{subequations}
%
which gives us the result we sought: 
%
\begin{align}
P_{\nu } = \frac{u_{\nu }}{3}
\,.
\end{align}

\subsection{Specific intensity along a ray}

We wish to see how the specific intensity \(I_{\nu }\) changes along a beam of light rays. Let us consider two positions \(1, 2\) along the beam, separated by a distance \(R\). Then, by definition we will have, for \(i =1, 2\):
%
\begin{align}
\dd{E_{i}} = I_{\nu, i} \dd{A_{i}} \dd{t_{i}} \dd{\Omega_{i}} \dd{\nu_{i}}
\,.
\end{align}

First of all we make the assumption of the gravitational field being weak: therefore \(\dd{t_{1}} = \dd{t_2 }\) and \(\dd{\nu_{1}} = \dd{\nu_2 }\). 
Now, we ask these two expressions to describe the same beam: the same photons will pass through \(\dd{A_1 }\) and  \(\dd{A_2 }\).  
Therefore, by conservation of energy, \(\dd{E_1 } = \dd{E_2 }\). 

This means that 
%
\begin{align}
I_{\nu, 1} \dd{A_1 } \dd{\Omega_1 } = 
I_{\nu, 2} \dd{A_2 } \dd{\Omega_2 }
\,.
\end{align}
%

We can treat the photons' motion as time-reversal symmetric: so, whether they pass through \(\dd{A_1 }\) or \(\dd{A_2 }\) first is irrelevant. 
So, since the linear scale of the differential are element is negligible compared to \(R\) we can consider all the photons which come through \(\dd{A_2 }\) to be coming from the apparent size of \(\dd{A_1 }\) from position \(2\). 
Therefore, the differential solid angle will look like 
%
\begin{align}
\dd{\Omega_2 } = \frac{ \dd{A_1 } }{R^2}
\,,
\end{align}
%
and we can apply the same reasoning reversing the photons' motion to find the same, alternate relation with \((1 \leftrightarrow 2)\). 
We can use this to write 
%
\begin{subequations}
\begin{align}
I_{\nu , 1} \frac{ \dd{A_1 }}{ \dd{\Omega_2 }} &= 
I_{\nu , 2} \frac{ \dd{A_2 }}{ \dd{\Omega_1 }}  \\
I_{\nu ,1 } R^2 &= I_{\nu , 2} R^2  \\
I_{\nu , 1} &= I_{\nu , 2}
\,.
\end{align}
\end{subequations}

This means that, under our assumptions, the specific intensity is conserved: 
%
\begin{align}
\dv{I_{\nu }}{s} = 0
\,,
\end{align}
%
where \(s\) is a parameter describing the light ray's trajectory.
This is useful since, if the variation of the specific intensity is zero in a vacuum, then its variation in the presence of matter will only be due to transfer phenomena, and the sign of the variation will describe whether energy is being added or removed. 

\section{Radiative transfer}

In general, as radiation passes through matter, its specific intensity changes.
This is due to emission and absorption, but also to scattering, which preserves the total number of photons: even in the low-energy limit it can change the angular distribution of the radiation, and in general it also changes the energy of the photon.

\subsection{Emission}

Emission is a process through which photons are created. 
We can define the grey emission coefficient \(j\) and the monochromatic emission coefficient \(j_{\nu }\) as: 
%
\begin{subequations}
\begin{align}
\dd{E} &= j \dd{V} \dd{\Omega } \dd{t}  \\
\dd{E} &= j_{\nu } \dd{V} \dd{\Omega } \dd{t} \dd{\nu }
\,,
\end{align}
\end{subequations}
%
they quantify the energy added to the radiation field per unit volume, solid angle (in order to account for the direction of emission) and unit time. For the monochromatic coefficient, we restrict ourselves to radiation emitted in the range from \(\nu \) to \(\nu + \dd{\nu }\).

In the case of an isotropic emission we can integrate over the solid angle to find  
%
\begin{align}
P_{\nu } = 4 \pi j_{\nu }
\,,
\end{align}
%
the radiated power per unit volume and frequency. 

Another useful concept is the emissivity \(\epsilon_{\nu }\): it is the energy added to the radiation field per unit time, frequency and mass in the direction described by \(\dd{\Omega }\). We express the infinitesimal mass as \(\dd{m}\), so that 
%
\begin{align}
\dd{E} = \epsilon_{\nu } \rho \dd{V} \dd{t} \dd{\nu } \frac{ \dd{\Omega }}{4\pi }
\,,
\end{align}
%
so that the emissivity and the emission coefficient are connected by 
%
\begin{align}
j_{\nu } = \frac{\epsilon_{\nu } \rho }{4 \pi }
\,.
\end{align}

We wish to describe the variation in specific intensity due to this emission. 
We consider a beam of cross section \(\dd{A}\) going through a length \(\dd{s}\), so that the volume it occupies is \(\dd{V} = \dd{A} \dd{s}\). 

Now, if we compare the definitions of \(j_{\nu }\) and \(I_{\nu }\) we find that they differ by a factor \(\dd{V} / \dd{A} = \dd{s}\), the length of the beam cylinder we defined. 

The difference between the specific intensities at the start and end of the cylinder would be zero without emission, now instead their difference can be calculated from the energy added; as we said most of the differentials simplify and we get that the variation of specific intensity is
%
\begin{align}
\dd{I_{\nu }} = j_{\nu } \dd{s}
\,.
\end{align}

\subsection{Absorption}

Absorption is described by a coefficient \(\alpha_{\nu }>0\), which is dimensionally an inverse length. 
The absorption law which defined the coefficient gives the decrease in radiative intensity for radiation of intensity \(I_{\nu }\) crossing an absorbing medium of length \(\dd{s}\):
%
\begin{align}
\dd{I_{\nu }} =- \alpha_{\nu } I_{\nu } \dd{s}
\,.
\end{align}

Why should the variation in intensity be proportional to the intensity itself? We give a simple argument: 
let us assume that absorption is due to randomly absorbers with number density \(n\) and (frequency dependent) cross section \(\sigma_{\nu }\).  

Let us consider our usual cylinder with cross sectional area \(\dd{A}\) and length \(\dd{s}\): the number of absorber in it will be \(\dd{N} = n \dd{A} \dd{s}\). 
The total effective cross section area presented for absorption will be \(\sigma_{\nu } \dd{N}\). 
The energy contained in photons in this cross sectional area will be lost: the energy lost \(- \dd{I_{\nu }}\) can be calculated as 
%
\begin{subequations}
\begin{align}
- \dd{I_{\nu }} \dd{A} \dd{t} \dd{\Omega} \dd{\nu }
&=  I_{\nu } \qty(\sigma_{\nu } n \dd{A} \dd{s}) \dd{t} \dd{\Omega } \dd{\nu }  \\
- \dd{I_{\nu }} &= - n \sigma_{\nu } I_{\nu } \dd{s}
\,,
\end{align}
\end{subequations}
%
which is the relation written above, with \(n \sigma_{\nu } = \alpha_{\nu }\). 
The number density is proportional to the mass density: \(n \overline{m} = \rho \) where \(\overline{m} \) is the average mass of a particle. Therefore, we can express \(\alpha_{\nu }\) as 
%
\begin{align}
\alpha_{\nu } = \rho \kappa_{\nu } = n \overline{m} \frac{ \sigma_{\nu }}{\overline{m}}
\,,
\end{align}
%
so we can see that \(\kappa_{\nu }\) is a cross sectional area per unit mass. It is called the \emph{mass absorption coefficient} or the \emph{opacity}. 

\paragraph{Conditions for validity}

This holds as long as the inter-absorber distances \(d \sim n^{-1/3}\) are large compared to the linear scale of the cross section \(\sigma_{\nu }^{1/2}\): we ask 
%
\begin{align}
\sigma_{\nu }^{1/2} \ll n^{-1/3}
\,,
\end{align}
%
and also we must assume that the absorbers are independent and randomly distributed (at least locally).
These assumptions are usually met in astrophysical systems. 

\subsection{The radiative transfer equation}

We can account for both absorption and emission in a combined equation for the derivative with respect to the beam length travelled \(s\) of the specific intensity \(I_{\nu }\): 
%
\begin{align}
\dv{I_{\nu }}{s} = - \alpha_{\nu } I_{\nu } + j_{\nu }
\,,
\end{align}
%
and we can see that in the absence of emission and absorption it is unchanged as we have shown before.
If \(j_{\nu }\) and \(\alpha_{\nu }\) are known we can integrate this differential equation to find the specific intensity. 

This will \emph{not} be the case when we will include scattering: the scattering term will not just depend on \(I_{\nu }\) but in its integral on the sphere, making this an integro-differential equation. 

\subsubsection{Solutions to the transfer equation in simple cases}

If there is only emission, that is, only \(j_{\nu }\) is nonzero, the intensity increases linearly in \(s\): 
%
\begin{align}
\dv{I_{\nu }}{s} = j_{\nu } \implies 
I_{\nu } = I_{\nu }(0) + \int_{s_0 }^{s} j_{\nu } (\widetilde{s}) \dd{\widetilde{s}} 
\,.
\end{align}

If there is only absorption, that is, only \(\alpha_{\nu }\) is nonzero, then the intensity decreases exponentially in \(s\):
%
\begin{align}
\dv{I_{\nu }}{s} = - \alpha_{\nu } I_{\nu } \implies 
I_{\nu } = I_{\nu }(s_0 ) \exp(- \int_{s_0 }^{s} \alpha_{\nu }(\widetilde{s}) \dd{\widetilde{s}})
\,.
\end{align}

\subsection{Optical depth and source function}

The optical depth \(\tau_{\nu }\) is defined so that it changes by 1 when the intensity of light at frequency \(\nu \) changes \(e\)-fold: its differential is
%
\begin{align}
\dd{\tau_{\nu }} = \alpha_{\nu } \dd{s}
\,,
\end{align}
%
so that the solution in the absorption-only case reads \(I_{\nu } \propto e^{-\int \dd{\tau }} = e^{-\tau }\).

So, a useful distinction to make is based on the magnitude of \(\tau \), since it quantifies how much light can shine through a medium: 
\begin{enumerate}
  \item if \(\tau \gg 1\) the medium is said to be \emph{opaque} or \emph{optically thick};
  \item if \(\tau \ll 1\) the medium is said to be \emph{transparent} or \emph{optically thin};
  \item if \(\tau \approx 1\) the medium is said to be \emph{translucent}. 
\end{enumerate}

If we define the source function 
%
\begin{align}
S_{\nu } = \frac{j_{\nu }}{\alpha_{\nu }}
\,,
\end{align}
%
we can write the radiative transfer equation as 
%
\begin{align}
\dv{I_{\nu }}{\tau_{\nu }} = - I_{\nu } + S_{\nu } \marginnote{Divided through by \(\alpha_{\nu }\), used definition of \(\tau_{\nu }\).}
\,.
\end{align}

\subsection{A formal solution of the radiative transfer equation}

We can solve this equation by defining \(Y_{\nu } = I_{\nu }e^{\tau_{\nu }}\), which obeys 
%
\begin{align}
\dv{Y_{\nu }}{\tau_{\nu }} = \dv{I_{\nu }}{\tau_{\nu }} e^{\tau_{\nu  }} + I_{\nu } e^{\tau_{\nu }}
\,,
\end{align}
%
so we can multiply the radiative transport equation to get 
%
\begin{subequations}
\begin{align}
\dv{I_{\nu }}{\tau_{\nu }} e^{\tau_{\nu }} &= - I_{\nu }e^{\tau_{\nu }} + S_{\nu } e^{\tau_{\nu }}  \\
\dv{Y_{\nu }}{\tau_{\nu }} &= S_{\nu } e^{\tau_{\nu }}  \\
Y_{\nu }(\tau_{\nu }) &= Y_{\nu } (0 ) +  \int_{0}^{\tau_{\nu }} S_{\nu } (\widetilde{\tau}_{\nu }) e^{\widetilde{\tau}_{\nu }} \dd{\widetilde{\tau}_{\nu }}  \\
I_{\nu }(\tau_{\nu }) &= I_{\nu } (0) e^{-\tau_{\nu }}
+ \int_{0}^{\tau_{\nu }} S_{\nu }(\widetilde{\tau}_{\nu }) e^{\widetilde{\tau}_{\nu } - \tau_{\nu }} \dd{\widetilde{\tau}_{\nu }} \marginnote{Divided through by \(e^{\tau_{\nu }}\).}
\,,
\end{align}
\end{subequations}
%
which has a direct intuitive meaning: the intensity at a certain point must be computed accounting for the initial one and emission all through the beam before the point we are considering, and each of these contributions to the emission is weighted by an exponential factor: the relevance of a term decreases if the optical distance increases. 

If \(S_{\nu } \) is a constant, we have the simplified expression 
%
\begin{align}
I_{\nu } (\tau_{\nu }) = I_{\nu }(0) e^{-\tau_{\nu }} 
+ S_{\nu } (\tau_{\nu }) \qty(1 - e^{-\tau_{\nu }})
\,,
\end{align}
%
and we can see that for large optical depths the intensity is dictated purely by the source at that point. 

\end{document}
