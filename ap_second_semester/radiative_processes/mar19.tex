\documentclass[main.tex]{subfiles}
\begin{document}

\subsubsection{The transfer equation for scattering}

\marginpar{Sunday\\ 2020-8-16, \\ compiled \\ \today}

If we only account for true emission and absorption the transfer equation looks like 
%
\begin{align}
\dv{I_\nu }{s} = - \alpha_{\nu } I_\nu + j_\nu 
\,.
\end{align}

For isotropic and conservative scattering we can modify this equation by adding on the scattering absorption and emission terms: 
%
\begin{align}
\dv{I_\nu }{s} = - (\alpha_{\nu } + \alpha_{\nu }^{(s)}) I_\nu + j_\nu + \alpha_{\nu }^{(s)} J_\nu 
\,.
\end{align}

Now we cannot give a formal solution anymore: the derivative of the intensity now depends not only on the intensity itself but on its average over all solid angles, \(J_\nu  = \expval{I_\nu }_{\Omega }\).
This will typically pose a problem: scattering is often the dominant phenomenon in radiative transfer for astrophysical systems. 

In order to solve the equation we can resort to numerical methods: for example we can use an iterative relaxation procedure in which we start off by computing the solution without scattering, calculate the mean intensity and plug it as a fixed value into the next iteration, and keep going. 

We can write the transfer equation as 
%
\begin{align}
\dv{I_\nu }{ s} 
= (\alpha_{\nu } + \alpha_{\nu }^{(s)}) 
\qty(- I_\nu + \frac{j_\nu + \alpha_{\nu }^{(s)} J_\nu }{\alpha_{\nu } + \alpha_{\nu }^{(s)}}) 
= (\alpha_{\nu } + \alpha_{\nu }^{(s)}) 
\qty(-I_\nu  + S_\nu )
\,,
\end{align}
%
where we define a new form for the source function: 
%
\begin{align}
S_\nu = \frac{j_\nu + \alpha_{\nu }^{(s)} J_\nu }{\alpha_{\nu } + \alpha_{\nu }^{(s)}}  
\,,
\end{align}
%
where, as long as Kirkhoff's law holds, we can substitute \(j_\nu = \alpha_{\nu } B_\nu \). 

The optical depth is derived from the \emph{total} absorption coefficient: 
%
\begin{align}
\dd{\tau_{\nu }} = (\alpha_{\nu } + \alpha_{\nu }^{(s)}) \dd{s}
\,.
\end{align}

With these definitions, we can write the transfer equation like before, 
%
\begin{align}
\dv{I_\nu }{\tau_{\nu }} = - I_\nu + S_\nu 
\,.
\end{align}

If there is scattering this is only apparently simple, the formulation only hides the complexity. 

\subsection{Mean free path}

We defined absorption as \(\alpha_{\nu } = n \sigma_{\nu }\): this is just the inverse of the mean free path, so we have (for true absorption):
%
\begin{align}
\ell_{\nu } = \frac{1}{\alpha_{\nu }}
\,,
\end{align}
%
and we can define a mean free path for scattering in the exact same way, with \(\alpha_{\nu }^{(s)}\). 
These will be the mean free paths of a photon before it undergoes that specific process, if we want to compute the mean free path of a photon before it undergoes \emph{either one} then we can just take the inverse of \(\alpha_{\nu } + \alpha_{\nu }^{(s)}\), the total absorption coefficient. 

Now, consider a medium with scattering, emission and absorption. 
Typically, a photon will be emitted, scatter a few times, and then be absorbed. Let us say that between emission and absorption it is scattered \(N\) times, and let us call the spatial intervals it travels between these scatterings \(\vec{r}_i\), where \(i\) goes from 1 to \(N\). 
The total distance travelled will look like \(\vec{R} = \sum _{i} \vec{r}_i\). 

The average of its square value will be 
%
\begin{align}
\expval{R^2} = \sum _{ij} \expval{\vec{r}_i \vec{r}_j}
= \sum _{i} \expval{r_i^2} + 2 \sum _{i<j} \expval{\vec{r}_i \vec{r}_j}
\,.
\end{align}

The mixed averages \(\expval{r_i r_j}\) evaluate to zero, since each scattering is isotropic and independent; on the other hand each of the \(\expval{r_i^2}\) is equal to \(\ell_\nu^2\), the square of the mean free path. 
This means that we have \(\expval{R^2} = N \ell_\nu^2\), meaning that the average distance occurring between each scattering is given by \(\ell_{*} = \sqrt{N} \ell_{\nu }\). 

Now, let us suppose that the photon is being scattered in a medium whose characteristic length is \(L\). Then typically a photon will need to be scattered \(N\) times before escaping the medium, where \(L = \sqrt{N} \ell_{\nu }\). 
This means that 
%
\begin{align}
\sqrt{N} = \frac{L}{\ell_{\nu }} = \alpha_{\nu } L \sim \tau_{\nu }
\,,
\end{align}
%
where the last order-of-magnitude relation comes from the fact that the \emph{differential} of the optical path is given by \(\dd{\tau }_{\nu } = \alpha_{\nu } \dd{s}\). 

This means that, at least in terms of order of magnitude, \(N \sim \tau_{\nu }^2\). This holds as long as the medium is optically thick, that it, \(\tau_{\nu }\) is larger (ideally much larger) than 1. 

What happens instead if the medium is optically thin, with \(\tau_{\nu } < 1\)? 
Then, we can neglect emission and scattering emission, and only consider scattering absorption. Then, the transfer equation will read 
%
\begin{align}
\dv{I_\nu }{\tau_{\nu }} = - I_\nu 
\,,
\end{align}
%
which is solved by \(I_\nu = I_\nu (0) e^{-\tau_{\nu }}\). 
This can also be written, by adding \(I_\nu (0)\) to both sides, as 
%
\begin{align}
I_\nu (0 ) - I_\nu  &= I_\nu (0) \qty(1 - e^{-\tau_{\nu }}) \\
\frac{I_\nu (0) - I_\nu }{I_\nu } &= 1 - e^{-\tau_{\nu }} \approx \tau_{\nu }
\,.
\end{align}

Now, the average number of scatterings will be less than 1 and will correspond to the average relative intensity lost, \(N = \Delta I_\nu / I_\nu (0)\). Therefore, in this case we will have \(N \sim \tau_{\nu }\). 

Putting together these two limiting cases, we can roughly say that we will have 
%
\begin{align}
N \approx \max \qty(\tau_{\nu },  \tau_{\nu }^2)
\,.
\end{align}

After a mean free path the photon can be either scattered or absorbed. 
The probability that it is absorbed as opposed to being scattered is given by 
%
\begin{align}
\epsilon_{\nu } = \frac{\alpha_{\nu }}{\alpha_{\nu } + \alpha_{\nu }^{(s)}}
\,,
\end{align}
%
while the probability that it is scattered is \(1 - \epsilon_{\nu }\), and this last quantity is typically called the \textbf{single scattering albedo}.

The average number of mean free paths travelled before absorption will be\footnote{This can be shown using the identity 
%
\begin{align}
\sum _{i=1}^{\infty } i p^{i-1} = \frac{1}{(1-p)^2}
\,,
\end{align}
%
which can be proven by differentiating the geometric series and bringing the derivative into the sum, which converges absolutely. 

Then, we can compute the average number of scatterings as 
%
\begin{align}
\expval{N} = \sum _{i=1}^{\infty } \epsilon_{\nu } i (1 - \epsilon_{\nu })^{i-1} = \frac{\epsilon_{\nu }}{(1 - (1 - \epsilon_{\nu }))^2} = \frac{1}{\epsilon_{\nu }}
\,.
\end{align}} 
%
\begin{align}
N = \frac{1}{\epsilon_{\nu }}
\,,
\end{align}
%
so we can make the following manipulation to find an explicit expression for the mean path between emission and absorption: 
%
\begin{align}
\ell_{*} = N \ell_{\nu }^2 = \frac{\ell_{\nu }^2}{\epsilon_{\nu }}
= \frac{1}{(\alpha_{\nu } + \alpha_{\nu }^{(s)} )^2} \frac{\alpha_{\nu }+ \alpha_{\nu }^{(s)}}{\alpha_{\nu }} 
= \frac{1}{\alpha_{\nu } (\alpha_{\nu } + \alpha_{\nu }^{(s)})}
\,.
\end{align}

This means that 
%
\begin{align}
\ell_{*} = \frac{1}{\sqrt{\alpha_{\nu } (\alpha_{\nu } + \alpha_{\nu }^{(s)})}}
\,.
\end{align}

Now we make use again of the order-of-magnitude relation \(\tau \sim \alpha L\) where \(L\) is the length scale of the medium; if we multiply \(L\) by \(1 / \ell_{*}\), the effective absorption coefficient, we get 
%
\begin{align}
\tau \sim \frac{L}{\ell_{*}} = \sqrt{L^2 \alpha_{\nu } (\alpha_{\nu } + \alpha_{\nu }^{(s)})} = \sqrt{\tau_{\nu } (\tau_{\nu } + \tau_{\nu }^{(s)})}
\,,
\end{align}
%
which is larger than \(\tau_{\nu }\) alone would be. This is called the \textbf{effective optical depth}; the fact that this is larger than \(\tau_{\nu }\) means that scattering traps the photon in the region for a longer time than it would remain there with absorption alone. 

\section{Radiative diffusion}

This is a way to approximately solve the radiative transfer equation under certain assumptions, the main one being that the medium should be very \textbf{optically thick}.

Another assumption we will make is the \textbf{plane-parallel} approximation: this means that the properties of the medium only vary with respect to one coordinate, which we will call \(z\). 
We will further assume that the properties of the radiation also only vary along \(z\). 

Now, this is about the \emph{spatial} dependence of the quantities, however certain ones like the radiation intensity \(I_\nu \) are also intrinsically vectors, so they also depend on the direction. 
However, we will have cylindrical symmetry for rotations around the \(z\) axis: therefore, the intensity will be a function of \(z\) and of the angle \(\theta \) between the ray and the \(z\) axis. 

The differential length travelled by the ray can be expressed in terms of the distance travelled along the \(z\) axis as
%
\begin{align}
\dd{s} = \frac{ \dd{z}}{\cos \theta } = \frac{ \dd{z}}{\mu }
\,,
\end{align}
%
where we define the usual shorthand, \(\mu  = \cos \theta \).

The radiative transfer equation reads 
%
\begin{align}
\dv{I_\nu }{s} = \qty(- I_\nu + S_\nu ) \qty(\alpha_{\nu } + \alpha_{\nu }^{(s)})
\,,
\end{align}
%
where we can substitute 
%
\begin{align}
\dv{I_\nu }{s} = \mu \pdv{I_\nu }{z}
\,,
\end{align}
%
and using this we can express the radiative transfer equation like 
%
\begin{align}
I_\nu (z, \mu ) = S_\nu - \frac{1}{\alpha_{\nu } + \alpha_{\nu }^{(s)}}
\mu \pdv{I_\nu }{z}
\,.
\end{align}

Now we make use of the assumption that the optical depth is large. This means that \(\tau \sim (\alpha_{\nu }+ \alpha_{\nu }^{(s)}) \ell\) is large, where \(\ell\) is the characteristic scale of the system. 

\todo[inline]{Should we not say explicitly that, besides being optically thick, the medium's thickness should be ``slowly varying''?}

This means that the term proportional to \(\pdv*{I_\nu }{z}\) is small compared to the source function, so we can apply perturbation theory. 

To zeroth order, assuming thermal equilibrium, we will have 
%
\begin{align}
I_\nu^{(0)} \approx S_\nu^{(0)} = B_\nu (T)
\,.
\end{align}

Then, the first-order approximation can be found by inserting the zeroth-order expression into the equation: 
%
\begin{align}
I_{\nu }^{(1)} (z, \mu ) = B_\nu - \frac{\mu }{\alpha_{\nu } + \alpha_{\nu }^{(s)}} \pdv{B_\nu }{z} 
\,.
\end{align}

An important thing to note is the linear relation between \(I_\nu \) and the cosine \(\mu \) of the angle. 

With this first-order result we can compute the flux: if we were to use the zeroth-order one we would get zero, since blackbody radiation is isotropic. 
The flux is given by the integral over the solid angle: 
%
\begin{align}
F_{\nu }(z) = \int I_\nu^{(1)} (z, \mu ) \mu \dd{\Omega }
= 2 \pi \int \dd{\theta } I_\nu^{(1)} (z, \mu ) \cos \theta \sin \theta 
\,,
\end{align}
%


\end{document}
