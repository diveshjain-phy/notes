\documentclass[main.tex]{subfiles}
\begin{document}

\subsection{Spectrum of synchrotron radiation}

\marginpar{Saturday\\ 2020-8-29, \\ compiled \\ \today}

We will use a rough, heuristic approach. 
Recall that the spectrum depends on 
%
\begin{align}
\frac{ \dd{w}}{ \dd{\omega } \dd{\Omega }} \propto \abs{\hat{E}(\omega )}^2
\,,
\end{align}
%
where \(\hat{E}(\omega )\) is the Fourier transform of the electric field \(E(t)\).

When we compute the Fourier transform we do so referring to the time as seen by the observer; we will see that the observer in any specific direction will only see the radiation for a very short amount of time each revolution because of relativistic beaming. 
In order to see this, consider a single charge moving along a helical path.  

Let us choose two points 1 and 2 along the trajectory, such that the path length between them is \(\Delta s\) and the angle difference is \(\Delta \theta \). Let us also denote the radius of curvature of the trajectory as \(a\): then we will have \(\Delta s = a \Delta \theta \). 

We know that radiation is emitted in a cone with aperture \(\sim 1/\gamma \): let us choose the two points so that they are at the edges of where the radiation starts and then stops being received by the observer.
Then, radiation emitted between the two points will always hit the observer.

We know that in this case \(\Delta \theta  = 2 / \gamma \) by the geometry of the setup, since \(2 / \gamma \) is the angle between two diametrically opposed points in the cone. 
The equation of motion is 
%
\begin{align}
m \gamma \dv{\vec{v}}{t} = \frac{q}{c} \vec{v} \times \vec{B}
\,,
\end{align}
%
so the modulus of the velocity change, for small enough time differences, is
%
\begin{align}
\frac{ \abs{ \Delta \vec{v}}}{ \Delta t} = \frac{q}{mc \gamma } v B \sin \alpha 
\,,
\end{align}
%
where \(\alpha \) is the angle between the velocity and \(\vec{B}\), which is called the \emph{pitch angle}. 
Since the modulus of \(\vec{v}\) is constant, we can write \(\abs{ \Delta \vec{v}} = v \Delta \theta\), so that 
%
\begin{align}
v \frac{ \Delta \theta }{\Delta t} = \frac{q}{mc \gamma } vB \sin \alpha 
\,,
\end{align}
%
but since the path travelled is given by \(\Delta s = v \Delta t\) we have 
%
\begin{align}
v^2 \frac{ \Delta \theta }{ \Delta s} &= \frac{q}{mc \gamma } v B \sin \alpha  \\
 \frac{ \Delta \theta }{ \Delta s} &= \frac{q}{mc v \gamma } B \sin \alpha \\
 \frac{ 2/ \gamma  }{ \Delta s} &= \frac{q}{mc v \gamma } B \sin \alpha \\
 \Delta s = \frac{2 mc v}{q B \sin \alpha }
\,,
\end{align}
%
which is related to the frequency \(\omega _B = q B / mc \gamma \) by 
%
\begin{align}
\Delta s = \frac{2 b}{\gamma \omega _B \sin \alpha  }
\,.
\end{align}

Then, we can derive the time needed to go from point 1 to point 2: 
%
\begin{align}
\Delta t = \frac{\Delta s}{v} = \frac{2}{\gamma \omega _B \sin \alpha }
\,,
\end{align}
%
however we must be careful: although this is the time as measured in the lab frame, the motion of the particle is highly relativistic and the radiation is emitted in two different places. 

Let us call \(t_1^A\) and \(t_2^A\) the times at which the radiation corresponding to  the start and the end of the pulse arrive at the observer. 
They will be given by 
%
\begin{align}
t_1^{A} = t_1 + \frac{L}{c} + \frac{\Delta s}{c} 
\qquad \text{and} \qquad
t_2^{A} = t_2 + \frac{L}{c}  
\,.
\end{align}

This means that 
%
\begin{align}
t_1^A- t_2^{A} = t_2 - t_1 - \frac{\Delta s}{c}  
= \frac{2}{\gamma \omega_B \sin \alpha } - 2 \frac{v/c}{\gamma \omega _B \sin \alpha }
= \frac{2}{\gamma \omega_B \sin \alpha } \qty(1 - \frac{v}{c})
\,,
\end{align}
%
but if the motion is highly relativistic we can approximate 
%
\begin{align}
1 - \frac{v}{c} \approx 1 - \qty(1 - \frac{1}{2 \gamma^2}) = \frac{1}{2 \gamma^2}
\,,
\end{align}
%
therefore 
%
\begin{align}
\Delta t^{A} = \frac{2}{\gamma \omega _B \sin \alpha } \frac{1}{2 \gamma^2} = \frac{1}{\gamma^3 \omega _B \sin \alpha }
\,,
\end{align}
%
so the reduction of the pulse duration is shorter by a factor \(\sim 1/ \gamma^3\) than \(1 / \omega _B\): it becomes \textbf{very short!}

Therefore, the spectrum will be rather broad in frequency, of the order \(\sim \gamma^3 \omega _B\). 

Let us define the \textbf{critical pulsation} and the \textbf{critical frequency}:
%
\begin{align}
\omega _c = \frac{3}{2} \gamma^3 \omega _B \sin \alpha 
\qquad \text{and} \qquad
\nu _c = \frac{3}{4 \pi } \gamma^3 \omega _B \sin \alpha 
\,,
\end{align}
%
which will roughly be around the cutoff of the spectrum. 

We can see by looking at the old formulas we derived that the dependence of the emitted power per unit solid angle on the angle is always in the form 
%
\begin{align}
\frac{ \dd{w}}{ \dd{t} \dd{\Omega }} = F(\gamma \theta )
\,,
\end{align}
%
which is a consequence of relativistic beaming. Therefore, this will also hold for the dependence of the electric field.

Now, we want to find out how \(\theta \) is related to \(t\): we know that \(\theta = s/ a\) while \(t = (s/v) (1 - v/c)\); so 
%
\begin{align}
t &= \frac{a \theta }{v} \qty(1 - \frac{v}{c}) \\
\gamma t &= \frac{\gamma  a \theta }{v} \qty(1 - \frac{v}{c}) \\
\gamma \theta &= \gamma t \frac{v}{a} \qty(1 - \frac{v}{c})^{-1}
\approx \gamma t \omega _B \sin \alpha 2 \gamma^2 \propto \omega _c t
\,.
\end{align}

Therefore, the electric field depends on time like \(E(t) \propto g(\omega _c t)\) for some function \(g\). 
Its Fourier transform will then look like 
%
\begin{align}
\hat{E}(\omega ) &\propto \int_{- \infty }^{\infty } g(\omega _c t) e^{-i \omega t} \dd{t}  \\
&\propto \int_{-\infty }^{\infty } g(z) e^{-i \frac{\omega}{\omega _c} z} \dd{z}
\,,
\end{align}
%
meaning that \emph{the dependence of \(\hat{E}(\omega )\) on \(\omega \) is only through the ratio \(\omega / \omega _c\).}

Recall that the power per unit solid angle is given by 
%
\begin{align}
\frac{ \dd{w}}{ \dd{\omega } \dd{\Omega }} \propto \abs{\hat{E}(\omega )}^2
\,,
\end{align}
%
so if we integrate over the solid angle and divide by the period we find that the power per unit frequency is 
%
\begin{align}
\frac{ \dd{w}}{ \dd{t} \dd{\omega }} = c_1 F \qty(\frac{\omega }{\omega _c} )
\,,
\end{align}
%
where \(c_1 \) is some constant and \(F\) is some function. 
We can evaluate \(c_1 \) since we know what is the total radiated power: 
%
\begin{align}
\dv{w}{t} = P 
= \frac{2}{3} r_0^2 c \gamma^2 \beta_{\perp}^2B^2 
= \frac{2}{3} \frac{q^{4} \beta^2 \sin^2 \alpha B^2 \gamma^2}{m^2 c^3}
\,,
\end{align}
%
where we used the fact that \(\beta _\perp^2 = \beta^2 \sin^2 \alpha \) and \(r_0^2= q^{4} / m^2 c^{4}\). 
In terms of the constant \(c_1 \) and the function \(F\) this will be the integral over the frequency space: 
%
\begin{align}
\dv{w}{t} = c_1 \int_0^{\infty} F(\frac{\omega }{\omega _c}) \dd{\omega }  = 
c_1 \omega _c \int_0^{\infty } F(z) \dd{z} \overset{!}{=} 
\frac{2}{3} \frac{q^{4} \beta^2 \sin^2 \alpha B^2 \gamma^2}{m^2 c^3}
\,,
\end{align}
%
so we can write the value of the constant \(c_1 \) as 
%
\begin{align}
c_1 &= \frac{2}{3} \frac{q^{4} \beta^2 \sin^2 \alpha B^2 \gamma^2}{m^2 c^3} \frac{2}{3 \gamma^3 \omega _B \sin \alpha } \frac{1}{
\int_0^{\infty } F(z) \dd{z} 
}   \\
&= \frac{q^3 \beta^2  \sin \alpha B}{mc^2} \frac{1}{\int_0^{ \infty } F(z) \dd{z}}
\,.
\end{align}

As we shall show later the integral will evaluate to \(\int_0^{ \infty } F(z) \dd{z} = 2 \pi / \sqrt{3}\), so, as long as the particle is ultrarelativistic (\(\beta \approx 1\)) we will have 
%
\begin{align}
c_1 = \frac{q^3 \sin \alpha B}{mc^2} \frac{\sqrt{3}}{ 2 \pi }
\,,
\end{align}
%
therefore 
%
\begin{align}
\frac{ \dd{w}}{ \dd{t} \dd{\omega }} = \frac{\sqrt{3}}{ 2 \pi }
\frac{q^3 B \sin \alpha }{mc^2} F \qty( \frac{\omega }{\omega _c})
\,.
\end{align}

The explicit expression of \(F\) can be derived \cite[par.\ 6.4]{rybickiRadiativeProcessesAstrophysics1979}, but we will not do it here. It is given by the modified Bessel function 
%
\begin{align}
F(z) =  z \int_{z}^{\infty } k_{5/3} (x) \dd{x}
\,.
\end{align}

It is a smooth function of \(z\), going to zero at 0 and \(+ \infty \), and it has its peak around \num{.29}, where it reaches a value close to 1 (this means that the peak of synchrotron radiation is around \(\omega \sim \num{.29} \omega _c\)). Its decay is then exponential.
Asymptotically, we have that for \(z \ll 1\) 
%
\begin{align}
F(z) \sim \frac{4}{\sqrt{3} \Gamma (1/3)} \qty(\frac{z}{2})^{1/3}
\,,
\end{align}
%
while for \(z \gg 1\): 
%
\begin{align}
F(z) \sim \qty(\frac{\pi }{2})^{1/2} e^{-z} z^{1/2}
\,.
\end{align}

\subsubsection{Synchrotron emission from a nonthermal electron distribution}

In many astrphyiscal applications we need to consider the emission from electrons whose energies do not follow a thermal distribution but instead something like a powerlaw; the energy losses and gains caused by synchrotron emission actually cause the spectrum to get closer to a powerlaw, which gives further justification for considering this situation.

So, we will consider a population such that the number density of electrons per unit energy is given by
%
\begin{align}
\dv{N}{E} = C E^{-P}
\,.
\end{align}

Since the energy is given by \(E = mc^2 \gamma \), we can also write 
%
\begin{align}
\dd{N} = C \gamma^{-P} \dd{\gamma }
\,.
\end{align}

Then, the emission per unit volume, frequency and time will be 
%
\begin{align}
\frac{ \dd{w}}{ \dd{t} \dd{V} \dd{\omega }} 
= \int_{\gamma_1 }^{\gamma_2 } \frac{ \dd{N}}{ \dd{\gamma }} \frac{ \dd{w}}{ \dd{t} \dd{\omega }}  \dd{\gamma }  \\
&= \int_{\gamma_1 }^{\gamma_2 } c \gamma^{-P } c_1 F \qty(\frac{\omega }{\omega _c}) \dd{\gamma }
\,.
\end{align}

Remember that \(\omega_c\) depends on \(\gamma \), as \(\omega _c \propto \gamma^{2}\), so we cannot bring it out of the integral. 
We can write this integral, keeping only the variable parts, as 
%
\begin{align}
\frac{ \dd{w}}{ \dd{t} \dd{V} \dd{\omega }} 
&\propto \int_{\gamma_1 }^{\gamma_2 } \gamma^{-P} F \qty(\frac{\omega }{\omega _c}) \dd{\gamma }  \\
&\propto \int_{x_1 }^{x_2 } \omega^{-P / 2} x^{P/2} \omega^{1/2} \qty(x)^{3/2} F \qty(x) \dd{ x}  \\
&= \int_{x_1 }^{x_2 } \omega^{-(P-1)/2} x^{(P-3) / 2} F \qty(x) \dd{x}   \\
&= \omega^{- (P-1 ) / 2} \int_{x_1 }^{x_2 } x^{(P-3) / 2} F(x) \dd{x}
\,,
\end{align}
%
\todo[inline]{in the second equality, should the integration bounds not change?}
where \(x = \omega / \omega _c\). 
The integral bounds \(x_1 \) and \(x_2 \) will depend on \(\omega \) in general --- depending on what the regime of validity of the powerlaw --  but we will brutally approximate \(x_1 \sim 0 \) and \(x_2 \sim + \infty \).
In this case, the integral will not depend on \(\omega \), therefore 
%
\begin{align}
\frac{ \dd{w}}{ \dd{t} \dd{V} \dd{\omega }} 
\propto \omega^{- (P-1) / 2}
\,.
\end{align}

So, the spectrum of the synchrotron radiation emitted by an electron population whose energies are distributed according to a powerlaw will \textbf{also be a powerlaw}.
The new spectral index is given by \(S = (P-1) / 2\). 

\end{document}
