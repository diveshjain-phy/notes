\documentclass[main.tex]{subfiles}
\begin{document}

\marginpar{Thursday\\ 2020-8-27, \\ compiled \\ \today}

We can compute this integral if we know the distribution of the velocity moduli \(f(v)\). The angular integral is nontrivial: the single-velocity cross section has an angular dependence. 

The calculation is then tricky, the full result is relatively recent, going back only to the 1980s. We will not discuss it here.
We are able to reduce the CSK to a single integral in \(\dd{v}\). 

Now, let us give an explicit expression for the velocity distribution \(f(v)\): since the electrons are relativistic we need to use a \textbf{relativistic Maxwellian}, also called a Jüttner distribution, which looks like
%
\begin{align}
f(v) = \frac{\gamma^{5} \exp(- \gamma / \Theta )}{4 \pi c^3 \Theta K_2 (1/\Theta )}
\qquad \text{where} \qquad
\Theta = \frac{k_B T}{mc^2}
\,,
\end{align}
%
where \(\gamma \) is the Lorentz gamma factor of the electron, while \(K_2 \) is the modified Bessel function of order 2.
We will then find a cross section depending on the rescaled temperature \(\Theta \). 
The integral over \(\dd{v}\) can then be evaluated numerically. 

What we find is that the cross section is strongly peaked, both in energy (temperature, \(\Theta \)) and in angle (\(\xi = \cos \theta \)).

We can then compute the total absorption coefficient, by integrating across all possible outgoing photon energies and angles:
%
\begin{align}
\sigma (\epsilon ; \Theta ) &= \int \dd{\Omega }' \dd{\epsilon }' \sigma (\epsilon \to \epsilon ', \xi ; \Theta ) \\
&= \frac{n \sigma_T}{2 \epsilon K_2 (1 / \Theta )}
\int_0^{\infty } \dd{z} \sigma_0 (\epsilon z) \exp(- \frac{z + 1/z}{2 \Theta })
\,,
\end{align}
%
where \(\sigma_0 \) is a known function. This integral is quite easy to compute numerically. 
What we find is quite similar to the drop of the Klein-Nishina cross section, for high temperatures the drop happens at lower energies and is smoother. 

\subsubsection{Energy exchange rate}

We want to see how much energy is transferred from photons to electrons and vice versa. We start off by considering the energy exchanged by a single electron and many photons. 

The electron will be moving with a velocity \(v\), and we consider the photon flux coming from a direction range \(\dd{\Omega }\). We also consider the distribution across photon energies, so that we have the number flux of photons \(\dd{f} = (I/\epsilon ) \dd{\epsilon } \dd{\Omega }\), where \(I\) is the specific intensity.
With this, we can compute the number of scatterings per unit time, solid angle of emission \(\dd{\Omega }' \) and emission energy \(\dd{\epsilon }'\) as 
%
\begin{align}
\dd{N_s} = (\sigma \dd{\Omega }' \dd{\epsilon }') \dd{f}
\,.
\end{align}

Then, the energy change per unit time will be \((\epsilon' - \epsilon ) \dd{N_s}\). 
This is all differential: in order to get the full energy change we need to integrate. For a thermal photon distribution we will have 
%
\begin{align}
\dot{E}_s = \int \dd{\Omega } \dd{\epsilon } \dd{\Omega }' \dd{\epsilon }' 
(\epsilon ' - \epsilon ) \frac{I}{\epsilon } \sigma (\epsilon \to \epsilon ' , \xi ; \Theta )
\,.
\end{align}

Now, we can see that this depends on the number of scatterings, the electron and the radiation temperatures. 
Let us assume that the distribution of the photons is Planckian as well, with a temperature \(T_r\), and define \(\Theta _r = k_B T_r / m c^2\). 

Then, we can find that for small \(\Theta _r\) and small \(\Theta \) we have 
%
\begin{align}
\frac{\dot{E}_s}{E_s} = \qty(4 \Theta - \Theta _r) \qty(1 + \frac{5}{2} \Theta - 21 \frac{\zeta (5)}{\zeta (4)} \Theta _r + \dots)
\,,
\end{align}
%
where \(\zeta \) is  the Riemann zeta function.

\todo[inline]{The 4 is \emph{inside} the parenthesis, right? at thermal equilibrium photons have mean energy \(4 k_B T_e\)\dots See eg \cite[eq.\ 7.36]{rybickiRadiativeProcessesAstrophysics1979}, but I've seen the result often cited elsewhere as well.}

The lowest order contribution is then, for an isotropic photon distribution (not only a blackbody),
%
\begin{align}
\frac{\dot{E}_s}{E_s} = 4 \Theta - \frac{\expval{\epsilon^2}}{\expval{\epsilon }}
\,,
\end{align}
%
so if the photons are colder than \(4 k_B T_e\), where \(T_e\) is the temperature of the electrons, they gain energy.



What happens if a photon \textbf{scatters off the electron distribution many times}? 

Let us assume we have an isotropic electron distribution with number density \(n\), and that the scattering is isotropic as well.
We shall use the cross section for Thomson scattering, \(\sigma _T\); this scattering is not isotropic but it has forward-backward symmetry, which is enough for our considerations.
 

The photon mean free path is then \(\lambda _s = 1 / (n \sigma _T)\), while the mean time between scatterings is \(t_s = \lambda _s / c\). 

The mean number of times the photon scatters in travelling a length \(L\) is something we already discussed, and the final result was 
%
\begin{align}
N_s = \frac{L^2}{\lambda _s^2} = \qty(n \sigma _T L)^2 = \tau _s^2
\,,
\end{align}
%
where \(\tau _s\) is the average scattering optical depth, if \(\tau _s \gg 1\), and \(N_s \sim \tau _s\) if \(\tau _s\) was \(\ll 1\). Then, \(N_s \sim \max (\tau _s, \tau _s^2)\). 

This is useful in order to define the Compton \(y\) parameter: as long as the photon energy \(\epsilon \) is much smaller than \(4 k_B T_e\), we have 

\todo[inline]{At least, it looks to me like we need to make that assumption: the photon cannot thermalize further than that temperature}
%
\begin{align}
y = \expval{\frac{\Delta \epsilon }{\epsilon }} N_s \approx \frac{4 k_B T}{mc^2 } \max(\tau _s , \tau _s^2)
\,,
\end{align}
%
where \(\expval{\Delta \epsilon / \epsilon }\) is the mean fractional energy change per scattering, which as we saw earlier was \(4 \Theta \) at lowest order for a nonrelativistic thermal distribution in the Thomson  limit. 
This number \(y \) tells us how efficient Comptonization is: by how much does the photon's energy change as it travels through the medium?

If this \(y\) comes out to be much smaller than 1, then we expect to see little effect on the primary spectrum (of the star, for instance) while if \(y \geq 1\)  there will be large spectral deformations. 
We can have small \(y\) both if the energy change \(\Delta \epsilon / \epsilon \) is small, or if the medium is optically thin: \(\tau _s\) is small. 

\chapter{Spectral modification in plasmas}

\section{The Kompaneets equation}

This is the tool we use in order to understand what kind of spectral modifications we expect to see for radiation passing through a cloud in which repeated scatterings may occur. 
It was developed in the 1950s by a Russian physicist working on nuclear weapons. 

It is basically a way to rewrite the \textbf{Boltzmann equation}, which is the equation describing the evolution of a photon distribution undergoing scatterings with electrons: 
%
\begin{align}
\pdv{n}{t} = c \int \dd[3]{p} \dd{\Omega } \dv{\sigma }{\Omega } 
\qty[f(\vec{p}') n(\epsilon ') (1 + n(\epsilon )) - f(\vec{p}) n(\epsilon ) (1 + n(\epsilon '))]
\,,
\end{align}
%
where \(f(\vec{p})\) is the phase space density of electrons. 
The reason why there is a \(1 + n(\epsilon )\) factor is to account for \emph{stimulated scattering}.
\todo[inline]{which is\dots?}

We will assume that the energy of the photons is much smaller than \(mc^2\), so \(\epsilon \ll 1\), and that the electrons are nonrelativistic. We can then approximate 
%
\begin{align}
\frac{\epsilon'}{\epsilon } = \frac{1 - \beta \cos \theta }{1 - \beta \cos \theta ' + \epsilon (1 - \cos \theta ) / \gamma } \approx 1 + \beta \cos \theta ' - \beta \cos \theta = 1 + \vec{\beta} \cdot (\vec{\Omega}' - \vec{\Omega})
\,.
\end{align}

Conservation of 3-momentum and energy gives us 
%
\begin{align}
\vec{p}' = \vec{p} + \epsilon \vec{\Omega} - \epsilon ' \vec{\Omega}' 
\qquad \text{and} \qquad
E' = E + \epsilon - \epsilon '
\,,
\end{align}
%
while if the electrons are thermal as well as nonrelativistic their distribution will be 
%
\begin{align}
f(E) = n (2 \pi m k_B T)^{-3/2} \exp(- \frac{E}{\Theta }) 
\,.
\end{align}

Note that all the energies are in units of \(mc^2\), where \(m\) is the electron mass, and the momenta are in units of \(mc\). 

We further assume that the energy change of the photon is small compared to the mean electron energy, 
%
\begin{align}
\Delta = \frac{\epsilon ' - \epsilon }{\Theta } = \frac{\epsilon \vec{\beta} \cdot (\vec{\Omega}' - \vec{\Omega})}{\Theta } \ll 1 
\,,
\end{align}
%
where we used the relation between the photon energies before and after the scattering which follows from the Lorentz transformation. 

\end{document}
