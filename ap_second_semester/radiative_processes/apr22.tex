\documentclass[main.tex]{subfiles}
\begin{document}

\subsection{The physical effects of Comptonization}

\marginpar{Thursday\\ 2020-8-27, \\ compiled \\ \today}

As we have said it is hard to find numerical solutions to the Kompaneets equation, however there is an interesting quantity whose evolution we can study analytically: the photon energy density. 
The energy density can be calculated as 
%
\begin{align}
u = \int \dd[3]{p} h \nu \dv{N}{ \dd[3]{x}}
= \int \dd[3]{p} h \nu   \frac{2 n}{h^3} 
= \frac{8 \pi (k_B T)^{4}}{h^3c^3} \int_0^{\infty } \dd{x} x^3 n
\,,
\end{align}
%
where, as usual, \(x = h \nu / k_B T\), and the last equality comes about by substituting this and \(\dd[3]{p} = p^2 \dd{p} \dd{\Omega }\) and assuming isotropy. 

In order to see how it changes, let us differentiate this energy density with respect to time (rescaled, as in the Kompaneets equation, by the mean time between two scatterings): 
%
\begin{align}
\dv{u}{t_s} 
= \frac{8 \pi (k_B T)^{4}}{h^3c^3} \int_0^{\infty } \dd{x} x^3 \pdv{n}{t_s}
= \frac{8 \pi (k_B T)^{4}}{h^3c^3} \int_0^{\infty } \dd{x} x^3 \frac{\Theta}{x^2} \pdv{}{x} \qty[x^{4 } (n' + n + n^2)]
\marginnote{Used the Kompaneets equation.}
\,.
\end{align}

Now, let us suppose that \(n\) is very small, so that we can neglect \(n\) and \(n^2\) inside the derivative, keeping \(n'\)
\todo[inline]{Not super clear why \(n \ll n'\) if \(n\) is small should hold\dots }
: this yields 
%
\begin{align}
\dv{u}{t_s} \approx \frac{8 \pi (k_B T)^{4}}{h^3 c^3} \Theta \int_0^{\infty } \dd{x} x \pdv{[x^{4} n']}{x}
\,.
\end{align}

We integrate by parts twice and assume that \(x^{k} n' \to 0 \) for \(k> 3\): this give us 
%
\begin{align}
\dv{u}{t_s} \approx \frac{8 \pi (k_B T)^{4}}{h^3 c^3} 4 \Theta \int_0^{\infty } \dd{x} x^3 n = 4 \Theta u
\,.
\end{align}

The solution to this is an exponential: 
%
\begin{align}
u(t_s) = u(0 ) \exp( 4 \Theta t_s) = u(0 ) \exp( t / t_c )
\,,
\end{align}
%
where \(t_c = 1/ (4 \Theta n \sigma _T c)\) is called the Compton time. 
This means that the radiative energy density increases \textbf{exponentially fast}, over a characteristic time of the order \(t_c\). 

\todo[inline]{Wait, is this exponential OK? it diverges as time increases! }

\subsection{Relativistic Kompaneets equation}

We made the hypotheses that \(\Theta \ll 1\) and \(\epsilon \ll 1\) at the start: the electrons and photons were nonrelativistic.
If we keep all the other hypotheses (isotropy, homogeneity, small fractional energy change and so on) can we generalize the Kompaneets equation to relativistic particles?

We will need to use a relativistic Maxwellian (Jüttner distribution) instead of a Maxwellian, and the Klein-Nishina cross section instead of the Thomson one. This will allow us to deal with generic \(\Theta \) and \(\epsilon \): the result is 
%
\begin{align}
\pdv{n}{t} = \frac{1}{\epsilon^2} \pdv{}{\epsilon } \qty[ \alpha (\epsilon , \Theta ) \qty(\Theta \pdv{n}{\epsilon } + n + n^2)]
\,,
\end{align}
%
where \(\alpha \) is a certain known function, 
%
\begin{align}
\alpha (\epsilon , \Theta ) = \frac{\epsilon^4}{2 K_2 (1 / \Theta )} \int_0^{\infty } \dd{z} z^2 \alpha_0 (\epsilon z) \exp( - \frac{z + 1/z}{2 \Theta })
\,,
\end{align}
%
where \(\alpha_0\) is a known analytical function, whose expression is omitted here. 

The form of the relativistic Kompaneets equation is very similar to the nonrelativistic one. 

% 15.50 

\end{document}