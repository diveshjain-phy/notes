\documentclass[main.tex]{subfiles}
\begin{document}

\marginpar{Friday\\ 2020-8-28, \\ compiled \\ \today}

An interesting case in which we have dynamical Comptonization is that of spherical infall of material, for example into a black hole: at least close to the event horizon, the particles will be in free fall. 

In this case we recover a powerlaw high-energy tail for soft photon input.

\subsection{Consequences of comptonization}

\subsubsection{Comptonization in X-ray binaries}

Comptonization plays a role in X-ray binaries, which are binary systems in which one of the objects is a compact object (black hole or neutron star).

Because of Roche lobe overflow, matter forms a gas stream flowing from the donor star to the compact object; there it forms an accretion disk and sometimes a jet. 

There are Low Mass and High Mass X-ray binaries, the latter have a very complex phenomenology and mostly emit in the X-rays. 
The spectrum can change dramatically over time, but the typical ingredient is the accretion disk. 

The simplest model for this is a Shakura-Sunyaev disk (SSD), which is (geometrically) thin: its height is much smaller than its width.
The distribution of the angular velocities of matter in such a disk will be Keplerian: each element of gas will describe a circular orbit. 
Also, these disks are optically thick in the direction normal to the disk. 

This has the consequence that each annulus will need to be in local thermal equilibrium, and it will need to radiate like a blackbody at the local surface temperature, 
%
\begin{align}
T(R) = \qty[\frac{3 GM \dot{M}}{8 \pi \sigma R _{\text{in}}^3 } \qty( \frac{R _{\text{in}}}{T})^{3} \qty(1 - \sqrt{\frac{R _{\text{in}}}{R}})]^{1/4}
\,,
\end{align}
%
where \(T\) is the temperature, \(R\) is the radius of the annulus from the center, \(R _{\text{in}}\) is the inner radius of the accretion disk, \(M\) is the mass of the compact object, \(\dot{M}\) is its accretion rate, while \(\sigma \) is the Stefan-Boltzmann constant. 

The reference temperature multiplying the dependence on the radius is 
%
\begin{align}
T _{\text{in}} = \qty[\frac{3 GM \dot{M}}{8 \pi \sigma R _{\text{in}}^3 }]^{1/4} \approx \SI{1}{keV}
\,,
\end{align}
%
so we can see that the main emission will be in the X-rays. 

The total emission we see will be a superposition of the emission curves of all the annuli, which yields three regions: at the low- and high-energy boundaries the emission looks like that of a blackbody, going like \(\nu^{2}\) and \(\exp(- h \nu  / k_B T)\) respectively, but in the intermediate region there is a peculiar dependence on \(\nu^{1/3}\) of the frequency spectrum. 

If we look at the X-ray spectrum of a binary like Cyg X-1, we can reproduce it by superimposing a blackbody spectrum with a powerlaw tail at high energies. 

In order to understand this we must consider a slightly more complicated model: beyond the disc we need to have a region called the \textbf{corona}, which is less dense than the disk itself.
It will contain energetic electrons --- we are not sure whether they are thermal --- which are able to up-scatter the radiation, so they must be hotter than the disc electrons. 
Then, we will have unsaturated comptonization, which we know to produce a power-law tail. 

\section{Synchrotron and cyclotron radiation}

We will describe the radiation emitted by relativistic particles which are accelerated by magnetic fields. 

First, we need to generalize some results we found in the dipole approximation for nonrelativistic particles.

The four-momentum is defined as \(p^{\mu } = m u^{\mu }\), where \(u^{\mu }= \gamma (c, \vec{v})\) and \(\gamma = 1/ \sqrt{1 - v^2 / c^2}\). 

The four-acceleration is 
%
\begin{align}
a^{\mu } = \dv{ u^{\mu }}{\tau }
\,,
\end{align}
%
where \(\tau \) is the proper time. The equation of motion of a particle is 
%
\begin{align}
m a^{\mu } = F^{\mu } = \dv{p^{\mu }}{\tau }
\,.
\end{align}

This defines the four-force. If the force is electromagnetic, it can be expressed from the Faraday tensor \(F_{\mu \nu }\). 
The force three-vector is 
%
\begin{align}
\vec{F} = e \qty(\vec{E} + \frac{\vec{v}}{c} \times \vec{B})
\,,
\end{align}
%
which can be generalized to 
%
\begin{align}
F^{\mu } = \frac{e}{c} F^{\mu }_{\nu } u^{\nu }
\,,
\end{align}
%
a linear multiple of the four-velocity and the Faraday tensor. 

The equation of motion will then read \(m a^{\mu } = (e/c) F^{\mu }_{\nu } u^{\nu } \), whose time component will be 
%
\begin{align}
\dv{p^{0}}{t} = \frac{e}{c} F^{0}_{\nu } u^{\nu } = \frac{e}{c} \vec{E} \cdot \vec{v}
\,,
\end{align}
%
where we can recall that \(p^{0}\) is the energy of the particle (divided by \(c\)), \(\gamma mc\), so this is telling us that the work on the particle is equal to \(e \vec{E} \cdot \vec{v}\).    

On the other hand, the space components of the equation are 
%
\begin{align}
\dv{p^{i}}{t} = e \qty(E^{i} + \frac{\vec{v}}{c} \times \vec{B})
\,.
\end{align}

They give us the variation of the relativistic momentum, \(p^{i} = \gamma m v^{i}\).

We need to generalize the Larmor formula to relativistic particles. 
Let us start by defining the instantaneous rest frame of a particle: it is the (non-inertial) frame in which the particle is always stationary. 
If we are in this frame at a fixed time we can use the nonrelativistic formula locally, since the particle will have nonrelativistic velocities at least for a small while around this point. 

The emission from the nonrelativistic Larmor formula depends on \(\sin^2 \Theta \), so it is not isotropic; however it is (parity) symmetric, so the change of momentum due to the emission of radiation is zero. 

\todo[inline]{He says that the Larmor formula has no angular dependence but this is false!}

So, in the (primed) rest frame the momentum emission will be zero, \(\dd{p}' = 0\).
This means that we can transform to the lab frame to get the energy change as \(\dd{w} = \gamma \dd{w}'\), where \(w\) denotes the energy. 
Also, the time interval will change like \(\dd{t} = \gamma \dd{t'}\). 

Therefore, the emitted power will be 
%
\begin{align}
\dv{w}{t} = \frac{ \gamma \dd{w}'}{ \gamma \dd{t'}} = \dv{w'}{t'}
\,,
\end{align}
%
so the emitted power is the same in both frames, it is a Lorentz invariant (at least as long as emission has forward-backward symmetry). 

In the rest frame the emitted power is 
%
\begin{align}
\dv{w'}{t'} = \frac{2 q^2}{3c^3} \abs{\vec{a}'}^2
\,,
\end{align}
%
where \(\vec{a}'\) is the 3-velocity in the LRF. In frame the acceleration is nonzero, however the velocity is zero: this means that \(\gamma = 1\), and therefore \(a_0 \sim \dv*{\gamma }{t} = 0\). So, 
%
\begin{align}
\abs{a^{\prime \mu } a^{\prime }_{\mu }} = \abs{\vec{a}'}
\,.
\end{align}

We have then shown that we can calculate the power using the Lorentz-invariant quantity \(a^{\mu } a_{\mu }\) in any frame. 
We will distinguish between the components of the acceleration which are parallel to the velocity, \(a_{\parallel}\), and the perpendicular ones, \(a_{\perp}\). 
Under the boost, they transform like 
%
\begin{align}
a'_{\parallel} = \gamma^3 a_{\parallel}
\qquad \text{and} \qquad
a'_{\perp} = \gamma^2 a_\perp
\,.
\end{align}
 
Therefore, the power can be calculated using the components of the 3-acceleration in the lab frame as 
%
\begin{align}
\dv{w}{t} = \frac{2 q^2}{3c^3} \qty(a^{\prime 2}_{\parallel} + a^{\prime 2}_{\perp}) = \frac{2 q^2 \gamma^{4}}{3 c^3} \qty(a^2_{\perp} + \gamma^2 a^2_{\parallel})
\,.
\end{align}

One might wonder: since \(a_{\mu } a^{\mu }\) is an invariant, why can't we just use the acceleration in the lab frame directly without all the gammas? If we wanted to, we'd also need the \(a^{0}\) component which is nontrivial in the lab frame. What we did is a way to avoid having to compute it. 

\end{document}
