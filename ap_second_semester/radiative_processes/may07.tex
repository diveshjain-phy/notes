\documentclass[main.tex]{subfiles}
\begin{document}

\subsection{Sychrotron absorption}

\marginpar{Tuesday\\ 2020-9-1, \\ compiled \\ \today}

We have seen what the total absorption coefficient (accounting for stimulated emission) is for a two-level system. 

We want to use this expression in order to evaluate the total absorption due to the synchrotron process: so, we need to generalize it to a system with a continuum of energy levels, the free particle states with arbitrary velocity. 

We will approach this by discretizing the space of possible energies of the particle.
There is a slight complication, in that for a given energy \(h \nu \) there are now \emph{many} pairs of levels having that energy between them. We will need to sum over them, still denoting all the higher-energy states in the pairs by ``2'' and the lower-energy states in the pairs by ``1''.
In this discretized description, then, we shall have an expression like 
%
\begin{align}
\alpha _\nu  = \frac{h \nu }{4 \pi } \sum _{E_1 } \sum _{E_2 } 
\qty[n(E_1 ) B_{12} - n(E_2 ) B_{21} ] \phi_{12}
\,,
\end{align}
%
where the sum is performed across all the possible energy levels \(E_1 \) and \(E_2 \) such that \(E_2 - E_1 = h \nu \), while \(n(E_1 )\) and \(n(E_2 )\) are the respective number densities of the two energy states. 
On the other hand, \(\phi_{12} \) is the transition width. 

The result we derived earlier was found by assuming that emission and absorption are isotropic; this is not true anymore since the magnetic field offers a preferential direction.
However, we can work around this problem by assuming that the magnetic field is ``tangled'', so that the direction of its value at a randomly chosen point in space is essentially uniformly distributed along the sphere.

We can also quickly evaluate the emission coefficient \(j_\nu \) (which is the power emitted per unit volume, solid angle and frequency): if the power per unit frequency emitted by a single electron is \(P(E, \nu )\) then the coefficient is 
%
\begin{align}
j_\nu = \frac{ \dd{w}}{ \dd{t} \dd{\nu }} \frac{n(E)}{4 \pi }
= P(E, \nu ) \frac{n(E)}{4 \pi }
\,,
\end{align}
%
where \(n(E)\) is the density of electrons at a specific energy \(E\). 
Note that this is not a differential quantity since we discretized the energy levels, so that they are countable. 
We are also assuming isotropicity, by the same reasoning as before.

The coefficient \(j_\nu \) can also be expressed in terms of the Einstein coefficients, as 
%
\begin{align}
j_\nu = \frac{h \nu }{4 \pi } \phi_{21}  (\nu ) n_2 A_{21} 
\,,
\end{align}
%
which like before can be generalized to the situation in which we have many possible energy levels giving rise to the same transition as 
%
\begin{align}
j_\nu = \sum _{E_1 } \frac{h \nu }{4 \pi }  \phi_{21} n_2 A_{21} 
= \frac{h \nu }{4 \pi } n_2  \sum _{E_1 } \phi_{21} A_{21} 
= \frac{n_2 }{4 \pi } P(E_2, \nu  )
\,,
\end{align}
%
meaning that now we know 
%
\begin{align}
P(E_2 , \nu ) = h \nu \sum _{E_1} \phi_{21} A_{21} 
\,.
\end{align}

From the \emph{detailed balance} relations we can also relate the Einstein coefficients as 
%
\begin{align}
A_{21} = \frac{2 h \nu^3}{c^2} B_{21} 
\,,
\end{align}
%
which we can substitute in, to find 
%
\begin{align} \label{eq:spectral-power-from-einstein-B}
P(E_2 , \nu ) = h \nu \qty( \frac{2 h \nu^3}{c^2}) \sum _{E_1} \phi_{21} B_{21} 
\,.
\end{align}

The reasoning we went to these great lengths to express the spectral power as a function of the Einstein coefficients is that we hope to invert the relation we found, giving us the coefficients as a function of the spectral power.

We have an expression for \(\alpha_\nu \) in terms of the coefficients \(B_{21} \) and \(B_{12} \); however we know from the detailed balance relations that, as long as the statistical weights of the energy levels are equal (which holds for us since we are considering free states), \(B_{21} = B_{12} \), so we can write the expression as 
%
\begin{align}
\alpha _\nu &=  \frac{h \nu }{4 \pi } \sum _{E_1 } \sum _{E_2 } B_{21} \qty(n(E_1 ) - n(E_2 ) ) \phi_{21} \\
&= \frac{h \nu }{4 \pi } 
\sum _{E_2 } \qty(n(E_2 - h \nu ) - n(E_2 ))
\sum _{E_1 } B_{21} \phi_{21} \marginnote{Used \(E_2 - E_1 = h \nu \).}  \\
&= \frac{h \nu }{4 \pi }
\sum _{E_2 } \qty(n(E_2 - h \nu ) - n(E_2 ))
\frac{P(E_2, \nu )}{h \nu \frac{2 h \nu^3}{c^2}}  \marginnote{Used equation \eqref{eq:spectral-power-from-einstein-B}.}\\
&= \frac{c^2}{8 \pi h \nu^3 } 
\sum _{E_2 } \qty(n(E_2 - h \nu ) - n(E_2 ))
P(E_2 , \nu )
\,.
\end{align}

This works well for a system which we can discretize: however, if we want to use the language of the continuum of states we will need to turn the sum into an integral.
In order to do this, we need to introduce the distribution function of electrons in momentum space, \(f(\vec{p})\), which we assume to be isotropic. Using it, we write 
%
\begin{align}
\alpha _\nu = 
\frac{c^2}{8 \pi h \nu^3}
\int \dd[3]{p_2 } \qty[ f(\overline{p}_2 )- f(p_2 )]P (E_2 , \nu )
\,,
\end{align}
%
where \(p_2 \) is the modulus of the momentum associated with \(E_2 \), and similarly \(\overline{p}_2\) is associated with \(E_1 = E_2 - h \nu \). 

Let us apply this for the well-known case in which the electron distribution is thermal, so that 
%
\begin{align}
f(\vec{p}) = k \exp(- \frac{E}{k_B T})
\,,
\end{align}
%
which means that 
%
\begin{align}
f(\overline{p}_2) - f(p_2) &= k \qty(\exp(- \frac{E - h \nu }{k_B T}) - \exp(- \frac{E}{k_B T}))  \\
&= k \exp(- \frac{E}{k_B T}) \qty(\exp(\frac{h \nu }{k_B T}) - 1)
= f(p_2) \qty(\exp(\frac{h \nu }{k_B T}) - 1)
\,.
\end{align}

We can plug this into the expression we have for the absorption  coefficient  to find 
%
\begin{align}
\alpha _\nu &=
\frac{c^2}{8 \pi h \nu^3} \int \dd[3]{p} f(p) \qty(\exp(\frac{h \nu }{k_B T}) - 1) P(E, \nu )  \\
\frac{c^2}{8 \pi h \nu^3} \qty(\exp(\frac{h \nu }{k_B T}) - 1) \underbrace{\int \dd[3]{p} f(p) P(E, \nu ) }_{4 \pi j_\nu }  \\
&= \frac{j_\nu}{B_\nu }
\,,
\end{align}
%
since the rest of the expression we have is precisely the inverse of the Planck function. 
We have \textbf{recovered Kirkhoff's law}, which is expected since the electrons are assumed to be in thermal equilibrium. 

Now we want to proceed in the general case in which the distribution of the electron energies is not Maxwellian; a specific case of interest is a power-law spectrum, which as we have seen can be the effect of synchrotron radiation.

\todo[inline]{Are we sure about this? was the powerlaw tail not the effect of Comptonization?} 

We wish to switch our integral from one over the particle momentum to one over the particle energy; in order to simplify the change of variable we will make the assumption that the particles are ultrarelativistic, therefore \(E \approx p c\). The change of variable is then 
%
\begin{align}
4 \pi p^2 \dd{p}=  4 \pi \frac{E^2}{c^2} \frac{\dd{E}}{c} = \frac{4 \pi E^2}{c^3} \dd{E}
\,.
\end{align}
%
while the number of particles in this differential element is given by 
%
\begin{align}
\dd{N} = N(E) \dd{E} = 4 \pi p^2 \dd{p} f(p) = \frac{4 \pi E^2}{c^3} \dd{E} f(p)
\,,
\end{align}
%
so we can identify 
%
\begin{align}
N(E) = \frac{4 \pi E^2}{c^3} f(p)
\qquad \text{or} \qquad
f(p) = \frac{N(E)}{E^2} \frac{c^3}{4 \pi }
\,,
\end{align}
%
where \(f(p)\) is calculated at the momentum \(p = E /c\).  
Plugging this into the expression for the absorption coefficient we find 
%
\begin{align}
\alpha _\nu = \frac{c^2}{8 \pi h \nu^3} \int \frac{4 \pi }{c^3} E^2 \dd{E} \frac{c^3}{4 \pi } \qty[\frac{N(E - h \nu )}{(E - h \nu )^2} - \frac{N(E)}{E^2}] P (E, \nu )
\,.
\end{align}

We will further assume that the particle energy \(E\) is much smaller than the photon energy \(h \nu \); this is justified by the fact that we are not using the full machinery of QED which would be required if the calculation was nonclassical. 

\todo[inline]{Not clear how \(E \ll h \nu \) means classical\dots
Is it not actually \(h \nu \ll E\)? the photon would be relativistic then}

If this is the case, then we can expand the expression in a power series around \(E\), so that it reads 
%
\begin{align}
\alpha _\nu  = - \frac{c^2}{8 \pi h \nu^3} \int E^2 P(E, \nu ) \pdv{}{E} \qty[ \frac{N(E)}{E^2}] h \nu 
\,.
\end{align}

Let us restrict this result to a powerlaw electron distribution, so that \(N(E) = C E^{-P}\) for some \(P \in \mathbb{R}^{+}\). 
If this is the case, then the expression inside the integral reads 
%
\begin{align}
- E^2 \dv{}{E} \qty(\frac{N(E)}{E^2})
= - E^2 \dv{}{E} \qty[C E^{-P-2}]
= C E^2 (P+2) E^{-P-3} = (P+2) \frac{N(E)}{E}
\,,
\end{align}
%
so the absorption coefficient will be approximately 
%
\begin{align}
\alpha _\nu = (P+2) \frac{c^2}{8 \pi \nu^2} \int \frac{N(E)}{E} P(E, \nu ) \dd{E} 
\,.
\end{align}

We must insert the known expression for the power radiated by a single charge as \(P(E, \nu )\); this can be evaluated to finally find 
%
\begin{align}
\alpha _\nu \propto \nu^{- (P+4) / 2}
\,.
\end{align}

Now, recall that in general the definition of the source function \(S_\nu \) is 
%
\begin{align}
S_\nu = \frac{j_\nu }{\alpha _\nu } = \frac{P(\nu )}{4 \pi \alpha _\nu }
\,,
\end{align}
%
where \(P(\nu )\) is the total spectral power, and since we know that its dependence on \(\nu \) is as \(P(\nu ) \propto \nu^{- (P-1) / 2}\), we can calculate the general source function for a powerlaw distribution: 
%
\begin{align}
S_\nu \propto \frac{\nu^{- (P-1 ) / 2}}{\nu^{- (P+4) / 2}} = \nu^{- \frac{P - 1 + P - 4}{2}} = \nu^{5/2}
\,,
\end{align}
%
which is \textbf{independent of} \(P\)! 

Knowing the source function we can make certain predictions about the emission of a medium which is optically thick at least for the low-energy part of the spectrum. 

We have shown at the beginning of the course that if the optical depth in a medium is large then the specific intensity \(I_\nu \) becomes close to the source function: \(I_\nu  \sim S_\nu \). 
In the case of synchrotron emission we have seen that \(\alpha_\nu \sim \nu^{- (P+4) / 2}\), a \emph{decreasing} function of the frequency, so the absorption will be large at low energies and small at high energies. 
The same will generally hold for the optical depth, since it is proportional to the absorption coefficient if the length (\(\sim\) size of the medium) is fixed. 

On the other hand, if the medium is optically thin then \(I_\nu \sim j_\nu \sim P(\nu )\). 

So, if we make a plot (let us make it log-log as is usually done) of the specific intensity \(I_\nu \) as a function of frequency \(\nu \) we will have a ``thick'' region at low \(\nu \) where the intensity increases as \(I_\nu \sim \nu^{5/2}\), and a ``thin'' region at high \(\nu \) where the intensity decreases as \(I_\nu \sim \nu^{- (P-1) / 2}\). 

\end{document}
