\documentclass[main.tex]{subfiles}
\begin{document}

\section{Radiative Transitions}

\marginpar{Saturday\\ 2020-9-5, \\ compiled \\ \today}

This topic will be treated more quickly than what was originally planned for lack of time. 

The basic question we will ask today is: given the knowledge of the various energy states of a certain atom, how do we characterize the radiation emitted by the various transitions of this atom? 

First, we need to find out which transitions are actually permitted and thus can be observed in reality: \textbf{selection rules} will help with this. We will then determine what is the strength of the radiation from the transitions which are allowed. 

A note: no transition in the cases we will treat is strictly \emph{forbidden}, meaning that it is actually \emph{impossible} for it to happen; rather, we use the word as a shorthand to denote transitions whose probability is quite small, so that the intensity of radiation they emit is vanishingly small (at least as compared to the ``allowed'' transitions). 

We will use a \textbf{semi-classical} approach, treating the atom quantum-mechanically and radiation classically. 
The justification for this (which amounts to not using QED, basically) is that the self-interaction of photons is negligible. 

\todo[inline]{Is the condition we should check not that the number of photons is very large, so that any single-photon effect washes out?}

The transition probability per unit time from an initial state \(i\) to a final state \(f\) is denoted \(w_{fi}\).
There is an explicit quantum-mechanical expression for this quantity in terms of the Hamiltonian of the system and the wavefunctions corresponding to the two states of the electrons. 

Because of the detailed balance equation we have that, at equilibrium, \(w_{fi} = w_{if}\). 

The general expression for the transition rate contains integrals in the form 
%
\begin{align}
\int \phi _f^{*} e^{i \vec{k} \cdot \vec{r}} \vec{\ell} \cdot \vec{\nabla}_j \phi _i \dd[3]{r}
\,,
\end{align}
%
where \(\vec{\ell}\) is a unit vector in the direction of the polarization of the wave \cite[eq.\ 10.13 onwards]{rybickiRadiativeProcessesAstrophysics1979}, while \(\vec{k}\) is the wavevector and \(\vec{r}\) is the position vector.

In the \textbf{long-wave} approximation, in which the wavelength of the radiation is much larger than the characteristic scale of the atom, we can expand 
%
\begin{align}
e^{i \vec{k} \cdot \vec{r}} \approx 1 + i \vec{k} \cdot \vec{r} + \frac{1}{2} \qty(i \vec{k} \cdot \vec{r})^2 + \dots
\,.
\end{align}

Let us make the long-wave condition more explicit: typically we will have \(\abs{\vec{r}} \sim a_0 = \hbar / (m_e \alpha c)\), the Bohr radius, while the wavevector's magnitude will typically be \(\abs{\vec{k}} \sim \omega / c\). So, the quantity we want to be small is 
%
\begin{align}
\vec{k} \cdot \vec{r} \sim \frac{a_0 \omega }{c} = a_0 \frac{\Delta E}{\hbar c} 
\,,
\end{align}
%
where \(\Delta E\) is the transition energy.
Since the binding energy is electrostatic in nature, the typical value of \(\Delta E\) is \(\Delta E \sim Z e^2 / 2 a_0 \). Then, the Bohr radius simplifies, and we get 
%
\begin{align}
\vec{k} \cdot \vec{r} \sim \frac{Z e^2}{2} \frac{1}{\hbar c} = \frac{Z \alpha }{2} \ll 1
\,,
\end{align}
%
so the approximation of taking the scalar product to be small works as long as we consider atoms which are not too far down the periodic table. 

Truncating the series at the lowest order is equivalent to the dipole approximation. The link comes from the fact that \(Z \alpha \approx v / c\), at least to first order. 
This means that, as long as the condition is satisfied, the electron is moving non-relativistically around the nucleus.  

In the dipole approximation we do indeed have certain transitions with probability exactly equal to zero, but this is an artifact of the approximation. 

If we consider an ensemble of atoms whose orientations are random, then the \emph{unpolarized} transition rate reads 
%
\begin{align}
\expval{w_{fi}} = \frac{4 \pi^2}{3 \hbar^2 c}
\abs{d_{fi}}^2 \mathcal{I}(\omega_{fi})
\,,
\end{align}
%
where \(\mathcal{I}(\omega _{fi}) = 4 \pi J(\omega _{fi})\) is the intensity of radiation integrated over all solid angles and computed at the transition frequency, while \(d_{fi}\) is the matrix element of the electric dipole operator \(\hat{d} = e \sum_j \hat{r}_j\) between the initial and final state. 

Now, this is related to the Einstein \(B\) coefficient as 
%
\begin{align}
\expval{w_{lu}} = B_{lu} J(\nu _{ul})
\,,
\end{align}
%
where we switched from \(i\) and \(f\), meaning initial and final, to \(l\) and \(u\), meaning lower and upper energy state. 

We have the conversion 
%
\begin{align}
J(\nu _{ul}) = \frac{\mathcal{I}(\nu _{ul})}{4 \pi }
= \frac{2 \pi }{4 \pi } \mathcal{I}(\omega _{ul})
= \frac{\mathcal{I}(\omega _{ul})}{2}
\,.
\end{align}

\todo[inline]{What? a change by a factor \(2 \pi \) in the \emph{argument} results in a change in \(2 \pi \) in the \emph{function}? This only holds if intensity is a \emph{linear} function of the frequency!}

Therefore, 
%
\begin{align}
\expval{w_{ul}} = B_{lu} \frac{\mathcal{I}(\omega _{ul})}{2}
\,,
\end{align}
%
which means that the Einstein \(B\) coefficient can be expressed in terms of the dipole operator as 
%
\begin{align}
B_{lu} = \frac{8 \pi^2}{4 c \hbar^2} \abs{d_{ul}}^2
\,.
\end{align}

It is common to express this in terms of a quantity called the \emph{oscillator strength} for that given transition, however in order to introduce that formalism we need to generalize what we have found to transitions between degenerate energy levels. 

If we have several degenerate upper levels, while the final lower state has a statistical weight \(g_l\), then we can write 
%
\begin{align}
B_{lu} = \frac{32 \pi^2}{3 c \hbar^2} \frac{1}{g_l} \sum _{l} \abs{d_{ul}}^2
\,.
\end{align}

We are averaging over the initial states, and summing over the final ones.

\todo[inline]{why is this expression multiplied by a factor 4 compared to the nondegenerate one?}

\todo[inline]{Why is \(g_l\) outside the sum?}

The definition of the oscillator strength \(f_{lu}\) is then given by expressing the Einstein coefficient as 
%
\begin{align}
B_{lu} &= \underbrace{\frac{4 \pi^2 e^2}{h \nu_{ul} m c}}_{\text{classical}} f_{lu}  \\
f_{lu} &= \frac{2 m }{3 \hbar^2 g_l c^2} (E_u - E_l) \sum \abs{d_{lu}}^2  
\,.
\end{align}

The quantity \(f_{lu}\) is called the \textbf{oscillator strength}; the reason for this definition is that the quantity multiplying \(f_{lu}\) in the expression for the Einstein coefficient is the coefficient we would get with a classical oscillator; the oscillator strength is the quantum correction to it.

\section{Line broadening}

So far, we have implicitly assumed that the transition can be described by a single frequency, but as was discussed earlier in the course this is not really the case; instead, the line should be described by a certain function \(\phi (\nu )\) which is peaked around that frequency. 

There are several processes which can broaden a line beyond the most basic time-energy uncertainty relation. 

\subsubsection{Doppler broadening}

We are often dealing with plasmas which contain ions moving at high velocities due to thermal motion; since this is a random motion the distribution of the velocities' components along the observation direction will be quite wide. 
Since we need to move from the atom rest frame to the observer frame, we will need to consider the Doppler effect.

The velocity component along the observation direction affects the effective frequency of the radiation as measured by us. 
Let us make this reasoning quantitative. 

Suppose that the observation direction is the \(z\) axis, and let us denote the velocity component in that direction as \(v_z\). 
Further, let us assume that the velocity is nonrelativistic so that we can work to first order in \(v/c\). Then, the Doppler shift reads 
%
\begin{align}
\nu - \nu_0  = \nu_0 \frac{v_z}{c} 
\,,
\end{align}
%
where the (variable) frequency \(\nu \) is in the observer frame, while the (fixed) frequency \(\nu_0 \) is in the atom frame. 

Since the velocities are assumed to be nonrelativistic and corresponding to thermal equilibrium, their distribution will be Maxwellian, meaning that each of the three components will have a Gaussian distribution: explicitly, the number density of atoms whose velocity component is near \(v_z\) in an interval of width \(\dd{v_z}\) is 
%
\begin{align}
\dd{N} \propto \exp(- \frac{m v_z^2}{2 k_B T}) \dd{v_z}
\,.
\end{align}

We can refer this expression to the frequency change by inverting the Doppler relation: we find 
%
\begin{align}
v_z = c \frac{\nu - \nu_0 }{\nu_0 }
\implies 
\dd{v_z} = \frac{c}{\nu_0 } \dd{\nu }
\,,
\end{align}
%
using which we can refer the Gaussian distribution of the velocity components back to the emitted frequency distribution as 
%
\begin{align}
\dd{N} \propto \phi (\nu ) \propto \exp(- \frac{m c^2 (\nu - \nu_0 )^2}{2 k_B T \nu_0^2}) \dd{\nu }
\,.
\end{align}

From this, we can say that the line profile function is given by 
%
\begin{align}
\phi (\nu ) = \frac{1}{ \Delta \nu _D \sqrt{\pi }} \exp(- \frac{(\nu - \nu_0 )^2}{\Delta \nu_D^2})
\,,
\end{align}
%
where the width of the line is given by
%
\begin{align}
\Delta \nu _D = \frac{\nu_0 }{c} \sqrt{\frac{2 k_B T}{m}}
\,,
\end{align}
%
and the normalization is fixed by imposing \(\int_0^{\infty  } \phi (\nu ) \dd{\nu } = 1\).  

\subsubsection{Natural broadening}

This effect is due to the fact that if a level has a lifetime of \(\Delta t\), then there must be a spread \(\Delta E\) in the transition energies of the level so that \(\Delta t \Delta E \gtrsim \hbar\). 
Since no level has an infinite lifetime, not even at \(T =0 \) can a transition be truly monochromatic. 

The decay rate of a state \(n\) is given by 
%
\begin{align}
\gamma = \sum _{n'} A _{n n'}
\,,
\end{align}
%
where \(n'\) is a state with a lower energy than \(n\) into which the atom can decay, while \(A_{nn'}\) is the Einstein \(A\) coefficient for the transition from \(n\) to \(n'\). 

It can be shown that the line profile due to natural broadening is given by a Lorentzian or Lorentz profile, 
%
\begin{align}
\phi (\nu ) = \frac{1}{4 \pi^2} \frac{\gamma }{(\nu - \nu_0 )^2 + (\gamma / 4 \pi )^2}
\,.
\end{align}

It can also be shown that certain other effects, such as microturbulence, can be described by a similar profile: this is then accomplished by defining a ``generalized transition rate'' \(\Gamma \) to replace the transition rate \(\gamma \). 
We will use the letter \(\Gamma \) in the next section. 

\subsubsection{Combined Doppler and Lorentz profile}

We can give a description of the combined effects of the two kinds of broadening we discussed so far, which is needed since they will both occur in general. 

The Lorentz profile referred to the observed frequency of the transition (which can be shifted due to the Doppler effect) is given, like before, as 
%
\begin{align}
\phi (\nu ) = \frac{\Gamma / 4 \pi^2}{(\nu - \nu_0^{\text{obs}} )^2 + (\Gamma / 4 \pi )^2}
\,,
\end{align}
%
where 
%
\begin{align}
\nu_0^{\text{obs}} = \nu_0 + \nu_0 \frac{v_z}{c}
\,,
\end{align}
%
so we can refer the profile to the atom frame like 
%
\begin{align}
\phi (\nu ) = \frac{\Gamma / 4 \pi^2}{\qty[\nu - \nu_0 \qty(1 + v_z / c)]^2 + (\Gamma / 4 \pi )^2}
\,.
\end{align}

In order to find the observed broadening we need to average this over all the possible \(v_z\) according to their Gaussian distribution: this yields 
%
\begin{align}
\phi (\nu ) &= \frac{\Gamma }{4 \pi^2}
\int_{- \infty }^{\infty } \frac{\sqrt{m / 2 \pi k_B T}\exp(- m v_z^2 / 2 k_B T)}{\qty[\nu - \nu_0 \qty(1 + v_z/c)]^2 + (\Gamma / 4 \pi )^2} \dd{v_z}  \\
&= \frac{1}{\Delta \nu_D \sqrt{\pi }} H(a, u)
\,,
\end{align}
%
where we introduced the Voigt function, which cannot be computed analytically:
%
\begin{align}
H(a, u) = \frac{a}{\pi }  \int_{- \infty }^{\infty } \frac{e^{- y^2} \dd{y}}{a^2 + (u-y)^2}
\,,
\end{align}
%
whose arguments are \(a = \Gamma / 4 \pi \Delta \nu _D\) and \(u = (\nu - \nu_0 ) / \Delta \nu _D\). 
This finally yields the Voigt profile, which is basically the convolution of the Gaussian and Lorentzian profiles. 

\end{document}
