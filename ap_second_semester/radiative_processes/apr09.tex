\documentclass[main.tex]{subfiles}
\begin{document}

\section{Electron scattering}

\subsection{Compton scattering onto an electron at rest}

\marginpar{Sunday\\ 2020-8-23, \\ compiled \\ \today}

We need to account for the fact that light has a quantum nature, which is not addressed in the classical treatment of scattering. 
Also, now we will account for the momentum of the photon. 

We start of with an electron at rest, and a photon with energy \(h \nu \) and momentum \(h \nu / c\) impinging on it. 
After the scattering, the photon will have energy \(h \nu '\) and momentum \(h \nu ' / c\), while the electron will have momentum \(m v \gamma \). 
Let us call the angle between the direction of the incoming photon and the direction of the outgoing one \(\theta \). 

In terms of four-vectors, we can express the momenta of the photon before and after as 
%
\begin{align}
k^{\mu } = \frac{\epsilon}{c} \left[\begin{array}{c}
1 \\ 
\vec{\Omega}
\end{array}\right]
\qquad \text{and} \qquad
k^{\prime \mu } = \frac{\epsilon'}{c} \left[\begin{array}{c}
1 \\ 
\vec{\Omega}'
\end{array}\right]
\,,
\end{align}
%
where \(\vec{\Omega}\) and \(\vec{\Omega}'\) are unit vectors defining the propagation directions, such that \(\vec{\Omega} \cdot \vec{\Omega}' = \cos \theta \). 
On the other hand, the momenta of the electron will be 
%
\begin{align}
p^{\mu }  = \left[\begin{array}{c}
mc \\ 
\vec{0}
\end{array}\right]
\qquad \text{and} \qquad
p^{\prime \mu } = \gamma \left[\begin{array}{c}
mc \\ 
m \vec{v}
\end{array}\right]
\,.
\end{align}

Since the particles are unchanged after the scattering, both the incoming and outgoing momenta must satisfy \(p^{\mu } p_{\mu } = - m^2 c^2\) and \(k^{\mu } k_{\mu } = 0\) (in any frame: they are Lorentz scalars). 
Because of momentum conservation, we can also impose the four equations 
%
\begin{align}
p^{\mu } + k^{\mu } = p^{\prime \mu } + k^{\prime \mu }
\,.
\end{align}

Solving the system of equations yields 
%
\begin{align}
\epsilon' = \frac{\epsilon }{1 + \frac{\epsilon }{mc} \qty(1 - \vec{\Omega} \cdot \vec{\Omega}')}
\,,
\end{align}
%
implying that we must have \(\epsilon' \leq \epsilon \): the photon will lose energy in the scattering. 
The calculation which yields the differential cross section of the scattering is quite complicated and requires the full machinery of QED; here we just give the result: 
%
\begin{align}
\frac{ \dd{\sigma }  }{ \dd{\vec{\Omega} }'} 
= \frac{r_0^2}{2 } \qty(\frac{\epsilon '}{\epsilon })^2
\qty[ \frac{\epsilon}{\epsilon '} + \frac{\epsilon'}{\epsilon } - 1 + \vec{\Omega} \cdot \vec{\Omega}'] 
\,,
\end{align}
%
where \(\epsilon \) is the energy divided by \(m c^2\) and \(r_0 \) is the classical electron radius. 
If we substitute in our formula for the energy, using \(\xi = \cos \theta = \vec{\Omega} \cdot \vec{\Omega}'\), we find 
%
\begin{align}
\frac{ \dd{\sigma }}{ \dd{\Omega }' \dd{\epsilon }'} 
= \frac{r_0^2}{2} \frac{1 + \xi^2}{\qty(1 + \epsilon (1 - \xi ))^2}
\qty[1 + \frac{\epsilon^2 (1 - \xi )^2}{(1 + \xi )^2 (1 + \epsilon (1 - \xi ))}]
\delta \qty(\epsilon ' - \frac{\epsilon }{1 + \epsilon (1 - \xi )})
\,.
\end{align}

In the low energy limit, \(h \nu \ll m_e c^2 \) or \(\epsilon \to 0\), we find 
%
\begin{align}
\frac{ \dd{\sigma }}{ \dd{\Omega }' \dd{\epsilon }'} 
=
\frac{r_0^2}{2} (1 + \xi^2) \delta (\epsilon ' - \epsilon )
\,,
\end{align}
%
which is the Thompson cross section. 

Let us now define the \textbf{Compton scattering kernel} \(\sigma \): it is the differential cross section times the electron density,
%
\begin{align}
\sigma (\epsilon \to \epsilon ', \xi ) = n_e \frac{ \dd{\sigma }}{ \dd{\Omega }' \dd{\epsilon }'}
\,.
\end{align}

We can integrate this in order to find the total cross section presented by the electrons to photons of an energy \(\epsilon \): 
%
\begin{align}
\sigma (\epsilon ) &= \int \dd{\Omega '} \dd{\epsilon '} \sigma (\epsilon \to \epsilon ', \xi ) \\
&= \frac{3}{4} n_e \sigma_{T} \qty[ \qty(\frac{1 + \epsilon }{\epsilon^3}) 
\qty(\frac{2 \epsilon (1 + \epsilon )}{1 + 2 \epsilon }- \log \qty(1 + 2 \epsilon ))
+ 
\frac{1}{2 \epsilon } \log \qty(1 + 2 \epsilon )
- 
\frac{1 + 3 \epsilon }{(1 + 2 \epsilon )^2}
]
\,.
\end{align}

How does this differ from the Thompson cross section? For \(\log \epsilon \lesssim -1 \) we have \(\sigma \approx \sigma_T\), while as \(\epsilon \) increases the cross section goes to zero. 
In the low energy limit we have the expansion 
%
\begin{align}
\sigma (\epsilon )\approx \sigma_T \qty(1 -2 \epsilon + \frac{26}{5} \epsilon^2)
\,.
\end{align}

The introduction of these nonconservative aspects complicates the radiative transfer equation for scattering.
The absorption term is 
%
\begin{align}
- I (\epsilon , \Omega ) n_e \int \dd{\Omega }' \dd{\epsilon }' \frac{ \dd{\sigma }}{ \dd{\Omega }' \dd{\epsilon }' } =     - I (\epsilon , \Omega  ) \sigma (\epsilon )
\,,
\end{align}
%
while for the emission term the intensity must go inside the integral, so we have 
%
\begin{align}
n_e \int \dd{\Omega }' \dd{\epsilon '} I(\epsilon ', \Omega ') \frac{ \dd{\sigma }}{ \dd{\Omega }' \dd{\epsilon }' } = \int \dd{\Omega }' \dd{\epsilon }' \sigma (\epsilon \to \epsilon ', \xi )
\,,
\end{align}
%
which cannot be expressed in terms of the integrated kernel \(\sigma (\epsilon )\). 

For Thomson scattering the absorption term is similar, with \(n_e \sigma_T\) instead of \(\sigma (\epsilon )\). 
The emission term, on the other hand, can now be evaluated to yield  
%
\begin{align}
\int \dd{\Omega }' \dd{\epsilon }' \sigma (\epsilon \to \epsilon ', \xi )
= \frac{r_0^2}{2} n_e \int \dd{\Omega '} \dd{\epsilon '} I(\epsilon ', \Omega ') (1 + \xi^2) \delta (\epsilon ' - \epsilon ) \sim n_e \sigma_T J(\epsilon )
\,,
\end{align}
%
as long as we neglect the angular part \(1 + \xi^2\). 

\end{document}
