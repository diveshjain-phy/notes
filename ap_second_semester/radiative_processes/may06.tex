\documentclass[main.tex]{subfiles}
\begin{document}

\subsection{Polarization of synchrotron radiation}

\marginpar{Saturday\\ 2020-8-29, \\ compiled \\ \today}

Measuring polarization is in general difficult, but still it is important to be able to provide an estimate of how much radiation is actually polarized. 

In order to describe synchrotron radiation we will need some reference unit vectors. Let us define \(\vec{\epsilon}_{\perp}\) and \(\vec{\epsilon}_{\parallel}\) such that, given the observation direction \(\vec{n}\) and the magnetic field \(\vec{B}\), \(\vec{\epsilon}_\perp\) is perpendicular to both, while \(\vec{\epsilon}_\parallel\) is perpendicular to \(\vec{n}\) only, while being in the plane defined by \(\vec{n}\) and \(\vec{B}\). 

These two unit vectors define a basis for the radiation seen in the direction \(\vec{n}\), since as we recall the polarization of electromagnetic radiation is always transverse. 

It can be shown that the power spectrum emitted in each polarization is given by 
%
\begin{align}
\eval{\frac{ \dd{w}}{ \dd{t} \dd{\omega }}}_{\perp} &= \frac{\sqrt{3} q^3 B \sin \alpha }{4 \pi m c^2} \qty(F + G) \\
\eval{\frac{ \dd{w}}{ \dd{t} \dd{\omega }}}_{\parallel} &= \frac{\sqrt{3} q^3 B \sin \alpha }{4 \pi m c^2} \qty(F - G)
\,,
\end{align}
%
where \(F\) is the function of \(\omega / \omega _c\) we defined earlier, 
%
\begin{align}
F \qty(\frac{\omega }{\omega _c}) = \frac{\omega}{\omega _c} \int_{\omega / \omega _c}^{\infty } K_{5/3} (z) \dd{z} 
\,,
\end{align}
%
while the new function \(G\) is defined by 
%
\begin{align}
G \qty(\frac{\omega }{\omega _c}) = \frac{\omega}{\omega _c} K_{2/3} \qty( \frac{\omega}{\omega _c})
\,.
\end{align}

The total emitted power per unit frequency is given by the sum of the two contributions in each polarization; so the terms containing \(G\) simplify, and we are left with the \(F\) term only, the result we found earlier. 

The polarization fraction at each frequency for a single electron will be given by 
%
\begin{align}
\Pi (\omega ) 
= \frac{P_\perp (\omega ) - P_\parallel (\omega )}{P_\perp (\omega ) + P_\parallel (\omega )}
= \frac{G( \omega / \omega _c) }{F(\omega / \omega _c)}
\,,
\end{align}
%
while the polarization fraction for a whole family of electrons whose energies are distributed according to a powerlaw will be given by the ratio of the integrals of \(G\) and \(F\) weighted by the electron distributions:  
%
\begin{align}
\Pi &= \frac{\int_0^{\infty } G (\omega / \omega _c) \gamma^{-P} \dd{\gamma }}{\int_0^{\infty } F(\omega  / \omega _c) \gamma^{-P} \dd{\gamma }}  \\
&= \frac{\int_0^{\infty } G (x) \gamma^{-P} \dd{\gamma }}{\int_0^{\infty } F(x) \gamma^{-P} \dd{\gamma }}
\,,
\end{align}
%
where we must be careful, since \(x\) is a function of \(\gamma \) in the integral. Specifically, \(x = \omega / \omega _c \sim 1/ \gamma^2\), so \(\gamma \sim x^{-1/2}\), therefore \(\dd{\gamma } \propto -1/2 x^{-3/2} \dd{x}\). This yields, up to multiplicative factors which cancel out in the ratio: 
%
\begin{align}
\Pi = \frac{\int_0^{\infty } G(x) x^{(P-3) / 2} \dd{x}}{\int_0^{\infty } F(x) x^{(P-3) / 2} \dd{x}}
\,.
\end{align}

These integrals of special functions are evaluated in the literature, we have general explicit expressions for expressions like \(\int x^{\mu } F(x) \dd{x}\) and similarly for \(G\).

Substituting these in, we find 
%
\begin{align}
\Pi = \frac{1 + P}{P + 7/3}
\,.
\end{align}

This is always \(<1\) as it should be, however with very steep powerlaws (high \(P\)) it can reach values of \(\Pi \) which are arbitrarily close to 1.

If, on the other hand, we are only considering one electron, we can compute the \emph{frequency-integrated} polarization fraction 
%
\begin{align}
\Pi 
= \frac{\int_0^{\infty } G(x) \dd{\omega }}{\int_0^{\infty } F(x) \dd{\omega }}
= \frac{\int_0^{\infty } G(x) \dd{x}}{\int_0^{\infty } F(x) \dd{x }}
\,,
\end{align}
%
which coincides with the polarization fraction from a population of electrons if we take \(P = 3\). Therefore, we find \(\Pi = (1+3) / (3 + 7/3) = 3/4\). 

\section{Einstein coefficients}

In order to understand what the absorption from synchrotron radiation looks like we need to take a step back and consider the Einstein coefficients. 

Let us reconsider Kirkhoff's law: if emitters and absorbers are in thermal equilibrium, then the following relation holds: 
%
\begin{align}
j_\nu = \alpha _\nu B_\nu 
\,.
\end{align}

This law holds at the macroscopic level: however, the process giving rise to it have a microscopic origin, so we should be able to give a microscopic version of this law. 

Let us consider an atom with two energy levels, at \(E\) and \(E + h \nu_0\) respectively. 
In the population these will have a different statistical weight: let us call these weights \(g_1 \) and \(g_2 \). 

Einstein identified the three main processes which can occur when such an atom interacts with radiation: 
\begin{enumerate}
    \item spontaneous emission, where the system starts off in the higher energy level and decays to the lower one, emitting a photon of energy \(h \nu_0 \). This can happen regardless of the presence of an external radiation field. 
    The Einstein \(A\) coefficient describes this: \(A_{21} \) is the transition probability per unit time connected to this process. 
    \item Absorption, where the atom starts off in the lower energy state and absorbs a photon of energy \(h \nu_0 \), going to the higher energy level.
    It should be stressed that the absorption is not actually monochromatic; because of several effects it is described by a distribution \(\phi (\nu )\) which is sharply peaked around \(\nu_0\), and which is normalized so that \(\int \phi (\nu ) \dd{\nu } = 1\). 
    Then, the absorption probability per unit time is given by 
    %
    \begin{align}
    B_{12} \int_0^{\infty } J_\nu \phi (\nu ) \dd{\nu }
    \,,
    \end{align}
    %
    where \(J_\nu \) is the mean intensity of the external radiation field. 
    \item Stimulated emission, which is a consequence of the quantum nature of light. Planck's law only holds if this is included. This corresponds to the fact that the emission probability is enhanced by the presence of a radiation field. This is similar to how absorption occurs, in that the transition probability per unit time is 
    %
    \begin{align}
    B_{21} \int_0^{\infty } J_\nu \phi (\nu ) \dd{\nu }
    \,.
    \end{align}    
\end{enumerate}

Let us consider a system of many two-level atoms, whose number density is \(n\). Some of them will be in level 1 and some will be in level 2; let us call the corresponding number densities \(n_1 \) and \(n_2 \).

If we are in thermodynamic equilibrium, then the transition rate from 1 to 2 must equal that from 2 to 1. This can be written as 
%
\begin{align}
\underbrace{n_1 B_{12} \overline{J}}_{\text{absorption}} &= \underbrace{n_2 A_{21}}_{\text{emission}} + \underbrace{n_2 B_{21} \overline{J}}_{\text{stimulated emission}}  \\
\overline{J} &= \int_0^{\infty } J_\nu \phi (\nu ) \dd{\nu }
\,.
\end{align}

We can solve this equation for \(\overline{J}\): it comes out to be 
%
\begin{align}
\overline{J} = \frac{A_{21} / B_{21} }{ (n_1 / n_2 ) (B_{12} / B_{21} ) - 1}
\,.
\end{align}

Since we know that the system is in thermodynamic equilibrium, we can also write that the ratio of the number densities of the two populations will be 
%
\begin{align}
\frac{n_1}{n_2 } = \frac{g_1  \exp( - E / k_B T)}{g_2 \exp(- E / k_B T - h \nu_0 / k_B T)} = \frac{g_1}{g_2 } \exp( \frac{h \nu_0}{k_B T})
\,.
\end{align}

Inserting this in the previous relation we find 
%
\begin{align}
\overline{J} = \frac{A_{12} / B_{21} }{(g_1 B_{12} /  g_2 B_{21} ) \exp(h \nu_0 / k_B T) - 1 }
\,.
\end{align}

Also, we know that the mean intensity \(J_\nu \) will be equal to the Planck function \(B_\nu \). The Planckian is quite slowly varying while \(\phi (\nu )\) is very peaked around \(\nu_0 \): therefore, it is safe to say that 
%
\begin{align}
\overline{J} = \int_0^{\infty } B_\nu \phi (\nu ) \dd{\nu } \approx B_{\nu_0 }
\,,
\end{align}
%
since \(\phi (\nu )\) is normalized. Equating this to the expression we found above, we get 
%
\begin{align}
\overline{J} =  B_{\nu_0 } = \frac{2 h \nu^3}{c^2} \frac{1}{\exp( \frac{h \nu_0 }{k_B T }) - 1} &= \frac{A_{12} / B_{21} }{(g_1 B_{12} /  g_2 B_{21} ) \exp(h \nu_0 / k_B T) - 1 }
\,.
\end{align}

This must hold for any temperature \(T\), so we can identify the terms in the two similar expressions: this tells us that 
%
\begin{align}
\frac{g_1 B_{12} }{g_2 B_{21} } = 1 
\qquad \text{and} \qquad
\frac{2 h \nu^3}{c^2} = \frac{A_{12} }{B_{21} }
\,.
\end{align}

These are called the \textbf{detailed balance relations}. They must be satisfied even if the system is outside of thermal equilibrium, since the temperature \(T\) does not appear in them. 

\todo[inline]{What? How? Why? We used the hypothesis of thermal equilibrium a lot\dots}

\subsubsection{The emission and absorption coefficients}

Let us assume that radiation which is emitted in the transition \(2 \to 1\) has the same frequency as the radiation which is absorbed in the transition \(1 \to 2\), so both processes will be characterized by the funciton \(\phi (\nu )\). 

With this, we can express the energy emitted per unit volume, solid angle, frequency and time as 
%
\begin{align}
J_\nu \dd{V} \dd{t} \dd{\Omega } \dd{\nu } &= \frac{h \nu_0}{4 \pi } \phi (\nu ) n_2 A_{21} \dd{\Omega } \dd{V} \dd{t} \dd{\nu }  \\ 
J_\nu &= \frac{h \nu_0}{4 \pi } \phi (\nu ) n_2 A_{21}  
\,.
\end{align}

For absorption, we can apply a similar line of reasoning: 
the energy absorbed per unit volume and time is 
%
\begin{align}
\dd{w}_\nu = 
\dd{V} \dd{t} h \nu_0 n_1 B_{12} \frac{1}{4 \pi } \int \dd{\Omega } \dd{\nu } I_\nu \phi (\nu )
\,,
\end{align}
%
which we can specify to the energy taken out of the beam in the specific frequency and solid angle unit range by removing the integral. 
If we write the volume element as \(\dd{V} = \dd{A} \dd{s}\), where \(\dd{s}\) is the distance travelled along the path of the radiation, and if we remember the radiative absorption law \(\dd{I_\nu } = - \alpha _\nu I_\nu \dd{s}\), we can write 
%
\begin{align}
- \dd{I_\nu } = -\frac{ \dd{w}_\nu }{ \dd{A} \dd{t} \dd{\Omega } \dd{\nu }}
&= \underbrace{h \nu_0 n_1 \frac{B_{12}}{4 \pi } \phi(\nu )}_{\alpha _\nu } I_\nu  \dd{s} \\ 
\alpha_\nu &= \frac{h \nu_0  }{4 \pi } n_1 B_{12} \phi (\nu )
\,.
\end{align}

We could also write \(h \nu\) instead of \(h \nu_0\) if we were to approximate \(\phi (\nu )\)  as a delta function \(\phi (\nu ) = \delta (\nu - \nu_0 )\).

Now, since stimulated emission depends on \(\overline{J}\), it is convenient to treat it as a kind of ``negative absorption'': with the same steps as before we find that its coefficient is 
%
\begin{align}
\alpha_\nu^{(s)} = - \frac{h \nu_0 }{4 \pi } n_2 B_{21} \phi (\nu ) 
\,.
\end{align}

The total absorption coefficient will then be the sum of these: 
%
\begin{align}
\alpha _\nu = \frac{h \nu_0 }{4 \pi } \phi (\nu ) \qty(n_1 B_{12} - n_2 B_{21} )
\,.
\end{align}

In general, when we talk about absorption we always mean absorption minus stimulated emission. 

\end{document}