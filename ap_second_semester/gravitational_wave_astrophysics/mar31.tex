\documentclass[main.tex]{subfiles}
\begin{document}

\marginpar{Tuesday\\ 2020-3-31, \\ compiled \\ \today}

\section{Spin and B-field evolution of NSs}

What is the structure of neutron stars? 
Their radius is of the order of \SI{15}{km}, but this is very much a field of investigation. 
We have a carbon atmosphere of the height of a few centimeters, 
the core has nuclear density, of the order of \SI{e15}{g/cm^3}. 

The spin frequencies are of the order of \SI{700}{Hz} to \SI{.05}{Hz}. 
Their magnetic fields are of the order of \SI{e13}{G}, they lose rotational energy at something like \(\dot{E} \sim \num{e5} L_{\odot}\). 

Their masses are of the order \(\num{1.4}M_{\odot}\). 
They are like a giant atomic nucleus, with \(A \sim \num{e57}\) and densities of the order of 2 to 10 times the density of nuclei. 

They are very precise clocks. Although they are slowly slowing down, they do so predictably. 

The radio emission is of the order \SI{.48}{Jy}, and if we were to interpret it as blackbody radiation we get absurdly high temperatures. 
There must be a different mechanism. 

The total amount of radiation detected from radio pulsars is very little. 

See radio telescope in Sardinia!

We have detected radio pulsars mostly on our side of the galaxy, since the radio waves have a hard time going through the galactic center.

\subsection{Pulse profiles} 

Sometimes we have a single pulse, sometimes we have two, and they can have different shapes. 

The emission in different wavelengths can be misaligned, depending on the geometry. 
Radio, optical and X-ray photons can 

See \cite[]{shapiroBlackHolesWhite1983}.

We can use the scattering and the dispersion relation in the ISM to gauge the distance of the pulsars. The distance is one over the slope of the curve of phase to frequency. 

This needs a measure of the density of free electrons on the ISM, so it is quite uncertain, it has an uncertainty of around \SI{20}{\percent}.

We can constrain the stellar wind mass-loss rate of a B-star using pulsars. 

\todo[inline]{What is the angular resolution with which we can determine the positions of these?}

Spin evolution of pulsars: we have a rotating dipole, so there is emission of radiation. 

We can give a characteristic age by \(\tau = P / 2 \dot{P}\). 

We know that the strain of the GWs is given by 
%
\begin{align}
h_{\mu \nu } = \frac{2G}{c^{4}d} \ddot{Q}_{\mu \nu }
\,,
\end{align}
%
so we need an asymmetric mass distribution rotating. 

We know that 
%
\begin{align}
L_{GW} \sim \frac{G}{5c^{5} } \expval{\dot{\ddot{Q}}_{\mu \nu } \dot{\ddot{Q}}_{\mu \nu }}
\,,
\end{align}
%
and typically 
%
\begin{align}
\dot{\ddot{Q}} \sim \frac{MR^2}{T^3} \sim \frac{M v^3}{R}
\,,
\end{align}
%
where \(M\), \(v\) and \(R\) are characteristic masses, velocities and radii of the system. 

A neutron star with a mountain with height \SI{1}{mm} has a \(L_{GW}\) of 60 orders of magnitude larger than that of a \SI{20}{m} steel cylinder spinning at the speed of sound.

We can measure the braking index: 
%
\begin{align}
n = \frac{\ddot{\Omega} \Omega }{\dot{\Omega}^2}
\,,
\end{align}
%
which determines the evolution of the pulsar in the \(P - \dot{P}\) diagram. \(n=3\) is what we would have for a pure dipole, while if we had only decay by gravitational radiation we would see \(n=5\). 

The true age of a pulsar is given by 
%
\begin{align}
t = \frac{P}{(n-1) \dot{P}} \qty[1 - \qty(\frac{P_0 }{p})^{n-1}]
\,.
\end{align}

Only a tiny part of the lost energy goes to radio waves. 

The slope in the \(P, \dot{P}\) diagram is given by \(2-n\). 

Magnetars: a type in neutron star with a very high magnetic-field decay: they cannot be radio pulsars since their \(X\)-ray luminosity is \(\gg \dot{E} _{\text{rot}}\). 

Largest magnetar \(\gamma \)-ray burst: the atmosphere was measured to be fluctuating at its same period! 

We can measure \emph{proton} synchrotron emission lines. 

There is evidence that over time the pulsars tend to align the spin axis and the N-S magnetic axis. 
This is very difficult to measure. 

What would be the frequency range of the GW emission 

\end{document}
