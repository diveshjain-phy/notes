\documentclass[main.tex]{subfiles}
\begin{document}

\marginpar{Friday\\ 2020-3-13, \\ compiled \\ \today}

Today we introduce the concept of \emph{invariant mass}: we consider a system of \(N\) particles, each with energy \(E_{k}\) and momentum \(\vec{p}_{k}\). 
Their four-momenta are \(p_{k} = \qty(E_{k} / c, \vec{p}_{k})\). 

The total 4-momentum is 
%
\begin{align}
p_{\text{tot}} = \sum_{k=1}^{N} p_{k}
\,.
\end{align}

The scalar product between two of these is defined as 
%
\begin{align}
p_1 \cdot p_{2} = \frac{E_1 E_2 }{c^2} - \vec{p}_{1} \cdot \vec{p}_{2}
\,.
\end{align}

We define the quantity 
%
\begin{align}
\sqrt{s} = \sqrt{p^2 _{\text{tot}}}
= \sqrt{\big(\sum _{k} E_{k}\big)^2 - \big|\sum _{k} \vec{p}_{k}\big| c^2}
\,.
\end{align}

Let us consider a fixed target experiment: a particle 1 incident upon a screen, call it 2. 
Let us consider the frame in which the screen is at rest. 

The momenta are \(p_1 = \qty(E_1 / c,  \vec{p}_{1})\) and \(p_2 = \qty(m_2 c, \vec{0})\). 
The total momentum is then 
%
\begin{align}
p _{\text{tot}} = \qty(\frac{E_1}{c} + m_2 c , \vec{p}_{1})
\,,
\end{align}
%
so 
%
\begin{align}
\sqrt{s} = \qty(\qty(E_1 + m_2 c^2)^2 - p_1^2)
\,,
\end{align}
%
so 
%
\begin{align}
\sqrt{s} = \sqrt{m_1^2c^{4} + m_2^2c^{4} + 2 E_1 m_2 c^2}
\,.
\end{align}

If, instead, both particles are moving then  
%
\begin{align}
\sqrt{s} = \sqrt{m_1^2c^{4} + m_2^2c^{4} + 2 \qty(E_1 E_2 - \abs{p_1 } \abs{p_2 } c^2 \cos \theta )}
\,,
\end{align}
%
where \(\theta \) is the angle between the particles' trajectories. For a head-on collision 
%
\begin{align}
\sqrt{s} = 
\sqrt{m_1^2c^{4} + m_2^2c^{4} + 2 \qty(E_1 E_2 - \abs{p_1 } \abs{p_2 } c^2)}
\,.
\end{align}

The flux of cosmic particles decreases approximately as a powerlaw with the energy. 

Exercise: at the LHC the particles can be accelerated to energies of around
%
\begin{align}
E \approx \SI{7}{TeV}
\,,
\end{align}
%
so we want to compute the COM energy \(\sqrt{s}\) for two protons colliding head-on. 



\end{document}
