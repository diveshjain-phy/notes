\documentclass[main.tex]{subfiles}
\begin{document}

\section{Dark Matter}

\marginpar{Wednesday\\ 2020-5-13, \\ compiled \\ \today}

Yesterday we described the formation of the first nuclei. 
The amount of baryons in terms of energy density is of the order of 4 to \SI{5}{\percent}. 

We know that the total value of \(\Omega _{\text{tot}}  \sim 1\). 
The density fraction due to dark energy is around \SI{70}{\percent}, while \(\Omega _{\text{matter}} \sim \SI{30}{\percent}\).

So, most (something like \SI{80}{\percent}) of the matter is not made of baryons. 

We are then looking for non-baryonic matter with \(\Omega \sim \SI{25}{\percent}\).

There were interesting studies about how the lensing of light is affected by Dark Matter. 
All the evidence is based on our knowledge of gravitational interactions. 
Are we sure that Newton's law of gravitation is right on such large scales? There are MOND theories.

However, the hints about the existence of non-baryonic DM come from very different directions. 
We then take the standard view: Dark Matter exists. 

Could the Higgs boson be the source of Dark Matter? No. 
It decays too fast.

We have similar problems for the \(Z\) boson. So, in the end we must have some other neutral particle.

So, neutrinos? Maybe. They are light, of course, but there are many. 

We get 
%
\begin{align}
\Omega_{\nu } = \frac{\rho^{0}_{\nu }}{\rho _{\text{crit}}} 
\,,
\end{align}
%
where \(n^{0}_{\nu } \approx \SI{339.5}{cm^{-3}}\). So, this comes out to be 
%
\begin{align}
\Omega_{\nu } = \frac{ \sum _{i} m_{\nu_{i}}}{h^2 \SI{93.14}{eV}}
\,.
\end{align}

Recall that ``electron neutrino'' is a current eigenstate, not a mass eigenstate. 
The bound we have is of the order \(m_{\nu_{e}} \lesssim \SI{1}{eV}\).

We also have bounds for the square of the mass differences. 
In the end, this means that the term \(\sum m_{\nu }\) can be at most a few \SI{}{eV}.

In spite of the fact that they are very numerous, neutrinos have too small a mass. 

Dark Matter is also crucial in the formation of Large Scale Structure.
It must ``clump'' in order to do so. This means that it must be nonrelativistic.  But neutrinos are relativistic today! 

We must ask that the mass of Dark Matter, \(M_X\), be larger than the temperature of decoupling. This is then called ``Cold Dark Matter''.

Relativistic DM kills all of the density fluctuations. 

If neutrinos were the source of DM, the first structures would form at a very large scale. 

There are simulations used in order to compare these models.

What does the Dark Matter density profile look like in a galaxy? 
CDM would have a ``cusp'' in density at the center, a really heavy region. 

Could we also have ``Warm'' Dark Matter? An intermediate situation between HDM and CDM. 

Dark Matter around \SI{1}{keV} in mass would be a candidate for this. 

Let us consider \textbf{Thermal Dark Matter}: it has been in equilibrium with the plasma for some time in the early universe.

This means that at that time we can know the number density of those particles. 
Consider a heavy particle \(\ce{X}\), with mass larger than, say, \SI{10}{GeV}, and which is stable or quasi-stable (that is, such that its lifetime is long compared to the age of the universe).

We have different kinds of equilibrium: kinematic and chemical equilibrium. If we have a scattering like 
%
\begin{align}
\ce{X} + \ce{A} \to \ce{X} + \ce{B}
\,,
\end{align}
%
this does not change the number of particles \(\ce{X}\). It only distributes the momenta. 

If the annihilation rate of \ce{X} is lower than the expansion of the universe, it stays as it is. 
Then, we have the Boltzmann suppression factor in the distribution, \(\exp(- m_X / T)\). 

The final number of \ce{X} is reduced by a factor \(\exp(- m_X / T _{\text{freezeout}})\).

What is the window of the parameters of this particle which could give us the \(\Omega_{DM}\) we observe?
If the particle is a Weakly Interacting Massive Particle, we get the ``WIMP miracle'': with very simple assumptions, we get \(\Omega \) between 1 and \num{e-1}. 

The fact that the LHC has not seen any new particles at the electroweak scale poses a problem, even though the WIMP possibility looks good. 

We can expect that there are more than one kind particle constituting DM.
A possibility is the presence of a ``Dark Sector'' in the SM. 
It is generally assumed that there is a sort of ``portal'', which makes the ordinary matter sector allowing the sectors to communicate.

A very simple kind of dark sector would be constituted by right-handed neutrinos: they only interact with the Higgs boson, with terms like \(\overline{\nu}_{L} H^{0} \nu_{R}\).

\end{document}
