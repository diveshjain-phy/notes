\documentclass[main.tex]{subfiles}
\begin{document}

\section{Dark Matter}

\marginpar{Wednesday\\ 2020-5-13, \\ compiled \\ \today}

% Yesterday we described the formation of the first nuclei. 
% The amount of baryons in terms of energy density is of the order of 4 to \SI{5}{\percent}. 
Let us start our discussion of dark matter with the \textbf{baryon budget}: how much do baryons contribute to the total energy density of the universe?

In the \emph{high-redshift universe} we have: 
\begin{enumerate}
    \item at nucleosynthesis (\(z \sim \num{e9}\)) the baryonic matter fraction was \(\Omega_{B}^{BBN} = \num{.04(2)}\); 
    \item at CMB emission (\(z \sim \num{e3}\)) the baryonic matter fraction was \(\Omega_{B}^{CMB} \approx \Omega_{B}^{BBN} \approx \SI{5}{\percent}\);
    \item at \(z \sim 3\) we have observations of the Lyman \(\alpha \) (``frest''?) of the intergalactic medium: this again yields \(\Omega_{B} \approx \SI{5}{\percent}\).
\end{enumerate}

\todo[inline]{what is written there?}

In the recent universe (\(z \sim 0\)) we have observations of galaxy clusters, in which we measure a matter fraction of \(\Omega_{M} \approx \num{.26}\), of which about \SI{15}{\percent} is made of baryons: so, we get an estimate of around \(\Omega_{B} \approx \SI{4}{\percent}\). 

So, the fraction of luminous baryonic matter is \(\Omega_{B}^{\text{lumin}} \approx \SI{1}{\percent}\), with a corresponding density of \(\rho _{\text{lumin}} \approx \SI{9e-29}{kg m^{-3}}\). 
\todo[inline]{Not really clear how we estimate this based on what is mentioned above.}

On the other hand, our estimate for the total baryonic mass fraction is \(\Omega_{B}^{\text{tot}} \approx \SI{5}{\percent}\), with a corresponding density of \(\rho_{b} \approx \SI{4e-28}{kg m^{-3}}\).

We know that the total value of \(\Omega _{\text{tot}}  \sim 1\). 

The density fraction due to dark energy is around \SI{70}{\percent}, while \(\Omega _{\text{matter}} \sim \SI{30}{\percent}\): this can be deduced from the gravitational potential energy, which is inferred from galactic rotation curves, and the mass distribution in clusters. 
The corresponding matter density is around \(\rho_{M} \approx \SI{2e-27}{kg m^{-3}}\).

So, most (something like \SI{80}{\percent}) of the matter is not made of baryons. 
We are then looking for non-baryonic matter with \(\Omega \sim \SI{25}{\percent}\).
More precisely, the latest results give 
%
\begin{align}
\Omega_{DM} \approx \SI{26.4}{\percent} = \SI{84.4}{\percent} \Omega_{M}
\,.
\end{align}

Accounting for our uncertainty in \(H_0 \), we can write \(\Omega_{B} = \num{.022}h^{-2} \approx 4 \divisionsymbol \SI{5}{\percent}\).

How sure are we that dark matter is indeed the best way to describe galactic rotation curves and such?

It makes other predictions beyond them: there were interesting studies about how the lensing of light is affected by Dark Matter. 
All the evidence is based on our knowledge of gravitational interaction: Are we sure that Newton's law of gravitation is right on such large scales? There are MOND theories, in which it does not.
However, the hints about the existence of non-baryonic DM come from very different directions.\footnote{And I'd be remiss not to cite \url{https://xkcd.com/1758/}.}
We then take the standard view: Dark Matter exists. 

Could the Higgs boson be the source of Dark Matter? No. 
It decays too fast.
We have similar problems for the \(Z\) boson. So, in the end we must have some other neutral particle.

So, neutrinos? Maybe. They are light, of course, but there are many. 

We get 
%
\begin{align}
\Omega_{\nu } = \frac{\rho^{0}_{\nu }}{\rho _{\text{crit}}} 
\,,
\end{align}
%
where we used the known neutrino number density today \(n^{0}_{\nu } \approx \SI{339.5}{cm^{-3}}\).
So, this comes out to be 
%
\begin{align}
\Omega_{\nu } = \frac{ \sum _{i} m_{\nu_{i}}}{h^2 \SI{93.14}{eV}}
\,.
\end{align}

Now, \(h^2 \sim 1/2\), so this is very roughly 
%
\begin{align}
\Omega_{\nu } \sim \frac{\sum m_\nu }{\SI{50}{eV}}
\,.
\end{align}

Recall that ``electron neutrino'' is a current eigenstate, not a mass eigenstate. 
The bound we have is of the order \(m_{\nu_{e}} \lesssim \SI{1}{eV}\).

We also have bounds for the square of the mass differences. 
In the end, this means that the term \(\sum m_{\nu }\) can be at most a few \SI{}{eV}.

Even if all the three neutrinos' masses were at the very top of their current experimental bounds at \( m_\nu \sim \SI{2}{eV}\) each we would have \(\Omega_{\nu } < \num{.1}\), which is too small to fit observations of \(\Omega_{DM} \sim \num{.25}\). 

In spite of the fact that they are very numerous, neutrinos have too small a mass. 

\subsection{Hot \& cold DM}

% Dark Matter is also crucial in the formation of Large Scale Structure.
% It must ``clump'' in order to do so: this means that it must be nonrelativistic.  But neutrinos are still relativistic today!

Let us call a generic DM particle \(X\). It will decouple at a certain temperature \(T^{D}_{X}\). Then, \textbf{cold} DM is such that \(M_X \gg T_X^{D}\), while for \textbf{hot} DM we have \(M_X \ll T_X^{D}\). 
So, cold DM is \emph{nonrelativistic} when it decouples, while hot DM is \emph{relativistic} when it does.

We can also have \textbf{warm} DM, in the case where \(M_X \sim T_D\).

% We must ask that the mass of Dark Matter, \(M_X\), be larger than the temperature of decoupling. This is then called ``Cold Dark Matter''.

% Relativistic DM kills all the density fluctuations. 

Neutrinos would be an example of hot dark matter, since they decouple around \SI{1}{eV} while their mass is smaller than that. 

Whether DM is hot or cold\footnote{\url{https://www.youtube.com/watch?v=kTHNpusq654}} is relevant for the formation of structures: relativistic matter has a hard time clumping at small scales. 
% If neutrinos were the source of DM, the first structures would form at a very large scale. 

For HDM, in the period \(T_D > T > M_X\) the DM is free-streaming, so it quickly fills in under-dense regions, and the density perturbations are washed out.

The warm case is at around \(M_X \gtrsim \SI{1}{eV}\), while HDM has \(M_X < \SI{1}{eV}\). 

There are simulations used in order to compare these models: what is seen is that for HDM the first structures to form are the very largest --- superclusters of galaxies ---, which then fragment into smaller pieces. 
This type of evolution is in stark disagreement with our observations.
With this line of reasoning we have bounded the fraction of DM which may be hot: most of it will be cold. 
So, neutrinos may be a part of DM, but they cannot be the dominant part of it.

% What does the Dark Matter density profile look like in a galaxy? 
% CDM would have a ``cusp'' in density at the center, a really heavy region. 

% Could we also have ``Warm'' Dark Matter? An intermediate situation between HDM and CDM. 

Dark Matter around \SI{1}{keV} in mass would be a candidate for this. 

Let us consider \textbf{Thermal Dark Matter}, which has been in equilibrium with the plasma for some time in the early universe, so it is cold when it decouples.
By assuming chemical equilibrium we can figure out the number density of those particles.
Consider a heavy particle \(\ce{X}\), with mass larger than, say, \SI{10}{GeV}, and which is stable or quasi-stable (that is, such that its lifetime is long compared to the age of the universe).

Then, its equilibrium number density will be given by 
%
\begin{align}
n_X ^{\text{eq}} = g_X \qty(\frac{M_X T}{2 \pi })^{3/2} \exp(- \frac{M_X}{T})
\,.
\end{align}

The number of \(X\) in a certain comoving volume can only change by annihilation and creation: \(X \overline{X} \leftrightarrow \text{light particles}\).

If the rate \(\Gamma \) of this process is \(>H\) at a certain temperature \(T\), then the number density of \(X\) is indeed the equilibrium one. 
However, as \(T\) decreases below \(M_X\) we will reach a point at which the light particles will not be energetic enough to form \(X\) pairs. 
At this point, only the annihilation of \(X \overline{X}\) pairs will be able to occur, so their number will diminish. 

After some more time, their encounters will not be likely anymore, so we say that the abundance of \(X\) \emph{freezes out}.
From that moment onward, \(n_X\) will not change in the comoving frame.

Note that this may still be before \emph{kinetic} equilibrium is reached: scatterings like \(X + A \to X + B\) can still happen. 
The number of \(X\)s is governed by the Boltzmann equation: 
%
\begin{align}
\dv{n_X}{t} + 3 H n_X  = - \expval{\sigma _{\text{annihilation}}v } \qty(n_X^2 - n_{X, \text{eq}}^2)
\,.
\end{align}

The cross section \(\sigma _{\text{annihilation}}\) is averaged over all the equilibrium momentum space distribution functions of the particle species, and summed over the possible annihilation channels.

% We have different kinds of equilibrium: kinematic and chemical equilibrium. If we have a scattering like 
% %
% \begin{align}
% \ce{X} + \ce{A} \to \ce{X} + \ce{B}
% \,,
% \end{align}
% %
% this does not change the number of particles \(\ce{X}\). It only distributes the momenta. 

% If the annihilation rate of \ce{X} is lower than the expansion of the universe, it stays as it is. 
% Then, we have the Boltzmann suppression factor in the distribution, \(\exp(- m_X / T)\). 

% The final number of \ce{X} is reduced by a factor \(\exp(- m_X / T _{\text{freezeout}})\).

\todo[inline]{Here start many calculations to get the density of \(X\) today\dots}

\subsection{WIMPs}

What is the window of the parameters of this particle which could give us the \(\Omega_{DM}\) we observe?

If the particle is a Weakly Interacting Massive Particle, we get the ``WIMP miracle'': with very simple assumptions, we get \(\Omega \) between 1 and \num{e-1}. Let us show this.

\begin{claim}
For a heavy relic particle the following holds:
%
\begin{align}
\Omega_{X} h^2 \approx \num{e-10} 
\qty(\frac{\SI{}{GeV^{-2}}}{\expval{\sigma v}_{\text{eff}}})
\frac{1}{\sqrt{g_* (t_f)}}
\log(
    \frac{g_X M_X M_P \expval{\sigma v}}{(2 \pi )^{3/2} \sqrt{g_*}}
)
\,.
\end{align}
\end{claim}


We apply this to a particle with mass \(M_X \sim \SI{100}{GeV}\), which interacts through the weak interaction and therefore has 
%
\begin{align}
\expval{\sigma v} \sim \frac{\alpha_{w}^2}{M_X^2} \sim \frac{\num{e-4}}{\num{e4}} \SI{}{GeV^{-2}}
\,.
\end{align}

The values of \(\sqrt{g_*}\) we need are \(\sqrt{g_*}\sim 10\) at \(T > \SI{200}{GeV}\) and \(\sqrt{g_*} \sim 3\) at \(\SI{100}{MeV}\).

Plugging these in we get 
%
\begin{align}
\Omega_{X} h^2 &\approx 2 \times \num{e-10} \num{e8} \frac{1}{10} \log(\frac{\num{e2} \num{e19} \num{e-7}}{\num{e2}})  \\
\Omega_{X} &\approx 2 \times  2\times  \num{e-3} 12 \log 10 \approx \num{e-1}
\,,
\end{align}
%
since \(h^2 \sim 1/2\).

So, a particle with \(M_X \sim \SI{100}{GeV}\) which is weakly coupled, \(\alpha_{w} \sim \num{.01}\) could explain our dark matter observations!

So, the WIMP possibility looks good; alas, the LHC has not seen any new particles at the electroweak scale. 

We have a result of \(\Omega_{X}\) between \(\num{.1}\) and \(\num{1}\), but the higher end of this range is not good for our purposes: we know that \(\Omega_{DM} < \num{.25}\). 
So, we can put a lower limit on \(\sigma _{\text{annihilation}}\) in order to avoid the over-production of dark matter (or, more accurately, we want to avoid the fact that too much of it remains).
What we find is 
%
\begin{align}
\abs{\sigma _{\text{ann}} v} >  \num{.3} \divisionsymbol \SI{10e-8}{GeV^{-2}}
\,.
\end{align}

Now, if the mass of \(X\) were large (like, more than \SI{10}{TeV}) then we would get \(\Omega_{X} > 10\), which would definitely over-close the universe. So, we expect DM not be super massive. 

There are ways to get around this restriction: one is a strongly coupled theory, in which case \(X\) would be annihilated more; then it would be a Strongly Interacting Massive Particle but it would work. 

Another possibility is for \(X\) not to have been in thermal equilibrium, so that we have non-thermal DM, so that their number is not linked to the abundance of other particle species. 
\todo[inline]{But wait, what does thermal equilibrium have to do with it? We should only consider chemical equilibrium for the variation of the number of \(X\), right?}

For regular WIMPs, though, the bound \(M_X < \SI{100}{TeV}\) remains. 

We can expect that there are more than one kind particle constituting DM.
A possibility is the presence of a ``Dark Sector'' in the SM. 
It is generally assumed that there is a sort of ``portal'', which makes the ordinary matter sector allowing the sectors to communicate.

A very simple kind of dark sector would be constituted by right-handed neutrinos: they only interact with the Higgs boson, with terms like \(\overline{\nu}_{L} H^{0} \nu_{R}\).

How do we \textbf{search for WIMPs}? There are two ways, one is direct: we set up a target and wait for \(X\) to hit it. 
If we set up a target with \SI{10}{kg} of nuclei with \(A \sim 100\), and have \(M_X \sim \SI{100}{GeV}\) and \(\sigma_0 \sim \SI{e-10}{GeV^{-2}}\), then we will have 
%
\begin{align}
\text{event rate} &\approx 
v_X n_X N_A \sigma_{AX}  \\
&\approx v_X \frac{\rho_{DM}^{\text{local}}}{M_X} N_A \sigma_{AX}
\,,
\end{align}
%
where \(N_A \approx \num{6e25}\) is the number of \(A\) nuclei, \(\sigma_{AX} \sim 10 A \sigma_0 \) is the cross section of the nucleus-\(X\) process, and \(\rho_{DM}^{\text{local}} \approx \SI{.3}{GeV / cm^3}\).

This then yields (after multiplying by \(\hbar^2 c^2\), and taking \(v_X \sim  \SI{2}{km/s}\)) 
%
\begin{align}
\text{event rate} \approx \SI{e-8}{s^{-1}}
\,,
\end{align}
%
or about 1 event a year.
    
The other method is indirect, we look for the products of \(X \overline{X}\) annihilation in our astronomical observations: photons, protons, electrons, neutrinos or their antiparticles.

\end{document}
