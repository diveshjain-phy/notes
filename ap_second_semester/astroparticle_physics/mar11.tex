\documentclass[main.tex]{subfiles}
\begin{document}

\marginpar{Wednesday\\ 2020-3-11, \\ compiled \\ \today}

\chapter{The particle physics Standard Model}

[For this lecture, we can refer to Peskin, chapter 2.]

We wish to describe what we described yesterday as ``matter'' and ``radiation''.

The problem is similar to the one we have in classical mechanics, an initial value problem: given the positions and velocities of the particles at a certain starting time \(t_0 \) we wish to compute their state at a later time \(t\).

This classical description in which the particles are not wavelike fails at the microscopic level: we want to give a quantum description of such a system of particles. 
We will derive it from the classical description using the standard tools of quantization. 
We start with a refresher of the classical description.

\subsection{The classical description of a system of particles}

Our aim is to compute and solve the equations of motion.
The usual approach is to use Hamilton's variational principle: it is the principle of least action, but it is not usually referred to as such: we are actually not \emph{minimizing} the action but finding a \emph{stationary point} for it.
This could also be a maximum or a saddle point.

The action functional \(S\) depends on the coordinates \(q_i(t)\) of the particles at at time \(t\), on the derivatives of these positions \(\dot{q}_{i}(t)\) which represent the velocities of the particles at a time \(t\). 
We usually write \(S \qty[q_{i}(t), \dot{q}_{i}(t)]\). 

If we fix \(q(t_0 )\) and \(q(t_{f})\), the positions at some initial and final time \(t_f\), we can then trace out a path \(q(t)\) and perturb it by \(\delta q(t)\); we fix \(\delta q(t_0 ) = \delta q(t_f) = 0\). 

Under this perturbation of the path \(q \rightarrow q + \delta q\), the action changes to \(S \rightarrow S + \delta S\). 
We then ask that \(\delta S =0\). 

\(S\) is an action: is dimensions are those of an energy times a time. 
In terms of the Lagrangian \(L\), the action is defined as 
%
\begin{align}
S \qty[q_i(t), \dot{q}_{i}(t)] = \int_{t_0 }^{t_f} L (q_i (t), \dot{q}_{i}(t)) \dd{t}
\,,
\end{align}
%
which means that the Lagrangian must have the dimensions of an energy. 
We will make use of a quantity called the Lagrangian density: 
%
\begin{align}
L = \int \mathscr{L} \qty(\phi (\vec{x}), \partial_{\mu } \phi (\vec{x})) \dd[3]{x}
\,.
\end{align}

From a finite number of particles we move to considering a field \(\phi (\vec{x})\): this means that, in a certain sense, we are considering an infinite number of particles. 

The dependence of the Lagrangian on the \(q_i\) and \(\dot{q}_{i}\) shifted to a dependence on the spacetime coordinates \(x\) and their 4-derivatives \(\partial_{\mu } x\).
It could depend on many fields simultaneously, we omit this dependence for simplicity.
Now, this Lagrangian density has the dimensions of an energy per unit volume. 

Then, the action, computed in a region \(\Omega \) of 4-dimensional spacetime, is 
%
\begin{align}
S = \int_{\Omega } \dd[4]{x} \mathscr{L} \qty(\phi (x), \partial_{\mu } (\phi (x)))
\,.
\end{align}

Now that we have established the notation, we can apply the action principle: we consider an infinitesimal variation of the field \(\phi \rightarrow \phi  + \delta \phi \). 
We require this variation to vanish not only at the initial and final time, but over all the boundary \(\partial \Omega \): 
%
\begin{align}
\eval{\delta \phi }_{\partial \Omega } =0
\,.
\end{align}

Then, imposing \(\delta S =0\) is equivalent to the Euler-Lagrange equations: 
%
\begin{align}
\pdv{\mathscr{L}}{\phi (x)} - \pdv{}{x^{\mu }} \qty(\pdv{\mathscr{L}}{\qty(\partial_{\mu} \phi (x))}) = 0
\,.
\end{align}

If we have many fields \(\phi_{r}\), then we have a set of E-L equations for each of them. 
This is still classical: for example, classical (relativistic) electrodynamics is formulated in this way.

The momenta are 
%
\begin{align}
\pi (x) = \pdv{\mathscr{L}}{\dot{\phi} (x)}
\,,
\end{align}
%
where a dot denotes a time derivative
and using this we define the Hamiltonian density by 
%
\begin{align}
\mathscr{H} (x) = \pi (x) \dot{\phi} (x) - \mathscr{L}(\phi, \partial_{\mu }\phi )
\,,
\end{align}
%
and similarly to the Lagrangian we have the full Hamiltonian \(H\)
%
\begin{align}
H = \int \dd[3]{x} \mathscr{H}
\,.
\end{align}

Now, these fields (\(\phi \) and \(\pi \)) are classical fields: we will apply classical quantization to them. 
They will need to satisfy the classical commutation relations: 
%
\begin{align}
\qty[\phi (\vec{x}, t), \pi (\vec{x}', t)] &= i \hbar \delta (\vec{x} - \vec{x}')  \\
\qty[\phi (\vec{x}, t), \phi  (\vec{x}', t)] &=
\qty[\pi (\vec{x}, t), \pi  (\vec{x}', t)] = 0
\,.
\end{align}
%
% \todo[inline]{\(\pi \) or \(\dot{\phi}\)?}
% \todo[inline]{What does dot mean?}
% \todo[inline]{Dependence on \(x\)?}

\subsection{A relativistic reminder}


the energy-momentum four-vector is 
%
\begin{align}
p^{\mu } = \left[\begin{array}{cccc}
E & \vec{p} c
\end{array}\right]
\,,
\end{align}
%
where the greek index \(\mu \) can take values from 0 to 3. 

The metric signature used here is the the mostly minus one. So, \(p^{\mu } q_{\mu } = E_{p} E_{q} - \vec{p} \cdot \vec{q}\), since we raise and lower indices using the metric \(\eta_{\mu \nu }\). 

The square norm of the 4-momentum is \(p \cdot p  = p^2 = E^2 - \abs{p}^2 c^2\). It is Lorentz invariant.

In the rest frame of the observer, \(p^{\mu } = \qty[E_{0}, \vec{0}]\), and this \(E_0\) is just (\(c^2\) times)
 the mass of the particle: this is the \emph{definition} of mass. 

When the relation is satisfied we have 
%
\begin{align}
p^2 &= E^2 - \abs{p}^2 c^2= \qty(mc^2)^2  \\
E &= c \sqrt{\abs{p}^2 + (mc)^2}
\,.
\end{align}

% \todo[inline]{There is an extra \(c\) in the notes.}

When this relation is satisfied we say we are \emph{on shell}: for virtual particles, instead, this might not be satisfied.

We will use natural units: \(\hbar = c = 1\). 

This means that we equate energies (\SI{}{eV}) and angular velocities (\SI{}{Hz}); also we equate times (\SI{}{s}) and lengths (\SI{}{m}). 

The rest energy of the electron is \(m_e \approx \SI{511}{keV}\). 
Let us consider an electron with a momentum \(p\) of the order of its mass \(m_e\): then, its uncertainty in position is of the order 
%
\begin{align}
\frac{\hbar}{p c} \approx \frac{1}{m_e} \approx \SI{4e-11}{cm}
\,.
\end{align}

The dimensions of the lagrangian density, in natural units, are those of an energy to the fourth power, or a length to the \(-4\), or a mass to the fourth. 

Another useful exercise is to calculate the coupling of the electromagnetic field: 
%
\begin{align}
V(r) = \frac{e^2}{4 \pi \epsilon_{0} r} = \frac{e^2}{4 \pi } \frac{1}{r} 
\,,
\end{align}
%
since we set \(\epsilon_{0} = \mu_0 = 1\). 
We can introduce the electromagnetic \(\alpha \): this is 
%
\begin{align}
\alpha = \frac{e^2}{4 \pi } \times \frac{1}{\hbar c} 
\,.
\end{align}

This then becomes adimensional: \(\alpha \approx 1 / 137\). 
It represents the strength of the electromagnetic interaction: the strength of the coupling of the photon to the electron.  
The fact that it is \(\sim \num{e-2}\) is important: it allows us to work in a perturbative way, in powers of \(\alpha \). 

What is the coupling of the strong and weak interactions? This will be discussed.

Next time, we will discuss symmetries and symmetry breaking. 

\end{document}
