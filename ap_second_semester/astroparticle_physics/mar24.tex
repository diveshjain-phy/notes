\documentclass[main.tex]{subfiles}
\begin{document}

\section{Relativistic wave equations}

\marginpar{Tuesday\\ 2020-3-24, \\ compiled \\ \today}

Last week, we moved from the usual Schrödinger equation to the Klein Gordon equation by moving from \(E = p^2 / 2m\) to \(E^2 = p^2 + m^2\). 
The latter is covariant under Lorentz transformations. 

There are, however, two issues with the Schrödinger equations:
one is that it is not relativistic, while the second one is more subtle. 

The Schrödinger equation describes the evolution of a single particle in time, while when we deal with elementary particles a one-particle description is unsuitable: we must have conservation of probability, so we are unable to describe situations in which a particle disappears by decaying into other particles, or we have inelastic collisions. 

This is the reason why we need a multiparticle description. 
An example of the issues which arise in the single-particle relativistic description is the Klein paradox. 

So, we will introduce the so-called second quantization formalism. 
We are going to interpret \(\phi \) as a \textbf{quantum field operator}, instead of a scalar field. Let us make this explicit. 
We take the KG equation from the first quantization to the KG equation describing a Quantum Field Theory. 

\(\phi (x)\) will be an operator which can destroy or create particles. 
We start from the quantization of the Hamiltonian density, \(\mathscr{H}\): we consider its ground state \(\ket{0}\). 
This corresponds to a state in which there is no particle, and is called the \textbf{vacuum} state. 

We will also have higher-energy states, in which we will have one or more particles: how do we describe these? 
A one particle state \(\ket{\varphi_{1 } (p)}\) can be acted upon by the operator \(\phi (x)\): it is destroyed, yielding a state which is aligned with the vacuum. Formally, we have 
%
\begin{align}
\bra{0} \phi (x) \ket{\varphi_1 (p)} = e^{-ipx}
\,,
\end{align}
%
where \(p^{0}\) is the positive energy: \(p^{0} = + E_p = + \sqrt{\vec{p}^2 + m^2}\). 

Now, let us consider the complex conjugate of this matrix element: we swap the bra and ket and take the adjoint of the operator, to get 
%
\begin{align}
\bra{\varphi_1 (p)} \phi ^\dag \ket{0} = e^{ipx}
\,.
\end{align}

This means that, if the first equation describes a particle propagating with momentum \(p^{\mu }\), this new equation will now describe a particle propagating with momentum \(- p^{\mu }\). 

This will now have a negative energy. 
This was a problem historically, now we give the solution directly. 

Let us introduce a particle \(\ket{\varphi_2 (p)}\), such that 
%
\begin{align}
\bra{0} \phi ^\dag (x) \ket{\varphi_2 (p)} = e^{-ipx}
\,,
\end{align}
%
so now this particle  has the same mass \(m\), but this new particle \(\ket{\varphi_2}\) differs from \(\ket{\varphi_1 }\) only for the charge, which is now opposite. 

\todo[inline]{Add clarification: in general we change the sign of the coupling, which is in front of the V in the Lagrangian.}

Do note that charge does not exclusively mean electric charge! We can also have other kinds of charges. 
For example, neutrinos have the charge of lepton number. 

Particles can be their own antiparticles, if they have zero charge.
So, we can interpret the negative energy solution as the presence of an antiparticle: 
a negative energy particle would correspond to a positive energy antiparticle. 

A generic field theory is defined by its action, which is written from the density Lagrangian: a free Lagrangian is  
%
\begin{align}
\mathscr{L} = \frac{1}{2} \qty(\partial^{\mu } \phi \partial_{\mu } \varphi - m^2 \varphi^2)
\,,
\end{align}
%
and if we impose \(\var{S}=0\) we get precisely the KG equation as our equation of motion. 

\subsection{Spin 1}

We discuss 3D vectors \(V^{i}\), which transform under rotations \(R_{ij} \in SO(3)\) as 
%
\begin{align}
V^{i} \rightarrow V^{\prime i} = R_{ij}V^{j}
\,.
\end{align}

Now, the matrix element from before reads 
%
\begin{align}
\bra{0} V^{i}(x) \ket{v(p, \epsilon )} = \epsilon^{i} e^{-ipx}
\,,
\end{align}
%
where we need to account for the momentum \(p\) and the polarization \(\epsilon^{i} \). 
If we want to move to a relativistic description, we get the same thing, with the spatial index \(i\) being replaced by a 4-dimensional index \(\mu \):
%
\begin{align}
\bra{0} V^{\mu }(x) \ket{v(p, \epsilon )} = \epsilon^{\mu } e^{-ipx}
\,.
\end{align}

The problem is now the normalization of these states: what is the value of \(\braket{v}{v}\)? 
With the definition we gave, this is equal to \(\epsilon^{\mu } \epsilon_{\mu }\), not 1! 
In general this will not be positive. Take the photon: then, we have the two helicities, which are transverse degrees of freedom. 
For example, we can have \(\epsilon ^{\mu } = (0,1,1,0)\), which in our choice of metric, which is the mostly negative one. 

So, we could impose \(\braket{v}{v} = - \epsilon^{\mu } \epsilon_{\mu }\): but what if we make a boost?
\todo[inline]{The square modulus is covariant though!}

This is in general a problem. 

The electromagnetic field strength is defined as 
%
\begin{align}
A^{\mu }  = \qty(\varphi (x), \vec{A})
\,,
\end{align}
%
and the EM field strength is \(F^{\mu \nu } = 2 \partial^{[\mu } A^{\nu ]}\). So, we have 
%
\begin{align}
F^{i0} &= - \nabla^{i} \varphi - \partial_{t} A^{i} = E^{i}  \\
 F^{ij} &=  - 2\nabla^{[i} A^{j]} = \epsilon^{ijk} B^{k}
\,.
\end{align}

The Maxwell equations follow from the density Lagrangian 
%
\begin{align}
\mathscr{L} = - \frac{1}{4} F^{\mu \nu } F_{\mu \nu } - j^{\mu } A_{\mu }
\,,
\end{align}
%
where \(j^{\mu }\) is an external current. This yields \(\partial_{\mu } F^{\mu \nu } = j^{\nu }\). 

The photon is a massless vector boson. This description applies in general to a massless vector field. 
Are there massive vector fields?
Yes, the weak interaction is described by a massive vector boson \(W^{\pm}_{\mu }\).

\subsection{Spin 1/2}

Now we discuss spin \(1/2\) particles: we will need to have first derivatives on either side, as opposed to the second equations in the KG equation.

Our ansatz for what will be called the Dirac equation is: 
%
\begin{align}
i \partial_{t} = - i \vec{\alpha} \cdot \vec{\nabla} + \beta m
\,.
\end{align}

Is it possible to find four numbers (3 encoded in the vector \(\vec{\alpha} \), one more in \(\beta \)) so that the square of this relation is \(E^2 = p^2+m^2\) and that we still retain Lorentz invariance? 

If we try to impose these conditions, we find that there are no solutions: there are no such four numbers. 

We can, however, find a solution if we allow \(\vec{\alpha}\) and \(\beta \) to be matrices: specifically, \(4 \times 4 \) matrices, which must obey the \textbf{anticommutation relations}: 
%
\begin{align}
\qty{\alpha_{i}, \alpha_{j}} &= 2\delta_{ij} 
\qquad \text{and} \qquad
\qty{\alpha_{i}, \beta } = 0
\,,
\end{align}
%
and now we define 
%
\begin{align}
\gamma^{0} = \beta 
\qquad \text{and} \qquad
\gamma^{i} = \beta \alpha^{i}
\,,
\end{align}
%
which must be 4D, as we said, and the simplest representation is called the Dirac representation:
%
\begin{align}
\gamma^{0} = \left[\begin{array}{cc}
\mathbb{1} & 0 \\ 
0 & \mathbb{1}
\end{array}\right] 
\qquad \text{and} \qquad
\gamma^{i} = \left[\begin{array}{cc}
0 & \sigma^{i}   \\ 
\sigma^{i} & 0
\end{array}\right]
\,.
\end{align}

It can be verified that then 
%
\begin{align}
\qty{\gamma^{\mu }, \gamma^{\nu }} = 2 \eta^{\mu \nu }
\,.
\end{align}

We can move between representations of these matrices using unitary transformations. 
Inserting these, we find: 
%
\begin{align}
\qty[ i \gamma^{0} \pdv{}{t} + i \vec{\gamma} \cdot \vec{\nabla} - m \mathbb{1}] \varphi (x) = 0
\,,
\end{align}
%
which means that the wavefunction must be 4-dimensional as well, since we are acting on it with a 4D operator. 
We are going to call such an object a \textbf{spinor}. 
If we denote \(\gamma^{\mu } = \qty(\gamma^{0}, \vec{\gamma}) \) we can write the Dirac equation as 
%
\begin{align}
\qty(i \gamma^{\mu }\partial_{\mu } - m) \varphi (x) &= 0  \\
\qty(i \slashed{\partial} - m ) \varphi (x) &= 0
\,,
\end{align}
%
where we have defined the notation \(\slashed{x} = \gamma^{\mu } x_{\mu }\). 
As we shall see tomorrow morning, this equation is extremely rich in structure. 

In the Westminster abbey, this equation is inscribed as a homage to Paul Dirac.

\end{document}
