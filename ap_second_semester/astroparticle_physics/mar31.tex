\documentclass[main.tex]{subfiles}
\begin{document}

\marginpar{Tuesday\\ 2020-3-31, \\ compiled \\ \today}

% Last time we saw our first field theory: QED. 

We have associated spin \(1/2\) fermions to matter fields. 
Spin \(1\) particles, instead, are vector bosons, such as the photon \(\gamma\), the gluon \(g\) and the weak \(W^{\pm}\) and \(Z^{0}\) bosons. These are the radiation fields.  
We will explore this in more detail. 

This connection comes from the \textbf{spin-statistics} theorem: it states that particles with integer spin obey Bose-Einstein statistics, while particles with half-integer spin obey Fermi-Dirac statistics.
Particles which are ``matter'' (electrons, quarks and such) are fermions, while particles which are ``force carriers'' (photons, weak-interaction \(W\) and \(Z\) particles,  gluons) are bosons.

This is due to the fact that in order to consistently quantize Dirac's theory we need to use \textbf{anticommutators} instead of field commutators to replace the Poisson brackets of the classical theory. 

Up until now, we have observed only particles with spins \(0\), \(1/2\) and \(1\). 

We could have a symmetry called supersymmetry, which connects fermions and bosons.

The graviton has spin \(2\); in supergravity the graviton has a fermion partner called the ``gravitino'' with spin \(3/2\). 

\subsection{Scattering}

How do we normalize the states in relativistic theory? 
Classically we did 
%
\begin{align}
\braket{p_1 }{p_2  } = (2 \pi )^3 \delta^{(3)} (\vec{p}_{1} - \vec{p}_{2})
\,,
\end{align}
%
in the relativistic case instead we will do 
%
\begin{align}
\braket{p_1 }{p_2  } = 2 E_{p_1 } (2 \pi )^3 \delta^{(3)} (\vec{p}_{1} - \vec{p}_{2})
\,.
\end{align}

This is a Lorentz invariant normalization (although not manifestly so --- see Peskin \cite[sec.\ 3.5]{peskinConceptsElementaryParticle2019} for a proof). 

The relativistic volume element is given by 
%
\begin{align}
\int \frac{ \dd[3]{p}}{(2\pi )^3} \rightarrow
\int \frac{ \dd[4]{p}}{(2 \pi )^{4}} (2 \pi ) \delta (p^2-m^2)
= \int \frac{ \dd[3]{p}}{(2\pi )^3} \frac{1}{2E_p}
\,,
\end{align}
%
where we integrated over the variable \(p^{0}=E\), which removed the \(\delta (p^2-m^2) = \delta (E - \sqrt{p^2+m^2}) / 2E\). 

This way, we have the completeness relation 
%
\begin{align}
\int \frac{ \dd[3]{p}}{(2 \pi )^3} \frac{1}{2 E_p } \dyad{p} = \mathbb{1}
\,.
\end{align}

States have the dimension of an energy to the -1, field operators have the dimension of an energy. 

There are two main kinds of processes which are considered in particle physics. 
The first is a \textbf{decay} process: we have a particle \(A\) decaying into a possibly multi-particle state \(f\). 
We are interested in the decay \emph{rate} of this process. 

The probability of survival of particle \(A\) at time \(t\) is in the form \(\mathbb{P}(t) = \exp(-t / \tau_{A})\), so we define the decay rate
%
\begin{align}
\Gamma_{A} = \frac{1}{\tau_{A}}
\,,
\end{align}
%
which has the dimensions of a frequency, or equivalently an energy. 

Generally the decay of a certain particle species can happen through different \textbf{channels}, that is, into different kinds of particles.
We can define the branching ratios as 
%
\begin{align}
BR(A \to f) = \frac{ \Gamma (A \to f)}{\Gamma_{A}}
\,.
\end{align}

Another process of interest is a \textbf{scattering} process of \(n \to m \) particles. 
There is no particle number conservation: we can create and destroy as many particles as we like. 
These types of processes are described by their \emph{cross section}, which is an effective area corresponding to how aligned the trajectories of the incoming particles must be in order for them to interact.

This cross section allows us to compute the average time for an interaction to occur: if two particles' interaction becomes so rare that they cannot interact within a Hubble time then they are said to have \emph{decoupled}. 

We start by considering fixed-target experiments: we want to know how many events per second we will have, which will be given by 
%
\begin{align}
\frac{\text{\# events}}{\text{second}} = n_A \times v_A \times \sigma 
\,,
\end{align}
%
where \(\sigma \) is the cross section, \(v_A\) is the velocity of the incoming particles, while \(n_A\) is the number density of particles in the beam. The number density of the particles in the target is accounted for inside of \(\sigma \). 
This tells us that \(\sigma\) has the dimensions of an area. 

If we have two beams of particles coming towards each other, the term will look like \(n_A n_B (v_A - v_B) \ell_B A_b \sigma \). 

In general, we will be interested in the differential cross section 
%
\begin{align}
\frac{ \dd{\sigma }}{ \dd[3]{p_1 } \dd[3]{p_2 } \dots \dd[3]{p_{n}}}
\,.
\end{align}

This is just a definition, and it is not covariant, if we want to integrate we still need to use the covariant momentum element. We can integrate it in \(\dd[3]{p_1} \dots \dd[3]{p_n}\) in order to recover the total cross section, but inside it we have more information about the angular properties of the process.

For a scattering process like \(A+B \rightarrow 1 + \dots + n\)
we will need to compute things like 
%
\begin{align} \label{eq:s-matrix-element-feynman-amplitude-decay}
\bra{12 \dots n} T \ket{AB} = \mathcal{M} (A+B \rightarrow 1 + \dots + n) (2\pi )^{4} \delta^{(4)} (P_A + P_B - \sum p_i)
\,,
\end{align}
%
where \(T\) is the time evolution, and we defined the invariant scattering amplitude \(\mathcal{M}\). 
If we want to compute the width \(\Gamma_{A}\) for a decay process, we need to define the phase space integral:
%
\begin{align}
\int \dd{\Pi_{n}} = \underbrace{\prod_{i} \int \frac{ \dd[3]{p_i}}{2 E_i (2 \pi )^3} 
(2\pi )^{4} \delta^{(4)} (P_A - \sum _{i} p_i)}_{\text{phase space integral}}
\,,
\end{align}
%
so we will have what is called Fermi's golden rule:
%
\begin{align}
\Gamma_{A} = \frac{1}{2 M_A} \int \dd{\Pi_{n}} \abs{\mathcal{M} (a \rightarrow f)}^2
\,,
\end{align}
%
since in order to get the probability we need to take the square of the amplitude. 

Now, the Feynman amplitude \(\mathcal{M}\)'s dimensionality can be inferred from equation \eqref{eq:s-matrix-element-feynman-amplitude-decay}: the dimension of a state is \(1 / [M]\), the dimension of a four dimensional delta function is \(1/ [M]^{4}\), while the time evolution operator is dimensionless, so we have 
%
\begin{align}
[M]^{-n-1} = 
[\mathcal{M}]  [M]^{-4} \implies [\mathcal{M}] = [M]^{3-n}
\,,
\end{align}
%
where \(n\) is the number of outgoing particles. Here by \([M]\) we mean the dimensions of a mass, energy, inverse length or inverse time. 
The dimension of the phase space element can similarly be found to be \([M]^{2n - 4}\); so we can check that Fermi's golden rule is dimensionally consistent: its dimensions read 
%
\begin{align}
[\Gamma_A ] = [M]^{-1} [M]^{2n-4}  \qty([M]^{3-n})^2 = [M]
\,,
\end{align}
%
which makes sense, since the decay rate is an inverse time.

The expression we gave is \textbf{polarized}, that is, by deciding on the initial and final states we are fixing the spins of the particles.
This may be useful in some cases, but often we cannot control the spins of the incoming particles, and/or we cannot select only outgoing ones with a certain spin configuration.
So, what is done usually is to \textbf{average} over the initial polarizations' probabilities, and to \textbf{sum} over the final ones. Note that in doing this we must add probabilities, not amplitudes: in this context we are dealing with classical mixtures.

For cross sections, Fermi's golden rule reads: 
%
\begin{align}
\sigma (A+B \to f) = \frac{1}{2 E_A E_B \abs{v_A - v_B}}
\int \dd{\Pi_2 } \abs{\mathcal{M}(A+B \to f)}^2
\,,
\end{align}
%
where (if we are in the COM frame) the phase space integral for the two final particles case is (nontrivially! see the TP notes \cite[sec.
 4.2.3]{tissinoTheoreticalPhysicsNotes2020}) given by: 
%
\begin{subequations}
\begin{align}
\int \dd{\Pi_{2}} &= \int \frac{\dd[3]{p}}{(2 \pi )^3} \frac{1}{2 E_1 } \frac{1}{2 E_2 } (2 \pi ) \delta (E_{CM} - E_1 - E_2 )   \\
&= \frac{1}{8 \pi } \frac{2p}{E_{CM}} \int  \frac{\dd{\Omega}}{4 \pi } 
\,.
\end{align}
\end{subequations}

The dimensionality check now gives \([\mathcal{M}] = [M]^{2-n}\), so 
%
\begin{align}
[\sigma ] = [M]^{-2} [M]^{2n-4} \qty([M]^{2-n})^2 = [M]^{-2}
\,,
\end{align}
%
which makes sense: it is a length squared, and lengths are inverse masses.

A kind of process which appears often is a \textbf{resonance}: something like \(A+B \to X \to A+B\), where we have a certain short-lived intermediate state \(X\), whose decay rate is \(\Gamma \).
The nonrelativistic Breit-Wigner formula describes this: it is given by 
%
\begin{align}
\mathcal{M} \sim \frac{1}{E - E_X + i \Gamma/ 2}  
\,,
\end{align}
%
where \(E_X\) is the energy of \(X\), while \(E\) is the center-of-mass energy of the process. 

The paradigmatic process for these kinds of interactions is the process \(e^{+}e^{-} \to \mu^{+}\mu^- \). 
In the noted it is sketched how one would go about discussing such a process. 

\end{document}
