\documentclass[main.tex]{subfiles}
\begin{document}

\marginpar{Tuesday\\ 2020-3-31, \\ compiled \\ \today}

Last time we saw our first field theory: QED. 

We have associated spin \(1/2\) fermions to matter fields. 
Spin \(1\) particles, instead, are vector bosons, such as the photon \(\gamma\), the gluon \(g\) and the weak \(W^{\pm}\) and \(Z^{0}\) bosons. These are the radiation fields.  
We will explore this in more detail. 

This connection comes from the \textbf{spin-statistics} theorem: it states that particles with integer spin obey Bose-Einstein statistics, while particles with half-integer spin obey Fermi-Dirac statistics.

Up until now, we have observed only particles with spins \(0\), \(1/2\) and \(1\). 

We could have a symmetry called supersymmetry, in which fermions and bosons are associated. 

The graviton has spin \(2\); in supergravity the graviton has a fermion partner called the ``gravitino'' with spin \(3/2\). 

How do we normalize the states in relativistic theory? 
Classically we did 
%
\begin{align}
\braket{p_1 }{p_2  } = (2 \pi )^3 \delta^{(3)} (\vec{p}_{1} - \vec{p}_{2})
\,,
\end{align}
%
in the relativistic case instead we will do 
%
\begin{align}
\braket{p_1 }{p_2  } = 2 E_{p_1 } (2 \pi )^3 \delta^{(3)} (\vec{p}_{1} - \vec{p}_{2})
\,.
\end{align}

The relativistic volume element is given by 
%
\begin{align}
\int \frac{ \dd[3]{p}}{(2\pi )^3} \rightarrow
\int \frac{ \dd[4]{p}}{(2 \pi )^{4}} (2 \pi ) \delta (p^2-m^2)
= \int \frac{ \dd[3]{p}}{(2\pi )^3} \frac{1}{2E_p}
\,,
\end{align}
%
where we integrated over the variable \(p^{0}=E\), which removed the \(\delta (p^2-m^2) = \delta (E - \sqrt{p^2+m^2}) / 2E\). 

Let us consider a \textbf{decay} process: we have a particle \(A\) decaying into a possibly multi-particle state \(f\). 
We are interested in the decay \emph{rate} of this process. 

Another process of interest is a \textbf{scattering} process of \(n \to m \) particles. 
There is no particle number conservation: we can create and destroy as many particles as we like. 
These types of processes are described by their \emph{cross section}, which is an effective area corresponding to how aligned the trajectories of the incoming particles must be in order to interact.

This cross section allows us to compute the average time for an interaction to occur: if two particles' interaction becomes so rare that they cannot interact within a Hubble time then they are said to have \emph{decoupled}. 

The probability of survival of particle \(A\) at time \(t\) is in the form \(\mathbb{P}(t) = \exp(-t / \tau_{A})\), so we define 
%
\begin{align}
\Gamma_{A} = \frac{1}{\tau_{A}}
\,,
\end{align}
%
which has the dimensions of a frequency, or equivalently an energy. 

We can define the branching ratios as 
%
\begin{align}
BR(A \to f) = \frac{ \Gamma (A \to f)}{\Gamma_{A}}
\,.
\end{align}

We can consider fixed-target experiments: we want to know how many events per second we will have, which will be given by 
%
\begin{align}
\frac{\text{\# events}}{s} = n_A \times v_A \times \sigma 
\,,
\end{align}
%
where \(\sigma \) is the cross section, \(v_A\) is the velocity of the incoming particles, while \(n_A\) is the number density of particles in the target. 

If we have two beams of particles coming towards each other, the term will look like \(n_A n_B (v_A - v_B) \ell_B A_b \sigma \). 

In general, we will be interested in the differential cross section 
%
\begin{align}
\frac{ \dd{\sigma }}{ \dd[3]{p_1 } \dd[3]{p_2 } \dots \dd[3]{p_{n}}}
\,.
\end{align}

This is just a definition, and it is not covariant, if we want to integrate we still need to use the covariant momentum element. 

For a scattering process like \(A+B \rightarrow 1 + \dots + n\)
we will need to compute things like 
%
\begin{align}
\bra{12 \dots n} T \ket{AB} = \mathcal{M} (A+B \rightarrow 1 + \dots + n) (2\pi )^{4} \delta^{(4)} (P_A + P_B - \sum p_i)
\,,
\end{align}
%
where \(T\) is the time evolution, and we defined the invariant scattering amplitude \(\mathcal{M}\). 
If we want to compute the width \(\Gamma_{A}\) for a decay process, we need to define the phase space integral:
%
\begin{align}
\int \dd{\Pi_{n}} = \underbrace{\prod_{i} \int \frac{ \dd[3]{p_i}}{2 E_i (2 \pi )^3} 
(2\pi )^{4} \delta^{(4)} (P_A - \sum _{i} p_i)}_{\text{phase space integral}}
\,,
\end{align}
%
so we will have what is called Fermi's golden rule:
%
\begin{align}
\Gamma_{A} = \frac{1}{2 M_A} \int \dd{\Pi_{n}} \abs{\mathcal{M} (a \rightarrow f)}^2
\,,
\end{align}
%
since in order to get the probability we need to take the square of the amplitude. 

[Discussion on the dimensionality of \(\mathcal{M}\)]

If we are in the COM frame, the phase space integral for the initial particles reads: 
%
\begin{align}
\int \dd{\Pi_{2}} &= \int \frac{\dd[3]{p}}{(2 \pi )^3} \frac{1}{2 E_1 } \frac{1}{2 E_2 } (2 \pi ) \delta (E_{CM} - E_1 - E_2 )   \\
&= \frac{1}{8 \pi } \frac{2p}{E_{CM}} \int  \frac{\dd{\Omega}}{4 \pi } 
\,.
\end{align}

The nonrelativistic Breit-Wigner formula is given by 
%
\begin{align}
\mathcal{M} \sim \frac{1}{E - E_R + i \Gamma/ 2}  
\,.
\end{align}

The paradigmatic process for these kinds of interactions is the process \(e^{+}e^{-} \to \mu^{+}\mu^- \). 
In the noted it is sketched how one would go about discussing such a process. 

\todo[inline]{So we are assuming \(\abs{\psi (t)}^2 \sim \exp( - \Gamma t)\), but then how does the probability of finding increase? The two exponentials do not meet nicely!}

\end{document}
