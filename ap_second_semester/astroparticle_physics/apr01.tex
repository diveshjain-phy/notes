\documentclass[main.tex]{subfiles}
\begin{document}

\marginpar{Wednesday\\ 2020-4-1, \\ compiled \\ \today}

This and next week we will finish the introduction to particle physics, then we will start discussing the open problems in cosmology and astroparticle physics. 

We consider the following process: 
%
\begin{align}
e^{+} e^{-} \to \mu^- \mu^+
\,,
\end{align}
%
where the mass of the electron is around \(m_e \sim \SI{.5}{MeV}\), the mass of the muon is around \(m_{\mu } \sim \SI{100}{MeV}\). 

Digression: there are different families of leptons, the first encompasses \(e, \nu_{e}, u, d\); the second encompasses \(\mu, \nu_{\mu }, c, s\) and the third encompasses \(\tau, \nu_{\tau }, t, b\). 
The characteristics of the four members of the family are well-known, and between families the characteristics are the same: the only thing which varies between the families is the mass. 

So, Rabi famously asked ``who ordered the fermions''? 

Coming back to our problem: the state \(\ket{e^{+}e^{-}}\) must be annihilated by the EM current \(j^{\mu } = \overline{\psi} \gamma^{\mu } \psi \); it is then converted to a photon, which however is not on mass shell --- it cannot be since we cannot go to its rest frame. 
It is then called a \emph{virtual photon}. 

Then, this photon decays to a muon-antimuon pair: then we will have a term \(\bra{\mu^- \mu^+} j^{\mu }\).

Let us call \(p_{-}\) and \(p_{+}\) the momenta of the electron and positron, and \(p^{\prime }_{-}\) and \(p^{\prime }_{+}\) those of the muon and antimuon. 

The momentum \(q\) of the photon cannot have \(q^2  =0 \), but this is fine: it is just an excitation. 

The physics of the process is all contained in the matrix element \(\mathcal{M}(e^{+}e^{-} \to \mu^+ \mu^-)\). How do we calculate it? we have 
%
\begin{align}
\mathcal{M}(e^{+}e^{-} \to \mu^+ \mu^-)
(-e)
\bra{\mu^- \mu^+} j^{\mu } \ket{0}
\frac{1}{q^2}
(-e)
\bra{0} j_{\mu } \ket{e^{+} e^{-}}
\,,
\end{align}
%
where the \(-e\) factor is because of the EM coupling to the photon. 
The Breit-Wigner factor looks like 
%
\begin{align}
\frac{1}{P^2-M_R^2}
\,,
\end{align}
%
but for the photon we have no mass, therefore we only get a factor \(1/q^2\). 

\todo[inline]{What is \(j^{\mu }\) explicitly?}

The Feynman diagrams are just a way to collect the Feynman rules needed to compute the process, they are not meant to represent how the process ``looks like''. 

Let us take the ultrarelativistic limit, in which the energy of the process is much larger than the muon's mass. 
If this is the case, then we can set \(m_e = m_\mu  = 0\). 

Let us consider the Dirac equation, in the case in which the mass \(m\) is equal to zero: then we get 
%
\begin{align}
i \slashed{\partial} \psi = 0
\,.
\end{align}

Let us use the chiral representation for the \(\gamma \) matrices: 
%
\begin{subequations}
\begin{align}
\gamma^{\mu } = \left[\begin{array}{cc}
0 & \sigma^{\mu } \\ 
\overline{\sigma}^{\mu } & 0
\end{array}\right] 
\,,
\end{align}
\end{subequations}
%
where we mean by \(\sigma^{0} = \mathbb{1}\), and \(\overline{\sigma}^{\mu } = (\sigma^{0}, - \sigma^{i})\).

Let us then split the spinor \(\psi \) into 
%
\begin{subequations}
\begin{align}
\psi = \left[\begin{array}{c}
\psi_{L} \\ 
\psi_{R}
\end{array}\right]
\,,
\end{align}
\end{subequations}
%
where \(\psi_{L, R}\) are two-component spinors. This allows us to write two two-dimensional equations: 
%
\begin{subequations}
\begin{align}
i \overline{\sigma}^{\mu } \partial_{\mu } \psi_{L} &= 0 \\
i \sigma^{\mu } \partial_{\mu } \psi_{R} &= 0
\,,
\end{align}
\end{subequations}
%
so if we have no mass the equations decouple. 

We can do the same thing if the Dirac equation is coupled to the EM field, since the issue is with the structure of the \(\gamma^{\mu }\), it does not matter if we have \(\gamma^{\mu } \DD_{\mu }\) or \(\gamma^{\mu } \partial_{\mu }\). 

So, for the right-handed spinor we have:
%
\begin{align}
\qty(i \partial_{t} + i \sigma_{3} \partial_3 ) \psi_{R}
\,,
\end{align}
%
so if we have a plane wave 
%
\begin{align}
\psi_{R} = U_R (p) e^{-iEt + i \vec{p} \cdot \vec{x}}
\,.
\end{align}
%
the equation reads 
%
\begin{subequations}
\begin{align}
\qty(E - p \sigma^{3}) U_R = \left[\begin{array}{cc}
E-p & 0 \\ 
0 & E+p
\end{array}\right] U_{R} = 0
\,,
\end{align}
\end{subequations}
%
so we must have two solutions: they look like 
%
\begin{subequations}
\begin{align}
\psi_{R} = \left[\begin{array}{c}
1 \\ 
0
\end{array}\right] e^{-i Et + i E x_3 }
\qquad \text{and} \qquad
\psi_{R} = \left[\begin{array}{c}
0 \\ 
1
\end{array}\right] e^{+i Et + i E x_3 }  
\,.
\end{align}
\end{subequations}

\todo[inline]{How can we move between these using \(P\), \(C\) and \(T\)?}

In the end, we have that our quantum field operator \(\psi_{R}\) acts as 
%
\begin{align}
\bra{0}\psi_{R} \ket{e^{-}_{E} (p)} = U_R(p) e^{-ipx}
\,.
\end{align}

Then, we can have it acting in the other way as 
%
\begin{align}
\bra{e^{+}_{L}(p)} \psi_{R} \ket{0} = v_L (p) e^{+ipx}
\,.
\end{align}

\todo[inline]{Why is it flipped like this?}

If we were to repeat the analysis for the other spinor, we would get the specular result. 

At high energies we can move from electron-left to an electron-right, but at a price \(m / E\). 

In order to describe this spin, we introduce the helicity quantum number, which is defined as 
%
\begin{align}
h = \hat{p} \cdot \vec{s}
\,,
\end{align}
%
the projection of the spin along the direction of motion. 
For \(\psi_{R}\), we have the \((1,0)\) state with helicity \(h = 1/2\), while the state \((0,1)\) is a positron with helicity \(h = - 1/2\). 

If \(m=0\), then helicity is exactly conserved. 

Accounting for everything, the cross section comes out to be 
%
\begin{align}
\sigma = \frac{4 \pi }{3} \frac{\alpha^2}{E^2_{CM}}
\,.
\end{align}
%
we could have guessed the dependence on \(\alpha^2  / E^2_{CM}\), but for the numerical factor we needed to do the full computation. 
This is because the cross section is a length square, so it must depend on the inverse square of our only energy parameter, \(E_{CM}\). 

Also, the coupling was fixed: we are working in QED, so we only have a coupling constant: \(e\), so we will have terms \(e^2 / 4 \pi = \alpha \) inside of \(\mathcal{M}\), so we will get \(\alpha^2\) inside of \(\abs{\mathcal{M}}^2\). 
The \(4 \pi \)s will cancel because of the phase-space angular integrals.

If we have done this, can we apply it? suppose we want to compute the cross section \(\sigma (e^{+}e^{-} \to \text{hadrons})\). How could we do it?
% Well, all these processes look similar, but 
We can make a similar kind of reasoning, but we get a multiplicity of 3: this indicates that we have a different type of charge.

We shall see that this is related to certain kinds of internal symmetries of our field theory. 

\end{document}
