\documentclass[main.tex]{subfiles}
\begin{document}

\marginpar{Tuesday\\ 2020-3-10}

\section{Introduction}

Professor Antonio Masiero, \url{antonio.masiero@unipd.it}

There are two courses, for Astrophysics and Cosmology and for Physics, which bear the same name. 
The other one is by Francesco D'Eramo: it assumes a knowledge of Quantum Field Theory. 

This course, instead, only requires knowledge of Quantum Mechanics. 
The first part of this course is devoted to an introduction about the basics of Quantum Field Theory and gauge theories.

``By the end of the 20th century [\dots] we have a comprehensive, fundamental theory of all observed forces of nature which has been tested and might be valid from the Planck length scale of \SI{e-33}{cm} to the edge of the universe \SI{e28}{cm}''. David Gross, 2007. 

The task in APP is to be able to discuss such a fundamental theory. 

First of all, we need to address the two standard models: the \(\Lambda \)CDM model for cosmology and the Standard Model of particle physics.

There are points of friction between the two Standard Models.
There are also several questions: neutrinos' mass, what caused inflation\dots 

These problems have a common denominator: the interplay between particle physics, cosmology and astrophysics.
What we seek is new physics, beyond the two standard models. 

Books: M.E. Peskin, ``Concepts of Elementary Particle Physics''. The book is addressed to students who are not experts in QFT and particle physics, rather, it provides the fundamental knowledge for these topics.

The exam is a colloquium, an oral exam, for which we can prepare a presentation on a specific topic. 
There is no issue if we do not precisely remember a specific formula, it is about going deep in the concepts.

\subsection{An overview of the astroparticle physics landscape}

\paragraph{Fundamental particles: the SM of particle physics}

Elementary particles make up ordinary matter. 
Fermions have spin \(1/2\), and are composed of quarks: 
%
\begin{align}
\left[\begin{array}{ccc}
u & c & t \\ 
d & s & b
\end{array}\right]
\,,
\end{align}
%
leptons: 
%
\begin{align}
\left[\begin{array}{ccc}
\nu_{e} & \nu_{\mu } & \nu_{\tau } \\ 
e & \mu  & \tau 
\end{array}\right]
\,.
\end{align}

The muon and tau particles are similar to electrons, but with higher mass. 

These particles' interactions are mediated by 12
vector bosons, which have spin 1: these are ``radiation'' (the term is outdated).

\begin{enumerate}
  \item gluons (\(g\)) mediate the strong nuclear interaction, there are 8 of them;
  \item the \(W^{\pm}\) and \(Z^{0}\) bosons mediate the weak interaction;
  \item the photon (\(\gamma \)) mediates the electromagnetic interaction. 
\end{enumerate}

There was a need for a mechanism to provide mass to the weak bosons and the fermions: this is accounted for by the Higgs boson, which is a scalar (that is, it has spin 0). 
This realizes the electroweak symmetry breaking.

The issue is that gravity is missing.
In order to describe it in this scheme we would need a way to quantize it: all of these particles are actually excitations of quantum fields. 

There are two marvelous 20th century theories, but they are not compatible. 

\paragraph{Unification of interactions}

In 1687 Newton unified two domains of interactions: the terrestrial phenomena and the celestial phenomena, establishing the universality of gravitational interactions. 

In 1865 Sir Maxwell unified electricity and magnetism into electromagnetism. 

In 1967 Glashow, Weinberg and Salam propose the Standard Model of Particle Physics, unifying the Electromagnetic and Weak interactions. 
This is not a true unification: it is more appropriate to say that they are ``mixed together'' into the Electroweak interaction.  

In the Standard Model, there is a kind of ``frontier'' around \SI{100}{GeV}: below this energy, we see two interactions: the electromagnetic and the weak interaction. 
They are very much different: photons are massless, so the interaction has an infinite range, while the weak bosons are massive. 

How can these be unified? We will see; above \SI{100}{GeV} this apparent profound difference disappears in favour of the electroweak interaction. 
This is a phase transition.

Is the electroweak interaction above \SI{100}{GeV} massless or not?
Above this energy there is still a difference between the coupling constants of the two interactions.
Above this energy, the \(W\) and \(Z\) bosons are no longer massive. 

Above \SI{100}{GeV} the strong interaction is separated from the electroweak one. Maybe there is an energy at which the electroweak interaction is unified with the strong one? 
We shall explore this topic: there are theories (Grand Unified Theories, GUT) in which there is such a unification.

As the energy increases, the coupling of the strongest interactions becomes weaker.

The energy scale, however, is very large: around \SI{e16}{GeV}: this is a ``science fiction'' energy scale, it is extremely large. 
This is close to the Planck mass: \(M_P  \sim \SI{e19}{GeV}\), so we might not be able to describe this energy range with vanilla SM. 

\paragraph{The Standard Model of Cosmology}

Now we can work backwards in our energy scale: as time progresses forward from the Big Bang, the energy of particles decreases. 

The symmetry group of the Grand Unified Theory is broken, so we get subgroups; at each transition some symmetry is broken. 

The EM + weak into electroweak transition is not speculative: we have observed it at the LHC. 
On the other hand, the electroweak + strong into GUT transition is speculative. 

When, in the expansion of the universe, we reach an energy per particle of \(\approx \SI{1}{GeV}\) we have a new transition: the quark-hadron transition, so free quarks become confined into hadrons such as protons and neutrons. 

Around \SI{1}{MeV} we have a new transition: nucleosynthesis. 

Then, we reach recombination, which is when the radiation we see as the CMB is released.

There must be new physics somewhere: there is no room in the SM for dark matter, the matter-antimatter asymmetry, the mass of neutrinos. 

\end{document}
