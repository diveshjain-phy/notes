\documentclass[main.tex]{subfiles}
\begin{document}

\chapter{Early Universe}

\section{Basics of cosmology}

\marginpar{Tuesday\\ 2020-5-5, \\ compiled \\ \today}

We establish some basic notions in cosmology. We use natural units \(\hbar = c = k_B = 1\), so that we have the equivalence of mass, energy, temperature and angular frequency. 

The Planck units are defined by also setting \(G =1\).
The classic textbook is Kolb-Turner \cite[]{kolbEarlyUniverse1994}.
For something more recent, we have Gorbunov and Rubakov \cite[]{gorbunovIntroductionTheoryEarly2011}.

The FLRW metric is given by 
%
\begin{align}
\dd{s^2} = \dd{t}^2  - a^2(t) \gamma_{ij} \dd{x^{i}} \dd{x^{j}}
\,,
\end{align}
%
where \(\gamma_{ij}\) is the metric of a unit 3-sphere, 3-hyperboloid or 3-plane.

The Hubble parameter is given by \(H(t) = \dot{a}(t) / a(t)\). The Einstein equations read 
%
\begin{align}
R_{\mu \nu } - \frac{1}{2} g_{\mu \nu } R= 8 \pi G T_{\mu \nu }
\,.
\end{align}

The pressure is assumed to follow the law \(P = w \rho \), and fluids are characterized by their \(w\). 
The stress-energy tensor is \(T_{\mu \nu } = \diag(\rho , - P, -P, -P)\).

The Friedmann equations read 
%
\begin{align}
\qty( \frac{\dot{a}}{a})^2 &= \frac{8 \pi G}{3} \rho - \frac{k}{a^2}  \\
\dot{\rho} + 3 \frac{\dot{a}}{a}(\rho + P) &=0  \\
P &= P(\rho ) 
\,.
\end{align}

The dynamical \(\ddot{a}\) Friedmann equation is a consequence of the first two. 

If we have nonrelativistic matter, which is commonly called ``dust'', with \(w = 0\), we get \(\rho \propto a^{-3}\) and \(a(t) \sim t^{2/3}\).
So, the energy density scales like \(\rho (t) \sim t^{-2}\). 

The Hubble parameter, on the other hand, is given by \(H(t) = 2 / (3 t)\). This is not actually verified, since the assumption we are making of the universe being filled with matter does not hold. 

For relativistic matter, we have \(P = \rho /3\), so the density scales like \(\rho \sim a^{-4}\), \(a(t) \sim t^{1/2}\), \(H = 1/(2t)\) and \(\rho = \frac{3}{32 \pi G t^2} \).

The density scales like 
%
\begin{align}
\rho = \frac{\pi^2}{30} g_{*} T^{4}
\,,
\end{align}
%
and the Hubble parameter scales like 
%
\begin{align}
H^2 = \frac{8\pi G}{3} \rho = \frac{8}{90} \pi^3 g_{*}
\,.
\end{align}

Let us consider the vacuum: suppose we had a relation like 
%
\begin{align}
T_{\mu \nu } = \rho _{\text{vac}} \eta_{\mu \nu }
\,,
\end{align}
%
so that \(P = - \rho _{\text{vac}}\).

This is equivalent to the introduction of a \(- 8 \pi G \Lambda g_{\mu \nu }\) term to the left-hand side of the field equations. 

\subsection{The \(\Lambda \)CDM model}

The ingredients are: 
\begin{enumerate}
    \item Nonrelativistic matter: baryons, dark matter, and also neutrinos as long as \(m_{\nu } \gtrsim \SI{-3}{eV}\). 
    \item Relativity.
    \item Dark energy: is it vacuum energy? 
\end{enumerate}

The Hubble parameter is given by 
%
\begin{align}
H^2 = \frac{8 \pi G}{3} \qty(\rho_{M} + \rho _{\text{rad}} + \rho_{\Lambda } + \rho _{\text{curv}})
\,.
\end{align}

The critical density is given by 
%
\begin{align}
\rho_{c} = \frac{3}{8 \pi G} H_0^2 \approx \SI{5e-6}{GeV / cm^3}
\,.
\end{align}

This is obtained by taking \(h = \num{.7}\). 

We can take the ratios of \(\rho_{i} / \rho_{c} = \Omega_{i})\), this quantifies how much of the density of the universe is in the form of a certain kind of fluid. 

What is the vacuum energy predicting by the Standard Model? 
We know that there is symmetry breaking. 

What remains when we take the VEV are the scalar terms, since the vector fields are zero in the vacuum. Specifically, we are left with 
%
\begin{align}
\mu^2 \phi^2 + \lambda \phi^{4}
\,.
\end{align}

The order of magnitude of the VEV of the Higgs is around \SI{100}{GeV}. 
The dimensions of \(\rho = E / L^3\) are those of an energy to the fourth power. So, our estimate will be \(\rho \sim \qty(\SI{100}{GeV})^{4}\). 

The estimate from cosmology is \SI{e-10}{eV^{4}}; the one from particle physics is 120 orders of magnitude larger.

\end{document}