\documentclass[main.tex]{subfiles}
\begin{document}

\marginpar{Wednesday\\ 2020-4-15, \\ compiled \\ \today}

Last lecture we started the discussion of a new kind of interaction: we had treated QED and QCD, now we introduced the nuclear weak interaction. 

We would like to introduce a sort of ``QWD'', and we will see that this is possible in the standard model, by unifying the electromagnetic and weak interactions. 

We have seen that fermions are described by a four-component spinor: 
%
\begin{align}
\Psi = \left[\begin{array}{c}
\psi_{L} \\ 
\psi_{R}
\end{array}\right]
\,,
\end{align}
%
where \(\psi_{L, R}\) are both two-component vector. The right handed part has helicity \(h = + 1/ 2\), the left handed part has helicity \(h = - 1 / 2\).

The puzzling part of the weak interaction is that it seems like the only part of the fermion entering \(\beta \)-decay is the left-handed part.
In the late seventies, the professor's master thesis stated that this was a mystery. Still, we do not know why this interaction breaks parity. 

We need to describe the experimental results: so, we need a projector onto the left-handed components. 
We already introduced the Dirac \(\gamma^{\mu }\) matrices. 
Now we introduce 
%
\begin{align}
\gamma^{5} = \left[\begin{array}{cc}
- \mathbb{1} & 0 \\ 
0 & \mathbb{1}
\end{array}\right]
\,,
\end{align}
%
which has the important property that it anticommutes with all the Dirac matrices: 
%
\begin{align}
\qty{\gamma^{5}, \gamma^{\mu }} = 0 
\,,
\end{align}
%
and actually we can write it as 
%
\begin{align}
\gamma^{5} = i \gamma^{0}\gamma^{1}\gamma^{2}\gamma^{3}
\,.
\end{align}

By its form we can readily see that \(\psi_{L}\) is an eigenstate of \(\gamma^{5}\) with eigenvalue \(-1\), while \(\psi_{R}\) has eigenvalue \(+1\). 
So, the matrix 
%
\begin{align}
\frac{\mathbb{1} - \gamma^{5}}{2} 
\,
\end{align}
%
projects a state onto the left-handed subspace.
The Dirac matrices can be written as 
%
\begin{align}
\gamma^{\mu } \left[\begin{array}{cc}
0 & \sigma^{\mu }  \\ 
-\overline{\sigma}^{\mu } & 0
\end{array}\right]
\,,
\end{align}
%
where 
%
\begin{align}
\sigma^{\mu } = \left[\begin{array}{cc}
\mathbb{1} & \sigma^{\mu }
\end{array}\right]
\qquad \text{and} \qquad
\overline{\sigma}^{\mu } = \left[\begin{array}{cc}
\mathbb{1} & - \sigma^{\mu }
\end{array}\right]
\,.
\end{align}
%

Now, then, we discard the left handed components: for \(m=0\) we write 
%
\begin{align}
i \overline{\sigma} \partial \psi_{L} =0\\
i \sigma \partial \psi_{R}=0
\,,
\end{align}
%
so we can discard the second equation. 
We can introduce a current involving only the left-handed component: 
%
\begin{align}
j^{\mu +}_{L} = \nu ^+_{L} \overline{\sigma}^{\mu } e_{L} + u ^+_{L} \overline{\sigma}^{\mu } d_{L} + \dots
\,.
\end{align}

Notice that we did not introduce here an index \(a\) as we had done for the color charge. 

This is a charged current.

Why are there no terms involving \(\sigma^{\mu }\)? We projected along the first two components, so in the terms 
%
\begin{align}
\overline{u} \gamma^{\mu } \qty( \frac{1-\gamma^{5}}{2})d = \overline{u} \gamma^{\mu } d_L
\,
\end{align}
%
the last two terms vanish, we do not need to write them.
This theory of weak interaction is called a \(V-A\) theory: in the current we have the terms 
%
\begin{align}
\frac{1}{2} \qty[ \underbrace{\overline{u} \gamma^{\mu } d}_{V} - \underbrace{\overline{u} \gamma^{\mu } \gamma^{5} d}_{A}]
\,,
\end{align}
%
where the term \(V\) is a vector current, while the term \(A\) is an axial current. 

We know that a density Lagrangian has a dimension of 4, so from the expression \(\mathscr{L} = \overline{\psi} m \psi \) we get that the fermion must have dimension \(3/2\). 
By dimension, we mean power of the mass.

When we have a term like \(j^{\mu } j_{\mu } \times \const\), the constant must have a dimension of \(-2\). 
The constant is called \(G_F\), and we have that the cross section goes like \(\sigma \sim G_F^{2} \sim [m]^{-4}\)? this does not work, we need something with dimension \(2\) multiplying it, since the cross section has dimension \(-2\). 

Our only free parameter is \(s\), the square of the center of mass energy. So, we will have 
%
\begin{align}
\sigma \sim G_F^2 s
\,,
\end{align}
%
which means that if we increase the beam energy the cross section increases. 
\(s\) can diverge in principle, but \(\sigma \) is connected to a probability: this means that we have a unitarity violation. 

This can be fixed with the introduction of a charged mediator \(W\).

How massive must it be? We can measure \(G_F \sim \SI{e-5}{GeV^{-2}}\). 

In the Breit-Wigner resonance formula we will need to insert 
%
\begin{align}
\frac{1}{q^2- M_W^2}
\,,
\end{align}
%
which avoids the divergence. 
In order for this to work, we need this boson to be of the order of \SI{10}{GeV} to \SI{100}{GeV}.

Now we need to describe this as a gauge theory. The symmetry group will be \(SU(2)_{L}\). We will have \(2^2-1 = \) three generators. 

This process is going to fail: the kinetic term of these vector bosons will look like 
%
\begin{align}
F^{\mu \nu, i} F_{\mu \nu }^{i}
\,,
\end{align}
%
which will never have a quadratic term \(M^2 W^{\mu } W_{\mu }\). 
So, the bosons will never be massive like we need.
This is called the Intermediate Vector Boson theory. 

Since our plain Yang-Mills theory does not have vector bosons, we can insert them manually. 
However, if we do we are explicitly breaking the gauge symmetry.

We lose renormalizability: renormalization means that we can absorb all the infinities into a finite number of parameters.

We have a dilemma: Yang-Mills theories are based on the power of symmetry, but they predict massless mediators. 

\end{document}
