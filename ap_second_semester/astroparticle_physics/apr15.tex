\documentclass[main.tex]{subfiles}
\begin{document}

\marginpar{Wednesday\\ 2020-4-15, \\ compiled \\ \today}

We would like to introduce a sort of ``Quantum Weak Dynamic'' theory, and we will see that this is possible in the standard model, by unifying the electromagnetic and weak interactions. 

\subsection{The current-current model for the weak interaction}

We have seen that fermions are described by a four-component spinor: 
%
\begin{subequations}
\begin{align}
\Psi = \left[\begin{array}{c}
\psi_{L} \\ 
\psi_{R}
\end{array}\right]
\,,
\end{align}
\end{subequations}
%
where \(\psi_{L, R}\) are both two-component vectors. The right handed part has helicity \(h = + 1/ 2\), the left handed part has helicity \(h = - 1 / 2\).

The puzzling part of the weak interaction is that it seems like the only part of the fermion entering \(\beta \)-decay is the left-handed part.
In the late seventies, the professor's master thesis stated that this was a mystery. Still, we do not know why this interaction breaks parity. 

We need to describe the experimental results mathematically: so, we need a projector onto the left-handed components. 
We already introduced the Dirac \(\gamma^{\mu }\) matrices. 
Now we introduce the \textbf{fifth} gamma matrix: in the chiral representation it reads
%
\begin{align}
\gamma^{5} = \left[\begin{array}{cc}
- \mathbb{1} & 0 \\ 
0 & \mathbb{1}
\end{array}\right]
\,,
\end{align}
%
which has the important property that it anticommutes with all the Dirac matrices: 
%
\begin{align}
\qty{\gamma^{5}, \gamma^{\mu }} = 0 
\,,
\end{align}
%
and actually we can write it as 
%
\begin{align}
\gamma^{5} = \frac{i}{4!} \epsilon_{\mu \nu \rho \sigma } \gamma^{\mu } \gamma^{\nu } \gamma^{\rho } \gamma^{\sigma } =  i \gamma^{0}\gamma^{1}\gamma^{2}\gamma^{3}
\,.
\end{align}

By its form we can readily see that \(\psi_{L}\) is an eigenstate of \(\gamma^{5}\) with eigenvalue \(-1\), while \(\psi_{R}\) has eigenvalue \(+1\). 
So, the matrix 
%
\begin{align}
\frac{\mathbb{1} \pm \gamma^{5}}{2} 
\,
\end{align}
%
projects a state onto the left- or right-handed subspace (minus for the left handed one).
If we apply a parity transformation \(P \vec{x} \to - \vec{x}\) to the current \(\overline{\psi} \gamma^{5} \psi \) we get \(- \overline{\psi} \gamma^{5} \psi \). 

The Dirac matrices can be written as 
%
\begin{subequations}
\begin{align}
\gamma^{\mu } \left[\begin{array}{cc}
0 & \sigma^{\mu }  \\ 
-\overline{\sigma}^{\mu } & 0
\end{array}\right]
\,,
\end{align}
\end{subequations}
%
where 
%
\begin{align}
\sigma^{\mu } = \left[\begin{array}{cc}
\mathbb{1} & \sigma^{\mu }
\end{array}\right]
\qquad \text{and} \qquad
\overline{\sigma}^{\mu } = \left[\begin{array}{cc}
\mathbb{1} & - \sigma^{\mu }
\end{array}\right]
\,.
\end{align}
%

We have already seen that if we set \(m=0\) the Dirac equation for the two left- and right-handed spinors decouples:
%
\begin{subequations}
\begin{align}
i \overline{\sigma} \cdot \partial \psi_{L} =0\\
i \sigma \cdot \partial \psi_{R}=0
\,.
\end{align}
\end{subequations}

We know that neutrons are made up of \(ddu\) quarks, while protons are \(uud\), so we must have a down quark turning into an up:
the Feynman diagram for \(\beta^{-}\) decay (from far away) looks like what is shown in figure \ref{fig:beta-decay}.

\begin{figure}[ht]
\centering
\feynmandiagram[layered layout, horizontal = x to c]{
x [particle = {\(d_L\)}] -- [fermion] c [blob] -- [fermion] u [particle={\(u_L\)}],
c -- [fermion] e [particle = {\(e^{-}_{L}\)}],
c -- [anti fermion] nu [particle = {\(\overline{\nu}_{L}\)}]
};
\caption{Beta decay: an unknown weak process.}
\label{fig:beta-decay}
\end{figure}

Only the left-handed spinors appear in the process. In order to describe this, we can introduce a current involving only the left-handed components: 
%
\begin{align}
j^{\mu +}_{L} &= \nu ^+_{L} \overline{\sigma}^{\mu } e_{L} + u ^+_{L} \overline{\sigma}^{\mu } d_{L} + \dots \\
j^{\mu -}_{L} &= e ^+_{L} \overline{\sigma}^{\mu } \nu_{L} + d ^+_{L} \overline{\sigma}^{\mu } u_{L} + \dots
\,.
\end{align}

This current will appear in the Feynman amplitude in this fashion: 
%
\begin{align}
\mathcal{M} \sim \frac{4 G_F}{\sqrt{2}} j^{\mu +}_{L} j^{\mu -}_{L} 
\,.
\end{align}

These terms can be written without specifying \(L\) or \(R\), as:
%
\begin{align}
\overline{u} \gamma^{\mu } \qty( \frac{1-\gamma^{5}}{2})d 
= \frac{1}{2} \qty(\underbrace{\overline{u}\gamma^{\mu } d}_{V} - \underbrace{\overline{u}\gamma^{\mu } \gamma^{5} d}_{A} )
= \overline{u}_L \gamma^{\mu } d_L
\,.
\end{align}

Notice that we did not introduce here an index \(a\) as we had done for the color charge. 

The interaction boson mediating the weak interaction is called a \textbf{charged current}, since it must have an electric charge.

This theory of weak interaction is called a \(V-A\) theory: in the current we have distinguished the terms \(V\) --- a vector current --- and  \(A\), an axial current. This distinction is a group representation one: the vector term transforms like a vector, while the axial term transforms like a pseudovector.

We know that a density Lagrangian has a dimension of 4, so from the expression \(\mathscr{L} = \overline{\psi} m \psi \) we get that the fermion must have dimension \(3/2\).\footnote{By ``dimension'' we mean mass dimension: the dimension of a mass \(m\) is 1.} 
So, the current \(j^{\mu } \sim \overline{\psi} \gamma^{\mu } \psi \) has a dimension of 3.
In the Lagrangian the current term will look like \(G_F j^{\mu } j_{\mu }\), so the constant \(G_F\) must have a dimension of \(-2\). 

Experimentally, we find \(G_F \sim \SI{e-5}{GeV^{-2}}\). 

This kind of \(V-A\) theory has been successful in the description of processes such as: 
%
\begin{align}
n &\to p e^{-} \overline{\nu}_{e}  \\
\mu &\to \nu_{\mu }  e^{-} \overline{\nu}_{e}  \\
\pi^{-} &\to \mu^- \overline{\nu}_{\mu } \text{ or } e^{-} \overline{\nu}_{e}
\,.
\end{align}

% The constant is called \(G_F\), and we have that the cross section goes like \(\sigma \sim G_F^{2} \sim [m]^{-4}\)? this does not work, we need something with dimension \(2\) multiplying it, since the cross section has dimension \(-2\). 
We can estimate the order of magnitude of the cross section of weak interaction processes without drawing any Feynman diagrams: we know that they must depend on \(G_F^2\), which has dimension \(-4\), while the cross section has dimension \(-2\): what can we multiply it by to fix the dimensionality?

Our only free parameter is \(s = (p_1 + p_2 )^2\), the square of the center of mass energy. So, we will have 
%
\begin{align}
\sigma \sim G_F^2 s
\,,
\end{align}
%
which means that if we increase the beam energy the cross section increases. 

This is a problem: \(s\) can diverge as we raise the energy of our colliders, but \(\sigma \) is connected to a probability: this means that we have a unitarity violation. 

The problem becomes worse if we consider \(n\) loops: then we will have \(\sigma \sim G_F^{2n} s^{m}\), where by dimensional analysis we find \(m = 2n-1\): this increases with \(n\)!

This can be fixed with the introduction of a charged mediator \(W\) instead of a four-current vertex with a coupling \(G_F\).
This new boson will have a vertex with both of the currents. 
This is called \textbf{Intermediate Vector Boson} theory. 

The mass of this boson is denoted as \(M_W\); then in the amplitude we will need to insert a factor 
%
\begin{align}
\frac{1}{q^2- M_W^2}
\,,
\end{align}
%
which avoids the divergence. Here \(q^2\) is the square of the four-momentum of the carrier: in simple processes it will just be given by \(q^2 = s\). 
Then, we go from \(\sigma \sim G_F^2s\) to 
%
\begin{align}
\sigma \sim \qty( \frac{1}{q^2 - M_W^2})^2 s 
\,,
\end{align}
%
which scales as \(\sigma \sim 1 / s\) when \(s\) diverges.

A massless mediator would not work, even though it would fix the ultraviolet problems of our theory, because of the observed behaviour at low energies.
The interaction we see does not have an infinite range, like the photon; instead it has a very short range. 
Because of the measured value of \(G_F\), we expect this boson's mass \(M_W\) to be of the order of \SI{100}{GeV}, since at low energies the Breit-Wigner factor reduces to 
%
\begin{align}
\frac{g_w^2}{q^2 - M_W^2} \sim \frac{g_w^2}{M_W^2} \sim G_F
\implies M_W \sim \frac{g_w}{\sqrt{G_F}}
\,,
\end{align}
%
and we expect the coupling \(g_w\) to be of order one. 
We see no \(q^2\) dependence in \(G_F\) at low energies \(\lesssim \SI{30}{GeV}\), which supports this hypothesis.

Now we need to describe this as a gauge theory. 
The symmetry group will be \(SU(2)_{L}\) and we will have \(2^2-1 = \) three generators.
The photon is described by \(U(1)\) symmetry, and it has no mass term \(\sim m^2 A^{\mu } A_{\mu }\). 
If we were to insert such a term we would lose gauge symmetry. 
We must then solve the problem of implementing a gauge theory description of a massive boson.

Let us consider the problem kinematically: if a vector boson is massive we can go to its rest frame, so the polarization vector has \emph{three} degrees of freedom: so, this new boson will need to have a new degree of freedom.

If we break gauge freedom, there will be no way anymore to describe our vector boson by imposing it with the Gupta-Bleuer condition: so, we will get states of negative probability. 

Also, gauge invariance ensures the \textbf{renormalizability} of the theory: the possibility to be able to change the parameters off the theory with the energy so that we do not get diverging contributions from loop diagrams. 

% This process is going to fail: the kinetic term of these vector bosons will look like 
% %
% \begin{align}
% F^{\mu \nu, i} F_{\mu \nu }^{i}
% \,,
% \end{align}
% %
% which will never have a quadratic term \(M^2 W^{\mu } W_{\mu }\). 
% So, the bosons will never be massive like we need.
% This is called the Intermediate Vector Boson theory. 

% Since our plain Yang-Mills theory does not have massive vector bosons, we can insert them manually. 
% However, if we do we are explicitly breaking the gauge symmetry.

% We lose renormalizability: renormalization means that we can absorb all the infinities into a finite number of parameters.

% We have a dilemma: Yang-Mills theories are based on the power of symmetry, but they predict massless mediators. 

\end{document}
