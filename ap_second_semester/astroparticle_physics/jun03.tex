\documentclass[main.tex]{subfiles}
\begin{document}

\section{Baryogenesis}

\subsection{Electroweak baryogenesis}

\marginpar{Wednesday\\ 2020-6-3, \\ compiled \\ \today}

The classical Lagrangian is invariant under \(q \to e^{i \alpha } q\), which gives us the classical \(B\)-number conservation; and it is also invariant under \(\ell_i \to e^{i \beta_{i}} \ell_i\), where \(i\) labels the leptons' generations. 

This gives us conservation of the three lepton numbers. 
This holds as long as the neutrinos are massless, if they are not we get a single symmetry \(U(1)_L\), where \(L = L_e + L_\mu + L_c\). 

However, at the quantum level the currents associated with these symmetries are anomalous: we have 
%
\begin{align}
\partial_{\mu } j^{\mu B} = 3 \frac{g_s^2}{32 \pi^2} W^{\mu \nu , a} \widetilde{W}_{\mu \nu }^{a} 
\qquad \text{and} \qquad
\partial_{\mu } j^{\mu L} = \frac{g_s^2}{32 \pi^2} W^{\mu \nu , a} \widetilde{W}_{\mu \nu }^{a} 
\,,
\end{align}
%
where 
%
\begin{align}
W^{a}_{\mu \nu } = 2 \partial_{[\mu } W^{a}_{\nu ]} + g \epsilon^{abc} W^{b}_{\mu } W^{c}_{\nu }
\,
\end{align}
%
is the field strength tensor of the \(SU(2)_L\) field and \(\widetilde{W}\) is the Hodge dual: 
%
\begin{align}
\widetilde{W}^{a}_{\mu \nu } = \frac{1}{2} \epsilon_{\mu \nu \rho \sigma} W^{\rho \sigma , a} 
\,.
\end{align}

This is a result which holds for any axial current \(j^{\mu }_{A} \sim \overline{\psi} \gamma^{\mu } \gamma^{5} \psi \): we can write its divergence as proportional to the norm of the field strength. 

Now, for the strong interaction gluons interact equally with left-handed and right-handed charges. 
On the contrary, for the weak interaction only the left-handed charges interact with the bosons, so neither \(B\) nor \(L\) are conserved. 

Geometrically, this is connected to the fact that 
%
\begin{align}
\int W \widetilde{W} \dd[4]{x} = \int \dd{S_{\mu }} K^{\mu } \neq 0
\,,
\end{align}
%
where \(W_{\mu \nu } \widetilde{W}^{\mu \nu } = \partial_{\mu } k^{\mu }\).

For an abelian symmetry, the integral always vanishes, while for a nonabelian one it can be nonzero: there is a ``winding number'' 
%
\begin{align}
\nu = - \frac{i g_2^3}{24 \pi^2} \int \dd{S_\mu } \epsilon_{\mu \nu \rho \sigma } \Tr[A_{\nu }A_{\rho }A_{\sigma }]
\,.
\end{align}

We have topologically distinct vacua: \emph{sphalerons}. \emph{Instantons} (nonpertubative ``tunneling'' solutions) can go between them. 

Even in a process with \(\Delta \nu \neq 0\), though, we have \(\Delta L = \Delta B\), so \(L-B\) is always conserved. 

The height of the potential barrier is of the order \(E _{\text{sph}} \sim M_W / g_2^2\). 

At zero temperature, transmission is only possible through tunneling, in which case we have \(\Gamma \sim e^{-4 \pi / \alpha_{w}}\), so \(\Gamma \sim \num{e-165}\): practically zero. 

However, at nonzero temperature we have 
%
\begin{align}
\Gamma _{\text{sph}} \sim T^{4} \exp( - \frac{M_W(T)}{\alpha_{w} T})
\,,
\end{align}
%
but if \(T > v\) we go into the unbroken electroweak regime, when \(M_W(T ) = 0\). So, at high energies \(\Gamma _{\text{sph}} \sim \alpha_{w} T^{4}\). 

\todo[inline]{Where did that \(\alpha_{w}\) come from? }

The electroweak \(B\)-breaking processes are in thermal equilibrium if the rate per particle \(\Gamma _{\text{sph}} / n \sim \Gamma _{\text{sph}} / T^3 \sim \alpha_w^{4} T\) is larger than \(H(T) = T^2 / M_P^{*}\). 


So, sphalerons are in equilibrium as long as 
%
\begin{align}
M_W \sim \SI{100}{GeV} < T < \SI{e12}{GeV} \sim \alpha_w^{4}M_P^{*}
\,.
\end{align}

In this regime, then, \(B-L\) is conserved but \(B\) and \(L\) are not separately: could we get baryogenesis here? 

There are issues with two of the Sakharov conditions: \(CP\) violation, and out-of-equilibrium processes. 
\(CP\) violation in the standard model is due to the \(V_{CKM}\) mixing matrix. 
If we only had two fermion generations this matrix would only have one free parameter and no phase: therefore, we would have no \(CP\) violation. Three generators is the minimal number. 

The degree of \(CP\) violation in the standard model is quantified by the Jarlskog invariant \(J\). This is nonzero iff the matrix \(V_{CKM}\) cannot be brought to a real form, in \(O(3)\). 
All \(CP\) violations in the SM are proportional to \(J\), and experimentally we know that \(J \approx \num{3e-5}\). 

The baryon asymmetry predicted by the Standard model is of the order 
%
\begin{align}
\delta_{CP} \approx \num{e-20}
\,,
\end{align}
%
so it is too small to account for the observed \(n_B / n_\gamma \sim
\num{e-10}\). 

If there are physics beyond the SM which will contribute to this, they are likely to show up in the electric dipole of the electron and the neutron.

As for the out-of-equilibrium condition: the ELW phase transition is not first order in the SM; and the timescale of the weak interactions is much shorter than the age of the universe at that point. 

\subsubsection{Leptogenesis}

With this mechanism, we have baryogenesis through leptogenesis, since \(B - L\) is conserved. 
There is a connection between the smallness of \(m_\nu \) and \(B\) asymmetry. 

The see-saw mechanism gives neutrinos a small mass by introducing a \(\nu_{R}\) with a large Majorana mass, and the Majorana mass term has \(\Delta L \neq 0\). 
Sphaleron processes can exchange \(B\) for \(L\), so we can get some baryons. 

For the right-handed neutrino, we have two options: one option is that we do not introduce a \(\nu_{R} \) and only write a Majorana mass term like  
%
\begin{align}
h \nu_{L}^{i} \Delta^{ij} \nu_{L}^{j}
\,,
\end{align}
%
where \(\Delta^{ij}\) is a \(SU(2)_L\) triplet, and \(ij\) are \(SU(2)_L\) indices. 
This must have \(\expval{\Delta } = v' \ll v \approx \SI{100}{GeV}\). 

The other option is that we do introduce \(\nu_{R}\), and write a term like 
%
\begin{align}
h \overline{\nu}_{L} \phi \nu_{R}
\,.
\end{align}

Then, the neutrino would be a Dirac particle, and we would need to have \(h \ll 1\).

If we impose the global symmetry \(U(1) _{\text{Left}}\), then we can not have a term like \(M \nu_{R} \nu_{R}\). 
If we do not impose it, we can have the term, with any value for \(M\).
This mass might correspond to some new symmetry, under which \(\nu_{R}\) is not a singlet anymore. 

Then, we can add new terms to the Lagrangian, such as 
%
\begin{align}
\Delta \mathscr{L}_{R} = M_{\alpha } \overline{N}^{c}_{\alpha R} N_{\alpha R} + y_{\alpha \beta } L_{\alpha, L} \phi L_{\beta , R}
\,,
\end{align}
%
where \(N_R = \nu_{R}\) is the simplest option. 
Here \(\alpha \) and \(\beta \) are generation indices. 

\todo[inline]{Some unintelligible computations follow, about the interference of tree level and one-loop processes.}

The experimentally-determined value for \(m_\nu \) from neutrino oscillations is compatible with leptogenesis. 

\subsection{An interesting coincidence}

The baryon and dark matter content of the universe are less than an order of magnitude apart; however the mechanisms to generate them are different, since there are no Sakharov conditions for DM. 

Is there a correlation? Maybe we have ``asymmetric Dark Matter''. 

The dark energy domination phase is quite recent: we had \(\Omega_{M} \sim \Omega_{DE}\) only at \(z \sim \num{.55}\). 

% \section{Summary}

% The two standard models work well together for nucleosynthesis, less so for dark matter, dark energy, baryon asymmetry, primordial inflation. 

% Also, the Glashow-Weinberg-Salam SM cannot account for the neutrino masses. 
 
\end{document}
