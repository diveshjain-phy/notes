\documentclass[main.tex]{subfiles}
\begin{document}

\marginpar{Wednesday\\ 2020-4-22, \\ compiled \\ \today}

% Last time we considered the spontaneous breaking of a \emph{global} symmetry: we saw the appearance, corresponding to the breaking of the \(U(1)\) symmetry, of massless Goldstone bosons.
% Each broken generator has a corresponding Goldstone boson.

% We had a quadratic term in the Lagrangian, and a quartic term. 

What would \(\mu^2<0\) physically mean? It would be a tachyonic particle.

The field \(\sigma (x)\) is called the \textbf{Higgs field}; the would-be Goldstone boson disappears yielding the longitudinal polarization of \(A_{\mu } (x)\). 
% The Vacuum Expectation Value goes from 0 to \(v\): in this case we have broken the \(U(1)\) symmetry.

% The imaginary part of the field \(\phi \) corresponds to a massless scalar field.

% Every time we have a certain global symmetry described by a group \(G\) with generators \(t^{a}\), which is broken to a subgroup \(G'\) (which can also be just the identity) with generators \(t^{i}\), we can identify the broken generators with Goldstone bosons.

% We have the Dirac equation \(\qty(i \slashed{\partial} - m) \psi =0\), where \(\psi \) has a global \(U(1)\) symmetry. 
% This can\emph{not} be generalized to a local \(U(1)\) symmetry, unless we introduce a compensating gauge field.

% Let us consider the Lagrangian from yesterday for the scalar field with a kinetic, quadratic and quartic term. Can we do \(\partial_{\mu } \to \DD_{\mu }\) as we did with QED, to generalize the \(U(1)\) symmetry of \(\phi \)?

% We can do it and encounter no issues. 

% What if \(\mu^2<0\)? The result is surprising: we do as before, perturbing around the vacuum \(\phi = v\), writing \(\phi = v + \sigma (x) + i \eta (x)\). 
% This is around pages 92--93 of the notes.

% We start from \(\phi \), which has 2 dof since it is a complex scalar, and \(A_{\mu }\), which also has 2 dof since it is a massless vector. 

% We would expect to get the fields \(\sigma \) and \(\eta \): however, the field \(\eta \) does not appear, it is ``eaten up'' by the vector field \(A_{\mu }\).
% We only find a massive real scalar \(\sigma (x)\), which has 1 dof, and a massive vector boson \(A_{\mu }\), which has 3 dof.

% This is the \emph{transmutation}, the scalar degree of freedom is absorbed to a degree of freedom in the vector field.

We are curing two different problems: we have found massive gauge bosons, and removed the unphysical Goldstone bosons.

\subsubsection{A more general formulation: \(SU(2)\) symmetry breaking}

This section follows Peskin pretty closely \cite[sec.\ 16.2]{peskinConceptsElementaryParticle2019}.

We can repeat this in the general case, with the group \(G\) being broken to \(G'\). Suppose, for clarity, that this is \(SU(2)\) being broken to \(U(1)\).

The three vectors \(A^{a}_{\mu }\) generate rotations around the axes \(x^{a}\) respectively, as \(a = 1, 2, 3\). 

The adjoint representation of \(SU(2)\) is given by three real scalar fields \(\phi^{a}\); their covariant derivative is given by 
%
\begin{align}
\DD_{\mu } \phi^{a} 
= \partial_{\mu } \phi^{a}
+ g \epsilon^{abc} A^{b}_{\mu } \phi^{c}
\,,
\end{align}
%
since the representation matrices in the adjoint representation are \((t^{b}_{G})_{ac} = i f^{abc}\). 

We want to choose a potential \(V(\phi )\) which is minimized by a configuration with \(\expval{\abs{\phi^{a}}} = v\). 
A vacuum for this potential is, for example, given by \(\phi^{a }= v \delta^{a3}\): it retains some rotational invariance (around the \(\hat{3}\) axis), but invariance for rotations around the other two axis is broken. 

A perturbative expansion around such a vacuum will then be given by 
%
\begin{align}
\phi (x) = \qty(\pi^{1} (x), \pi^{2}(x), v + h(x))
\,,
\end{align}
%
where \(\pi^{1, 2}\) are the would-be goldstone bosons, which will contribute to the longitudinal components of \(A^{1,2}_{\mu }\), the bosons which become massive. 

On the other hand, \(A^{3}_{\mu } \) remains massless. 
As we go to the unitary gauge we can eliminate \(\pi^{1, 2}\): so we get 
%
\begin{align}
\phi (x) = (0, 0, v + h(x))
\,.
\end{align}

We know that 
%
\begin{align}
\DD_{\mu } \phi^{a} 
&= \partial_{\mu } h \delta^{3a} 
+ g \epsilon^{abc} A^{b}_{\mu } \phi^{c}  \\
&= \partial_{\mu } h \delta^{3a}
+ g \epsilon^{ab3} A^{b}_{\mu } (v + h(x))
\,.
\end{align}

The kinetic term of \(\phi^{a}\) will then become: 
%
\begin{align}
\frac{1}{2} \qty(\DD_{\mu } \phi^{a})^2
&= \frac{1}{2} \DD_{\mu } \left[\begin{array}{ccc}
0 & 0 & v+h
\end{array}\right]
\DD^{\mu } \left[\begin{array}{c}
0 \\ 
0 \\ 
v+h
\end{array}\right]  \\
&= \frac{1}{2} \qty[
    g \epsilon^{ab3} A^{b}_{\mu } (v+h) 
    + \partial_{\mu } h \delta^{a3}
]^2  \marginnote{Square entails contraction of both \(a\) and \(\mu \).}\\
&= \frac{g^2 v^2}{2} \epsilon^{ab3} A^{b}_{\mu } \epsilon^{ad3} A^{d, \mu } + \mathcal{O}(h) \\
&= \frac{g^2 v^2}{2} \qty(A^{1}_{\mu } A^{\mu 1} + A^{2}_{\mu } A^{\mu 2 } ) + \mathcal{O}(h)
\,.
\end{align}

So, \(A^{1, 2}\) became massive, with mass \(M_W^2 = g^2 v^2\), while \(A^{3}\) remained massless.

As \(SU(2)\) is broken to \(U(1)\), the 3 generators \(A_{\mu }^{a}\), with \(a = 1, 2, 3\) are broken to give two massive vector bosons (\(W^{\pm}\)) and one massless vector boson (the photon), while the real field \(\sigma \) is the Higgs field.

The two \(W\) bosons are not actually the components \(A^{1, 2}\): instead, we choose them to be the eigenvalues of rotations around the \(z\) axis: 
%
\begin{align}
W^{\pm}_{\mu } = \frac{1}{\sqrt{2}} \qty(A^{1}_{\mu } \mp i A^{2}_{\mu }) 
\,,
\end{align}
%
since the rotation matrix around the \(z\) axis in the spin-1 representation is given by 
%
\begin{align}
J^{3} = \left[\begin{array}{ccc}
0 & -i & 0 \\ 
i & 0 & 0 \\ 
0 & 0 & 0
\end{array}\right]
\,,
\end{align}
%
whose eigenvectors are indeed \((1, \mp i, 0)\) with eigenvalues \(\pm 1\). 

This is called the \textbf{Georgi-Glashow model}. It predicts two massive bosons of the \(V-A\) interaction. 
However, it does not work well phenomenologically. 

\subsubsection{Hypercharge: why the GG model does not work}

This was a computationally easy example to clarify what this mechanism looks like, not the one we actually will use: we will have to choose another symmetry group.

The issue lies with the fact that we are associating electric charge with the generator of rotations about \(\hat{3}\).
We know that an electron current can be turned into an electronic neutrino current, and that the neutrino is electrically neutral.
They must belong to the same isospin multiplet in order for this to happen.
Isospin (\(I\)) is what we call rotation in the three-dimensional space the \(SU(2)\) symmetry was initially defined in. 
Rotation about the \(\hat{3}\) axis then corresponds to the operator \(I_3\); the algebra of these operators is that of a rotation representation, \([I_i, I_j] = i \epsilon_{ijk} I_k\).
When we say ``isospin multiplet'' we mean that the value of \(I^2 = j (j+1)\) is fixed; then the value of \(I_3\) will go from \(-j\) to \(j\).

We cannot have a doublet with only \(\nu \) and \(e^{-}\) because of the fact that the neutrino is neutral: the eigenvalues of the rotation about the \(3\) axis would have to be \(0\) and \(1\).
We need a particle with eigenvalue \(-1\) with respect to rotations about the \(\hat{3}\) axis, so with positive charge.

We then predict the existence of a heavy electron \(E^{+}\), such that the triplet looks like 
%
\begin{align}
\left[\begin{array}{c}
E^{+} \\ 
\nu  \\ 
e^{-}
\end{array}\right]
\,,
\end{align}
%
where the eigenvalues of this rotation around the \(\hat{3}\) axis are \(-1, 0, 1 \) respectively.
This heavy electron was not observed; the experimental bound is at \(M_{E^{+}}\leq \SI{400}{GeV}\) currently. 
We would expect to then see processes like \(u + E^{+} \to d + \nu \), but we do not.

\todo[inline]{Is the positron \(e^{+}\) not a candidate for this? Why do we say that this particle would need to be ``heavy''? Or are we already implicitly considering antiparticles maybe?}

This will be a way to unify electromagnetic and weak interactions, and we will get to use a single coupling constant for our new electroweak theory.

\subsection{Electroweak symmetry breaking}

The correct symmetry group for the electroweak theory, as Glashow showed in 1961, is 
%
\begin{align}
SU(2)_{L} \otimes U(1) _{\text{hypercharge}}
\,,
\end{align}
%
where the hypercharge is usually denoted as \(Y\).
We have three vector fields for \(SU(2)\), which we call \(A_{\mu}^{i}\), and also a vector field for \(U(1)\): \(B_{\mu }\).
Note that these do not have a direct correspondence with the fields we get after SSB: hypercharge is different from electromagnetic charge.

The pair neutrino-electron is a \(I = 1/2\) doublet under \(SU(2) \times U(1)\), we will see that one component of this doublet has charge \(+\) while the other has charge 0: its transformation will look like 
%
\begin{align}
\left[\begin{array}{c}
\nu_{L} \\ 
e^{-}_{L}
\end{array}\right]
\to 
\underbrace{e^{-i \vec{\alpha} \cdot \vec{\sigma} / 2}}_{SU(2)}
\underbrace{e^{-i \beta  /2 }}_{U(1)}
\left[\begin{array}{c}
\nu_{L} \\ 
e^{-}_{L}
\end{array}\right]
\,.
\end{align}

The bosons in this theory start off as massless, and therefore the fields 
%
\begin{align}
W^{\pm}_{\mu } = \frac{1}{\sqrt{2}}  \qty(A^{1}_{\mu } \mp i A^{2}_{\mu })
\,
\end{align}
%
are massless as well. However, we want massive \(W\) bosons with \(M_W \sim \SI{100}{Gev}\)! 
At the end of the sixties Weinberg and Salam introduced the Higgs mechanism in this theory. 
The idea is to interpret the scalar field as a \(I= 1/2\) representation of rotation: it will transform under the symmetries of the theory as 
%
\begin{align}
\varphi =
\left[\begin{array}{c}
\varphi^{+} \\ 
\varphi^{0}
\end{array}\right]
\to 
e^{i \vec{\alpha} \cdot \vec{\sigma} /2}
e^{i \beta / 2}
\left[\begin{array}{c}
\varphi^{+} \\ 
\varphi^{0}
\end{array}\right]
\,.
\end{align}

So, the lepton doublet has a charge \(-1/2\) with respect to the \(U(1)\) symmetry, while the field has a symmetry \(+1/2\). 

Suppose the potential of the field reads: 
%
\begin{align}
V(\phi ) = - \mu^2 \abs{\phi }^2 + \lambda \abs{\phi }^{4}
\,,
\end{align}
%
whose minimum is defined by 
%
\begin{align}
0= - 2 \mu^2 \phi + 4 \lambda \phi \abs{\phi }^2
\,,
\end{align}
%
therefore the minimum is a sphere (since the two complex fields correspond to four real degrees of freedom) with \(\abs{\phi }^2 =  \abs{\phi^{+}}^2 + \abs{\phi^{0}}^2 = \mu^2 / 2 \lambda \); so we define \(v = \mu / \sqrt{\lambda }\).

Up to a rotation, the VEV of our field will be 
%
\begin{align}
\expval{\phi }_{0} = \frac{1}{\sqrt{2}}
\left[\begin{array}{c}
0 \\ 
v
\end{array}\right]
\,.
\end{align}

This vacuum breaks the \(SU(2)_L \times U(1)_{Y}\) symmetry. 
We can expand around it in the following manner: 
%
\begin{align}
\phi (x) 
= 
\left[\begin{array}{c}
\pi^{+}(x) \\ 
\frac{v + h(x) + i \pi^{3}(x)}{\sqrt{2}}
\end{array}\right]
\,,
\end{align}
%
where \(\pi^{+} = (\pi^{1} + i\pi^{2}) / \sqrt{2}\) is a complex perturbation encompassing two real degrees of freedom. 

As we did before, we can gauge away the unphysical Goldstone bosons \(\pi^{1, 2, 3}\); the real field \(h(x)\) remains. 

How do the bosons of this theory \textbf{couple to fermions}? The couplings can only have specific forms, fixed by gauge invariance: the covariant derivative will read: 
%
\begin{align}
\DD_{\mu } \Psi  = 
\qty(\partial_{\mu } 
- i g A^{a}_{\mu } I^{a} 
- i g' B_{\mu } Y
)\Psi 
\,,
\end{align}
%
where \(g\) and \(g'\) are the coupling of \(SU(2)\) weak isospin and \(U(1)\) hypercharge respectively; \(I^{a}\) are the generators of \(SU(2)\) in the \(SU(2)\) representation acting on \(\Psi \), and \(Y\) is the hypercharge, a scalar. 

In principle, \(g\) and \(g'\) are independent, so we keep them distinct. Let us now apply \(\DD_{\mu } \) to the field \(\phi \) around the vacuum with only the \(h \) perturbation, as 
%
\begin{align}
\DD_{\mu } \phi = \qty(\partial_{\mu } 
- i g A^{a}_{\mu } I^{a} 
- i g' B_{\mu } Y
) \left[\begin{array}{c}
0 \\ 
\frac{v + h(x)}{\sqrt{2}}
\end{array}\right]
\,,
\end{align}
%
of which we take the square norm, up to zeroth order in \(h\): 
%
\begin{align}
\abs{\DD_{\mu } \phi }^2
&= \frac{1}{2} \left[\begin{array}{cc}
0 & v
\end{array}\right]
\qty(g A^{a}_{\mu } \frac{\sigma^{a}}{2} + g' B_{\mu } \frac{1}{2})
\qty(g A^{b, \mu } \frac{\sigma^{b}}{2} + g' B^{\mu } \frac{1}{2})
\left[\begin{array}{c}
0 \\ 
v
\end{array}\right] \qty(+ \mathcal{O}(h)) \\
&= \frac{1}{2} \underbrace{ g^2v^2  \frac{1}{4} 
\qty(A^{1}_{\mu } A^{1, \mu } + A^{2}_{\mu } A^{2, \mu })}_{\underbrace{(gv/2)^2}_{M_W^2} W^{+}_{\mu } W^{\mu, -}}
+ \frac{1}{2} \frac{v^2}{4} 
\qty(- g A^{3}_{\mu } + g' B_{\mu })^2
\,.
\end{align}

So, we have found two massive \(W^{\pm}\) bosons, and the \textbf{combination} \(- g A^{3}_{\mu } + g' B_{\mu }\) has also gained mass. 

Let us define \(\theta_{w}\), the Weinberg angle, by \(\tan \theta_{w} = g ' / g\). 
Then we will have 
%
\begin{align}
\cos \theta_{w} = \frac{g}{\sqrt{g^2 + g^{\prime 2}}}
\qquad \text{and} \qquad
\sin \theta_{w} = \frac{g'}{\sqrt{g^2 + g^{\prime 2}}}
\,,
\end{align}
%
and we define the electromagnetic field \(A_{\mu }\) and the \(Z_{\mu }\) boson by the two orthogonal combinations: 
%
\begin{align}
A_{\mu } &= \sin \theta_{w} A^{3}_{\mu } + \cos \theta_{w} B_\mu \\
Z_{\mu } &= \cos \theta_{w} A^{3}_{\mu } - \sin \theta_{w} B_\mu  
\,,
\end{align}
%
so that the kinetic term reads 
%
\begin{align}
\frac{1}{2} \frac{v^2}{4} \qty(- g A^{3}_{\mu } + g' B_{\mu })^2 
&= \frac{m_Z^2}{2} Z^{\mu }Z_{\mu } 
\,,
\end{align}
%
where \(m_Z^2 = (g^2 + g^{\prime 2}) v^2/ 4 \), while there is no mass term for \(A\): so, \(m_A = 0\).

The residual gauge symmetry can be identified with \(U(1)_{\text{em}}\).
When the symmetry is broken, we are left with a long-range interaction and a short range one.

% Do note that \(U(1)\)-hypercharge \emph{is} broken.
% However, a certain combination of the generators gives us the \(U(1)\) electromagnetic.
The combination under which the vacuum is invariant is 
%
\begin{align}
T^{3} + Y
\,,
\end{align}
%
where \(T^{3}\) is the third generator of \(SU(2)\), while \(Y\) is the generator of hypercharge.
This corresponds to 
%
\begin{align}
\phi \to e^{i \vec{\alpha} \cdot \vec{\sigma} / 2} e^{i \beta /2} \phi 
\,,
\end{align}
%
with \(\alpha^{3} = \beta \): this is due to the fact that 
%
\begin{align}
\frac{1}{2} \qty(\sigma^{3} + \mathbb{1}) =  \left[\begin{array}{cc}
1 & 0 \\ 
0 & 0
\end{array}\right]
\,,
\end{align}
%
whose exponential leaves the second component of \(\phi = (0, v/\sqrt{2})\) invariant.

In summary, the SSB has the form 
%
\begin{align}
SU(2)_L \times U(1)_{Y} \to U(1)_{\text{em}}
\,,
\end{align}
%
and it yields the two \(W^{\pm}\) bosons, each with mass \(gv/2\), and the \(Z\) boson with mass \(\sqrt{g^2+g^{\prime 2}} v/2\). 
The masses of the \(W\) and \(Z\) bosons are related by: 
%
\begin{align}
m_W = m_Z \cos \theta_{w}
\,,
\end{align}
%
so if we can devise some experiment which will measure \(\theta_{w}\) we can predict \(m_W / m_Z\). 

The full expression of the covariant derivative reads, using the notation \(c_w = \cos \theta_{w}\), \(s_w = \sin \theta_{w}\) and \(\sigma^{\pm} = \qty(\sigma^{1} \mp i \sigma^{2}) / \sqrt{2}\):
\todo[inline]{not sure about the last one} 
%
\begin{align}
\DD_{\mu } \Psi &= \qty[
    \partial_{\mu } 
    -i \frac{g}{\sqrt{2}} \qty(W^{+}_{\mu } \sigma^{+}+ W^{-}\sigma^{-})
    - ig \qty(c_{w} Z_\mu + s_{w}A_{\mu }) I^{3}
    - ig \qty(- s_{w} Z_\mu + c_{w} A_{\mu }Y)
] \Psi  \\
&= \qty[\partial_{\mu } 
    - i \frac{g}{\sqrt{2}} \qty(W^{+}_{\mu } \sigma^{+}+ W^{-}\sigma^{-})
    - i e A_{\mu }Q 
    - i \frac{g}{c_w} Z_\mu Q_Z
] \Psi 
\,,
\end{align}
%
where we defined \(e = g s_w = g' c_w \), and 
%
\begin{align}
Q = I^{3} + Y 
\qquad \text{and} \qquad
Q_Z = I^{3} - s_w^2 Q
\,.
\end{align}

\todo[inline]{The next bit is not super clear.}

If the mass of the fermion described by \(\Psi \) is zero we can decouple the left- and right-handed parts of the spinor. 
We will get interactions between the left handed components, not between the right-handed components, since the \(W^{\pm}\) do not couple with them.
\todo[inline]{How does that come about exactly?}

We recover the \(V-A\) structure: the left- and right-handed spinors have different quantum numbers with respect to \(SU(2)_L\) and \(U(1)_Y\) but the same \(Q\), which is the quantum number corresponding to \(U(1)_{\text{em}}\). 

The \(\psi_{L}\) are \(SU(2)_L\) \textbf{doublets}, with \(I = 1/2\); the \(\psi_{R}\) are \(SU(2)_L\) \textbf{singlets}, with \(I = 0\).

We then choose the values of the hypercharge such that the values of \(Q\) are compatible with experiment. 

\begin{figure}[H]
\centering
\begin{tabular}{cccc}
Particle & \(I^{3}\) & \(Y\) & \(Q\)\\
\hline
\(\nu_{e, L}\) & \(+ 1 /2\) & \(- 1/2\)& \(0\) \\
\(\nu_{e, R}\) & \(0\) & \(0\)& \(0\) \\
\(e^-_{L}\) & \(- 1 /2\) & \(- 1/2\)& \(-1\) \\
\(e^-_{R}\) & \(0\) & \(-1\)& \(-1\) \\
\(u^-_{L}\) & \(+ 1 /2\) & \(1/6\)& \(2/3\) \\
\(u^-_{R}\) & \(0\) & \(2/3\)& \(2/3\) \\
\(d^-_{L}\) & \(-1 /2\) & \(1/6\)& \(-1/3\) \\
\(d^-_{R}\) & \(0\) & \(-1/3\)& \(-1/3\) \\
\end{tabular}
\label{tab:fermions-quantum-numbers}
\caption{Electroweak quantum numbers of the fermions.}
\end{figure}

% A combination which is broken is 
% %
% \begin{align}
% Q_{z} = T^{3} - \sin^2 (\theta_{w}) Q
% \,,
% \end{align}
% %
% where \(\theta_{w}\) is the Weak, or Wonder angle, defined by 
% %
% \begin{align}
% \tan(\theta_{w}) = \frac{g'}{g}
% \,,
% \end{align}
% %
% where \(g'\) and \(g\) are the coupling constants relative to \(U(1)\) and \(SU(2)\) respectively.

% The model is called the Glashow-Weinberg-Salam model. This is the Standard Model of the electroweak interaction.

% How is this unification since we have two coupling constants? The constants are not the coupling constants of weak and electromagnetic interaction. The photon is given by
% %
% \begin{align}
% A_{\mu } = \sin(\theta_{w}) A^{3}_{\mu } + \cos(\theta_{w}) B_{\mu }
% \,,
% \end{align}
% %
% while the orthogonal combination is 
% %
% \begin{align}
% Z^{0}_{\mu } = \cos(\theta_{w}) A^{3}_{\mu } - \sin(\theta_{w}) B_{\mu }
% \,,
% \end{align}
% %
% so we cannot decouple the weak and electromagnetic bosons.
% This is why the symmetry is called the electroweak symmetry.

In terms of \(SU(2)_L\) multiplets, we separate: 
%
\begin{align}
\left[\begin{array}{c}
\nu_{L} \\ 
e^{-}_{L}
\end{array}\right]
&&
e^{-}_{R}
&& 
\left[\begin{array}{c}
u_L \\ 
d_L
\end{array}\right]
&&
u_R
&&
d_R
\,.
\end{align}

This is a \textbf{generation} (or family) of fermions: the ``electronic'' one, and we have two more: the muonic and tauonic ones.

\todo[inline]{Is this next bit useful?}
If \(\mu^2\) is a function of \(T\), then we can get a phase transition.
This is called the electroweak phase transition.

\end{document}
