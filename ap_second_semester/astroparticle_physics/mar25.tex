\documentclass[main.tex]{subfiles}
\begin{document}

\marginpar{Wednesday\\ 2020-3-25, \\ compiled \\ \today}

Today we are going to examine the third case of wave equations: we discussed the KG equation for a scalar particle, then we moved to a vector particle, and now we are going to consider a spin-1/2 particle. 


One has to prove that the equation \(\qty(i \slashed{\partial} - m)  \varphi (x )=0\) is Lorentz-covariant. 
Also, solutions of the Dirac equation are also solutions of the Klein-Gordon equation, therefore they satisfy \(E^2=p^2+m^2\). 
The latter is relatively easy to prove: we apply the operator \(i \slashed{\partial} + m\) to the Dirac equation and find 
%
\begin{align}
\qty(i \slashed{\partial} + m) \qty(i \slashed{\partial} - m) \psi (x) &= 0  \\
\qty(- \gamma^{\mu } \gamma^{\nu } \partial_{\mu } \partial_{\nu } - m^2 ) \psi (x) &= 0  \\
(\square + m^2 )\psi (x) &= 0 
\,,
\end{align}
%
where we used the fact that, since the derivatives \(\partial_{\mu } \partial_{\nu } \) commute, we can substitute \(\gamma^{\mu } \gamma^{\nu }\) with half of their anticommutator, which equals \(\eta^{\mu \nu }\). 

Since we have proven the equivalence, we expect the Dirac equation to also have negative energy solutions.

Proving the covariance is harder, even though the equation looks covariant, since \(\gamma^{\mu }\) does not transform as a vector \emph{a priori}. 

The object \(\psi (x)\) is a spinor: it is neither a scalar, nor a vector. A priori we can say that its transformation law will look like 
%
\begin{align}
\psi (x) \rightarrow \psi^{\prime } (x) = S(\Lambda ) \psi (x)
\,,
\end{align}
%
where \(S(\Lambda )\) is a unitary transformation associated with the Lorentz transformation \(\Lambda \), and by imposing the covariance of the Dirac equation we find that we must have 
%
\begin{align}
S (\Lambda ) \gamma^{\mu } S^{-1} (\Lambda ) = \qty(\Lambda^{-1})^{\mu }_{\nu } \gamma^{\nu }
\,.
\end{align}

So, in order to find the explicit form of \(S(\Lambda )\) we write an infinitesimal Lorentz transformation: it can be shown that we can write it as
%
\begin{align}
\tensor{\Lambda }{^{\mu }_{\nu }} = \tensor{\eta }{^{\mu }_{\nu }} + \delta \tensor{\omega}{^{\mu }_{\nu }}
\,,
\end{align}
%
where \(\delta \tensor{\omega }{^{\mu }_{\nu }}\) is an antisymmetric matrix, if it has nonzero \(0i\) components it gives boosts, if it has nonzero \(ij\) components it gives rotations. 
The result for the form of \(S\) is:
%
\begin{align}
S = \mathbb{1} + \frac{1}{8 } \qty[\gamma_{\mu }, \gamma_{\nu }] \delta \tensor{\omega }{^{\mu \nu }}
\,.
\end{align}

If we make a rotation, for example, we get 
%
\begin{align}
\psi (x) \rightarrow \exp(i \frac{i}{2} \qty[\gamma_{i}, \gamma_{j}] \delta \tensor{\omega }{^{ij}}) \psi (x) 
\,,
\end{align}
%
and typically one uses the shorthand rotation 
%
\begin{align}
\frac{i}{2} \qty[\gamma_{i}, \gamma_{j}] \overset{\text{def}}{=} \sigma_{ij}
\,.
\end{align}

If, for example, we want to perform a rotation by an angle \(\varphi \) around the \(z\) axis we get 
%
\begin{align}
\psi' (x') = S(\Lambda) \psi (x) = \exp( \frac{i}{2} \varphi \sigma_{12}) \psi (x)
\,,
\end{align}
%
where 
%
\begin{align}
\sigma_{12} = \frac{i}{2} \qty[\gamma^{1}, \gamma^{2}]
= \left[\begin{array}{cc}
\sigma_3 & 0 \\ 
0 & \sigma_3 
\end{array}\right]
\,,
\end{align}
%
where we used the fact that \(\qty[\sigma_i, \sigma_{j}] = i \epsilon_{ijk}\sigma_{k}\). 

This means that a spinor \(\psi (x)\) reacts in a peculiar way to rotations: it rotates by an angle \( \varphi /2 \) if we perform a Lorentz rotation of an angle \(\varphi \); its periodicity is \(4\pi \).  

We introduce the \emph{chiral representation} of the gamma matrices: 
%
\begin{subequations}
\begin{align}
\gamma^{0} = \left[\begin{array}{cc}
0 & -\mathbb{1} \\ 
- \mathbb{1} & 0
\end{array}\right]
\qquad \text{and} \qquad
\vec{\gamma} = \left[\begin{array}{cc}
0 & \vec{\sigma}  \\ 
-\vec{\sigma} & 0
\end{array}\right]
\,.
\end{align}
\end{subequations}

Then, for Lorentz boosts we have 
%
\begin{subequations}
\begin{align}
\sigma_{0i} = \frac{1}{2} \qty[\gamma_0, \gamma_{i}] = -i \left[\begin{array}{cc}
\sigma_{i} & 0 \\ 
0 & - \sigma_{i}
\end{array}\right]
\,,
\end{align}
\end{subequations}
%
while for rotations we have 
%
\begin{subequations}
\begin{align}
\sigma_{ij} = \frac{i}{2} \qty[\gamma_{i}, \gamma_{j}] = \epsilon_{ijk} \left[\begin{array}{cc}
\sigma_{k} & 0 \\ 
0 & \sigma_{k}
\end{array}\right]
\,,
\end{align}
\end{subequations}
%
which is useful since it gives us block-diagonal matrices. 
So, we can interpret the spinor as being made up of two components: 
%
\begin{subequations}
\begin{align}
\psi (x) = \left[\begin{array}{c}
\psi_1  \\ 
\psi_2  \\ 
\psi_3   \\ 
\psi_4 
\end{array}\right] 
= \left[\begin{array}{c}
\eta  \\ 
\omega 
\end{array}\right]
\,,
\end{align}
\end{subequations}
%
on which Lorentz transformations act independently. 
This is relevant since, for example, if we deal with an electron, we  will describe it with a 4-component spinor, however we will be able to divide it into two components \(e_L\) and \(e_R\), which are two component spinors on which we can act independently. 
We have effectively divided our representation of the Lorentz group into the sum of two irreps, of dimension \((1/2, 0)\) and \((0, 1/2)\) respectively. 

This will become very concrete when we will discuss how many degrees of freedom were present in the original plasma. 

Since a spinor \(\psi \) also solves the KG equation, it will be able to be written as 
%
\begin{subequations}
\begin{align}
\psi = \left[\begin{array}{c}
a \\ 
b \\ 
c \\ 
d
\end{array}\right] e^{-i px}
\,,
\end{align}
\end{subequations}
%
but what are the relations between the coefficients? 
We consider a simple case, that of \(\vec{p} =0\), so that we are in the rest frame of the particle. 

Then, the Dirac equation reads: 
%
\begin{subequations}
\begin{align}
\qty(\gamma^{0}E - m ) \left[\begin{array}{c}
a \\ 
b \\ 
c \\ 
d
\end{array}\right]=0
\,,
\end{align}
\end{subequations}
%
so we need to choose a representation for the \(\gamma^{0}\) in order to write this explicitly. We choose the Dirac representation, in which 
%
\begin{subequations}
\begin{align}
\gamma^{0} = \left[\begin{array}{cc}
\mathbb{1} & 0 \\ 
0 & -\mathbb{1}
\end{array}\right]
\,.
\end{align}
\end{subequations}

Then, the equation reads 
%
\begin{subequations}
\begin{align}
\left[\begin{array}{cccc}
E-m & 0 & 0 & 0 \\ 
0 & E-m & 0 & 0 \\ 
0 & 0 & -E-m & 0 \\ 
0 & 0 & 0 & -E-m
\end{array}\right] 
\left[\begin{array}{c}
a \\ 
b \\ 
c \\ 
d
\end{array}\right] e^{-ipx}
= 0
\,,
\end{align}
\end{subequations}
%
so we get a solution in which \(E=m\), and a solution in which \(E = -m \). So, in general we write the two linearly independent  solutions, respectively with positive and negative energy, as 
%
\begin{subequations}
\begin{align}
\psi = \left[\begin{array}{c}
\xi  \\ 
0
\end{array}\right] e^{-imt}
\qquad \text{and} \qquad
\Psi = \left[\begin{array}{c}
0 \\ 
\eta 
\end{array}\right] e^{+imt}
\,.
\end{align}
\end{subequations}

The negative energy solution, as we will see, represents the antiparticle of the Dirac fermion.

The assumption we made, \(\vec{p} = 0\), does not actually mean we lose generality: we can simply boost into the rest frame of the particle.
If we do this, we get the general 
%
\begin{subequations}
\begin{align}
\left[\begin{array}{cc}
E-m & - \vec{\sigma} \cdot \vec{p} \\ 
\vec{\sigma} \cdot \vec{p} & -E-m
\end{array}\right]
\left[\begin{array}{c}
\left[\begin{array}{c}
a \\ 
b
\end{array}\right] \\ 
\left[\begin{array}{c}
c \\ 
d
\end{array}\right]
\end{array}\right]
e^{-imt + i \vec{p} \cdot \vec{x}}
= 0
\,,
\end{align}
\end{subequations}
%
so we can decompose our solution into 
%
\begin{align}
\psi &= u^{s} e^{-i p  x} & E&>0 \\
\psi &= v^{s} e^{i p x} & E&<0
\,,
\end{align}
%
where \(s\) is an index denoting which 2D unit vector we are considering, that is, \(s = 1, 2\). 
So, now we come to the interpretation: we introduce the existence of an antifermion, which corresponds to the solution to the Dirac equation with negative energy, but it has the opposed momentum: so, it has positive energy.
Then, both of our solutions have positive energy and evolve forward in time, and both have the same mass. 

Now, we make the jump to second quantization: we start interpreting \(\psi (x) \) as an operator, which can destroy a one-state particle or create a particle starting from the vacuum. 

We will have 
%
\begin{align}
\bra{0} \psi (x) \ket{e^{-}(p,s)} = u^{s} (p) e^{-ipx}
\,.
\end{align}

Now, the one-particle state \(\ket{e^{-}(p, s)}\) is promoted to a spinor. 
On the other hand, we have the creation operator \(\psi ^\dag\): 
%
\begin{align}
\bra{e^{-}(p,s)} \psi ^\dag (x) \ket{0} = u^{s\dag} e^{-ipx}
\,.
\end{align}

Now, the tricky question is to introduce the negative-energy solution. 
The \(\psi ^\dag\) operator will destroy this state, while \(\psi \) will create it. 
So we will write an equation like 
%
\begin{align}
\bra{0} \psi ^\dag (x) \ket{e^{+} (p, s)} 
= v^{s \dag} (p) e^{-ipx} 
\,,
\end{align}
%
where we would write \(v^{s}(p)\) if we were considering the negative energy particle, instead we are looking at the antiparticle.

Now, we will be able to operate with \(\psi (x)\) on the vacuum, to find 
%
\begin{align}
\bra{e^{+}(p, s)} \psi (x) \ket{0} = v^{s} (p) e^{ipx}
\,.
\end{align}

\subsection{Photon-fermion coupling}

Yesterday we discussed the Lagrangian of the free photon field, 
%
\begin{align}
\mathscr{L} \propto F^{\mu \nu } F_{\mu \nu }
\,,
\end{align}
%
but as we said there can also be coupling to external currents, which we did not quantize. 
However, now we quantized the electron: so, can we construct the external current \(j_{\mu }\) in the coupling term \(j_{\mu }A^{\mu }\)? 

The first attempt would be to write something like 
%
\begin{align}
j^{\mu } \sim \psi ^\dag \gamma^{\mu } \psi 
\sim e^{+} \gamma^{\mu } e^{-}
\,,
\end{align}
%
but we would need to check whether it is a vector: in fact, it does not transform correctly. 

Is this at least a Hermitian operator? Well, its adjoint is 
%
\begin{align}
\qty(\psi ^\dag \gamma^{\mu } \psi ) ^\dag = \psi ^\dag \qty(\gamma^{\mu })^\dag \psi 
\,,
\end{align}
%
but 
%
\begin{align}
\qty(\gamma^{\mu })^\dag = \qty(\gamma^{0}, - \gamma^{i}) \neq \gamma^{\mu }
\,.
\end{align}

One finds that the correct definition is to have 
%
\begin{align}
\overline{\psi} \overset{\text{def}}{=}
\psi ^\dag \gamma^{0}
\,,
\end{align}
%
and then 
%
\begin{align}
j^{\mu } = \overline{\psi} \gamma^{\mu } \psi 
\,
\end{align}
%
is the correct definition. This, then, transforms as a 4-vector; it can also be shown starting from Dirac's equation (and its conjugate) that this is a conserved current: \(\partial_{\mu } j^{\mu } = 0\). 

Now, in order to couple the EM field to the electron we will use minimal coupling: 
%
\begin{align}
\partial_{\mu } \rightarrow \DD_{\mu } = \partial_{\mu } + ie A_{\mu }
\,,
\end{align}
%

so the Dirac equation will read 
%
\begin{subequations}
\begin{align}
\qty( i \slashed{\DD} -m ) \psi &= 0  \\
\qty(i \gamma^{\mu } \qty(\partial_{\mu } + ie A_{\mu }) -m ) \psi &= 0 
\,.
\end{align}
\end{subequations}

From which density Lagrangian can we derive the Dirac equation? It turns out to be 
%
\begin{align}
\mathscr{L} = \overline{\psi} \qty(i \gamma^{\mu } \DD_{\mu } -m ) \psi 
\,,
\end{align}
%
so if we want to describe both the EM field, the electron and their interaction, we have 
%
\begin{align}
\mathscr{L} (e, A_{\mu }) &= 
\underbrace{- \frac{1}{4} F^{\mu \nu } F_{\mu \nu }}_{\text{free EM field}} + \underbrace{\overline{\psi} \qty(i \slashed{\partial} - m) \psi}_{\text{electron}} 
- \underbrace{e\overline{\psi} \slashed{A} \psi}_{\text{electron-EM coupling}}  \\
&= - \frac{1}{4} F^{\mu \nu }F_{\mu \nu } + \overline{\psi} \qty(i \slashed{\DD} - m )\psi 
\,.
\end{align}

This is the density Lagrangian of Quantum ElectroDynamics. 
This is the first interacting QFT which was constructed, and it was extraordinarily successful. 

Its predictions for the anomalous magnetic moment of the electrons were exceptional: we can solve it perturbatively to different orders in 
%
\begin{align}
\alpha = \frac{e^2}{4 \pi } \approx \frac{1}{137}
\,.
\end{align}



\end{document}
