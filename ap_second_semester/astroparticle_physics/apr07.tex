\documentclass[main.tex]{subfiles}
\begin{document}

\marginpar{Tuesday\\ 2020-4-7, \\ compiled \\ \today}

Last time we wrote the Lagrangian of QED: 
%
\begin{align}
\mathscr{L}_{QED} = - \frac{1}{4} F^{\mu \nu } F_{\mu \nu } 
+ \overline{\psi} \qty(i \gamma^{\mu } \DD_{\mu } -m) \psi 
\,,
\end{align}
%
where \(\DD_{\mu } = \partial_{\mu } + i e A_{\mu }\).

This possesses several symmetries: Lorentz (actually, Poincaré) invariance, \(P\) (parity), \(C\) (charge conjugation), \(T\) (time inversion). 

We are also interested in the \emph{internal} symmetries of this Lagrangian: all the aforementioned symmetries were spacetime ones.

We know that \(\overline{\psi} = \psi ^\dag \gamma^{0}\), so if we transform \(\psi \to e^{i \theta } \psi \) the Lagrangian is unchanged, since we also have \(\overline{\psi} \to e^{-i \theta } \overline{\psi} \).
Here, we are taking a constant phase angle \(\theta \): it comes out of the derivative unchanged. 
This is called a \(U(1)\) global symmetry, since it is the same everywhere in space and since a phase is the same as a \(1 \times 1 \) unitary matrix.

The following section differs from the notes. 
Let us ignore the interactions of electrons with the EM fields in QED. 
Our world is made of electrons, and we want to describe these free propagating electrons. 
We take the Lagrangian 
%
\begin{align}
\mathscr{L}_{\text{Dirac}} = \overline{\psi} \qty(i \partial_{\mu } \gamma^{\mu } -m) \psi 
\,,
\end{align}
%
which still has the symmetry \(\psi \to e^{i \alpha } \psi \). 

Is this invariant also with respect to a \emph{local} \(U(1)\) symmetry? This looks like \(\psi (x) \to e^{i \alpha (x)} \psi (x)\), where \(\alpha (x)\) is a continuous spacetime scalar function. 

This is a ``promotion'' of the symmetry: why? It seems like a global symmetry is a more general thing\dots However, the global symmetry is a special case of the local one. 

This local symmetry is called a \(U(1)\) \emph{gauge} symmetry. 
Properly speaking, the global symmetry is also a gauge one but it is commonly called just a global symmetry. 

Substituting in, we get 
%
\begin{align}
e^{-i \alpha (x)} \overline{\psi} \qty[i \partial_{\mu } \gamma^{\mu } - m] e^{i \alpha (x)} \psi  
&= 
\overline{\psi} \qty[i \partial_{\mu } \gamma^{\mu } - m] \psi  
 + \overline{\psi} \psi i \gamma^{\mu }\partial_{\mu } \alpha 
\,,
\end{align}
%
so we see that the Lagrangian is \emph{not} invariant under this gauge symmetry in general. 

If we want the symmetry to hold, we need to introduce a \emph{compensating} field to cancel out the term.

The answer is that the thing to add is a vector field \(A_{\mu }\). 
Then, if we transform 
%
\begin{align}
A_{\mu } \to A_{\mu } - \frac{1}{e} \partial_{\mu } \alpha 
\,
\end{align}
%
this compensates the change, as long as the vector is coupled to the fermion with a term \(\overline{\psi} e \gamma^{\mu } A_{\mu } \psi \) in the Lagrangian. 
This is a profound result: if we wanted to describe a world with only electrons, and we want to have this electron be symmetric with respect to the \(U(1)\) gauge symmetry \(e \to e^{i \alpha (x)} e\) then we \emph{must} have a vector coupled to it. 

Then, we should also insert a term describing the propagation of the free vector \(A_{\mu }\). 

A quote: ``And she said `let there be symmetry', and there was light''.

We could have just proven that the QED Lagrangian is invariant with respect to \(U(1)\) symmetry: however, this reasoning illustrates the point that the symmetry requires the insertion of photons. 

One might ask: how do you know that this is the correct symmetry? 
The method is trial and error. 

Could we have something which is more complicated than a spin 1 mediator? It is basically a guessing game, we see what works. 

This \(U(1)\) gauge symmetry is an abelian symmetry, but we also have non-abelian ones: the \(T^{a}\) are Hermitian operators called generators, their Lie algebra which is defined by 
%
\begin{align}
\qty[T^{a}, T^{b}] = if^{abc} T^{c}
\,.
\end{align}

The \emph{structure coefficients} \(f^{abc}\) are antisymmetric in their first two indices. 

To say that these are generators means that any infinitesimal transformation can be written as 
%
\begin{align}
\Phi \to \qty(1 + i \alpha^{a} t^{a}_{R}) \Phi 
\,.
\end{align}

We will be interested in Lie groups \(SU(n)\) with \(N\geq 2\), which are \(N \times N\) unitary matrices with determinant \(1\).

For example, recall that \(SU(2)\) corresponds to \(SO(3)\). 

Some time ago we discussed the annihilation  of \(e^{+} e^{-}\) into hadrons: we can get protons and neutrons, pions, kaons\dots

However we can simplify by discussing only the creation of quarks. 

When we compute the cross sections, our calculation seems to be wrong by a factor 3. If we multiply it by 3 we get the correct result. 
So there are three types of quarks: we categorize them by ``color'', even though it has nothing to do with colors. 

We associated QED  with \(U(1)\): is there a group corresponding to Quantum Chromo Dynamics?
Can we do this with the weak interaction as well? 

Since there are three quarks, we are drawn to represent them as triplets. 
Since we know that unitary matrices are nice, we try \(SU(3)\). 

We call this symmetry \(SU(3)_{\text{color}}\). 

Consider now \(SU(N)\): we have 
%
\begin{align}
\Tr \qty[t^{a}_{N} t^{b}_{N}] = \frac{1}{2} \delta^{ab}
\,,
\end{align}
%
where \(t^{a}_{N}\) are the generators of the Lie algebra. 

We define the following: 
%
\begin{align}
\qty(t^{a}_{G})^{bc} \overset{\text{def}}{=} f^{abc}
\,,
\end{align}
%
which satisfy 
%
\begin{align}
\Tr \qty[t^{a}_{G} t^{b}_{G}] = f^{acd} f^{bcd} = C(G) \delta^{ab}
\,,
\end{align}
%
where the constant \(C(G)\) is just the dimension \(N\) in our case. 

Consider the Lagrangian 
%
\begin{align}
\mathscr{L} = \overline{\psi} \qty(i \gamma^{\mu } \partial_{\mu } - m ) \psi 
\,.
\end{align}

This can describe an electron. Now, let us add an index \(j\) in the group \(G\), according to which the particle transforms in the \(R\)-dimensional (in our case \(R=3\)) representation:
\begin{align}
\mathscr{L} = \overline{\psi}_{j} \qty(i \gamma^{\mu } \partial_{\mu } - m ) \psi_{j}
\,.
\end{align}

Under the local action of a group \(G\) the particle transforms like 
%
\begin{align}
\psi_{j} (x) \overbrace{\to}^{G} 
\psi^{'}_{j} (x)
= \qty(1 + i \alpha^{a}(x) t^{a}_{R})_{jk} \psi_{k} 
\,,
\end{align}
%
where \(a\) is an index going from 0 to \(N^2-1\). 

Now we have to do the same thing we did in the \(U(1)_{\text{em}}\) case. 

We have the same problem as before: we need to cancel a term with one index. So, we will need to take a derivative like 
%
\begin{align}
\partial_{\mu } \qty(\alpha (x) \psi ) = \qty(\partial_{\mu } \alpha ) \psi  
\,,
\end{align}
%
even though now ``\(\alpha \)'' is actually a product of many terms. The solution will still be in the form 
%
\begin{align}
\DD_{\mu } \to \partial_{\mu } - i g A_{\mu }^{a} t^{a}_{R}
\,,
\end{align}
%
where \(g\) is the strength of the interaction, commonly written as \(g_{s}\) in the case of the strong force. 
The index \(a\) goes from \(1\) to \(N^2-1 = 8\). 

The beautiful thing is the fact that starting only from the symmetry requirement we can get a full theory. 

We now attribute actual \emph{existence} to these quantum fields: they are interaction bosons. 
They must transform like 
%
\begin{align}
A^{a}_{\mu }(x) \to A^{a}_{\mu } (x) + \frac{1}{g} \DD_{\mu } \alpha^{a}(x)
\,,
\end{align}
%
and then the Lagrangian will be 
%
\boxalign{
\begin{align}
\mathscr{L} =
- \frac{1}{4} F^{\mu \nu a} F^{a}_{\mu \nu } + \overline{\psi}
\qty(i \gamma^{\mu } \DD_{\mu }-m) \psi 
\,,
\end{align}}
%
which looks the same as the QED Lagrangian, but we must be careful: what is \(F_{\mu \nu }^{a}\)? We might think it is 
%
\begin{align}
F^{a}_{\mu \nu } = 2 \partial_{[\mu } A^{a}_{\nu ]}
\,,
\end{align}
%
but this is not enough: inside the covariant derivative in \(\DD_{\mu } \alpha \) in the transformation law we also have \(A_{\mu }\), so we get an additional term, and the final formula looks like 
%
\begin{align}
F^{a}_{\mu \nu } = 2 \partial_{[\mu } A^{a}_{\nu ]} + g f^{abc} A^{b}_{\mu } A^{c}_{\nu }
\,.
\end{align}

This corresponds to the fact that, while there is no photon-photon interaction since \(U(1)\) is abelian,  \(SU(3)\) is not: so, we do have gluon-gluon interaction. 

This is crucial in very high energy situations, such as in the early universe. 

\end{document}
