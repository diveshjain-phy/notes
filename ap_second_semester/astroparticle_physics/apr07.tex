\documentclass[main.tex]{subfiles}
\begin{document}

\subsection{Gauge symmetries in QED}

\marginpar{Tuesday\\ 2020-4-7, \\ compiled \\ \today}

Last time we wrote the Lagrangian of QED: 
%
\begin{align}
\mathscr{L}_{QED} = - \frac{1}{4} F^{\mu \nu } F_{\mu \nu } 
+ \overline{\psi} \qty(i \gamma^{\mu } \DD_{\mu } -m) \psi 
\,,
\end{align}
%
where \(\DD_{\mu } = \partial_{\mu } + i e A_{\mu }\).

This possesses several symmetries: Lorentz (actually, Poincaré) invariance, \(P\) (parity), \(C\) (charge conjugation), \(T\) (time inversion). 

We are also interested in the \emph{internal} symmetries of this Lagrangian: all the aforementioned symmetries were spacetime ones.

We know that \(\overline{\psi} = \psi ^\dag \gamma^{0}\), so if we transform \(\psi \to e^{i \theta } \psi \) the Lagrangian is unchanged, since we also have \(\overline{\psi} \to e^{-i \theta } \overline{\psi} \).
Here, we are taking a \emph{constant} phase angle \(\theta \): it comes out of the derivative unchanged. 
This is called a \(U(1)\) \textbf{global} symmetry, since it is the same everywhere in space and since a phase is the same as a \(1 \times 1 \) unitary matrix.

The following section differs from the notes. 
Let us ignore the interactions of electrons with the EM fields in QED. 
Our world is made of electrons, and we want to describe these free propagating electrons. 
We take the Lagrangian 
%
\begin{align}
\mathscr{L}_{\text{Dirac}} = \overline{\psi} \qty(i \gamma^{\mu } \partial_{\mu }  -m) \psi 
\,,
\end{align}
%
which still has the symmetry \(\psi \to e^{i \alpha } \psi \). 

Is this invariant also with respect to a \emph{local} \(U(1)\) symmetry? This looks like \(\psi (x) \to e^{i \alpha (x)} \psi (x)\), where \(\alpha (x)\) is a continuous spacetime scalar function. 

This is a ``promotion'' of the symmetry: why? It seems like a global symmetry is a more general thing\dots However, the global symmetry is a special case of the local one. 

This local symmetry is called a \(U(1)\) \emph{gauge} symmetry. 
Properly speaking, the global symmetry is also a gauge one but it is commonly called just a global symmetry. 

Substituting in, we get 
%
\begin{align}
e^{-i \alpha (x)} \overline{\psi} \qty[i \partial_{\mu } \gamma^{\mu } - m] e^{i \alpha (x)} \psi  
&= 
\overline{\psi} \qty[i \partial_{\mu } \gamma^{\mu } - m] \psi  
 + \overline{\psi} \psi i \gamma^{\mu }\partial_{\mu } \alpha 
\,,
\end{align}
%
so we see that the Lagrangian is \emph{not} invariant under this gauge symmetry in general. 

If we want the symmetry to hold, we need to introduce a \emph{compensating} field to cancel out the term.

The answer is that the thing to add is a vector field \(A_{\mu }\). 
Then, if we transform 
%
\begin{align}
A_{\mu } \to A_{\mu } - \frac{1}{e} \partial_{\mu } \alpha 
\,
\end{align}
%
this compensates the change, as long as the vector is coupled to the fermion with a term \(\overline{\psi} e \gamma^{\mu } A_{\mu } \psi \) in the Lagrangian. 
This is a profound result: if we wanted to describe a world with only electrons, and we want to have this electron be symmetric with respect to the \(U(1)\) gauge symmetry \(e \to e^{i \alpha (x)} e\) then we \emph{must} have a vector coupled to it. 

Then, we should also insert a term describing the propagation of the free vector \(A_{\mu }\), the kinetic term \(\propto F^{\mu \nu } F_{\mu \nu }\).

A quote: ``And she said `let there be symmetry', and there was light''.

We could have just proven that the QED Lagrangian is invariant with respect to \(U(1)\) symmetry: however, this reasoning illustrates the point that the symmetry requires the insertion of photons. 

One might ask: how do you know that this is the correct symmetry? 
The method is trial and error. 

Could we have something which is more complicated than a spin 1 mediator? It is basically a guessing game, we see what works. 

\section{QCD}

This section can also be followed from Peskin \cite[sec.\ II.11]{peskinConceptsElementaryParticle2019}.

This \(U(1)\) gauge symmetry is an abelian symmetry, but we also have non-abelian ones: we denote by \(T^{a}\) the generators of the group, their Lie algebra is defined by 
%
\begin{align}
\qty[T^{a}, T^{b}] = if^{abc} T^{c}
\,.
\end{align}

The \emph{structure coefficients} \(f^{abc}\) are manifestly antisymmetric in their first two indices. It can also be shown that they are fully antisymmetric. If the group is abelian, we have \(f^{abc}\) identically, but this need not be the case.

To say that these are generators means that any infinitesimal transformation can be written as 
%
\begin{align}
\Phi \to \qty(1 + i \alpha^{a} t^{a}_{R}) \Phi 
\,,
\end{align}
%
where \(\alpha^{a}\) are the parameters of the infinitesimal transformation, while \(t^{a}_{R}\) are Hermitian matrices of dimension \(d_R\) which make up the representation of the group. There are \(d_G\) of them, where \(d_G\) is the dimension of the group.

The finite unitary transformation mapping \(\Phi \to U(\alpha ) \Phi \) can be recovered from here by 
%
\begin{align}
U(\alpha ) = e^{i \alpha^{a} t^{a}_{R}} 
\,.
\end{align}

We will be interested in Lie groups \(SU(n)\) with \(N\geq 2\), which are \(N \times N\) unitary matrices with determinant \(1\).

For example, recall that \(SU(2)\) has a 2-to-1 correspondence with \(SO(3)\). \(SU(2)\) describes the rotation of spinors, the generators of their rotation are \(\sigma^{i} / 2\). 

Last time we discussed the annihilation  of \(e^{+} e^{-}\) into hadrons: we can get protons and neutrons, pions, kaons\dots

However we can simplify by discussing only the creation of quarks. 

When we compute the cross sections, our calculation seems to be wrong by a factor 3. If we multiply it by 3 we get the correct result. 
So there are three types of quarks: we categorize them by ``color'', even though it has nothing to do with colors. 

We associated QED  with \(U(1)\): is there a group corresponding to Quantum Chromo Dynamics?
Can we do this with the weak interaction as well? 

Since there are three quarks, we are drawn to represent them as triplets. 
Since we know that unitary matrices are nice, we try \(SU(3)\). 
We call this symmetry \(SU(3)_{\text{color}}\). 

We start by giving some general results for \(N\)-dimensional groups \(SU(N)\): we normalize their representation by imposing
%
\begin{align}
\Tr \qty[t^{a}_{N} t^{b}_{N}] = \frac{1}{2} \delta^{ab}
\,,
\end{align}
%
where \(t^{a}_{N}\) are the Hermitian generators of an \(N\)-dimensional unitary representation. If we generalize to an \(R\)-dimensional representation, we will have 
%
\begin{align}
\Tr \qty[t^{a}_{G} t^{b}_{G}] = C(R) \delta^{ab}
\,,
\end{align}
%
where \(C(R)\) is some constant depending only on the dimension of the representation. 

A special representation we can choose is the \textbf{adjoint} representation, which is the one under which the generators of the Lie algebra transform; it is defined by: 
%
\begin{align} \label{eq:adjoint-representation-definition}
\qty(t^{a}_{G})^{bc} \overset{\text{def}}{=} i f^{abc}
\,.
\end{align}

By making use of the Jacobi identities we can show that this is indeed a valid representation of the group, its dimension is that of the group and we have:
%
\begin{align}
\Tr \qty[t^{a}_{G} t^{b}_{G}] = f^{acd} f^{bcd} = C(G) \delta^{ab}
\,,
\end{align}
%
where the constant \(C(G)\) is just the dimension \(N\) for the adjoint representation.
For \(SU(N)\) the dimension is \(N^2 - 1\): the representation consists of \(N^2-1\) matrices, each \(N \times N\).

\subsection{Non-abelian gauge theory: Yang-Mills}

Consider the Lagrangian 
%
\begin{align}
\mathscr{L} = \overline{\psi} i \gamma^{\mu } \partial_{\mu }  \psi 
\,.
\end{align}

This can describe an electron. Now, let us add an index \(j\) in the group \(G\), according to which the particle transforms in the \(R\)-dimensional (in our case \(R=3\)) representation:
\begin{align}
\mathscr{L} = \overline{\psi}_{j} i \gamma^{\mu } \partial_{\mu } \psi_{j}
\,.
\end{align}

Under the local action of a group \(G\) the particle transforms like 
%
\begin{align}
\psi_{j} (x) \overset{G}{\to} 
\psi^{'}_{j} (x)
= \qty(1 + i \alpha^{a}(x) t^{a}_{R})_{jk} \psi_{k} 
\,,
\end{align}
%
where \(a\) is an index going from 1 to \(N^2-1\). 

We have the same problem we had with \(U(1)_{\text{em}}\): we need to cancel the term coming from the derivative of \(\alpha (x)\), with a one-index object.
The variation in the Lagrangian is
%
\begin{align}
\delta \mathscr{L} = \overline{\psi}_{j} i \gamma^{\mu } \qty(i \partial_{\mu } \alpha^{a} (x) t^{a}_{R, jk}) \psi_{k}
\,.
\end{align}

The solution is the same: we introduce a coupling to the derivative, which takes the form
%
\begin{align}
\DD_{\mu } \to \partial_{\mu } - i g A_{\mu }^{a} t^{a}_{R}
\,,
\end{align}
%
where \(g\) is the strength of the interaction, which is also written as \(g_{s}\) in the case of the strong force. 
The index \(a\) goes from \(1\) to \(N^2-1 = 8\): we must introduce a vector field \(A\) for each \emph{generator} of the required symmetry. 

The beautiful thing is the fact that starting only from the symmetry requirement we can get a full theory. 

We now attribute actual \emph{existence} to these quantum fields: they are interaction bosons. 
They must transform like 
%
\begin{align}
A^{a}_{\mu }(x) &\to  A^{a}_{\mu } (x) + \frac{1}{g} \partial_{\mu } \alpha^{a} (x) + \overbrace{A^{b}_{\mu } f^{abc} \alpha^{c} (x)}^{\text{new nonabelian term}} 
 \\
&=  A^{a}_{\mu } (x) + \frac{1}{g} \DD_{\mu } \alpha^{a}(x)
\,.
\end{align}

This is derived making use of the adjoint representation definition \eqref{eq:adjoint-representation-definition}: we are equating 
%
\begin{align}
A^{b}_{\mu } f^{abc} \alpha^{c} &= -i A^{b}_{\mu } t^{b}_{R} \alpha^{a}  \\
A^{b}_{\mu } f^{abc} \alpha^{c} &= -i A^{b}_{\mu } (i f^{bca}) \alpha^{c} \\
A^{b}_{\mu } f^{abc} \alpha^{c} &= + A^{b}_{\mu } f^{abc} \alpha^{c} 
\marginnote{We can do cyclic permutations of the indices of \(f^{abc}\).}
\,.
\end{align}

If we assign physical reality to these fields we must give them kinetic terms in the Lagrangian, which will then be 
%
\boxalign{
\begin{align}
\mathscr{L} =
- \frac{1}{4} F^{\mu \nu a} F^{a}_{\mu \nu } + \overline{\psi}
\qty(i \gamma^{\mu } \DD_{\mu }-m) \psi 
\,,
\end{align}}
%
which looks the same as the QED Lagrangian, but we must be careful: what is \(F_{\mu \nu }^{a}\)?
We might think it is 
%
\begin{align}
F^{a}_{\mu \nu } = 2 \partial_{[\mu } A^{a}_{\nu ]}
\,,
\end{align}
%
but this is not enough: inside the covariant derivative in \(\DD_{\mu } \alpha \) in the transformation law we also have \(A_{\mu }\), so we get an additional term, and the final formula looks like 
%
\begin{align}
F^{a}_{\mu \nu } = 2 \partial_{[\mu } A^{a}_{\nu ]} + g f^{abc} A^{b}_{\mu } A^{c}_{\nu }
\,.
\end{align}

This corresponds to the fact that, while there is no photon-photon interaction since \(U(1)\) is abelian,  \(SU(3)\) is not: so, we do have gluon-gluon interaction. 
Notice that this expression has abelian symmetries as a special case: the term \(fAA\) is antisymmetric in the two bosons, so if they commute it vanishes. 
The square of this field strength is then both Lorentz and gauge invariant.

This field strength can be interpreted as the Riemann tensor of the Lie group manifold: 
%
\begin{align}
\qty[\DD_\mu , \DD_\nu] = -ig F^{a}_{\mu \nu } t^{a}_{R}
\,.
\end{align}

This is crucial in very high energy situations, such as in the early universe. The dynamics of the field are now more complicated than the ones we find in electrodynamics due to the nonlinear terms.

The wavefunction \(\psi\) which appears in the Lagrangian will transform in some \(d\)-di\-men\-sio\-nal representation of \(G = SU(3)_c\). 
It will have indices like \(\psi_{\alpha , i}\): \(\alpha \) is a four-dimensional spinorial index, while \(i\) is a three-dimensional color index. 

The strong-interaction coupling constant \(g_s \) is dimensionless, we also define the parameter
%
\begin{align}
\alpha_{s} = \frac{g_s^2}{4 \pi }
\,.
\end{align}

The coupling term of the gluons with the fields can be made explicit as 
%
\begin{align}
\overline{\psi}_{\alpha, i} \gamma^{\mu }_{\alpha \beta } A^{a}_{\mu } \qty(t^{a})_{ij} \psi_{\beta, j}
\,,
\end{align}
%
where \(\mu \) is a Lorentz index, \(i\) and \(j\) are color indices, while \(\alpha \) and \(\beta \) are spinorial indices.

\end{document}
