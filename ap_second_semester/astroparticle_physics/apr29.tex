\documentclass[main.tex]{subfiles}
\begin{document}

\subsection{Symmetries of the SM}

\marginpar{Wednesday\\ 2020-4-29, \\ compiled \\ \today}

Let us write the Standard Model Lagrangian \emph{after} spontaneous symmetry breaking (so, at low energies): 
%
\begin{align}
\mathscr{L} = - \frac{1}{4} \sum _{a} \qty(F_{\mu \nu }^{a})^2
+ m_W^2 W^{+}_{\mu } W^{-, \mu }
+ \frac{1}{2} m_Z^2 Z_{\mu } Z^{\mu }
+ \sum _{f} \overline{\psi}_{f} \qty(i \slashed{\DD} - m_f) \psi_{f} 
+ \frac{1}{2} \qty(\partial_{\mu } h)^2 - V(h)
\,,
\end{align}
%
where \(a\) runs over all our \(8 + 3 + 1 = 12\) gauge bosons (8 gluons, the \(W^{\pm }\) and \(Z^{0}\), the photon), and the covariant derivative is given by: 
%
\begin{align}
\DD_{\mu f} = \partial_{\mu } - i e Q_f A_{\mu } 
- i \frac{g}{\cos \theta_{w}} Q_{Zf} Z_{\mu } 
- i g_s A^{a}_{\mu } t^{a}
\,.
\end{align}

\todo[inline]{Something that should have come up before: what's up with the \(W\) mass term? It does not look like a usual mass term\dots }

The weak force interactions are diagonal for the leptons, proportional to \(V_{CKM}\) for the quarks.

Let us discuss the remaining symmetries of the model. 
These are symmetries which are not imposed on the model by hand, instead they follow automatically from the terms which are allowed by the chosen gauge symmetries.

\subsubsection{Baryon and Lepton number conservation}

We define 
%
\begin{align}
B(\text{Quark}) = \frac{1}{3} \qquad \text{and} \qquad
B(\text{antiQuark}) = - \frac{1}{3}
\,,
\end{align}
%
while for any other particle we assign 0.
This is conserved, but we need not impose it: the most general Lagrangian we write with our symmetries has it.
This is a global symmetry: \(U(1)_{B}\). 

The way to find it is to construct a baryonic current \(j^{\mu }_{B}\), in such a way that it is conserved in our Lagrangian, then the integral of \(j^{0}_{B}\) will give the conserved charge.

This has heavy consequences: the lightest baryon is the proton. If it were to decay, this would not conserve baryon number. 

If we are allowed to violate baryon number conservation we can have a decay like \(p \to e^{+ } + \gamma \), which would occur and destabilize atoms.

We get a bound on the lifetime of the proton of 
%
\begin{align}
\tau _{\text{proton}} > \SI{e32}{yr}
\,.
\end{align}

We also have lepton number conservation.
This is more interesting since we have neutrinos, which are hard to detect.

We will see a method to give mass to neutrinos, which will however violate lepton number conservation.

\subsubsection{\(C\), \(P\) and \(T\) symmetries}

Strong and EM interaction conserve these three symmetries, while the coupling of the weak bosons to the fermions violate \(C\) and \(P\) maximally.
If the couplings were real-valued, \(CP\) symmetry would be preserved, but the \(V_{CKM}\) matrix has a physical phase, so this is not the case. 

By the \(CPT\) theorem, \(T\) is also violated.

\subsubsection{Flavour number conservation}

The only thing which couples different families of fermions together is the \(CKM\) matrix, which gives mixing terms between the quarks, as 
%
\begin{align}
\overline{u}_{i} \gamma^{\mu } (V_{CKM})_{ij} \phi_{j}
\,,
\end{align}
%
and the Cabibbo angle measures how much this happens; we then expect to see decays like \(u \to s + W^{+}\). 

For leptons, instead, we expect perfect conservation of lepton number since there is no mixing. Experimentally this seems to be quite well verified: if process like \(\mu \to e \gamma \) or \(\tau \to e \gamma \) happen they must do so extremely rarely. 

\subsection{Neutrino masses}

% The Yukawa Lagrangian is written as 
% %
% \begin{align}
% \mathscr{L}_{\text{Yukawa}}
% = - y_{e} L ^\dag_{a} \phi_{a} e_{R}
% -y_{d} O ^\dag_{a} \phi_{a} d_{r}
% -y_{a} O ^\dag_{a} \epsilon_{ab} \phi ^\dag_{b} u_{R} 
% + \text{h.c.}
% \,.
% \end{align}

% Commonly this is written as \(\overline{\psi}_{L} \phi \psi_{R}\), this however does not mean \(\overline{\psi} = \psi ^\dag \gamma^{0}\), since it is applied to a 2-component spinor. 

In the mass term we add to the Yukawa Lagrangian,
the index \(a\) is an SU(2) index, and the Higgs field looks like 
%
\begin{subequations}
\begin{align}
\phi = \left[\begin{array}{c}
\phi^{+} \\ 
\phi^{0}
\end{array}\right]
\,;
\end{align}
\end{subequations}
%
we have two terms like 
%
\begin{subequations}
\begin{align}
\left[\begin{array}{cc}
\nu_{L} & e^{-}_{L}
\end{array}\right]
\left[\begin{array}{c}
\phi^{+} \\ 
\phi^{0}
\end{array}\right]
e_{R}
\,,
\end{align}
\end{subequations}
%
and also 
%
\begin{subequations}
\begin{align}
\left[\begin{array}{cc}
\nu_{R}^{c} & e^{+}_{R}
\end{array}\right]
\left[\begin{array}{c}
\phi^{+} \\ 
\phi^{0}
\end{array}\right]
e_{R}^{-}
\,.
\end{align}
\end{subequations}

This means that the neutrino is massless: the VEV of \(\phi_{a}\) is \((0, v)\), so only the \textbf{second} component gains mass after SSB.
So, we get Dirac mass terms for the electron but not for the neutrino.

This is equivalent to saying that the mass matrix for the charged leptons is diagonal: there is no mixing.

How can we detect whether neutrinos have mass, \textbf{experimentally}? 
A path is \textbf{\(\beta \)-decay}: 
\begin{align}
n \to p + e^{-} + \overline{\nu}_{e}
\,.
\end{align}

We look at the so-called Kurie plot: on the horizontal axis we put the measured energy of the electrons \(E_e\), on the vertical axis we put the square root of the number of electrons measured at that energy (in a small range around it: it is a histogram): \(\sqrt{N_e}\). 
This plot is expected to look like a straight line if the neutrinos are massless, with some electrons reaching the maximum center-of-mass energy; on the other hand if neutrinos are massive the kinematics of the process change and the curve dips down earlier.

People have measured this to a very high degree of precision.
What we have found up to now is a bound \(m_{\nu_{e}} < \SI{2}{eV}\) in these beta decay experiments. 

The result that the neutrinos are massive did not come from here. 

Other paths are \textbf{neutrino oscillations} and bounds from \textbf{cosmology}. 
From the neutrino oscillations we have the result that, at least for some neutrinos, \(m_{\nu } \neq 0\).

However, in the Standard Model neutrinos are massless. How can we model their having mass?

Ettore Majorana proposed a way to give a mass for the neutrino (or any neutral fermion, really).

% The symmetries of the SM are, generally, \(\text{Lorentz} \otimes \text{Gauge}\).
% We then write all the terms we can which are invariant under these. 

% From these we can derive conservation laws, such as that for the Baryon number. 

% Parity and charge are massively violated.

The Dirac Lagrangian, as we saw, after SSB reads: 
%
\begin{align}
\mathscr{L} _{\text{Dirac}} = \psi ^\dag_{R} \qty(i \sigma \cdot \partial) \psi_{R} + \psi ^\dag_{L} \qty(i \overline{\sigma} \cdot \partial) \psi_{L} 
- m \qty(
 \psi ^\dag_{R} \psi_{L} 
+ \psi ^\dag_{L} \psi_{R} 
)
\,,
\end{align}
%
where \(m = m_f = y_f v/ \sqrt{2}\). 
However, in our SM we did not introduce a right-handed neutrino! 
So, one possibility to give mass to the left-handed neutrino is to introduce a right-handed one, with a term like 
%
\begin{align}
\Delta \mathscr{L} _{\text{Yukawa}} = 
- y^{ij}_{\nu } L ^{\dag, i}_{a} \epsilon_{ab} \phi^{*}_{b} \nu_{R}^{j}
\,,
\end{align}
%
in analogy to the term we have for quarks.
This yields a term \(\nu_{L} ^\dag \phi_0 \nu_{R} \), which near the vacuum gives mass to the neutrinos, with a term like 
%
\begin{align}
m^{ij}_{\nu } = - y^{ij}_{\nu } \frac{v}{\sqrt{2}}
\,.
\end{align}

We know that if the neutrinos do indeed have a mass it is very small, of the order of an \SI{}{eV} or less. 
So, we choose a basis in which the interaction of the \(\nu_{L}\) with the charged leptons are ``diagonal'': \(\overline{\nu}_{L}^{i} \gamma^{m } \ell^{i}_{L} W_{\mu }\), where \(\ell\) is a lepton (electron, muon, tauon).

As before we perform a change of variables making use of the unitary matrices \(U\): 
%
\begin{align}
e^{i}_{R} \to U^{(e)}_{R, ij} e^{j}_{R}
\qquad \text{and} \qquad
L^{i} \to U_{L, ij}^{(e)} L^{j}
\,.
\end{align}

Then, the fermion mass couplings will be: 
%
\begin{align}
y_\ell = U_L^{(e)} Y_\ell U_R^{(e),\dag}
\,,
\end{align}
%
and they can be chosen to be diagonal. However, in general \(y_\nu \) is not diagonalized by these unitary matrices! 

The current eigenstates \(\nu_{L}^{(e, \mu , \tau )}\) which are produced at the charged current vertex are \textbf{different} from the mass eigenstates \(\nu_{1, 2, 3}\) with masses \(m_{\nu_{1, 2, 3}}\).

If we wish to diagonalize \(y_\nu \), we can do so as 
%
\begin{align}
y_\nu' = U_L^{(e), \dag} y_\nu = U_L^{(\nu)} Y_\nu U_R^{(\nu )}
\,.
\end{align}

The transition matrices \(U\) have three angles and one phase in freedom (which is to say, the other degrees of freedom in the matrix can be absorbed into global phases for the other fields). 
They are the Pontecorvo-Maki-Nakagawa–Sakata mixing matrix. 

% CP symmetry moves us from \(e^{-}_{L}\) to \(e^{+}_{L}\) to \((e^{+})_{R}\).

% If we have CP symmetry, then if in our model we have a \(\nu_{L}\) we will also need to have a \(\nu^{c}_{R}\).
% That is, there are only two degrees of freedom between them.

% In the SM we introduce only left-handed neutrinos. However, we can also introduce 
% %
% \begin{align}
% \nu_{R} \iff (\nu^{c})_{L}
% \,.
% \end{align}

Let us connect these musings to experiment.
From cosmic ray interactions we have charged pions, which decay into 
%
\begin{align}
\pi^{+} \to \mu^+ + \nu_{\mu }
\,,
\end{align}
%
where the muonic neutrino we get is a \emph{flavour} eigenstate, since it is produced in a weak-interaction vertex; it is a linear combination of the mass eigenstates. 

Then, the muon decays into 
%
\begin{align}
\mu^+ \to e^{+} + \nu_{e} + \overline{\nu}_{\mu }
\,.
\end{align}

So, naively we expect twice as many \(\nu_{\mu }\) as \(\nu_{e}\).

Now, we know that for any particle \(p = \sqrt{E^2 - m^2} \approx E - m^2 / 2E\) as long as \(m\) is small compared to \(E\) (which is definitely the case for our very light neutrinos). So, the momenta of the mass eigenstates of the neutrinos at a fixed energy \(E\)  are 
%
\begin{align}
p_i \approx E - \frac{m_i^2}{2E}
\,,
\end{align}
%
for \(i = 1, 2, 3\). So, if the masses are different then the momenta will be slightly (\(\mathcal{O} (m/E)\)) different. 

The wavefunction of the muonic neutrino \(\nu_{\mu }\) will look like 
%
\begin{align}
\sum _{i} V^{PMNS}_{(\mu ), i} e^{i (E - m_i^2 /2E) x}
\,,
\end{align}
%
so we can see that the different components will go out of phase! This is very interesting, since even though the difference in the momenta is small we can measure at macroscopically different values of \(x\), so that the phase difference is large! 

Let us consider only two neutrino species for simplicity: \(\nu_{e}\) and \(\nu_{\mu }\). In order to get the mass eigenstates from the flavour ones we will have a matrix \(V\) depending on a mixing angle \(\theta \): 
%
\begin{align}
V = \left[\begin{array}{cc}
\cos \theta  & -\sin \theta  \\ 
\sin \theta  & \cos \theta 
\end{array}\right]
\,.
\end{align}

Then, the probability of observing a \(\nu_{\mu }\) given that we started with one will look like 
%
\begin{align}
\mathbb{P} (\nu_{\mu } \to \nu_{\mu })
&= \abs{
    \cos^2\theta e^{-i (m_1^2 / 2E )x}+
    \sin^2\theta e^{-i (m_2^2 / 2E )x}
}^2  \\
&= 1 - \sin^2 (2 \theta ) \sin^2 \qty( \frac{ \delta m^2}{2E} x)
\,,
\end{align}
%
so we will observe an oscillation length scale of 
%
\begin{align}
L = \frac{4 \pi E}{ \delta m^2} \approx \SI{2.48}{m} \qty( \frac{E}{\SI{}{MeV}}) \qty( \frac{ \SI{}{eV^2}}{ \delta m^2})
\,.
\end{align}

The amplitude of this oscillation is maximal if \(\theta= \pi /4\). 
Note that in order for oscillations to occur we need \(\theta \neq 0\) and \(\delta m^2 \neq 0\). 

Experimentally, we see them: at a neutrino detector we distinguish downward neutrinos, which come from the sky above the detector, and upward neutrinos, which come from the other side of the Earth (through it). 
We see the correct ratio \(\nu_{\mu } / \nu_{e} \approx 2\) for downward neutrinos, while for upward neutrinos the muonic flux is suppressed (and the electron-neutrino flux is isotropic).

So, we are seeing \(\nu_{\mu } \leftrightarrow \nu_{\tau }\) flavour mixing with a length scale comparable to the Earth's diameter. 

\todo[inline]{Are we sure about this? could it not be that the length scale is (much) smaller, and we are underestimating the suppression?}

The parameters we get are 
%
\begin{align}
\delta m^2_{(\mu, \tau )} \approx \SI{2.4e-3}{eV^2} \approx \qty(\SI{5e-2}{eV})^2
\,,
\end{align}
%
and \(\sin^2\theta \approx 0.4\).

Also, we detect fewer neutrinos from the Sun than we'd expect from the standard Solar model; this gives us bounds with regards to \(e\) to \(\mu , \tau \) oscillations on the order of \(\abs{\delta m} \approx \SI{e-2}{eV}\) and \(\sin^2\theta \approx 0.3\).

From these measurement, we get that all the neutrino masses are within about a tenth of an electronVolt of each other, two of them being close.
We do not know the hierarchy: is the lone one (conventionally \(3\)) heavier or lighter than the other two?

We recently saw some indications of \(CP\) violations in neutrino physics: this is indicated by differences in the probabilities when considering oscillations between neutrino flavours and ones between \emph{anti}neutrino flavours.

% We can build detectors in order to see these. We expect a muonic neutrino and antineutrino. 

% The state \(\nu_{\mu }\) is a charge eigenstate, not a mass eigenstate: in general it will be written as \(a \nu_1 + b n_2 +c \nu_3 \).

% Its time-evolution will then exhibit an oscillation in the probability to still find a \(\nu_{\mu }\), since the mass components oscillate at different frequencies.

% The experiment OPERA found that some of the neutrinos from CERN going to Gran Sasso were converted to \(\nu_{\tau }\) and such. 

% Also, Kamiokande found a ratio different from 2 of \(\nu_{\mu }\) to \(\nu_{e}\).

% So, if there is an oscillation it means that the neutrinos must indeed be massive, or at least some of them must be.

% We put a Dirac mass term for the \(\nu_{R}\): the mass will look like 
% %
% \begin{align}
% m_{\nu } =  y_{\nu } v
% \,.
% \end{align}

% But we know that the masses of the neutrinos are very small, from beta decay but also from structure formation in the early universe.

There are two possibilities in order to give mass to the neutrinos. 
The first is to introduce right-handed neutrinos, so that we have a Dirac mass term: \(m \overline{\nu}_{L} \nu_{R}\). 

This adds two degrees of freedom to our model, since each neutrino has 2 dof. 

Now, \(\nu_{L}\) has only two degrees of freedom: denoting by \(\alpha \)and \(\beta \) a Weyl index (spinorial, from 1 to 2) we can add to the Lagrangian a term like \(\nu_{L}^{\alpha } \nu_{L}^{\beta } \epsilon_{\alpha \beta }\). This is a singlet under Lorentz transformations. 

This term would violate lepton number conservation, since each of the left handed neutrinos increases \(L\) by 2! However we did not include it at the start, it was an accidental outcome of our model. 

However, \(\nu_{L}\) belongs to a \(SU(2)_L\) doublet \((\nu_{L}, e_L)\), with \(T^{\nu_{L}}_{3} = + 1/2\). Therefore, the term \(\nu_{L} \nu_{L}\) would also be a \(SU(2)_L\) triplet. 

Now, the bare neutrino has nonzero hypercharge and neither do two of them. 
\todo[inline]{And so\dots?}

Could we do something like a Yukawa coupling, \(\nu_{L} \nu_{L} \phi \)? No, since \(\phi \) is a \(SU(2)_L\) doublet, and in the product between a doublet and a triplet we do not get a singlet. 
Also, this term would have nonzero hypercharge, so it would not be \(U(1)_Y\) invariant either. 

However, if we consider \(\phi \phi \) we get a doublet times a doublet, and \(2 \times 2 = 1 + 3 \). 

\todo[inline]{Does the spin algebra really work just like that?}

So, we can add to the Lagrangian a term like 
%
\begin{align}
\Delta \mathscr{L} = 
y^{M}_{ij}
\qty(L^{i}_{a \alpha } \epsilon_{ab } \phi_{b })
\qty(L^{j}_{c \beta } \epsilon_{cd } \phi_{d })
\epsilon_{\alpha \beta }
\,,
\end{align}
%
where \(i, j\) are flavour indices, \(\alpha , \beta \) are Weyl spinorial indices (from 1 to 2), while \(a, b\) are \(SU(2)_L\) indices.

The tensor \(y^{M}_{ij}\) contains the Yukawa couplings, and it is adimensional. 

If we dimension-count we find a dimension of 5, so we must divide by a mass: this will be a new mass parameter \(M\), so that the term will be 
%
\begin{align}
y^{M}_{ij} \frac{L^{i} \phi L^{j} \phi }{M}
\,.
\end{align}

Then, after SSB we substitute \(\phi \) with its VEV to get 
%
\begin{align}
M^{\nu }_{ij} = y^{M}_{ij} \frac{v^2}{2M}
\,.
\end{align}

The left handed neutrino \(\nu_{L}\) is associated with its antiparticle, \(\qty(\nu^{c})_{R}\). 
If we draw a Feynman diagram for this interaction, we will see a vertex with two fermionic lines coming towards it and two scalar ones. 
Both of the fermionic lines will have arrows pointing towards the vertex, since neither of them is conjugated! This shows lepton number violation graphically. 
\begin{figure}[ht]
\centering
\feynmandiagram[layered layout, horizontal = x to l2 ]{
l1 -- [fermion] x -- [anti fermion] l2, 
a [crossed dot] -- [scalar] x,
b [crossed dot]  -- [scalar] x,
};
\caption{}
\label{fig:majorana-lepton-number-violation}
\end{figure}

This is a so-called \textbf{Majorana mass term}.
It also entails lepton number violation. 

% So, since \(v \sim \SI{100}{GeV}\) and \(m_{\nu } \lesssim \SI{1}{eV}\) we must have \(y_{\nu } \sim \num{e-11}\).

% The Dirac mass term is 2+2 degrees of freedom. 

% If \(L\) is a lepton doublet, we can write a term \(L \phi \).

% However, \(L\) has an index: but we can consider a term 
% %
% \begin{align}
% L^{i} \phi L^{j} \phi \epsilon_{ij}
% \,.
% \end{align}

% We can do this for the neutrinos but not for the other particles since they are uncharged. 

% So the term looks like 
% %
% \begin{align}
% \nu_{L} \phi^{0} \nu_{L} \phi^{0}
% \,.
% \end{align}

% When we compute the VEV we get a mass for the neutrino. 

% This is for the left-handed neutrino. The dimension of this term is 5, since the fermions have dimension \(3/2\), while the Higgs has dimension 1: we should put a mass term in the denominator.
% We will then have a term which looks like 
% %
% \begin{align}
% y^2_{\nu } \frac{L L \phi^2}{M}
% \,,
% \end{align}
% %
% so that in the end \(m_{\nu } = y^2_{\nu } v^2 / M\).

% This \(M\) must be very large in order to account for the small neutrino mass.

By orders of magnitude, the Majorana mass term reads 
%
\begin{align}
M_{ij}^{\nu } \sim \frac{y^{M}_{ij}v^2}{M}
\,,
\end{align}
%
and we know that \(v \sim \SI{100}{GeV}\). If we assume \(y \sim \num{e-1} \divisionsymbol \num{e-2}\) we find \(M \sim \num{e13} \divisionsymbol \SI{e14}{GeV}\). 
If, on the other hand, we assume \(M \sim v\) then we must have \(y \sim \num{e-13}\). Either \(y\) is very small or \(M\) is very large, or some combination of the two.  

We also have the \textbf{seesaw mechanism}. 
If we add a right-handed neutrino to our model, we could also get terms like \(\nu_{R} \nu_{R}\) (more precisely, \(\nu_{R, \alpha } \nu_{R, \beta } \epsilon_{\alpha \beta }\)): the right-handed neutrino is a singlet with respect to all the gauge symmetries (it being a singlet with respect to hypercharge follows from the fact that it is one with respect to \(SU(2)_L\) and it is uncharged).
This is why it is called a ``sterile'' neutrino, it is not affected by gauge transformations. 

The term \(\nu_{R} \nu_{R}\) is Lorentz and gauge invariant; like the \(\nu_{L} \nu_{L}\) term it explicitly breaks lepton number conservation. 

So, if we add to our Lagrangian a Majorana mass term for the right-handed neutrino:  
%
\begin{align}
\Delta \mathscr{L} = \frac{1}{2} M_{ij} \nu^{i}_{R, \alpha } \nu^{j}_{R, \beta }\epsilon_{\alpha \beta }
\,,
\end{align}
%
\todo[inline]{BTW, why are we always antisymmetrizing the neutrinos' spinorial indices?}

we have no issues related to SSB of the \(SU(2)_L \times U(1)_Y\) gauge symmetry: \(M_{ij}\) can be at any energy scale it likes. 

This is a sort of seesaw model: the mass matrix (on both axes we have \(\nu_L\) and the \(\nu_{R}\)) looks something like 
%
\begin{subequations}
\begin{align}
\left[\begin{array}{cc}
0 & m _{\text{Dirac}} \\ 
m _{\text{Dirac}} & M
\end{array}\right]
\,.
\end{align}
\end{subequations}

% The thing is: \(\nu_{R}\) has no gauge numbers. 
The mass of the right-handed neutrino is not ``protected'' by the gauge symmetry, it can be as large as it likes. 
This term then gives left-handed neutrinos very small masses by giving right-handed neutrinos very large ones. 

% If neutrinos possess a Majorana mass, 

It is important to find out whether the mass of the neutrino is due to a Majorana or Dirac mass term.

The experiments which can determine this are called \emph{neutrinoless double beta decay}: a double \(\beta \) decay in which the two neutrinos produces would annihilate violating lepton number conservation.

If this were found, it would mean that neutrinos have a Majorana mass term.
This concludes the discussion of the particle physics standard model. Next week we are going to start looking at the standard model of cosmology.

\end{document}
