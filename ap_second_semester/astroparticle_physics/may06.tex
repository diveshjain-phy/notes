\documentclass[main.tex]{subfiles}
\begin{document}

\marginpar{Wednesday\\ 2020-5-6, \\ compiled \\ \today}

Today we will discuss the \textbf{thermodynamics of the early universe}. 

We start off by discussing particles in thermal equilibrium, but the interesting thing is the transition between this equilibrium and non-equilibrium.

We use the usual approximation of a dilute gas, we take a Boltzmann distribution function \(f(p)\), from which we can compute \(n\), \(\rho \) and \(P\).

We can make some useful approximations in the relativistic limit \(T \gg m\) and in the nonrelativistic limit \(T \ll m\).

\todo[inline]{Add reference to Fundamentals notes, chapter 3}

A very important parameter is \(g_{*}\): it is the \emph{effective} number of degrees of freedom: it is computed as 
%
\begin{align}
g_{*} = \sum _{i \in \text{bosons}} g_{i} \qty(\frac{T_{i}}{T})^{4} + \frac{7}{8} \sum _{i \in \text{fermions}} g_{i} \qty( \frac{T_i}{T})^{4}
\,,
\end{align}
%
since we need to weigh the contributions to the degrees of freedom according to whether the particles are in equilibrium or not, and when the particle decoupled. 

Let us compute \(g_{*}\) in a couple of examples. First, let us consider \(T \ll \SI{}{MeV}\). 

So, in the standard model we only have photons and the three neutrino species. So, we get 
%
\begin{align}
g_{*} (T\ll\SI{}{MeV}) = 2 + \frac{7}{8} \times 3 \times 2
\,,
\end{align}
%
and we must ask: are these particles still coupled at the time we are considering?
The temperature of the neutrinos is given by 
%
\begin{align}
T_{\nu } = \sqrt[3]{ \frac{4}{11}} T_{\gamma }
\,,
\end{align}
%
so in the end we get \(g_{*} \approx 3.37\).

If we consider a temperature around \(\SI{1}{MeV} < T < \SI{100}{MeV}\), we also will have to account for (?).

In the era of radiation domination, we get 
%
\begin{align}
a(t) \propto \sqrt{t}
\,,
\end{align}
%
and the Hubble parameter scales like 
%
\begin{align}
H \approx 166 \sqrt{g_{*}} \frac{T^2}{M_{\text{Pl}}} = \frac{T^2}{M _{\text{Pl}}}
\,.
\end{align}

So, we get a time of the order of 
%
\begin{align}
t \approx \qty( \frac{T}{\SI{}{MeV}})^2 \SI{}{s}
\,.
\end{align}

How do we quantitatively describe coupling? We take the rate \(\Gamma_{i}^{\nu }\) from our particle to another, considering all possible processes.

Let us suppose that \(T > M_W\), so roughly \(T > \SI{100}{GeV}\).

The rate is given by 
%
\begin{align}
\Gamma_{\nu } = \sigma n v 
\,,
\end{align}
%
and the physics is really given by the cross section \(\sigma \): the other factors are always roughly the same, \(n \sim T^{3}\) and \(v \sim 1\).

We have the coupling constant 
%
\begin{align}
\frac{e^2}{4 \pi } = \alpha _{\text{em}}
\,,
\end{align}
%
and the charge is \(e = g \sin(\theta_{w})\), where we know experimentally that \(\sin^2\theta_{w} \approx \num{.24}\).

So, we know that the scaling is 
%
\begin{align}
\sigma \sim \qty( \frac{g^2_{w}}{4 \pi })^{2}
\sim \qty(\alpha^2_{w})
\,,
\end{align}
%
and on the denominator we need a mass square: but the only energy scale we know of here is \(T\), so we finally get \(\sigma \sim \alpha^2_{w} / T^2\). So, in the end we find 
%
\begin{align}
\sigma \sim \frac{\alpha^2_{w}}{T^2} T^3 \times 1 \sim \alpha^2_{w} T
\,.
\end{align}

The result we get is: \(T > H\) when 
%
\begin{align}
T < \frac{\alpha^2_{w }M _{\text{Pl}}}{\sqrt{g_{*}}}
\,.
\end{align}

In another case (which?) we have \(\sigma \sim T^{2}\), so \(\Gamma \sim T^{5}\). 

We must compare to the expansion of the universe. 

The end result is \(\Gamma_{\nu } > H\) as long as \(T > \SI{1}{MeV}\), roughly. So, for larger temperatures the neutrinos are coupled, for smaller temperatures they are decoupled. Below \SI{1}{MeV} they completely decouple. 

The ``relic neutrinos'' are colder than the relic photons. Moreover, their number density is smaller. 
The entropy density is given by 
%
\begin{align}
s = \frac{\rho + P}{T} = \frac{4}{3} \frac{\rho}{T}
\,.
\end{align}

The temperature of neutrinos today is around \(T^{0}_{\nu } \approx \SI{1.96}{K}\). 

Next week we will consider nucleosynthesis. 

\end{document}
