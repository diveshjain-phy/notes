\documentclass[main.tex]{subfiles}
\begin{document}

\subsection{Axions}

\marginpar{Wednesday\\ 2020-5-20, \\ compiled \\ \today}

The search for WIMPs can be indirect or direct: direct searches look for DM interacting with a target with a large cross section, indirect searches look for the recoil of particles. 

There might be a DM ``wind'' passing through the Earth. We would then expect a seasonal modulation of this effect, because of the Earth's orbit around the Sun.

Is there a modulation of this kind in the neutrino flux? Yes, at Gran Sasso they have shown this with very high confidence. 

We have our SM particles and the ``Dark Sector'': there might be a ``portal'' between them, for instance the Higgs boson.

Suppose the Dark Sector had a \(U(1)\) symmetry: then, we would have a sort of ``dark photon''. 
If this symmetry were spontaneously broken, we could then have a mixing between the two sectors.

A particular light pseudoscalar DM candidate is called the \textbf{axion}. 
Its interest is not only as a DM candidate, but also in other contexts: it is linked to BTSM physics below the electroweak scale.

It was introduced by particle physicists first, and then it was understood that it might be useful as DM.

Weak interactions violate CP symmetry. This is due to an effect related to the electroweak interaction. 

In the QCD Lagrangian we have a kinetic term \(G_{\mu \nu }^{a} G^{\mu \nu }_{a}\); then we can also add a term \(G_{\mu \nu } \widetilde{G}^{\mu \nu }\), where \(\widetilde{G}\) is the Hodge dual of the field strength.
This would produce a CP-violating term. We can write it but people were not worried: this term can be written as a 4-divergence, so it could be removed.

However, it could be shown that due to instantons --- nonperturbative quantum effects --- there was an anomalous current, the term could not be neglected. 

% \todo[inline]{So, it could not be written as a divergence anymore?}
The term can still be written as a divergence, but because of the instanton the asymptotic values of the field are not zero anymore: therefore, the surface integral in the action is not zero anymore. 

Then, a term \(\theta_0   G \widetilde{G}\) was added. 
This also affected the phases of quark transitions. 
Then, we introduce a term 
%
\begin{align}
\overline{\theta} = \theta_0  + \arg \qty(\det M)
\,,
\end{align}
%
where \(M\) is the quark mass matrix. This \(\overline{\theta}\) will then multiply the \(G \widetilde{G}\). This is a free parameter of the theory: the quark mass matrix depends on the Yukawa couplings, and \(\theta_0 \) is completely free.
This should be \(\overline{\theta} < \num{e-9}\) to comply with experiment: it looks like fine-tuning!

There is no physical reason why the parameter should be small.
This is weird: there must be some hidden symmetry.

We can introduce a \(U(1)\) global symmetry, called the Pacci-Quinn symmetry. 
This is discussed by Rubakov.

Since the field \(H\) feels the symmetry, in terms like \(\overline{Q}_{L} H d_R\). 
The spontaneous breaking of this symmetry yields a goldstone boson called the axion. 

The mass of this pseudo-Goldstone boson depends on the scale at which this symmetry is broken, \(Q\). 

% We will have terms like \(a F^{\mu \nu } F_{\mu \nu }\) if it coupled to photons.

We can introduce additional scalars, which are singlets or doublets under SU(2), whose VEV is much larger than the electroweak scale.

If the photons inside a star could convert into axions, they could speed up the cooling down of a star: our stellar evolution observations then allow us to give a bound
%
\begin{align}
f_{PQ} > \SI{e9}{GeV}
\,.
\end{align}

What happens to the axions after they are produced? The axion is light, since \(m_a \sim 1 / f_Q\). The mass is small, like \(< \SI{e-5}{eV}\).
They oscillate, moving around the minimum in their potential.
Most off the energy is contained in this oscillation, the mass contributes very little. 

This is then a type of DM which is Cold, but also very light!
In order to not over-close the universe we must also have \(f_{PQ} < \SI{e12}{GeV}\).

How would we detect axions? They interact with photons, so we can have an experiment in which we have photon conversion into an axion.

We have a spectacular effect which is called ``light shining through a wall'': we send a photon towards a wall, it will not pass through usually. 
Suppose that we put a source of a strong magnetic field: if there are axions, we could have photon-photon conversion into an axion, which can pass through the wall and then emerges as an axion. We then put another strong source of magnetic field which can deconvert the axion into a photon.

So far, this has not provided any evidence for the existence of the axion.

\end{document}
