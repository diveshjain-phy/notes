\documentclass[main.tex]{subfiles}
\begin{document}

\subsection{Neutral current interactions}

\marginpar{Tuesday\\ 2020-4-28, \\ compiled \\ \today}

We have introduced a term in our model which yields an important and testable prediction: the existence of a weak \emph{neutral} current in addition to the charged ones.
Let us make some simple examples for the three types of currents we have: the charged current with the \(W^{\pm}\) bosons mediates interactions like 
%
\begin{align}
\nu_{L} + d_L \to e^{-}_{L} + u_L
\,,
\end{align}
%
the electromagnetic current \(A_{\mu }\) and the neutral current \(Z^{0}_{\mu }\) mediate interactions like 
%
\begin{align}
e^{-}_{L} + u_L \to e^{-}_{L} + u_L
\,,
\end{align}
%
however there is a distinction to be made: the electromagnetic current can only be coupled to charged particles, so a process like 
%
\begin{align}
\nu_{L} + u_L \to \nu_{L} + u_L
\,
\end{align}
%
can only be mediated by the \(Z^{0}\) boson, since the neutrino is neutral!
This kind of interaction has been experimentally observed!

\todo[inline]{So, there are certain processes which can be mediated by both the \(Z^{0}\) boson and the photon. Does this mean that people needed to revise their estimate of the electromagnetic coupling, since electromagnetism was not the only contributor to those scatterings?}

\section{The Standard Model of particle physics}

The Lagrangian reads:
%
\boxalign{
\begin{align}
\mathscr{L} = - \frac{1}{4} F_{\mu \nu } F^{\mu \nu }
+ i \overline{\psi} \slashed{\DD} \psi  
+ \psi_{i} y_{ij} \psi_{j} \phi 
+ \text{h.\ c.} 
+\abs{\DD_{\mu } \phi }^2 - V(\phi )
\,.
\end{align}}

Let us unpack the terms: 
\begin{enumerate}
    \item \(F_{\mu \nu } F^{\mu \nu }\) encompasses the kinetic and self-interaction terms of all the interaction bosons (photon, \(W^{\pm}\), \(Z^{0}\), gluons);
    \item \(i \overline{\psi} \slashed{\DD} \psi \) encompasses all the kinetic terms of the fermions (\(\overline{\psi} \gamma^{\mu } \partial_{\mu } \psi  \)) and the interaction terms between the fermions and the gauge bosons: \(g \overline{\psi} \gamma^{\mu }A_{\mu } \psi \);
    \item \(\psi_{i} y_{ij} \psi_{j} \phi \) is the interaction between the Higgs field and the fermions, which gives mass to the fermions and which we will discuss in the next section;
    \item \(\abs{\DD_{\mu } \phi }^2\) is the kinetic term of the scalar (Higgs) field \(\phi \), also including the interactions between it and the vector bosons;
    \item \(V(\phi )\) is the potential for the scalar field, it must be at least quartic.  
\end{enumerate}

``h.\ c.'' means Hermitian Conjugate, it is needed in order to ensure that everything is self-adjoint. 

The symmetry which determines this theory is 
%
\begin{align}
\text{Lorentz} \otimes 
SU(3)_{c}  \otimes
SU(2)_{L} \otimes
U(1)_{Y}
\,.
\end{align}

Starting from the symmetry, the only other prescription needed are the quantum numbers of the various particles: we can make a table. 
As we have seen, the fermions are divided into left-handed isospin doublets and right-handed singlets. 

\begin{figure}[H]
\centering
\begin{tabular}{rcccc}
& Name & \(T_3\) (SU(2)) & \(Y\) & \(Q\) \\
\cline{2-5}
\ldelim\{{4}{*}[Leptons (no \(SU(3)\) charge)] &
\(\nu_{L}\) & \(+ 1 / 2\) & \(- 1/ 2\) & 0  \\
& \(e^{-}\) & \(- 1/ 2\) & \(- 1/ 2\) & \(-1\)  \\
& \(\cancelto{}{\nu_{R}}\) & 0 & 0 & 0 \\
& \(e^{-}_{R}\) & 0 & \(-1\) & \(-1\) \\
 \\
\ldelim\{{2}{*}[Higgs scalar] &
\(\phi^{+}\) & \(+1 / 2\) & \(+ 1/2\) &  \(+1\) \\
&\(\phi^{0}\) & \(-1 / 2\) & \(+1 /2 \) & \(0\) \\
 \\
\ldelim\{{4}{*}[Quarks (\(SU(3)_c\) triplets)] &
\(u_L\) & \(+ 1/ 2\) & \(1 / 6\) & \( 2 / 3\) \\ 
& \(d_L\)& \(- 1/ 2\) & \(1 / 6\) & \(- 1/ 3\) \\
& \(u_R\)& \(0\) & \(2/3\) & \(2/ 3\) \\
& \(d_R\)& \(0\) & \(- 1/3\) & \(- 1/ 3\) 
\end{tabular}
\label{tab:particles}
\caption{Particles}
\end{figure}

Here we write only one of the generations for both leptons and quarks.
The quantum numbers for the other generations are the same.
An unanswered question is \emph{why} exactly there are three families.

We are always computing the electric charge as \(Q = T_3 + Y\), where \(T_3\) is the generator of \(SU(2)_L\) rotations around the \(\hat{3}\) axis.
Sometimes we write ``1'' for the right-handed quarks under \(T_3 \), this is not the eigenvalue but it is used to mean ``singlet''.

For the \(SU(3)\) we have 8 generators (corresponding to the gluons), for \(SU(2)\) we have three generators, for \(U(1)\) we only have 1 generator: these four, after SSB, are three massive bosons and the photon.

The assignment of the charges cannot come from the symmetry, it must be determined experimentally.

The right-handed neutrino does not appear here: it should be invariant under any gauge transformation. This would then be the \emph{sterile} neutrino: it would not interact with matter by any of the forces of the Standard Model.

It makes sense not to include it \emph{a priori} then, however we can keep it in mind: it is a DM candidate.

Can we say anything about its possible mass? There are schemes in which the RH neutrino enters, and in which it can be assigned a mass. However, in general this is a free parameter: in some contexts we can give it a specific value, but we cannot say anything about its mass from basic Standard Model theory.

There are experiments at Fermilab which could point indirectly at their existence.

The rule is always: 
\begin{quote}
    \emph{Write all the terms in your Lagrangian which are invariant under your symmetry and which have dimension \( 4\).}
\end{quote}

Should we not include also some prescription for the couplings? 
No, this is included in the symmetry: if we have the coupling constants \(g_S\), \(g\) (weak) and \(g'\) (hypercharge) we can determine everything else.

The issue is that, before including the Higgs boson, everything is massless.

Gluons and quarks have the same ``destiny'': there is confinement when \(g_S\) becomes large.

% The \(SU(2)_L \times U(1)_{Y}\) symmetry is spontaneously broken to \(U(1)_{\text{em}}\). If we had 4 massless DoF before, they must still be there but they have become massive.

% These three particles would have become Goldstone bosons, but instead they become the longitudinal component of the three vector bosons.

The charged Higgs component \(\phi^{+}\) gives mass to the \(W^{\pm}\) vector bosons of the electroweak interaction, the \(\phi^{0}\) gives mass to the \(Z^{0}\) vector boson. 

% So, we know how to write the kinetic terms \(F^{\mu \nu } F_{\mu \nu }\), and the mass terms \(\overline{\psi} \slashed{D} \psi \). 

\subsection{Fermions' mass terms}

We have the massless \(G^{a}\) and \(A^{\mu }\), and the massive \(W^{\pm}\) and \(Z^{0}\). These are the terms we've written so far: where do the \(\overline{\psi} m  \psi \) terms come from?



The kinetic term has dimension \(2^2\), which is fine, the wavefunction has dimension \(3/2\) while the derivative has dimension \(1\).
So, also a term \(\overline{\psi} m \psi \) would be fine dimensionally. Is it symmetric? 
Let us consider the issue with the left- and right-handed components. If we were to write something like \(\overline{\psi}_L \psi_L \) or \(\overline{\psi}_{R} \psi_{R}\) we would get 
%
\begin{align}
\overline{\psi}_{L/R} \psi_{L/R} = \overline{\psi} 
\qty(\frac{1 \mp \gamma_5 }{2}) 
\qty(\frac{1 \pm \gamma_5 }{2}) 
\psi 
= 0
\,,
\end{align}
%
since we are computing the product of two orthogonal projectors. 

So, the only terms which do not vanish look like \(\overline{\psi}_{L} \psi_{R} = \overline{\psi}_{R} \psi_{L} = \overline{\psi} \psi \).

\todo[inline]{So he writes, but it does not work! We have two different components, \(\overline{\psi} \psi \) and \(\overline{\psi} \gamma^{5} \psi \)!}

Let us consider electrons for example: an object like \(m \overline{e}_{L} e_R\) is a Lorentz invariant, but is it also gauge invariant?

The spinor \(e_L\) is in an \(SU(2)\) doublet, while the \(e_R\) is an \(SU(2)\) singlet.
So, the object is not a singlet: it is not invariant under \(SU(2)_{L}\), but each term in the Lagrangian must be a scalar with respect to the symmetry group (that is, it must be \emph{symmetric}). 
In fact, this term is only \(U(1) _{\text{em}}\) invariant, neither its isospin nor its hypercharge are zero.

So, we cannot include this term.
Does this mean that fermions are massless?
Before introducing \(\phi \), in the pure Yang-Mills theory, they indeed are.

There is only one kind of field which is invariant under spacetime transformations: a scalar. So, we could introduce a term like 
%
\begin{align}
\overline{\psi}_{L}^{i} \phi_{i} \psi_{R}
\,,
\end{align}
%
where \(i\) is an index going from 1 to 2, an \(SU(2)\) index: \(\phi \), the Higgs field, is an \(SU(2)_L\) doublet!
This is not a mass term, but an interaction term between the Higgs field and the electron-left and electron-right.

Then, the way the mechanism works is by the fact that the VEV of \(\phi \) is \(v \neq 0\), so at low energies we will see an effective mass term.
The term we put in the Lagrangian will look like 
%
\begin{align}
y^{ij} \overline{\psi}^{i}_{L} \phi \psi_{R}^{j}
\,,
\end{align}
%
where now the indices \(i\) and \(j\) run over the possible fermions.
We will then have 
%
\begin{align}
M_\ell = y_{\ell} v 
\,,
\end{align}
%
and the matrix will be \(y^{ij} = \delta^{ij} y^{j}\).

Similarly we will have mass terms for the leptons, up and down quarks.
These are free parameters of the theory, and they must be different. 
The VEV of the Higgs field is fixed by the masses of the \(W\) and \(Z\) bosons: it is of the order \SI{100}{GeV}.

But the mass of the electron is \(m_e \sim \SI{511}{keV}\): so, we must have \(y_e \sim \num{e-5}\). 

The price we pay for this is the fact that the number of free parameters increases. This is a ``dirty part'' of the theory.

Why are these parameters so different? This is the \emph{flavour problem}.

\todo[inline]{Stuff is not super clear here.}

Let us write the most general set of Yukawa couplings \cite[eq.\ 18.24]{peskinConceptsElementaryParticle2019}: 
%
\begin{align}
\mathscr{L} _{\text{coupling}}
= - y^{ij}_{e} L ^{\dag, i}_{a} \phi_{a} e^{j}_{R}
+ y^{ij}_{d} Q ^{\dag, i}_{a} \phi_{a} d^{j}_{R}
- y^{ij}_{u} Q ^{\dag, i}_{a} \epsilon_{ab} \phi_{b}^{*} u^{j}_{R} 
+ \text{h.\ c.}
\end{align}

The matrices \(y^{ij}_{f}\) are called the Yukawa matrices. The indices \(i \) and \(j\) label fermion generations (1, 2, 3 for electron-lik, muon-like and tau-like).
They are not Hermitian in general, however we can construct 
%
\begin{align}
y_f y_f ^\dag = U_L^{(f)} Y_f U_L^{(f), \dag} 
\qquad \text{and} \qquad
y_f ^\dag y_f = U_R^{(f)} Y_f U_R^{(f), \dag}
\,,
\end{align}
%
and if the matrix \(\sqrt{Y_f}\) is block-diagonal in generation space then we can write 
%
\begin{align}
y_f = U_L^{(f)} \sqrt{Y_f} U_R^{(f)\dag}
\,.
\end{align}

So, if we change variables as 
%
\begin{align}
e^{i}_{R} \to U^{(e)}_{R, ij} e^{j}_{R}
\qquad \text{and} \qquad
L^{i} \to U^{(e)}_{L, ij} L^{j}
\,,
\end{align}
%
the matrices \(U_{L, R}\) disappear from the couplings.

This works well for the leptons; for the quarks there is an issue, in their coupling to the \(W\) boson we are left with a term 
%
\begin{align}
u ^\dag_{L} (i \overline{\sigma}^{\mu }) d_L &= u ^\dag _L U_L^{(u)\dag} (i \sigma^{\mu }) U_L^{(d)} d_L  \\
&= u ^\dag_L (i \overline{\sigma}^{\mu }) \underbrace{U_L^{(u)\dag}U_L^{(d)}}_{V_{CKM}} d_L
\,.
\end{align}

\todo[inline]{But why does this happen? what is it about the quarks that makes those matrices not cancel?}

\todo[inline]{This all has to do with the fact that mass eigenstates and flavour eigenstates are different, methinks.}

% We could have a term which looks like 
% %
% \begin{align}
% \overline{u}_{L} \gamma^{\mu } d_{L} W_{\mu }
% \,.
% \end{align}

% Thi is not the most general thing: we could write 
% %
% \begin{align}
% \overline{u}_{L} U_{L} ^{\dag, (u)} \gamma^{\mu } U_{L}^{(d)} d_{L} W_{\mu }
% \,.
% \end{align}
% %
% The product \(U ^\dag U\) is called \(V_{CKM}\), the Cabibbo matrix, which gives us \(CP\) violation in the Standard model.

% It does seem, though, that the amount of CP violation in the Standard Model is too small to satisfy the Sacharov conditions.

% There are many parameters: the couplings, the mass terms inside \(y^{ij}\), the terms \(\mu \) and \(\lambda \) inside of \(V(\phi )\).

In the \(V_{CKM}\) matrix we have the freedom to choose three (Euler) angles (it is a rotation matrix basically) and a phase.

\end{document}
