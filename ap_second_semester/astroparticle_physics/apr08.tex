\documentclass[main.tex]{subfiles}
\begin{document}

\subsection{Quantum Chromo Dynamics}

\marginpar{Wednesday\\ 2020-4-8, \\ compiled \\ \today}

Asymptotic freedom is not a political slogan from the sixties!

With this lecture and the next we should finish discussing the standard model of particle physics. 

Today we are going to explore a property which is profoundly different between EM and the strong interaction. 

We introduced the quantity \(\alpha_{EM} = e^2 / 4 \pi \): how do we measure this parameter?
We know that the interaction term looks like 
%
\begin{align}
\mathscr{L} \sim e j^{\mu }A_{\mu }
\,,
\end{align}
%
so we can measure \(e\) using the cross sections of processes, such as \(e^{+} e^{-} \to \mu^{+} \mu^{-}\); the free parameters are just \(e\) and the COM energy, so we are done. 
But how do we do it without an accelerator? We could do the Millikan drop experiment, for example. 

But we are doing Quantum Field Theory: beyond physical particles we can also have virtual particles, which despite the name can influence physical processes: inside the ``black box'' within which the interactions occur we can have off-shell processes. 

In fact, in the aforementioned decay the photon is off mass shell; but we can go beyond. The photon can create and then annihilate an \(e^{+} e^{-}\) pair. 

The vacuum is a \emph{quantum} vacuum, which does correspond to a minimal energy but it is filled by the continuous creation and annihilation of these particle-antiparticle pairs, since there is an indetermination between time and energy. 
It is important that after being created these pairs are indeed destroyed.
This is called \emph{vacuum polarization}

The pairs are virtual but they have an effect. The coupling constant \(\alpha \) has a numerical value of around \(1 / 137\), and the loop diagrams of increasing number of loops depend on increasing powers of \(\alpha \). 
The fact that \(\alpha \) is small allows us to work perturbatively. 

This works, the predictions of QED are extremely precise and correspond to experiment.
For instance, the anomalous magnetic moment of the electron: in a QFT, when an electron interacts with a photon, there can be other particles. 

The polarization of the vacuum creates a screening effect. 
So, we expect the effective \(\alpha_{EM}\) to be larger. 
We compare the Millikan experiment (\(E \sim 0\)), where we find \(\alpha \sim 1/137\), with LEP: now \(E \sim \SI{100}{GeV}\), and we find \(\alpha \sim 1/128\)!

What is the theoretical relation between \(\alpha \) and \(Q\)? It comes out to be 
%
\begin{align}
\alpha (Q) = \frac{\alpha_0 }{1 - \frac{2 \alpha_0 }{3 \pi } \log(\frac{Q}{Q_0 })}
\,,
\end{align}
%
where \(Q\) is the energy at which we measure. 

What happens if we compute this for the strong interaction? We will have 
%
\begin{align}
\alpha_3 = \alpha _{\text{strong}} = \frac{g^2_{s}}{4 \pi }
\,.
\end{align}

The problem is that this is a large number: but our tools are perturbative! How do we compute the cross sections then? 
The thing is, with increasing energy this \(\alpha \) becomes \emph{smaller}! 
Why is this? The gluons are self interacting, because of the \(A \wedge A\) term in the field strength of the strong interaction, as opposed to photons. 

In the EM case, we had \(\gamma \to e^{+} e^{-} \to \gamma \) as a virtual particle process; in the chromodynamics case instead we have \(g \to g + g \to g \): it is the case (even though it is not easy to prove) that this loop contribution has the opposite sign! We have 
%
\begin{align}
\dv{g_{s}}{\log Q} = \beta (g_s) 
\qquad \text{where} \qquad
\beta = - \frac{11}{3} C(G) +  \frac{4}{3} n_f \frac{1}{2}  
\,,
\end{align}
%
where \(C(G)\) is the Casimir of the adjoint representation, so for QCD it is 3, while \(n_f\), the number of fermions, is 6. Then, the final result is negative! 

This is called asymptotic freedom: \(\alpha \) goes to zero with \(Q\). 
If the energy is high enough, the quarks and gluons are effectively decoupled. 

So, at the low energies we look at in atomic nuclei the quarks are confined: no one has ever observed a free quark, not even at LHC. 
This is called hadronization, since the quarks are always bound into hadrons. ALICE, at LHC, is looking for the phase transition between this ``infrared slavery'' of the quarks and their free state. 

Since \(\alpha_{S}\) decreases, while \(\alpha_{EM}\) increases, there should be a point at which they cross and the electromagnetic interaction becomes stronger than the strong one. 

Actually, it might be possible that all of the interactions are manifestations of the same kind of interaction. 

The crucial point is that the strength of an interaction depends on the energy. 

We know that the time needed for hadronization is of the order \(\tau _{\text{had}} \sim \SI{e-23}{s}\), while the time for a pion's decay is \(\tau (\pi^{+}) \sim \SI{3e-8}{s}\). 
The time for beta decay is \(\tau (n \to p + e^{-} + \overline{\nu}_{e}) \sim \SI{880}{s}\). 

This can give us a first signal for the fact that the weak interaction is \emph{weak}. 

We know that the neutron is \(u d d \) in terms of quarks, while the proton is \(u u d\): so, in the beta decay one down quark must become an up quark. 
There is no term in QED or QCD in which this can happen: QED and QCD conserve flavour.

But we do see beta decay: so, we must introduce a new \emph{weak interaction}.

Beta decay is experimentally observed to be a three-body process, which is why observations of it were the first indication of the existence of neutrinos. 
Beta-decay is not \(P\)-symmetric; on the other hand, QED and QCD are.

\end{document}