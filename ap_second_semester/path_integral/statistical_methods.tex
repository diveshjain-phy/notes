\documentclass[main.tex]{subfiles}
\begin{document}

\section{Statistical methods in cosmology}

% Ergodicity: temporal and spatial averages coincide
% Fair Sample Hypothesis: the universe is homogeneous and isotropic

From \cite[sec.\ 4]{nataleNoteCorsoDi2017}.

Two point correlation function: 
%
\begin{align}
1 + \xi (r_{12} ) = \frac{ \dd{P}}{ \dd{P} _{\text{indep}}} = \frac{ \dd{P}}{ n^2 \dd{V_1} \dd{V_2}}
\,.
\end{align}
%

Fractal dimension! 
The number of galaxies within a radius \(R\) around a given one scales like \(R^{3-\gamma }\).

Hierarchical models: \(N\)-point correlation functions can be calculated from the two-point one. 

Bias model: \(\delta _g = b \delta \) with constant \(b\), where \(\delta_g \) is the density perturbation for galaxies and \(\delta \) is the one for dark matter.

Power spectrum definition, which by Wiener-Khinchin is the Fourier transform of the two-point correlation function. 
Expression for \(\xi \) in terms of \(P\) and Bessel functions as a single integral.

\end{document}