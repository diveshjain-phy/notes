\documentclass[main.tex]{subfiles}
\begin{document}

\section{Applications}

Definitions from \cite[]{matarresePathintegralApproachLargescale1986}. 

The second connected correlation function, computed with a smoothing at a scale \(R\), can be used to define a variance: \(\xi^{(2)} (x, x) \sim \sigma _R^2\). 
With this we can then quantify ``peaks'' in the density perturbation \(\delta (x) = (\rho - \expval{\rho }) / \expval{\rho }\). 

Specifically, we define a function \(\rho_{\nu , R}\) so that it is equal to \(1\) when the perturbation is \(\nu \) sigmas above the average (which is \(\delta = 0\) by definition), and \(0\) otherwise: 
%
\begin{align}
\rho_{\nu , R} = \Theta (\delta (x) - \nu \sigma _R)
\,.
\end{align}

We can define an \(N\)-point correlation for this ``boolean density'': 
%
\begin{align}
\Pi^{(N)}_{\nu , R} (x_1 \dots x_N) = \expval{\prod_{r=1}^{N} \rho_{\nu , R}(x_r)} 
\,
\end{align}
%
quantifies the probability that the perturbation is above the threshold at all the positions \(x_r\). 
We then define the \emph{peak} correlation function 
%
\begin{align}
\xi^{(N)}_{\text{dis}, \nu , R} = \expval{\frac{\prod_{r=1}^{N} \rho_{\nu , R} (x_r)}{\expval{\rho_{\nu , R}}^{N} }} - 1
\,.
\end{align}

% \todo[inline]{understand its physical meaning}

According to equations 16 and 22 in \cite[]{matarresePathintegralApproachLargescale1986} the two-point correlation function with two different smoothing scales and thresholds is allowed to have zero-crossings.  

\end{document}
