\documentclass[main.tex]{subfiles}
\begin{document}

\section{Applications}

Definitions from \cite[]{matarresePathintegralApproachLargescale1986}. 

The second connected correlation function, computed with a smoothing at a scale \(R\), can be used to define a variance: \(\xi^{(2)} (x, x) \sim \sigma _R^2\). 
With this we can then quantify ``peaks'' in the density perturbation \(\delta (x) = (\rho - \expval{\rho }) / \expval{\rho }\). 

Specifically, we define a function \(\rho_{\nu , R}\) so that it is equal to \(1\) when the perturbation is \(\nu \) sigmas above the average (which is \(\delta = 0\) by definition), and \(0\) otherwise: 
%
\begin{align}
\rho_{\nu , R} = \Theta (\delta (x) - \nu \sigma _R)
\,.
\end{align}

We can define an \(N\)-point correlation for this ``boolean density'': 
%
\begin{align}
\Pi^{(N)}_{\nu , R} (x_1 \dots x_N) = \expval{\prod_{r=1}^{N} \rho_{\nu , R}(x_r)} 
\,
\end{align}
%
quantifies the probability that the perturbation is above the threshold at all the positions \(x_r\). 
We then define the \emph{peak} correlation function 
%
\begin{align}
\xi^{(N)}_{\text{dis}, \nu , R} = \expval{\frac{\prod_{r=1}^{N} \rho_{\nu , R} (x_r)}{\expval{\rho_{\nu , R}}^{N} }} - 1
\,.
\end{align}

% \todo[inline]{understand its physical meaning}

According to equations 16 and 22 in \cite[]{matarresePathintegralApproachLargescale1986} the two-point correlation function with two different smoothing scales and thresholds is allowed to have zero-crossings.  

\todo[inline]{What does \(\xi_{cc} \propto \xi_{gg}\) mean? These are the two-point correlation functions for rich cluster and galaxies. Do they correspond to \(\xi^{(N)}_{\text{dis}, \nu, R}\) for different values of \(\nu \)? What is the connection between this observable quantity and the mathematical formalism of peak correlation functions?}

The paper also finds \emph{scaling relations} between a certain order \(N\)-point peak correlation function and lower order ones, as well as the background correlation functions \cite[eqs.\ 24--25]{matarresePathintegralApproachLargescale1986}. 
The result is interesting since it is general, allowing for the computation to be performed with a general non-Gaussian background; in the Gaussian case it reduces to 
%
\begin{align}
\zeta (1, 2, 3) &= \xi (1, 2) \xi (2, 3) + \xi (1, 3) \xi (2, 3) + \xi (1, 2) \xi (1, 3) + \xi (1, 2) \xi (2, 3) \xi (1, 3)  \\
\zeta &\sim \sum \xi^2 + \xi^3
\,,
\end{align}
%
where \(\zeta \) is the three-point peak correlation function, \(\xi \)is the two point one, and we denote \(x_i \equiv i\) for compactness. 
The second expression is just a compact form for the first.

This expression does not seem to fit observational data: specifically, the \(\xi^3\) term has not been found in observations. 
The aforementioned general expression can be written as 
%
\begin{align}
\zeta = F \qty(\sum \xi^2 + \xi ^3) + (F-1) \qty(1 + \sum \xi ) 
\,,
\end{align}
%
where \(F = F(1, 2, 3) = 1\) in the Gaussian case. 

\todo[inline]{I do not understand the motivation behind equation 28 in the same paper? Is it meant to make it so the \(\xi^3\) term vanishes? }

Moving on to \cite[]{bertschingerPathIntegralMethods1987}. 
Perturbative methods work well in the initial stages, then the nonlinearity kicks in and the higher order terms become very relevant: Monte Carlo methods are then more suitable to calculate the functional integrals. 

They are able to put constrains on the realizations of the Gaussian random field! 
BBKS gives us peaks; the path integral approach allows us to more precisely estimate the surrounding  distribution, by the use of constraints. 

\end{document}
