\documentclass[main.tex]{subfiles}
\begin{document}

\section{Applications} \label{sec:applications}

% Definitions from \cite[]{matarresePathintegralApproachLargescale1986}. 

% The second connected correlation function, computed with a smoothing at a scale \(R\), can be used to define a variance: \(\xi^{(2)} (x, x) \sim \sigma _R^2\). 
% With this we can then quantify ``peaks'' in the density perturbation \(\delta (x) = (\rho - \expval{\rho }) / \expval{\rho }\). 

% Specifically, we define a function \(\rho_{\nu , R}\) so that it is equal to \(1\) when the perturbation is \(\nu \) sigmas above the average (which is \(\delta = 0\) by definition), and \(0\) otherwise: 
% %
% \begin{align}
% \rho_{\nu , R} = \Theta (\delta (x) - \nu \sigma _R)
% \,.
% \end{align}

% We can define an \(N\)-point correlation for this ``boolean density'': 
% %
% \begin{align}
% \Pi^{(N)}_{\nu , R} (x_1 \dots x_N) = \expval{\prod_{r=1}^{N} \rho_{\nu , R}(x_r)} 
% \,
% \end{align}
% %
% quantifies the probability that the perturbation is above the threshold at all the positions \(x_r\). 
% We then define the \emph{peak} correlation function 
% %
% \begin{align}
% \xi^{(N)}_{\text{dis}, \nu , R} = \expval{\frac{\prod_{r=1}^{N} \rho_{\nu , R} (x_r)}{\expval{\rho_{\nu , R}}^{N} }} - 1
% \,.
% \end{align}

% % \todo[inline]{understand its physical meaning}

% According to equations 16 and 22 in \cite[]{matarresePathintegralApproachLargescale1986} the two-point correlation function with two different smoothing scales and thresholds is allowed to have zero-crossings.  

% \todo[inline]{What does \(\xi_{cc} \propto \xi_{gg}\) mean? These are the two-point correlation functions for rich cluster and galaxies. Do they correspond to \(\xi^{(N)}_{\text{dis}, \nu, R}\) for different values of \(\nu \)? What is the connection between this observable quantity and the mathematical formalism of peak correlation functions?}

The paper also finds \emph{scaling relations} between a certain order \(N\)-point peak correlation function and lower order ones, as well as the background correlation functions \cite[eqs.\ 24--25]{matarresePathintegralApproachLargescale1986}. 
The result is interesting since it is general, allowing for the computation to be performed with a general non-Gaussian background; in the Gaussian case it reduces to 
%
\begin{align}
\zeta (1, 2, 3) &= \xi (1, 2) \xi (2, 3) + \xi (1, 3) \xi (2, 3) + \xi (1, 2) \xi (1, 3) + \xi (1, 2) \xi (2, 3) \xi (1, 3)  \\
\zeta &\sim \sum \xi^2 + \xi^3
\,,
\end{align}
%
where \(\zeta \) is the three-point peak correlation function, \(\xi \)is the two point one, and we denote \(x_i \equiv i\) for compactness. 
The second expression is just a compact form for the first.

This expression does not seem to fit observational data: specifically, the \(\xi^3\) term has not been found in observations. 
The aforementioned general expression can be written as 
%
\begin{align}
\zeta = F \qty(\sum \xi^2 + \xi ^3) + (F-1) \qty(1 + \sum \xi ) 
\,,
\end{align}
%
where \(F = F(1, 2, 3) = 1\) in the Gaussian case. 

\todo[inline]{I do not understand the motivation behind equation 28 in the same paper? Is it meant to make it so the \(\xi^3\) term vanishes? }

\todo[inline]{Is the \(\xi^3\) term predicted by Gaussian statistics but not observed?}

Moving on to \cite[]{bertschingerPathIntegralMethods1987}. 
Perturbative methods work well in the initial stages, then the nonlinearity kicks in and the higher order terms become very relevant: Monte Carlo methods are then more suitable to calculate the functional integrals. 

They are able to put constrains on the realizations of the Gaussian random field! 
BBKS gives us peaks; the path integral approach allows us to more precisely estimate the surrounding  distribution, by the use of constraints. 

In \cite[]{hoffmanConstrainedRealizationsGaussian1991} an improved algorithm for constrained realizations of Gaussian fields is presented.

\todo[inline]{To understand: what constrained realizations are practically useful? Do we impose the presence of a (proto-galaxy?) and generate the density distribution around it?} 

\textcite[]{bertschingerPathIntegralMethods1987} found a method with complexity \(\order{(N_c^2 + 1) N}\) working in Fourier space (so, in a plane wave basis), 
Binney and Quinn (1990) improved upon it by working in spherical harmonics --- 
for certain symmetric types of constrains this decreases \(N_c\) significantly. 

\textcite[]{hoffmanConstrainedRealizationsGaussian1991} work by constraining the Fourier \emph{phases} of the components of the random field. 

\todo[inline]{I don't see where \(\overline{f}(r) = \expval{f(r) | \Gamma } = \xi _i (r) \xi^{-1}_{ij} c_j\) comes from in general --- I can see it in the specific case \(C_i = f(r_i)\) through.} 

Looking at \cite[]{matarreseAbundanceHighRedshift2000}. 
They account for non-gaussianity through field (overdensity or gravitational potential) in the form \(\alpha \phi + \epsilon (\phi^2 - \expval{\phi^2 }) \), where \(\phi \) is Gaussian with zero mean. 
They use the Press–Schechter formalism, considering the regions in which the linearized density perturbation  goes beyond \(\Delta  _c \approx 1.687\).
Figure 4: the effect of adding the non-Gaussian tails is to boost by an order of magnitude the probability \(\mathbb{P}(> \delta _c | z_c, R)\), where \(z_c\) is the redshift we are considering while \(R\) is the filtering scale. 

If we perturb the density we need \(\abs{\epsilon} \lesssim \num{e-2}\), if we perturb the gravitational field we need \(\abs{\epsilon } \gtrsim \num{e2}\), in either case these are needed in order to see a significant departure from the Gaussian case. 
The latter statements tells us that non-Gaussianity predicted by inflationary models would be undetectable from structure formation. 

\todo[inline]{Does equation 51 solve the problem of halos-inside-halos which was mentioned as an issue in early Press-Schechter theory?}

Section 5: using high-\(z\) galaxies is advantageous because ``they directly sample the galaxy scale, and are always in virialized halos''. I get the fist point, but not the second: are more recent galaxies not in virialized halos? 

At higher \(z\) the density PDF is more sensitive to non-Gaussianity: the critical over-density is \(\delta _c = \Delta _c / D(z_c  | \Omega_{0, m}, \Omega_{0, \Lambda })\), where \(D\) is the linear growth factor: it seems like for most cosmological models \(D^{-1} \propto 1+z\), with an order-1 constant, from figure 3 in \cite[]{matarreseAbundanceHighRedshift2000}. 
More details should be found in \cite[]{hamiltonFormulaeGrowthFactors2001}. 

In the paper the functional integrals are explicitly performed to find 
%
\begin{align}
\mathbb{P}(> \delta _c | z_c, R) = \frac{1}{2 \pi i} \int_{- \infty }^{\infty } \frac{ \dd{\lambda }}{\lambda } \exp( - i \lambda \delta _c (z_c) + \mathscr{W}(\lambda )) + \frac{1}{2}
\,,
\end{align}
%
where \(\mathscr{W}(\lambda )\) is the \emph{cumulant generator}, while \(\lambda \) is related to the magnitude of the source term. The explicit expression for \(\mathscr{W}\) is 
%
\begin{align}
\begin{split}
\mathscr{W}(\lambda ) &=
- i \lambda C - \frac{1}{2} \int \dd[3]{y} \dd[3]{z} 
\bigg[
    \lambda^2 \alpha^2 F_R (\abs{y }) \qty[\mathscr{K}'_\lambda ]^{-1} (y, z)
    F_R(\abs{z}) + \\
    &\phantom{=}\ + \delta _D (y-z) 
    \log \qty[ \delta _D (y-z) - 2i \lambda \epsilon F_R (\abs{y}) \xi _\phi (y-z)]
\bigg]
\,,
\end{split}
\end{align}
%
where: 
\todo[inline]{Add more details of the computation, and definitions of the parameters (?)}

So, \(\delta _c\) increases with redshift, therefore we see from equation 66 that the skewness term becomes less and less relevant. 
This can be seen in figs.\ 4 and 5 as well. 

Connection with observation! 
We have a sample of 6 \(z \sim 5 \divisionsymbol 6  \) galaxies, whose masses, SFR and \(\sigma _v\) are estimated from spectroscopic data.
Approximately speaking, from the amount of galaxies found per area surveyes, at this redshift we find \(\sim 1\) galaxy per \SI{3e-4}{\deg^2}. 

Estimating the number of galaxies formed with masses within \(\num{2e10} M_{\odot} < M < \num{4e11}M_{\odot}\) and formation redshifts within \(6 < z_c < 8\) yields a number density of galaxies about 16 times lower than the observed value; non-Gaussianity can raise the tail of the density distribution to give the required order-10 increase of galaxy formation. 

Precise mass determination is crucial to figure out whether we actually need non-Gaussianity to explain the data. 

Equation 4.25 in \textcite[]{verdeMultivariateJointPDF2013} is a calculation tool: it allows us to calculate the average of an observable \(A\) for a non-Gaussian random field in a discretized space. 

\textcite[]{planckcollaborationPlanck2018Results2019}: I guess I can interpret the PDF for the nongaussianity parameter, but not much else\dots

\textcite[]{celoriaPrimordialNonGaussianity2018}: is EEE, TTT and such related to the three-point functions computed with different polarizations?

\end{document}
