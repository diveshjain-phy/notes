\documentclass[main.tex]{subfiles}
\begin{document}

\section{Path integral basics}

Following \cite[]{zaidiFunctionalMethods1983}.

We start from the space of square-integrable functions \(q(x)\), endowed with a product and an orthonormal basis \(\phi _n\).
We consider (multi-)linear \emph{functionals}, which are maps from the space of square-integrable functions (or from tuples of them) to \(\mathbb{R}\) or \(\mathbb{C}\). 
These can be represented as functions of infinitely many variables, countably so if we use the basis \(\phi _n\), uncountably so if we use the continuous basis \(x\).

A functional \(F[q]\) can be represented as a power series 
%
\begin{align}
F[q] = \sum _{n=0}^{\infty } \frac{1}{n!} \prod_{i=1}^{n} \int \dd{x_i} q(x_i) f(x_1, \dots, x_n)
\,.
\end{align}

Examples of this are the exponential series corresponding to the function \(f(x)\), mapping \(q(x)\) to \(e^{(f, q)}\) where the brackets denote the scalar product in the space, and the Gaussian series corresponding to the kernel \(K(x, y)\), mapping \(q(x)\) to \(e^{(q, K, q)}\), where 
%
\begin{align}
(q, K, q) = \int \dd{x} \dd{y} q(x) q(y) K(x, y)
\,.
\end{align}

\textbf{Functional derivatives} describes how the output of the functional changes as the argument goes from \(q(x)\) to \(q(x) + \eta (x)\), where \(\eta (x)\) is small. 
This will be a linear functional of \(\eta \) to first order, so we define the functional derivative with the expression 
%
\begin{align}
\eval{F[q+\eta ] - F[q]}_{\text{linear order}} = \int \eta (y) \fdv{F}{q (y)} \dd{y}
\,.
\end{align}

The analogy to finite-dimensional spaces is as follows: the functional derivative \(\fdv*{F}{q(y)}\) corresponds to the \emph{gradient} \(\nabla^{i} F\), while the integral in the previous expression corresponds to the \emph{directional derivative} \((\nabla^{i} F) \eta^{j} g_{ij}\).
The metric is present since the gradient is conventionally defined with a vector-like upper index; in our infinite-dimensional space the scalar product is given by the integral.

Practically speaking, the most convenient way to calculate a functional derivative is by taking \(\eta (x)\) to be such that it only differs from zero in a small region near \(y\), and let us define 
%
\begin{align}
\delta \omega = \int \eta (x) \dd{x}
\,.
\end{align}

Then, we define 
%
\begin{align}
\fdv{F}{q(y)} = \lim_{ \delta \omega \to 0} \frac{F[q + \eta ] - F[q]}{ \delta \omega }
\,.
\end{align}

In order for the limit to be computed easily, it is convenient for \(\eta (x)\) to be in the form \(\delta \omega \times \text{fixed function}\),
so that we are only changing the normalization as we shrink \(\delta \omega \).
A common choice is then 
%
\begin{align}
\eta (x) = \delta \omega  \delta (x-y)
\,.
\end{align}

If we apply this procedure to the identity functional \(q \to q\), we find 
%
\begin{align}
\fdv{q(x)}{q(y)} = \lim_{ \delta \omega  \to 0} \frac{q (x) + \delta \omega \delta (x-y) -q (x)}{ \delta \omega } = \delta (x-y)
\,.
\end{align}

The variable \(q\) is one-dimensional, if instead we wanted to consider a multi-dimensional coordinate system \(q_\alpha \) by the same reasoning we would find 
%
\begin{align}
\fdv{q_\alpha (x)}{q_\beta (y)} = \delta_{\alpha \beta } \delta (x-y)
\,.
\end{align}

An example: the functional derivative of a functional \(F_n\) defined by 
%
\begin{align}
F_n[q] = \int f(x_1 , \dots, x_n) q(x_1 )\dots q(x_n) \dd{x_1} \dots \dd{x_n}
\,,
\end{align}
%
where \(f\) is a symmetric function of its arguments, is given by 
%
\begin{align}
\fdv{F_n}{q(y)} = n \int f(x_1, \dots, x_{n-1}, y) q(x_1 )\dots q(x_{n-1}) \dd{x_1 } \dots \dd{x_{n-1}}
\,,
\end{align}
%
a function of \(y\). 

A \textbf{linear transformation} is in the form 
%
\begin{align}
q(x) = \int K(x, y) q'(y) \dd{y}
\,.
\end{align}

If this transformation has an inverse, which is characterized by the kernel \(K^{-1}\), then we must have the orthonormality relation 
%
\begin{align}
\int K(x,y) K^{-1} (y, z) \dd{y} =
\int K^{-1}(x,y) K (y, z) \dd{y} =
\delta (x-z)
\,.
\end{align}

We can do \textbf{Legendre transforms}: if we have a functional \(F\) we can differentiate with respect to the coordinate \(q\) to find 
%
\begin{align}
\fdv{F[q]}{q(x)} = p(x)
\,,
\end{align}
%
in analogy to the momentum in Lagrangian mechanics. Then, we can map \(F[q]\) to a new functional \(G[p]\) which will only depend on the momentum: 
%
\begin{align}
G[p]= F[p] - \int q(x) p(x) \dd{x}
\,.
\end{align}

We can also define functional integration, by 
%
\begin{align}
\int F[q] \qty[ \dd{q}] = \int \hat{F}(\qty{q_i}) \prod_i \dd{q_i}
\,.
\end{align}

On the right-hand side we are using the expression of the functional as a function of infinitely many variables which we discussed above;
we are then integrating over each of the coordinates in this infinite dimensional function space.

This integral will not always exist, however in the cases in which it does we can change variables. 
Let us consider a linear change of variable, whose kernel is \(K(x, y)\), such that (compactly written) \(q = K q'\). 

Then, we want to compute the integral 
%
\begin{align}
\int F[Kq'] \qty[ \dd{Kq'}] 
\,
\end{align}
%
as an integral in \(\qty[ \dd{q}]\): in order to do so, we need to relate the two functional measures. 
We start by expressing both \(q\) and \(q'\) in terms of an orthonormal basis \(\phi _i\): inserting this into the linear transformation law we get 
%
\begin{align}
q(x) &= \int K(x, y) q'(y) \dd{y}  \\
\sum _{i} q_i \phi _i(x) &= \int K(x, y) \sum _{j} q'_j \phi _j (y) \dd{y}  \\
\sum _{i} q_i \underbrace{\int \phi _i (x) \phi _k (x) \dd{x}}_{ \delta_{ik}} 
&= 
\sum _{j} q'_j \underbrace{\int K(x, y) \phi _j (y) \phi _k (x) \dd{y} \dd{x}}_{ k_{jk}}  \\
q_k &= \sum _{j} q'_j k_{jk}
\,.
\end{align}

Then, the measure will transform with the determinant \(\det K = \det k\): 
%
\begin{align}
\qty[ \dd{q}] = \abs{\pdv{q}{q'}} \qty[ \dd{q'}] = \det K \qty[ \dd{q'}]
\,.
\end{align}


\end{document}
