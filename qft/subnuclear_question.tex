\documentclass{article}

\usepackage[utf8]{inputenc}

\usepackage{textcomp}
\usepackage[T1]{fontenc}
\usepackage{multirow}
\usepackage{float}
\usepackage[caption = false]{subfig}
\usepackage{longtable}
\usepackage{listings}
\usepackage{mathtools}
\DeclareMathOperator{\tr}{Tr}
\usepackage{commath}
\usepackage{bbold}
\usepackage{xcolor}
\usepackage{physics}
%\usepackage[margin=1.8cm]{geometry}

\usepackage{tikz-cd}
\usepackage{amsmath}
\usepackage{amsfonts}
\usepackage{amssymb}
\usepackage{amsthm}
\usepackage{graphicx}
\usepackage[colorinlistoftodos]{todonotes}
\usepackage[colorlinks=true, allcolors=blue]{hyperref}
\usepackage{siunitx}
\sisetup{separate-uncertainty=true}

\usepackage[sc]{mathpazo}
\linespread{1.05}         % Palladio needs more leading (space between lines)
\usepackage[T1]{fontenc}

\newcommand{\diag}[1]{\text{diag}\qty(#1)}
\newcommand{\const}{\text{const}}
\newcommand{\sign}[1]{\text{sign}\qty(#1)}
\renewcommand{\H}{\mathcal{H}}
\renewcommand{\dim}{\text{dim}}
\newcommand{\C}{\mathbb{C}}
\newcommand{\R}{\mathbb{R}}
\newcommand{\N}{\mathbb{N}}
\newcommand{\Z}{\mathbb{Z}}

\renewcommand{\var}[1]{\text{var} \qty(#1)}
\newcommand{\average}[1]{\langle #1 \rangle}

\newcommand\mybox[1]{%
  \fbox{\begin{minipage}{0.9\textwidth}#1\end{minipage}}}


\begin{document}

%\section*{Resonances}

{\centering Jacopo Tissino, 1152348, 22 June 2019}

\textbf{Describe the concept of \emph{resonance} in particle physics, the Breit-Wigner distribution and its use in data analysis.}

Many particles we would like to study are not stable, but are insteaad in shallow
minima of their potential: they can tunnel or be excited easily through them, and
thus decay before they can be observed moving through space.

We can study them in the brief period of time between when they are formed and when
they decay: suppose we have a process \(\ce{a} + \ce{b} \rightarrow \text{(metastable state)} \rightarrow \ce{c} + \ce{d} \).

If we do a scattering experiment on the intermediate state of the reaction, we expect the cross section to be higher if a metastable resonance state is formed.

This qualitative statement can be formalized:

\begin{equation}
    \sigma(E)  =
    \frac{(2J+1)}{(2s_a + 1) (2s_b + 1)} \frac{4 \pi }{k^2}
    \frac{\Gamma_i \Gamma_f}{(E-M_R)^2 + (\Gamma/2)^2}
\end{equation}

where

\begin{itemize}
    \item \(\sigma\) is the scattering integral cross section,
    \item \(k = 2 \pi / \lambda\) is the angular wavenumber of the particle we are scattering off the resonance (or, since we are working in \(c=\hbar=1\), its energy),
    \item \(E = \sqrt{s} \) is the center of mass energy,
    \item \(J\) is the spin of the resonant state,
    \item \(s_{a, b}\) are the spins of the reacting particles,
    \item \(\Gamma_{i,f}\) are the initial and final partial widths of the reaction, respectively the width for \(\ce{a} + \ce{b} \rightarrow \text{(metastable state)}\) and \(\text{(metastable state)} \rightarrow \ce{c} + \ce{b} \),
    \item \(\Gamma\) is the resonance width: the square modulus of the wavefunction of the resonance contains the term \(\exp(- \Gamma t) \),
    \item \(M_R\) is the mass of the resonant state.
\end{itemize}


We derive this formula in the case \(\ce{a}, \ce{b} = \ce{c}, \ce{d}\).
Recall the fact that, if the incoming wavefunction is a planar wave and the outgoing function is of the form \(\psi =  f(\theta, \varphi) \exp(pr/i \hbar) / r \), then the differential scattering cross section is

\begin{equation}
    \dv{\sigma}{\Omega} = \abs{f(\theta, \varphi)}^2
\end{equation}



\end{document}
