\documentclass[main_zanardi.tex]{subfiles}

\begin{document}

Zanardi from the University of Southern California.

\section{Distances over quantum state spaces}

\paragraph{What we will talk about}

What is a metric? "Trace norm distance"

Information theore
tical protocol

Battacharya distance; its infinitesimal version: the Fischer metric, the Fubini-Study metric: many body physics, Quantum Phase Transitions.

At zero temperature, we only have the ground state, there is no entropy, so how do transitions work?

The statements preceded by the word "claim" are left as exercises.

\subsection{Quantum theory recap}

We have \emph{states} and \emph{observables}. We call the system $S$, and its associated Hilbert space \(\H\). (No real quantum system is going to be truly isolated).

A state \(\rho\) is a density matrix, an observable is a Hermitian operator over \(\H\).

\begin{equation}
  S(\H) = \qty{\rho |\, \rho \geq 0, \, \Tr \rho = 1}
\end{equation}

Then \(\forall \phi \in \H: \ev{\rho}{\phi} \geq 0\), \(\rho = \rho ^\dag\).

\(\sigma (\rho) = \text{spectrum of } \rho = \qty{p_i}_i\). \(d = \dim (\H)\).

\(p_i \geq 0\), \(\sum_i p_i = 1\).

So quantum theory is just noncommutative probability theory.

\paragraph{Single qubit}.

\(\H \sim \mathbb{C}^2$, $\dim \H = 2$, $\rho = \frac{1}{2} \qty(\mathbb{1} + \vec{\lambda} \cdot \vec{\sigma})\).

\(\vec{\lambda}\) is the Bloch vector, \(\vec{\sigma}\) are the Pauli matrices.
If \(\abs{\vec{\lambda}} = 1\), we have a pure state.

\paragraph{Exercise}
Prove that \(\rho\) being a density matrix is equivalent to \(\abs{\lambda}\leq 1\), and that \(\abs{\lambda} = 1 \iff \rho^2 = \rho \iff \rho = \ketbra{\psi}\).

Creating convex combinations of states should always be allowed,  and we see this in the fact that the space of states is convex.

States are usually decomposable as probabilistic mixtures, which are convex combinations in "Bloch space" (or its generalizations).

\subsection{Telling states apart} We are given two states \(\rho _{1, \, 2}\): what is the measurement which maximizes the probability we will be able to tell one from the other?

Probability vectors: \(\vec{p} = (p_i)\), \(\vec{q} = (q_i)\). We can use the "\(\ell_1\)" metric:

\begin{equation}
  d(p, q) = \sum_i \abs{p_i - q_i}
\end{equation}

Claim: this is a distance.

Can we do the same for quantum states? In general, \([\rho_1,  \rho_2] \neq 0\), we do not have a common eigenbasis. Could we use the Frobenius metric? Eeh, not really.

We have a map from observables to numbers: \(A \rightarrow \Tr \qty(\rho A) = \langle A \rangle_\rho\).

We can do \(\sup{\abs{{\langle A \rangle_{\rho_1} - \langle A \rangle_{\rho_2}}}}\) (Over \(\norm{A} = 1\)).

\begin{equation}
  \norm{A} = \sup_{\psi \neq 0} \frac{\norm{A\psi}}{{\psi}}
\end{equation}

So our distance is

\begin{equation}
  d = \sup_{\norm{A}=1} \abs{\Tr \qty[A(\rho_1 - \rho_2)]}
\end{equation}

We know that

\begin{align}
  \abs{\Tr \qty(AB)} &= \sum_i b_i \ev{A}{i} \\
  &\leq \sum_i \abs{b_i} \ev{A}{i} \\
  &\leq \norm{A} \sum_i \abs{b_i} \\
  &\leq \norm{A} \Tr \abs{B}
\end{align}

Where \(B = \sum_i b_i \ketbra{i}\)),
and we call \(\norm{B}_1 = \Tr \abs{B}\), where the modulus of the operator can be thought of eigenvalue-wise (diagonalizing the operator, and then flipping the sign of all the negative eigenvalues).

So

\begin{equation}
  d \leq \norm{A} \norm{\rho_1 - \rho_2}_1 = \norm{\rho_1 - \rho_2}_1
\end{equation}

Claim: this in an equality (there \emph{always} exists an $A$ to do the job).

Do we get back the classical case if the matrices commute? Claim: yes.

We call \(D(\rho_1, \rho_2) = \frac{1}{2} \norm{\rho_1 - \rho_2}_1\). (since the $d$ we used before is upper-bounded by 2, by the triangular inequality).

"the duals of self-adjoint operators are traceles"?

\paragraph{Measurements}
Von Neumann orthogonal measurement we know about.

Generalized measurement: we have an ancillary system, mearure this system and then trace over it. This is not described by an orthogonal projection:

\paragraph{Positive Operator-Valued Measurement}

We have finitely many \(\qty{E_i}_i\), \(E_i \geq 0\),  \(\sum_i E_i = \mathbb{1}\).

\(\rho \rightarrow p_i = \Tr (\rho E_i)\).

2-element POVM: \(E_{1, \, 2} \geq 0\), we have our states \(\rho_{1, \, 2}\).

Say we get the states with \(50\%\) probability each, and we wish to distinguish them:

\begin{equation}
  P(\text{success}) = \frac{1}{2} \qty[\Tr(E_1 \rho_1) + \Tr(E_2 \rho_2)]
\end{equation}

\begin{equation}
  P(\text{error}) = \frac{1}{2} \qty[\Tr(E_1 \rho_2) + \Tr(E_2 \rho_1)]
\end{equation}

We want to maximize \(P(\text{success})\). We can rewrite it as:

\begin{align}
  P(\text{success}) &= \frac{1}{2} \qty[\Tr(E_1 \rho_1) + \Tr\qty((\mathbb{1} - E_1) \rho_2)] \\
  &= \frac{1}{2}\qty[1 + \Tr \qty(E_1 (\rho_1 - \rho_2))]
\end{align}

to maximize over $E_1$. The optimum (Claim) is

\begin{equation}
  P(\text{success}) = \frac{1}{2}\qty[1 + \frac{1}{2} \norm{\rho_1 + \rho_2}_1]
\end{equation}

Hellstrom optimal measurement?
This is 1 if they are maximally different, $1/2$ if they are indistinguishable.

$E_1$ should be the projection over the positive eigenvalues of the difference between the matrices.

\paragraph{Bhattacharyya distance}

We have two probability vectors \(\vec{p} = \qty(p_i)_i\), \(\vec{q} = \qty(q_i)_i\).
Normalized in the euclidean metric, if we take the square root component by component: \(V_p = \qty(\sqrt{p_i})_i\).

\begin{equation}
  d_B (\vec{p}, \vec{q}) = \cos^{-1} (\vec{V_p}, \vec{V_q})
\end{equation}

Claim: this is a distance.

\begin{equation}
  d_B (\vec{p}, \vec{q}) = \cos^{-1} \qty(\sum_i \sqrt{p_i q_i})
\end{equation}

\paragraph{Quantize it!}

Let us focus on the pure state case:
we have a POVM \(\mathbb{E} = \qty{E_i}\), and two probability distributions \(\rho = \ketbra{\phi}\), \(\sigma = \ketbra{\psi}\)

\begin{equation}
  P_\phi (i) \defeq \Tr (E_i \rho) = \ev{\phi}{E_i}
\end{equation}

and similarly for \(\psi\).

The Bhattacharyya distance is

\begin{equation}
  d_B (\phi, \psi) = \sup _\mathbb{E} d_B\qty(\vec{P_\phi}, \vec{P_\psi})
\end{equation}

Theorem:

\begin{equation}
  d_B (\phi, \psi) = \cos^{-1} \abs{\braket{\phi}{\psi}}
\end{equation}

this is the Fubini-Study metric over a projective Hilbert space.

We can generalize this to differential geometry.

\begin{greenbox}
  In this particular case this can be expressed as
  \(\abs{\braket{\phi}{\psi}} = \sqrt{\Tr \qty(\rho\sigma)}\)
  but the result does not generalize.
\end{greenbox}
\end{document}
