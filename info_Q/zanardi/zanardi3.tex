\documentclass[main_zanardi.tex]{subfiles}

\begin{document}

\section{Quantum Fischer metric}

Eravamo arrivati a definire la \textit{quantum Fisher information} \(F^Q\) come:

\begin{align*}
F^Q \propto \Tr (\rho L^2)
\end{align*}

dove \(L\) è la derivata logaritmica simmetrica, definita come la soluzione all'equazione di Lyapunov:

\begin{align*}
\frac{1}{2}\{ L,\rho\} = \rho' = \frac{1}{2}(L\rho + \rho L)
\end{align*}

Inserendo la rappresentazione spettrale di $\rho$:

\begin{align*}
\rho = \sum_i p_i \ket{i}\bra{i}
\end{align*}

Si trova:

\begin{align*}
F^Q \propto \sum_{ij=1}^d \frac{\abs{\bra{i}\rho'\ket{j}}^2}{p_i + p_j} = \sum_{\mathbb{E}} F^{\text{class}}
\end{align*}

Si trova che $F^Q$ è proprio la scelta ottimizza l'informazione di Fisher \textit{classica} su tutte le possibili POVM $\mathbb{E}=\{E_i\}$, dove la $F^{\mathrm{class}}$ è data da:

\begin{align*}
F^{\text{class}} = \int \dd{x} \frac{(p')^2}{p}
\end{align*}

Nel caso $\rho$ sia uno stato puro, si ha $\rho^2 = \rho = \ket{\psi}\bra{\psi}$, e differenziando tale relazione si ottiene $\rho'\rho + \rho \rho' = \rho'$, che ha la stessa forma dell'equazione di Lyapunov. Ma allora:

\begin{align*}
L \propto \rho' = \ket{\psi'}\bra{\psi} + \ket{\psi}\bra{\psi'}
\end{align*}

e si giunge a:

\begin{align*}
F^{Q}_{\mathrm{puro}} \propto \bra{\psi}L^2 \ket{\psi} &= \braket{\psi'|\psi'} - |\braket{\psi|\psi'}|^2 = \norm{\ket{\psi'}}
-|\braket{\psi|\psi'}|^2 =\\
&= \norm{\ket{\psi_\perp'}}^2
\end{align*}

dove $\ket{\psi_\perp'}=(1-\ket{\psi}\bra{\psi})\ket{\psi'}$.\\

La formula per l'informazione di Fisher per stati puri può essere ottenuta in un altro modo, partendo dalla metrica di Fubini-Study:

\begin{align*}
d_{FS} ([\phi], [\psi]) = \cos^{-1}(|\braket{\psi|\phi}|)
\end{align*}

dove con $[\cdot]$ si intendono tutti i ket che rappresentano lo stesso stato quantistico - in altre parole la distanza di Fubini-Study è una distanza definita nello \textit{spazio proiettivo} $\mathcal{P}\mathcal{H}$.\\
Il prodotto scalare $\mathcal{F}=|\braket{\psi|\phi}|$ che compare come argomento nel $\cos^{-1}$ è detto \textbf{Fidelity}. Si ha:

\begin{align*}
\mathcal{F} = \begin{cases}
0 & \text{quando } \psi \perp \phi\\
1 & \Leftrightarrow [\phi]=[\psi]
\end{cases}
\end{align*}

Consideriamo due stati infinitamente vicini tra loro, nel senso $\phi = \psi + \Delta\psi$. Prendendo il $\cos$ di entrambi gli argomenti:

\begin{align*}
\cos \dd_{FS} = |\braket{\psi|\psi+\Delta\psi}|
\end{align*}

Dato che $\dd_{FS}\approx 0$, possiamo espandere entrambi i membri in serie, troncando al secondo ordine:

\begin{align*}
1-\frac{1}{2} ds^2 = \left|\braket{\psi|\psi + \psi'd\lambda + \frac{1}{2}\psi'' d\lambda^2}\right|
\end{align*}

Espandiamo per linearità il prodotto scalare:

\begin{align}
1-\frac{1}{2}ds^2 = \left| \underbrace{\braket{\psi|\psi} }_{=1}+ \braket{\psi|\psi'}d\lambda + \frac{1}{2} \braket{\psi|\psi''}d\lambda^2 \right|
\label{eqn:espansione-metrica}
 \end{align}

Poiché l'evoluzione unitaria non cambia la norma di uno stato, differenziando:

\begin{align*}
\braket{\psi|\psi} = 1 \xRightarrow{\partial_t} \braket{\psi|\psi'} + \braket{\psi'|\psi} =0
\end{align*}

Ma allora, riarrangiando:

\begin{align*}
\braket{\psi|\psi'} = - \braket{\psi'|\psi} = - \overline{\braket{\psi|\psi'}}
\end{align*}

Ciò significa che $\braket{\psi|\psi'}$ deve essere puramente immaginario:

\begin{align*}
\braket{\psi|\psi'} = \pm i |\braket{\psi|\psi'}|
\end{align*}

Differenziando ancora una volta otteniamo:

\begin{align*}
 \braket{\psi|\psi'} + \braket{\psi'|\psi} =0 \xRightarrow{\partial_t} \braket{\psi'|\psi'} + \braket{\psi|\psi''} + \braket{\psi''|\psi} + \braket{\psi'|\psi'} = 0
\end{align*}

E riarrangiando:

\begin{align*}
\underbrace{\braket{\psi|\psi''} + \braket{\psi''|\psi}}_{2\op{Re}\braket{\psi|\psi''}} = -2\braket{\psi'|\psi'}
\end{align*}

E otteniamo un ulteriore vincolo:

\begin{align*}
\op{Re}\braket{\psi|\psi''} = -\braket{\psi'|\psi'}
\end{align*}

\textbf{Nota}: non è detto che $\ket{\psi'}$ sia normalizzato!

Possiamo ora calcolare il valore assoluto in (\ref{eqn:espansione-metrica}), che sarà pari alla somma dei quadrati di parte reale e immaginaria. Rimuovendo subito gli ordini di $d\lambda$ maggiori di $2$ si ottiene:

\begin{align*}
\sqrt{\left(1+\frac{1}{2}\op{Re}\braket{\psi|\psi''}\right)^2 + |\braket{\psi|\psi'}|^2 d\lambda^2} + O(d\lambda^3)
\end{align*}

(verificare). Inserendo i calcoli appena visti:

\begin{align*}
= \sqrt{\left(1-\frac{1}{2}\braket{\psi'|\psi'} d\lambda^2\right)^2 + |\braket{\psi|\psi'}|^2 } \propto \sqrt{1-\braket{\psi'|\psi' }d\lambda^2 + |\braket{\psi|\psi'}|^2 d\lambda^2}
\end{align*}

Ed espandendo otteniamo finalmente:

\begin{align*}
\mathcal{F} \approx 1-\frac{1}{2}(\braket{\psi'|\psi'} - |\braket{\psi|\psi'}|^2) = 1-\frac{1}{2}ds^2
\end{align*}

E quindi:

\begin{align*}
\dd{s^2} = \braket{\psi'|\psi'}-|\braket{\psi|\psi'}|^2 d\lambda^2
\end{align*}

Ritroviamo allora nella metrica di Fubini-Study la Quantum Fisher Metric.

\begin{equation}
  \sqrt{\qty(1+ \frac{1}{2} \Re(\braket{\psi}{\psi''} )^2) + \abs{\braket{\psi}{\psi'} }^2 \dd{\lambda^2} + O(\dd{\lambda^2}) }
\end{equation}

It seems like we're missing something but we are not.

\begin{equation}
  = \sqrt{\qty(1 - \frac{1}{2} \braket{\psi'}{\psi'} )^2 \dd{\lambda^2} + \abs{\braket{\psi}{\psi'} }^2 }
  \propto \sqrt{1 - \braket{\psi'}{\psi'} )^2 \dd{\lambda^2}  + \abs{\braket{\psi}{\psi'} }^2 \dd{\lambda^2}}
\end{equation}

Therefore the fidelity is

\begin{equation}
  F \sim 1 - \frac{1}{2} \qty(\braket{\psi'}{\psi'} )^2 \dd{\lambda^2}  + \abs{\braket{\psi}{\psi'} }^2 \dd{\lambda^2})
  \sim 1 - \frac{1}{2}\dd{s^2}
\end{equation}

so \(\dd{s^2} = \qty(\braket{\psi'}{\psi'} )^2  + \abs{\braket{\psi}{\psi'} }^2) \dd{\lambda^2}\)
is the expression of the \textbf{Quantum Fisher Metric} in the \(\dim \qty[ \supp{\rho}] = 1\) case.

Now \(\lambda\) is our control parameter and it lives on a smooth manifold, and \(\psi = \psi(\lambda)\) is the Ground State of a hamiltonian \(H\).

We have a map like: \(\lambda \rightarrow H(\lambda) \rightarrow \text{Groundd State }  = \P H\).

Let us consider a perturbative setting: \(H(\lambda) = H_0 + \lambda V\).

We remember the perturbative expression

\begin{equation}
  \ket{\widetilde{\psi}_0 } = \ket{\psi_0} + \lambda \ket{\psi_0^{(1)}} + O(\lambda^2)
\end{equation}

where

\begin{equation}
  \ket{\psi_0 ^{(1)} } = \sum _{n>0} \frac{\dyad{\psi_n ^{(0)}} V \ket{\psi_0 ^{(0)} }}{E_n ^{(0)}  -  E_0 ^{(0)} } \perp \ket{\psi_0}
\end{equation}

and \(H_0 \ket{\psi_n} = E_n \ket{\psi_n}\).

With this, we can get a new formula for the metric. We write \(\lambda V = \dd{H} \). Now, in the expression for the Fisher Metric the term  \(\braket{\psi_0}{\psi_0 '} = 0 \). So:

\begin{equation}
  \dd{s^2} = \norm{\ket{\psi_0 '}} = \sum _{n>0}  \frac{\abs{\bra{\psi_n} \dd{H} \ket{\psi_0}} }{(E_n - E_0)^2}
\end{equation}

This is different from regular second order perturbation theory because of the square in the denominator. It is the Fubini-Study metric pulled back over the parameter manifold.

We want to do the "GR thing". We have a pseudorimannian parameter manifold, and the divergences of the metric are where the physics happens: if the metric is large, some very close events are distingishable by a very large measurement.

If the first excited state is very close to the ground state, which happens in quantum phase transitions, we have a blow-up of the metric. (In the thermodinamical limit, where the system size \(N \rightarrow \infty\)).

\begin{equation}
   \lim _ {N \rightarrow \infty} \Delta_N(\lambda) = \begin{cases}
     > 0: \qquad \text{Gapped}  \\
     0: \qquad \text{Gap-less: a Quantum controlled Phase Transition}
 \end{cases}
\end{equation}

\textbf{Claim}: if \(\lambda \in M \) is gapped, then the metric is at most extensive (\(\dd{s^2} \leq cN = O(N) \)).

Then it follows that if \(\dd{s^2} = N^\alpha \) with \(\alpha>1\), then we can tell the system is gapless.

\begin{greenbox}
  \textbf{Proof}:

  \begin{subequations}
  \begin{align}
    \dv{s^2}{\lambda^2}  &= \sum _{n>0}  \frac{\abs{\bra{\psi_n} V \ket{\psi_0}}}{(E_n - E_0)^2}  \\
    &\leq \frac{1}{\Delta^2} \sum _{n>0} \abs{\bra{\psi_n} V \ket{\psi_0}}^2  \\
    &=  \frac{1}{\Delta^2} \bra{\psi_0} V   \sum _{n>0} \dyad{\psi_n} V \ket{\psi_0}
      - \ketbra{\psi_0}{\psi_0}
  \end{align}
  \end{subequations}

  \textbf{TO CHECK}

  So,

  \begin{subequations}
  \begin{align}
    \qty(\dv{s}{\lambda})^2  &=  \frac{1}{\Delta^2} \qty(\ev{V^2}{\psi_0} - \ev{V^2}{\psi_0}^2)  \\
    &= G \dd{V}
  \end{align}
  \end{subequations}

  where we consider \(V = \sum _{j}  V_j \) and \(G _{ij} = \expval{V_i V_j} - \expval{V_i} \expval{V_j} = G(\abs{i-j} ) \).

  So,

  \begin{equation}
    \qty(\dv{s}{\lambda} ) \leq \frac{1}{\Delta^2} N \sum _{k} G(k)
  \end{equation}

  but it is known that for gapped systems \(G\) decays eponentially; so

  \begin{equation}
    \qty(\dv{s}{\lambda} ) \leq \frac{1}{\Delta^2} N \sum _{k} \exp(-\frac{\abs{G}}{\text{something}}  )
  \end{equation}

  the point is that those exponentials are finite:

  \begin{equation}
    \qty(\dv{s}{\lambda} ) \leq \frac{1}{\Delta^2} N \widetilde{C}
  \end{equation}

  which proves the theorem.
\end{greenbox}

We will use \(N\) qubits, and

\begin{equation}
  H_{xy} = -\frac{1}{2} \qty(\sum _{i=1}   ^{N} \frac{1 + \gamma}{2} \sigma_i ^x \sigma_{i+1}^x \frac{1-\gamma}{2} \sigma^y_i \sigma_{i+1}^y - h \sigma_i^{z} )
\end{equation}

this maps our system to free fermions which are exactly solvable: see Jordan Kahuma (?).
\end{document}
