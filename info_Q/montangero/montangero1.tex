\documentclass[main_montangero.tex]{subfiles}
\begin{document}

\section{Shor's algorithm}

It is a method to factor a product of large numbers.

\paragraph{Motivation}

Alice wants to communicate a message \(P\) to Bob.
Bob generates a public key \( K _{\text{Pu}} \) and a private key \( K _{\text{Pr}}  \), he sends the public key \( K _{\text{Pu}} \) to Alice, who encodes the message with an algorithm \(E\) which depends on the public key:

\begin{equation}
  C = E _{K _{\text{Pu}} } (P) = P ^e \mod N
\end{equation}

where \( N \) is chosen such that \( N = pq \), with \( p, q \in \Z_{\text{prime}} \), \(\Phi = (p-1) (q-1)\), \( 1<e<\Phi \),  and \( \text{GCD}(\Phi, e) = 1 \).

She then sends \( C \) to Bob, who uses \( K _{\text{Pr}} \) to decode it with an algorithm \(D\):

\begin{equation}
  P = D _{K _{\text{Pr}}} C^d \mod N % P (d, N)??
\end{equation}

where \( d \) is chosen such that \( de = 1 \mod \Phi \).

Factoring \( N \) is equivalent to finding the period of a function: the \emph{order} \( r \) is the number such that \( x ^{r} = 1 \mod N  \) , \( f(r) = x^r \mod N \).

If \( r \) is even, then \( y = x ^{r/2}  \), so \( y^2 = 1 \mod N \) therefore \( (y+1)(y-1) = 0 \mod N \).

Therefore \( (y+1)(y-1) = kN \) for some \( k \in \N \), so we have found the factors.

\paragraph{The algorithm}

Given \( N = pq \), we have the following steps:

\begin{enumerate}
  \item Choose \( x<N \). If it divides \( N \), we are done;
  \item Find the order \( r \) such that \( f(r) = x^r \mod N \); \label{item:qft}
  \item If \( r \) is even, we have the factors. If it is not, start over.
\end{enumerate}

The quantum step is in step \ref{item:qft}.


\paragraph{Step 1}

\subparagraph{Hypotheses} These are not actually needed but they make treating the problem much simpler, and there is not much to learn in generalizing: we assume \( N = 2^n \) and \( N/r = m \in \N \).

As always we cannot directly encode our function as a unitary transformation since it will be periodic, therefore not injective, therefore not unitary. So we encode it taking the input along, as

\begin{equation}
   U: \ket{x}\ket{0} \longmapsto \ket{x} \ket{f(x)}
\end{equation}

We start from \( \ket{0}^{\otimes n} \), apply \(N\) Hadamards and get \( \ket{\psi_0} \) = superposition of all possible states, and with this
we prepare

\begin{equation}
  \ket{\psi_1} = \frac{1}{\sqrt{2^n}} \sum _{x=0} ^{N}   \ket{x} \ket{f(x)}
\end{equation}

\paragraph{Step 2}

We measure the second registry, and obtain \( \ket{\overline{f}} \). Then the first registry must contain all the combinations which generate that state: so:

\begin{subequations}
\begin{align}
  \ket{\psi_2}
  &= \frac{1}{\sqrt{m}}\sum _{j=0} ^{m-1} \ket{x_0 + jr} \ket{\overline{f(x_0)}}  \\
  &= \qty[\frac{1}{\sqrt{m}}\sum _{j=0} ^{m-1} \ket{x_0 + jr} ] \otimes \ket{\overline{f(x_0)}}
\end{align}
\end{subequations}

\paragraph{Step 3}

We want to find \( r \), so we can do a quantum Fourier transform. It can be slow  to actually measure the full transform for generic functions but in our case the transform is applied to a function which is already periodic

\begin{equation}
  \ket{\psi _3} = \text{QFT}\qty{\ket{\psi_2}} = \frac{1}{\sqrt{mN}} \sum _{y=0} ^{N-1} \sum _{j=0} ^{m-1} \exp(2 \pi i (x_0 + jr) y/N) \ket{y}
\end{equation}

\paragraph{Step 4}

We compute the probability of obtaining a specific value \(\overline{y} \) from a measure of the registry:

\begin{subequations}
\begin{align}
  \P\qty(\overline{y})
  &= \frac{1}{Nm} \abs{\sum _{j=0} ^{m-1} \exp(2 \pi i (x_0 + jr) \overline{y}/N) }  \\
  &= \frac{1}{r} \abs{\frac{1}{m} \sum_j \exp(2 \pi i j \overline{y} /m)}
\end{align}
\end{subequations}

Claim: the states with nonzero probability to be found are those with \( \overline{y} = km \), where \( k \in 0, \dots,  r \).

\subparagraph{Example}

\( P(\overline{y} = 0) = 1/r \abs{1/m \sum_j 1} = 1/r \). Our function is periodic with period \(r\), and there are \(r-1\) other analogous states. So, the probability is saturated and there is no other possible outcome.

So all the states we get are in the form \( \overline{y}=km = kN/r \). We know $N$, we measured \( \overline{y} \), so:

\begin{itemize}
  \item if \( k=0 \), we failed;
  \item if \( k\neq 0 \), we set \( \overline{y}/N = \overline{k}/r \) and find the solution in polynomial time.
\end{itemize}

\( P(\text{success}) \sim 1 \) dopo \( O(\log(\log(r))) \).

Recall \( n = \log N \): the complexity of Shor's algorithm scales as \( O(n^2 \log n \log \log n) \), whereas the classical algorithm scales as \( \exp(O(\sqrt[3]{n\log n})) \).

It is important to emphasize that no classical algorithm has been found which runs in polynomial time, but it has \emph{not} been proven that it is impossible for one to be found.

\paragraph{Example of period search}

\begin{equation}
  f(x) = \frac{1}{2} \qty(\cos \pi x + 1)
\end{equation}

\[
 f:
\left|
  \begin{array}{rcl}
    \qty{0,1}^3 & \longrightarrow & \qty{0,1} \\
    0,2,4,6 & \longmapsto & 1 \\
    1,3,5,7 & \longmapsto & 0 \\
  \end{array}
\right.
\]

As always, we mix the notation for \(n\)-qubit states, \(n\)-bit numbers expressed in binary and in decimal.

So \( N = 2^3 = 8 \). \( r=2 \), \( m = N/r = 4 \).

\subparagraph{Step 1}

\begin{equation}
  \ket{\psi _1 }  =  \frac{1}{\sqrt{8}} \sum_{m=0} ^{7} \ket{x}_1 \ket{f(x)}_2
\end{equation}

\subparagraph{Step 2}

\begin{equation}
  \ket{\psi_2} = \frac{1}{2} \qty(\ket{1}+\ket{3}+\ket{5}+\ket{7})_1 \otimes \ket{0}_2
\end{equation}

\subparagraph{Step 3}

We map \( j \rightarrow \frac{1}{\sqrt{8}} \sum_k \exp(2 \pi i j k / 8) \ket{k} \)


\begin{subequations}
\begin{align}
  \ket{\psi_3}
  &= \frac{1}{2 \sqrt{8}} (\ket{0} + e^{i \pi/4} \ket{1} \\
  &+ \dots + \ket{0} + e^{3 i \pi /4}\ket{1})_1 \otimes \ket{0}_2 \\
  &= \frac{1}{\sqrt{2}} \qty(\ket{0}+\ket{4} )
\end{align}
\end{subequations}

We either measure 0 or 4. So, if it is 0 we have failed, if it is 4 we have \( \overline{y} = 4 \), therefore \( k=1 \)  works and \( r=2 \).

\end{document}
