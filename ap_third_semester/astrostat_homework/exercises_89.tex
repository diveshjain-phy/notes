\documentclass[main.tex]{subfiles}
\begin{document}

\section*{Exercise 8}

For clarity, we denote with Greek indices those ranging from 1 to \(N\), the size of the vector of data; and with Latin indices those ranging from 1 to \(M\), the number of templates. 
Then, the \(\chi^2\) reads 
%
\begin{align}
\chi^2 = \qty(d_\alpha - A_{i} t_{i \alpha }) C^{-1}_{\alpha \beta } 
\qty(d_\beta - A_j t_{j \beta })
\,,
\end{align}
%
where the Einstein summation convention has been used. 
We want to maximize this as the amplitudes vary: therefore, we set the derivative with respect to \(A_k\) to zero,
%
\begin{align}
\pdv{\chi^2}{A_k} = -2 t_{k \alpha } C^{-1}_{\alpha \beta } \qty(d_\beta - A_j t_{j \beta }) = 0
\,,
\end{align}
%
which means that \(d_ \beta = A_j t_{j \beta }\), as long as the linear map \(t_{k \alpha } C^{-1}_{\alpha \beta }\) is nondegenerate. This is a linear system: \(N\) equations in \(M\) unknowns.

\todo[inline]{A bit unsure about where to go from here, need to ponder it a little more. }

\end{document}
