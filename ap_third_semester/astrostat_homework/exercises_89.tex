\documentclass[main.tex]{subfiles}
\begin{document}

\subsection*{Exercise 8}

For clarity, we denote with Greek indices those ranging from 1 to \(N\), the size of the vector of data; and with Latin indices those ranging from 1 to \(M\), the number of templates.

We are assuming that the data have a Gaussian distribution with a covariance matrix \(C\), and we are modelling their mean \(\mu_\alpha  \) as a sum of templates \(t_{i \alpha}\) with coefficients \(A_i\):
%
\begin{align}
\mu _\alpha = t_{i \alpha } A_i
\,,
\end{align}
%
where the Einstein summation convention has been used. 
Therefore, the likelihood is proportional to 
%
\begin{align}
\mathscr{L}(d_\alpha | A_i) \propto \exp(- \frac{1}{2} \qty(d_\alpha - A_{i} t_{i \alpha }) C^{-1}_{\alpha \beta } 
\qty(d_\beta - A_j t_{j \beta }))
\,.
\end{align}

The normalization only depends on the covariance matrix \(C_{\alpha \beta }\), which we assume is fixed.
Therefore, maximizing the likelihood\footnote{Which is equivalent to maximizing the posterior if we are using a flat prior.} is equivalent to minimizing the \(\chi^2\), which reads 
%
\begin{align}
\chi^2 = \qty(d_\alpha - A_{i} t_{i \alpha }) C^{-1}_{\alpha \beta } 
\qty(d_\beta - A_j t_{j \beta })
\,.
\end{align}

We want to maximize this as the amplitudes vary: therefore, we set the derivative with respect to \(A_k\) to zero,
%
\begin{align}
\pdv{\chi^2}{A_k} = -2 t_{k \alpha } C^{-1}_{\alpha \beta } \qty(d_\beta - A_j t_{j \beta }) = 0
\,,
\end{align}
%
which means that 
%
\begin{align}
t_{k \alpha } C^{-1}_{\alpha \beta } d_\beta = (t_{k \alpha } C^{-1}_{\alpha \beta }  t_{j \beta }) A_j
\,,
\end{align}
%
a linear system of \(M\) equations (indexed by \(k\)) in the \(M\) variables \(A_j\). 
If we denote the evaluations of bilinear forms in the data (\(N\)-dimensional) space with brackets, as \(a_\alpha C_{\alpha \beta } b_\beta \overset{\text{def}}{=} (a | C |b)\), this reads 
%
\begin{align}
(t | C^{-1} | d)_k &= (t | C^{-1} | t)_{kj} A_j  \\
\qty[(t | C^{-1} | t)^{-1}]_{mk} (t | C^{-1} | d)_k  &=
\underbrace{ \qty[(t | C^{-1} | t)^{-1}]_{mk}
(t | C^{-1} | t)_{kj}}_{= \delta_{mj}} A_j = A_m  \\
A_m &= \qty[(t | C^{-1} | t)^{-1}]_{mk} (t | C^{-1} | d)_k
\,,
\end{align}
%
where the inverse of \((t | C^{-1} | t)\) is to be computed in the \(M\)-dimensional vector space. 

\subsection*{Exercise 9}

Our model for the mean value is in the form \(\mu (\Theta , A) = A \overline{x}(\Theta )\), where \(\overline{x}\) is a generic function of \(\Theta \), while \(A\) is our scale parameter.\footnote{This is not specified in the problem, but it seems natural to think that \(\abs{\overline{x}(\Theta )}\) is a constant for varying \(\Theta \). } 
Our likelihood then reads 
%
\begin{align}
\mathscr{L}(x | \Theta , A) = \underbrace{\frac{1}{(2\pi )^{N/2} \sqrt{\det C}}}_{B_1 }
\exp(- \frac{1}{2} (x - A \overline{x}(\Theta ))^{\top} C^{-1} (x- A \overline{x}(\Theta )))
\,.
\end{align}

If the priors for both \(A\) and \(\Theta \) are flat, this corresponds to the joint posterior \(P (\Theta , A | x)\). 
We want to marginalize over \(A\), which amounts to integrating over it: dropping the dependence on \(\Theta \) of \(\overline{x}\) and defining \(V = C^{-1}\) we find
%
\begin{align}
P(\Theta | x) 
&= B_1  \int \exp(- \frac{1}{2} (x - A \overline{x})^{\top} V (x- A \overline{x})) \dd{A}  \\
&= B_1  \int \exp(- \frac{1}{2} \qty(x^{\top} V x -2 A \overline{x}^{\top} V x + A^2 \overline{x}^{\top} V \overline{x})) \dd{A} 
\marginnote{Used the symmetry of \(V\).}
\,.
\end{align}

Applying the formula for the single-variable Gaussian integral \eqref{eq:single-variable-gaussian-integral} (the bilinear forms are all evaluated to yield scalars, we are only integrating over the scalar \(A\)!) we then get 
%
\begin{align}
P(\Theta | x) &= \underbrace{B_1  \exp(- \frac{1}{2} x^{\top} V x )}_{B_2 } \exp( \frac{(\overline{x}^{\top} V x)^2}{4 (\overline{x}^{\top} V \overline{x})}) \sqrt{ \frac{\pi }{\overline{x}^{\top} V \overline{x}}}  \\
&= B_2 \sqrt{ \frac{\pi}{\overline{x}^{\top}V \overline{x}}}
\exp(\frac{\overline{x}^{\top} \Omega \overline{x}}{4 \overline{x}^{\top}V \overline{x}})
\,,
\end{align}
%
where we defined the bilinear form \(\Omega = V x x^{\top} V^{\top}\).\footnote{With explicit indices, \(\Omega_{im} = V_{ij} x_j x_k V_{km}\).}

% We can observe that all information about the scale of \(\overline{x}\) has been lost: if we map \(\overline{x} \to A \overline{x}\) the argument of the exponential does not change, therefore the only change comes from the factor in front, and the posterior \(P\) is mapped to \(P / A\). 

Let us consider a simple example of this as a sanity check: suppose that \(x\) is two-dimensional, and \(\overline{x}(\Theta ) = (\cos \Theta , \sin \Theta )^{\top}\); further, suppose that \(V\) is diagonal, so that 
%
\begin{align}
V = \left[\begin{array}{cc}
\sigma_x^{-2} & 0 \\ 
0 & \sigma _y^{-2}
\end{array}\right]
\,.
\end{align}

Also, suppose that the observed data parameter is 
%
\begin{align}
x = A_x \left[\begin{array}{c}
\cos \varphi  \\ 
\sin \varphi 
\end{array}\right]
\,.
\end{align}

Then, the multiplicative constant in front of the marginalized posterior reads 
%
\begin{align}
B_2 = B_1 \exp(- \frac{1}{2} A_x^2 \qty( \frac{\cos^2\varphi}{\sigma^2_x} + \frac{\sin^2\varphi}{\sigma^2_y}))
\,;
\end{align}
%
while the bilinear form \(\Omega \) is 
%
\begin{align}
\Omega &= A_x^2
\left[\begin{array}{cc}
\sigma_x^{-2} & 0 \\ 
0 & \sigma _y^{-2}
\end{array}\right]
\left[\begin{array}{cc}
\cos^2 \varphi  & \cos \varphi \sin \varphi  \\ 
\cos \varphi \sin \varphi  & \sin^2 \varphi 
\end{array}\right]
\left[\begin{array}{cc}
\sigma_x^{-2} & 0 \\ 
0 & \sigma _y^{-2}
\end{array}\right]  \\
&= A_x^2\left[\begin{array}{cc}
\cos^2 \varphi / \sigma_x^{4} & \cos \varphi \sin \varphi / \sigma_x^2 \sigma _y^2 \\ 
\cos \varphi \sin \varphi / \sigma_x^2 \sigma _y^2 & \sin^2 \varphi / \sigma_y^{4}
\end{array}\right]
\,.
\end{align}

Then, when we  evaluate the marginalized posterior we will find something in the form
%
\begin{align}
P(\Theta | x) &= B_1 \sqrt{\pi } \qty( \frac{\cos^2\Theta }{\sigma_x^2} + \frac{\sin^2 \Theta }{\sigma _y^2})^{-1/2}
\exp( A_x^2 F(\Theta , \varphi ))
\,,
\end{align}
%
where \(F (\Theta , \varphi )\) is some function whose specific form does not really matter;\footnote{For completeness, here is the full expression: 
%
\begin{align}
\begin{split}
F(\Theta , \varphi ) &=
- \frac{1}{2} \qty( \frac{\cos^2\varphi}{\sigma^2_x} + \frac{\sin^2\varphi}{\sigma^2_y})+  \\
&\phantom{=}\ 
+ \frac{1}{4} \qty( \frac{\cos^2 \Theta}{\sigma _x^2} + \frac{\sin^2 \Theta }{\sigma _y^2})^{-1}
\qty[ 
    \frac{\cos^2 \Theta \cos^2 \varphi }{\sigma _x^{4}}
    +2\frac{\cos \Theta \sin \Theta \cos \varphi \sin \varphi  }{\sigma _x^{2} \sigma _y^{2}}
    +\frac{\sin^2 \Theta \sin^2 \varphi }{\sigma _y^{4}}
]
\,.
\end{split}
\end{align}
%
} the point is that the amplitude of the observed data vector, \(A_x\), appears only as a multiplicative prefactor: its exact value will be taken care of by the evidence, and it cannot affect the shape of the distribution. 
Therefore, we see that by marginalizing over \(A\) we have ``forgotten'' any scaling information about \(x\). 

\end{document}
