\documentclass[main.tex]{subfiles}
\begin{document}

\section*{Exercise 8}

For clarity, we denote with Greek indices those ranging from 1 to \(N\), the size of the vector of data; and with Latin indices those ranging from 1 to \(M\), the number of templates. 
Then, the \(\chi^2\) reads 
%
\begin{align}
\chi^2 = \qty(d_\alpha - A_{i} t_{i \alpha }) C^{-1}_{\alpha \beta } 
\qty(d_\beta - A_j t_{j \beta })
\,,
\end{align}
%
where the Einstein summation convention has been used. 



\end{document}
