\documentclass[main.tex]{subfiles}
\begin{document}

\section{Exercise 4}

% \marginpar{Tuesday\\ 2020-10-6, \\ compiled \\ \today}

After being given a probability distribution \(\mathbb{P}(x)\), we define the \emph{characteristic function} \(\phi \) as its Fourier transform, which can also be expressed as the expectation value of \(\exp(- i \vec{k} \cdot \vec{x})\): 
%
\begin{align}
\phi (\vec{k}) = \int \dd[n]{x} \exp(- i \vec{k} \cdot \vec{x}) \mathbb{P}(x) 
= \mathbb{E} \qty[ \exp(- i \vec{k} \cdot \vec{x})]
\,.
\end{align}

\begin{claim}
A multivariate normal distribution 
%
\begin{align}
\mathcal{N}(\vec{x} | \vec{\mu}, C)
&= \frac{1}{(2\pi )^{n/2} \sqrt{\det C}} \eval{\exp(- \frac{1}{2} \vec{y}^{\top} C^{-1} \vec{y})}_{\vec{y} = \vec{x} - \vec{\mu}}
\,,
\end{align}
%
has a characteristic function equal to 
%
\begin{align}
\phi (\vec{k}) = \exp(- i \vec{\mu}\cdot \vec{k} - \frac{1}{2} \vec{k}^{\top} C \vec{k}) 
\,.
\end{align}
\end{claim}

\begin{proof}[Proof: completing the square]
The integral we need to compute is given, absorbing the normalization into a factor \(N\), by 
%
\begin{align}
\phi (\vec{k}) = N \int \dd[n]{x} \eval{\exp(- i \vec{k} \cdot \vec{x} - \frac{1}{2} \vec{y}^{\top} C^{-1} \vec{y})}_{\vec{y} = \vec{x} - \vec{\mu}}
\,.
\end{align}

The only integrals we really know how to do are Gaussian ones, so we want to rewrite the argument of the exponential so that it is a quadratic form. The manipulation goes as follows, considering the opposite of the argument the exponential in order to have less minus signs and defining the symmetric matrix \(V = C^{-1}\):
%
\begin{align}
i \vec{k} \cdot \vec{x} + \frac{1}{2} \vec{y}^{\top} V \vec{y}
&= i \vec{k} \cdot \vec{x} + 
\frac{1}{2} \vec{x}^{\top} V \vec{x}
- \vec{x}^{\top} V \vec{\mu } 
+ \frac{1}{2} \vec{\mu}^{\top} V \vec{\mu}  \\
&= \frac{1}{2} \vec{x}^{\top} V \vec{x}
+ \vec{x}^{\top}\qty(
    i \vec{k} - V \vec{\mu}
)
+ \frac{1}{2} \vec{\mu}^{\top} V \vec{\mu}  \\
\begin{split}
&= \underbrace{\frac{1}{2} \qty(\vec{x} + V^{-1} (i \vec{k} - V \vec{\mu}))^{\top} V \qty(\vec{x} + V^{-1} (i \vec{k} - V \vec{\mu}))}_{\Circled{1}} +\\
&\phantom{=}\ \underbrace{- \frac{1}{2} \qty(i \vec{k} - V \vec{\mu})^{\top} V^{-1} \qty(i \vec{k} - V \vec{\mu}) + \frac{1}{2} \vec{\mu}^{\top} V \vec{\mu}}_{\Circled{2}}
\,,
\end{split} 
\end{align}
%
which we can now integrate, since it is now a quadratic form in terms of a shifted variable, \(\vec{x} + \vec{p}\), where the constant (with respect to \(\vec{x}\)) vector \(\vec{p}\) is given by \(V^{-1}(i \vec{k} - V \vec{\mu} )\).\footnote{
In the last step we applied the matrix square completion formula: for a  symmetric matrix \(A\) and vectors \(\vec{x}\), \(\vec{b}\) we have 
%
\begin{align}
&\frac{1}{2} \qty(\vec{x} + A^{-1} \vec{b})^{\top} A \qty(\vec{x} + A^{-1} \vec{b}) - \frac{1}{2} \vec{b}^{\top} A^{-1} \vec{b}  =\\
&= \frac{1}{2} \qty[ \vec{x}^{\top} A \vec{x}
+  \vec{x}^{\top} A A^{-1} \vec{b}
+  \qty(A^{-1} \vec{b})^{\top} A \vec{x}
+  \qty( A^{-1} \vec{b})^{\top} A A^{-1} \vec{b}
-  \vec{b}^{\top} A^{-1} \vec{b}]  \\
&=\frac{1}{2} \qty[ \vec{x}^{\top} A \vec{x}
+  \vec{x}^{\top} \vec{b}
+  \vec{b}^{\top} (A^{-1})^{\top} A \vec{x}
+  \vec{b}^{\top} (A^{-1})^{\top}  \vec{b}
-  \vec{b}^{\top} A^{-1} \vec{b}]  \\
&= \frac{1}{2} \vec{x}^{\top} A \vec{x}
+ \vec{b}^{\top} \vec{x}
\,,
\end{align}
%
which we used with \(\vec{b} = i \vec{k} - V \vec{\mu}\).} 

Now, shifting the integral from one in  \(\dd[n]{x}\) to one in \(\dd[n]{(x + p)}\) does not change the measure, since the Jacobian of a shift  is the identity.
Then, we have 
%
\begin{align}
\phi (\vec{k}) &= N \int \dd[n]{(x+p)} \exp(- \Circled{1} - \Circled{2})  \\
&= N \sqrt{\frac{(2\pi )^{n}}{\det V}} \exp(- \Circled{2})  \\
&= \underbrace{\frac{1}{\sqrt{\det V \det C}}}_{= 1} \exp(- \Circled{2})
\,,
\end{align}
%
since the determinant of the inverse is the inverse of the determinant.

Now, we only need to simplify \(\Circled{2}\): 
%
\begin{align}
\Circled{2} &= - \frac{1}{2} \qty[
    - \vec{k}^{\top} V^{-1} \vec{k}
    - 2i \vec{\mu}^{\top} V V^{-1} \vec{k}
    + \vec{\mu}^{\top} V V^{-1} V \vec{\mu}
]
+ \frac{1}{2} \vec{\mu}^{\top} V \vec{\mu}  \\
&= \frac{1}{2} \vec{k}^{\top} C \vec{k} + i \vec{\mu}^{\top} \vec{k}
\,,
\end{align}
%
inserting which into the exponent yields the desired result. 
\end{proof}

\begin{proof}[Proof: diagonalization]
We now follow a different approach: the covariance matrix \(C\) is symmetric, so we will always be able to find an orthogonal matrix \(O\) (satisfying \(O^{\top} = O^{-1}\)) such that \(C = O^{\top} D O\), where \(D\) is diagonal. 
We will then also have \(V = C^{-1} = O^{\top} D^{-1} O\).
Let us denote the eigenvalues of \(D\) as \(\lambda _i\), and the eigenvalues of \(D^{-1}\) as \(d_i = \lambda _i^{-1}\).

Defining \(\vec{z} = O \vec{x}\), \(\vec{m} = O \vec{\mu}\), \(\vec{u} = O \vec{k}\) the negative of the argument of the integral becomes:
%
\begin{align}
 i \vec{k} \cdot \vec{x} 
 + \frac{1}{2}
\qty(\vec{x} - \vec{\mu})^{\top} C^{-1} 
\qty(\vec{x} - \vec{\mu}) &=
 i \vec{u} \cdot \vec{z} 
+ \frac{1}{2} (\vec{z} - \vec{m})^{\top} D^{-1} (\vec{z} - \vec{m})  \\
&= 
i \vec{u} \cdot \vec{z} 
+ \frac{1}{2} \sum _{i} d_i (z_i - m_i)^2   \\
&= \sum _{i} \qty[i u_i z_i + \frac{d_i}{2} \qty(z_i^2 + m_i^2 - 2 m_i z_i)]  \\
&= \sum _{i} \qty[z_i^2 \frac{d_i}{2} + z_i \qty(i u_i - m_i d_i) + \frac{d_i}{2} m_i^2]
\,.
\end{align}

With this, and since by \(\det O = 1\) we have \(\dd[n]{z} = \dd[n]{x}\), we can decompose our Gaussian integral into a product of Gaussian integrals:
%
\begin{align}
\phi (\vec{k}) &= 
N \int \dd[n]{x} \exp(-  i \vec{k} \cdot \vec{x} 
 - \frac{1}{2}
\qty(\vec{x} - \vec{\mu})^{\top} C^{-1} 
\qty(\vec{x} - \vec{\mu}))  \\
&= N \int \dd[n]{z} \exp(- \sum _{i} \qty[z_i^2 \frac{d_i}{2} + z_i \qty(i u_i - m_i d_i) + \frac{d_i}{2} m_i^2])  \\
&= N \prod_i \int \dd{z_i}
\exp(- z_i^2 \frac{d_i}{2} - z_i \qty(i u_i - m_i d_i) - \frac{d_i}{2} m_i^2)  \\
&= N \prod_i \sqrt{\frac{2\pi }{d_i}} \exp(\frac{(i u_i - m_i d_i)^2}{2 d_i} - \frac{d_im_i^2}{2}) \\
&= \frac{1}{\sqrt{\det C \det V}} \prod_i \exp(\frac{- u_i^2 + m_i^2 d_i^2 - 2 i u_i m_i d_i }{2 d_i} - \frac{d_i m_i^2}{2}) \\
&= \exp(\sum _{i} \qty[- \frac{u_i^2}{2d_i} - i u_i m_i])  \\
&= \exp(- \frac{1}{2} \vec{u}^{\top} C \vec{u} - i \vec{u} \cdot \vec{m})  \\
&= \exp(- \frac{1}{2} \vec{k}^{\top} C \vec{k} - i \vec{k} \cdot \vec{\mu })
\,,
\end{align}
%
where we have used the expression for the single-variable Gaussian integral: 
%
\begin{align}
\int \dd{z} \exp(- a z^2 + bz + c) = \sqrt{ \frac{\pi}{a}} \exp( \frac{b^2}{4 a} + c)
\,,
\end{align}
%
which comes from the one-variable completion of the square: 
%
\begin{align}
-az^2 + bz + c = -a \qty(z - \frac{b}{2a})^2 + \frac{b^2}{4a} + c
\,.
\end{align}

Also, we used the fact that orthogonal transformation do not change fully-contracted objects, such as scalar products or bilinear forms.
\end{proof}

\section{Exercise 5}

We can calculate the moments of a distribution through its characteristic function: 
%
\begin{align}
\mathbb{E} \qty[x_\alpha^{n_\alpha } \dots x_\beta^{n_\beta }]
= \eval{\frac{\partial^{n_\alpha \dots n_\beta } \phi (\vec{k})}{\partial(- i k_\alpha )^{n_\alpha } \dots \partial(-i k_\beta )^{n_\beta }}}_{\vec{k} = 0}
\,.
\end{align}

In the multivariate Gaussian case we can then calculate the mean (component by component) as 
%
\begin{align}
\mathbb{E}(x_\alpha ) &= \eval{\pdv{\phi (\vec{k})}{(-i k_\alpha )}}_{\vec{k} = 0}  \\
&= \eval{\pdv{}{(-i k_\alpha )}}_{\vec{k} = 0}
\exp(- \frac{1}{2} \vec{k}^{\top} C \vec{k} - i \vec{k} \cdot \vec{\mu })  \\
&= \eval{\qty[- i \sum_\beta  k_\beta  C_{\beta  \alpha } + \mu _\alpha ] \exp(- \frac{1}{2} \vec{k}^{\top} C \vec{k} - i \vec{k} \cdot \vec{\mu })}_{\vec{k} = 0}  \\
&= \mu _\alpha 
\,,
\end{align}
%
where we used the fact that the differentiation of a symmetric bilinear form is as follows: 
%
\begin{align}
\pdv{}{k_\alpha } \qty(\sum _{\beta \gamma } k_\beta k_\gamma C_{\beta \gamma }) = 2 \sum _{\beta \gamma } \delta_{\beta \alpha } k_\gamma C_{\beta \gamma } = 2 \sum_{\gamma } k_\gamma C_{\alpha \gamma }
\,.
\end{align}

The covariance matrix can be computed by linearity as
%
\begin{align}
\widetilde{C}_{\alpha \beta } = \mathbb{E}\qty[\qty(x_ \alpha - \mathbb{E}(x_\alpha )) \qty(x_\beta - \mathbb{E}(x_\beta ))] 
= \mathbb{E}\qty[x_\alpha  x_\beta ] - \mu _\alpha \mu _\beta 
\,,
\end{align}
%
the first term of which reads as follows: 
%
\begin{align}
\mathbb{E}[x_\alpha x_\beta ] &= \eval{\pdv[2]{\phi (\vec{k})}{(-i k_\beta  )}{(-i k_\alpha  )}}_{\vec{k} = 0}  \\
&= \eval{\pdv{}{(-i k_\beta )}}_{\vec{k} = 0} \qty[-i \sum _{\beta } k_\beta C_{\beta \alpha } + \mu _\alpha ] \exp(- \frac{1}{2} \vec{k}^{\top} C \vec{k} - i \vec{k} \cdot \vec{\mu })  \\
&= C_{\alpha \beta } + \mu _\alpha \mu _\beta 
\,,
\end{align}
%
therefore, as expected, \(\widetilde{C}_{\alpha \beta }\) is indeed \(C_{\alpha \beta }\). 

\section{Exercise 6}

\begin{claim}
The characteristic function of a multivariate Gaussian is, up to normalization, a multivariate Gaussian. 
\end{claim}

\begin{proof}
The characteristic function is the exponential of (minus) 
%
\begin{align}
\frac{1}{2} \vec{k}^{\top} C \vec{k} + i \vec{k} \cdot \vec{\mu} 
&=  \frac{1}{2} \qty(\vec{k} + i C^{-1} \vec{\mu })^{\top} C 
\qty(\vec{k} + iC^{-1}\vec{\mu }) + \frac{1}{2} \vec{\mu }^{\top} C^{-1} \vec{\mu }  
\,,
\end{align}
%
which means that the characteristic function is in the form 
%
\begin{align}
\phi (\vec{k}) = \const \times \exp(- \frac{1}{2} (\vec{k} -\vec{m})^{\top} C (\vec{k} - \vec{m}) )
\,,
\end{align}
%
a multivariate normal with mean \(\vec{m} = -i C^{-1} \vec{\mu}\) and covariance matrix \(C^{-1}\), the inverse of the covariance matrix of the corresponding MVN.  
\end{proof}

\end{document}