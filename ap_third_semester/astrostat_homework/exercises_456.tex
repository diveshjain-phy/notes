\documentclass[main.tex]{subfiles}
\begin{document}

\section{Exercise 4}

% \marginpar{Tuesday\\ 2020-10-6, \\ compiled \\ \today}

After being given a probability distribution \(\mathbb{P}(x)\), we define the \emph{characteristic function} \(\phi \) as its Fourier transform, which can also be expressed as the expectation value of \(\exp(- i \vec{k} \cdot \vec{x})\): 
%
\begin{align}
\phi (\vec{k}) = \int \dd[n]{x} \exp(- i \vec{k} \cdot \vec{x}) \mathbb{P}(x) 
= \mathbb{E} \qty[ \exp(- i \vec{k} \cdot \vec{x})]
\,.
\end{align}

\begin{claim}
A multivariate normal distribution 
%
\begin{align}
\mathcal{N}(\vec{x} | \vec{\mu}, C)
&= \frac{1}{(2\pi )^{n/2} \sqrt{\det C}} \eval{\exp(- \frac{1}{2} \vec{y}^{\top} C^{-1} \vec{y})}_{\vec{y} = \vec{x} - \vec{\mu}}
\,,
\end{align}
%
has a characteristic function equal to 
%
\begin{align}
\phi (\vec{k}) = \exp(- i \vec{\mu}\cdot \vec{k} - \frac{1}{2} \vec{k}^{\top} C \vec{k}) 
\,.
\end{align}
\end{claim}

\begin{proof}[Proof: completing the square]
The integral we need to compute is given, absorbing the normalization into a factor \(N\), by 
%
\begin{align}
\phi (\vec{k}) = N \int \dd[n]{x} \eval{\exp(- i \vec{k} \cdot \vec{x} - \frac{1}{2} \vec{y}^{\top} C^{-1} \vec{y})}_{\vec{y} = \vec{x} - \vec{\mu}}
\,.
\end{align}
\end{proof}

\end{document}