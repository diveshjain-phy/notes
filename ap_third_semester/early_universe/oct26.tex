\documentclass[main.tex]{subfiles}
\begin{document}

\marginpar{Monday\\ 2020-10-26, \\ compiled \\ \today}

Our scalar field is given by 
%
\begin{align}
\varphi (\vec{x}, t) = \varphi_0 (t) + \delta \varphi (\vec{x}, t)
\,,
\end{align}
%
a background value, which we have studied up to now, plus some quantum fluctuations. 

The full field obeys the KG equation: 
%
\begin{align}
\ddot{\varphi}(\vec{x}, t) + 3 H \varphi (\vec{x}, t)
- \frac{\nabla^2 \varphi (\vec{x}, t)}{a^2} 
= - \pdv{V}{\varphi }
\,,
\end{align}
%
which we can Taylor expand up to linear order around the background value if we assume that the perturbation is indeed small: 
%
\begin{align}
\ddot{ \delta \varphi} + 3 H \dot{\delta \varphi}
- \frac{\nabla^2}{a^2} \delta \varphi 
= - \pdv[2]{V(\varphi )}{\varphi } \delta \varphi 
\,.
\end{align}

We want to see how the dynamics of \(\delta \varphi \) affect the super-horizon scales. 
The first order equation for \(\delta \varphi \) is coupled to the background one: 
%
\begin{align}
\ddot{\varphi_0} (t) + 3 H \dot{\varphi_0 } (t) = - \eval{\pdv{V}{\varphi }}_{\varphi_0 }
\,.
\end{align}

Let us differentiate it with respect to time, assuming that we are in a phase of De Sitter expansion with \(\dot{H} \approx 0 \): 
%
\begin{align}
\dot{\ddot{\varphi_0}} + 3 H \ddot{\varphi} 
= - \eval{\pdv[2]{V}{\varphi }}_{\varphi_0 } \dot{\varphi_0 }
\,,
\end{align}
%
which looks similar to the equation for \(\delta \varphi \), except for the Laplacian. 

We can neglect the Laplacian on superhorizon scales: for one, on large scales the variations are negligible. More formally, in Fourier space that term is \(k^2 \delta \varphi / a^2\), to be compared with things like \(3 H \dot{ \delta  \varphi} \sim 3 H^2 \delta \varphi \), since the characteristic time of the variations of the field is \(1/H\).  

Now, on superhorizon scales \(k \gg r_H\), which means that \(k \ll aH = r_H^{-1}\), therefore \(k^2 / a^2 \ll H^2\). 
We can then neglect the Laplacian term. 

What we should do now is to consider the Wronskian of the two-equation system:
%
\begin{align}
W (x, y) &= \dot{x} y - x \dot{y}  \\
W(\dot{\varphi_0}, \delta \varphi ) &= \ddot{\varphi_0} \delta \varphi - \dot{\varphi_0} \dot{ \delta \varphi}
\,.
\end{align}

It can be shown that if the Hubble parameter is approximately constant then \(\dot{W} = - 3 HW\), meaning \(W \sim \exp(- 3 H t)\), which quickly goes to zero.

If \(W = 0\), then the two variables \(\dot{\varphi_0} \) and \(\delta \varphi \) are dependent, then we can decompose \(\delta \varphi \) like 
%
\begin{align}
\delta \varphi (\vec{x}, t) = - \delta t (\vec{x}) \dot{\varphi_0}(t)
\,.
\end{align}

Therefore, the full field is given by reverse-Taylor expanding as 
%
\begin{align}
\varphi (\vec{x}, t) &= \varphi_0 (t) - \delta t(\vec{x}) \dot{\varphi_0}(t)  \\
&\approx \varphi_0 (t - \delta t(\vec{x}))
\,.
\end{align}

This expression has a very clear physical interpretation: point by point, the field goes through the same evolution, taking on the same values, only at different times at each point. 

This holds as long as we have a single scalar field, in multi-field models of inflation this will not be the case anymore. 

What we want to compute is the final effect of these fluctuations on super-horizon scales. 
The equation for \(\delta \varphi \) reads 
%
\begin{align}
\ddot{ \delta \varphi} + 3 H \dot{ \delta \varphi} 
- \frac{\nabla^2 \delta \varphi }{a^2} = V''(\varphi_0 ) \delta \varphi 
\,.
\end{align}

We move to Fourier space: 
%
\begin{align}
\delta \varphi (\vec{x}, t) = \frac{1}{(2\pi )^{3}} \int \dd[3]{k} 
e^{i \vec{k} \cdot \vec{x}} \delta \varphi _{\vec{k}}(t)
\,.
\end{align}

Since the field is real, \(\delta \varphi _{\vec{k}} = \delta \varphi _{- \vec{k}}^{*}\). 
These Fourier modes evolve independently of each other up to linear order. 

Why do we make a 3D Fourier transform and not a 4D one? We have time-dependent coefficients in the equation, such as \(H\). 
We Fourier transform to take account of the spatial symmetry: it is not useful to transform in time since there is no time translation symmetry. 

By transforming we are implicitly using a stationary wave basis, which is natural in a spatially flat universe: in a non-flat universe instead of \(e^{i \vec{k} \cdot \vec{x}}\) we should use generalized solution of the Helmholtz equation 
%
\begin{align}
\nabla^2 Q_{\vec{k}} + k^2 Q_{\vec{k}} = 0
\,,
\end{align}
%
the eigenvalue equation for the Laplacian operator. 
These are \emph{time-independent} waves, while propagating waves \(\exp(- i k_\mu x^{\mu })\) would not suit our needs. 
In Fourier space the equation reads 
%
\begin{align}
\ddot{ \delta \varphi} _{\vec{k}} + 3 H \dot{ \delta \varphi}_{\vec{k}}
+ \frac{k^2}{a^2} \delta \varphi _{\vec{k}} = - V'' (\varphi_0 ) \delta \varphi _{\vec{k}}
\,.
\end{align}

We will need to use the standard tools of second quantization: we start by introducing \(\hat{ \delta \varphi} = a \delta \varphi \), and we use conformal time, \(a \dd{ \tau } = \dd{t}\). Then, we will have 
%
\begin{align}
\delta \varphi (\vec{x}, \tau ) = \int \frac{ \dd[3]{k}}{(2 \pi)^3}
\qty[u_k (\tau ) a_{\vec{k}} e^{i \vec{k} \cdot \vec{x}} + u_k^{*} (\tau ) a_{\vec{k}}^\dag e^{-i \vec{k} \cdot \vec{x}}] 
\,.
\end{align}

The functions \(u\) are classical functions of time, while \(a\) and \(a ^\dag \) are the annihilation operators.
They are defined so that \(a_{\vec{k}} \ket{0} = 0\) for any \(\vec{k}\), and similarly \(\bra{0} a ^\dag_{\vec{k}} = 0\). 

Here, \(\ket{0}\) is called the \emph{free vacuum state}.

We impose the normalization condition 
%
\begin{align}
u_k^{*} u_k' - u_k u_k^{*, \prime} = -i
\,,
\end{align}
%
which is equivalent to requiring 
%
\begin{align}
\qty[a_{\vec{k}}, a_{\vec{k}'} ^\dag] = \hbar \delta^{(3)} (\vec{k} - \vec{k}')
\,,
\end{align}
%
while \(\qty[a_k, a_{k'}] = \qty[a_k ^\dag, a_{k'}^\dag] = 0\). 

In flat spacetime, once the commutation relations are fixed we are done: the solutions are known to be plane waves, with \(u_{\vec{k}} \sim \exp(- i \omega _k t) / \sqrt{2 \omega _k}\), and \(\omega _k = \sqrt{k^2 + m^2}\). 

Now, instead, we need to make some assumptions. Because of the equivalence principle, for small distances and time intervals the solutions should reproduce the flat-spacetime ones. 
From this guiding principle, we require 
%
\begin{align}
\frac{k}{aH} \to \infty \implies 
u_k (\tau ) = \frac{e^{-i k \tau }}{\sqrt{2 k}}
\,,
\end{align}
%
which is called the \emph{bunch-Davies vacuum choice}.
In the denominator we only have \(k\) since \(k^2\gg m^2\) for the large scales we are considering. 

What is the equation for \(u\)? In conformal time (\(' = \dv{}{\tau }\)) it is 
%
\begin{align}
u''_k (\tau ) + \qty[k^2  - \frac{a''}{a} + a^2 \pdv[2]{V}{\varphi }] u_k (\tau ) = 0
\,,
\end{align}
%
which is associated to the rescaled variable \( \hat{ \delta \varphi} = a \delta \varphi \). 
We can also see the reason why we have chosen to use the rescaled \(\delta \varphi \) instead of the regular one: what we have found with this ansatz is basically a harmonic oscillator with a time-dependent frequency, changing according to the accelerated expansion of the universe. 
The ansatz yields a canonically-normalized kinetic term in the harmonic oscillator. 

In order to solve the equation, we will assume de-Sitter expansion with \(H = \const\), a massless scalar field with \(m_\varphi^2 = \pdv*[2]{V}{\varphi } = 0\). 
Then, we know that \(\dd{\tau } = \dd{t} / a = \dd{t} e^{-Ht}\), so 
%
\begin{align}
\tau = - \frac{1}{H} e^{-Ht} = - \frac{1}{aH}
\,.
\end{align}

Conventionally we choose the integration bounds so that \(\tau \) runs from negative infinity to zero. This is just a matter of convention, a constant time shift has no effect on the dynamics. We also have 
%
\begin{align}
\frac{a''}{a} = \frac{2}{\tau^2}
\,,
\end{align}
%
so if \(\lambda _{\text{phys}} = a \lambda \ll H^{-1}\), therefore \(\lambda \gg aH\). Therefore, being in the subhorizon scale means that 
%
\begin{align}
\frac{a''}{a} = 2 a^2 H^2 \ll k^2
\,,
\end{align}
%
therefore on subhorizon (microscopic) scales the equation reads 
%
\begin{align}
u''_k + k^2 u_k = 0
\,,
\end{align}
%
solved by 
%
\begin{align}
u_k = \frac{e^{-ik \tau }}{\sqrt{2 k }}
\,.
\end{align}

As expected, we recover the regular flat-spacetime solution.
However, what is really interesting to us are the cosmological, superhorizon solutions: this means \(k \ll a H\), so 
%
\begin{align}
u''_k (\tau ) - \frac{a''}{a} u_k (\tau ) = 0
\,,
\end{align}
%
which is a second-order equation: it will have two independent solutions, and it can be shown that the generic solution is written as 
%
\begin{align}
u_k(\tau ) = \underbrace{B(k) a(\tau )}_{\text{growing mode}} + \underbrace{C(k) a^{-2}(\tau )}_{\text{decaying mode}}
\,.
\end{align}

We neglect the decaying mode, since even if it is excited it will quickly decay.
The physical fluctuation \(\delta \varphi \) is proportional to \(u_k /a \sim B(k)\), therefore we can see that the ``growing mode'' is actually asymptotically constant on superhorizon scales. 

We then want to determine the scale of \(B(k)\). 

We need to match the subhorizon and superhorizon solutions, at the point \(k = aH\). 
%
\begin{align}
\abs{B(k)} a &= \abs{\frac{e^{-ik \tau }}{\sqrt{2 k}}}   \\
\abs{ \delta \varphi _k} &= \abs{B(k)} = \frac{1}{a \sqrt{2k}} 
\,,
\end{align}
%
\todo[inline]{recover last bit}

The fluctuation gets ``frozen in'' at horizon crossing: 
%
\begin{align}
\abs{ \delta \varphi _k} = \frac{H}{2 k^3}
\,.
\end{align}
%
This is called a \emph{gravitational amplification mechanism}.
This is analogous to pair production from vacuum under a strong electrostatic field. The electric field separates the newly-formed electron-positron pair. 

\end{document}