\documentclass[main.tex]{subfiles}
\begin{document}

\marginpar{Monday\\ 2020-10-5, \\ compiled \\ \today}

These next few lectures, we will consider the motivations for inflationary models. 
The problems they solved were the \textbf{shortcomings of the Hot Big Bang} model. 

\begin{enumerate}
    \item The horizon problem;
    \item the flatness problem;
    \item unwanted relics / magnetic monopole problem.
\end{enumerate}

We start by recalling some basic elements in cosmology. 
In order to describe a homogeneous and isotropic universe we use the FLRW metric: 
%
\begin{align}
\dd{s}^2  = - c^2 \dd{t}^2 + a^2(t) \qty[ \frac{ \dd{r}^2}{1 - k r^2} + r^2 \dd{\Omega}^2]
\,,
\end{align}
%
where \(\dd{\Omega}^2 = \dd{\theta}^2 + \sin^2 \theta \dd{\varphi }^2
\). The quantity \(a(t)\) is called the \emph{scale factor}.
The coordinates \(r\), \(\theta \) and \(\varphi \) are called \emph{comoving coordinates}. 

Physical distances and comoving distances are related by 
%
\begin{align}
\lambda _{\text{phys}} = a(t) \lambda _{\text{comoving}}
\,.
\end{align}

The constant  \(k\) is the spatial curvature of the universe, which can always be rescaled so that it is equal to 
\begin{enumerate}
    \item \(+1\) for a spatially closed universe;
    \item \(0\) for a spatially flat universe;
    \item \(-1\) for a spatially open universe.
\end{enumerate}

In terms of the scale factor we define the \textbf{Hubble parameter} 
%
\begin{align}
H = \frac{\dot{a}}{a}
\,,
\end{align}
%
which describes the rate at which the universe expands. 

The dynamics of gravity are described by the Einstein equations: 
%
\begin{align}
G_{\mu \nu } = 8 \pi G T_{\mu \nu }
\,,
\end{align}
%
where \(T_{\mu \nu }\) is the energy-momentum tensor of the particle species filling the universe, while \(G_{\mu \nu } = R_{\mu \nu } - R g_{\mu \nu } / 2\) is the Einstein tensor, describing curvature. 

These can be derived from an action principle through the action 
%
\begin{align}
S = \underbrace{\frac{1}{16 \pi G} \int R \sqrt{-g} \dd[4]{x}}_{S_{EH}} + 
S_{\text{matter}}
\,.
\end{align}

Often we use an ideal fluid energy-momentum tensor: 
%
\begin{align}
T_{\mu \nu } = \rho u_{\mu } u_{\nu } + P h_{\mu \nu }
\,,
\end{align}
%
where \(h_{\mu \nu } = u_{\mu } u_{\nu } + g_{\mu \nu }\) is a projector onto the space orthogonal to the four-velocity. This does not account for any anisotropy, it is the most symmetric energy-momentum tensor. This is a diagonal 

In order to solve the Einstein equations we can proceed with some assumptions, without needing to know the action for all the fundamental fields. The perfect fluid S-E tensor has all the FLRW symmetries, as long as \(\rho \) and \(P\) are only functions of time. 

We are \emph{not} saying that this S-E tensor is only allowed if we are in a FLRW universe. 
\todo[inline]{Clarify\dots }

Requiring the FLRW symmetries means that the S-E tensor must be diagonal, however we can have viscosity as long as it is not \emph{shear} but \emph{bulk} viscosity, which adds onto the diagonal terms.

Inserting the FLRW metric into the Einstein equation yields the Friedmann equations: 
%
\begin{align}
\frac{\dot{a}^2}{a^2} &= \frac{8\pi G}{3} \rho - \frac{k}{a^2}  \\
\frac{\ddot{a}}{a} &= - \frac{4 \pi G}{3} \qty(\rho + 3 P)  \\
\dot{\rho} &= - 3 \frac{\dot{a}}{a} \qty(\rho + P)
\,.
\end{align}

The first two can be derived from the Einstein equations directly, the third comes from the ``conservation law'' \(\tensor{T}{_{\mu \nu }^{;\nu }} = 0\). 

They are not independent, only two are.
We have too many parameters: \(a\), \(\rho \) and \(P\), but only two independent equations, so we ``close'' the system of equations with an equation of state, commonly \(P = P(\rho ) = w \rho \). 

These equations of state describe many kinds of fluids (approximately):
dust with \(w = 0\), which means \(\rho \propto a^{-3}\); radiation with \(w = 1/3\), which means \(\rho \propto a^{-4}\); a cosmological constant with \(w = -1\), which means \(\rho = \const\).

In general as long as \(w \neq 1\) we have \(\rho \propto a^{-3(1+w)}\) and \(a \propto t^{2 / 3 (1+w)}\).

\subsection{The horizon problem}

The \textbf{particle horizon}, denoted as \(d_H (t)\), is given by 
%
\begin{align}
d_H (t) = a(t) \int_{0}^{t} \frac{c \dd{\tau }}{a(\tau )}
\,,
\end{align}
%
and it sets the radius of a sphere centered at an observer \(O\). The points inside this sphere have been able to have causal interactions with observer \(O\) in the time from the Big Bang to \(t\).

It is the proper distance (as measured today) which could have been travelled by light starting at the beginning and moving in a geodesic. 
It can be derived from the FLRW metric by assuming radial light-like motion
%
\begin{align}
\dd{s}^2 = - c^2 \dd{t}^2 + a(t)\frac{ \dd{r}^2}{1 - kr^2} = 0
\,,
\end{align}
%
and setting \(k = 0\):\footnote{This is a good approximation for early times, even if the universe is not flat.}
%
\begin{align}
c \dd{t} = \pm  a(t) \dd{r}
\,,
\end{align}
%
which we can use to calculate the \emph{comoving distance} from the point of emission to today, which we then multiply by the scale factor calculated at a chosen point. 

We know that the scale factor goes to 0 as \(t\) goes to 0, so the integral giving us \(r_H\) could diverge. We can show that \(d_H\) is finite as long as \(\alpha = 2 / 3 (1+w)\) is smaller than one, meaning that \(w > - 1/3\), which is equivalent to \(\ddot{a} < 0\).
In a decelerating universe, the particle horizon is finite.

In general, the calculation yields
%
\begin{align}
d_H (t) = \frac{3(1 +w)}{1 + 3w} ct
\,.
\end{align}

With \(w = 0\), a spatially flat matter-dominated universe, \(d_H = 3 ct\). This is called an Einstein-De Sitter universe. 
With \(w = 1/3\), a spatially flat radiation-dominated universe, we have \(d_H = 2 ct\). 

Another way to characterize causality is the Hubble radius: 
%
\begin{align}
r_C(t) = \frac{c}{H(t)}
\,.
\end{align}

The characteristic time of expansion is \(\tau (H ) =H^{-1}\). 
We can show that in a FLRW universe, typically after a Hubble time the scale factor doubles. 

Since 
%
\begin{align}
H(t) = \frac{2}{3 (1+w)} \frac{1}{t} 
\,,
\end{align}
%
we can write
%
\begin{align}
R_H \approx \frac{1 + 3w}{2} d_H (t) \approx d_H (t)
\,.
\end{align}

\todo[inline]{check! exercise}

This will \emph{not} happen in an inflationary universe.
The two are similar in a regular FLRW universe, while they differ a lot if there is inflation. 
The particle horizon takes into account all the past history of an observer, the Hubble radius does not care about it: it only described causal connections taking place in a time interval taking place in a Hubble time. 

Let us introduce the \emph{comoving Hubble radius}: \(r_H (t)\), given by 
%
\begin{align}
r_H (t) = \frac{r_C (t)}{a(t)}
\,.
\end{align}

Let us plot this for a matter or radiation-dominated FLRW universe.

In radiation domination, \(r_H \propto \sqrt{t}\), while in matter domination \(a \sim t^{2/3}\) so \(r_H \propto t^{1/3} \sim a^{1/2}\). 

This comoving radius is then always increasing, initially faster and then slower. 

Instead, consider the comoving particle horizon: \(d_H (t) / a(t)\), so just the integral in the definition of \(d_H(t)\): 
%
\begin{align}
\frac{d_H (t)}{a(t)} = \int \frac{c \dd{t}}{a} = \int \frac{ \dd{a}}{a} \underbrace{\frac{c}{a H}}_{r_H}
\,,
\end{align}
%
so we can see that the comoving \emph{particle horizon} is the logarithmic integral over the scale factor of the comoving \emph{Hubble radius}: as we mentioned before, this takes into account the whole past history. 

In a matter dominated universe, \(d_H = 2 r_C \approx 5 h^{-1} \SI{}{Gpc}\).

Now we discuss the horizon problem, which is best understood in a comoving plot. We neglect dark energy for simplicity.

If we choose a fixed comoving size \(\lambda \), we get in our model that in early times \(\lambda \) is super-horizon, then at a certain point it crosses the horizon, becoming smaller than \(r_H \). The time at which \(r_H = \lambda \) is called the \emph{horizon crossing} time, \(t_H (\lambda )\). 

For times earlier than \(t_H (\lambda )\), by definition it is impossible for points at a distance \(\lambda \) to be causally connected. 
This happens for every scale, and it means that for many regions we are interested in there cannot have been causal connection in the early universe. 
But, today we observe the universe to exhibit the same properties across the whole sky, even though the regions were causally disconnected earlier. 

This is most directly expressed in terms of CMB photons. They would have become causally connected at the quadrupole scale (separations of \SI{90}{\degree}) almost \emph{today}. 

We can compute the size of the horizon at the last scattering epoch: this subtends an angle in the sky of around \SI{1}{\degree}; however we observe photons with the same temperature on much larger scales, this was already seen by COBE with an angular resolution of \SI{7}{\degree}. 

\todo[inline]{re-do calculation!}

Photons which could not have been in causal contact in the HBB model are observed to have the same temperature. 

The inflationary solution to this issue is to think that, before the radiation-dominated epoch, the comoving Hubble radius decreased for a certain period of time.

This allows the parts of the sky to have been in causal contact in the early universe. 

This means that \(\ddot{a} > 0\) in the inflationary phase, or \(w < - 1/3\).
These are only the \emph{kinematics} of inflation, we are not yet discussing how it might come about. 

\end{document}
