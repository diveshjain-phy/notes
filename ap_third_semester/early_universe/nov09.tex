\documentclass[main.tex]{subfiles}
\begin{document}

\marginpar{Monday\\ 2020-11-9, \\ compiled \\ \today}

\todo[inline]{Recover first hour --- I was sleeping.}

For these models of inflation we find \(\delta \rho / \rho \sim \order{\lambda^{1/2}}\), where \(\lambda \) is the coupling constant between the inflaton field and the thermal bath.

Typically people considered values for \(\lambda \) between \num{.1} and \num{1}, which means we would get anisotropies of the order of 100:
way too large, since from CMB anisotropies we know that \(\delta \rho / \rho \sim \num{e-5}\).
In order to match this observational value we would need \(\lambda \sim \num{e-10}\), however this is not compatible with us being sure to have thermal equilibrium between the inflaton reheating and its thermal bath. 

Thermal equilibrium and sufficiently small density perturbations seem to be in conflict with each other. 

This is the reason why \textbf{chaotic inflationary models} (Linde 1983): disconnecting inflation from phase transitions. 
Suppose we have a flat potential \(V(\varphi ) \equiv V_0 \). 
This would lead to unending De Sitter inflation with \(- \infty < \varphi < \infty \); so instead of it we consider a roughly flat potential \(V(\varphi ) = (\lambda /4) \varphi^{4}\), with \(\lambda \ll 1\). 

The constraint we need to require is that \(V(\varphi ) \lesssim M_P^{4}\), which ensures that quantum gravity corrections are not needed.
Then, we must constrain \(\varphi \) to lie between plus and minus \(M_P / \lambda^{1/4}\). 
But if \(\lambda \ll 1\), \(\varphi \) will be able to attain initial values larger than the Planck mass: this allows for slow-roll inflation. 

These models are called chaotic because the values of \(\varphi \) are taken to be randomly distributed, and in some regions inflation will not take place. 
At the Planck time \(t = t_P\) this is required: there, \(\Delta E \Delta T \gtrsim 1\), so the potential \(V(\varphi )\) must have an uncertainty larger or equal than \(M_P\). 

Now having \(\lambda \sim \num{e-10}\) is not an issue: the inflaton is not coupled, and we are fine with this. 

The high temperature corrections to this potential are negligible.

As an example we consider the \textbf{Coleman-Weinberg potential}: 
%
\begin{align}
V(\varphi ) = \frac{B \sigma^{4}}{2}
+ B \varphi^{4} \qty[\log \qty(\frac{\varphi^2}{\sigma^2}) - \frac{1}{2}]
\,,
\end{align}
%
a typical potential which can characterize symmetry breaking. 
The peculiarity of this potential is the logarithm: it comes from the one-loop corrections for the interactions between the field and other particles. 

This has a minimum at \(\varphi = \sigma \), while near the origin it is quite flat: we can approximate it as 
%
\begin{align}
V(\varphi ) \approx \frac{B \sigma^{4}}{2} - \lambda \frac{\varphi^{4}}{4} 
\,,
\end{align}
%
where \(\lambda = \abs{4 B \log (\varphi^2 / \sigma^2)} \approx \qty(\num{10} \divisionsymbol \num{100}) B\). 

Very close to the origin, we will have \(V(\varphi ) \approx B \sigma^{4} / 2 = \const\), so 
%
\begin{align}
H^2 \approx \frac{8 \pi G}{3} V(\varphi ) \approx \frac{4 \pi G}{3} B \sigma^{4}
\,.
\end{align}

In the original model of new inflation, this was considered with \(B \approx \num{e-3}\), and \(B = \frac{25}{16} \alpha^2 _{\text{GUT}}\), and \(\alpha _{\text{GUT}} = g _{\text{GUT}}^2 / 4 \pi \). 

Since it pertained to a GUT, we also had \(\sigma \sim T_C \sim \SI{e15}{GeV}\), therefore \(H^2 \approx \qty(\SI{e10}{GeV})^2\). 
\todo[inline]{why?}

Do these kinds of models actually work? The first question is: can they solve the horizon problem? The number of \(e\)-folds is given by 
%
\begin{align}
N = \int _{t_i}^{t_F} H \dd{t} = \int_{\varphi _i}^{\varphi _f} \frac{H}{\dot{\varphi}} \dd{\varphi }
\,,
\end{align}
%
and recalling \(3 H \dot{\varphi} \approx V' (\varphi )\), plus \(H^2= 8 \pi G V(\varphi ) /3\), we find 
%
\begin{align}
N = - 8 \pi G \int_{\varphi _i}^{\varphi _f} \frac{V(\varphi )}{V'(\varphi )} \dd{\varphi }
\,.
\end{align}

Using the near-origin approximation, we get \(V'(\varphi ) \approx - \lambda \varphi^3\), so 
%
\begin{align}
N &\approx  - 3 H^2 \int_{\varphi _i}^{\varphi _f} \frac{ \dd{\varphi }}{- \lambda \varphi^3}   \\
&\approx 3 \lambda^{-1} H^2 \int_{\varphi _i}^{\varphi _f} \frac{ \dd{\varphi }}{\varphi^3} = \frac{3}{2} \frac{H^2}{\lambda } \qty( \frac{1}{\varphi _i^2} - \frac{1}{\varphi_f^2})
\,.
\end{align}

When does inflation end?
We know that slow-roll holds as long as \(\abs{V''} < H^2\), so we can say that \(\varphi _f\) is determined by \(\abs{V''(\varphi _f)} \sim 10 H^2\), and we know that \(V''(\varphi ) = - 3 \lambda \varphi^2\): 
therefore, \(\varphi _f = - 3^{1/2} H / \lambda^{1/2}\). 

If \(B \approx \num{e-3}\), this means 
%
\begin{align}
\varphi _f \approx \frac{3 H^2}{\lambda } \approx \qty(30 \divisionsymbol 300) H^2
\,,
\end{align}
%
and we know that \(H^2 \approx \qty(\SI{e10}{GeV})^2\).

Then, an initial condition which works is \(\varphi _i \approx \num{e8} \divisionsymbol \SI{e9}{GeV}\). 
This means that \(\varphi _i \sim H / 10\), which means that \(N \gtrsim 1000\) easily.
So, we can solve the horizon and flatness problems without issue. 

However, we will see that there are indeed problems with this model. 
The amplitude of the primordial fluctuations is too large, the quantum fluctuations of the inflaton are too large, invalidating the semiclassical approach. 

\end{document}
