\documentclass[main.tex]{subfiles}
\begin{document}

\marginpar{Wednesday\\ 2020-12-9, \\ compiled \\ \today}

CDM particles, definitionally, decouple when they are \emph{not} relativistic, so \(z_f = m_\psi / T > 1\) at freezeout; recall that freezeout is defined as the moment when the inequality \(\Gamma _A / A \lesssim 1\) starts to be satisfied, where \(\Gamma _A\) refers to the annihilation process \(\psi \overline{\psi} \leftrightarrow X \overline{X}\). 

If we define \(y = n_\psi / s\), we can draw the usual decay curve \(y = y(z)\), which is constant and then has a decay (something like an inverted sigmoid); at decoupling the amount stops changing. 
The Boltzmann equation reads 
%
\begin{align}
\dv{y}{z} = \frac{- zs \expval{\sigma _A \abs{v}}}{H (z=1)} \qty(y^2 - y^2 _{\text{eq}})
\,.
\end{align}

Typically we can model the term \(\expval{\sigma_A \abs{v}}\) as a powerlaw: 
%
\begin{align}
\expval{\sigma_A \abs{v}} &\sim T^{n} \\
\expval{\sigma_A \abs{v}} &= \sigma_0 \qty( \frac{T}{m})^{n} = \sigma_0 \overline{z}^{n}
\,,
\end{align}
%
therefore the differential equation for the abundance \(y\) reads 
%
\begin{align}
\dv{y}{z} &= - \frac{\lambda}{z^{2 + n}} \qty(y^2 - y^2 _{\text{eq}}) \\
\lambda = \frac{z^3 s \sigma_0 }{H(z=1)}
\,,
\end{align}
%
but recall that 
%
\begin{align}
z^3 &= \qty( \frac{m}{T})^3  \\
s &= \frac{2 \pi^2}{45} g_* T^3  \\
H(z=1) &=  \qty(\frac{8 \pi G}{3})^{1/2} \qty(\frac{\pi^2}{30})^{1/2}
g_*^{1/2} \underbrace{T^2}_{= m_\psi^2}
\,,
\end{align}
%
and with all these substitutions we can write 
%
\begin{align}
\lambda \approx \num{.264} m_P m_\psi \sigma_0 \frac{g_{*s}}{g_*^{1/2}}
\,.
\end{align}

Further, we can model the equilibrium abundance as 
%
\begin{align}
y _{\text{eq}} = \num{.145} \qty(\frac{g_*}{g_{*s}}) z^{3/2} e^{z}
\,,
\end{align}
%
with \(z > 1\). 

Now, let us define \(\Delta = y - y _{\text{eq}} \); then, denoting derivatives with respect to \(z\) with primes we get
%
\begin{align}
\Delta' = - y' _{\text{eq}} - \frac{\lambda}{z^{2+n}} \qty(\Delta + 2 y _{\text{eq}}) \Delta 
\,,
\end{align}
%
which can be solved by distinguishing two regimes: the first is for early times, defined by \(1 < z < z_f\), meaning that the particle is relativistic but still coupled.
This tells us that both \(\Delta \) and \(\Delta '\) are small; with this we can write 
%
\begin{align}
\Delta &\approx \frac{z^{2+n}}{\lambda } \frac{- y' _{\text{eq}}}{\Delta + 2 y _{\text{eq}}}  \\
&\approx \frac{z^{2+n}}{\lambda } \frac{- y' _{\text{eq}}}{2 y _{\text{eq}}}  \\
&\approx \frac{z^{2+n}}{\lambda }
 \frac{y _{\text{eq}}}{2 y _{\text{eq}}} = \frac{z^{2+n}}{2 \lambda }
\,,
\end{align}
%
since when we differentiate \(y _{\text{eq}}\) one term is negligible.
\todo[inline]{I think, to check.}

On the other hand, at late times \(z \gg z_f\) we get \(y(z) \gg y _{\text{eq}} (z)\),\footnote{This is true since \(y\) is the \emph{actual} amount of particles in a comoving volume, while \(y _{\text{eq}}\) is the amount which \emph{would} be reached if there were equilibrium, which cannot happen due to decoupling.} therefore \(\Delta \approx y\). 
With this approximation, we get 
%
\begin{align}
\Delta ' = - \frac{\lambda}{z^{2+n}} \Delta^2
\,,
\end{align}
%
and solving this we find 
%
\begin{align}
- \frac{\Delta '}{\Delta^2} &\approx \frac{\lambda }{z^{2 + n}}  \\
\frac{1}{\Delta _\infty } - \frac{1}{\Delta _f} &\approx \eval{- \frac{\lambda}{n+1} z^{-n-1}}_{z = z_f}^{z = \infty } \\
\frac{1}{\Delta_{\infty }} - \frac{1}{\Delta _f} &\approx  + \frac{\lambda }{z_f^{n+1} (n+1) }
\,,
\end{align}
%
but \(y_\infty < y_f\), meaning that we can neglect the \(y_f\) term and so 
%
\begin{align}
y_\infty \approx \frac{z_f^{n+1}(n+1)}{\lambda } \approx
\frac{\num{3.79} (n+1) z_f^{n+1}}{(g_{*s} / g_*^{1/2}) m_P m_\psi \sigma_0 }
\,.
\end{align}

From this, using \(n_\psi = s_0 y_\infty \) and the expression \(s_0 \approx \SI{2e3}{cm^{-3}}\) (if one neutrino species is relativistic today), then we can calculate 
%
\begin{align}
\Omega_{\psi_0 } h^2 = \frac{\num{.75e9}(n+1)z_f^{n+1}}{(g_{*s} / g_*^{1/2}) m_P  \sigma_0} \SI{}{GeV^{-1}} 
\,.
\end{align}

Here we used \(\Omega _{\psi_0 } = m_\psi n_{\psi_0 }  / \rho _{\text{crit}}\), with \(\rho _{\text{crit}} \approx \SI{e4}{eV cm^{-3} h^2}\).
We appear to have eliminated the dependence on \(m_\psi \), but this is not completely the case: (something) still depends at least weakly on it. 

The crucial result from this manipulation is 
%
\begin{align}
\Omega_{\psi_0 } h^2 \propto \frac{1}{\sigma_0 }
\,.
\end{align}

This is somewhat intuitive: if the particle interacts more then its abundance has more time to decrease with the Boltzmann suppression. 

Manipulating the expression with some typical values, we get 
%
\begin{align}
\Omega_{\psi_0} h^2 \approx \order{1} (n+1) \qty(\frac{z_f}{10})^{n+1}
\qty(\frac{g_*}{100})^{1/2} \qty(\frac{100}{g_{*s}}) \frac{\SI{e-38}{cm^2}}{\sigma_0 }
\,.
\end{align}

The natural normalization arising for \(\sigma_0\) is \SI{e-2}{pbn} (picobarns), a very small cross-section, which is the typical order of magnitude of weak-interaction cross-sections. 
This is what is called the \textbf{WIMP miracle}, where WIMP means Weakly-Interacting Massive Particle.

The freezout epoch is the one at which the abundance \(y\) departs from its equilibrium value: then, we ask that \(\Delta (z_f) = y(z_f) - y _{\text{eq}} (z_f) = c y _{\text{eq}}\), with \(c\) being a coefficient of order unity.

Plugging in the expression we derived earlier, we get 
%
\begin{align}
\Delta(z_f) = \lambda^{-1} \frac{z_f^{n+2} y_{\text{eq}}(z_f)}{2 y _{\text{eq}}(z_f) + \Delta (z_f)} = \frac{z_f^{n+2}}{\lambda } \frac{1}{2 + c}
\,,
\end{align}
%
therefore 
%
\begin{align}
\frac{z_f^{n+2}}{\lambda } \frac{1}{2 + c}
&= c a z_f^{3/2} e^{-z_f}  \\
a &= \num{.45} \frac{g_\psi }{g_{*s}}
\,,
\end{align}
%
a transcendental equation which can be solved iteratively, yielding 
%
\begin{align}
z_f^{(0)} = \log \qty[c (c+2) a \lambda ]
\,,
\end{align}
%
and a good approximation is found to be \(c ( c+ 2) = n+1\). 
The second iteration then yields 
%
\begin{align}
z_f^{(1)} = \log \qty[c (c+2) a \lambda ] - \qty( \frac{1}{2} + n) \log \qty[c (c+2) a \lambda ]
\,,
\end{align}
%
which means that \(z_f\) does indeed depend on the mass of the particle, although it is a weak dependence since it is in the logarithm. 

As an example, if \(m_\psi \) were of the order of \SI{100}{GeV} and if we had \(\sigma_0 \sim \SI{e-38}{cm^{-2}}\), then we would get 
%
\begin{align}
z_f \approx \begin{cases}
    17 & \text{for } n=0 \\
    14 & \text{for } n=1 \\
    12 & \text{for } n=2 
\end{cases}
\,.
\end{align}

This means that \(T_f \approx m_\psi / 10\), therefore \(T_f \approx 10 \divisionsymbol \SI{100}{Gev}\) is the typical order of magnitude of the decoupling temperature. 
The relics we discussed are \emph{thermal}, but non-thermal relics are also possible; their abundance is much more model-dependent. 
For example, axions might be produced through a ``misalignment mechanism'', and other relics might be produced through a ``freeze-in mechanism''.

\chapter{Cosmological perturbation within GR}

We will treat the ``gauge issue'' for perturbations. Einstein's equations are invariant under diffeomorphisms, so we must be careful and use invariant quantities. 
We already used 
%
\begin{align}
\mathcal{J} = - H \frac{ \delta \rho}{\rho } - \hat{\Phi}
\,,
\end{align}
%
where \(\hat{\Phi}\) is a scalar perturbation of the metric. 

Then, we will try to get a proper treatment of cosmological perturbations within GR.
We have already seen how to treat them in a Newtonian way (i.e.\ Jeans instability). 
This is not ok when the wavelength of the perturbation is larger than the Hubble radius.

The Poisson equation reads 
%
\begin{align}
\nabla^2 \varphi = 4 \pi G \delta \rho  = 4 \pi G \overline{\rho} \delta 
\,,
\end{align}
%
so in terms of characteristic lengths we get 
%
\begin{align}
\frac{\varphi}{\lambda^2 _{\text{phys}}} \sim \frac{3}{2} H^2 \delta 
\,,
\end{align}
%
therefore 
%
\begin{align}
\varphi \sim \qty(\frac{\lambda _{\text{phys}}}{\lambda _H})^2 \delta 
\,,
\end{align}
%
and reintroducing the speed of light 
%
\begin{align}
\frac{\varphi}{c^2} \sim \qty(\frac{\lambda _{\text{phys}}}{\lambda _H})^2 \delta 
\,.
\end{align}

We know that typically \(\delta \sim \num{e-5}\) for primordial perturbations, so when \(\lambda _{\text{phys}} \ll \lambda _H\) the Newtonian treatment is fine; while when the perturbations are outside the horizon the perturbations are frozen. 

We will adopt a perturbative approach, which in GR is a perturbation of the geometry itself. 
Consider a generic tensor field \(T\): a perturbation of this quantity is defined as 
%
\begin{align}
\Delta T = T - T_0 
\,,
\end{align}
%
where \(T\) is evaluated in the perturbed universe, while \(T_0 \) is evaluated in the unperturbed FLRW spacetime. 
We are then comparing two tensors, however we can do so only if they are defined at two different spacetime locations. 
This is not allowed in differential geometry: in order to make the comparison meaningful we need a one-to-one correspondence between the perturbed and unperturbed spacetimes (denoted respectively as \(\mathcal{M}\) and \(\mathcal{M}_0\)); this amounts to making a specific gauge choice. 

A gauge transformation is a change of the map \(\mathcal{M}_0 \to \mathcal{M}\) which \emph{keeps the coordinates on \(\mathcal{M}_0\) fixed}; this is different from a coordinate transformation.

The gauge issue comes from the freedom of always making a gauge transformation. 

A coordinate system is defined by a threading of spacetime into lines (corresponding to fixed spatial coordinates) and by a slicing of spacetime into hypersurfaces (corresponding to fixed time).

A gauge transformation can be defined in a completely coordinate-free way.

We can change our point of view: instead of changing the point in \(\mathcal{M}\) to which our point in \(\mathcal{M}_0\) maps, we can see the transformation as a change in the point in \(\mathcal{M}_0\) from which we start to reach the fixed point in \(\mathcal{M}\).

If we have a quantity \(T\) which is a scalar under coordinate transformation, then it still \emph{can} change if we make a gauge transformation. 

\end{document}
