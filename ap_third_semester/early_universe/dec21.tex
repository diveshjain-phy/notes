\documentclass[main.tex]{subfiles}
\begin{document}

\marginpar{Monday\\ 2020-12-21, \\ compiled \\ \today}

Let us continue with gauge-invariant cosmological perturbations.
Let us now give a gauge-invariant definition for the matter velocity: 
%
\begin{align}
2 v_s = 2 v^{\parallel} + \chi^{\parallel, \prime}
\,,
\end{align}
%
where, as usual, by ``matter'' we mean anything which goes on the right-hand side of the Einstein equations. 

This velocity is related to the amplitude of the shear tensor for the matter velocity: specifically, from the shear tensor \(\sigma_{\mu \nu }\) we can define the quantity \(\qty(\sigma^{ij} \sigma_{ij} /2 )^{1/2}\). 

For the scalar part, we have (following the notation by Bardeen)
%
\begin{align}
\epsilon_m &= \delta \rho + \rho_0' \qty(v^{\parallel} + \omega^{\parallel} )  \\
\rho_0 &= \rho_0 (\eta )
\,.
\end{align}

Note that energy density perturbations themselves are not gauge invariant. 
The quantity \(\epsilon _m\) corresponds to \(\delta \rho \) in the gauge where \(v^{\parallel} + \omega^{\parallel}\), which means that we are selecting constant-\(\eta \) hypersurfaces which are orthogonal to the worldlines of the fluid: the fluid's rest frame. 

The quantity \(v^{\parallel} + \omega^{\parallel}\) enters into the expression for \(T^{0}_{i}\), which describes the momentum flux of the fluid. 
We also define 
%
\begin{align}
2 E_g = 2 \delta \rho + \rho_0' \qty(2 \omega^{\parallel} - \chi^{\parallel, \prime})
\,,
\end{align}
%
which is also equal to \(\delta \rho \) in the gauge in which \(2 \omega^{\parallel} + \chi^{\parallel, \prime} = 0\), which is the zero-shear or Poisson gauge. 

We can also construct vector perturbations: 
%
\begin{align}
\Psi_i = \omega^{\perp}_i - \chi_i^{\perp, \prime}
\,.
\end{align}

This is related to the amplitude of the vector geometric component of the geometric shear \(\sigma_{\mu \nu }\). 
This term describes frame-dragging effects. 

The matter velocity can be described as 
%
\begin{align}
V^{i}_{s} = v^{i}_{\perp} + \chi^{i, \prime}_{\perp}
\,.
\end{align}

Also, we can define 
%
\begin{align}
V^{i}_{c} = v^{i}_{\perp} + \omega^{i}_{\perp}
\,.
\end{align}

This is related to the amplitude of the vorticity tensor \(\omega_{\mu \nu }\) (and, specifically, the quantity \(\qty(\omega^{ij} \omega_{ij} / 2)^{1/2}\)). 

As for tensor perturbation modes, linear tensor perturbations are automatically gauge-invariant (at linear order, at least): 
%
\begin{align}
\chi^{T, \prime}_{ij} =
\chi^{T}_{ij} 
\,.
\end{align}

Let us write the equations of motion for these linear cosmological perturbations. 
We start from the Einstein equations \(G_{\mu \nu } = 8 \pi G T_{\mu \nu }\), and the Bianchi identities \(T^{\mu \nu }_{; \nu } = 0\). 
The Einstein tensor is defined as 
%
\begin{align}
G_{\mu \nu } &= R_{\mu \nu } - \frac{1}{2} g_{\mu \nu } R  \\
R_{\mu \nu } &= R^{\alpha }_{\mu \alpha \nu }  \\
R^{\alpha }_{\mu \nu \beta }&\sim \pdv{\Gamma }{x} + \Gamma^2  \\
\Gamma^{\mu }_{\nu \rho } &\sim g^{-1} \pdv{g}{x}
\,.
\end{align}

We can compute the nonvanishing Christoffel symbols in the unperturbed case (see, for example, the notes for Theoretical Cosmology), and then the perturbations of these: 
%
\begin{align}
\delta \Gamma^{0}_{00} &= \psi '  \\
\delta \Gamma^{0}_{0i} &= \partial_{i} \psi + \frac{a'}{a} \partial_{i} \omega^{\parallel}  \\
\delta \Gamma^{i}_{00} &= \frac{a'}{a} \partial^{i} \omega^{\parallel} + \partial^{i} \omega^{\parallel, \prime} + \partial^{i} \psi  \\
\delta \Gamma^{0}_{ij} &= -2 \frac{a'}{a} \psi \delta_{ij} - \partial_{i} \partial_{j} \omega^{\parallel} - 2 \frac{a'}{a} \phi \delta_{ij} 
- \phi ' \delta_{ij} + \frac{a'}{a} D_{ij} \chi^{\parallel} + \frac{1}{2} D_{ij} \chi^{\parallel, \prime}  \\
\delta \Gamma^{i}_{0j} &= - \phi \delta^{i}_{j} + \frac{1}{2} D^{i}_{j} \chi^{\parallel, \prime}  \\
\delta \Gamma^{i}_{jk} &= \dots
\,.
\end{align}

The spatial components of the metric can be written as 
%
\begin{align}
g_{ij} = \underbrace{a^2(\eta ) \qty[1 - 2 \phi ] \delta_{ij}}_{a^2(\eta , \vec{x})} + \dots
\,,
\end{align}
%
where we can express 
%
\begin{align}
a(\eta , \vec{x}) = a(\eta ) \qty(1 - \phi (\eta , \vec{x}))
\,,
\end{align}
%
therefore \(\delta a = - a \phi \), which means that 
%
\begin{align}
\delta \qty( \frac{a^{\prime}}{a}) = - \phi '
\,.
\end{align}

The unperturbed Ricci tensor reads 
%
\begin{align}
R_{00} &= - 3 \frac{a''}{a} + 3 \qty( \frac{a'}{a})^2  \\
R_{ij} &=  \qty[\frac{a''}{a} + \qty(\frac{a'}{a})^2] \delta_{ij} 
\,.
\end{align}

Its perturbation is 
%
\begin{align}
\delta R_{00} &= \frac{a'}{a} \nabla^2 \omega^{\parallel} + \nabla^2 \omega^{\parallel, \prime} + 3 \phi^{\parallel} + 3 \frac{a'}{a} \phi' + 3 \frac{a'}{a} \psi '  \\
\delta R_{0i} &= \frac{a'}{a} \partial_{i} \omega^{\parallel} + \qty(\frac{a'}{a})^2 \partial_{i} \omega^{\parallel} +2 \partial_{i} \phi' + 2 \frac{a'}{a} \partial_{i} \psi + \frac{1}{2} \partial_{k} D^{k} \qty(\chi^{, i}_{\parallel})'  \\
\delta R_{ij} &= \dots
\,.
\end{align}

\todo[inline]{Copy full expressions from the notes.}

The Ricci scalar \(R\) is given by \(R = (6/a^2) (a' / a)\) in flat FRLW, while its perturbation is 
%
\begin{align}
\delta R &= \frac{1}{a^2} \qty(- 6 \frac{a'}{a} \nabla^2\omega^{\parallel} - 2 \nabla^2 \omega^{\parallel, \prime} - 2 \nabla^2 \psi - 6 \phi^{\parallel} - 6 \frac{a'}{a} \psi' - 18 \frac{a'}{a} \phi ' - 12 \frac{a''}{a} \psi + 4 \nabla^2 \phi + \partial_{k} \partial^{i} D^{k}_{i} \chi^{\parallel})
\,.
\end{align}

This expression is fully general, in a specific gauge it can be significantly simplified.  
The stress-energy tensor we will use is given by 
%
\begin{align}
T_{\mu \nu } = \rho u_\mu u_\nu + p h_{\mu \nu } + \Pi_{\mu \nu }
\,,
\end{align}
%
where \(\Pi^{\mu }_{\mu } = \Pi_{\mu \nu } u^{\nu } = 0\). This allows us to account for imperfections in the fluid. The only nonvanishing components of \(\Pi \) are the spatial ones \(\Pi_{ij}\). 
This is true in any frame. 

We can write it as 
%
\begin{align}
\Pi_{ij} = D_{ij} \Pi^{\parallel} + 2\Pi^{\perp}_{(i, j)} + \Pi^{T}_{ij}
\,,
\end{align}
%
where, as usual, \(D_{ij} = \partial_{i} \partial_{j} - (1/3) \delta_{ij} \nabla^2\). 
The component 
%
\begin{align}
- T^{0}_{0} &= \rho_{0} (\eta ) + \delta \rho (\eta , \vec{x})  \\
&= \rho_0 (\eta ) \qty(1 + \delta ) 
\,,
\end{align}
%
while 
%
\begin{align}
p = p_0 (\eta ) \qty(1 + \Pi _L) 
\,,
\end{align}
%
where \(\Pi _L = \delta p / p_0 (\eta )\). 

The unperturbed spatial components of the stress-energy tensor read:
%
\begin{align}
T^{i}_{j} &= p_0 (\eta ) \qty[ \qty(1 + \Pi _L) \delta^{i}_{j} + \Pi_{T, j}^{i}]  \\
\Pi_{T, j}^{i} &= \frac{\Pi^{i}_{j}}{p_0 (\eta )}
\,.
\end{align}

The perturbation reads 
%
\begin{align}
\delta T^{0}_{i} = \qty(\rho_0 + p_0 ) \qty(v_i + \omega _i )
\,.
\end{align}

This is quite general: it is the density of the \(i\)-component of the fluid's momentum, or the flux of energy in the \(i\)-th direction.

The linearly perturbed \(00\) EFE reads 
%
\begin{align}
\frac{3 a'}{a} \qty(\hat{\phi} + \frac{a'}{a} \psi ) - \nabla^2 \qty(\hat{\phi} + \frac{a'}{a} \sigma ) = - 4 \pi G a^2 \delta \rho 
\,,
\end{align}
%
where \(\hat{\phi} = \phi + (1/6) \nabla^2 \chi^{\parallel}\) and \(\sigma = - \omega^{\parallel} + (1/2) \chi^{\parallel, \prime}\). 

The \(0i\) equation is 
%
\begin{align}
\hat{\phi}' + \frac{a'}{a} \psi = - 4 \pi G a^2 \qty(\rho_0 + p_0 )V
\,,
\end{align}
%
where \(V = v^{\parallel} + \omega^{\parallel}\). 
These two are not really ``evolution'' equations, they should be interpreted as constraints for the evolution of the other components. 
On the other hand, from the trace of the \(ij\) equations we find 
%
\begin{align}
\hat{\phi}'' + 2 \frac{a'}{a} \hat{\phi}' + \frac{a'}{a} \psi +
\qty[2 \qty(\frac{a'}{a})' + \qty(\frac{a'}{a})^2] \psi 
= 4 \pi G a^2 \qty(\Pi _L + \frac{2}{3} \nabla^2 \Pi _T)p_0 
\,,
\end{align}
%
where the \(T\) in \(\Pi _T = \Pi^{\parallel} / p_0 (\eta )\) means ``traceless''. 
From the traceless part of these equations we find 
%
\begin{align}
\sigma ' + 2 \frac{a'}{a} \sigma + \hat{\phi} - \psi = 8 \pi G a^2 \Pi _T p_0 
\,.
\end{align}

So far we have not chosen a gauge; let us now put ourselves in the Poisson gauge. Here, 
%
\begin{align}
\omega^{\parallel} = 0 = \chi^{\parallel}
\,.
\end{align}

Then, \(\hat{\phi} = \phi = - \Phi _H\), and \(\psi = \Psi _A\). 
Now \(\sigma = - \omega^{\parallel} + (1/2) \chi^{\parallel, \prime} = 0\). 
Then, the equations reads 
%
\begin{align}
\phi - \psi &= 8 \pi G a^2 \Pi_T p_0  \\
\Phi _H + \Psi _A  &= - 8 \pi G a^2 \Pi _T  p_0 
\,.
\end{align}

If the anisotropic stress can be neglected, then \(\phi = \psi \). 
If this is the case, then we can write the evolution equation in a simpler way: \(\Pi _T\) vanishes, by definition \(\Pi _L p_0 = \delta p\), which we can split into 
%
\begin{align}
\delta p = c_s^2 \delta \rho + \delta p _{\text{non-adiab}}
\,,
\end{align}
%
and considering only the adiabatic part we find 
%
\begin{align}
\Phi _H'' + 3 \qty(1 + c_s^2) \frac{a'}{a} \Phi _H' + 
\qty[2 \qty(\frac{a'}{a})' + \qty(1 + 3 c_s^2) \qty(\frac{a'}{a})^2
- c_s^2 \nabla^2] \Phi _H = 0
\,,
\end{align}
%
which is an isolated evolution equation for \(\Phi _H\), a wave-like propagation equation. 

The Poisson gauge is the most Newton-like one: the evolution equation becomes 
%
\begin{align}
- \nabla^2 \Phi _H &= 4 \pi G a^2 \underbrace{\qty(\delta \rho - \frac{3 a'}{a} (\rho_0 + p_0 )V )}_{\epsilon_{m}}  \\
&= 4 \pi G a^2\epsilon_m
\,,
\end{align}
%
a Poisson equation. 

For the vector perturbation \(\Psi_i\), we find 
%
\begin{align}
\nabla^2\Psi _i = 16 \pi G a^2 \qty(\rho_0 + p_0 ) V_{i, c}
\,,
\end{align}
%
while for the tensor perturbations, starting from the traceless part of the \(ij\) EFE, we get 
%
\begin{align}
\chi_{ij}^{\prime \prime, T} + 2 \frac{a'}{a} \chi^{T, \prime}_{ij} - \nabla^2 \chi^{T}_{ij} = 16 \pi G a^2p_0 (\eta ) \Pi^{T}_{ij}
\,.
\end{align}

We should also perturb for the matter source of the equations: \(T^{\mu \nu }_{; \nu } =0 \). 
The energy density continuity equation (\(\mu = 0\)) reads 
%
\begin{align}
\delta \rho ' + \frac{3 a'}{a} \qty(\delta p + \delta \rho ) - 3 \qty(\rho_0 + p_0 )  \hat{\phi}' + \qty(\rho_0 + p_0 ) \nabla^2 \qty(V + \sigma ) &= 0
\,,
\end{align}
%
while for \(\mu = i\) we get 
%
\begin{align}
V' + \qty(1 + 3 c_s^2) \frac{a'}{a} V + \psi + \frac{1}{(\rho_0 + p_0 )} \qty(\delta p + \frac{2}{3} p_0 \nabla^2 \Pi _T) &= 0
\,.
\end{align}

The curvature perturbation on uniform energy density hypersurfaces is 
%
\begin{align}
\zeta = - \hat{\phi} - H \frac{ \delta \rho }{\rho_0 } = - \hat{\phi} - \frac{a'}{a} \frac{ \delta \rho }{\rho_0 '}
\,,
\end{align}
%
where, as usual, \(\hat{\phi} = \phi + (1/6) \nabla^2 \chi^{\parallel}\). 

In a uniform energy density gauge \(\zeta = - \hat{\phi}\). 
On super-horizon scales, the Laplacian in the energy density continuity equation can be taken to vanish, and we can express it as 
%
\begin{align}
\zeta ' = - \frac{a'}{a} \frac{ \delta p}{\rho_0 + p_0 }
\,,
\end{align}
%
but we must evaluate this in the uniform energy density gauge the adiabatic contribution to the pressure perturbation vanishes, therefore we are left with 
%
\begin{align}
\zeta ' = - \frac{a'}{a} \frac{ \delta p _{\text{non-adiabatic}}}{\rho_0 + p_0 }
\,.
\end{align}

For single-field models of slow-roll inflation, we find 
%
\begin{align}
\delta p _{\text{non-adiabatic}} \propto \frac{k^2 \Phi _H}{a^2} \approx 0 \implies \zeta \approx \const 
\,.
\end{align}

The continuity equation for vector perturbations reads 
%
\begin{align}
\qty[(\rho_0 +p_0 ) V_{ic}]' + \frac{4 a'}{a} (\rho_0 + p_0 ) V_{ic} 
= - \nabla_k \qty(\Pi^{\perp, k}_{, i} + \Pi^{\perp, k}_{i})
\,.
\end{align}

By Kelvin's circulation theorem, vorticity is conserved along trajectories unless there are dissipative effects.
Then, the divergence on the right-hand side of this equation vanishes, therefore we can write the left-hand side as 
%
\begin{align}
a^3 \qty(\rho_0 + p_0 ) V_{ic} a = \const
\,.
\end{align}

This amounts to a momentum times \(a\), so the equation represents the conservation of the intrinsic angular momentum. 

\end{document}
