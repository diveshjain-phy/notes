\documentclass[main.tex]{subfiles}
\begin{document}

\marginpar{Monday\\ 2020-10-12, \\ compiled \\ \today}

As we saw last time, inflation provides an ``attractor solution'' to the flatness problem. 

We must impose 
%
\begin{align}
\frac{1 - \Omega _i^{-1}}{1 - \Omega_0^{-1}} \gtrsim 1
\,
\end{align}
%
in order to solve the flatnesss problem. 
Since \(( 1- \Omega^{-1}) \rho a^2\) is a constant, this ratio is equal to 
%
\begin{align}
\frac{1 - \Omega _i^{-1}}{1 - \Omega_0^{-1}}
= 
\frac{\rho_0 a_0^2}{\rho _i a_i^2} 
= 
\frac{\rho_0 a_0^2}{\rho _{\text{eq}} a _{\text{eq}}^2}
\frac{{\rho _{\text{eq}} a _{\text{eq}}^2}}{\rho _f a_f^2} 
\frac{\rho _f a _f^2}{\rho _i a_i^2} 
\,,
\end{align}
%
and since \(\rho \propto a^{-3 (1+w)}\) the term we are considering scales like \(\rho a^2 \propto a^{-(1 + 3w)}\). 
Substituting this in, accounting for the fact that the first term is in the matter-dominated epoch while the second is in the radiation-dominated one we have 
%
\begin{align}
\frac{1 - \Omega _i^{-1}}{1 - \Omega_0^{-1}}
&= 
\qty(\frac{a_0 }{ a _{\text{eq}}})^{-1} 
\qty(\frac{a _{\text{eq}}}{a_f})^{-2}
\qty(\frac{a_f}{ a_i})^{-(1 + 3 w_f)}
e^{N \abs{1 + 3 w_f}} = \frac{1 - \Omega_i^{-1}}{1 - \Omega_0^{-1}} X
\,,
\end{align}
%
where \(w_f\) is the equation of state in the inflationary phase, while 
%
\begin{align}
X &= \frac{a_0 }{ a _{\text{eq}}}\qty(\frac{a _{\text{eq}}}{a_f})^2 \\
&\approx \num{60}
\,.
\end{align}


Then, 
%
\begin{align}
e^{N \abs{1 + 3 w_f}} &\gtrsim \underbrace{\frac{1 - \Omega _i^{-1}}{1-\Omega_0 }}_{\gtrsim 1} X \\
N _{\text{min}} &= \frac{\log X}{\abs{1 + 3 w_f }} \approx 60 \divisionsymbol 70
\,,
\end{align}
%
where we assumed \(w_f \sim -1\). There is some model dependence, specifically regarding the transition from inflation to radiation domination. 

It is interesting that this is similar to the number of \(e\)-folds needed to solve the horizon problem. 

We can also write the expression, defining \(N = p N _{\text{min}}\), as
%
\begin{align}
1 - \Omega_0^{-1} = \frac{1 - \Omega_i^{-1}}{\exp((p -1) N _{\text{min}} (1 + 3 w_f))}
\,,
\end{align}
%
so, taking \(p = 2\) and \(w_i = -1\) we have 
%
\begin{align}
1 - \Omega_0^{-1} = (1 - \Omega _i^{-1} ) e^{-2 N _{\text{min}}} = (1-\Omega _i^{-1}) X^{-1}
\,.
\end{align}

Constraining \(\Omega_0\) allows us to constrain \(\Delta N = N - N _{\text{min}}\). 
Right now we have \(\abs{1 - \Omega_0 } < \SI{.4}{\percent}\) (at \SI{95}{\percent} CL) we can say \(\Delta N \gtrsim 5\).

\subsection{The flatness problem as an age problem}

Consider the HBB model, at \(t < t _{\text{eq}}\): the radiation-dominated epoch. 

The Planck time, the only characteristic time of the universe a priori, is \(t _{\text{Pl}} \approx \SI{e-43}{s}\). 
If the universe is spatially closed, then \(t _{\text{collapse}} = 2 t_m\). 
We would expect for the time of collapse to be of the order of the Planck time. 

If the universe was spatially open, we would expect that for \(t > t_* \sim t _{\text{Pl}}\) we would have curvature domination: \(a (t)/ a(t _*) \sim t / t_* \).

Therefore, we would have \(t_0 = t _{\text{Pl}} T _{\text{Pl}} / T_0 \sim \SI{e-11}{s}\). 

In order for this to not happen, we need the energy density term to finely balance the curvature term in the first Friedmann equation.

\subsection{The unwanted relics problem}

This is also known as the ``magnetic monopoles problem''.
Consider a massive particle \(X\), such that \(\Omega_{0X} \gg 1\). 

Typically, \(\Omega_{0X} \sim 1 / \sigma _A\). 

Historical examples are cosmic topological defects, arising from the SSB of some gauge theory. 
Also, we could have cosmic strings: one-dimensional defects, arising from the SSB of \(U(1)\).
Magnetic monopoles could come from a Grand Unified Theory.

Domain walls arise from the SSB of discrete symmetries, 3D textures arise from the SSB of \(U(2)\).

Other examples of unwanted relics are gravitinos (spin \(3/2\) superpartners of gravitons, with \(m \sim \SI{100}{GeV}\)) or spin-0 \emph{moduli} from superstring theory. 

We will explore how these can overclose the universe.  
We start with SSB in a cosmological context. 
SSB means that the ground state has less symmetry than the full theory.
How do we describe it in the context of an expanding universe?

If go back in time to when the temperature was very high, we expect a restoration of the symmetry. 

We start with a lot of symmetry, and as the universe expands we lose it. There are \emph{phase transitions} accompanying this change. 

\end{document}