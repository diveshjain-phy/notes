\documentclass[main.tex]{subfiles}
\begin{document}

\marginpar{Wednesday\\ 2020-12-9, \\ compiled \\ \today}

We were discussing how the envelope of the neutron star can be treated in the plane-parallel approximation. 

The conductivity tensor is 
%
\begin{align}
k'_{ij} = \left[\begin{array}{cc}
k_\parallel & 0 \\ 
0 & k_\perp
\end{array}\right]
\,,
\end{align}
%
and so the components of the tensor in the unprimed frame are 
%
\begin{align}
k_{ij} = \left[\begin{array}{cc}
k_\parallel \cos^2 \phi + k_\perp \sin^2\phi  & (k_\parallel - k_\perp) \sin \phi \cos \phi  \\ 
(k_\parallel - k_\perp) \sin \phi \cos \phi  & 
k_\parallel \sin^2 \phi + k_\perp \cos^2\phi
\end{array}\right]
\,,
\end{align}
%
which will be used in the law 
%
\begin{align}
q_i = - \sum _{j} k_{ij } \pdv{T}{x _j}
\,.
\end{align}

Let us denote \(q_z = q_2\), which will be given by 
%
\begin{align}
q_z = - k_{21} \pdv{T}{x} - k_{22} \pdv{T}{z}
\,,
\end{align}
%
and we know that this will correspond to blackbody emission: \(q_z = \sigma T^{4}\).

Now, \(\pdv*{T}{z} \approx (T - T_0 ) / L\), while \(\pdv*{T}{x} \approx - T/R\).
We know that \(T \ll T_0 \), and \(L \ll R\): then, 
%
\begin{align}
\abs{ \pdv{T}{z}} \approx \abs{- \frac{T_0}{L} } \gg \pdv{T}{x}
\,.
\end{align}

So, taking only the most significant term we get 
%
\begin{align}
\sigma T^{4} &= - k_{22} \pdv{T}{z} = - \qty(
k_\parallel \sin^2 \phi + k_\perp \cos^2\phi) \pdv{T}{z}   \\
&= \underbrace{- k_\parallel \pdv{T}{z}}_{\sigma T_P^{4}} \qty( \sin^2 \phi + \frac{k_\perp}{k_\parallel} \cos^2 \phi )
\,,
\end{align}
%
where we defined the new temperature \(T_P\), since on dimensional grounds that term was a thermal emissivity. If we take \(\phi = \pi /2\) we get the temperature at the pole: therefore, \(T_P\) is the polar temperature.

As long as \(k_\perp \ll k_\parallel\), we get \(\sigma T^{4} \approx \sigma T^{4} \sin^2 \phi \). Note that the angle \(\phi = \pi /2 - \Theta \).
\todo[inline]{Is this true?}

This then tells us that \(T = T_P \sqrt{\abs{\cos \Theta }}\).

The dipolar magnetic field is given by 
%
\begin{align}
\vec{B} = \frac{B_p}{2} \qty(\frac{R}{r})^3 \qty(2 \cos \theta \hat{e}_r + \sin \theta \hat{e}_\theta )
\,,
\end{align}
%
so 
%
\begin{align}
\vec{B}(r = R) = \frac{B_p}{2} \qty(2 \cos \theta \hat{e}_r + \sin \theta \hat{e}_\theta )
\,.
\end{align}

The cosine of \(\Theta \), the angle between the \(\vec{B}\) field and the radial direction, is 
%
\begin{align}
\cos \Theta = \frac{\vec{B} \cdot \hat{e}_r}{\abs{\vec{B}}} = \frac{B_r}{B} = \frac{B_p}{2} \frac{2 \cos \theta }{B} 
\,,
\end{align}
%
so, taking geometric considerations into account, we get 
%
\begin{align}
\cos \Theta = \frac{2 \cos \theta }{\sqrt{1 + 3 \cos^2 \theta }}
\,.
\end{align}

Therefore, the temperature monotonically decreases going from the pole to the equator. 
At the equator it is not really zero, but it is an order of magnitude lower than the polar temperature.

\todo[inline]{Plot temperature as a function of angle.}

This has an important observational implication: the emission from the NS is not a single blackbody, but instead a superposition of a blackbody for each latitude.

We need to account for relativistic effects.

\section{Accretion onto Neutron Stars}

We start by introducing the light-cylinder radius.
We define \(v_\phi = R_{LC} \Omega = c\), so \(R_{LC} = c/  \Omega \).

This is on the order of 
%
\begin{align}
R_{LC} = \frac{cP}{2 \pi } \approx \SI{5e9}{cm} \frac{P}{\SI{1}{s}}
\,.
\end{align}

After this length, the magnetic field lines must disconnect. 

The Alfven radius is the one at which the magnetic pressure equals the round pressure: 
%
\begin{align}
\frac{B^2}{8 \pi } = \frac{1}{2} \rho v^2
\,,
\end{align}
%
and we can approximate the radius at which this occurs by using Bondi flow, therefore assuming spherical symmetry. 
This gives us 
%
\begin{align}
4 \pi r^2\rho v = \dot{M}
\,,
\end{align}
%
therefore \(v = \sqrt{GM /r}\), and \(\rho = \dot{M} / 4 \pi r^2 v \), so 
%
\begin{align}
\frac{B^2}{8 \pi } = \frac{1}{2} \frac{\dot{M}}{4 \pi r^2v}
= \frac{1}{2} \frac{\dot{M}}{4 \pi r^2} \sqrt{ \frac{GM}{r}}
\,,
\end{align}
%
and let us consider \(B \approx (B_P /2) (R / r)^3\). 
This yields 
%
\begin{align}
\frac{B_p^2 R^{6}}{\sqrt{GM} \dot{M}} = r^{6} r^{-2} r^{-1/2} = r^{7/2}
\,,
\end{align}
%
therefore 
%
\begin{align}
r_A &= \qty(\frac{B_p^2 R^{6}}{\sqrt{GM} \dot{M}})^{2/7}  \\
&\approx \SI{3e8}{cm} \times \qty(\frac{B_p}{\SI{e8}{G}})^{4/7}
\qty(\frac{R}{\SI{10}{cm}})^{18/7} \qty(\frac{M}{M_{\odot}})^{1/7} 
\qty(\frac{\dot{M}}{\SI{e7}{g/s}})^{-2/7}
\,.
\end{align}

Typically, \(r_A < R_{LC}\), which justifies our use of a dipolar magnetic field. 

\end{document}
