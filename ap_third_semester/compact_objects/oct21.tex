\documentclass[main.tex]{subfiles}
\begin{document}

\marginpar{Wednesday\\ 2020-10-21, \\ compiled \\ \today}

About the differences between the Kerr spacetime and the spacetime outside a rotating star: the spacetime outside a rotating body is indeed \emph{not} Kerr. 

Kerr is Petrov type D, which means nonradiative: we would need to compute the Weyl invariants of the metric. 
If we do the calculation, we find that the quadrupole moment of Kerr is \(Q = J / M\). 
If we compute the invariant telling us whether a spacetime is radiative or not, we find that it is nonradiative iff \(Q = J / M\). 
The Hartle-Thorne approximation allows us to work up to a certain value of \(\Omega / \Omega _k\), close to the mass shedding limit.

The thing we find is that in general for a star the quadrupole is \(Q \neq Q _{\text{Kerr}}\), however for small values of the rotation the spacetimes converge. 

A reference for this: \textcite{bertiRotatingNeutronStars2005}. 

To leading order, deviations from type-D are driven by deviations of \(Q \) from \(Q _{\text{Kerr}}\). 

Back to what we were saying: we have an explicit expression for the pressure of a degenerate electron gas. 

Let us consider the ultrarelativistic limit first, \(x_F \gg 1\): 
%
\begin{align}
P &\approx \frac{\pi m^4 c^{5}}{h^3}
\qty[x_F^2 \frac{2}{3} x_F^2 + \log \dots]  \\
&\approx \frac{\pi m^4 c^{5}}{h^3} \frac{2}{3} x_F^{4} \propto n^{4/3} \propto \rho^{4/3}
\,.
\end{align}

In the opposite limit, \(x_F \ll 1\), we would need to expand up to fifth order to see through all the cancellations: skipping all that mess, we find
%
\begin{align}
P \approx \frac{8}{15} \frac{m^4 c^5}{h^3} x_F^{5} \propto n^{4/3}  \propto \rho^{5/3}
\,.
\end{align}

The results are similar: in both cases, \(P \propto \rho^{\gamma }\), with \(\gamma = 5/3\) and \(4/3\) respectively.

This allows us to compute the maximum mass that a spherical equilibrium configuration can reach if it is supported by the degeneracy pressure of the electron gas alone: the \textbf{Chandrasekhar mass}.
This is the maximum mass of a white dwarf. 

We need to start with the \textbf{Lane-Emden equation}, which is also due to work by Chandrasekhar. 

We have a spherical star with no nuclear burning. 
The only forces are due to gravity and pressure. This is a good description of a white dwarf. 
Sometimes white dwarfs can have some burning on their surface if they are in binary systems, if they accrete fresh mass. 

The equation of hydrostatic equilibrium reads 
%
\begin{align}
\dv{P}{r} = - \frac{G m(r) \rho }{r^2}
\,,
\end{align}
%
where \(m(r)\) is the mass contained within a shell of radius \(r\):
%
\begin{align}
m(r) = \int_0^{r} 4 \pi r^2 \rho \dd{r} 
&& 
\dv{m}{r} = 4 \pi r^2 \rho 
\,.
\end{align}

We want to couple these two equations in order to get a single one: we will find a second-order equation. 
We start from 
%
\begin{align}
\frac{r^2}{\rho }\dv{P}{r} &= - G m  \\
\dv{}{r} \qty(\frac{r^2}{\rho }\dv{P}{r}) &=
- G \dv{m}{r} = - 4 \pi G r^2 \rho  \\
\frac{1}{r^2} \dv{}{r} \qty( \frac{r^2}{\rho } \dv{P}{r}) &= - 4 \pi G \rho 
\,.
\end{align}

We also need to specify an equation of state, which we will have in the form \(P(\rho )\): we assume \(P = K \rho^{\gamma }\), a \textbf{polytropic} EoS, with constant \(\gamma \).
Sometimes this is also written through \(\gamma = 1 + 1/n\), where \(n\) is called the polytropic index. 

A convenient way to solve the equation is to substitute \(\rho = \lambda \phi^{n}\): then, \(P = \lambda^{1+1/n} K \phi^{n+1}\). 
Then, if we assume that the function \(\phi \) is dimensionless and such that \(\phi (0) = 1\), we have \(\lambda = \rho _c\). 

We need to compute 
%
\begin{align}
\dv{P}{r} &= \dv{}{r} \qty( K \lambda^{1 + 1/n} \phi^{n+1})  \\
&= K \lambda^{1 + 1/n} (n+1) \phi^{n} \dv{\phi }{r} 
\,.
\end{align}

Then, the equation reads 
%
\begin{align}
K \lambda^{1/n} (n+1) \frac{1}{r^2} \dv{}{r} \qty(r^2 \dv{\phi }{r} )
&= - 4 \pi G \lambda \phi^{n}  \\
a^2 \frac{1}{r^2} \dv{}{r} \qty(r^2 \dv{\phi }{r}) &= - \phi^{n} 
\,,
\end{align}
%
where 
%
\begin{align}
a^2 = \frac{K (n+1) \lambda^{-1 + 1/n}}{4 \pi G}
\,.
\end{align}

The physical dimensions of \(a\) are those of a length. 
We then introduce a new radial coordinate \(\xi = r /a\), and we have the right amount of \(a\)s on the left-hand side to adimensionalize everything: 
%
\begin{align}
\frac{1}{\xi^2} \dv{}{\xi } \qty(\xi^2 \dv{\phi }{\xi }) = - \phi^{n}
\,,
\end{align}
%
which is the usual formulation of the Lane-Emden equation. 

What are the boundary conditions we need to set? If we assume that \(\rho (r=0) = \rho _c\), we can set \(\lambda = \rho _c\) and \(\phi (0) = 1\).

\end{document}
