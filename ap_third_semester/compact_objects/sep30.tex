\documentclass[main.tex]{subfiles}
\begin{document}

\section{A journey into the life of a massive star}

\marginpar{Wednesday\\ 2020-9-30, \\ compiled \\ \today}

Stars whose mass is \(M \gtrsim 8 M_{\odot}\) go supernova at the end of their life. 
During their lifetime, hydrogen fuses through two channels: the p-p chain and the CNO cycle. 

In the CNO cycle, four protons turn into a \ce{^{4}He} nuclide, two positrons, two electron neutrinos using heavier nuclides as catalysts. 

The critical temperature above which the CNO cycle dominates is around \(T_c \sim \SI{2e7}{K}\). For the Sun, less than \SI{8}{\percent} of energy production is through the CNO cycle. 

When the temperature of the core reaches a value around \(1 \divisionsymbol\SI{2e8}{K}\) and the density is around \(\num{e8} \divisionsymbol \SI{e9}{g / cm^3}\), helium starts to burn in the \(3 \alpha \) process, becoming \ce{^{12}C}. 
The \(Q\)-value here is around \SI{7.27}{MeV}.

As soon as we have carbon, this can fuse with an \(\alpha \) particle giving rise to a nucleus of oxygen with \(Q \approx \SI{7.16}{MeV}\). 
This oxygen can further catch an \(\alpha \) particle, making a \ce{^{20}Ne} nuclide. 

Then, we have a temperature around \SI{5e8}{K} and a density around \SI{3e6}{g / cm^3}. Carbon starts to fuse with itself, making sodium, magnesium, and more neon (plus an \(\alpha \) particle). 

% This can happen in a controlled manner as long as the matter is not degenerate. (?)
% \todo[inline]{Understand this, mention of ``carbon flash''}
If you want carbon burning to proceed in a steady way, it must occur in a nondegenerate electron gas.
This occurs only if the star is quite massive, more than \(8 M_{\odot}\). Otherwise, it is an explosive process. 

Now the core temperature reaches \SI{e9}{K}. The energy of a typical photon is quite high, \(h \nu \sim k_B T \sim \SI{100}{keV}\). 
Suppose there are neon nuclei in the core (this will be the case since they are a product of fusion). 

It is not hard for a Neon to lose an \(\alpha \) particle through photodissociation, this produces an Oxygen. 
The energy required for this is of the order \SI{4.7}{MeV}, at the high energy tail we have a few photons at this energy. 

This is the ``neon burning phase'', after which we have oxygen and magnesium. Oxygen is the next candidate for nuclear burning, and after a further contraction the star starts burning it. It fuses with itself to produce \ce{^{28}Si} plus an \(\alpha \), or \ce{^{32}S}.

Sulfur cannot fuse with itself, the potential barrier is too high. 
Through successive \(\alpha \) captures, the star synthesizes elements in the ``iron peak'': iron, nickel, cobalt. 

The core tries to contract in the attempt to get them to burn, but they have the maximum possible binding energy per nucleon. 
So, the contraction continues. 

If the mass of the contracting core exceeds the Chandrasekhar limit, it cannot become an electron-degenerate object. 
The mass of the core is always in excess of this limit mass for the stars which are massive enough to reach this stage of stellar burning. 

Iron is photodissociated to make helium nuclei first, then bare protons, electrons and neutrons. 
Protons and electrons can combine into neutrons. The core becomes more and more neutrons rich, but the reaction also produces neutrinos, which can fly away.

The \emph{neutron} degeneracy pressure can stop the collapse in certain cases: this is how a neutron star is formed. 
The threshold between neutron stars and black holes is hard to determine, but generally speaking with \(8 M_{\odot} < M < 25 M_{\odot}\) a neutron star is formed, while for larger masses the core collapses further to form a black hole. 

The freefall velocity is a significant fraction of the speed of light.
What are the statistics? how many NS and BH are there in our galaxy?

We can model the distribution of star masses in our galaxy with the distribution, the IMF, as a Salpeter IMF,\footnote{See the evil organization in Mission Impossible.} which is given by 
%
\begin{align}
N(m) \propto m^{-\alpha }
\,,
\end{align}
%
where \(\alpha \approx \num{2.35}\). Then, we can calculate the number of stars which have more than \(8 M_{\odot}\) by integrating: we find something proportional to \(8^{-1.35}\), while the number of stars which have more than \(25 M_{\odot}\) we get something proportional \(25^{-1.35}\). These will give us the amount of compact objects.
The proportionality constant depend on the minimum and maximum mass of stars, but we can calculate the ratio of the two without concern for it. We find 
%
\begin{align}
\frac{N_{BH}}{N_{NS} + N_{BH}} = \qty(\frac{8}{25})^{-1.35} \approx \num{.2}
\,.
\end{align}

The present rate of supernova explosions in the galaxy is around 1 per century, or \SI{e-2}{yr^{-1}}. 
In the age of the galaxy (around \SI{e10}{yr}), we will then have had around \(\num{e8}\) compact objects. 

How much can we trust this figure? Kind of, the true number is closer to \num{e9}, about \SI{1}{\percent} of the number of stars in the galaxy.

The galaxy roughly looks like a cylinder with radius \(R \sim \SI{60}{kpc}\) and height \(H \sim \SI{1}{kpc}\).
Its volume will then be \(V = 2 \pi R^2 H \approx \SI{e13}{pc^{3}}\). Then, the number density of compact objects is around \SI{0.1}{pc^{-3}}.

The typical separation between them will be something like \SI{20}{pc}. Compact objects are close, common! Beware!

The closest compact object we know of is a neutron star \SI{60}{pc} away: this is on the same order of magnitude.

\paragraph{Compactness}
The gravitational radius characterizing an object is 
%
\begin{align}
R_g = \frac{GM}{c^2}
\,,
\end{align}
%
while the Schwarzschild radius is \(2 R_g\). For the Sun, this is approximately \SI{1.5}{km}. It's small.

The value \(R_g / R\) is \num{.5} for black holes, \num{.15} for neutron stars, \num{e-4} for white dwarfs.

\end{document}
