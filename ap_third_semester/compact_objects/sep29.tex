\documentclass[main.tex]{subfiles}
\begin{document}

\marginpar{Tuesday\\ 2020-9-29, \\ compiled \\ \today}

\subsection*{Introduction}

Tuesdays and Wednesday at 14.30 PM in room P1A, Paolotti building.
22 people. 

This course overlaps with ``Computational Astrophysics'' by professor Mapelli.

The examination is an oral one, done either online or live. 

We start with a brief overview of the final fates of massive stars. 
We have white dwarfs, neutron stars and black holes under the category of ``compact objects'', but white dwarfs are not really that compact. 

We then discuss accretion onto compact objects, and neutron stars.  
An open question: what is the EOS of ultradense neutron matter?

``Accretion power in astrophysics'', ``The physics of Compact Objects'', ``Astrofisica Relativistica I \& II'', ``Astrofisica delle Alte Energie''.

\end{document}
