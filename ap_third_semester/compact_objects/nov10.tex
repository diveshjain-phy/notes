\documentclass[main.tex]{subfiles}
\begin{document}

\marginpar{Tuesday\\ 2020-11-10, \\ compiled \\ \today}

We have introduced the function 
%
\begin{align}
D(R) = \frac{1}{2} R^2 \Sigma (\lambda \overline{v})
\qty(\pdv{\Omega }{R})^2 \geq 0
\,,
\end{align}
%
which can be zero only if either \(\lambda \overline{v} = 0\) (no viscosity) or \(\pdv{\Omega }{R} = 0\) (rigid rotation). 

The viscosity is given by the presence of meso-scale structures called \textbf{eddies}, and we define \(\lambda \overline{v} = \nu\), the kinematic viscosity. 
Newton's law of viscosity gives the viscous stress \(\tau \) in terms of the velocity gradient times the \emph{dynamic} viscosity \(\mu \): 
%
\begin{align}
\tau = \mu \dv{v}{y}
\,,
\end{align}
%
and the two viscosities are related by \(\nu = \mu / \rho \). 
This is exemplified in Couette flow. 
This is an empirical law, there are many non-Newtonian fluids.  

In dimensional terms, we can gather from Newton's law that the units of the dynamic viscosity \(\mu \) are \SI{}{kg / m s}. 
Therefore, the kinematic viscosity has units of \SI{}{m^2/s}.
At least on dimensional grounds, then, identifying \(\lambda \overline{v}\) with \(\nu \) makes sense. 

If the carriers of motion are ions, then \(\lambda \sim \lambda _d\), the Debye scale, and the average velocity of this Brownian motion is roughly \(\overline{v} \sim c_s\). 
We can compute the Reynolds number as 
%
\begin{align}
\Re = \frac{\text{inertial  forces}}{\text{viscous forces}}
\approx \frac{\dv*{v}{t}}{ \frac{\mu S}{m} \dv{v}{y}} \approx \frac{\dv{v}{R} v}{\frac{\mu }{m} S \dv{v}{R}} \approx \frac{v}{\mu S / m} \approx \frac{vR}{\mu / \rho } = \frac{vR}{\nu }
\,.
\end{align}

Then, we get \(\Re = v R / \lambda \overline{v}\): but if the viscosity is indeed due to ions' motion we find \(\Re \gtrsim \num{e14}\). 
Even changing our assumptions a bit we will find flow which is definitely turbulent. 

If the carriers were ions, then the flow would be turbulent. 
For water in a pipe, the transition between laminar and turbulent is at around \(\Re \gtrsim \num{e3}\). 

\todo[inline]{Why can't the flow be turbulent?}

Then, we will have \(\nu = \nu _{\text{turb}} = \lambda _{\text{turb}} \overline{v} _{\text{turb}}\). What are these two parameters' values?

If \(H\) is the height of the disk, then we must have \(\lambda _{\text{turb}} \lesssim H\); also we must have \(\overline{v} _{\text{turb}} \lesssim c_s\) --- it can be shown that supersonic turbulence is unstable.
Then, the kinematic viscosity is bounded as \(\nu _{\text{turb}} \lesssim H c_s\).

This is somewhat uninteresting, it is just an inequality.
What we then do is called an \(\alpha \)-prescription: we write 
%
\begin{align}
\nu _{\text{turb}} = \alpha H c_s
\,,
\end{align}
%
encompassing all our ignorance in the dimensionless parameter \(0 < \alpha \lesssim 1\). 

The accretion disk described by this is called a Shakura-Sunyaev Disk (SSD). 

In general \(\alpha \) will be a function of pressure, temperature and such, but we will simplify the problem by assuming \(\alpha \) is a constant. 
What we hope to find is that the physical observables are not very sensitive to the value of \(\alpha \). 

We will make certain assumptions: 
\begin{enumerate}
    \item the disk must be geometrically \emph{thin}, at any radius \(R\) the height \(H\) of the disk must be \(H(R) \ll R\);
    \item we will decompose the velocity as \(\vec{v} = v_\phi \hat{u}_\phi + v_R \hat{u}_R\), and assume that \(v_\phi \gg v_R\);
    \item the system has azimuthal symmetry.
\end{enumerate}

We will work in cylindrical coordinates, and have dependence only on \(R\) and \(z\). 

Mass flux will be inward; the differential mass element is \(\Delta m = 2 \pi R \Delta R \Sigma \) so 
%
\begin{align}
\pdv{\Delta m}{t} = v_R (R, t) 2 \pi R \Sigma (R, t) - v(R + \Delta R, t) 2 \pi (R + \Delta R) \Sigma (R + \Delta R) \approx - 2 \pi R \pdv{(\Sigma R v_R)}{R}t
\,,
\end{align}
%
with \(v_R <0\). 

Then,  
%
\begin{align}
R \pdv{\Sigma }{t} = - \pdv{}{R} \qty(R V_R \Sigma )
\,.
\end{align}

The angular momentum in the ring is given by 
%
\begin{align}
\Delta L = 2 \pi R \Delta R \Sigma R^2 \Omega 
\,,
\end{align}
%
so 
%
\begin{align}
\begin{split}
\pdv{\Delta L}{t}
&= \pdv{G}{R} \Delta R 
+ v_R( R, t) 2 \pi R \Sigma (R, t) R^2\Omega (R, t) \\
&\phantom{=}\ - v_R(R + \Delta R, t) 2 \pi (R + \Delta R) \Sigma (R + \Delta R, t)( R + \Delta R)^2 \Omega (R + \Delta R, t)
\end{split}  \\
&= \pdv{G}{R} \Delta R - 2 \pi \Delta R \pdv{}{T} \qty(v_R R R^2 \Omega \Sigma )
\,.
\end{align}

Then, like before 
%
\begin{align}
\pdv{}{t} \qty(2 \pi R \Delta R \Sigma R^2 \Omega ) &= 2 \pi R R^2 \Delta R \pdv{\Omega }{t}   \\
&= \pdv{G}{R} \frac{\Delta R}{2 \pi }  - 2 \pi \Delta R \pdv{}{R} \qty(v_R R R^2 \Omega \Sigma )
\,,
\end{align}
%
so, assuming that \(\Omega = \Omega (R)\) is the Keplerian angular velocity we get 
%
\begin{align}
R \pdv{}{t} \qty(R^2\Omega \Sigma ) &= \pdv{G}{R} \frac{1}{2 \pi } - \pdv{}{R} \qty(v_R R R^2 \Omega \Sigma )  \\
&= \pdv{G}{R} \frac{1}{2 \pi } - \Omega R^2 \pdv{}{R} \qty(v_R R \Sigma ) - v_R R \Sigma \pdv{}{R} \qty(\Omega R^2) 
\end{align}
\begin{align}
R^3 \Omega \pdv{\Sigma }{t} - R^2 \Omega \pdv{}{R} \qty(v_R R \Sigma ) 
= \underbrace{R^2 \Omega \qty(R \pdv{\Sigma }{t} + \pdv{}{R} \qty(v_R R \Sigma ))}_{= 0 \text{ by mass conservation}}
&= \frac{1}{2 \pi } \pdv{G}{R} - v_R R \Sigma \pdv{}{R} (\Omega R^2)
\,,
\end{align}
%
so the momentum equation reads 
%
\begin{align}
\frac{1}{2 \pi } \pdv{G}{R} - v_R R \Sigma \pdv{}{R} (\Omega R^2) = 0
\,,
\end{align}
%
or 
%
\begin{align}
v_R R \Sigma &= \frac{1}{2 \pi } \pdv{G}{R} \qty(\pdv{(\Omega R^2)}{R})^{-1}  \\
R \pdv{\Sigma }{R} &= - \pdv{}{R} \qty[ \frac{1}{2 \pi } \pdv{G}{R} \pdv{(\Omega R^2)}{R}]
\,.
\end{align}

Since \(\Omega = \sqrt{GM / R^3} \) is the Keplerian angular velocity, we have 
%
\begin{align}
G = 2 \pi R \Sigma \nu R^2 \dv{\Omega }{R}
\,,
\end{align}
%
we get 
%
\begin{align}
\pdv{\Sigma }{t} = \frac{3}{R} \pdv{}{R} \qty[R^{\prime \prime 2} \pdv{}{R} (\nu \Sigma R^{\prime \prime 2})] 
\,.
\end{align}

This is a diffusion-like equation, an initially peaky function will drift towards a broader function. 

We can solve for \(v_R\): we find 
%
\begin{align}
v_R &= \frac{1}{2 \pi R \Sigma } \pdv{G}{R} \pdv{(R^2 \Omega )}{R}  \\
&= - \frac{3}{\Sigma R^{\prime \prime 2} } \pdv{}{R} \qty(\nu \Sigma R^{\prime \prime 2}) 
\,.
\end{align}

The ratio of \(\Sigma / T\), where \(T\) is the characteristic time of the evolution of the system, is roughly 
%
\begin{align}
\frac{\Sigma}{T} \approx \frac{\nu \Sigma }{R^2}
\,,
\end{align}
%
therefore \( 1/ T \approx \nu / R^2\).

Also, the velocity is roughly 
%
\begin{align}
v_R \approx \frac{1}{\Sigma R} \nu \Sigma = \frac{\nu}{R} = \frac{R}{T}
\,.
\end{align}

How does the mixing timescale \(T\) compare to the timescales mentioned earlier? 
We will have \(v_\phi \approx R / t _{\text{dyn}}\), \(v_R \approx R / t _{\text{visc}}\), where \(T = t _{\text{visc}}\). 
This makes sense together with our assumption that \(v_\phi \gg v_R\). 

Why is there turbulence? A strong candidate is \textbf{magneto-rotational instability}. 

\end{document}
