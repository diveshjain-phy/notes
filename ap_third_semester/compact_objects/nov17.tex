\documentclass[main.tex]{subfiles}
\begin{document}

\section{Stationary SSD}

\marginpar{Tuesday\\ 2020-11-17, \\ compiled \\ \today}

We know that \begin{enumerate}
    \item \(\rho = \Sigma / H\);
    \item \(H = c_s (R^3 / GM)^{1/2} = R c_s / v_\phi \);
    \item \(c_s^2 = P / \rho \);
    \item \(P = P _{\text{gas}} + P _{\text{rad}} = k_B \rho T / (\mu m_p) + a T^{4} / 3\).
\end{enumerate}

From the center of the disk radiation can travel outward, however it will be optically thick: \(\tau = \int_{0}^{r} \alpha \dd{s} >1\).
We can approximate it as \(\tau \approx \kappa _R \rho H = \kappa _R \Sigma > 1\).

Radiative transport in a slab can be treated analytically in the diffusion approximation to yield the radiative flux 
%
\begin{align}
F(z) = \frac{16 \sigma T^3}{3 \kappa _R \rho } \pdv{T}{z} = - \frac{4}{3} \frac{\sigma}{\kappa _R \rho } \pdv{(T^{4})}{z}
\,.
\end{align}

We can define the surface as the height at which \(\tau =1\).

The flux crossing the \(z = 0\) surface is 
%
\begin{align}
F(0) \approx \frac{4}{3} \frac{\sigma }{\kappa _R \rho _c} \frac{T_c^{4}}{H} \approx \frac{4}{3} \frac{\sigma T_c^{4}}{\tau _R}
\,.
\end{align}

\todo[inline]{not zero?}

The flux at the surface, on the other hand, is
%
\begin{align}
F(s) \approx \frac{4}{3} \frac{\sigma}{\tau _s} T_s^{4} \approx \frac{4}{3} \sigma T_s^{4}
\,.
\end{align}

So, their ratio is 
%
\begin{align}
\frac{F(s)}{F(0)} = \qty(\frac{T_s}{T_c})^{4} \tau _c 
\,.
\end{align}

We expect \(T_s < T_c\), as is natural if flux is going from inside to outside. Then, \((T_s / T_c)^{4} \ll 1\), so unless \(\tau _c\) is extremely large (and, as we will see, it is not) we get \(F(s) < F(0)\). 
The difference \(F(0) - F(s) \approx F(0)\) corresponds to the produced energy \(D(R)\), so 
%
\begin{align}
\frac{4}{3} \sigma \frac{T_c^{4}}{\tau _c} \approx D(R) = \frac{3 GM \dot{M}}{8 \pi R^3} \qty[1 - \qty(\frac{R _{\text{in}}}{R})^{1/2}]
\,.
\end{align}

This will be another assumption for us. 
Also, we will use the relations 
%
\begin{align}
\tau &= \kappa _R \Sigma  \\
v \Sigma &= \frac{\dot{M}}{3 \pi } \qty[1 - \qty(\frac{R _{\text{in}}}{R})^{1/2}]  \\
v_R &= - \frac{3 v}{2R} \qty[1 - \qty(\frac{R _{\text{in}}}{R})^{1/2}]^{-1}  \\
\kappa _R &= \kappa _R (\rho , T, \dots)  \\
v &= v (\rho , T, \dots) = \alpha H c_s
\,.
\end{align}

Then, we can write 
%
\begin{align}
\sigma T_s^{4} &= D(R) = \frac{3 GM \dot{M}}{8 \pi R^3} \qty[1 - \qty(\frac{R _{\text{in}}}{R})^{1/2}]  \\
T_s &= \qty(\frac{3 GM \dot{M}}{8 \pi \sigma })^{1/4} R^{-3/4} \qty[1 - \qty(\frac{R _{\text{in}}}{R})^{1/2}]^{1/4}  \\
&\approx \qty(\frac{3 GM \dot{M}}{8 \pi \sigma R _{\text{in}}^3 })^{1/4} \qty( \frac{R _{\text{in}}}{R})^{3/4}
\marginnote{If  \(R \gg R _{\text{in}}\).}
\,.
\end{align}

Then, we can see that the temperature decreases as a function of \(R\).

Let us define a typical surface temperature \(T _s^{\text{typical}}\) as 
%
\begin{align}
T _s^{\text{typical}}
= 
\qty(\frac{3 GM \dot{M}}{8 \pi \sigma R _{\text{in}}^3 })^{1/4}
\approx \SI{e7}{K} \qty(\frac{\dot{M}}{\SI{e17}{g /s}})^{1/4}
\qty(\frac{M}{M_{\odot}})^{1/4} \qty(\frac{R _{\text{in}}}{\SI{e6}{cm}})^{-3/4}
\,,
\end{align}
%
which yields soft \(X\)-rays, at around \SI{1}{keV}. 
Using the complete formula, we find that the temperature is in the form 
%
\begin{align}
T^{4} \propto \frac{1}{x^{3/4}} \qty(1 - \frac{1}{x^{1/2}})
\,,
\end{align}
%
where \(x = R / R _{\text{in}}\). 
We can maximize this, and we find that the maximum temperature of the disk is found to be \(T _{\text{max}} \approx \num{.5} T^{\text{typical}}_s\). 

There are several radiative processes taking place in an accretion disk: 
\begin{enumerate}
    \item electron scattering;
    \item thermal free-free emission.
\end{enumerate}

The latter is dominant, and the characteristic Rosseland mean opacity is 
%
\begin{align}
\kappa _R^{\text{bremss}} \approx \num{6.6e22} \rho T^{-7/2} \SI{}{cm^2 g^{-1}}
\,.
\end{align}

We find an algebraic system with all the equations, and finally we get 
%
\begin{align}
H = \SI{1.7e8}{cm} \times \alpha^{-1/10} M^{3/20} M^{-3/8} R_{10}^{9/8} f^{3/5}
\,,
\end{align}
%
from which we can confirm that \(R \gg H\). 
We can also calculate the central optical depth: 
%
\begin{align}
\tau _c = 33 \alpha^{-4/5} M_{16}^{1/5} f^{4/5}
\,.
\end{align}



\end{document}
