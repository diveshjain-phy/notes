\documentclass[main.tex]{subfiles}
\begin{document}

\marginpar{Monday\\ 2020-11-23, \\ compiled \\ \today}

Next homework assignment: exercise 10 for December, exercise 7 just before the exam. 

 \section{SN Ia observation}
 
Standard candles have a uniform intrinsic luminosity \(L\), so that measuring the flux \(f\) at Earth allows us to measure the luminosity distance as 
%
\begin{align}
d_L = \sqrt{\frac{L}{4 \pi f}}
\,,
\end{align}
%
and we quantify fluxes in terms of magnitudes: 
%
\begin{align}
\mu = m - M = 5 \log_{10} \qty( \frac{d_L}{\SI{10}{pc}})
\,,
\end{align}
%
where \(m\) and \(M\) are the apparent and intrinsic magnitudes.

The dispersion of the intrinsic magnitudes is not zero, but it is rather small, on the order of \SI{.5}{mag}.

There is a correlation between the peak luminosity of the SN light curve and the length of the decay; also there is a correlation between color and peak width.
Correcting for both of these, there is a dispersion of the order of \SI{.2}{mag}. 
Therefore, the final error in luminosity distance is on the order of \SI{10}{\percent}. 

The magnitudes of the stretch and color corrections (\(\alpha \) and \(\beta \)) are not precisely known, we need to include them as nuisance parameters in our model. 

\todo[inline]{How do we know that the spread is that small?}

The distance moduli depend on the redshifts \(z_i\) and on the cosmological parameters \(\vec{p} = h, \Omega _m, \Omega _\Lambda \); however \(h\) is fully degenerate with the intrinsic magnitude \(M\). 
This degeneracy must be broken in some way, for example with Cepheids or red giants. 
With SN Ia only we can, however, measure \(\Omega _m\) and \(\Omega _\Lambda \) by marginalizing over \(\alpha \), \(\beta \), \(H_0 \) and \(M\). 

Classically what was done was \(\chi^2\) minimization, however there are several problems with it. 
\textcite[]{marchImprovedConstraintsCosmological2011} improved the result through the use of Bayesian Hierarchical Models. 

\end{document}
