\documentclass[main.tex]{subfiles}
\begin{document}

\marginpar{Monday\\ 2020-11-2, \\ compiled \\ \today}

We can compute the average value of some function \(g\) of our parameters \(\vec{\theta}\), defined as 
%
\begin{align}
\mathbb{E} (g(\vec{\theta})) = \int g(\vec{\theta}) p(\vec{\theta}) \dd{\vec{\theta}}
\,,
\end{align}
%
as 
%
\begin{align}
\hat{E} (g(\vec{\theta})) = \frac{1}{N} \sum _{i=1}^{N} g(\vec{\theta}_i)
\,,
\end{align}
%
which converges as \(N \to \infty \) to the true value, as long as the \(\vec{\theta}_i\) are iid sampled according to their distribution \(p(\vec{\theta})\).
This converges to the true value with a variance \(\var{\hat{E}} = \sigma^2 / N\). 

However, it is difficult to sample points from a generic multidimensional distribution. 
This is the problem we will treat now. 

A possible solution is rejection sampling. See the numerical methods course for a detailed explanation of the method. 

In short, in order to sample from a distribution \(f(x)\) we choose a distribution \(g(x)\) which \emph{embeds} the generic one we have: this means that \(g(x) \geq M f(x)\) for some real number \(M\). 
We generate numbers \(x\) according to \(g(x)\), and then we reject the number we generated a fraction \(M f(x) / g(x)\) of the time (this can be done through a uniform distribution).
This yields samples distributed according to \(f(x)\).

This can be wasteful, especially in high dimensions, due to the \textbf{curse of dimensionality}. 
Consider the volume of a \(D\)-dimensional sphere and a \(D\)-dimensional cube with a side equal to the diameter of the sphere. 
The ratio is \(\pi /4\) in two dimensions, \(\pi /6\) in three dimensions, and it approaches zero for \(D \to \infty \).

So, rejection sampling fails in practice. 

\end{document}
