\documentclass[main.tex]{subfiles}
\begin{document}

\marginpar{Monday\\ 2020-12-21, \\ compiled \\ \today}

There will be a couple of optional additional lectures uploaded to the Moodle. 

Today we will discuss \cite[]{hobsonCombiningCosmologicalDatasets2002}.
The idea is to combine the \(\chi^2\) for the various experiments with various weights, which are our hyperparameters. 

Do we actually need these hyperparameters, or can we avoid using them by simply adding the \(\chi^2\)? This is a model selection problem. 

In the hyperparameter model the likelihood is not a Gaussian anymore. 
The weights must be larger than \(0\), we can take them to be larger than 1  if the corresponding experiment was overestimating its errorbars. 


\end{document}
