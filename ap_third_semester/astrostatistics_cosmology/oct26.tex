\documentclass[main.tex]{subfiles}
\begin{document}

\marginpar{Monday\\ 2020-10-26, \\ compiled \\ \today}

We have seen if the gradient of the log-posterior is linear in the parameters: \(\vec{\nabla} L (\vec{x}) = A \vec{x} + \vec{c}\) then the covariance matrix is \(\Sigma_{ij} = \qty(- A^{-1})_{ij}\).
Matrix inversion is slow, so we seek faster methods in general. 

Are there situations in which the gradient of the log-posterior is indeed linear?
We need the noise to be Gaussian --- this is common, by the CLT this approximation is good if there are many concurring sources. 
For simplicity, we will assume that the measurements are statistically independent --- this is not strictly necessary, what we will show holds even if this is not true.
Formally, this means that for two measurements \(d_i\) and \(d_j\) we have \(\mathbb{P}(d_i, d_j) = \mathbb{P}(d_i) \mathbb{P}(d_j)\). 

Then, the likelihood reads 
%
\begin{align}
\mathscr{L} (d_i | \vec{x}) = \frac{1}{\sqrt{2 \pi } \sigma }
\exp(- \frac{1}{2} \frac{(d_i - \mu )^2}{\sigma^2})
\,.
\end{align}

Let us assume that only the \emph{mean} is a function of the parameters of our theory (and of the observation number!) \(\mu = \mu_i (\vec{x})\). 
As usual, with the independence assumption we find 
%
\begin{align}
\mathscr{L} (d_i | \vec{x}) &\propto \prod_i 
\exp(- \frac{1}{2} \frac{(\qty(d_i - \mu (\vec{x}))^2)}{\sigma^2})   \\
L &\propto - \frac{1}{2} \sum _{i=1}^{n} \frac{(d_i - \mu _i (\vec{x}))^2}{\sigma_i^2} 
\,.
\end{align}

A sum of squares of Gaussian variables is distributed like a \(\chi^2\) variable (with \(n\) degrees of freedom, in our case): in this case, \(L = - \chi^2 / 2\), so minimizing the \(\chi^2\) corresponds to minimizing the likelihood.

A crucial condition in order for the problem to be linear is that the mean \(\mu _i (\vec{x})\) must be a linear function of the parameter vector: \(\mu _i (\vec{x}) = A \vec{x} + \vec{c}\).
Assuming a flat prior, our log-posterior reads 
%
\begin{align}
\log P = L \propto \chi^2 = \sum _{i=1}^{n} \frac{1}{\sigma_i^2}
\qty[ d_i - \sum _{j} A_{ij} x_j - c_i ]^2
\,.
\end{align}

What we want to show is that this implies that the gradient of the posterior, evaluated at the maximum, is linear in the parameter: it reads 
%
\begin{align}
\pdv{L}{x_k} = \sum _{i} \frac{2}{\sigma _i^2}
\qty[ d_i - \sum _{j} A_{ij} x_j - c_i ] A_{ik}
\,.
\end{align}

This is linear: one could see it by eye, but let us also compute its second derivative to make sure: 
%
\begin{align}
\pdv[2]{L}{x_l}{x_k}
= 
\sum _{i} \frac{2}{\sigma _i^2}
\qty[- A_{il} A_{ik}] = -2 \qty( A^{\top} \Sigma^{-1} A)_{lk}
\,.
\end{align}

This is a constant, so the initial expression was indeed linear.
This is the usual least-squares fitting: people often simply do this without much care for the assumptions they are making.
 
\subsection{Frequentist vs Bayesian}

The conceptual meaning of a credible interval is different from that of a confidence interval. 
Sometimes, especially when we have few data points, they are also quantitatively different. 

\begin{itemize}
    \item For a frequentist a parameter \(T\) is a fixed unknown number, not a random variable;
    \item data \(x\) are random variables given by their frequency with which they occur in many repetitions;
    \item both the Bayesian and the frequentist build a \emph{statistical estimator} \(f\), which is a function giving an estimate for the parameter starting from the data: \(f(x)\) yields \(t\), an estimate for \(T\); 
    \item the distribution of the estimator --- which is a random variable since its argument is --- is called a \emph{sampling distibution} (for example, a \(\chi^2\) distribution);
    \item the frequentist builds a \(b \%\) confidence interval, which is an interval for \(t\) such that it will contain \(T\) \(b \%\) of the time.
\end{itemize}

In the Gaussian limit the confidence interval and the credible interval coincide. 
In the frequentist case we write the Confidence Interval as 
%
\begin{align}
\mathbb{P}\qty(-\num{1.96} 
\leq \frac{\mu_0 - \mu }{\sigma / \sqrt{n}}
\leq \num{1.96}) \approx \num{.95}
\,,
\end{align}
%
while in the Bayesian case we write the Credible interval as 
%
\begin{align}
\mathbb{P}\qty(\mu_0 - \num{1.96} \frac{\sigma }{\sqrt{n}}
\leq \mu \leq \mu_0 + \num{1.96} \frac{\sigma }{\sqrt{n}}) \approx \num{.95}
\,.
\end{align}

These two are written in terms of different variables: \(\mu_0\), the estimate of the mean from the data, in the frequentist case, and the parameter \(\mu \) in the Bayesian case. 

If we also need to estimate the standard deviation we need to use a Student's \(t\) distribution: it converges to a Gaussian, but with few data it has fatter tails.

Now, we see an example of the failings of frequentist statistics \cite[]{vanderplasFrequentismBayesianismIII2014}.
The probability is given by \(\mathbb{P}(x | T) = \exp(T - x) [x \geq T]\). We are given a dataset \([10, 12,  15]\). 
A frequentist will build an estimator: we have
%
\begin{align}
\expval{x} = \int x \mathbb{P}(x) \dd{x}  = T +1
\,,
\end{align}
%
So the arithmetic mean of \(x\) converges to \(T +1\), therefore we average the results and subtract one. 
Our confidence interval will then look like \(\hat{T} -2 \sqrt{N} \leq T \leq T + 2 \sqrt{N}\). 
This is derived from an asymptotic estimate, but it is close to the true result.
We find \([\num{10.2}, \num{12.5}]\): but we know that \(T \leq 10\)! 

In a Bayesian approach, we do not even need to take a uniform prior for \(T \leq \min (\text{data})\). This is fixed by the data, since the likelihood includes a Heaviside theta. 
The likelihood then looks like 
%
\begin{align}
\mathbb{P}(T | x) = \exp(T - x) [T < x]
\,.
\end{align}

The \(\alpha \)-interval we get is \(T = x _{\text{min}} + \log \alpha  / N\). 
\todo[inline]{To include more of the discussion.}

\end{document}
