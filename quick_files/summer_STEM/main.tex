\documentclass{article}
\usepackage[utf8]{inputenc}

\usepackage{textcomp}
\usepackage[T1]{fontenc}
\usepackage{multirow}
\usepackage{float}
\usepackage[caption = false]{subfig}
\usepackage{longtable}
\usepackage{listings}
\usepackage{mathtools}
\DeclareMathOperator{\tr}{Tr}
\usepackage{commath}
\usepackage{bbold}
\usepackage{xcolor}
\usepackage{physics}
%\usepackage[margin=1.8cm]{geometry}

\usepackage{tikz-cd}
\usepackage{amsmath}
\usepackage{amsfonts}
\usepackage{amssymb}
\usepackage{amsthm}
\usepackage{graphicx}
\usepackage[colorinlistoftodos]{todonotes}
\usepackage[colorlinks=true, allcolors=blue]{hyperref}
\usepackage{siunitx}
\sisetup{separate-uncertainty=true}

\usepackage[sc]{mathpazo}
\linespread{1.05}         % Palladio needs more leading (space between lines)
\usepackage[T1]{fontenc}

\newcommand{\diag}[1]{\text{diag}\qty(#1)}
\newcommand{\const}{\text{const}}
\newcommand{\sign}[1]{\text{sign}\qty(#1)}
\renewcommand{\H}{\mathcal{H}}
\renewcommand{\dim}{\text{dim}}
\newcommand{\C}{\mathbb{C}}
\newcommand{\R}{\mathbb{R}}
\newcommand{\N}{\mathbb{N}}
\newcommand{\Z}{\mathbb{Z}}

\renewcommand{\var}[1]{\text{var} \qty(#1)}
\newcommand{\average}[1]{\langle #1 \rangle}

\newcommand\mybox[1]{%
  \fbox{\begin{minipage}{0.9\textwidth}#1\end{minipage}}}

\usepackage[italian]{babel}

\usepackage[
backend=biber,
style=alphabetic,
sorting=nyt,
backref=true
]{biblatex}

\addbibresource{STEM_Indivisibility.bib}

\title{L'indivisibilità del fotone: STEM 2020}

\begin{document}

\maketitle

Iniziamo con un quiz: \textbf{cos'è la luce}?

\begin{enumerate}[label=\emph{\alph*})]
    \item Un'onda.
    \item Una particella.
    \item Sia un'onda che una particella. 
    \item Né un'onda né una particella. 
    \item Mi astengo.
\end{enumerate}

La domanda, posta in questo modo vago (e provocatorio), rischia di allontanarsi dall'ambito di ricerca della Fisica: non siamo generalmente interessati a descrivere l'\emph{essenza fondamentale} di un oggetto, bensì cerchiamo di fornire modelli che ne descrivano il comportamento e che siano in grado di predire i risultati degli esperimenti che facciamo. 
Questo approccio all'investigazione spesso riesce a darci grande chiarezza sul come pensare agli oggetti fisici. 

% TODO keep this bit?

Tenendo questo in mente, cerchiamo di rispondere alla domanda.
Innanzitutto, dobbiamo chiarire qual'è l'oggetto fisico che vogliamo descrivere.

Partiamo dalla descrizione più intuitiva: la luce è quello che vediamo con gli occhi. 
Vedremo che, investigando questo fenomeno nel dettaglio, questa definizione risulterà restrittiva e ci porterà a ridefinire quella che vediamo con gli occhi come \emph{luce visibile} --- quindi, c'è anche della ``luce invisibile''.

Per ora, tuttavia, partiamo dalle prime teorie della luce, che volevano descrivere solo quella visibile.

\section{Newton versus Huygens}

Lo storico dibattito è quello fra Newton e Huygens. 

Newton (1642--1726/1727), il più famoso dei due, proponeva una \textbf{teoria corpuscolare} della luce. 

Secondo questa teoria, la luce è composta di corpuscoli molto leggeri che viaggiano ad una velocità finita. 

Quali fenomeni voleva spiegare Newton con questa teoria?
Riflessione e rifrazione, innanzitutto: lui ha dimostrato che facendo passare la luce per un prisma si può scomporla nelle sue componenti cromatiche, e --- qui la chiave! --- questo processo è reversibile: con un altro prisma è possibile ricomporre la luce bianca.

La luce, dunque, viene deviata quando passa per un'interfaccia, e di quanto viene deviata dipende dal suo colore. 

La riflessione invece è ciò che accade per uno specchio: la luce ``rimbalza'' ad un angolo uguale a quello di incidenza.

La teoria di Newton ha delle spiegazioni per questi fenomeni, seppur non completamente soddisfacenti.
Newton la mette in pratica per costruire telescopi: il telescopio Newtoniano utilizza solo specchi invece che lenti, eliminando quindi l'aberrazione cromatica che le caratterizza. 

La teoria corpuscolare ha dietro l'apparato della nuovissima meccanica Newtoniana: la luce viene descritta come un insieme di piccole particelle con massa molto ridotta che si muovono molto velocemente rispettando le leggi di Newton: inerzia, \(\vec{F} = m \vec{a}\) e azione-reazione.

Un vantaggio della teoria corpuscolare è il fatto che non richiede un \emph{etere luminifero} (portatore della luce). 

La teoria di Huygens è \textbf{ondulatoria}: vede la luce come un'onda che si propaga in un qualche tipo di etere, in modo simile a come si comportano le onde nell'acqua poco profonda.

Questa teoria spiega, al contrario di quella corpuscolare, il fatto che all'interfaccia fra due mezzi ci siano sia riflessione che rifrazione: nella meccanica ondulatoria questo è un fatto naturale e atteso.

Nel contesto della meccanica ondulatoria, si può derivare la legge di Snell, che governa l'angolo al quale la luce viene diffratta. 

Iniziamo però con i preliminari della teoria ondulatoria: guardiamo il caso monodimensionale per semplicità, guardare il problema in più dimensioni è utile ed interessante ma le caratteristiche principali si possono evincere già da qui. 

La caratteristica principale delle onde è la periodicità: partiamo da quella che si chiama onda \emph{monocromatica}, il caso più semplice, che ha un periodo ben definito.
Le equazioni che ora scriviamo potrebbero descrivere, ad esempio, la variazione del livello dell'acqua lungo una linea che tracciamo in uno stagno: chiamiamola \(A\).
\(A\) può assumere un valore positivo o negativo, se vale 0 significa che il livello è quello che avrebbe lo stagno senza perturbazioni.

Se fissiamo un istante temporale \(t_0 \) e ``fotografiamo'' l'onda, la formula che la descrive è 
%
\begin{align}
\eval{A(x)}_{t = t_0 } = A_0 \sin(2 \pi \frac{x}{\lambda } +\varphi )
\,.
\end{align}

\(A_0 \) è l'ampiezza massima che raggiunge l'onda, l'altezza dei picchi e la profondità delle valli. \(\lambda \) è una lunghezza caratteristica dell'onda, detta \emph{lunghezza d'onda}: è la distanza fra due picchi vicini, o fra due valli vicine; se \(x\) --- la posizione spaziale lungo la linea --- varia aumentando o diminuendo di esattamente \(\lambda\) l'ampiezza non cambia.
Questo è dovuto al fatto che la funzione seno ha una periodicità di \(2 \pi \). 

Il termine \(\varphi \) si chiama \emph{fase}, ed esprime il fatto che nel punto che chiamiamo \(x=0\) l'onda può essere in un massimo, un minimo, o a qualunque punto della sua evoluzione. 
Visto che \(\varphi \) è arbitrario, è arbitraria anche la scelta di un seno piuttosto che un coseno: cambiando \(\varphi \) di \(\pi /2\) possiamo passare dall'uno all'altro. 

Se vogliamo includere la dipendenza dal tempo, otterremo una formula di questo tipo: 
%
\begin{align}
A(x, t) = A_0 \sin(2 \pi \qty( \frac{x}{\lambda } - \frac{t}{T}) + \varphi )
\,.
\end{align}

Abbiamo inserito un tempo caratteristico \(T\), chiamato \emph{periodo}: se fissiamo la posizione \(x\) e guardiamo solo un punto, l'onda torna al punto dov'era prima dopo un tempo \(T\). 
C'è un meno perché stiamo assumendo che l'onda avanzi verso destra, ovvero verso le \(x\) positive: se \(x\) aumenta di un poco, per mantenerci nella stessa posizione (``viaggiando con l'onda'', ovvero tornando allo stesso valore dell'argomento del seno) dobbiamo aumentare un poco anche \(t\), ovvero aspettare un poco. 

Se aspettiamo un tempo \(T\) l'onda si è spostata di una lunghezza \(\lambda \): quindi, la sua velocità è \(v = \lambda / T\). Spesso questo si scrive introducendo la \emph{frequenza}, \(f = 1 / T\), misurata in \(\SI{}{Hz} = 1 / \SI{}{s}\).

Così, scriviamo 
%
\boxalign{
\begin{align}
v = \lambda f
\,.
\end{align}}


\begin{exo}[Legge di Snell]
Da impostare a lezione, senza finire il conto. 
I fronti d'onda arrivano paralleli, il vettore d'onda ha un angolo \(\theta _{\text{in}}\) dalla normale. \(v = \lambda f\) deve valere sia dentro che fuori dal vetro: visto che \(f\) è la stessa, abbiamo 
%
\begin{align}
    \frac{ \lambda _{\text{out}}}{\lambda _{\text{in}}} = \frac{n _{\text{in}}}{n _{\text{out}}}
    \,,
\end{align}
%
dove definiamo l'indice di rifrazione con \(v = c/n\). 
    
Se \(D\) è la distanza fra le creste misurata lungo l'interfaccia, abbiamo 
%
\begin{align}
    \frac{\lambda_{i}}{D} = \sin(\theta_{i})
    \,,
\end{align}
%
per \(i = \text{in}, \text{out}\). 

Da qui si ricava 
%
\begin{align}
n _{\text{in}} \sin(\theta _{\text{in}}) = 
n _{\text{out}} \sin(\theta _{\text{out}})
\,.
\end{align}
\end{exo}

Abbiamo menzionato prima il fatto che la teoria ondulatoria prevede che questo fenomeno di rifrazione avvenga simultaneamente a quello della riflessione. 
La teoria permette di dare delle precise stime su quanta luce viene riflessa e quanta viene trasmessa. Non entriamo nel dettaglio --- chi è interessata può trovare una discussione più completa in diversi testi di ottica, ad esempio Vistnes \cite[]{vistnesReflectionTransmissionPolarization2018}. Menzioniamo solo le quantità in gioco: la luce trasmette energia, ma come la quantifichiamo? 
Immaginiamo di avere un sensore in grado di rilevare questa energia. Questo sensore avrà una certa area, e lo terremo acceso per un certo tempo. 
Se l'onda luminosa è costante nel tempo, l'energia che rileveremo tenendo il detector acceso per un tempo \(2T\) sarà due volte quella che avremmo rilevato in un tempo \(T\). Allo stesso modo, se raddoppiamo l'area del sensore l'energia che rileviamo raddoppierà.

Per questo motivo, quantifichiamo l'energia della luce usando l'\textbf{intensità}, che è definita come l'energia per unità di tempo e per unità di area: 
%
\begin{align}
I = \frac{E}{A t}
\,,
\end{align}
%
e si misura in \(\SI{}{J / m^2 / s}  = \SI{}{W/m^2}\). 

In termini di intensità possiamo definire i coefficienti di riflessione e trasmissione: se l'intensità dell'onda incidente è \(I _{\text{in}}\), l'intensità dell'onda riflessa è \(I_{r}\) mentre quella dell'onda trasmessa (ovvero rifratta) è \(I_{t}\), allora definiamo 
%
\begin{align}
\mathcal{R} = \frac{I_r}{I _{\text{in}}}
\qquad \text{e} \qquad
\mathcal{T} = \frac{I_t}{I _{\text{in}}}
\,,
\end{align}
%
che dovranno soddisfare \(\mathcal{R} + \mathcal{T} = 1\).

Un altro fenomeno spiegato dalla teoria ondulatoria è la diffrazione: se illuminiamo una fessura, e restringiamo la fessura, l'immagine luminosa inizialmente si restringe, ma a un certo punto inizia ad allargarsi.

Abbiamo definito la velocità della luce, ma quanto vale numericamente? 

\begin{exo}[Esperimento di Michelson, velocità della luce]

% Parametri: doppia distanza fra gli specchi \(D = \SI{3972.46}{ft}\), variazione lineare della posizione: \(d = \SI{114.85}{mm}\), raggio \(r = \SI{28.672}{feet}\), rivoluzioni al secondo: \(n = \SI{257.36}{Hz}\).

L'apparato che lo scienziato americano Michelson ha utilizzato nel 1880 \cite[]{michelsonExperimentalDeterminationVelocity1880} per misurare la velocità della luce permette una misura molto accurata.
Per una descrizione dell'apparato e una figura vedere Wikipedia \cite[]{wikipediacontributorsFizeauFoucaultApparatus2019}.

Alcuni parametri di una singola misura di Michelson: distanza fra gli specchi \(D = \SI{605.40}{m}\); variazione lineare della posizione: \(d = \SI{114.85}{mm}\), raggio \(r = \SI{8.7392}{m}\); rivoluzioni al secondo: \(n = \SI{257.36}{Hz}\).

Qual'è la velocità della luce che dovrebbe calcolare Michelson?

Formula corretta: 
%
\begin{align}
c _{\text{exp}} = \frac{2 \times 2 \pi \times 2 D n}{\arctan(d / r)} \approx \num{.994} c
\,.
\end{align}
\end{exo}

\section{Maxwell e Einstein}

Nei primi anni 1860, James Clerk Maxwell riesce a unificare tutte le osservazioni riguardanti i fenomeni elettrici e magnetici nelle sue celebri 4 equazioni: queste descrivono i due \emph{campi vettoriali} elettrico \(\vec{E}\) e magnetico \(\vec{B}\). 
Ad ogni punto nello spazio per ogni istante temporale sono assegnati questi due vettori, e inoltre abbiamo la carica elettrica \(Q\) e la corrente elettrica \(\vec{J}\). 

Le quattro equazioni si possono dividere in due gruppi: due equazioni sono identità che pongono dei vincoli su come si comportano e come interagiscono i campi elettrico e magnetico, le altre due invece coinvolgono la carica \(Q\) e la corrente \(\vec{J}\) e descrivono come queste generino i campi elettrico e magnetico, rispettivamente. 

Se si scrive le equazioni di Maxwell ponendo \(Q = \vec{J} = 0\), ovvero nel vuoto, si trova un'equazione che descrive un'onda che viaggia proprio alla velocità \(c\) di cui abbiamo parlato prima. 
Questo fa sospettare che la luce sia un fenomeno elettromagnetico! 

È proprio così: non entriamo nei dettagli, ma si può dimostrare sperimentalmente che agitando della carica elettrica avanti e indietro si può generare luce.

Qui entra in gioco l'unificazione di cui parlavamo: la \emph{luce visibile} di cui abbiamo iniziato a parlare all'inizio è solo una tipologia particolare di una classe più grande di fenomeni che possiamo chiamare \emph{luce}: la \textbf{radiazione elettromagnetica}.

\emph{Radiazione} è un termine generale per descrivere qualcosa che si diffonde nello spazio portando energia.

\todo[inline]{Inserire tabella/grafico con le varie lunghezze d'onda della luce.}

A fine 1800 la questione della luce sembrava risolta, ma ben presto si scoprì un effetto che metteva i bastoni fra le ruote: l'\emph{effetto fotoelettrico}.

Detta in sintesi, la questione è questa: se si manda della luce su certi materiali si può ``eccitare'' i loro elettroni, ovvero strapparli agli atomi dei quali facevano parte.
La parte che non funziona con la teoria ondulatoria di Maxwell è il fatto che, mantenendo la stessa intensità di luce ma aumentandone la \emph{frequenza} l'energia cinetica degli elettroni strappati aumenta. 

Questo non sembra avere senso: stiamo mandando la stessa quantità di energia, e gli elettroni uscenti ne hanno di più? 
(Non c'è violazione della conservazione dell'energia: gli elettroni sono più energetici ma ce ne sono di meno)

Si scopre così che a ogni frequenza si può associare una energia caratteristica, e la relazione che lega energia e frequenza è \emph{lineare}! La relazione è 
%
\begin{align}
E = h f
\,,
\end{align}
%
dove \(E\) è l'energia caratteristica, \(f\) è la frequenza della luce, mentre 
%
\begin{align}
h = \SI{6.62607015e-34}{J / Hz}
\,,
\end{align}
%
è la costante di proporzionalità, che si chiama \textbf{costante di Planck}. 
Nota interessante: il valore è esatto dal 2018. Questo numero è così importante che definiamo le nostre unità di misura a partire da esso.

Cosa vuol dire ``energia caratteristica''? 
L'ipotesi avanzata da Einstein e Planck per spiegare l'effetto fotoelettrico e altri fenomeni, come la radiazione di corpo nero, fu quella che la luce fosse suddivisa in quanti indivisibili, che poi furono chiamati \textbf{fotoni}, caratterizzati proprio da un'energia data da \(E = hf\). 

Il grande scienziato Millikan nella sua \emph{Nobel lecture} del 1924
\autocite[]{millikanRobertMillikanNobel1924} parla così della teoria della luce quantizzata: 

\begin{quotation}
``It may be said then without hesitation that it
is not merely the Einstein equation which is having extraordinary success at
the moment, but the Einstein conception as well.

But until it can account for the facts of interference and the other effects
which have seemed thus far to be irreconcilable with it, we must withhold
our full assent.''
\end{quotation}

Era chiaro, dunque, che la teoria corpuscolare non poteva spiegare del tutto 

\printbibliography

\end{document}
