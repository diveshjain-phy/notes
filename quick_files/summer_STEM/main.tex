\documentclass{article}
\usepackage[utf8]{inputenc}

\usepackage{textcomp}
\usepackage[T1]{fontenc}
\usepackage{multirow}
\usepackage{float}
\usepackage[caption = false]{subfig}
\usepackage{longtable}
\usepackage{listings}
\usepackage{mathtools}
\DeclareMathOperator{\tr}{Tr}
\usepackage{commath}
\usepackage{bbold}
\usepackage{xcolor}
\usepackage{physics}
%\usepackage[margin=1.8cm]{geometry}

\usepackage{tikz-cd}
\usepackage{amsmath}
\usepackage{amsfonts}
\usepackage{amssymb}
\usepackage{amsthm}
\usepackage{graphicx}
\usepackage[colorinlistoftodos]{todonotes}
\usepackage[colorlinks=true, allcolors=blue]{hyperref}
\usepackage{siunitx}
\sisetup{separate-uncertainty=true}

\usepackage[sc]{mathpazo}
\linespread{1.05}         % Palladio needs more leading (space between lines)
\usepackage[T1]{fontenc}

\newcommand{\diag}[1]{\text{diag}\qty(#1)}
\newcommand{\const}{\text{const}}
\newcommand{\sign}[1]{\text{sign}\qty(#1)}
\renewcommand{\H}{\mathcal{H}}
\renewcommand{\dim}{\text{dim}}
\newcommand{\C}{\mathbb{C}}
\newcommand{\R}{\mathbb{R}}
\newcommand{\N}{\mathbb{N}}
\newcommand{\Z}{\mathbb{Z}}

\renewcommand{\var}[1]{\text{var} \qty(#1)}
\newcommand{\average}[1]{\langle #1 \rangle}

\newcommand\mybox[1]{%
  \fbox{\begin{minipage}{0.9\textwidth}#1\end{minipage}}}

\usepackage[italian]{babel}

\usepackage[
backend=biber,
style=alphabetic,
sorting=nyt
]{biblatex}

\addbibresource{STEM_Indivisibility.bib}

\title{L'indivisibilità del fotone: STEM 2020}

\begin{document}

\maketitle

Iniziamo con un quiz: \textbf{cos'è la luce}?

\begin{enumerate}[label=\emph{\alph*})]
    \item Un'onda.
    \item Una particella.
    \item Sia un'onda che una particella. 
    \item Né un'onda né una particella. 
    \item Mi astengo.
\end{enumerate}

La domanda, posta in questo modo vago (e provocatorio), rischia di allontanarsi dall'ambito di ricerca della Fisica: non siamo generalmente interessati a descrivere l'\emph{essenza fondamentale} di un oggetto, bensì cerchiamo di fornire modelli che ne descrivano il comportamento e che siano in grado di predire i risultati degli esperimenti che facciamo. 
Questo approccio all'investigazione spesso riesce a darci grande chiarezza sul come pensare agli oggetti fisici. 

% TODO keep this bit?

Tenendo questo in mente, cerchiamo di rispondere alla domanda.
Innanzitutto, dobbiamo chiarire qual'è l'oggetto fisico che vogliamo descrivere.

Partiamo dalla descrizione più intuitiva: la luce è quello che vediamo con gli occhi. 
Vedremo che, investigando questo fenomeno nel dettaglio, questa definizione risulterà restrittiva e ci porterà a ridefinire quella che vediamo con gli occhi come \emph{luce visibile} --- quindi, c'è anche della ``luce invisibile''.

Per ora, tuttavia, partiamo dalle prime teorie della luce, che volevano descrivere solo quella visibile.

Lo storico dibattito è quello fra Newton e Huygens. 

Newton (1642--1726/1727), il più famoso dei due, proponeva una \textbf{teoria corpuscolare} della luce. 

Secondo questa teoria, la luce è composta di corpuscoli molto leggeri che viaggiano ad una velocità finita. 

Quali fenomeni voleva spiegare Newton con questa teoria?
Riflessione e rifrazione, innanzitutto: lui ha dimostrato che facendo passare la luce per un prisma si può scomporla nelle sue componenti cromatiche, e --- qui la chiave! --- questo processo è reversibile: con un altro prisma è possibile ricomporre la luce bianca.

La luce, dunque, viene deviata quando passa per un'interfaccia, e di quanto viene deviata dipende dal suo colore. 

La riflessione invece è ciò che accade per uno specchio: la luce ``rimbalza'' ad un angolo uguale a quello di incidenza.

La teoria di Newton ha delle spiegazioni per questi fenomeni, seppur non completamente soddisfacenti.
Newton la mette in pratica per costruire telescopi: il telescopio Newtoniano utilizza solo specchi invece che lenti, eliminando quindi l'aberrazione cromatica che le caratterizza. 

La teoria corpuscolare ha dietro l'apparato della nuovissima meccanica Newtoniana: la luce viene descritta come un insieme di piccole particelle con massa molto ridotta che si muovono molto velocemente rispettando le leggi di Newton: inerzia, \(\vec{F} = m \vec{a}\) e azione-reazione.

Un vantaggio della teoria corpuscolare è il fatto che non richiede un \emph{etere luminifero} (portatore della luce). 

La teoria di Huygens è \textbf{ondulatoria}: vede la luce come un'onda che si propaga in un qualche tipo di etere, in modo simile a come si comportano le onde nell'acqua poco profonda.

Questa teoria spiega, al contrario di quella corpuscolare, il fatto che all'interfaccia fra due mezzi ci siano sia riflessione che rifrazione: nella meccanica ondulatoria questo è un fatto naturale e atteso.

Nel contesto della meccanica ondulatoria, si può derivare la legge di Snell, che governa l'angolo al quale la luce viene diffratta. 

Iniziamo però con i preliminari della teoria ondulatoria: guardiamo il caso monodimensionale per semplicità, guardare il problema in più dimensioni è utile ed interessante ma le caratteristiche principali si possono evincere già da qui. 

La caratteristica principale delle onde è la periodicità: partiamo da quella che si chiama onda \emph{monocromatica}, il caso più semplice, che ha un periodo ben definito.
Le equazioni che ora scriviamo potrebbero descrivere, ad esempio, la variazione del livello dell'acqua lungo una linea che tracciamo in uno stagno: chiamiamola \(A\).
\(A\) può assumere un valore positivo o negativo, se vale 0 significa che il livello è quello che avrebbe lo stagno senza perturbazioni.

Se fissiamo un istante temporale \(t_0 \) e ``fotografiamo'' l'onda, la formula che la descrive è 
%
\begin{align}
\eval{A(x)}_{t = t_0 } = A_0 \sin(2 \pi \frac{x}{\lambda } +\varphi )
\,.
\end{align}

\(A_0 \) è l'ampiezza massima che raggiunge l'onda, l'altezza dei picchi e la profondità delle valli. \(\lambda \) è una lunghezza caratteristica dell'onda, detta \emph{lunghezza d'onda}: è la distanza fra due picchi vicini, o fra due valli vicine; se \(x\) --- la posizione spaziale lungo la linea --- varia aumentando o diminuendo di esattamente \(\lambda\) l'ampiezza non cambia.
Questo è dovuto al fatto che la funzione seno ha una periodicità di \(2 \pi \). 

Il termine \(\varphi \) si chiama \emph{fase}, ed esprime il fatto che nel punto che chiamiamo \(x=0\) l'onda può essere in un massimo, un minimo, o a qualunque punto della sua evoluzione. 
Visto che \(\varphi \) è arbitrario, è arbitraria anche la scelta di un seno piuttosto che un coseno: cambiando \(\varphi \) di \(\pi /2\) possiamo passare dall'uno all'altro. 

Se vogliamo includere la dipendenza dal tempo, otterremo una formula di questo tipo: 
%
\begin{align}
A(x, t) = A_0 \sin(2 \pi \qty( \frac{x}{\lambda } - \frac{t}{T}) + \varphi )
\,.
\end{align}

Abbiamo inserito un tempo caratteristico \(T\), chiamato \emph{periodo}: se fissiamo la posizione \(x\) e guardiamo solo un punto, l'onda torna al punto dov'era prima dopo un tempo \(T\). 
C'è un meno perché stiamo assumendo che l'onda avanzi verso destra, ovvero verso le \(x\) positive: se \(x\) aumenta di un poco, per mantenerci nella stessa posizione (``viaggiando con l'onda'', ovvero tornando allo stesso valore dell'argomento del seno) dobbiamo aumentare un poco anche \(t\), ovvero aspettare un poco. 

Se aspettiamo un tempo \(T\) l'onda si è spostata di una lunghezza \(\lambda \): quindi, la sua velocità è \(v = \lambda / T\). Spesso questo si scrive introducendo la \emph{frequenza}, \(f = 1 / T\), misurata in \(\SI{}{Hz} = 1 / \SI{}{s}\).

Così, scriviamo 
%
\boxalign{
\begin{align}
v = \lambda f
\,.
\end{align}}

\textbf{Esercizio: legge di Snell}

Da impostare a lezione, senza finire il conto. 
I fronti d'onda arrivano paralleli, il vettore d'onda ha un angolo \(\theta _{\text{in}}\) dalla normale. \(v = \lambda f\) deve valere sia dentro che fuori dal vetro: visto che \(f\) è la stessa, abbiamo 
%
\begin{align}
\frac{ \lambda _{\text{out}}}{\lambda _{\text{in}}} = \frac{n _{\text{in}}}{n _{\text{out}}}
\,,
\end{align}
%
dove definiamo l'indice di rifrazione con \(v = c/n\). 

Se \(D\) è la distanza fra le creste misurata lungo l'interfaccia, abbiamo 
%
\begin{align}
\frac{\lambda_{i}}{D} = \sin(\theta_{i})
\,,
\end{align}
%
per \(i = \text{in}, \text{out}\). 

Da qui si ricava 
%
\begin{align}
n _{\text{in}} \sin(\theta _{\text{in}}) = 
n _{\text{out}} \sin(\theta _{\text{out}})
\,.
\end{align}

Un altro fenomeno spiegato dalla teoria ondulatoria è la diffrazione: se illuminiamo una fessura, e restringiamo la fessura, l'immagine luminosa inizialmente si restringe, ma a un certo punto inizia ad allargarsi.

Abbiamo definito la velocità della luce, ma quanto vale numericamente? 

\textbf{Esercizio: esperimento di Michelson, velocità della luce.}

% Parametri: doppia distanza fra gli specchi \(D = \SI{3972.46}{ft}\), variazione lineare della posizione: \(d = \SI{114.85}{mm}\), raggio \(r = \SI{28.672}{feet}\), rivoluzioni al secondo: \(n = \SI{257.36}{Hz}\).

L'apparato che lo scienziato americano Michelson ha utilizzato nel 1880 \cite[]{michelsonExperimentalDeterminationVelocity1880} per misurare la velocità della luce permette una misura molto accurata.
Per una descrizione dell'apparato e una figura vedere Wikipedia \cite[]{FizeauFoucaultApparatus2019}.

Distanza fra gli specchi \(D = \SI{605.40}{m}\); variazione lineare della posizione: \(d = \SI{114.85}{mm}\), raggio \(r = \SI{8.7392}{m}\); rivoluzioni al secondo: \(n = \SI{257.36}{Hz}\).

Formula corretta: 
%
\begin{align}
c = \frac{2 \times 2 \pi \times 2 D n}{\arctan(d / r)} \approx \num{.994} c
\,.
\end{align}



\printbibliography

\end{document}
