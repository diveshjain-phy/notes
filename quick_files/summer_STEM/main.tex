\documentclass{article}
\usepackage[utf8]{inputenc}

\usepackage{textcomp}
\usepackage[T1]{fontenc}
\usepackage{multirow}
\usepackage{float}
\usepackage[caption = false]{subfig}
\usepackage{longtable}
\usepackage{listings}
\usepackage{mathtools}
\DeclareMathOperator{\tr}{Tr}
\usepackage{commath}
\usepackage{bbold}
\usepackage{xcolor}
\usepackage{physics}
%\usepackage[margin=1.8cm]{geometry}

\usepackage{tikz-cd}
\usepackage{amsmath}
\usepackage{amsfonts}
\usepackage{amssymb}
\usepackage{amsthm}
\usepackage{graphicx}
\usepackage[colorinlistoftodos]{todonotes}
\usepackage[colorlinks=true, allcolors=blue]{hyperref}
\usepackage{siunitx}
\sisetup{separate-uncertainty=true}

\usepackage[sc]{mathpazo}
\linespread{1.05}         % Palladio needs more leading (space between lines)
\usepackage[T1]{fontenc}

\newcommand{\diag}[1]{\text{diag}\qty(#1)}
\newcommand{\const}{\text{const}}
\newcommand{\sign}[1]{\text{sign}\qty(#1)}
\renewcommand{\H}{\mathcal{H}}
\renewcommand{\dim}{\text{dim}}
\newcommand{\C}{\mathbb{C}}
\newcommand{\R}{\mathbb{R}}
\newcommand{\N}{\mathbb{N}}
\newcommand{\Z}{\mathbb{Z}}

\renewcommand{\var}[1]{\text{var} \qty(#1)}
\newcommand{\average}[1]{\langle #1 \rangle}

\newcommand\mybox[1]{%
  \fbox{\begin{minipage}{0.9\textwidth}#1\end{minipage}}}

\usepackage[italian]{babel}

\title{L'indivisibilità del fotone: STEM 2020}

\begin{document}

\maketitle

\section{Prerequisiti}

\begin{enumerate}
    \item Cauchy-Schwartz: \(\expval{X^2} \geq \expval{X}^2 \)
\end{enumerate}

\section{Teoria}

\begin{enumerate}
    \item Teoria ondulatoria della luce
    \begin{enumerate}
        \item Coefficienti di trasmissione e riflessione
        \item Intensità
    \end{enumerate}
    \item Teoria particellare della luce
    \item Applicazioni di probabilità: probabilità condizionata, come misurare sperimentalmente le probabilità a partire dai conteggi osservati
    \item Concetti generali di rilevatori di particelle
    \begin{enumerate}
        \item Fotomoltiplicatori
        \item Coincidenze
    \end{enumerate}
    \item Apparato sperimentale
    \begin{enumerate}
        \item Cristallo birifrangente
        \item Beamsplitter
        \item Struttura del nostro apparato
    \end{enumerate}
    \item Previsioni dei nostri due modelli per l'apparato:
    \begin{enumerate}
        \item Definizione di \(g = P_{23}/(P_2 P_3) = I_{23}/(I_2 I_3)\)
        \item \(g \geq 1\) nel modello ondulatorio
        \item \(g = 0\) nel modello particellare
    \end{enumerate}
\end{enumerate}



\section{Laboratorio}

\end{document}
