\documentclass[main.tex]{subfiles}
\begin{document}

Sessione di 6h per la Summer STEM Academy dell'estate 2020. 

L'idea è che alla fine della sessione gli studenti abbiano un quadro chiaro di quale sia stato, a grandi linee, lo sviluppo storico delle teorie della luce e come, con apparati moderni, si possa mostrare che la luce è composta di parti indivisibili: i fotoni. 

La cosa importante è che tutti i passaggi logici che permettono di trarre una conclusione sulla natura della luce a partire dalla lettura dei numeri che escono dai rivelatori siano chiari, e che non rimangano parti ``magiche''. 

\section{Prerequisiti matematici}

\begin{enumerate}
    \item Cauchy-Schwartz: \(\expval{X^2} \geq \expval{X}^2 \)
    \item Statistica Bayesiana
\end{enumerate}

\section{Teoria}

\begin{enumerate}
    \item Teoria ondulatoria della luce
    \begin{enumerate}
        \item Coefficienti di trasmissione e riflessione
        \item Intensità
    \end{enumerate}
    \item Teoria particellare della luce
    \item Applicazioni di probabilità: probabilità condizionata, come misurare sperimentalmente le probabilità a partire dai conteggi osservati
    \item Concetti generali di rilevatori di particelle
    \begin{enumerate}
        \item Fotomoltiplicatori
        \item Coincidenze
    \end{enumerate}
    \item Apparato sperimentale
    \begin{enumerate}
        \item Cristallo birifrangente
        \item Beamsplitter
        \item Struttura del nostro apparato
    \end{enumerate}
    \item Previsioni dei nostri due modelli per l'apparato:
    \begin{enumerate}
        \item Definizione di \(g = P_{23}/(P_2 P_3) = I_{23}/(I_2 I_3)\)
        \item \(g \geq 1\) nel modello ondulatorio
        \item \(g = 0\) nel modello particellare
    \end{enumerate}
\end{enumerate}


\end{document}