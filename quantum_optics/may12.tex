\documentclass[main.tex]{subfiles}
\begin{document}

\section{Quantum Communications Move to Space}

\marginpar{Tuesday\\ 2020-5-12, \\ compiled \\ \today}

Today we want to show how quantum communications can be performed at very long distances, and how this can be used to investigate quantum foundations. 
We will see some examples.

We need to measure qubits in order to gain physical insight: depending on the kind of measurement we perform we get different results. \textbf{Quantum communication} is about sharing a quantum state between distant partners. 

We have the usual naming conventions: Alice, Bob and sometimes Charlie want to communicate, while Eve wants to disrupt. 

There is a European project called Open QKD. 
It tries to find applications of quantum communications to various practical scenarios.

The Holy Grail of cryptography is \textbf{unconditional security} in communication. 

We need to describe the degrees of freedom in our system. 
We use degrees of freedom which are resistant to propagation: examples are polarization modes and temporal modes. For this, we need an adequate spectral support.

Modes connected to angular momentum decay too quickly for true long-range. 

What do we want to do with Space Quantum Communications? 
To date, SQComm have been demonstrated up to LEO. We want to extend this. 

The orbits are classified into LEO, MEO and GEO. 
The low orbits pass by very quickly. 

We can derive theoretically how many common qubits \(M\) we need in order to get a shared key of length \(\ell\) with a quberr \(Q\).

Satellites are equipped with retroreflectors. 
This allows for time-of-flight measurements of the photons: so, we can do geodesy, measure precisely the deformation of the Earth's surface. 

We can do quantum communication at LEO! We send ``control'' pulses every \SI{100}{ms}, and then single photons at \SI{100}{MHz}.

\end{document}
