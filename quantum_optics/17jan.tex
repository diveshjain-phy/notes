\documentclass[main.tex]{subfiles}
\begin{document}

\section*{Fri Jan 17 2020}

Dialogue: ``take a photon''. 

Nonlinear crystals: in general the formula for the polarization vector in terms of the electric field looks like 
%
\begin{align}
P_{i} = \chi_{ij}^{(1)} E_{j} + \chi_{ijk}^{(2)} E_{j} E_{k} + \dots
\,,
\end{align}
%
with \(\chi^{(m)}\) being the \(m\)-th order susceptibility tensor: the linear susceptibility is what the basic index of refraction is based on, but the higher order interactions allow for interesting effects. 

For example, green laser pointer start off with an infrared light, and then cuts the frequency in half. 

An atom is ionized, the free electron is accelerated by the light's electric field (thus, it absorbs photons), and then it is reabsorbed. 

This practically allows for up-conversion of low-frequency light.

Typo in formula D4: the \(E^{(+)}\) is the one with the creation operator. 

The photon going straight is the ``pump'' photon, the other two nonlinear photons are the ``idler'' and ``signal''. 

Typical rates of production: \num{e-7} to \num{e-11}.

The process is called \emph{Spontaneous Parametric Down-Conversion}.

In type-1 SPDC we have the same polarization, and a cone is emitted. In type-2 SPDC they have orthogonal polarizations: then they perceive different susceptibility tensors, and two different cones of light are emitted. 
The intersection of the two cones are those in which we have polarization entangled light. 

With some weird second-quantization notation we can write the Bell state 
%
\begin{align}
\ket{\psi } = \frac{1}{\sqrt{2}} \qty(\ket{H}_{s} \ket{V}_{i} + \ket{V}_{s} \ket{H}_{i})
\,,
\end{align}
%
where \(s\) and \(i\) denote signal and idler, while \(H\) and \(V\) denote horizontal and vertical polarizations. 

Schrödinger: unlike the classical case, knowledge of a full system does \emph{not} come from the knowledge of all its parts. 

What is the quantum mechanical description of a beam splitter? 
It does \emph{not} work to multiply the transmission and reflection coefficients by annihilation operators. 
The actual operatorial description must describe all of the four sides of the BS: we will have a 
%
\begin{align}
\left[\begin{array}{c}
\hat{a}_{2}  \\ 
\hat{a}_{3}
\end{array}\right] = 
M \left[\begin{array}{c}
\hat{a}_0  \\ 
\hat{a}_{1}
\end{array}\right]
\,,
\end{align}
%
with \(M\) being a unitary matrix: 
%
\begin{align}
M = \exp(i \frac{\pi}{4} \qty(\hat{a}^\dag_{0} \hat{a}_{1} + \hat{a}_{0} \hat{a}_{1}^\dag))
\,,
\end{align}
%
so the process is coherent. So, if we send a state \(\ket{01}\) (in the Hilbert space of photons going right, down before the BS) to the BS it returns 
%
\begin{align}
\frac{1}{\sqrt{2}} \qty(i \ket{10} + \ket{01})
\,
\end{align}
%
in the space of photons going right, down after the BS. 
For coherent states we have 
%
\begin{align}
\ket{0 \alpha }  \rightarrow \ket{i \frac{\alpha}{\sqrt{2}}} \otimes \ket{\frac{\alpha}{\sqrt{2}}}
\,,
\end{align}
%
so there is no entanglement. 
If we send in  two photons, one from up and one from the left, we will always find two photons coming out to the right or downwards. The non-interactions between the photons and the BS destructively interfere. 

The photons must arrive ``at the same time'' for this: the time difference must be small compared to the coherence time of the beam, the time before which the waves have a \emph{phase jump}.  
%
\begin{align}
R _{\text{coincidences}} = 1 - \exp(- (\Delta \omega )^2 (t-t_0 )^2)
\,,
\end{align}
%
an inverse Gaussian: \(\Delta \omega \) is the \emph{bandwidth} and it gives the inverse of the std of the gaussian. 

\end{document}
