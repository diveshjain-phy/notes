\documentclass[main.tex]{subfiles}
\begin{document}

\marginpar{Friday\\ 2020-3-20, \\ compiled \\ \today}

Today we will discuss another application of entanglement: Quantum Key Distribution.

\section{Quantum Key Distribution}

The idea is that it is possible to exchange two keys between two people, in an \emph{unconditional} way, as opposed to the current way of doing security, which is based on hard-to-solve problems. 

If Alice and Bob have a shared key \(K\), they can use a \emph{one-time-pad} to communicate. 

Say \(X\) is our message: then Alice constructs \(Y = X \oplus K\), which is completely random since \(K\) is. 

When Bob receives the message, he does \(Y \oplus K = X \oplus K \oplus K = X\). 
This is old classical cryptography, it was discovered by Shannon. 

This works as long as \(K\) is indeed \emph{one-time}: if it is reused, an attacker can start reconstructing the message: otherwise, . 

So, the QKD is about transmitting the key. 
The most famous protocol to do this is BB84. 
Most of the things which will be covered today are covered in 
Rev Mod Phys 81, 1301 (2009).

The basis of the algorithm is a public quantum channel between Alice and Bob. The thing which is needed is for the channel to be \emph{verified}, so that Alice is sure to be talking to Bob. 
Eve can be watching the passing qubits.  

Alice can prepare four possible states, which she associates to two logical bits: \(\ket{0}\) and \(\ket{+}\) are associated with 0, while \(\ket{1}\) and \(\ket{-}\) are associated with 1. 

Bob either measures \(\sigma_{z}\) or \(\sigma_{x}\). 
It they use the same basis, Bob measures the same thing Alice sent. 

If they use a different basis, then the result is random, since \(\abs{\braket{\pm}{0}}^2 =  \abs{\braket{\pm}{1}}= 1 / 2\). 

So, there is a need to do \emph{sifting} later: after the transmission, Alice and Bob say publicly which bases they used. 
With this information, they discard the qubits for which they used different bases. 

Any measurement by the eavesdropper necessarily increases the Q-Bit Error Rate, which can be detected by Alice and Bob. 

If Eve performs an ``Intercept and Resend'' attack, \(1/2\) of the time she will get the basis wrong, and those times she will create an error half of the time. 
So, in this case we will have a QBER equal to \SI{25}{\percent}. 

This can be measured by selecting, a posteriori, some qubits to be used as a check. 

Alice could also use the states \(\ket{+i}\) and \(\ket{-i}\), which are eigenstates of \(\sigma_{y}\). Using these states we can increase the security. 

The probabilities can be changed: we can have probabilities \(\epsilon \) for \(\sigma_{x}\) and \(\sigma_{y}\), while we use \(1-2\epsilon \) probability for \(\sigma_{z}\). 

Then, we encode the key using only \(\sigma_{z}\), so we are communicating \(\approx1 -4 \epsilon  \) of the time. 
We use the qubits encoded with \(\sigma_{x}\) and \(\sigma_{y}\) to measure how many times Eve has been measuring, since she will be always measuring \(\sigma_{z}\). 

Eve cannot determine which basis is being used by Alice, since the mixed states corresponding to having \(\ket{0}\) or \(\ket{1}\) with equal probability and to having \(\ket{+}\) or \(\ket{-}\) with equal probability are the same, \(\rho = \frac{1}{2} \mathbb{1}\). 

This works well if the devices are working as expected.
If the implementation is insecure, QKD can still be breached: so, we want to get \emph{device-independent} implementations. 

A possible implementation is one in which we use an entangled source which gives us pairs with \(\ket{\Psi^{-}}\). 

Then, Alice measures \(\sigma_{x}\) and \(\sigma_{z}\) and so does Bob. 
Using the correlations between Alice and Bob we can determine whether the qubits are being generated entangled or not. 
The state is 
%
\begin{align}
\ket{\Psi^{-}} = \frac{1}{\sqrt{2}} \qty(\ket{01} - \ket{10})
= \frac{1}{\sqrt{2}} \qty(\ket{+-} - \ket{- +})
\,,
\end{align}
%
so the measurements are precisely anticorrelated for both bases 01 and +-. 

So, we can do QKD with black boxes.
However, this is difficult to implement. 

The receiver is more vulnerable to attacks, since they can receive any signal from the outside. 
So, we want to do \emph{measurement device independent} coding: Alice and Bob each prepare the four states with equal probabilities, they send it to an untrusted station \(C\), in which there is a beamsplitter. 
This performs a Bell measurement: if \(C\) sees a coincidence, then they communicate this to \(A\) and \(B\): this means that the measured state is \(\ket{\Psi^{-}}\). This means that the original qubits were opposite, but there is no information as to what the states originally were. 

If the operator at \(C\) does anything else other than the Bell measurement, this can be detected by Alice and Bob by looking at the QBER. 

By adding some more detectors we can also measure \(\ket{\Psi^{+}}\); then we do efficient BB84 by mostly sending in one basis. 
This way, we can also get an efficiency of \(1- \epsilon \). 

Now, how do we measure the secret Key Rate? 
It is defined by 
%
\begin{align}
r = I_{AB} - I_E 
\,,
\end{align}
%
where \(I_{AB}\) is the mutual information of Alice and Bob, while \(I_E\) is the information of the Eavesdropper. 

The definition of \(I_{AB}\) is: 
%
\begin{align}
I_{AB} = H(A) - H(A|B)
\,,
\end{align}
%
where \(H(A)\) is the Shannon entropy of Alice: 
%
\begin{align}
H(A) = - \sum_{x = 0,1} p_x \log_2  p_x
\,.
\end{align}

If the bits are random, then \(H(A )= 1\). On the other hand, 
%
\begin{align}
H(A|B) = - \sum _{\substack{a=0,1 \\ b=0,1} } p_{ab} \log_2 p_{a|b}
\,,
\end{align}
%
which is zero if Alice and Bob's bits are perfectly correlated. It can be shown that 
%
\begin{align}
H(A|B) = h_2  (Q) = -Q \log_2 Q - (1-Q) \log_2 (1-Q)
\,,
\end{align}
%
where \(Q\) is the error rate. 

So, we write this as \(I_{AB} = 1 - h_2 (Q)\). 
\(Q\) is the number of bits which are wrong, and the formula tells us that in order to correct these errors we need to reveal \(h_2 (Q)\). 

It can be shown that the information of the eavesdropper in the BB84 case is 
%
\begin{align}
I_E = h_2 (Q)
\,.
\end{align}

With the 6-state coding we have 
%
\begin{align}
I_E = Q + (1- Q) h_2 \qty(\frac{1 - 3 Q / 2}{1 - Q})
\,.
\end{align}

This is purely quantum: classically there is no connection between \(I_E\) and \(Q\). 
So, for BB84 the error rate must be \(Q \leq \SI{11}{\percent}\).

So, we need to do \emph{error correction} and then \emph{privacy amplification} so that the eavesdropper has no residual information. 
Privacy amplification reduces the key length by a factor \(1/r\), so we get bits which are surely secure. 

What we do, practically, is to use an attenuated laser. The state of the laser is a coherent one: 
%
\begin{align}
\ket{\alpha } = e^{- \abs{\alpha }^2 / 2} \sum _{n} \frac{\alpha ^{n}}{\sqrt{n!}} \ket{k}
\,.
\end{align}

We can attenuate the laser so that we have \(\abs{\alpha } = \mu \), with \(\mu = 1\): usually we will have zero photons, often one, but sometimes \(2\): this is an issue, since the attacker can take one and leave the other: this is called Photon Number Splitting. 

What can be done is to use a \emph{decoy state}: we use three values for the intensity, \(\mu = 1 \), but  also  \( \nu_1 = .1 \) and \(\nu_2 =  0\). Using this, we can see whether there is a PNS attack or not. 

So, from the source point of view we have no issue in using a classical source like a laser. 

\end{document}
