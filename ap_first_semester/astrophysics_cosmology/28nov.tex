\documentclass[main.tex]{subfiles}
\begin{document}

\marginpar{Thursday\\ 2019-11-28, \\ compiled \\ \today}
% \section*{Thu Nov 28 2019}

% We derived 
% %
% \begin{align}
%   \expval{T} = -\frac{1}{3} \frac{E _{\text{grav}}}{V}
% \,,
% \end{align}
% %
% so now we proceed: we want a relation between kinetic and gravitational energy densities. 
Now, the question we want to as is: is this equilibrium configuration \textbf{stable}? This is equivalent to asking whether the system is gravitationally bound, \(E _{\text{grav}} < 0\), which as we have shown is equivalent to \(\expval{P}> 0\).  

In order to answer this question we shall use a statistical-mechanics, microscopic approach. 

We consider a cubic box of volume \(V = L^3\) with \(N\) particles inside it, each of which has a velocity \(\vec{v} = (v_x, v_y, v_z)^{\top}\) and a momentum \(p\). Let us select a face of the box, which we assume to be perpendicular to the \(x\) axis. Each particle will hit it with a frequency \(t^{-1} = v_x / 2L\), and each time it does so it imparts upon it a momentum \(2 p_x\), since it is reflected backwards.

Summing over all the particles, the rate of momentum transfer (so, the force) in the direction \(x\) is given by 
%
\begin{align}
  \frac{N}{2L} \expval{2p_x v_x}
\,,
\end{align}
%
so the pressure upon that face will be the force divided by the area of the face
%
\begin{align}
P_x = \frac{N}{L} \expval{p_x v_x} \frac{1}{L^2} = \underbrace{\frac{N}{V}}_{= n} \expval{p_x v_x}
\,.
\end{align}

This will be the same for each direction by isotropicity: \(P_x = P_y = P_z\), and by the same argument we can write \(\expval{p_x v_x} = \expval{\vec{p} \cdot \vec{v}} / 3\): so 
%
\begin{align}
  P = \frac{n}{3} \expval{\vec{p} \cdot \vec{v}}
\,,
\end{align}
%
% in full generality. 
which, although we will not show it, generalizes to a configuration of any shape, and does not change if we consider quantum-mechanical or relativistic effects.
This is a simple expression for the \textbf{equipartition theorem}, a crucial result in Hamiltonian mechanics. 

Let us consider two limits: nonrelativistic and fully relativistic particles.

\paragraph{Nonrelativistic particles}

% In the nonrelativistic case we have 
% In this case the energy of a particle with momentum \(p \approx mv\) can be expressed as 
% %
% \begin{align}
%   \epsilon_{p} = mc^2 + \frac{p^2}{2m}
% \,,
% \end{align}
% %
% where \(p = mv\).
In this case, since \(\gamma \approx 1\) the four-momentum of the particles is approximately 
%
\begin{align}
p^{\mu } = \left[\begin{array}{c}
\gamma mc^2 \\ 
\gamma m \vec{v}
\end{array}\right]
\approx \left[\begin{array}{c}
mc^2 \\ 
m \vec{v}
\end{array}\right]
\,,
\end{align}
%
so \(\vec{p} = m \vec{v}\), which means \(\expval{\vec{p} \cdot \vec{v}}  = \expval{ m v^2}\). 

% In the ultrarelativistic case we have 
% %
% \begin{align}
%   \epsilon_{p} = pc
% \,,
% \end{align}
% %
% and the velocity is approximately the speed of light. 

Then, for a gas of nonrelativistic particles we can write the pressure as 
%
\begin{align}
  P = \frac{n}{3} \expval{ m v^2} = \frac{2}{3} \rho_{E_K}
\,,
\end{align}
%
where \(\rho_{E_K} = n m \expval{ v^2 /2}\) is the density of translational kinetic energy.

Combining this result with the fact that, as we have seen before, \(\expval{P}=- \rho _{\text{grav}} / 3\), we find 
%
\begin{align}
  - \frac{1}{3} \rho _{\text{grav}} = \frac{2}{3} \rho_{E_K} \implies
  2 E _{\text{K}} + E _{\text{grav}} = 0
\,,
\end{align}
%
% in the nonrelativistic approximation. 
which is an alternate statement of the \textbf{nonrelativistic} case of the \textbf{virial theorem}. 

The total energy is then given by \(E _{\text{tot}} = E_k + E _{\text{grav}} = - E_k\): this means that in general the system will be \textbf{bound} --- the kinetic energy is quadratic, so always positive --- and that the hotter it is, the more bound it is.

% We define: \(\Delta E _{\text{tot}} = - \Delta E _{\text{K}} = \frac{1}{2} \Delta E _{\text{grav}}\). 

% We know that 
% %
% \begin{align}
%   \expval{P} = \frac{1}{3} \frac{E _{\text{K}}}{V} = -\frac{1}{3} \frac{E _{\text{grav}}}{V}
% \,
% \end{align}
% %
% by the virial theorem: so the total binding energy is equal to zero, since this gives us 
% %
% \begin{align}
%   E _{\text{grav}} + E _{\text{K}} = E _{\text{tot}} = 0
% \,.
% \end{align}
% %

\paragraph{Relativistic case}

In this case \(v \approx c\), so \(\expval{p \cdot \vec{v}} \approx pc\). We can apply the reasoning from before, but the density of translational kinetic energy is given by 
%
\begin{align}
\rho_{E_K} = n (E - mc^2) = n (\gamma -1) mc^2 \approx \gamma mc^2 = npc 
\,,
\end{align}
%
so we have 
%
\begin{align}
P = \frac{n}{3} \expval{\vec{p} \cdot \vec{v}} = \frac{\rho _{E_k}}{3}
\,.
\end{align}

Then, we can apply the same reasoning as the nonrelativistic case, with the difference of the missing factor 2: we then get 
%
\begin{align}
\rho_{E_k} + \rho_{\text{grav}} = 0 \implies
E _{\text{grav}} + E_k = E _{\text{tot}} = 0
\,,
\end{align}
%
so the system is \textbf{unbound}, it does not have any constraint preventing it from dissociating. 

% while in the relativistic case we have: 
% %
% \begin{align}
%   P = \frac{1}{2} n \expval{pc} = \frac{1}{3} \times \text{translational KE density}
% \,.
% \end{align}

% We will show that, if a star is made of a gas of classical nonrelativistic particles it tends to be stable, if the particles are relativistic then it tends not to be stable.

% The virial theorem tells us that 

\paragraph{Adiabatic gas}

We have seen the limiting cases, now let us consider a slightly more general one: a gas undergoing an adiabatic transformation, such that \(P V^{\gamma }\) (with some real number \(\gamma \)) is constant.\footnote{A more realistic model would allow \(\gamma \) to vary, which it definitely does in the stages of stellar formation and evolution and even across a single transformation. We will not, however, get that deep in the weeds.}
We will show that this is equivalent to the equations of state considered in cosmology, where \(P = w \rho \). 
This will allow us to characterize the gravitational stability of the to-be star depending on the equation of state of the gas. 

We start by differentiating: \(\dd \qty(P V^{\gamma }) = 0\), which means that we also have \(\dd{ (\log (P V^{\gamma }))} = 0\), which we can expand into
%
\begin{align}
  \dd{\log (V^{\gamma })} + \dd{\log (P)} = 
  \gamma \frac{ \dd{V}}{V} + \frac{ \dd{P}}{P} = 0
\,,
\end{align}
%
so 
%
\begin{align}
  - (\gamma -1 ) P \dd{V}
  = P \dd{V} + V \dd{P} =
  \dd{(PV)} 
\,.
\end{align}
%
% and we know that for an adiabatic transformation 

% Isentropicity of the trasformation means
In an adiabatic transformation the entropy must not change: so, we can write
%
\begin{align}
  T \dd{S} = 
  \dd{E _{\text{in}}} + P \dd{V} = 0
\,,
\end{align}
%
which we can then write using the relation we derived previously:
%
\begin{align}
  \dd{E _{\text{in}}} &= \frac{1}{\gamma -1} \dd{(PV)} \\
  E _{\text{in}} &= \frac{PV}{\gamma -1}  \\
  P &= (\gamma - 1) \frac{E _{\text{in}}}{V} = (\gamma - 1) \rho _{\text{in}}
\,.
\end{align}
%
% and let us assume that \(\gamma \) is approximately constant in the transformation: this means 

We can then see that if we impose that the transformation be adiabatic, we find the equation of state \(P  =w \rho \), with \(\gamma -1 = w\). 

% %
% \begin{align}
%   E _{\text{in}} = \frac{PV}{\gamma -1}
% \,,
% \end{align}
% %
% so 
% %
% \begin{align}
%   P = (\gamma -1 ) \frac{E _{\text{in}}}{V}
% \,,
% \end{align}
% %
% which justifies the relations we used in cosmology, \(P = w \rho \) with \(w = \gamma -1\). 

% We can rewrite the equation from before as 
Using the fact that, as we have shown before, \(P = - \rho _{\text{grav}} / 3\), this means 
%
\begin{align}
- \frac{\rho _{\text{grav}}}{3} = (\gamma - 1) \rho _{\text{in}} \implies
  3(\gamma -1 ) E _{\text{in}} + E _{\text{gr}} = 0
\,,
\end{align}
%
which, together with the fact that the total energy of the star after the collapse is the initial energy plus the (negative) gravitational binding energy: \(E _{\text{tot}} = E _{\text{in}} + E _{\text{gr}}\), so 
%
\begin{align}
  E _{\text{tot}} = - (3 \gamma - 4) E _{\text{in}}
\,,
\end{align}
%
which means that \(\gamma > 4/3\) characterizes a bound system, while \(\gamma < 4/3\) characterizes a free system. 
This is consistent with what we have seen before: the limiting case \(\gamma = 4/3\) is equivalent to \(w = 1/3\), the equation of state of radiation (or ultrarelativistic matter), which as we have already seen is unbound.
% \(\gamma  = 1\) is equivalent to nonrelativistic matter with \(w = 0\), meaning no pressure at all: this matter will then collapse and 

From classical thermodynamics we know that, for instance, a monoatomic gas has \(\gamma = 5 /3\).

% There are two dangers: one is the fight against the pressure forces, one is the fight against the quantum forces (the Pauli exclusion principle) which do not allow the compression to happen further. 

\section{Jeans instability}

Now we discuss Jeans instability: 
%
\begin{align}
  E _{\text{grav}} = - f \frac{GM^2}{R}
\,,
\end{align}
%
where \(f\) is a numerical factor of the order \(1\), depending on the mass distribution. If the distribution is uniform, it is \(3/2\).
\todo[inline]{3/2?} 

The kinetic energy is 
%
\begin{align}
  E _{\text{K}} = \frac{3}{2} N k_B T
\,,
\end{align}
%
and we want to impose the condition 
%
\begin{align}
  f \frac{GM^2}{R} > \frac{3}{2} N k_B T 
\,,
\end{align}
%
and the Jeans criterion is this boundary of the stability region: 
%
\begin{align}
  f \frac{gM_J^2}{R} = \frac{3}{2} \frac{M_J}{\bar{m}} k_B T
\,,
\end{align}
%
where \(\bar{m} = M / N\). 
The \(J\) denotes the fact that we are considering the specific boundary mass on both sides. Simplifying the formula we find: 
%
\begin{align}
  M_J = \frac{3}{2} \frac{k_B T }{G \bar{m}} R
\,,
\end{align}
%
and we can reframe this in terms of the density, which is defined by 
%
\begin{align}
  M_J = \frac{4 \pi }{3} \rho _J R^3
\,.
\end{align}
%
We cube and multiply on both sides: 
%
\begin{align}
  \rho_J M_J^3 = \qty(\frac{3 k_B T}{2 G \bar{m}})^3R^3 \rho_J
\,,
\end{align}
%
so we get 
%
\begin{align}
  \rho_J = \frac{1}{M_J^3} \qty(\frac{3 k_B T}{G \bar{m}})^3 \qty( \frac{4 \pi }{3} \rho_J R^3) \frac{1}{4 \pi }
\,,
\end{align}
%
so 
%
\begin{align}
  \rho _J = \frac{3}{4 \pi M_J^2} \qty(\frac{3 k_B T}{2 G \bar{m}})^3
\,,
\end{align}
%
and we will habe stability if the density is larger than this. So, if we want a collapse, we must decrease the mass\dots

When the last scattering happens, the pions are decoupled from the photons. Dark matter behaves differently from conventional matter. 

We have 
%
\begin{align}
  \dot{\rho}_r = - 3H \qty(\rho _r + P_r)
\,,
\end{align}
%
and 
%
\begin{align}
  \dot{\rho}_m = -3H \qty(\rho _m + P_m )
\,,
\end{align}
%
and \(P_r = \rho_r / 3\), which scale like \(a^{-4} \) and also as \(T^{4}\), which means \(T \sim 1/a\).  
%
\begin{align}
  \dd \qty(\rho _m c^2 a^3) + P_m d a^3 = 0
\,,
\end{align}
%
where we usually approximate \(\rho _m c^2 = m_p n_b c^2\), but we can include more terms: 
%
\begin{align}
  \rho _m c^2 = m_p n_b c^2 \qty(1 + (\gamma -1 )^{-1} \frac{k_B T}{m_p c^2})
\,,
\end{align}
%
while the pressure is given by \(P = n_b k_B T\): so in the end we find 
%
\begin{align}
  \dd \qty(\qty(m_p n_b c^2 + \frac{3}{2} m_p n_b \frac{k_BT}{m_p})a^3) = - n_b k_B T \dd{a^3}
\,,
\end{align}
%
which after some computation gives us 
%
\begin{align}
  \frac{1}{2} \dd{T } = - T \frac{ \dd{a}}{a}
\,,
\end{align}
%
which implies \(T_m \propto a^{-2}\) after baryogenesis. 

\todo[inline]{This is for monoatomic baryonic matter, right?}

Let us start writing equations for the stellar interior. The continuity equation is 
%
\begin{align}
  \partial_{t} \rho + \nabla \cdot \qty(\rho \vec{v}) = 0
\,,
\end{align}
%
and also we have the Euler equation 
%
\begin{align}
  \partial_{t} \vec{v} + \qty(\vec{v} \cdot \vec{\nabla}) \vec{v}
  = - \frac{1}{\rho } \vec{\nabla} P - \vec{\nabla} \Phi 
\,,
\end{align}
%
which can be written using the convective time derivative: 
%
\begin{align}
  \frac{ \mathrm{D} }{\mathrm{D}t} = \partial_{t} + \vec{v} \cdot \nabla_{x} = u^{\mu } \partial_{\mu }
\,,
\end{align}
%
which allows us to write 
%
\begin{align}
  \frac{ \mathrm{D} }{\mathrm{D}t} \rho + \rho \nabla \cdot \vec{v} = 0
\,,
\end{align}
%
and 
%
\begin{align}
  \frac{ \mathrm{D} }{\mathrm{D}t} \vec{v} 
  = - \frac{1}{\rho } - \nabla \Phi 
\,.
\end{align}

What happens to the entropy? we define the entropy density \(s \) by \(S  = s \rho \). 

An isentropic process is one in which 
%
\begin{align}
  \frac{ \mathrm{D} S}{\mathrm{D}t} = 0
\,,
\end{align}
%
which in terms of the entropy density is 
%
\begin{align}
  \partial_{t} s + \vec{v} \cdot \vec{\nabla} s = 0 
\,.
\end{align}
%

The external force is provided by the gravitational field: 
%
\begin{align}
  \nabla^2 \Phi = 4 \pi G \rho 
\,.
\end{align}

Jeans looked for a simple solution, an ansatz, called the background solution and then tried to perturb it: if it is stable than it was a good solution. 

He started with \(\rho = \const\), \(\vec{v} = 0\), \(s = \const\), \(\Phi = \const\). 

It is obviously wrong! It cannot satisfy the Poisson equation. 

However, we start from it and add some \(\delta \rho \), \(\delta \vec{v}\) (which we just call \(\vec{v}\)), \(\delta s\) and \(\delta \Phi \); then we only keep the linear terms in these perturbations. 

We find: 
%
\begin{align}
  \partial_{t} \delta \rho  +
  \rho_0 \vec{\nabla} \cdot \vec{v} = 0 
\,,
\end{align}
%
%
\begin{align}
  \partial_{t} \vec{v} = - \frac{1}{\rho_0 } \nabla \delta P - \nabla \delta \Phi 
\,,
\end{align}
%
and 
%
\begin{align}
  \nabla^2 \delta \Phi = 4 \pi G \delta \rho 
\,,
\end{align}
%
and finally 
%
\begin{align}
  \partial_{t} \delta s = 0
\,.
\end{align}

We can expand
%
\begin{align}
  \delta P = \pdv{P}{\rho } \delta \rho + \pdv{P}{s} \delta s
\,,
\end{align}
%
and here we define 
%
\begin{align}
  c_s^2 = \pdv{P}{\rho }
\,. 
\end{align}

We will then consider an exponential solution: 
%
\begin{align}
  \delta \rho = \delta \rho_0 \exp(i \qty(\vec{k} \cdot \vec{x} - \omega r))
\,,
\end{align}
%
and similarly for \(\vec{v}\), \(s\), \(\Phi \). 

We will see that we will need to stick to \(\delta s =0\), and find a dispersion relation with \(\omega \) and \(\vec{k}\): it will be 
%
\begin{align}
  \omega^2 = c_s^2 \vec{k}^2 - 4 \pi G \rho 
\,,
\end{align}
%
so if the wavenumber is small enough we will have an imaginary \(\omega \). 

\end{document}
