\documentclass[main.tex]{subfiles}
\begin{document}

\marginpar{Thursday\\ 2019-11-28, \\ compiled \\ \today}
% \section*{Thu Nov 28 2019}

% We derived 
% %
% \begin{align}
%   \expval{T} = -\frac{1}{3} \frac{E _{\text{grav}}}{V}
% \,,
% \end{align}
% %
% so now we proceed: we want a relation between kinetic and gravitational energy densities. 
Now, the question we want to as is: is this equilibrium configuration \textbf{stable}? This is equivalent to asking whether the system is gravitationally bound, \(E _{\text{grav}} < 0\), which as we have shown is equivalent to \(\expval{P}> 0\).  

In order to answer this question we shall use a statistical-mechanics, microscopic approach. 

We consider a cubic box of volume \(V = L^3\) with \(N\) particles inside it, each of which has a velocity \(\vec{v} = (v_x, v_y, v_z)^{\top}\) and a momentum \(p\). Let us select a face of the box, which we assume to be perpendicular to the \(x\) axis. Each particle will hit it with a frequency \(t^{-1} = v_x / 2L\), and each time it does so it imparts upon it a momentum \(2 p_x\), since it is reflected backwards.

Summing over all the particles, the rate of momentum transfer (so, the force) in the direction \(x\) is given by 
%
\begin{align}
  \frac{N}{2L} \expval{2p_x v_x}
\,,
\end{align}
%
so the pressure upon that face will be the force divided by the area of the face
%
\begin{align}
P_x = \frac{N}{L} \expval{p_x v_x} \frac{1}{L^2} = \underbrace{\frac{N}{V}}_{= n} \expval{p_x v_x}
\,.
\end{align}

This will be the same for each direction by isotropicity: \(P_x = P_y = P_z\), and by the same argument we can write \(\expval{p_x v_x} = \expval{\vec{p} \cdot \vec{v}} / 3\): so 
%
\begin{align}
  P = \frac{n}{3} \expval{\vec{p} \cdot \vec{v}}
\,,
\end{align}
%
% in full generality. 
which, although we will not show it, generalizes to a configuration of any shape, and does not change if we consider quantum-mechanical or relativistic effects.
This is a simple expression for the \textbf{equipartition theorem}, a crucial result in Hamiltonian mechanics. 

Let us consider two limits: nonrelativistic and fully relativistic particles.

\paragraph{Nonrelativistic particles}

% In the nonrelativistic case we have 
% In this case the energy of a particle with momentum \(p \approx mv\) can be expressed as 
% %
% \begin{align}
%   \epsilon_{p} = mc^2 + \frac{p^2}{2m}
% \,,
% \end{align}
% %
% where \(p = mv\).
In this case, since \(\gamma \approx 1\) the four-momentum of the particles is approximately 
%
\begin{align}
p^{\mu } = \left[\begin{array}{c}
\gamma mc^2 \\ 
\gamma m \vec{v}
\end{array}\right]
\approx \left[\begin{array}{c}
mc^2 \\ 
m \vec{v}
\end{array}\right]
\,,
\end{align}
%
so \(\vec{p} = m \vec{v}\), which means \(\expval{\vec{p} \cdot \vec{v}}  = \expval{ m v^2}\). 

% In the ultrarelativistic case we have 
% %
% \begin{align}
%   \epsilon_{p} = pc
% \,,
% \end{align}
% %
% and the velocity is approximately the speed of light. 

Then, for a gas of nonrelativistic particles we can write the pressure as 
%
\begin{align}
  P = \frac{n}{3} \expval{ m v^2} = \frac{2}{3} \rho_{E_K}
\,,
\end{align}
%
where \(\rho_{E_K} = n m \expval{ v^2 /2}\) is the density of translational kinetic energy.

Combining this result with the fact that, as we have seen before, \(\expval{P}=- \rho _{\text{grav}} / 3\), we find 
%
\begin{align}
  - \frac{1}{3} \rho _{\text{grav}} = \frac{2}{3} \rho_{E_K} \implies
  2 E _{\text{K}} + E _{\text{grav}} = 0
\,,
\end{align}
%
% in the nonrelativistic approximation. 
which is an alternate statement of the \textbf{nonrelativistic} case of the \textbf{virial theorem}. 

The total energy is then given by \(E _{\text{tot}} = E_k + E _{\text{grav}} = - E_k\): this means that in general the system will be \textbf{bound} --- the kinetic energy is quadratic, so always positive --- and that the hotter it is, the more bound it is.

% We define: \(\Delta E _{\text{tot}} = - \Delta E _{\text{K}} = \frac{1}{2} \Delta E _{\text{grav}}\). 

% We know that 
% %
% \begin{align}
%   \expval{P} = \frac{1}{3} \frac{E _{\text{K}}}{V} = -\frac{1}{3} \frac{E _{\text{grav}}}{V}
% \,
% \end{align}
% %
% by the virial theorem: so the total binding energy is equal to zero, since this gives us 
% %
% \begin{align}
%   E _{\text{grav}} + E _{\text{K}} = E _{\text{tot}} = 0
% \,.
% \end{align}
% %

\paragraph{Relativistic case}

In this case \(v \approx c\), so \(\expval{p \cdot \vec{v}} \approx pc\). We can apply the reasoning from before, but the density of translational kinetic energy is given by 
%
\begin{align}
\rho_{E_K} = n (E - mc^2) = n (\gamma -1) mc^2 \approx \gamma mc^2 = npc 
\,,
\end{align}
%
so we have 
%
\begin{align}
P = \frac{n}{3} \expval{\vec{p} \cdot \vec{v}} = \frac{\rho _{E_k}}{3}
\,.
\end{align}

Then, we can apply the same reasoning as the nonrelativistic case, with the difference of the missing factor 2: we then get 
%
\begin{align}
\rho_{E_k} + \rho_{\text{grav}} = 0 \implies
E _{\text{grav}} + E_k = E _{\text{tot}} = 0
\,,
\end{align}
%
so the system is \textbf{unbound}, it does not have any constraint preventing it from dissociating. 

% while in the relativistic case we have: 
% %
% \begin{align}
%   P = \frac{1}{2} n \expval{pc} = \frac{1}{3} \times \text{translational KE density}
% \,.
% \end{align}

% We will show that, if a star is made of a gas of classical nonrelativistic particles it tends to be stable, if the particles are relativistic then it tends not to be stable.

% The virial theorem tells us that 

\paragraph{Adiabatic gas}

We have seen the limiting cases, now let us consider a slightly more general one: a gas undergoing an adiabatic transformation, such that \(P V^{\gamma }\) (with some real number \(\gamma \)) is constant.\footnote{A more realistic model would allow \(\gamma \) to vary, which it definitely does in the stages of stellar formation and evolution and even across a single transformation. We will not, however, get that deep in the weeds.}
We will show that this is equivalent to the equations of state considered in cosmology, where \(P = w \rho \). 
This will allow us to characterize the gravitational stability of the to-be star depending on the equation of state of the gas. 

We start by differentiating: \(\dd \qty(P V^{\gamma }) = 0\), which means that we also have \(\dd{ (\log (P V^{\gamma }))} = 0\), which we can expand into
%
\begin{align}
  \dd{\log (V^{\gamma })} + \dd{\log (P)} = 
  \gamma \frac{ \dd{V}}{V} + \frac{ \dd{P}}{P} = 0
\,,
\end{align}
%
so 
%
\begin{align}
  - (\gamma -1 ) P \dd{V}
  = P \dd{V} + V \dd{P} =
  \dd{(PV)} 
\,.
\end{align}
%
% and we know that for an adiabatic transformation 

% Isentropicity of the trasformation means
In an adiabatic transformation the entropy must not change: so, we can write
%
\begin{align}
  T \dd{S} = 
  \dd{E _{\text{in}}} + P \dd{V} = 0
\,,
\end{align}
%
which we can then write using the relation we derived previously:
%
\begin{align}
  \dd{E _{\text{in}}} &= \frac{1}{\gamma -1} \dd{(PV)} \\
  E _{\text{in}} &= \frac{PV}{\gamma -1}  \\
  P &= (\gamma - 1) \frac{E _{\text{in}}}{V} = (\gamma - 1) \rho _{\text{in}}
\,.
\end{align}
%
% and let us assume that \(\gamma \) is approximately constant in the transformation: this means 

We can then see that if we impose that the transformation be adiabatic, we find the equation of state \(P  =w \rho \), with \(\gamma -1 = w\). 

% %
% \begin{align}
%   E _{\text{in}} = \frac{PV}{\gamma -1}
% \,,
% \end{align}
% %
% so 
% %
% \begin{align}
%   P = (\gamma -1 ) \frac{E _{\text{in}}}{V}
% \,,
% \end{align}
% %
% which justifies the relations we used in cosmology, \(P = w \rho \) with \(w = \gamma -1\). 

% We can rewrite the equation from before as 
Using the fact that, as we have shown before, \(P = - \rho _{\text{grav}} / 3\), this means 
%
\begin{align}
- \frac{\rho _{\text{grav}}}{3} = (\gamma - 1) \rho _{\text{in}} \implies
  3(\gamma -1 ) E _{\text{in}} + E _{\text{gr}} = 0
\,,
\end{align}
%
which, together with the fact that the total energy of the star after the collapse is the initial energy plus the (negative) gravitational binding energy: \(E _{\text{tot}} = E _{\text{in}} + E _{\text{gr}}\), so 
%
\begin{align}
  E _{\text{tot}} = - (3 \gamma - 4) E _{\text{in}}
\,,
\end{align}
%
which means that \(\gamma > 4/3\) characterizes a bound system, while \(\gamma < 4/3\) characterizes a free system. 
This is consistent with what we have seen before: the limiting case \(\gamma = 4/3\) is equivalent to \(w = 1/3\), the equation of state of radiation (or ultrarelativistic matter), which as we have already seen is unbound.
% \(\gamma  = 1\) is equivalent to nonrelativistic matter with \(w = 0\), meaning no pressure at all: this matter will then collapse and 

From classical thermodynamics we know that, for instance, a monoatomic gas has \(\gamma = 5 /3\).

% There are two dangers: one is the fight against the pressure forces, one is the fight against the quantum forces (the Pauli exclusion principle) which do not allow the compression to happen further. 

\section{Jeans instability}

% 
Let us now try to understand the conditions under which a cloud of gas may become unstable and collapse onto itself to form a star (or a planet, for that matter). 
% Now we discuss Jeans instability: 

In general, the gravitational potential energy of a body whose characteristic size is \(R\) and whose mass is \(M\) is given by
%
\begin{align}
  E _{\text{grav}} 
  = - \int_{x, y \in V} \dd[3]{x} \dd[3]{y} \rho (x) \rho (y) \frac{G}{\abs{x - y}}  
  = - f \frac{GM^2}{R}
\,,
\end{align}
%
where \(f\) is a numerical factor depending on the mass distribution. 
If the object at hand is uniform-density sphere, we have \(f = 3/5\).
In general, the factor is of order 1. 

The kinetic component of the energy, on the other hand, is 
%
\begin{align}
  E _{\text{K}} = \frac{3}{2} N k_B T
\,.
\end{align}

% We can then see that the gravitational energy scales with \(R^{-1}\); 

The gravitational cloud is unstable the gravitational energy is larger than the kinetic energy:
\todo[inline]{Why should this be? The way \textcite[]{keetonStarPlanetFormation2014} discusses it makes more sense to me: he studies the response of the total energy to a decrease in radius, and checks that it is positive; it is not the same as what we are doing here!}
%
\begin{align}
  f \frac{GM^2}{R} > \frac{3}{2} N k_B T 
\,,
\end{align}
%
and the Jeans mass, \(M_J\), corresponds to the boundary of the stability region: the number of particles, \(N\), depends on it as \(N = M_J / \overline{m}\), where \(\overline{m}\) is the average particle mass.

The criterion then reads:
%
\begin{align}
  f \frac{gM_J^2}{R} &= \frac{3}{2} \frac{M_J}{\bar{m}} k_B T \\
  M_J &= \frac{3}{2} \frac{k_B T }{G \bar{m}} R
\,,
\end{align}
%
where we set \(f =1\), since we are only interested in an order-of-magnitude calculation.
% where \(\bar{m} = M / N\). 
% The \(J\) denotes the fact that we are considering the specific boundary mass on both sides. Simplifying the formula we find: 
%
% and we can reframe this in terms of the density, which is defined by 

As usual, we want to reframe our result in terms of densities: the Jeans mass corresponds to a Jeans density times the volume of the sphere:
%
\begin{align}
  M_J = \frac{4 \pi }{3} \rho _J R^3
\,.
\end{align}

In order to find out what this density is we start off by cubing the  
expression for the Jeans mass, and then substituting the expression for \(M_J\) in terms of \(\rho _J\):
% and multiply on both sides: 
%
\begin{align}
M_J^3 &= \qty(\frac{3 k_B T}{2 G \bar{m}})^3 R^3\\
&=  \qty(\frac{3 k_B T}{2 G \bar{m}})^3 \frac{3M_J}{4 \pi \rho _J} \\  
  \rho _J &= \frac{3}{4 \pi M_J^2} \qty(\frac{3 k_B T}{2 G \bar{m}})^3
\,.
\end{align}

Alternatively, we can write 
%
\begin{align}
\frac{4 \pi }{3} \rho _J R^3 &= \frac{3}{2} \frac{k_B T}{G \overline{m}} R \\
\rho _J &= \frac{9}{8 \pi } \frac{1}{R^2} \frac{k_B T}{G \overline{m}}
\,. \label{eq:jeans-density}
\end{align}
%


We will have an instability if the density is larger than this.
As we have seen in the previous section, a lower temperature facilitates the collapse. 
It should be stressed that the precise numerical coefficient will depend on the geometry of the cloud of material, this is not a hard rule but more of a guide for the understanding of the behavior of clouds.
% So, if we want a collapse, we must decrease the mass\dots

% When the last scattering happens, the pions are decoupled from the photons. Dark matter behaves differently from conventional matter. 

\todo[inline]{Here appears in the lecture the argument for the fact that the temperature of matter decreases as \(T \sim a^{-2}\); it does not really seem to fit with the rest of the chapter, perhaps it should go earlier?

I'll leave it here, commented out.}

% We have 
% %
% \begin{align}
%   \dot{\rho}_r = - 3H \qty(\rho _r + P_r)
% \,,
% \end{align}
% %
% and 
% %
% \begin{align}
%   \dot{\rho}_m = -3H \qty(\rho _m + P_m )
% \,,
% \end{align}
% %
% and \(P_r = \rho_r / 3\), which scale like \(a^{-4} \) and also as \(T^{4}\), which means \(T \sim 1/a\).  
% %
% \begin{align}
%   \dd \qty(\rho _m c^2 a^3) + P_m d a^3 = 0
% \,,
% \end{align}
% %
% where we usually approximate \(\rho _m c^2 = m_p n_b c^2\), but we can include more terms: 
% %
% \begin{align}
%   \rho _m c^2 = m_p n_b c^2 \qty(1 + (\gamma -1 )^{-1} \frac{k_B T}{m_p c^2})
% \,,
% \end{align}
% %
% while the pressure is given by \(P = n_b k_B T\): so in the end we find 
% %
% \begin{align}
%   \dd \qty(\qty(m_p n_b c^2 + \frac{3}{2} m_p n_b \frac{k_BT}{m_p})a^3) = - n_b k_B T \dd{a^3}
% \,,
% \end{align}
% %
% which after some computation gives us 
% %
% \begin{align}
%   \frac{1}{2} \dd{T } = - T \frac{ \dd{a}}{a}
% \,,
% \end{align}
% %
% which implies \(T_m \propto a^{-2}\) after baryogenesis. 

% \todo[inline]{This is for monoatomic baryonic matter, right?}

\subsubsection{Equations for stellar structure}

In order to properly study the dynamics of the stellar collapse, however, we need to analyze the differential equations which govern it. 
We will start out by doing so on a static background, following the original reasoning by Jeans (who, working in the early 1900s, did not know about the expansion of the universe). Then, we will discuss the effects of the universe's expansion on the gravitational instability. 
% Let us start writing equations for the stellar interior.

The \textbf{continuity equation}, imposed by mass conservation, is
%
\begin{align}
  \partial_{t} \rho + \nabla \cdot \qty(\rho \vec{v}) = 0
\,,
\end{align}
%
where \(\rho \) is the matter density while \(\vec{v}\) is the velocity field; the \textbf{Euler equation}, imposed by momentum conservation (assuming no viscosity), is 
%
\begin{align}
  \partial_{t} \vec{v} + \qty(\vec{v} \cdot \vec{\nabla}) \vec{v}
  = - \frac{1}{\rho } \vec{\nabla} P - \vec{\nabla} \Phi 
\,,
\end{align}
%
where \(P\) is the pressure while \(\Phi \) is the gravitational potential.

If we define the \textbf{convective} time \textbf{derivative},  
%
\begin{align}
  \frac{ \mathrm{D} }{\mathrm{D}t} = \partial_{t} + \vec{v} \cdot \nabla_{x} = u^{\mu } \partial_{\mu }
\,,
\end{align}
%
we can write the two equations as 
%
\begin{align}
  \frac{ \mathrm{D} }{\mathrm{D}t} \rho + \rho \nabla \cdot \vec{v} &= 0 \\
  \frac{ \mathrm{D} }{\mathrm{D}t} \vec{v} 
  &= - \frac{\nabla P}{\rho } - \nabla \Phi 
\,.
\end{align}

Lastly, the gravitational field \(\Phi \) must obey Poisson's equation:
%
\begin{align}
  \nabla^2 \Phi = 4 \pi G \rho 
\,.
\end{align}

Right now we have five equations (Euler is a vector equation, corresponding to three scalar ones) and six variables: \(\rho \), \(\Phi \), \(P\) and the three components of \(\vec{v}\). 
In order to be able to solve this system we need one more condition; typically this is provided as an equation of state, giving \(P\) in terms of the other variables. 

One way to go about this is to consider entropy: we define the entropy density \(s \) by the relation \(S  = s \rho \), where \(S\) is the (total?) entropy. 

We will consider isentropic processes, in which 
%
\begin{align}
  \frac{ \mathrm{D} s}{\mathrm{D}t} + s \vec{\nabla} \cdot \vec{v} = 0
  % \partial_{t} s + \vec{v} \cdot \vec{\nabla} s = 0
\,.
\end{align}

We introduce this, an additional eqution as well as an additional variable, in order to complete our equations with an equation of state in the form: 
%
\begin{align}
P = P(\rho , s)
\,.
\end{align}

Now, then, we are left with seven equations and seven variables: let us solve them!
This is in general very hard, no analytic solutions exist.

% Jeans looked for a simple solution, an ansatz, called the background solution and then tried to perturb it: if it is stable than it was a good solution. 
Jeans' approach, which we will follow, is to find a fixed background solution and then to perturb it. 
We are then looking to see whether the perturbation is dampened or amplified. Perturbations are always present, so this will tell us whether the configuration is stable or unstable.

\paragraph{Static ansatz}

Jeans' first ansatz was \(\rho = \rho_0 = \const\), \(\vec{v} = 0\), \(s = s_0 =  \const\), \(\Phi = \Phi_0 = \const\), \(P =P_0 = \const\). 

% It is obviously wrong! It cannot satisfy the Poisson equation. 
This is a \emph{very} simplified model, and it is not even self-consistent: unless \(\rho  = 0\), Poisson's equation cannot be satisfied, but we want to have matter in our proto-star.
We will ignore this problem, since despite it we get a physically meaningful result.
The equation cannot precisely hold, but in low-density regions it is not that far from equality.

% However, we start from it and add some \(\delta \rho \), \(\delta \vec{v}\) (which we just call \(\vec{v}\)), \(\delta s\) and \(\delta \Phi \); then we only keep the linear terms in these perturbations. 
We perturb the variables: for each variable we will have \(x = x_0 + \delta x\) (except \(\vec{v}\): since there is no \(\vec{v}_0 \), we just write \(\vec{v}\) instead of \(\delta \vec{v}\)). 

With this, the equations read:
%
\begin{align}
  \partial_{t} \delta \rho  +
  \rho_0 \vec{\nabla} \cdot \vec{v} &= 0 \\
  \partial_{t} \vec{v} &= - \frac{1}{\rho_0 } \vec{\nabla} \delta P - \vec{\nabla} \delta \Phi \\
  \nabla^2 \delta \Phi &= 4 \pi G \delta \rho  \\
  \partial_{t} \delta s &= 0
\,.
\end{align}

% We can expand
The pressure perturbation can be expressed in terms of the density and entropy ones:
%
\begin{align}
  \delta P = \underbrace{\eval{\pdv{P}{\rho }}_{s}}_{= c_s^2} \delta \rho + \pdv{P}{s} \delta s
\,,
\end{align}
%
where we recognize the constant-entropy derivative of the pressure with respect to the density: the square of the adiabatic speed of sound.

% In our study of these equations we will only consider first-order terms in the perturbations. 

% We will then consider an exponential solution: 
% %
% \begin{align}
%   \delta \rho = \delta \rho_0 \exp(i \qty(\vec{k} \cdot \vec{x} - \omega r))
% \,,
% \end{align}
% %
% and similarly for \(\vec{v}\), \(s\), \(\Phi \). 

% We will see that we will need to stick to \(\delta s =0\), and find a dispersion relation with \(\omega \) and \(\vec{k}\): it will be 
% %
% \begin{align}
%   \omega^2 = c_s^2 \vec{k}^2 - 4 \pi G \rho 
% \,,
% \end{align}
% %
% so if the wavenumber is small enough we will have an imaginary \(\omega \). 

\end{document}
