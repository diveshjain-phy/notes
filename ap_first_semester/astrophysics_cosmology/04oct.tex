\documentclass[main.tex]{subfiles}
\begin{document}

\section*{4 October 2019}

From yesterday we recall that \(k=0\) iff \(\rho (t) = \rho_C(t)\).

We can write \(H (t_0) = H_0 = 100 h \times \SI{}{\kilo\metre\per\second\per\mega\parsec} \).
Do note that \(\SI{1}{\mega\parsec} = \SI{3.086e22}{\metre}\).

In the American school, the pupils of Hubble thought \(h \sim 0.5\), while the French school thought \(h \sim 1\).
Now, we know that \(h \sim 0.7\). Some people find \(h\approx 0.67 \divisionsymbol 68\), others find \(h \approx 0.62\).

If we have \(H\) we can find \(\rho_{0C} = h^2 \times \SI{1.88e-28}{\gram\per\cubic\metre}\). We have defined \(\Omega(t) \defeq \rho / \rho_C\): recall that \(\sign{\Omega -1} = \sign k\).

So we want to measure the energy density in galaxies to figure out what \(\Omega\) is.

There is a professor called Schechter who introduced a \emph{universal} luminosity function of the universe.

\begin{equation}
  \Phi(L) = \frac{\Phi^*}{L^*} \qty(\frac{L}{L^*})^{-\alpha} \exp(-\frac{L}{L^*})
\end{equation}

These can be fit by observation: we find \(\Phi^* \approx \SI{e-2}{} h^3 \SI{}{\mega\parsec^{-3}}\), \(L^* \approx \num{e10}h^{-2} L_\odot \) and \(\alpha \approx 1\).

[Plot of this function: sharp drop-off at \(L=L^*\) ]

The integral for \(\mathscr L_g\) converges despite the divergence of \(\Phi(L)\) as \(L \rightarrow 0\): so we do not need to really worry about the low-luminosity cutoff.

The result of the integral is \(\mathscr L_g = \Phi^* L^* \Gamma(2-\alpha)\) where \(\Gamma\) is the Euler gamma function, and \(\Gamma(2-1) = 1\).
Numerically, we get \(\num{2e18} h L_\odot \SI{}{\per\cubic\mega\parsec}. \)

Spiral galaxies are characterized by rotation.

We plot the velocity of rotation of galaxies \(v\)  against the radius \(R\). This is measured using the Doppler effect.

We'd expect a roughly linear region, and then a region with \(v \sim R^{-1/2}\):
we apply \(G M(R) = v^2 (R) R\) (this comes from Kepler's laws or from the virial theorem). This implies

\begin{equation}
  v(R) \propto \sqrt{\frac{M(R)}{R}}
\end{equation}

So in the inside of the galaxy, where \(M(R) \propto R^3\), \(v \propto R\), while outside of it \(M(R)\) is roughly constant, so \(v \propto R^{-1/2}\).

Instead of this, we see the linear region and then \(v(R)\) is approximately constant. Is Newtonian gravity wrong? (GR effects are trivial at these scales).

An option is MOND: they propose that there is somthing like a Yukawa term at Megaparsec distances. They are wrong for some other reasons.

Another option is that what we thought was the galaxy, from our EM observations, is actually smaller than the real galaxy. We'd need mass obeying \(M (R) \sim R\): since \(M(R) = 4 \pi \int_0^{R_{\text{max}}} \dd{R} R^2 \rho(R)\), we need \(\rho(R) \propto R^{-2}\). This is a \emph{thermal} distribution (?): we call it the \emph{dark matter halo}.

People tend to believe that this matter is made up of beyond-the-standard-model particles, like a \emph{neutralino}.
An alternative is the \emph{axion}.

The total density of DM is \(\sim 5\) times more than that of regular matter.

If galaxies are not spiral, we look at other things: the Doppler broadening of spectral lines gives us a measure of the RMS velocity.

We will obtain the (nonrelativistic) virial theorem:

\begin{equation}
  2T + U = 0
\end{equation}


\end{document}
