\documentclass[main.tex]{subfiles}
\begin{document}

\section*{Thu Jan 09 2020}

Many of these topics will be discussed in more details in the course by Nicola Bartolo on cosmological perturbations and in the course by Giacomo Ciani on gravitational waves. 

Today we will focus on the LISA detector, which works for frequencies \(f \sim \num{e-5} \divisionsymbol \SI{e-1}{Hz}\), and on the cosmological stochastic background of GWs (CSGWB). 

We can make an analogy with the CMB: we expect to see a background of gravitational radiation coming from all directions. 

The stochastic background can have different origins: 
\begin{enumerate}
  \item astrophysical origin: a coherent superposition of a large number of astrophysical sources, which are too weak to be detected separately;
  \item cosmological origin: it is generated in the early universe by a variety of mechanisms: 
  \begin{enumerate}
    \item amplification of primordial tensor fluctuations via inflation; 
    \item GWs from phase transition around the \SI{}{TeV} scale (this is not the only energy scale at which they can be emitted, but it is the energy we'd need in order to detect them with LISA);
    \item GWs from topological defects.
  \end{enumerate}
\end{enumerate}

A stochastic background of cosmological origin is expected to be 
\begin{enumerate}
  \item isotropic;
  \item stationary;
  \item unpolarized, which means that the cross and plus polarizations will have the same amplitude. 
\end{enumerate}

Let us discuss the main properties of the frequency spectrum: it is characterized 
\begin{enumerate}
  \item in terms of (normalized) energy density per unit logarithmic interval of frequency: this is called \(h_0^2 \Omega_{GW} (f)\);
  \item in terms of the spectral density of the ensemble average of the Fourier component of the metric \(S_{h}(f)\);
  \item more on the experimental side: in terms of a characteristic amplitude of the stochastic background \(h_{c} (f)\).
\end{enumerate}

Let us define these three quantities. The energy density is given by 
%
\begin{align}
\Omega_{GW} = \frac{1}{\rho_{c}} \dv{\rho_{GW} }{\log f}
\,,
\end{align}
%
where \(\rho_{GW}\) is the energy density of SGWB, \(f\) is the frequency, while \(\rho_{c}\) is the present value of the critical energy density, defined as 
%
\begin{align}
\rho_{c} = \frac{3 H_0^2}{8 \pi G }
\,,
\end{align}
%
where we usually write \(H_0 = h_0 \times \SI{100}{km / (s Mpc)}\). 

So, usually we plot \(h_0^2 \Omega_{GW}\) in order to ignore the uncertainties on the measurements of \(H_0 \). 

We write stochastic GW at a given point \(\vec{x} = 0 \) in the transverse traceless (TT) gauge: \(h_{ii} = 0\) and \(\partial_{i} h^{ij}\): we get 
%
\begin{align}
h_{ab}(t) = \sum _{A = +, \times } \int_{\mathbb{R}} \dd{f} \int_{S^{2}} \dd{\Omega } \hat{h}_A (f, \Omega ) \exp( - 2 \pi i f t ) e^{A}_{ab} (\hat{\Omega})
\,,
\end{align}
%
where we must have the condition \(\hat{h}_A (f, \Omega ) = \hat{h}^{*}_{A} (-f, \Omega )\). Here \(\hat{\Omega}\) is a unit vector representing the direction of propagation of the wave, and \(\dd{\hat{\Omega}} = \dd{\cos(\theta )} \dd{\phi }\), \(e^{A}_{ab}\) are the polarization tensors: 
\begin{enumerate}
  \item \(e^{+}_{ab} (\hat{\Omega}) = 2\hat{m}_{[a} \hat{m}_{b]}\) ;
  \item \(e^{ \times }_{ab} (\hat{\Omega}) = 2\hat{m}_{(a} \hat{m}_{b)}\);
\end{enumerate}
where \(\hat{m}_{a, b}\) are unit vectors orthogonal to each other and to the propagation direction.

We have the condition \(e^{A}_{ab} e^{B ab} = 2 \delta^{AB}\). 

Then, assuming a SGWB which is isotropic, unpolarized and stationary we will have 
%
\begin{align}
\expval{\hat{h}^{*}_{A} (f, \hat{\Omega}) \hat{h}_{A'} (f',  \hat{\Omega }' )} = \delta (f- f') \frac{1}{4 \pi } \delta^{(2)}( \hat{\Omega}, \hat{\Omega }' ) \delta_{A A'} \frac{1}{2} S_{h}(f)
\,,
\end{align}
%
where 
%
\begin{align}
\delta^{(2)} (\hat{\Omega}, \hat{\Omega}')= \delta(\phi - \phi') \delta(\cos(\theta ) - \cos(\theta' ))
\,.
\end{align}

The factors of \(1 / 4 \pi \) and \(1 / 2\) are for convention, for normalization, so that \(\int S_{h} (f) \dd{f} = 1\). 

Then the remaining bit, \(S_{h}(f)\), is called the spectral density. 

So, using the equaitions we found so far, we get 
%
\begin{subequations}
\begin{align}
\expval{h_{ab}(t) h^{ab}(t)} &= 2 \int_{\mathbb{R}} \dd{f} S_h (f)  \\
&= 4 \int_{f=0}^{f= \infty } \dd{\log f} f S_h (f)
\,.
\end{align}
\end{subequations}

We define the characteristic amplitude \(h_{c}(f)\) as 
%
\begin{align}
\expval{h_{ab} (t) h^{ab } (t)} = 2 \int_{f=0}^{f = \infty} \dd{\log f} h_c^2 (f)
\,,
\end{align}
%
which means that \(h_c^2(f) = 2 f S_h(f)\). 

The last step is to relate \(h_c(f)\) and \(h_0^2 \Omega_{GW}(f)\). 

The energy density is defined as:
%
\begin{align}
\rho_{GW} =  \frac{1}{32 \pi G} \expval{\dot{h}_{ab} \dot{h}^{ab} }
\,,
\end{align}
%
where the average is performed over a wavelength, but by the ergodic theorem it can also be performed over a period.
So, 
%
\begin{align}
\rho_{GW} = \frac{4}{32 \pi G} \int_{f=0}^{f= \infty } \dd{(\log f)} f (2 \pi f)^2 S_h (f)
\,,
\end{align}
%
whicm means that 
%
\begin{subequations}
\begin{align}
\dv{\rho_{GW}}{\log f} &= \frac{\pi }{4 G} f^2 h_c^2(f)  \\
&= \frac{\pi}{2G} f^3 S_h (f)
\,,
\end{align}
\end{subequations}
%
so in the end we have 
%
\begin{align}
\Omega_{GW} (f) = \frac{2 \pi^2}{3 H_0^2} f^2 h_c^2 (f)
\,,
\end{align}
%
therefore 
%
\begin{align}
\Omega_{GW}(f) = \frac{4 \pi^2}{3 H_0^2} f^3 S_h (f)
\,.
\end{align}

The \emph{strain} is the quantity \(S_h\). 

An advantage of GWs is the fact that they decouple right after emission: they maintain the spectral shape. 

If we start from the Einstein-Hilbert action 
%
\begin{align}
S = \int \dd[4]{x} F g \frac{M_P^2}{2} R
\,,
\end{align}
%
and plug in a tensor-perturbed FRLW metric: 
%
\begin{align}
\dd{s^2} = - a^2 \qty(\dd{\eta^2  } + (\delta_{ij} + h_{ij}) \dd{\vec{x}}^2)
\,,
\end{align}
%
we get the equations of motion for gravitational waves. 
The scenario of slow-roll inflation gives rise to an energy spectrum which is unobservable with LISA as well as ground-based detectors. 

If we add an axion to the inflaton Lagrangian: 
%
\begin{align}
\mathcal{L} = \frac{1}{2} \partial_{\mu } \phi \partial^{\mu }\phi + V(\phi ) + \underbrace{\frac{\phi}{\Lambda } F_{\mu \nu } \widetilde{F}^{\mu \nu }}_{\text{Additional term}} 
\,,
\end{align}
%
we can see enhanced gravitational waves, which we hope would be detectable with LISA! However the enhancement is very small, we cannot see it. 

Chaotic preheating models predict higher amplitudes, however they are at very high frequencies. 

We can get observable signals by fine-tuning the parameters, unlikely scenario. 

What about phase transition, the collisions of primordial vacuum bubbles? We can get numerical estimates on the spectral shape of this signal, and they seem promising and detectable. 

What about cosmic defects? We can have Domain Walls, Cosmic Strings and Cosmic Monopoles. 

The prediction here is a flat spectrum at observable frequencies, at possibly observable amplitudes depending on the emission time. 



\end{document}
