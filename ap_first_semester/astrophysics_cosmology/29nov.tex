\documentclass[main.tex]{subfiles}
\begin{document}

\section*{Fri Nov 29 2019}

We want to derive the Jeans instability criterion, starting from the structure equations: 
%
\begin{enumerate}
    \item continuity;
    \item momentum conservation;
    \item Poisson for the gravitational field;
    \item entropy conservation. 
\end{enumerate}

Yesterday there was a mistake: the entropy conservation is actually 
%
\begin{align}
  \frac{\mathrm{D} s}{\mathrm{D} t} + s \nabla \cdot \vec{v} = 0 
\,.
\end{align}
%

There is no need to make an ansatz for the pressure, since we can compute \(P = P (\rho ,s )\): we have the relation seen last time, involving the speed of sound. 

Our \emph{ansatz } is 
%
\begin{align}
  \delta x_i = x_{i0} \exp( i \vec{k} \cdot \vec{x} - i \omega t)
\,,
\end{align}
%
with \(x_{i} = \rho, \vec{v}, \Phi , s\). 

Our equations become: 
%
\begin{align}
  i \omega \delta \rho_0 + \rho_0 i \vec{k} \cdot \vec{v_0} &=0  \\
  i \omega \vec{v_0} &= \frac{1}{\rho_0 } i \vec{k}
  \qty(c^2_s \delta \rho_0 + \pdv{P}{s} \delta s_0 ) 
  - i \vec{k} \delta \Phi_0 \\
  k^2 \delta \Phi &= 4 \pi G \delta \rho_0  \\
  \omega \delta s_0 &= 0
\,,
\end{align}
%
so one class of solutions will have \(\omega = 0\), that is, we consider time-independent ones: 
%
\begin{align}
  \rho_0 i \vec{k} \cdot \vec{v_0} &=0  \\
  0 &= \frac{1}{\rho_0 }  \vec{k}
  \qty(c^2_s \delta \rho_0 + \pdv{P}{s} \delta s_0 ) 
  - \vec{k} \delta \Phi_0 \\
  k^2 \delta \Phi &= 4 \pi G \delta \rho_0  
\,,
\end{align}
%
and we have a result from Helmholtz: the fact that every velocity field can be decomposed into \(\vec{v} = \nabla \Psi + \vec{T}\) with \(\nabla \cdot \vec{T} = 0\). 

As soon as we ask the perturbation to be time independent we only find vortical motions, with \(\vec{v} \perp \vec{k}\). 

Now we consider the velocity to be irrotational we consider the term \(\delta s_0 =0\), since \(\omega \neq 0\) in general. 
%
\begin{align}
  \omega \delta \rho_0 + \rho_0 \vec{k} \cdot \vec{v_0} &=0  \\
  \omega \vec{v_0} &= \frac{1}{\rho_0 } \vec{k}
  c^2_s \delta \rho_0  
  - \vec{k} \delta \Phi_0 \\
  k^2 \delta \Phi &= 4 \pi G \delta \rho_0 
\,,
\end{align}
%
which we can write as a linear system for the vector \(\delta \rho_0 , v_0, \delta \Phi_0 \). 

The 5x5 coefficient matrix is:
%
\begin{align}
  \left[\begin{array}{ccc}
  \omega  & \rho_0 \vec{k}  & 0 \\ 
  \frac{1}{\rho_0 } \vec{k} c_s^2 & \omega  & \vec{k} \\ 
  4 \pi G & 0 & k^2
  \end{array}\right]
  \,,
\end{align}
%
which has determinant 
%
\begin{align}
  \omega k^2 - \rho_0 \vec{k} \cdot \qty( \frac{1}{\rho_0 }\vec{k} c_s^2 - 4 \pi G \vec{k} ) = 0
\,,
\end{align}
%
which gives the dispersion relation \(\omega^2 = c_s^2 k^2 - \rho_0 4 \pi G \). 

A solution will be a combination of these. If \(\omega^2<0\), which can happen because of the minus sign. If that happens, instead of a propagating wave we have a standing wave. 

We have \(\omega = (4 \pi G \rho_0  )^{1/2}\), so the characteristic time is 
%
\begin{align}
  \tau = \frac{1}{\sqrt{4 \pi G \rho_0 }}
\,,
\end{align}
%


The free fall time can be computed to be: 
%
\begin{align}
  \tau _{\text{free fall}} = \qty(\frac{3 \pi }{ 32 G \rho_0 })^{1/2}
\,.
\end{align}
%

We have a typical Jeans wavenumber: 
%
\begin{align}
  k_J^2 = \frac{4 \pi G \rho_0 }{c_s^2}
\,,
\end{align}
%
corresponding to when the frequency becomes imaginary. 

We can compare this to a plasma of charged particles, with the electrostatic potential instead of the gravitational field. 
In that case we find 
%
\begin{align}
  \omega^2 = c_s^2 k^2 + \frac{4 \pi n_e e^2}{m_e}
\,,
\end{align}
%
where \(n_e\) is the number density of electrons, \(m_e\) is the electron mass. 

We have the following analogies: 
%
\begin{align}
  n_e &\rightarrow \rho_0 / m  \\
  m_e &\rightarrow m  \\
  e^2 &\rightarrow Gm^2  \\
  + &\rightarrow -
\,;
\end{align}
%
the first of these are due to the inertial mass being equal to the gravitational mass; the plus becoming a minus is due to the fact that we do not have negative charge in the gravitational setting, so there cannot be a screening effect. 

We now come back to the RW line element: 
%
\begin{align}
  \dd{s^2} = c^2 \dd{t^2} 
  -a^2(t) \qty(\frac{ \dd{r^2}}{1 - kr^2} + r^2 \dd{\Omega^2})
\,,
\end{align}
%
in which the most important thing is that \(a\) is a function of time. 

We will now use a Newtonian approach: we ignore spatial curvature, that is, set \(k=0\): this is good for small enough scales. The thing is: the time scales of gravitational collapse are long, and we cannot ignore the effects of spacetime expansion. 

We use a coordinate defined by \(\vec{r} = a(t) \vec{x}\), so that the line element becomes 
%
\begin{align}
  \dd{s^2} = c^2 \dd{t^2} - \dd{\abs{x}^2} - 
\abs{x}^2 \dd{\Omega^2}
\,. 
\end{align}
%
\todo[inline]{right?}

What will be done now could have been done by Newton: he did not just because he thought the universe was static. We have (dropping the vector sign, but still impying it): 
%
\begin{align}
  u = \dot{r} = \dot{a} x + a \dot{x} 
  = \frac{\dot{a}}{a} r + v
\,,
\end{align}
%
where \(v = a \dot{x}\) is called the peculiar velocity which galaxies and such can have.  

Let us forget about the pressure gradient and just look at how the instability evolves: this will also apply to dark matter which has no pressure. We look at 
%
\begin{align}
  \qty[\pdv[]{\rho }{t} ]_{\vec{r}} + \nabla_{\vec{r}} \qty(\rho \vec{u}) = 0 
\,,
\end{align}
%
the velocity equation is 
%
\begin{align}
  \qty[\pdv{\vec{u}}{r}]_{\vec{r}} + \qty(\vec{u} \cdot \nabla_{\vec{r}}) \vec{u} = - \nabla_{\vec{r}} \Phi 
\,,
\end{align}
%
and finally 
%
\begin{align}
  \nabla^2_{\vec{r}} = 4 \pi G \rho 
\,.
\end{align}

Our expression for \(\vec{u}\) is \(H \vec{r} + \vec{v}\), also we have 
%
\begin{align}
  \rho (\vec{r}, t) = \rho_{b} (t) + \delta \rho (\vec{r}, t)
\,,
\end{align}
%
a background plus a perturbation. 

To find the background, we assume that the density of matter is space independent but time dependent: then we find 
%
\begin{align}
  \Phi_{b } (\vec{r}, t) = \frac{2 \pi G}{3} \rho_{b} (t) r^2
\,,
\end{align}
%
which has gradient 
%
\begin{align}
  \nabla_{\vec{r}} \Phi_{b} = \frac{4 \pi G}{3} \rho_b (t) \vec{r} 
\,,
\end{align}
%
and laplacian 
%
\begin{align}
  \nabla^2_{\vec{r}} \Phi_{b} = 4 \pi G \rho_b
\,,
\end{align}
%
but this diverges at \(r \rightarrow \infty\)! this is not an issue with the solution, but with Newtonian mechanics. 

This is an elliptic differential equation: hyperbolic ones look like \( \square \Phi = 0\), elliptic ones like \(\nabla^2 \Phi =0  \), while parabolic ones (???)

We want to get equations in the comoving coordinates, not the local inertial ones. Take a generic function \(f(\vec{r}, t)\), which can also be expressed with respect to \((\vec{x}, t)\). 

Then, the difference between the derivatives is: 
%
\begin{align} \label{eq:comoving-derivatives-relation}
  \qty[\pdv{f}{t}]_{\vec{x}} =   \qty[\pdv{f}{t}]_{\vec{r}} + H \qty(\vec{r} \cdot \nabla_{\vec{r}}) f
\,.
\end{align}

Take the convective derivative 
%
\begin{align}
  \frac{\mathrm{D} f}{\mathrm{D}t } 
  = \pdv{f}{t}_{\vec{r}} + \pdv{f}{\vec{r}} \cdot \vec{\dot{r}} 
\,,
\end{align}
%
where \(\vec{\dot{r}
} = \vec{u} = H \vec{r} + \vec{v} \). 
From what we saw, this can be expressed as 
%
\begin{align}
  \pdv{f}{t}_{\vec{r}} + H \qty(\vec{r} \cdot \nabla_{\vec{r}}  ) + \qty(\vec{v} \cdot \nabla_{\vec{r}}) f 
\,,
\end{align}
%
but we can also do it the other way round, keeping \(\vec{x}\) constant: that way, we find 
%
\begin{align}
  \frac{\mathrm{D} f}{\mathrm{D}t } 
  = \pdv{f}{t}_{\vec{x}} + \pdv{f}{\vec{x}}
\,,
\end{align}
%
and identifying the terms we ge the desired relation, equation \eqref{eq:comoving-derivatives-relation}. 

Skipping some passages, from the continuity equation we get:
%
\begin{align}
  \pdv{\rho }{t}_{\vec{x}} + 3 H \rho + 
  \frac{1}{a} \nabla_{\vec{x}} \qty(\rho \vec{v}) = 0
\,,
\end{align}
%
while for the Euler equation we have: 
%
\begin{align}
  \pdv{H \vec{r}}{t}_{\vec{r}} + H^2 \qty(\vec{r} \cdot \nabla_{\vec{r}}) \vec{r} = - \nabla_{\vec{r}} \Phi_{b}
\,. 
\end{align}
%
The divergence term is just \(r^{j} \partial_{j} r^{i} = r^{i}\). Inserting the expression we know for the gravitational potential, whose gradient is proportional to \(\vec{r}\), we get: 
%
\begin{align}
  \vec{r} \qty(\dot{H} +  H^2 = - \frac{4 \pi G }{3} \rho_{b } )
\,,
\end{align}
%
which must hold for any \(\vec{r}\), so we drop it and recover one of the Friedmann equations! 
Note that we did not use any GR, everything was Newtonian. 

We will get another result: 
%
\begin{align}
  \pdv{\vec{v}}{t} + H \vec{v} + \frac{1}{a} \qty(\vec{v} \cdot \nabla_{\vec{x}}) \vec{v} = - \frac{1}{a} \nabla_{\vec{x}} \delta \Phi 
\,,
\end{align}
%
our new perturbed Euler equation; while the new Poisson equation is 
%
\begin{align}
  \nabla^2 \delta \Phi = a^2 4 \pi G \delta \rho 
\,.
\end{align}

Some cosmologists like to define 
%
\begin{align}
  \delta (\vec{x}, t) = \frac{ \delta \rho}{\rho _b} = \frac{\rho (\vec{x}, t) - \rho _b (t)}{\rho _b (t)}
\,,
\end{align}
%
which can be anywhere from \(-1\) to \(+ \infty \). 

Then, we can interpret a negative \(\delta \rho \) as a negative charge. 

Why the Newtonian approach? Nobody knows how to write down the equations for a general relativistic self-gravitating fluid. 

Is the Newtonian approximation a good one? In it, we have 
%
\begin{align}
  \nabla^2 \delta \Phi  = 4 \pi G \rho _b a^2 \delta 
\,,
\end{align}
%
can we make the weak field approximation? 

Typically 
%
\begin{align}
  \delta \Phi \sim 4 \pi G \rho _b a^2 \delta \lambda^2
\,,
\end{align}
%
where \(\lambda \) is the variation scale, from the Laplacian. 

From the Friedmann equation, \(H^2=  \frac{8 \pi G}{3} \rho_b\), then we have 
%
\begin{align}
  \delta \Phi \sim \frac{3}{2} H^2 \delta a^2 \lambda^2 \sim  \qty(\frac{\lambda^2  }{\lambda _{\text{hor}}^2}) \delta 
\,,
\end{align}
%
and \(\lambda \sim \SI{}{\mega\parsec}\), the galactic perturbation scale, while \(\lambda _{\text{hor}} \sim \SI{}{\giga\parsec}\). 

As long as the perturbations are only galactic, the Newtonian approximation is good. 

\end{document}