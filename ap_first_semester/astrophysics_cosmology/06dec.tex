\documentclass[main.tex]{subfiles}
\begin{document}

\section*{Fri Dec 06 2019}

Today we will discuss another possible \emph{caveat} for star formation: the case where the star collapses but does not ignite. 

After recombination the baryons' temperature decays more rapidly than the radiation's, so we get a temperature low enough to satisfy the instability criterion. 

The temperature stays low during the collapse: the thermal energy is used in order to break the bonds in \ce{H2} and ionize the hydrogen. 

Then we have a gas which is opaque to radiation: there is scattering, which means we lose energy through radiation very slowly. 
Then the Virial theorem is very close to being true. 

At the end of the process, even the mass does not matter anymore: we got \SI{2.6}{eV} as the temperature regardless of the mass. This is equivalent to around \SI{30e3}{K}.  
We must compare this to the ignition temperature: that is in the order of \SI{}{keV} (\SI{15e6}{K} is equivalent to around \SI{1.3}{keV}). 

After free-fall the radius is on the order of \SI{e10}{m} for a solar mass star, while the Sun's radius is smaller by two orders of magnitude. 

Now we discuss the \emph{conditions for stardom}: we need to account for the fermionic nature of protons and electrons, which will give us a maximum density due to the Pauli exclusion principle. 

The way we will treat this today will be quite rough. 

We know that the De Broglie wavelength is given by 
%
\begin{align}
  \lambda = \frac{h}{p} = \frac{2 \pi \hbar}{p}
\,.
\end{align}

How do we calculate \(p\)? we assume that the particles are nonrelativistic and apply \(E_{k} = p^2 /2m_e\). 

Objects which will not satisfy the conditions we talk about today become brown dwarves. 

The kinetic energy is of the order \(k_B T\), therefore 
%
\begin{align}
  p \sim \sqrt{2 m_e k_B T}
\,,
\end{align}
%
and the critical density is defined by 
%
\begin{align}
  \rho_c = \frac{\overline{m}}{\lambda^3}
\,,
\end{align}
%
where from the formula we found 
%
\begin{align}
  \lambda = \frac{2 \pi \hbar}{\sqrt{2 m_e k_B T}}
\,,
\end{align}
%
which gives approximately 
%
\begin{align}
  \rho _c \sim \overline{m} \frac{(m_e k_B T)^{3/2}}{(2 \pi \hbar)^{3}}
\,,
\end{align}
%
and from the virial theorem \(2 E_k + E _{\text{gr}}=0\), with 
%
\begin{align}
  E_k = \frac{3}{2} N k_B T = \frac{3}{2} \frac{M}{\overline{m}} k_B T
\,,
\end{align}
%
while 
%
\begin{align}
  E _{\text{gr}} = - \frac{G M^2}{R}
\,,
\end{align}
%
which means 
%
\begin{align}
  3 N k_B T = \frac{GM^2}{R}
\,,
\end{align}
%
and we can rewrite this as 
%
\begin{align}
  \frac{3 k_B T}{\overline{m}} = \frac{GM}{R}
\,,
\end{align}
%
and we can express the mass as 
%
\begin{align}
  M = \frac{4}{3} \pi \overline{\rho} R^3
\,,
\end{align}
%
so 
%
\begin{align}
  \frac{1}{R} = \qty(\frac{4 \pi }{3} \frac{\overline{\rho}}{M})^{1/3}
\,,
\end{align}
%
which gives us the result 
%
\begin{align}
  k_B T = \frac{GM \overline{m}}{3} \qty(\frac{4 \pi }{3} \frac{\overline{\rho}}{M})^{1/3}
\,,
\end{align}
%
and if we substitute the critical density in for \(\overline{\rho}\) we will get the maximum possible temperature allowed at a given mass. 

This yields
%
\begin{align}
  k_B T = \frac{GM \overline{m}}{3} \qty(\frac{4 \pi }{3 M})^{1/3} \frac{\overline{m}^{1/3}}{(2 \pi \hbar)} \qty(m_e k_B T)^{1/2}
\,,
\end{align}
%
which it is convenient to square: 
%
\begin{align}
  (k_B T)^2 = \frac{G^2M^2 \overline{m}^2}{9}
  \qty(\frac{4 \pi }{3 M})^{2/3} \frac{\overline{m}^{2/3}}{(2 \pi \hbar)^2} m_e k_B T 
\,,
\end{align}
%
so we can simplify, and up to an order-1 constant 
%
\begin{align}
  k_B T = \frac{G^2 \overline{m}^{8/3} M^{4/3}}{(2 \pi \hbar)^2} 
\,,
\end{align}
%
and inserting the ignition temperature of around \SI{1}{keV} we get \(M _{\text{min}} \sim 0.08 M_{\odot}\). 

This is confirmed experimentally. 

Let us consider the Sun. Its mass is \(M_{\odot} \approx \SI{1.99e30}{kg}\), the radius is \(R_{\odot} \approx \SI{6.96e8}{m}\), the electromagnetic luminosity is \(L_{\odot} = \SI{3.86e26}{W}\). 

The age of the Sun is around \(t_{\odot} \approx \SI{4.55e9}{yr}\), which is comparable to the age of the Universe. 

The central density is \(\rho _c \approx \SI{1.48e5}{kg m^{-3}}\), while the central temperature is \(T_c = \SI{1.56e7}{K}\), and the central pressure is around \(P_c = \SI{2.29e16}{Pa}\).  

The effective temperature is around \(T_E \approx \SI{5780}{K}\). 
\todo[inline]{Definition?}

What is the corresponding free fall time? It is much shorter than the age of the Sun: the Sun is not in free fall. 
\todo[inline]{What is the number?}

We know that 
%
\begin{align}
  \expval{P} = -\frac{1}{3} \frac{E _{\text{gr}}}{V}
\,,
\end{align}
%
where \(E _{\text{gr}} = - G M^2/R\) while \(V = 4 \pi R^3/3\): plugging the Sun's numbers we get 
%
\begin{align}
  \expval{P} = \SI{e14}{Pa}
\,,
\end{align}
%
100 times less than the central density. 

The density of the Sun is actually very similar to the density of water. 

Was it correct to use nonrelativistic equations? (???)

%
\begin{align}
  \expval{P} = \frac{\overline{\rho}}{\overline{m}} k_B T_I
\,,
\end{align}
%
where \(T_I\) is the mean internal temperature of the Sun. 
The value of \(\overline{m} \approx 0.61\) instead of 0.5 when considering the proper composition of the Sun. 

We get 
%
\begin{align}
  k_B T_I \approx \frac{G M_{\odot} \overline{m}}{3 R_{\odot} } \approx \SI{1.5}{keV} \approx \SI{6e6}{K}
\,.
\end{align}

We have evidence that the Sun is a blackbody, we can write 
%
\begin{align}
  L_{\odot} = 4 \pi R_{\odot} \sigma T^{4}_{E}
\,,
\end{align}
%
where \(\sigma \) is Stefan's constant: \(\sigma \approx \SI{5.67e-8}{W m^{-2} K^{-4}}\). 

In principle we could define another quantity: 
%
\begin{align}
  L_{\odot}^{\prime } = 4 \pi R_{\odot}^2 \sigma T_I^4
\,,
\end{align}
%
which does not fit the data. Why is this? 

We must consider a photon which comes from the interior of the Sun and goes towards the outside: it will follow a random walk scattering many times. However, depending on the density of electron in the outer regions (electron scattering dominates in the outside), the last scattering is in the outermost regions of the star. 

We need to deal with the Fourier equation. 
First of all, we need to write the Langevin equation: this is connected to the work by Einstein on Browinan motion. 

These are processes in which there is a ``random'' force (due to the fact that our description of the microscopic state is probabilistic), and possibly deterministic forces. 

\todo[inline]{What is the stuff about viscosity?}

The equation in the end is like: 
%
\begin{align}
  \dot{\vec{x}} = \vec{\eta}
\,,
\end{align}
\todo[inline]{What? }
%
with \(\expval{\eta _i (t) \eta_j (t')} = 2 D \delta_{ij} \delta (t - t')\): there is complete uncorrelation. This is the Markov property: the process has no memory.

This gives us a Gaussian distribution of the positions of the particles, and we get the equation: 
%
\begin{align}
  \pdv{P}{t} = D \nabla^2 P 
\,,
\end{align}
%
where \(P = P( \vec{x}, t)\). This is a \emph{parabolic equation}, the \emph{Fokker-Planck} formula. 
We need to give it both initial conditions and boundary conditions. 

\todo[inline]{What is \(P\)?}

There are three kinds of boundary conditions: 
\begin{enumerate}
  \item nothing: free boundary, the solution can diverge;
  \item a reflective boundary: particles ``bounce back'', in order to deal with this we use the images method, as in electromagnetism;
  \item an absorbing boundary: particles disappear if they reach the boundary. 
\end{enumerate}

This is equivalent to the Fourier transport equation. 

The solution without boundary is a Gaussian with variance \(\expval{x^2} =  2 D t\) and centered around zero. 

If \(\sigma = \sqrt{\expval{x^2}}\) becomes greater than the boundary then most of the particles have escaped. 

\(\vec{D}\) is the displacement vector of the Sun: it is 
%
\begin{align}
  \vec{D} = \sum _{i} \vec{l}_{i}
\,,
\end{align}
%
where the \(\vec{l}_i\) are the displacement vectors of its various steps. 
%
\begin{align}
  \expval{\vec{D}^2} = \sum _{i} \expval{\vec{l}_{i}^2} 
  + \sum _{i<j} \expval{\vec{l}_i \cdot \vec{l}_j}
\,,
\end{align}
%
but if we have isotropy then the scalar products have mean zero. This might be unphysical since the steps are larger at the boundary than at the center\dots

Then we find: 
%
\begin{align}
  \expval{\vec{D}^2} = N l^2 = R _{\odot}^2
\,,
\end{align}
%
where \(N = R _{\odot}^2 / l^2\). 

The time it takes for a photon to cover a distance \(l\) is \(t= l/c\). 
Then we have \(t_{RW} = Nt = R^2 _{\odot} l / (l^2 c)\), which means 
%
\begin{align}
  t_{RW} = \frac{R _{\odot}^2}{cl}
\,,
\end{align}
%
while in direct flight the photon would take \(t_0 = R _{\odot}/ c\): their ratio is 
%
\begin{align}
  \frac{t_{RW}}{t_0 } = \frac{R _{\odot}}{l}
\,,
\end{align}
%
and then 
%
\begin{align}
  L _{\odot} = L _{\odot}^{\prime } \frac{l}{R _{\odot}}
\,,
\end{align}
%
which means 
%
\begin{align}
  T_E = \qty(\frac{l}{R _{\odot}})^{1/4} T_I
\,,
\end{align}
%
so we can gather \(l\) by knowing the other three parameters: we get \(l = \SI{1}{mm}\). This is actually an average of the mean free paths. 

\begin{align}
  L = L' \frac{l}{R} = 4 \pi R _{\odot}^2 \sigma T_I^{4} \frac{l}{R}
\,,
\end{align}
%
and we know that \(k_B T_I = \frac{GM \overline{m}}{3 \hbar}\): replacin this we find 
%
\begin{align}
  L = 4 \pi R^2 _{\odot}\sigma  \qty(\frac{GM \overline{m}}{3 R k_B})^{4} \frac{l}{R}
  = \frac{(4 \pi )^2}{3^{5}} \frac{\sigma }{k_B^4} G^{4} \overline{m}^{4} \overline{\rho} l M^3  
\,.
\end{align}

What is the typical lifespan of a star? This gives us \(L \sim M^3\), which means \(\tau \sim M/L \sim M^{-2}\). 
Observationally, this matches the data pretty well. 

Now we discuss thermonuclear fusion. The reactions we need are 
%
\begin{subequations}
\begin{align}
  \ce{p + p} &\rightarrow \ce{d + e+ +} \nu_e  \\
  \ce{p + d} &\rightarrow \ce{^3He +} \gamma  \\
  \ce{^3He + ^3He} &\rightarrow \ce{^4He + p + p} 
\,,
\end{align}
\end{subequations}
%
which involves the weak interaction (for the first process: \(\tau \sim \SI{5e9}{yr}\)), EM interaction (for the second process: \(\tau \sim \SI{1}{s}\)) and strong interaction (for the third process: \(\tau \sim  \SI{3e5}{yr}\)). On the other hand, the process 
%
\begin{subequations}
\begin{align}
  \ce{n + p} &\rightarrow \ce{d} + \gamma   \\
  \ce{d + d} &\rightarrow \ce{^3He + n}  \\
  \ce{^3He + d}  &\rightarrow \ce{^4He + p}
\,,
\end{align}
\end{subequations}
%
which does not use the weak interactions. However, later there are no more free neutrons: this means that even if it is slower the first process is the only one which can happen. 

The net balance is 4 protons in, 1 \ce{^4 He} out. 

\end{document}
