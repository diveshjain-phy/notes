\documentclass[main.tex]{subfiles}
\begin{document}

\subsection{Conditions for stardom and brown dwarfs}

\marginpar{Friday\\ 2019-12-06, \\ compiled \\ \today}
% \section*{Fri Dec 06 2019}

% Today we will discuss another possible \emph{caveat} for star formation: the case where the star collapses but does not ignite. 

% After recombination the baryons' temperature decays more rapidly than the radiation's, so we get a temperature low enough to satisfy the instability criterion. 

% The temperature stays low during the collapse: the thermal energy is used in order to break the bonds in \ce{H2} and ionize the hydrogen. 

% Then we have a gas which is opaque to radiation: there is scattering, which means we lose energy through radiation very slowly. 
% Then the Virial theorem is very close to being true. 

% At the end of the process, even the mass does not matter anymore: we got \SI{2.6}{eV} as the temperature regardless of the mass. This is equivalent to around \SI{30e3}{K}.  
% We must compare this to the ignition temperature: that is in the order of \SI{}{keV} (\SI{15e6}{K} is equivalent to around \SI{1.3}{keV}). 

% After free-fall the radius is on the order of \SI{e10}{m} for a solar mass star, while the Sun's radius is smaller by two orders of magnitude. 

At this point, the proto-star reaching a high enough temperature to fuse hydrogen is not a given: it may not happen. 

In order to discuss the \emph{conditions for stardom} we need to account (albeit in an approximate way, as usual) for the fermionic nature of protons and electrons, which will give us a maximum density due to the Pauli exclusion principle: the particles will form a degenerate Fermi gas. 
Depending on the mass of the star, this maximum density may be reached before the ignition temperature: when this is the case, a \textbf{brown dwarf} is formed.
% The way we will treat this today will be quite rough. 

We estimate the minumum space a fermion can occupy with its De Broglie wavelength:
%
\begin{align}
  \lambda = \frac{h}{p} = \frac{2 \pi \hbar}{p}
\,,
\end{align}
%
where \(p\) is the momentum of the particle.
Since the temperature is still less than a \SI{}{keV}, both electrons and protons are nonrelativistic, so we can approximate \(E_{k} = p^2 /2m\), which will typically be of the order \(E_K \sim k_B T\).

% Objects which will not satisfy the conditions we talk about today become brown dwarves. 

% The kinetic energy is of the order \(k_B T\), therefore 
Then, the momentum and wavelength will be:
%
\begin{align}
  p \sim \sqrt{2 m k_B T} \implies \lambda \sim \frac{2 \pi \hbar}{\sqrt{m k_B T}}
\,.
\end{align}

We can see that \(\lambda \sim 1/\sqrt{m}\): this means that the wavelength is smaller (by a factor 40) for the protons than it is for the electrons, and they appear in equal numbers; therefore the first bound to be reached will be that of the degenerate electron gas, for this reason we will neglect protons now.

% and the critical density is defined by 
The critical density will be reached when we have an electron for each \(\lambda^3\); however in order to ensure local charge neutrality we must have protons distributed in the same way, and their mass contribution to the density will be the largest:

\todo[inline]{I'd actually say, then, that we should write \(\rho _c = m_p / \lambda^3\)! }
%
\begin{align}
  \rho_c = \frac{\overline{m}}{\lambda^3}
  \sim \overline{m} \frac{(m_e k_B T)^{3/2}}{(2 \pi \hbar)^{3}}
\,.
\end{align}
% %
% where from the formula we found mk
% %
% \begin{align}
%   \lambda = \frac{2 \pi \hbar}{\sqrt{2 m_e k_B T}}
% \,,
% \end{align}
%
% which gives approximately 
% %
% \begin{align}
%   \rho _c \sim \overline{m} \frac{(m_e k_B T)^{3/2}}{(2 \pi \hbar)^{3}}
% \,,
% \end{align}
%
% and from the virial theorem \(2 E_k + E _{\text{gr}}=0\), with 
% %
% \begin{align}
%   E_k = \frac{3}{2} N k_B T = \frac{3}{2} \frac{M}{\overline{m}} k_B T
% \,,
% \end{align}
% %
% while 
% %
% \begin{align}
%   E _{\text{gr}} = - \frac{G M^2}{R}
% \,,
% \end{align}
% %
% which means 

As we have seen earlier, since the virial theorem applies the temperature of the proto-star is tied to its gravitational binding energy by
%
\begin{align}
  3 N k_B T = \frac{GM^2}{R} \implies k_B T = \frac{GM \overline{m}}{3  R}
\,.
\end{align}
% %
% and we can rewrite this as 
% %
% \begin{align}
%   \frac{3 k_B T}{\overline{m}} = \frac{GM}{R}
% \,,
% \end{align}
% %

The mass can be expressed in terms of the average density as 
\(M = \frac{4}{3} \pi \overline{\rho} R^3 \), so 
%
\begin{align}
  \frac{1}{R} = \qty(\frac{4 \pi }{3} \frac{\overline{\rho}}{M})^{1/3}
\,,
\end{align}
%
which gives us the result 
%
\begin{align}
  k_B T = \frac{GM \overline{m}}{3} \qty(\frac{4 \pi }{3} \frac{\overline{\rho}}{M})^{1/3}
\,.
\end{align}

If we substitute the critical density \(\rho _c\) for \(\overline{\rho}\) we will get the maximum possible temperature allowed at a given mass \(M\): after some manipulation we get (up to a small constant, which we neglect since the calculation is rough anyway):
%
\begin{align}
  k_B T &= \frac{GM \overline{m}}{3} \qty(\frac{4 \pi }{3 M})^{1/3} \frac{\overline{m}^{1/3}}{(2 \pi \hbar)} \qty(m_e k_B T)^{1/2} \\
  (k_B T)^2 &= \frac{G^2M^2 \overline{m}^2}{9} 
  \qty(\frac{4 \pi }{3 M})^{2/3} \frac{\overline{m}^{2/3}}{(2 \pi \hbar)^2} m_e k_B T  \\
  k_B T &\approx \frac{G^2 \overline{m}^{8/3} M^{4/3} m_e}{(2 \pi \hbar)^2} 
\,.
\end{align}

Inserting the ignition temperature of around \SI{1}{keV} we get \(M _{\text{min}} \approx M_{\odot}\), which is almost the right order of magnitude: more accurate models agree with the observational result of \(M _{\text{min}} \approx \num{.08} M_{\odot}\). 

\section{The Sun and other stars}

Let us start off with a list of the physical characteristics of the Sun: its mass is \(M_{\odot} \approx \SI{1.99e30}{kg}\), its radius is \(R_{\odot} \approx \SI{6.96e8}{m}\), its bolometric luminosity\footnote{This means: the total luminosity, integrated across all the electromagnetic spectrum.} is \(L_{\odot} = \SI{3.86e26}{W}\). 

Its age is around \(t_{\odot} \approx \SI{4.55e9}{yr}\), which is comparable to the age of the Universe. 

At its core, the density is \(\rho _c \approx \SI{1.48e5}{kg m^{-3}}\), the temperature is \(T_c = \SI{1.56e7}{K} \approx \SI{1.3}{keV}\), and the pressure is around \(P_c = \SI{2.29e16}{Pa}\).  

The radiation emitted by the Sun approximately follows a blackbody curve, whose characteristic temperature is called the \emph{effective temperature}: it is around \(T_E \approx \SI{5780}{K} \approx \SI{.5}{eV}\). 
% \todo[inline]{Definition?}

% What is the corresponding free fall time? It is much shorter than the age of the Sun: the Sun is not in free fall. 
% \todo[inline]{What is the number?}

The power emitted by the Sun is quite small as a fraction of its total energy, so we can apply the virial theorem: one of its formulations we found is
%
\begin{align}
  \expval{P} = -\frac{1}{3} \frac{E _{\text{gr}}}{V}
\,,
\end{align}
%
where \(E _{\text{gr}} \approx - G M^2/R\) while \(V = 4 \pi R^3/3\): plugging the Sun's numbers we get 
%
\begin{align}
  \expval{P} 
  = \frac{G M_{\odot}^2}{4 \pi R_{\odot}^{4}}
  \approx \SI{e14}{Pa}
\,,
\end{align}
%
100 times less than the central pressure. 
Similarly, the average density can be computed as \(\expval{\rho } = M _{\odot} / V \approx \SI{1.4e3}{kg/m^3}\), just slightly more than the density of water! This is also roughly 100 times less than the central value.

% The density of the Sun is actually very similar to the density of water. 

% Was it correct to use nonrelativistic equations? (???)

By the ideal gas law (which does not precisely apply, but as we saw the Sun has a rather low density overall so it is a reasonable approximation) we can find the mean internal temperature of the Sun, \(T_I\), as:
%
\begin{align}
  \expval{P} = \frac{\expval{\rho}}{\overline{m}} k_B T_I
\,,
\end{align}
%
% where \(T_I\) is the mean internal temperature of the Sun. 
where \(\overline{m}\), the average particle mass, is \(\overline{m} \approx 0.61 m_H\) instead of \(\num{.5} m_H\) when considering the proper composition of the Sun, which has a noticeable fraction of helium and heavier elements. 

We get 
%
\begin{align}
  k_B T_I
  =  \frac{G M_{\odot}^2}{4 \pi R_{\odot}^{4}}
   \qty( \frac{M_{\odot}}{4 \pi R_{\odot}^3 / 3})^{-1} \overline{m}
  = \frac{G M_{\odot} \overline{m}}{3 R_{\odot} } \approx \SI{.5}{keV} \approx \SI{5e6}{K}
\,.
\end{align}

Again, we see the relation \(x _{\text{central}} \ll x _{\text{mean}} \ll x _{\text{surface}}\), which holds for \(x = \) density, pressure, temperature. 

% We have evidence that the Sun is a blackbody, we can write 
The measured bolometric luminosity of the Sun is consistent with 
%
\begin{align}
  L_{\odot} = 4 \pi R_{\odot}^2 \sigma T^{4}_{E}
\,,
\end{align}
%
where \(\sigma \) is Stefan's constant: \(\sigma \approx \SI{5.67e-8}{W m^{-2} K^{-4}}\). 

Why does this hold with the comparatively low temperature \(T_E\) and not with the mean internal temperature \(T_I\)?
We will now see that this is because the interior of the Sun is optically thick (opaque), so any photon from the interior cannot simply be emitted, it will undergo many scatterings before doing so. 

% In principle we could define another quantity: 
% %
% \begin{align}
%   L_{\odot}^{\prime } = 4 \pi R_{\odot}^2 \sigma T_I^4
% \,,
% \end{align}
% %
% which does not fit the data. Why is this? 

\subsection{Radiative diffusion}


% We must consider a photon which comes from the interior of the Sun and goes towards the outside: it will follow a random walk scattering many times.

% However, depending on the density of electrons in the outer regions (electron scattering dominates in the outside), the last scattering is in the outermost regions of the star. 

% We need to deal with the Fourier equation. 
% First of all, we need to write the Langevin equation: this is connected to the work by Einstein on Browinan motion. 

% These are processes in which there is a ``random'' force (due to the fact that our description of the microscopic state is probabilistic), and possibly deterministic forces. 

% \todo[inline]{What is the stuff about viscosity?}

The motion of the photon is Brownian, and the total displacement \(\vec{D}\) from production to emission will be written in terms of \(N\) short straight tracts, 
%
\begin{align}
\vec{D} = \sum _{i=1}^{N} \vec{\ell}_i
\,.
\end{align}

The process is inherently stochastic, and we will need to describe it as such. 
We need two equations in order to describe it: the Langevin and Fokker-Planck equations.

The \textbf{Langevin} equation gives us the derivative of the position of the particle in terms of a stochastic force \(\eta\):
%
\begin{align}
  \dot{\vec{D}}(t) = \vec{\eta} (t)
\,.
\end{align}

The requirements on \(\eta \) are that it should have zero mean: \(\expval{\vec{\eta}}(t) = 0\) and that it should satisfy \(\expval{\eta _i (t) \eta_j (t')} = 2 D \delta_{ij} \delta (t - t')\).
This means that it is spatially and temporally uncorrelated, making the process a Markovian one: it has ``no memory''.
The parameter \(D\) is called the \emph{diffusion constant}, its units are \SI{}{m^2 /s} (and it is not related to the displacement \(\vec{D}\)).s 

% there is complete uncorrelation.
% This is the Markov property: the process has no memory.

Nontrivially (we will not discuss the details of the derivation) this gives us the \textbf{Fokker-Planck} formula: 
%
\begin{align}
  \pdv{P}{t} = D \nabla^2 P 
\,,
\end{align}
%
where \(P\), a function of position and time, quantifies the probability density of finding a photon there.

Mathematically speaking this is a \emph{parabolic differential equation}, which concretely means that we need to give it both initial conditions and boundary conditions. 

% \todo[inline]{What is \(P\)?}

% In the There are three kinds of boundary conditions we can set: 
% \begin{enumerate}
%   \item nothing: free boundary, allowing the solution can diverge;
%   \item a reflective boundary: particles ``bounce back'', in order to deal with this we use the images method, as in electromagnetism;
%   \item an absorbing boundary: particles disappear if they reach the boundary. 
% \end{enumerate}
The kind of boundary condition we want for the study of the Sun is a so-called \emph{absorbing boundary}: physically, as a photon reaches the surface it is emitted, which from the inside looks like if the boundary ``absorbed it''.

% This is equivalent to the Fourier transport equation. 

The solution to the Fokker-Planck equation, with an impulsive initial condition like \(P(x, t=0) = \delta^{(3)} (x)\) is a Gaussian with variance \(\expval{x^2} =  2 D t\) and centered around zero: 
%
\begin{align}
P(x, t) = \frac{1}{(4 \pi D t)^{3/2}} \exp(- \frac{x^2}{ 4 D t})
\,.
\end{align}

This holds for \(0 < \abs{\vec{x}} < R_{\odot}\), while for \(\abs{x} > R_{\odot}\) the particles have escaped. 

Integrating the Gaussian within the stellar boundary at a time \(t\) yields the probability that any single photon emitted at the center of the Sun is still inside it after a time \(t\). 
Roughly, this be the case with  \(\sigma = \sqrt{2 Dt } \lesssim R_{\odot} \).

% If \(\sigma = \sqrt{\expval{x^2}}\) becomes greater than the boundary then most of the particles have escaped. 

Let us now consider the problem geometrically. The mean square value of the displacement will be given by
%
\begin{align}
  \expval{\vec{D}^2} = \sum _{i} \expval{\vec{\ell}_{i}^2} 
  + 2\sum _{i<j} \expval{\vec{\ell}_i \cdot \vec{\ell}_j}
\,,
\end{align}
%
but if we have isotropy then the scalar products have mean zero, since they are averages of two lengths times a cosine.
% This might be unphysical since the steps are larger at the boundary than at the center\dots

Then we find that, in order for the photon to have reached the boundary of the star on average it will need to have undergone \(N\) scatterings: if we set \(\expval{\vec{D}^2} = R_{\odot}^2\) we get
%
\begin{align}
  \expval{\vec{D}^2} = N \ell^2 = R _{\odot}^2
\,,
\end{align}
%
so \(N = R _{\odot}^2 / \ell^2\), where \(\ell\) is the typical path travelled between scatterings. 

The time it takes for a photon to cover a distance \(\ell\) is \(t= \ell/c\). 
Then, the total time taken in the random walk is given by \(t_{RW} = Nt = R^2 _{\odot} \ell / (\ell^2 c)\), which means 
%
\begin{align}
  t_{RW} = \frac{R _{\odot}^2}{c \ell}
\,,
\end{align}
%
while in direct flight the photon would only have taken \(t_0 = R _{\odot}/ c\): their ratio is 
%
\begin{align}
  \frac{t_{RW}}{t_0 } = \frac{R _{\odot}}{ \ell}
\,.
\end{align}

\todo[inline]{The argument which follows is still unconvincing to me: the photons take \(\sim \SI{50}{kyr}\) to come out of the Sun, so this process will have stabilized in the Sun's \SI{5}{Gyr} lifetime!

Estimating the mean free path as \(\ell \sim 1/ (n \sigma _T)\) could work, it yields \(\sim \SI{1}{cm}\) with the mean density and \(\sim \SI{0.1}{mm}\) with the central density, so perhaps since in the low-\(\ell\) regions the photons stay for a longer time this can average out to \SI{1}{mm} when doing the calculation properly.}

If \(L' _{\odot}\) is the luminosity for the Sun calculated using the average internal temperature \(T_I\) instead of the effective temperature \(T_E\) we get
%
\begin{align}
  L _{\odot} = L _{\odot}^{\prime } \frac{\ell}{R _{\odot}}
\,,
\end{align}
%
which means 
%
\begin{align}
  T_E = \qty(\frac{\ell}{R _{\odot}})^{1/4} T_I
\,,
\end{align}
%
so we can calculate \(\ell\) by knowing the other three parameters:
we get that the mean free path, averaged over the star, is \(\ell \approx \SI{1}{mm}\).
% This is actually an average of the mean free paths. 

The total solar luminosity is then:
%
\begin{align}
  L_{\odot} = L'_{\odot} \frac{\ell}{R} = 4 \pi R _{\odot}^2 \sigma T_I^{4} \frac{\ell}{R}
\,,
\end{align}
%
and we know that \(k_B T_I = \frac{GM \overline{m}}{3 R_{\odot}}\): inserting this we find 
%
\begin{align}
  L = 4 \pi R^2 _{\odot}\sigma  \qty(\frac{GM \overline{m}}{3 R k_B})^{4} \frac{\ell}{R}
  = \frac{(4 \pi )^2}{3^{5}} \frac{\sigma }{k_B^4} G^{4} \overline{m}^{4} \overline{\rho} \ell M^3  
\,.
\end{align}

The most important part of this result, which is observationally verifiable, is \(L \propto M^{3}\). 
% What is the typical lifespan of a star?
This allows us to estimate the lifespan of a star: the fraction of mass available to be turned into energy through fusion is a constant multiple of \(M\) (slighly less than \SI{1}{\percent}): therefore, we have \(\tau \propto M/L \propto M^{-2}\). 
More massive stars die earlier.

The relation \(L \propto M^{\alpha } \) for \(\alpha = 3\) is quite close to the data globally, in specific regions we can have better estimates for the powerlaw index: the value \(\alpha = 3.5\) is more commonly used for Main Sequence stars (which we will discuss later). 
% This matches the data pretty well for . 

The lifetime always scales like \(\tau \propto M^{1 -\alpha }\), and we always have \(\alpha >1\), so the fact that more massive stars die earlier always holds.

We then expect that each galaxy there can have been several ``generations'' of heavy stars, while the very lightest are still in their first generation. This can be observationally confirmed by looking at the abundances of elements, and comparing them with the expected production inside heavy stars. 

\subsection{Thermonuclear fusion}

The reaction chain which produces Helium in stellar cores is the following:
%
\begin{subequations}
\begin{align}
  \ce{p + p} &\rightarrow \ce{d + e+ +} \nu_e  \\
  \ce{p + d} &\rightarrow \ce{^3He} + \gamma  \\
  \ce{^3He + ^3He} &\rightarrow \ce{^4He + p + p} 
\,,
\end{align}
\end{subequations}
%
which involves the weak interaction (for the first process, whose lifetime is \(\tau \sim \SI{5e9}{yr}\)), EM interaction (for the second process: \(\tau \sim \SI{1}{s}\)) and strong interaction (for the third process: \(\tau \sim  \SI{3e5}{yr}\)).

The net balance is 4 protons in, one \ce{^{4}He} and some photons and neutrinos out, accounting for the \(\approx \SI{0.66}{\percent}\) mass difference. 

We can then see that the weak-interaction part of the chain is the bottleneck.

The chain we had during primordial nucleosynthesis, on the other hand, did not need any weak-interaction processes:
%
\begin{subequations}
\begin{align}
  \ce{n + p} &\rightarrow \ce{d} + \gamma   \\
  \ce{d + d} &\rightarrow \ce{^3He + n}  \\
  \ce{^3He + d}  &\rightarrow \ce{^4He + p}
\,.
\end{align}
\end{subequations}

The issue is that all the free neutrons quickly decayed after primordial nucleosynthesis, and free deuterium is scarce: secondary production chains can then prevail. 
% However, later there are no more free neutrons: this means that even if it is slower the first process is the only one which can happen. 

% The net balance is 4 protons in, 1 \ce{^4 He} out. 

\end{document}
