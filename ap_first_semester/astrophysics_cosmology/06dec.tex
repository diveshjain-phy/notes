\documentclass[main.tex]{subfiles}
\begin{document}

\section*{Fri Dec 06 2019}

Today we will discuss another possible \emph{caveat} for star formation: the case where the star collapses but does not ignite. 

After recombination the baryons' temperature decays more rapidly than the radiation's, so we get a temperature low enough to satisfy the instability criterion. 

The temperature stays low during the collapse: the thermal energy is used in order to break the bonds in \ce{H2} and ionize the hydrogen. 

Then we have a gas which is opaque to radiation: there is scattering, which means we lose energy through radiation very slowly. 
Then the Virial theorem is very close to being true. 

At the end of the process, even the mass does not matter anymore: we got \SI{2.6}{eV} as the temperature regardless of the mass. This is equivalent to around \SI{30e3}{K}.  
We must compare this to the ignition temperature: that is in the order of \SI{}{keV} (\SI{15e6}{K} is equivalent to around \SI{1.3}{keV}). 

After free-fall the radius is on the order of \SI{e10}{m} for a solar mass star, while the Sun's radius is smaller by two orders of magnitude. 

Now we discuss the \emph{conditions for stardom}: we need to account for the fermionic nature of protons and electrons, which will give us a maximum density due to the Pauli exclusion principle. 

The way we will treat this today will be quite rough. 

We know that the De Broglie wavelength is given by 
%
\begin{align}
  \lambda = \frac{h}{p} = \frac{2 \pi \hbar}{p}
\,.
\end{align}

How do we calculate \(p\)? we assume that the particles are nonrelativistic and apply \(E_{k} = p^2 /2m_e\). 

Objects which will not satisfy the conditions we talk about today become brown dwarves. 

The kinetic energy is of the order \(k_B T\), therefore 
%
\begin{align}
  p \sim \sqrt{2 m_e k_B T}
\,,
\end{align}
%
and the critical density is defined by 
%
\begin{align}
  \rho_c = \frac{\overline{m}}{\lambda^3}
\,,
\end{align}
%
where from the formula we found 
%
\begin{align}
  \lambda = \frac{2 \pi \hbar}{\sqrt{2 m_e k_B T}}
\,,
\end{align}
%
which gives approximately 
%
\begin{align}
  \rho _c \sim \overline{m} \frac{(m_e k_B T)^{3/2}}{(2 \pi \hbar)^{3}}
\,,
\end{align}
%
and from the virial theorem \(2 E_k + E _{\text{gr}}=0\), with 
%
\begin{align}
  E_k = \frac{3}{2} N k_B T = \frac{3}{2} \frac{M}{\overline{m}} k_B T
\,,
\end{align}
%
while 
%
\begin{align}
  E _{\text{gr}} = - \frac{G M^2}{R}
\,,
\end{align}
%
which means 
%
\begin{align}
  3 N k_B T = \frac{GM^2}{R}
\,,
\end{align}
%
and we can rewrite this as 
%
\begin{align}
  \frac{3 k_B T}{\overline{m}} = \frac{GM}{R}
\,,
\end{align}
%
and we can express the mass as 
%
\begin{align}
  M = \frac{4}{3} \pi \overline{\rho} R^3
\,,
\end{align}
%
so 
%
\begin{align}
  \frac{1}{R} = \qty(\frac{4 \pi }{3} \frac{\overline{\rho}}{M})^{1/3}
\,,
\end{align}
%
which gives us the result 
%
\begin{align}
  k_B T = \frac{GM \overline{m}}{3} \qty(\frac{4 \pi }{3} \frac{\overline{\rho}}{M})^{1/3}
\,,
\end{align}
%
and if we substitute the critical density in for \(\overline{\rho}\) we will get the maximum possible temperature allowed at a given mass. 

This yields
%
\begin{align}
  k_B T = \frac{GM \overline{m}}{3} \qty(\frac{4 \pi }{3 M})^{1/3} \frac{\overline{m}^{1/3}}{(2 \pi \hbar)} \qty(m_e k_B T)^{1/2}
\,,
\end{align}
%
which it is convenient to square: 
%
\begin{align}
  (k_B T)^2 = \frac{G^2M^2 \overline{m}^2}{9}
  \qty(\frac{4 \pi }{3 M})^{2/3} \frac{\overline{m}^{2/3}}{(2 \pi \hbar)^2} m_e k_B T 
\,,
\end{align}
%
so we can simplify, and up to an order-1 constant 
%
\begin{align}
  k_B T = \frac{G^2 \overline{m}^{8/3} M^{4/3}}{(2 \pi \hbar)^2} 
\,,
\end{align}
%
and inserting the ignition temperature of around \SI{1}{keV} we get \(M _{\text{min}} \sim 0.08 M_{\odot}\). 

This is confirmed experimentally. 

Let us consider the Sun. Its mass is \(M_{\odot} \approx \SI{1.99e30}{kg}\), the radius is \(R_{\odot} \approx \SI{6.96e8}{m}\), the electromagnetic luminosity is \(L_{\odot} = \SI{3.86e26}{W}\). 

The age of the Sun is around \(t_{\odot} \approx \SI{4.55e9}{yr}\), which is comparable to the age of the Universe. 

The central density is \(\rho _c \approx \SI{1.48e5}{kg m^{-3}}\), while the central temperature is \(T_c = \SI{1.56e7}{K}\), and the central pressure is around \(P_c = \SI{2.29e16}{Pa}\).  

The effective temperature is around \(T_E \approx \SI{5780}{K}\). 
\todo[inline]{Definition?}

What is the corresponding free fall time? It is much shorter than the age of the Sun: the Sun is not in free fall. 
\todo[inline]{What is the number?}

We know that 
%
\begin{align}
  \expval{P} = -\frac{1}{3} \frac{E _{\text{gr}}}{V}
\,,
\end{align}
%
where \(E _{\text{gr}} = - G M^2/R\) while \(V = 4 \pi R^3/3\): plugging the Sun's numbers we get 
%
\begin{align}
  \expval{P} = \SI{e14}{Pa}
\,,
\end{align}
%
100 times less than the central density. 

The density of the Sun is actually very similar to the density of water. 

Was it correct to use nonrelativistic equations? (???)

%
\begin{align}
  \expval{P} = \frac{\overline{\rho}}{\overline{m}} k_B T_I
\,,
\end{align}
%
where \(T_I\) is the mean internal temperature of the Sun. 
The value of \(\overline{m} \approx 0.61\) instead of 0.5 when considering the proper composition of the Sun. 

We get 
%
\begin{align}
  k_B T_I \approx \frac{G M_{\odot} \overline{m}}{3 R_{\odot} } \approx \SI{1.5}{keV} \approx \SI{6e6}{K}
\,.
\end{align}

We have evidence that the Sun is a blackbody, we can write 
%
\begin{align}
  L_{\odot} = 4 \pi R_{\odot} \sigma T^{4}_{E}
\,,
\end{align}
%
where \(\sigma \) is Stefan's constant: \(\sigma \approx \SI{5.67e-8}{W m^{-2} K^{-4}}\). 

In principle we could define another quantity: 
%
\begin{align}
  L_{\odot}^{\prime } = 4 \pi R_{\odot}^2 \sigma T_I^4
\,,
\end{align}
%
which does not fit the data. Why is this? 

We must consider a photon which comes from the interior of the Sun and goes towards the outside: it will follow a random walk scattering many times. However, depending on the density of electron in the outer regions (electron scattering dominates in the outside), the last scattering is in the outermost regions of the star. 



\end{document}