\documentclass[main.tex]{subfiles}
\begin{document}

% \marginpar{Monday\\ 2019-11-07, \\ compiled \\ \today}
% \section*{Thu Nov 07 2019}

% We talk about inflation again.
% The comoving horizon increases with time: 
% %
% \begin{align}
%   d_H (t) = a(t) \int_0^t \frac{c \dd{\widetilde{t}}}{a(\widetilde{t})} \sim ct \sim \frac{c}{H} \equiv \text{Hubble horizon}
% \,,
% \end{align}
% %
% it is also called the \emph{past event horizon}.
% The reason why we can plot the history of the universe in a single spacetime diagram is because there is a well-defined transformation which brings an infinite interval to a finite one, using a hyperbolic arctangent: this gives us a \emph{Penrose diagram}.

% The comoving Hubble radius is \(r_H = c / aH = c/ \dot{a}\). This can grow.

% If we have positive pressure \(p>0\), than the scale factor goes like \(a \propto t^{2/(3(1+w))}\) with \(w>0\), then the comoving radius increases with time.

% We can actually still get increasing comoving radii with a weaker condition: \(p> -\frac{1}{3} \rho c^2\). This is the actual boundary (it can be checked looking at the derivative of \(a\)).

% This is directly connected to the sign of the acceleration in the Friedmann equation.

% If these conditions are always met, Hawking and Ellis proved that a Big Bang is inevitable. 

% The inflation hypothesis is that, even though now we have \(\ddot{a}<0\) now (or, it was so for some time: now the expansion seems to be accelerating), there was a period in which we had \(\ddot{a}>0\).

% These are drawn as straight lines, but it is only qualitative: we are looking at the sign of the slope.

% Then, there was a time in the past at which the comoving horizon was as large as it is now: now we will see how much inflation there must have been in order to solve the horizon problem up to now, ignoring the fact that now the universe's expansion is accelerating.

% The inequality we want to impose is 
% %
% \begin{align}
%   r_H (t_i) \geq r_H (t_0 )
% \,,
% \end{align}
% %
% where \(t_0 \) is now while \(t_i\) is the beginning of inflation.

% A sphere with comoving radius \(d_H (t_i)\) will expand after inflation up to 
% %
% \begin{align}
%     d_H (t_i) \frac{a (t_f)}{a (t_i)}
%     \,,
% \end{align}
% %
% where \(t_f\) is the time of the end of inflation. 
% %
% \begin{align}
%     d_H (t_i) \frac{a (t_f)}{a (t_i)} \geq d_H (t_0 ) \frac{a(t_f)}{a(t_0 )}
%     \,.
% \end{align}
% %
% We want to see what the limiting condition is. 
% %
% \begin{align}
%   Z _{\text{min}} = \frac{d_H (t_0 )}{d_H (t_i)} \frac{a_f}{a_0} = \frac{H_i}{H_0 } \frac{a_f}{a_0}
% \,,
% \end{align}
% %
% \todo[inline]{is \(Z\) a redshift?}
% %
% \begin{align}
%   z _{\text{min}} \frac{H_i}{H_0 } \frac{a_f}{a_0} = \frac{H_i}{H_f} \frac{H_f}{H_0 } \frac{a_f}{a_0 }
% \,,
% \end{align}
% %
% or 
% %
% \begin{align}
%   \frac{H_f}{H_i} z _{\text{min}} = \frac{H_f}{H_0} \frac{a_f}{a_0}
% \,,
% \end{align}
% %
% in which we can insert our solution to the Friedmann equations, for the scale factor and Hubble parameter in function of time: 
% %
% \begin{align}
%   H(t) = H_{*} \qty(\frac{a(t)}{a_{*}})^{-\frac{3(1+w)}{2}}
% \,.
% \end{align}

% This can be found using the results we found some time ago: the expressions for \(a\) and \(H\) were equal up to a different thing multiplying the parenthesis, and a different exponent.

% This is of course an approximation, but it works. A better number can be found by integrating numerically over different more realistic equations of state.
% We find: 
% %
% \begin{align}
%   z = \frac{a_f}{a_i} = 
% \,,
% \end{align}
% %
% \todo[inline]{Put earlier}


% %
% \begin{align}
%   \frac{H_f}{H_i} = z _{\text{min}}^{- \frac{3 (1+w)}{2}}
% \,,
% \end{align}
% %
% therefore we get 
% %
% \begin{align}
%   z _{\text{min}} = 
%   \qty(\frac{H_f}{H_0 } \frac{a_f}{a_0})^{- \frac{2}{(1+3w _{\text{inf}})}}
% \,,
% \end{align}
% %
% where \(w _{\text{inf}} \) is calculated at the time of matter-radiation equality.

% So we get: 
% %
% \begin{align}
%   \frac{H_f}{H_0} 
%   = \frac{H_f}{H _{\text{eq}}} \frac{H _{\text{eq}}}{H_0 } 
%   = \qty(\frac{a_f}{a _{\text{eq}}})^{-2} \qty(\frac{a _{\text{eq}}}{a_0 })^{-3/2} 
%   = \qty(\frac{a_f}{a_0 })^{-2} \qty(\frac{a_0}{a _{\text{eq}}})^{-1/2}
% \,,
% \end{align}
% %
% which means that the minimum inflation redshift must be 
% %
% \begin{align}
%   z _{\text{min}} = \qty(\qty(\frac{a_f}{a_0 })^{-1} \qty(\frac{a_0 }{a _{\text{eq}}})^{1/2} )^{\frac{-2}{1 + 3 w _{\text{inf}}}}
% \,,
% \end{align}
% %
% so the result can be expressed in terms of temperatures: 
% %
% \begin{align}
%   \frac{a_0}{a_f} = \frac{T_f}{T_0 } = \frac{T_f}{T _{\text{pl}}} \frac{T _{\text{pl}}}{T_0 }  
% \,,
% \end{align}
% %
% where the \(T _{\text{pl}}\) is the Planck temperature,
% and \(a_0 / a _{\text{eq}} = 1 + z _{\text{eq}}\): in the end our result is 
% %
% \begin{align}
%   z _{\text{min}} = \qty(\frac{T _{\text{pl}}}{T_0 } (1+ z _{\text{eq}}) \frac{T_f}{T _{\text{pl}}})^{- \frac{2}{1 + 3 w _{\text{inf}}}}
% \,.
% \end{align}
% %

% Recall that \SI{1}{GeV} is equal to \SI{e13}{K}, and \(T _{\text{pl}} = \SI{e19}{GeV}\).
% Also, \(1 + z _{\text{eq}} = \num{2.3e4} \Omega h^2\) 

% \todo[inline]{What are these units?}

% We get 
% %
% \begin{align}
%   z _{\text{min}} \approx \num{e30} \frac{T_f}{T _{\text{pl}}}
% \,,
% \end{align}
% %
% but what is the early universe temperature at the end of inflation? It must allow baryiogenesis, but will still be less than one, but there is an upper bound based on the fact that we have not observed primordial gravitational waves from this time: it must be at most \num{e-3}, so we find that the minimum redshift is of the order \(z _{\text{min}} \lesssim \num{e30} \sim e^{60}\), or 60 \(e\)-folds.

% This is an order of order of magnitude estimate.

% From the Friedmann equation we get 
% %
% \begin{align}
%   1 = \Omega (t) - \frac{k c^2}{a^2H^2}
% \,,
% \end{align}
% %
% so \(\Omega (t) - 1 = k r_H^2\). Now, consider the \(\Omega_i\) of inflation: we get 
% %
% \begin{align}
%   \frac{\Omega -1}{\Omega_i -1} = \qty(\frac{r_{H0}}{r_{Hi}})^2 < 1 
% \,.
% \end{align}

\subsection{Mechanisms for inflation}

% What we discuss now might be outside of our possibilities of comprehension.
Both the horizon problem and the flatness problem are addressed by the theory of inflation; up until now we have seen what inflation \emph{does}, but we still have to discuss \emph{how} it occurs. 
The way to properly describe it is through the language of Quantum Field Theory in curved spacetime, which surely cannot be introduced here; this section is meant to just give a flavor of the mechanism.

% Inflation is equivalent to \(\ddot{a} > 0 \), which is equivalent to \( p< - \rho/ 3 \), since the second Friedmann equation reads
% %
% \begin{align}
%   \ddot{a} = - \frac{8 \pi G}{3} \qty(\rho + 3P) a
% \,.
% \end{align}

% A quantum Hamiltonian for a harmonic oscillator is 
In regular Quantum Field Theory the Hamiltonian of our theory is often in the form (this example is for a real scalar field, with \(a ^\dag\) and \(a\) being its creation and annihilation operator and \(k\) being the momentum, with corresponding energy \(E_k = \sqrt{m^2 + k^2}\)):
%
\begin{align}
H = \int \dd[3]{k} E_k \frac{a ^\dag a + a a ^\dag}{2} = 
\int \dd[3]{k} E_k \underbrace{a ^\dag a }_{N} + \underbrace{\int \dd[3]{k} \frac{\omega_k}{2} \qty[a, a ^\dag]}_{ \to \infty }
\,,
\end{align}
%
meaning that it can be expressed in terms of an integral of the number operator times the energy, plus an integral which is constant and which diverges, since \(\qty[a, a ^\dag] = \delta^{(3)} (0)\) and the ground state of each harmonic oscillator in our continuum of possible values of the momentum is nonzero. 

As long as we are writing our theory in flat spacetime (and without the Einstein equations) this is not an issue: we are just adding a constant to the Hamiltonian, which does not affect the equations of motions which depend on its derivatives.  
In GR this is not the case: this energy \emph{gravitates}, as we have seen any energy density does. 

Is the ground-state energy just an artifact of the mathematical description of the fields? If so, we do not have a problem; unfortunately this is not the case, we can see this ground state energy through the \textbf{Casimir effect}.

If we put two metallic plates close to each other in a vacuum we can detect an attractive force between them, due to the fact that long-wavelength fluctuations do not ``fit'' in the gap between the plates, decreasing the energy density between the plates compared to the one outside, which is equivalent to the binding energy of an attractive force.\footnote{This was demonstrated to actually occur: the first group to do the experiment with the original parallel-plate configuration was in Padua \cite[]{bressiMeasurementCasimirForce2002}!}

\todo[inline]{An interesting digression, but I'm not sure whether it fits in this section. }

% In QFT, we either have scalars, vectors or spinors.
In QFT fields are classified based on how they transform under rotations: scalars do not change, vectors come back to themselves after a rotation of \(2 \pi \), spinors come back to themselves after a rotation of \(4 \pi \). 
In the standard model generally ``matter'' particles are spinors, while ``force'' particles are vectors. Scalars are rare: the only one is the Higgs boson.

We want our mechanism for inflation to be a field with a nonzero expectation value, which should also satisfy the symmetries of the FLRW  metric.
Our only option then is a scalar field which is homogeneous (and automatically isotropic since it does not define a direction), but which can change in time. 

A vector and a spinor both define a direction in space, and thus do not satisfy the requirement of isotropy. 
A term like \(\overline{\psi} \psi \), where \(\psi \) is a spinor and \(\overline{\psi}\) is its adjoint, can actually respect the required symmetries. We will not explore this further, but it gives rise to what is called a \emph{fermion condensate}. 

So, we will add a scalar field \(\Phi \) to our model. 

% Can a scalar field have a nonzero expectation value, while respecting the Robertson-Walker symmetries? Yes, we just take a function of time.

% For a vector, we cannot have nonzero expectation: a nonzero expectation value gives us a preferred direction.
% For a spinor, the same holds.

% However, an object like \(\overline{\psi} \psi \) behaves like a scalar, even though it comes from a vector.

% There are almost no scalar particles in nature!
% The only one is the Higgs field.

\paragraph{The Lagrangian formulation of GR}

Usually, an action for the Standard Model particles in a general-relativistic setting will have a term containing the derivatives of the metric, \(S_g\), and a term containing the actions of all the Standard Model particles, \(S _{\text{SM}}\):
%
\begin{align}
S = S_g + S_{\text{SM}}
\,.
\end{align}

As in classical mechanics, the equations of motion are derived from a variational principle, \(\delta S = 0\); however since \(S\) is a function of many fields this actually contains the equations of motion for each of them. Varying with respect to the SM fields yields their equations of motion, while varying with respect to the inverse metric \(g^{\mu \nu }\) yields the Einstein equations: 
%
\begin{align}
\underbrace{R_{\mu \nu } - \frac{1}{2} R g_{\mu \nu }}_{\propto\displaystyle \fdv{S_g}{g^{\mu \nu }}} 
= \underbrace{8 \pi G T_{\mu \nu }}_{\propto\displaystyle \fdv{S _{\text{SM}}}{g^{\mu \nu }}}
\,.
\end{align}

In this Lagrangian approach, the very \emph{definition} of the stress-energy tensor is as a certain multiple of the functional derivative of the action of the particles with respect to the inverse metric. 

\paragraph{Coupling between a field and gravity}

We update the action by adding a term for the field \(\Phi \):
%
\begin{align}
  S = S_{\Phi } + S_{g} + S_{\text{SM}}
\,.
\end{align}


As usual the action is found by integrating a Lagrangian density, but since we have a nontrivial metric we need to use the invariant volume element \(\dd[4]{x} \sqrt{-g}\) (which for FLRW is just \(\dd[4]{x} a^3\)): 
%
\begin{align}
  S = \int \dd[4]{x} \sqrt{-g} \mathscr{L}
\,.
\end{align}

The gravitational Lagrangian is given in terms of the Ricci scalar, \(\mathscr{L}_g = R /16 \pi G\), and varying it with respect to the metric yields the left-hand side of the Einstein equations.

A typical Lagrangian for a particle of mass \(m\) whose position is described by the coordinates \(q\) is given by a kinetic term and a potential term:
%
\begin{align}
  \mathscr{L}  = \frac{m}{2} \dot{q}^2 - V(q)
\,,
\end{align}
%
for a scalar field in Minkowski spacetime its equivalent would be 
%
\begin{align}
\mathscr{L}_\Phi  = \frac{1}{2} \qty(\partial_{\mu } \Phi ) \qty(\partial^{\mu } \Phi )
-V(\Phi )
\,.
\end{align}
%

In the GR case it is customary to be more explicit about the metric appearing in the kinetic term; also, the derivatives should in general become covariant ones, although in this case there is no change since the covariant derivative of a scalar is equal to its partial one: \(\nabla_\mu \Phi  = \partial_{\mu } \Phi \). 
The Lagrangian then becomes: 
%
\begin{align}
\mathscr{L} = \frac{1}{2} g^{\mu \nu } \partial_{\mu } \Phi \partial_{\nu } \Phi - V(\Phi )
\,.
\end{align}

The potential may include different terms, a common one is a mass term, which looks like \(V(\Phi ) = m \Phi^2 / 2\). 

If we add a massive term, proportional to \(R \Phi^2\), we get that adding it to the global action looks like gravity.
\todo[inline]{Clarify}

\end{document}
