\documentclass[main.tex]{subfiles}
\begin{document}

\section*{Fri Nov 08 2019}

We continue the discussion from yesterday on the dynamics of inflation.

The Lagrangian for a scalar field in GR is 
%
\begin{align}
  \mathscr{L} = \frac{1}{2} g^{\mu \nu } \nabla_{\mu } \Phi  \nabla_{\nu } \Phi - V(\Phi )
\,.
\end{align}

The ``contravariant derivative'' does not exist.

\todo[inline]{Why?}

We can add a term \(\xi R \Phi^2\), which has the right dimensions. Actions are dimensionless since \(\hbar  =1\), and since \(\dd{s^2} = g_{\mu \nu } \dd{x^{\mu }} \dd{x^{\nu }}\) the metric is also dimensionless. Therefore the dimensional analysis of \(\int \dd[4]{x} \sqrt{-g} \mathscr{L}\) gives us that \(\mathscr{L}\) must have dimensions of \(m^{-4}\).

The field \(\Phi \) has the dimensions of a mass, which is an inverse length. The coupling constants are conventionally taken to be dimensionless: therefore if we are to add a term to the Lagrangian, it must be \( \xi \Phi^2 \) times an inverse square length: so we can insert \(R\).

The value of \(\xi \) is undetermined: \(\frac[i]{1}{6} \) gives us conformal symmetry, while in other cases we get \(\frac[i]{1}{4} \). 
A Weyl transformation allows us to remove the additional term: we move from the Jordan frame (where we \emph{do} have coupling between our scalar field and the curvature) to the Einstein frame, in which we do not.

This is a prototype for modified GR theories.

Varying the Einstein-Hilbert action with respect to the metric gives the LHS of the Einstein equations. If we vary with respect to something else we get the equations of motion of that other thing.

This means that the stress-energy tensor is just the functional variation of everything but the action of the metric in the global acction, with respect to the metric.

We get for our scalar field: 
%
\begin{align}
  T_{\mu \nu } (\Phi ) = \Phi_{,  \mu } \Phi_{, \nu } - g_{\mu \nu } \qty(\frac{1}{2} g^{\rho \sigma } \Phi_{, \rho } \Phi_{, \sigma } - V(\Phi ))
\,,
\end{align}
%
and then \(G_{\mu \nu } = 8 \pi G T_{\mu \nu }\).

We can get an explicit solution by using the symmetries of our spacetime: we assume that \(\Phi (x^{\mu }) = \varphi (t)\).
This however in QFT is an operator, but we cannot have an operator on the RHS of the EFE: so we do a semiclassical theory, an equivalent of Hartree-Fock: we take an average in the vacuum state, and our equations become 
%
\begin{align}
  G_{\mu \nu } = 8 \pi G \expval{\hat{T}_{\mu \nu } }_{0}
\,,
\end{align}
%
where we define the ground state as that one with the most symmetry allowed.

The symmetries we must consider are only rotations and translations.
There are no issues of commutation, since we do not quantize space unlike the quantum loop gravity people.

If we perturb, we get \(\Phi = \varphi + \delta \Phi \): so \(\expval{\Phi^2 } = \varphi^2 + 2 \expval{\varphi \delta \Phi } + \expval{ \delta \Phi^2}\), but the second term is zero since \(\expval{ \delta \Phi } = 0 \) and \(\varphi \)is constant.
The last term in these diverges. We do not know how to deal with it.
We therefore assume that it is small.

So: when computing the stress energy tensor we get only diagonal terms: a perfect fluid!

The energy density is the Hamiltonian:
%
\begin{align}
  \rho = T_{00 } = \frac{1}{2} \dot{\varphi }^2 + V(\varphi ) = H
\,,
\end{align}
%
while the pressure is the Lagrangian: 
%
\begin{align}
  P = \frac{1}{2} \dot{\varphi}^2 - V(\varphi ) = \mathscr{L}
\,.
\end{align}

We call the stuff in the universe which is not our field ``radiation'', with energy density \(\rho_r\). 
%
\begin{subequations}    
\begin{align}
    H^2 &= \frac{8 \pi G }{3} \qty(\frac{1}{2} \dot{\varphi }^2 + V + \rho_r) \\
    \frac{\ddot{a}}{a} &= - \frac{8 \pi G }{3} \qty(\dot{\varphi }^2 - V + \rho_r) \\
    \dot{\rho} _{\text{tot}} &= - 3 \frac{\dot{a}}{a} \qty(\rho _{\text{tot}} + P _{\text{tot}})
    \,,
\end{align}
\end{subequations}
%
but in the continuity equation we can split the contributions by inserting an unknown factor \(\Gamma \), the transfer of energy between the field and radiation. 
%
\begin{align}
  \dot{\rho} _\varphi &=  - 3 \frac{\dot{a}}{a} \dot{\varphi }^2 + \Gamma \\  
  \dot{\rho} _r &=  - 4 \frac{\dot{a}}{a} \rho_r - \Gamma  
\,,
\end{align}
%
which, denoting \(' = \partial_\varphi \): 
%
\begin{align}
  \dot{\rho }_\varphi &= \dot{\varphi } \ddot{\varphi } + V^{\prime } \dot{\varphi } \\
  \ddot{\varphi} \dot{\varphi } + V^{\prime } \dot{\varphi } &= -3 \frac{\dot{a}}{a} + \Gamma 
\,,
\end{align}
%
but we drop \(\Gamma \) since we assume there is little radiation.

One solution is \(\dot{\varphi } =0\), if not: 
%
\begin{align}
  \ddot{\varphi } + 3 \frac{\dot{a}}{a} = - V^{\prime }
\,,
\end{align}
%
recall the definition of
%
\begin{align}
  w = - \frac{1}{3} = \frac{P}{\rho }
  = \frac{\frac{1}{2} \dot{\varphi }^2 - V}{\frac{1}{2} \dot{\varphi }^2 + V}
\,,
\end{align}
%
so one possibility we have is 
%
\begin{align}
  \dot{\varphi }^2 \gg 2 \abs{V} \implies w = 1
\,,
\end{align}
%
or else 
%
\begin{align}
\dot{\varphi }^2 \ll 2 \abs{V} \implies w = - 1
\,.
\end{align}

The continuity equation gives us the Klein-Gordon equation again: it is tautological.

\(\varphi = \const\) was one of the first solutions proposed.
This model seems so fit the data.

Several proposals were made in the late seventies, early eighties.

A very simple model for a symmetry-breaking potential is the Ginzburg-Landau:
%
\begin{align}
  V \propto \qty(\Phi^2 - \sigma^2)^2
\,,
\end{align}
%
which gives a seeming ``mass term'' \(- 2 \varphi^2 \sigma^2\), which has the wrong sign: it is ``tachyonic''!

The configuration at \(\Phi = 0\) is unstable. The one at \(\abs{\Phi } = \sigma \) is not symmetric under \(\Phi \rightarrow - \Phi \).

People realized that QFT is a subcase of a condensed matter approach in which we have a thermal bath, an \emph{environment}.
This is \emph{finite temperature QFT}.

We consider then an \emph{effective potential} for the temperature: \(V_T (\Phi ) = V(\Phi ) + \) functions of \(T\).
This might be \(V(\Phi ) + \alpha \varphi^2 T^2 + \gamma T^{4}\), with positive \(\alpha \). The quadratic term then gives us a \emph{positive} mass term: at temperatures larger than some critical temperature we get stability at \(\Phi = 0\), but what happens if we lower the temperature?

Then, there is symmetry breaking.

Let us see how our Friedmann equations account for this situation.

The temperature of radiation is \(\rho _r  = \frac{\pi^2}{30 } g_{*} (r) T^{4}\). If we start with a universe which is radiation dominate, then it ends up to be De Sitter.

This is a consequence of the \emph{No Hair Cosmic Theorem}.

There is a potential barrier between the metastable ``\(\Phi = 0\)'' state, and the symmetry breaking other ones.
(even though it does not show in the fourth degree potential model).

This can happen through quantum tunneling, but there is a delay: a \emph{first order phase transition with supercooling} (by ``super'' what is meant is just that the temperature goes below \(T_C\) even though we still are in the center symmetric state).

We get bubbles of symmetry broken by fluctuation, expanding through the universe but never meeting because of the expansion.

This is the ``old inflation model''.

A new inflation model involves ``slow rolling''.

The equation \(\ddot{\varphi } + 3 H \dot{\varphi } = - V'\) looks like a regular equation of motion: after a time \(1/H\) the ``friction'' velocity-dependent term dominates.

Then we get a slow-roll regime: \(H^2  = \frac{8 \pi G}{3} V \) and \(\dot{\varphi} \approx - V^{\prime }/3H\).

We exploit the flatness of the potential.
There are quantum fluctuations during inflation.

The solution is \emph{chaotic inflation}, by Linde 1984:
Since \(\Delta E \Delta t \approx \hbar\), we do not know at which state we are actually. 
As time passes, the energy uncertainty decreases.
The initial condition for the distribution of the universe is then detemined by the uncertainty principle.

An alternative is \emph{eternal} chaotic inflation.
If a fluctuation increases the potential universe, then \(H^2\) increases, then the region feels a larger volume. The case where the field goes towards the minimum is unlikely.
Why did it happen? This can only be answered with the anthropic principle.

\end{document}