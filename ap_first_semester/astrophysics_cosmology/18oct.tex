\documentclass[main.tex]{subfiles}
\begin{document}

% \section*{Fri Oct 18 2019}
\marginpar{Friday\\ 2010-10-18, \\ compiled \\ \today}

% From yesterday: other consequences of inserting \(\Lambda \).
\subsection{Evolution of a dark energy dominated universe}

% Now we have a different approach from Einstein's: we insert the constant not in order to get a static universe but just as a measurable parameter of our theory. 
In order to find out how this parameter affects the universe's expansion, we consider a universe in which the only fluid behaves like the cosmological constant. 
So, we take the first Friedmann equation \eqref{eq:friedmann-1-cosmological-constant} in the absence of ordinary matter (\(\rho = 0\)) and with negligible spatial curvature (\(k= 0\)). This yields:

\begin{equation}
  \qty(\frac{\dot{a} }{a})^2 = \frac{\Lambda}{3}
\,.
\end{equation}

This is actually a good approximation for the asymptotic state of the universe, since the cosmological constant term is the only one which does not decay with the scale factor (and so, with time).

The solution to this differential equation is, as we mentioned in section \ref{sec:solutions-to-friedmann-equations}:
%
\begin{equation}
  a(t) = a_{*} \exp(\sqrt{\frac{\Lambda}{3}} \qty(t-t_{*})) 
\,,
\end{equation}
%
which can also be written as \(a \propto e^{Ht}\), since \(H = \dot{a} / a = \sqrt{\Lambda / 3}\).
This is called a steady-state solution, since the Hubble parameter is constant. 
It is also called a \emph{de Sitter} solution: it belongs to the maximally symmetric 4D spacetime solutions to the Einstein Equations: Minkowski, de Sitter and Anti de Sitter: the latter has \(\Lambda < 0\), the former has \(\Lambda >0\).

This actually seems to model the observed expansion of the universe well, and until recently it competed with the Big Bang theory. 

The fraction of the cosmic fluid which behaves like dark energy is bound to increase with time, since as we saw it is the only component which does not decrease in density over time. 

% The two other solutions are also called \emph{de Sitter} and look like cosines (?): they can be mapped into each other.

% \todo[inline]{How?}/

This is expressed formally using the so-called \emph{no-hair cosmic theorem}, which is actually a conjecture if it is meant to describe the universe: it states that asymptotically only the dark energy contribution is relevant: all the matter and everything else is forgotten. 
In order to interpolate between the current --- matter dominated, or in which at least matter has a sizeable contribution --- universe and the asymptotic one we can use a solution in the form
%
\begin{equation}
  a \propto \qty(\sinh(At))^{2/3}
\,,
\end{equation}
%
where we define \(2A/3 = \sqrt{\Lambda /3} \), since the hyperbolic sine is asymptotically close to an exponential.

\section{Curved models}

We seek solutions to the Friedmann equations for nonzero spatial curvature \(k\), for a universe containing nonrelativistic matter (\(w=0\)) without dark energy.
We make these assumptions since with them we can find an analytic solution. 

We can rewrite the two independent Friedmann equations as 
%
\begin{align}
  \dot{a}^2 &=  \frac{8 \pi G}{3} \rho a^2 - k  \\
  \rho  &= \rho_0 \qty(\frac{a}{a_0 })^{-3}
\,,
\end{align}
%
and now we will solve them with \(k = \pm 1\).

\medskip

\paragraph{Solutions to parametric ODEs}
In general, for an ODE like \(y = f(y')\) for the function \(y=y(x)\) with \(f'\) continuous we introduce \(y' \equiv p\), assuming  \(p \neq 0\): then \(y = f(p)\), which implies 
%
\begin{equation}
  y' = \dv{f}{p} p'
  \marginnote{Differentiated both sides of \(y = f(p)\)}
\,,
\end{equation}
%
which we can manipulate to get
%
\begin{equation}
  p = \dv{f}{p} p' \implies \dv{x}{p} = \frac{1}{p} \dv{f}{p}
\,,
\end{equation}
%
so we can get the solution by integration: we get an expression for \(x \) in terms of \(p\), which we will be able to invert since by assumption \(p' \neq 0\): so, we get
%
\begin{equation}
  x = \int   \frac{1}{p} \dv{f}{p}\dd{p}
  \qquad \text{and} \qquad
  y = f(p)
\,.
\end{equation}

\medskip

We use this for our problem: our differential equation looks like 
%
\begin{align}
\dot{a}^2 &= \frac{8 \pi G}{3} \rho a^2 - k  \\
\dot{a}^2 &= \frac{8 \pi G}{3} \rho_0 \frac{a_0^3}{a^3}a^2 - k \marginnote{Substituted \(\rho = \rho_0 a_0^3 / a^3\) from the third Friedmann equation.} \\
\dot{a}^2 &=  A a^{-1} - k
\,,
\end{align}
%
where we defined \(A \equiv 8 \pi G a_0^3 \rho_0 /3\).
We can rewrite this as 
%
\begin{equation} \label{eq:curved-models-general-ODE}
  a = \frac{A}{p^2+k} = f(p)
  \qquad \text{where} \qquad
  p = \dot{a} 
\,.
\end{equation}

Then, using the general formula we get: 
%
\begin{align}
t &= \int \frac{1}{p} \dv{f}{p} \dd{p} 
\qquad \text{where} \qquad
\dv{f}{p} = -\frac{2Ap}{(p^2+k)^2}  \\
&= \int \frac{-2A}{(p^2+k)^2} \dd{p}
\,.
\end{align}

\subsection{Positive curvature: a closed universe}

If \(k = +1\), then we can make the substitution \(p = \tan(\theta ) \), which is helpful since \(1 + p^2= \sec^2 \theta \); for the change of variable we have \(\dd{p} = \dd{\theta } \sec^2 \theta  \).
So, for the time we find: 
%
\begin{align}
  t &= -2A \int \frac{\sec^2\theta \dd{\theta }}{\sec^4\theta } \\
  &= -2A  \int \cos^2(\theta ) \dd{\theta }  = -A \qty(\theta + \sin(\theta ) \cos(\theta ) ) + \const
\,,
\end{align}
%
and we can apply the trigonometric identity \(\sin(\theta ) \cos(\theta ) = \sin(2 \theta ) \):
%
\begin{equation}
  t = -\frac{A}{2} \qty(2 \theta  + \sin(2 \theta ) ) + \const
\,.
\end{equation}

Now we can define \(2 \theta  = \pi - \alpha \), which allows for the simplification \(\sin(2\theta ) = \sin(\alpha )  \); also, we can express \(p=\tan \theta \) in terms of \(\alpha \).
This gives us 
%
\begin{align}
t = \frac{A}{2} \qty(\alpha - \sin(\alpha ) ) + \const
\qquad \text{and} \qquad
p = \tan(\frac{\pi}{2} - \frac{\alpha}{2})
\marginnote{Absorbed factor \(-A \pi /2\) into the constant}
\,.
\end{align}
%

We almost have our solution: inserting \(p(\alpha )\) into the main equation for \(a\) \eqref{eq:curved-models-general-ODE} we get
%
\begin{align}
  a &= \frac{A}{1 + \tan^2(\pi /2 - \alpha /2)} = A \cos^2 \qty(\frac{\pi}{2} - \frac{\alpha}{2})  
  \marginnote{Used \(1 + \tan^2x = 1/\cos^2x\).}
  \\
  &= \frac{A}{2} \qty(1+\cos(\pi - \alpha ) ) = \frac{A}{2} \qty(1- \cos(\alpha ) )
  \marginnote{Used \(\cos^2(x/2) = (1+\cos x) /2 \) and \(\cos x = - \cos(\pi -x)\).}
\,,
\end{align}
which should be complemented with the equation we found for \(t\): in the end, our solution looks like 
%
\begin{align}
t &= \frac{A}{2} \qty(\alpha - \sin \alpha ) + \const  \\
a &= \frac{A}{2} \qty(1 - \cos \alpha )
\,,
\end{align}
%
so, in order to interpret this physically we fix \(t=0 \iff \alpha = 0\), which sets ``const'' to zero, and we reinsert the constants\footnote{
We wish to express the constant \(A / 2\) in term of observables such as \(H_0 \) and \(\Omega_0 = \rho_0  / \rho_{0c}\). 
} we get
%
\begin{equation}
  a = a_0 \frac{\Omega_0}{2 (\Omega_0 -1)} \qty(1- \cos(\alpha ) ) 
\,,
\end{equation}
%
and
%
\begin{equation}
  t = \frac{1}{H_0 } \frac{\Omega_0}{2 (\Omega_0 -1)^{3/2}}(\alpha - \sin(\alpha ) )
\,,
\end{equation}
%
but now we switch from \(\alpha \) to \(\theta \) for historical reasons.

[Plot: \(a(\theta )\) vs \(\theta \)].

We have \(\dot{a} > 0 \)   when \(0 \leq \theta \leq \theta_m = \pi \), while \(\dot{a} < 0\) when \(\theta _m \leq \theta \leq 2 \pi \).
This is the \emph{turn-around} angle. The angles \(0\) and \(2 \pi \) correspond to the Big Bang and the Big Crunch.

At \(\theta _m\) we have: 
%
\begin{equation}
  a = a_0 \frac{\Omega_0}{\Omega_0 -1} 
\,,
\end{equation}
%
and 
%
\begin{equation}
  t = \frac{\pi}{2 H_0 } \frac{\Omega_0}{(\Omega_0 -1)^{3/2}}
\,,
\end{equation}
%
therefore if we set \(a=1\) at the current time we get 
%
\begin{equation}
  t_0 = \frac{1}{2H_0 } \frac{\Omega_0  }{(\Omega_0 -1)^{3/2}} \qty(\arccos (\frac{2}{\Omega_0 }-1) - \frac{2}{\Omega_0 } (\Omega_0 -1 )^{1/2})
\,,
\end{equation}
%
In a pure matter model we would have \(a \propto t^{2/3}\), which would imply \(H_0 = 2/(3 t_0 )\) or \(t_0 = 2 / (3 H_0 )\): this does not make sense! It would give us an age of the universe of the order \SI{e9}{yr}, while the measured age of the universe is \SI{1.4e10}{yr}.
This is for a flat universe.

For a given \(H_0 \), we would have a smaller \(t_0 \) with a closed universe.

\subsection{Negative curvature: an open universe}

For \(k = -1\) we do exactly the same steps with hyperbolic functions instead of trigonometric ones: we get 
%
\begin{equation}
  a ( \psi ) = a \frac{\Omega_0 }{2 (1-\Omega_0)} \qty(\cosh \psi -1)
\,,
\end{equation}
%
with \(\cosh\psi = 2/\Omega_0 -1\) and 
%
\begin{equation}
  t (\psi ) =  \frac{1}{H_0 } \frac{\Omega_0}{2 (\Omega_0 -1)^{3/2}}(\sinh \psi - \psi  )
\,.
\end{equation}
%

The same reasoning as before gives us a \(t_0 > 2 / (3 H_0 )\)! This is then more attractive.

Radiation's energy density \(\rho _r (a)\) can be seen as a function of the redshift: \(\rho _r (z ) = \rho _{0r} (1+z)^{4}\), since \((1+z) = a_0 / a\).

For matter \(\rho _m (z) = \rho_{0m} (1+z)^{3}\), for \(\Lambda \) instead \(\rho _\Lambda (z) = \rho _{0 \Lambda }\).

Let us define a moment called the \emph{equality redshift} \(z _{\text{eq}}\). This is when \(\rho _r (z _{\text{eq}}) = \rho _m (z _{\text{eq}})\). This means that 
%
\begin{equation}
  (1+z _{\text{eq}}) = \frac{\rho _{0,m}}{\rho _{0,r}} 
  = \frac{\Omega_{0,m}}{\Omega_{0,r}}
\,,
\end{equation}
%
where we divided and multiplied by the critical density.

We know that \(\Omega_{0,m}\) is around \num{0.3}, while for the radiation it would be easier to measure the density.
Neutrinos are matter now but they were radiation at the time of equality (???).

Accounting for everything, we think that 
%
\begin{equation}
  1 + z _{\text{eq}} \simeq \num{2.3e4} \Omega_{0, m} h^2
\,.
\end{equation}

This means that the recombination of electrons and protons into Hydrogen happened when the universe was already \emph{matter dominated}.

Another time is \(z_\Lambda \), defined by: \(\rho _m (z_{\Lambda }) = \rho _\Lambda (z_\Lambda )\). 
%
\begin{equation}
  1+ z_\Lambda = \qty(\frac{\rho_{0, \Lambda }}{\rho_{0,m}})^{1/3} \simeq \qty(\frac{0.7}{0.3})^{1/3}
\,,
\end{equation}
%
which implies that \(z_\Lambda \approx \num{0.33}\).

\chapter{The thermal history of the universe}

The model which won out is the \emph{hot Big Bang} model.

Consider radiation: we know from Stefan's law that \(\rho _r \propto T^{4}\), while \(\rho _r \propto a^{-4}\). Therefore we would expect \emph{Tolman's law} to hold: \(T \propto 1/a\).

By \emph{total number of galaxies} we mean the galaxies in our past light cone.

When (in natural units) we get temperature of the order of a certain elementary particle, then statistically that type of particle will usually be ultra-relativistic.

The number density of particles is: 
%
\begin{equation}
  n = \frac{g}{(2 \pi )^3} \int \dd[3]{q} f(\vec{q})
\,,
\end{equation}
%
in units where \(c= \hbar = k_B = 1\). The parameter \(g\) is the number of helicity states, \(q\) is the three-momentum. The function \(f(\vec{q})\) is a pdf in phase space.
We will have to integrate  over position and momentum: the metric we will get in the end must not depend on anything but time, by the cosmological principle.

There is no general rule for \(g\): for photons, we only have two spin states; the ``rule'' \(g = 2s +1 \) is not actually applied, for photons \(\vec{s} = 0\) is unphysical, for gravitons \(\abs{\vec{s}} \leq 1 \) is unphysical.
\(g\) accounts for all internal degrees of freedom: for atoms we also have vibration, rotation\dots

The energy density is 
%
\begin{equation}
  \rho = \frac{g}{(2 \pi )^3} \int \dd[3]{\vec{q}} E(q) f(\vec{q}) 
\,,
\end{equation}
%
where \(E^2 = q^2 + m^2\). For photons \(E = q\), for nonrelativistic particles \(E \approx m + q^2 / 2m\).

The adiabatic pressure is 
%
\begin{equation}
  P = \frac{g}{(2 \pi )^3} \int \dd[3]{\vec{q}} \frac{q^2 f(\vec{q})}{3E} 
\,,
\end{equation}
%
which comes from a consideration of the diagonal components \(T_{ii}\) of the stress energy tensor of particles. Alternatively, we can say that this comes from imposing \(\dd{E} = P \dd{V} + \dd{Q} \).

This definition gives us \(P = \rho /3\) for photons directly.

In full generality the distribution is 
%
\begin{equation}
  f(\vec{q}) = \qty(\exp(\frac{E- \mu }{T}) \pm 1 )^{-1}
\,,
\end{equation}
%
where we have a plus for fermions, and a minus for bosons. Here, \(\mu \) is the chemical potential: is becomes relevant when the gas becomes hot and dense.
It can be introduced as a Lagrange multiplier for changes in number of particles.

The Planck distribution is given by: 
%
\begin{equation}
  f_k(\vec{q}) = \qty(\exp(\frac{q}{T}) -1 )^{-1}
\,,
\end{equation}
%
since they are bosons with no chemical potential:

In general we can say that if for some species we have the reaction \(i+j \leftrightarrow k+l\), then \(\mu _i + \mu _j = \mu _k + \mu _l\). We can deduce them by the known relations: for example, from the annihilation of electron and positron we can derive \(\mu _{e^{+}} = - \mu_{e^{-}}\).
This rule is not trivial; it is an ansatz of thermodynamical equilibrium to get a solution of the Boltzmann equation which allows us to write the Saha equations.

\end{document}