\documentclass[main.tex]{subfiles}
\begin{document}

\section*{Fri Oct 18 2019}

From yesterday: other consequences of inserting \(\Lambda \).
Now we have a different approach from Einstein's: we insert the constant not in order to get a static universe but just as a measurable parameter of our theory. 
%
\begin{equation}
  \qty(\frac{\dot{a} }{a})^2 = \frac{8 \pi G }{3} \rho +\frac{\Lambda}{3} - \frac{k}{a^2}
\,,
\end{equation}
%
with \(\rho = \rho_0 (a_0 / a)^3\). The more the universe expands, the more the cosmological constant term dominates, since the other terms are inversely dependent on \(a\).

The asymptotic state, neglecting spatial curvature, gives us a steady-state solution: 
%
\begin{equation}
  a(t) \propto \exp(\sqrt{\frac{\Lambda}{3}} t) 
\,.
\end{equation}

This is called a \emph{de Sitter} solution. An alternative is found if we \emph{do} consider the spatial curvature, with \(k = \pm 1\). The two other solutions are also called \emph{de Sitter} and look like cosines (?): they can be mapped into each other.

\todo[inline]{How?}

These belong to the maximally symmetric solutions: Minkowski, dS and Anti de Sitter: the latter has \(\Lambda < 0\).

\todo[inline]{In AdS, do we have \(a \propto \exp(i k t) \)?}

The \emph{no-hair cosmic theorem} is actually a conjecture: it states that asymptotically only the dark energy contribution is relevant: all the matter and everything else is forgotten. 
%
\begin{equation}
  a \propto \qty(\sinh(At))^{2/3}
\,,
\end{equation}
%

where we define \(2A/3 = \sqrt{\Lambda /3} \) is a solution which \emph{interpolates} between the current --- matter dominated --- universe and the asymptotic one, since the hyperbolic sine is locally a simple exponential.

We can rewrite the Friedmann equation as 
%
\begin{equation}
  \dot{a}^2 = \frac{8 \pi G}{3} \rho a^2 - k
\,,
\end{equation}
%
and now we will solve it with \(k = \pm 1\).
In general, for an ODE like \(y = f(y')\) with \(f'\) continuous we introduce \(y' \equiv p\) with \(p \neq 0\): then \(y = f(p)\), which implies 
%
\begin{equation}
  y' = \dv{f}{p} p'
\,,
\end{equation}
%
and then 
%
\begin{equation}
  p = \dv{f}{p} p' \implies \dv{x}{p} = \frac{1}{p} \dv{f}{p}
\,,
\end{equation}
%
which gives by integration the solution: 
%
\begin{equation}
  x = \int  \dd{p} \frac{1}{p} \dv{f}{p}
  \qquad \text{with} \qquad
  y = f(p)
\,.
\end{equation}

Using this for our problem, we get \(\dot{a}^2 = A a^{-1} - k \), where \(A \equiv 8 \pi G a_0^3 \rho_0 /3\). We can rewrite this as 
%
\begin{equation}
  a = \frac{A}{p^2+k} 
  \qquad \text{where} \qquad
  p = \dot{a} 
\,.
\end{equation}

Then we get: 
%
\begin{equation}
  \dot{a} = p = -2A  p \dot{p} \frac{1}{(p^2+k)^2}
\,,
\end{equation}
%
and using our formula 
%
\begin{equation}
  t = -2A \int \frac{\dd{p} }{(p^2+k)^2}
\,.
\end{equation}

In order to go forward, we distinguish the cases: it \(k = +1\), then \(p = \tan(\theta ) \), therefore \(1 + p^2= \sec^2 \theta \) which implies \(\dd{p} = \dd{\theta } \sec^2 \theta  \).

For the time: 
%
\begin{equation}
  t = \int -2A \dd{\theta } \cos^2(\theta ) = -A \qty(\theta + \sin(\theta ) \cos(\theta ) ) + \const
\,,
\end{equation}
%
and we apply the trigonometric identity \(\sin(\theta ) \cos(\theta ) = \sin(2 \theta ) \):
%
\begin{equation}
  t = -\frac{A}{2} \qty(2 \theta  + \sin(2 \theta ) ) + \const
\,,
\end{equation}
%
now we can define \(2 \theta  = \pi - \alpha \), therefore \(\sin(2\theta ) = \sin(\alpha )  \): this gives us
\(t = A/2 (\alpha - \sin(\alpha ) )\) and \(p = 1/\tan(\alpha /2) = \tan(\pi /2 - \alpha /2) \).

So we almost have our solution: 
%
\begin{equation}
  a = \frac{A}{1 + \tan^2(\pi /2 - \alpha /2)} = A \cos^2(\frac{\pi}{2} - \frac{\alpha}{2})  = \frac{A}{2} \qty(1+\cos(\pi - \alpha ) ) = \frac{A}{2} \qty(1- \cos(\alpha ) )
\,,
\end{equation}
which should be complemented with the equation we found for \(t\).

Reinserting the constants we have:
%
\begin{equation}
  a = a_0 \frac{\Omega_0}{2 (\Omega_0 -1)} \qty(1- \cos(\alpha ) ) 
\,,
\end{equation}
%
and
%
\begin{equation}
  t = \frac{1}{H_0 } \frac{\Omega_0}{2 (\Omega_0 -1)^{3/2}}(\alpha - \sin(\alpha ) )
\,,
\end{equation}
%
but now we switch from \(\alpha \) to \(\theta \) for historical reasons.

[Plot: \(a(\theta )\) vs \(\theta \)].

We have \(\dot{a} > 0 \)   when \(0 \leq \theta \leq \theta_m = \pi \), while \(\dot{a} < 0\) when \(\theta _m \leq \theta \leq 2 \pi \).
This is the \emph{turn-around} angle. The angles \(0\) and \(2 \pi \) correspond to the Big Bang and the Big Crunch.

At \(\theta _m\) we have: 
%
\begin{equation}
  a = a_0 \frac{\Omega_0}{\Omega_0 -1} 
\,,
\end{equation}
%
and 
%
\begin{equation}
  t = \frac{\pi}{2 H_0 } \frac{\Omega_0}{(\Omega_0 -1)^{3/2}}
\,,
\end{equation}
%
therefore if we set \(a=1\) at the current time we get 
%
\begin{equation}
  t_0 = \frac{1}{2H_0 } \frac{\Omega_0  }{(\Omega_0 -1)^{3/2}} \qty(\arccos (\frac{2}{\Omega_0 }-1) - \frac{2}{\Omega_0 } (\Omega_0 -1 )^{1/2})
\,,
\end{equation}
%
In a pure matter model we would have \(a \propto t^{2/3}\), which would imply \(H_0 = 2/(3 t_0 )\) or \(t_0 = 2 / (3 H_0 )\): this does not make sense! It would give us an age of the universe of the order \SI{e9}{yr}, while the measured age of the universe is \SI{1.4e10}{yr}.
This is for a flat universe.

For a given \(H_0 \), we would have a smaller \(t_0 \) with a closed universe.

For \(k = -1\) we do exactly the same steps with hyperbolic functions instead of trigonometric ones: we get 
%
\begin{equation}
  a ( \psi ) = a \frac{\Omega_0 }{2 (1-\Omega_0)} \qty(\cosh \psi -1)
\,,
\end{equation}
%
with \(\cosh\psi = 2/\Omega_0 -1\) and 
%
\begin{equation}
  t (\psi ) =  \frac{1}{H_0 } \frac{\Omega_0}{2 (\Omega_0 -1)^{3/2}}(\sinh \psi - \psi  )
\,.
\end{equation}
%

The same reasoning as before gives us a \(t_0 > 2 / (3 H_0 )\)! This is then more attractive.

Radiation's energy density \(\rho _r (a)\) can be seen as a function of the redshift: \(\rho _r (z ) = \rho _{0r} (1+z)^{4}\), since \((1+z) = a_0 / a\).

For matter \(\rho _m (z) = \rho_{0m} (1+z)^{3}\), for \(\Lambda \) instead \(\rho _\Lambda (z) = \rho _{0 \Lambda }\).

Let us define a moment called the \emph{equality redshift} \(z _{\text{eq}}\). This is when \(\rho _r (z _{\text{eq}}) = \rho _m (z _{\text{eq}})\). This means that 
%
\begin{equation}
  (1+z _{\text{eq}}) = \frac{\rho _{0,m}}{\rho _{0,r}} 
  = \frac{\Omega_{0,m}}{\Omega_{0,r}}
\,,
\end{equation}
%
where we divided and multiplied by the critical density.

We know that \(\Omega_{0,m}\) is around \num{0.3}, while for the radiation it would be easier to measure the density.
Neutrinos are matter now but they were radiation at the time of equality (???).

Accounting for everything, we think that 
%
\begin{equation}
  1 + z _{\text{eq}} \simeq \num{2.3e4} \Omega_{0, m} h^2
\,.
\end{equation}

This means that the recombination of electrons and protons into Hydrogen happened when the universe was already \emph{matter dominated}.

Another time is \(z_\Lambda \), defined by: \(\rho _m (z_{\Lambda }) = \rho _\Lambda (z_\Lambda )\). 
%
\begin{equation}
  1+ z_\Lambda = \qty(\frac{\rho_{0, \Lambda }}{\rho_{0,m}})^{1/3} \simeq \qty(\frac{0.7}{0.3})^{1/3}
\,,
\end{equation}
%
which implies that \(z_\Lambda \approx \num{0.33}\).

\section{The thermal history of the universe}

The model which won out is the \emph{hot Big Bang} model.

Consider radiation: we know from Stefan's law that \(\rho _r \propto T^{4}\), while \(\rho _r \propto a^{-4}\). Therefore we would expect \emph{Tolman's law} to hold: \(T \propto 1/a\).

By \emph{total number of galaxies} we mean the galaxies in our past light cone.

When (in natural units) we get temperature of the order of a certain elementary particle, then statistically that type of particle will usually be ultra-relativistic.

The number density of particles is: 
%
\begin{equation}
  n = \frac{g}{(2 \pi )^3} \int \dd[3]{q} f(\vec{q})
\,,
\end{equation}
%
in units where \(c= \hbar = k_B = 1\). The parameter \(g\) is the number of helicity states, \(q\) is the three-momentum. The function \(f(\vec{q})\) is a pdf in phase space.
We will have to integrate  over position and momentum: the metric we will get in the end must not depend on anything but time, by the cosmological principle.

There is no general rule for \(g\): for photons, we only have two spin states; the ``rule'' \(g = 2s +1 \) is not actually applied, for photons \(\vec{s} = 0\) is unphysical, for gravitons \(\abs{\vec{s}} \leq 1 \) is unphysical.
\(g\) accounts for all internal degrees of freedom: for atoms we also have vibration, rotation\dots

The energy density is 
%
\begin{equation}
  \rho = \frac{g}{(2 \pi )^3} \int \dd[3]{\vec{q}} E(q) f(\vec{q}) 
\,,
\end{equation}
%
where \(E^2 = q^2 + m^2\). For photons \(E = q\), for nonrelativistic particles \(E \approx m + q^2 / 2m\).

The adiabatic pressure is 
%
\begin{equation}
  P = \frac{g}{(2 \pi )^3} \int \dd[3]{\vec{q}} \frac{q^2 f(\vec{q})}{3E} 
\,,
\end{equation}
%
which comes from a consideration of the diagonal components \(T_{ii}\) of the stress energy tensor of particles. Alternatively, we can say that this comes from imposing \(\dd{E} = P \dd{V} + \dd{Q} \).

This definition gives us \(P = \rho /3\) for photons directly.

In full generality the distribution is 
%
\begin{equation}
  f(\vec{q}) = \qty(\exp(\frac{E- \mu }{T}) \pm 1 )^{-1}
\,,
\end{equation}
%
where we have a plus for fermions, and a minus for bosons. Here, \(\mu \) is the chemical potential: is becomes relevant when the gas becomes hot and dense.
It can be introduced as a Lagrange multiplier for changes in number of particles.

The Planck distribution is given by: 
%
\begin{equation}
  f_k(\vec{q}) = \qty(\exp(\frac{q}{T}) -1 )^{-1}
\,,
\end{equation}
%
since they are bosons with no chemical potential:

In general we can say that if for some species we have the reaction \(i+j \leftrightarrow k+l\), then \(\mu _i + \mu _j = \mu _k + \mu _l\). We can deduce them by the known relations: for example, from the annihilation of electron and positron we can derive \(\mu _{e^{+}} = - \mu_{e^{-}}\).
This rule is not trivial; it is an ansatz of thermodynamical equilibrium to get a solution of the Boltzmann equation which allows us to write the Saha equations.

\end{document}