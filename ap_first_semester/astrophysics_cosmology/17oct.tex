\documentclass[main.tex]{subfiles}
\begin{document}

\section*{Thu Oct 17 2019}

If we neglect spatial curvature, which is small, we can write the luminosity distance as an integral which we can compute:
%
\begin{equation}
  d_L \equiv \qty(\frac{L}{4 \pi \ell})^{1/2}
\,.
\end{equation}
%

Our metric is the FLRW line element. Then, we can write \(d_L\) as: 
%
\begin{equation}
  d_L = a_0 (1+z) r(z)
\,.
\end{equation}
%

We now define the \emph{conformal time} \(\tau\): we want to impose \(a^2(\tau ) \dd{\tau^2} = \dd{t^2}  \).
Then, the RW line element becomes: 
%
\begin{equation}
  \dd{s^2} = a^2(\tau ) \qty(c^2 \dd{\tau^2} - \frac{\dd{r^2} }{1 - k r^2} - r^2 \dd{\Omega^2} )
\,.
\end{equation}

This is very important when we talk about zero-mass particles, with no intrinsic length scale.
Using the variable \(\chi\), we have: 
%
\begin{equation}
  \dd{s^2}  = a^2(\tau ) \qty(c^2 \dd{\tau^2} - \dd{\chi^2} - f^2_k (\chi ) \dd{\Omega^2})
\,,
\end{equation}
%
where \(f_k(\chi )=r\) is equal to \(\sin(\chi ) \), \(\chi \) or \(\sinh(\chi )\) if \(k\) is equal to \(1\), \(0\) or \(-1\).

If we look at photons moving radially, we get 
%
\begin{equation}
  \dd{s^2} = 0 = a^2(\tau ) \qty(c^2 \dd{\tau^2} - \dd{\chi^2})
\,,
\end{equation}
%
therefore \(c^2 \dd{\tau^2} = \dd{\chi^2}\): setting \(c=1\), we get \(\tau (t_0 ) - \tau (t_e) = \chi (r_e) - \chi (0)\), where a subscript \(e\) means ``emission''. 

\begin{equation}
  \dd{\tau} = \frac{\dd{t} }{a}  = \frac{\dd{a} }{a \dot{a} }
\,,
\end{equation}
%
and now recall \(a = a_0 /(1+z)\): differentiating this we get
%
\begin{equation}
  \dd{a} = - \frac{a_0 }{(1+z)^2} \dd{z}
\,,
\end{equation}
%
\begin{equation}
  \frac{\dd{a}}{a^2} = \frac{\dd{a} (1+z)^2}{a^2} = - \frac{\dd{z}}{ a_0 }
\,,
\end{equation}
%
which means 
%
\begin{equation}
  \dd{e} = - \frac{\dd{z}}{a_0 H(z)}
  \,.
\end{equation}

\todo[inline]{??? probabily there is wrong stuff here}

The Hubble parameter is given by 
%
\begin{equation}
  H^2= \frac{8 \pi G}{3} \rho - \frac{k c^2}{a^2}
\,,
\end{equation}
%
with density \(\rho (t) = \rho _r (t) + \rho _m (t) + \rho _\Lambda \), where the first term scales like \(a^{-4}\), the second \(a^{-3}\), the third is constant. Thus they scale like \((1+z)^{4}\), \((1+z)^{3}\) and so on.

Then we can write a law for the evolution of \(H(z) = H_0 E(z)\).
Recall the definitiion of \(\Omega (t)\): we can look at the \(\Omega_i (t)\) for \(i\) corresponding to matter, radiation and so on: 
%
\begin{equation}
  \Omega _i (z) = \frac{8 \pi G \rho _i (t)}{3 H^2(z)}
  = \frac{8 \pi G }{3 H^2} \frac{\rho _i (z)}{E^2(z)}
  \overset{\text{def}}{=} \Omega _{i,0} \frac{(1+z)^{\alpha }}{E^2(z)}
\,.
\end{equation}

In the case of radiation, \(p = \rho c^2 /3\), and then \(\alpha = 4\). 

For matter \(p=0\): \(\alpha = 3\).

For the cosmological constant \(\Lambda \): \(\alpha =0\).

For spatial curvature we have \(\alpha = 2\).

For the \(\Omega )\) corresponding to the curvature we define: \(\Omega _k = - t c^2/(a^2 H^2)\).

We must have 
%
\begin{equation}
  1 = \Omega _r + \Omega _m + \Omega _\Lambda + \Omega _k
\,.
\end{equation}

Recall the definition of \(E^2(z)\): 
%
\begin{equation}
  E^2(z) = \frac{H^2}{H_0^2} = 
  \Omega _{k, 0} + \Omega_{r, 0} (1+z)^4 + \Omega_{m, 0} (q+z)^{3}
\,.
\end{equation}

and to get \(E\) we just take the square root.
We have \(\tau (t_0 ) - \tau (t_e) = \chi (r_e)\).
Integrating: 
%
\begin{equation}
  \chi (r) = c \int_{a_t}^{a} \frac{\dd{a}  }{a \dot{a} }
  = \int \frac{\dd{z'} }{E(z')}
\,,
\end{equation}
%

therefore 
%
\begin{equation}
  r = f_k \qty(\frac{c}{a_0 H_0 } \int_0^z \frac{\dd{z'}   }{E(z')})
\,.
\end{equation}
%

Two weeks ago we defined the luminosity distance: now we can compute it.

\todo[inline]{How do we decide which \(k\) to use?}

Now,  suppose we are looking at a certain far-away object with angular size \(\Delta \theta \): we fix \(r\) in the RW line element, and look at a constant time: then we get a linear size corresponding to the angular one of 
%
\begin{equation}
  \dd{s}  = a(t) r \Delta \theta 
\,,
\end{equation}
%
which, when divided by \(\Delta \theta \), is called angular diameter \(D_A = a r = a_0 r(z) / (1+z)\).
This changes with distance... (?) 

\begin{equation}
 d_L = a_0 (1+z) r
\,,
\end{equation}
%
then 
%
\begin{equation}
  \frac{d_L}{d_A} = (1+z)^2
\,.
\end{equation}

Einstein thought that the universe had to be static.

Recall the Friedmann equations \eqref{eq:friedmann-equations}. Now, if we look at static solutions for matter (\(p=0\)): the third equation becomes the identity, the derivatives of \(a\) are zero: therefore the second equation gives us \(\rho \equiv 0\): there cannot be matter.

Now we know that the universe is neither static nor stationary.

Einstein modified his equations in order to get a static non-empty solution.

The Einstein equations read 
%
\begin{equation}
  G_{\mu \nu } = 8 \pi G T_{\mu \nu }
\,,
\end{equation}
%
when \(c=1\), where the Einstein tensor \(G_{\mu \nu }\) can be defined with 
%
\begin{equation}
  G_{ \mu \nu } = R_{\mu \nu } - \frac{1}{2} g_{\mu \nu }R
\,.
\end{equation}

Einstein added a term \(- \Lambda g_{\mu \nu }\) to the LHS of the Einstein equations. This is allowed since 

\begin{enumerate}
    \item it is symmetric;
    \item it has zero covariant divergence, since \(\Lambda \) is constant while \(\nabla_{\mu } g^{\mu \nu } = 0\).
\end{enumerate}

Then, we can rewrite the EE with a modified stress-energy tensor, to which we add \(\Lambda g_{\mu \nu } / 8 \pi G\). Comparing this to an ideal fluid tensor 
%
\begin{equation}
  T_{\mu \nu } = \left[\begin{array}{cccc}
  \rho & 0 & 0 & 0 \\ 
  0 & p & 0 & 0 \\ 
  0 & 0 & p & 0 \\ 
  0 & 0 & 0 & p
  \end{array}\right]
\,,
\end{equation}
%
we get \(\rho \rightarrow \rho + \Lambda/ 8 \pi G \) and \(p \rightarrow p + \Lambda  / 8 \pi G\).

Inserting this into the Friedmann equations we get: 
%
\begin{equation}
  \qty(\frac{\dot{a} }{a})^2= 
  \frac{8 \pi G}{3} \rho + \frac{\Lambda }{3}
  - \frac{k}{a^2}
\,,
\end{equation}
%

%
\begin{equation}
  \frac{\ddot{a}}{a} = - \frac{4 \pi G}{3} \rho + \Lambda 
\,,
\end{equation}
%
while in the third equation, in the contribution \(\widetilde{\rho }  + \widetilde{p} \) the two \(\Lambda \) terms cancel. Then for the first equation we get 
%
\begin{equation}
  \frac{8 \pi G}{3} \rho + \frac{\Lambda}{3} = \frac{k}{a^2}
\,,
\end{equation}
%
and for the second: 
%
\begin{equation}
  4 \pi G \rho = \Lambda 
\,.
\end{equation}

Then, 
%
\begin{equation}
  \Lambda \qty(\frac{1}{3}+ \frac{2}{3}) = \Lambda = \frac{k}{a^2}
\,,
\end{equation}
%
and we want a solution with \(k=1\), \(\Lambda > 0\).

A candidate for the cosmological constant term is the vacuum energy in QFT: however the estimate given there is around \num{e120} times off.

The next topic is a solution of the Friedmann equations.
We will try to do it with \(p=0\): \(\rho \propto a^{-3}\).

\end{document}
