\documentclass[main.tex]{subfiles}
\begin{document}

% \section*{Thu Oct 17 2019}

\marginpar{Thursday \\ 2019-10-17, \\ compiled \\ \today}

\section{Measuring distances} 

We want to be able to compute the comoving radius, given our knowledge of the evolution of the distribution of energy density in time.

% We will neglect spatial curvature. 
We have shown that the luminosity distance is given by:
%
\begin{equation}
  d_L \equiv \sqrt{\frac{L}{4 \pi \ell}}
  = a_0 (1+z) r(z)
\,.
\end{equation}

Also recall \emph{conformal time} \(\eta\), which is defined by its relation to cosmic time, \(a(\eta ) \dd{\eta} = \dd{t}\): it allows us to write the FLRW metric as
%
\begin{equation}
  \dd{s^2} = a^2(\eta ) \qty(c^2 \dd{\eta^2} - \frac{\dd{r^2} }{1 - k r^2} - r^2 \dd{\Omega^2} )
\,.
\end{equation}

This is very important when we talk about zero-mass particles, with no intrinsic length scale: the photon, which is our primary tool for astrophysical observations, is one of these.
This can be written in terms of the variable \(\chi\): 
%
\begin{equation}
  \dd{s^2}  = a^2(\eta ) \qty(c^2 \dd{\eta^2} - \dd{\chi^2} - f^2_k (\chi ) \dd{\Omega^2})
\,,
\end{equation}
%
where \(f_k(\chi )=r\) is equal to \(\sin(\chi ) \), \(\chi \) or \(\sinh(\chi )\) if \(k\) is equal to \(1\), \(0\) or \(-1\); in other words we either have \(\chi = \arcsin(r)\), \(\chi = r\) or \(\chi = \operatorname{arcsinh}(r)\).

If we look at photons moving radially we do not need to account for the angular part, and we find
%
\begin{equation}
  \dd{s^2} = 0 = a^2(\eta ) \qty(c^2 \dd{\eta^2} - \dd{\chi^2})
\,,
\end{equation}
%
therefore \(c^2 \dd{\eta^2} = \dd{\chi^2}\):  we get \(c \qty(\eta (t_0 ) - \eta (t_e)) = \chi (r_e) - \chi (r_0 )\), where a subscript \(e\) means ``emission'', while a subscript \(0\) means detection. 
We are choosing the negative sign when simplifying the square, since the problem we are considering is that of radiation starting from an astrophysical source and coming towards us: its radial coordinate \(\chi \) decreases when the temporal coordinate \(\eta \) increases. 

This means that we can find out the comoving distance \(\Delta \chi \) between two events by calculating the difference between their comoving times \(\Delta \eta \).
This is what was meant by the fact that this expression of the metric is useful for massless particles: the scale factor gets factored out, we can write the expression in a very simple way.

\begin{equation}
  \dd{\eta} = \frac{\dd{t} }{a}  = \frac{\dd{a} }{a \dot{a} }
\,,
\end{equation}
%
and now recall \((1+z) = a_0 / a\): we differentiate this with respect to time to find
%
\begin{equation}
  \dv{z}{t} = - \frac{a_0 }{a^2} \dot{a} = - \frac{a_0 H(z)}{a}
\,,
\end{equation}
%
% \begin{equation}
%   \frac{\dd{a}}{a^2} = \frac{\dd{a} (1+z)^2}{a^2} = - \frac{\dd{z}}{ a_0 }
% \,,
% \end{equation}
% %
% which means 
% %
% \begin{equation}
%   \dd{e} = - \frac{\dd{z}}{a_0 H(z)}
%   \,.
% \end{equation}
which means 
%
\begin{align}
\dd{\eta } = \frac{ \dd{t}}{a} = - \frac{ \dd{z}}{a_0 H(z)}
\marginnote{Took the inverse of the equation, split the differentials, used the definition of \(\eta \)}
\,,
\end{align}
%
so we get our final expression: 
%
\begin{align}
\dd{ \chi  } = \frac{c \dd{z}}{a_0 H(z)} \marginnote{Used the fact that \(\dd{\chi } = - c \dd{\eta }\).}[.2cm]
\,.
\end{align}

So, if we can find a way to parametrize the Hubble parameter \(H(z)\) in terms of the redshift we will be able to measure distances.
% \todo[inline]{??? probabily there is wrong stuff here}

The Hubble parameter is given by 
%
\begin{equation}
  H^2= \frac{8 \pi G}{3} \rho - \frac{k c^2}{a^2}
\,,
\end{equation}
%
where the density comes from several components: \(\rho (t) = \rho _r (t) + \rho _m (t) + \rho _\Lambda \), where the first term is the density of radiation and scales like \(a^{-4}\), the second is the density of matter and scales like \(a^{-3}\), the third is the density of dark energy and is constant.

In terms of the redshift, they scale like \((1+z)^{4}\), \((1+z)^{3}\) (and \((1+z)^{0}\)) respectively.

We express the Hubble parameter as a multiple of its value now: \(H(z) = H_0 E(z)\), where \(E(z)\) is an adimensional function.

Recall the definition of \(\Omega (t)\): it describes the ratio of the density of a certain type of fluid to the critical density. We can look at the \(\Omega_i (t)\) for \(i\) corresponding to matter, radiation and so on: 
%
\begin{equation}
  \Omega _i (z) = \frac{8 \pi G \rho _i (z)}{3 H^2(z)}
  = \frac{8 \pi G \rho_{i}(z=0)}{3 H_0 ^2} \times  \frac{\rho _i (z) / \rho_{i}(z=0)}{E^2(z)}
  = \Omega _{i,0} \frac{(1+z)^{\alpha }}{E^2(z)}
\,,
\end{equation}

where \(\alpha \) is the exponent of the scaling of the fluid: \(\alpha = 4\) for radiation, \(\alpha = 3\) for matter, \(\alpha = 0\) for the cosmological constant \(\Lambda \), while for spatial curvature \(\alpha = 2\).

\todo[inline]{
For the \(\Omega \) corresponding to the curvature we define: \(\Omega _k = - k c^2/(a^2 H^2)\).
Again, not sure how to interpret this: what is the EOS of spatial curvature?}

We must have 
%
\begin{equation}
  1 = \Omega _r + \Omega _m + \Omega _\Lambda + \Omega _k
\,.
\end{equation}

We can write an expression for \(E^2(z)\) by taking the ratio of the densities at emission versus now:
%
\begin{equation}
  E^2(z) = \frac{H^2}{H_0^2} = 
  \Omega _{\Lambda , 0} +  \Omega_{m, 0} (1+z)^{3}
  +\Omega_{r, 0} (1+z)^4
\,,
\end{equation}
%
and to get \(E\) we just take the square root.

Now we can finally compute our integral 
%
\begin{equation}
  \chi (z) = \frac{c}{a_0 H_0 } \int_{0}^{z} \frac{\dd{z'} }{E(z')}
\,,
\end{equation}
%

therefore 
%
\begin{equation}
  r = f_k \qty(\frac{c}{a_0 H_0 } \int_0^z \frac{\dd{z'}   }{E(z')})
\,.
\end{equation}
%

This does depend on \(k\), but the differences between positive and negative curvature are only relevant starting from third order.
If the curvature is zero, we get the comoving distance:
%
\begin{align}
d_C = r a_0 = \frac{c}{H_0 } \int_{0}^{z} \frac{ \dd{z'}}{E(z')}
\,.
\end{align}

If the curvature is not zero, we can still define a useful distance: the \emph{transverse comoving distance}, 
%
\begin{align}
d_M = a_0 r = a_0 f_k \qty(\frac{c}{a_0 H_0 } \int_0^z \frac{\dd{z'}   }{E(z')})
\,;
\end{align}
%
for \(k=0\) these two coincide.

\todo[inline]{Right? I didn't write this down but it seems natural.}

% Two weeks ago we defined the luminosity distance: now we can compute it.

Now,  suppose we are looking at a certain far-away object with angular size \(\Delta \theta \) and linear size \emph{at emission} of \(\Delta x\): then the \emph{angular diameter distance} is given, in the small-angle approximation, by
%
\begin{equation}
d_A = \frac{ \Delta x}{\Delta \theta } = a(t_e) r
= \frac{a_0 r_z}{1+z} = \frac{d_M}{1+z}
\,.
\end{equation}

Since the luminosity distance is given by 
%
\begin{equation}
 d_L = a_0 (1+z) r = d_M (1+z)
\,
\end{equation}
%
their ratio is  
%
\begin{equation}
  \frac{d_L}{d_A} = (1+z)^2
\,.
\end{equation}

\todo[inline]{This part is kind of confused, I think the concepts are well explained but there is some reorganization to do.}

\todo[inline]{Add summary of cosmological distances, with explanation of their physical meaning, drawing from \cite{hoggDistanceMeasuresCosmology2000} and \cite{davisExpandingConfusionCommon2004}.}

\section{The cosmological constant} \label{sec:cosmological-constant}

Einstein thought that the universe had to be static: it was a common notion at the time that it should be, almost a philosophical principle.\footnote{An interesting historical fact: this was corroborated by a calculation error on Einstein's part, which was later pointed out by Friedmann. Einstein thought \cite{einsteinCommentFriedmannPaper1922} that \(\nabla_{\mu } T^{\mu \nu }= 0\) implied \(\partial_{t} \rho =0\), while Friedmann pointed out \cite{friedmannAlexanderFriedmann1922} that the correct equation reads \( \partial_{t} \qty(\sqrt{-g} \rho ) =0\): the density of the universe is not forced to be time independent if the determinant of the metric changes accordingly. Even the best make mistakes.}
Now we know that the universe is neither static nor stationary.\footnote{The distinction between static and stationary is subtle but significant \cite{ludvigsenGeneralRelativityGeometric1999}: \emph{stationarity} is about the existence of a timelike Killing vector, while \emph{staticity} is about the timelike Killing vector being orthogonal to spacelike submanifolds. 
A concrete example: Schwarzschild geometry is both static and stationary, Kerr geometry is stationary but not static, FLRW geometry is neither, since there is no timelike Killing vector field.}

So, he sought static solutions (\(a = \const\)) for matter (\(P=0\)) to the Friedmann equations \eqref{eq:friedmann-equations}: if we set \(\dot{a} = \ddot{a} = 0\) 
the third equation becomes \(\dot{\rho}= 0\), the second equation gives us \(\rho \equiv 0\), and from the first we must also have \(k=0\): the only way to have a static matter-filled universe is for the density of matter to be zero, and for the spatial curvature to be also zero.

In order to satisfy what he thought was an empirical fact, Einstein modified his equations in order to get a static non-empty solution.

The Einstein equations read 
%
\begin{equation}
  G_{\mu \nu } = 8 \pi G T_{\mu \nu }
\,,
\end{equation}
%
when \(c=1\), where the Einstein tensor \(G_{\mu \nu }\) can be defined in terms of the Ricci curvature tensor \(R_{\mu \nu }\) and the scalar curvature \(R\) as:
%
\begin{equation}
  G_{ \mu \nu } = R_{\mu \nu } - \frac{1}{2} g_{\mu \nu }R
\,.
\end{equation}

This peculiar construction is the only one which can be made in terms of the curvature tensor and which is covariantly constant: \(\nabla_{\mu } G^{\mu \nu } = 0\). 
This is a necessary condition since \(\nabla_{\mu } T^{\mu \nu }=0\): the Einstein equations state that they are proportional, so if we take the covariant derivative of the equations we must get the identity \(0=0\).

% \begin{bluebox}
% Here, \(R = g^{\mu \nu } R_{\mu \nu }\) is the scalar curvature, while \(R_{\mu \nu } = R^{\rho }_{\mu \rho \nu }\) is the Ricci tensor, and finally \(R^{\mu }_{\nu \rho \sigma }\) is the Riemann tensor, defined in terms of the affine connection \(\Gamma^{\mu }_{\nu \rho }\) as 
% %
% \begin{align}
% R^{\mu }_{\nu \rho \sigma } = -2 \qty(\Gamma^{\mu }_{\nu [\rho , \sigma ]} + \Gamma^{\alpha }_{\nu [\rho } \Gamma^{\mu }_{\sigma ] \alpha })
% \,,
% \end{align}
% %
% where commas denote coordinate differentiation. The connection \(\Gamma^{ \mu }_{\nu \rho } \) 
% \end{bluebox}

Einstein added a term \(- \Lambda g_{\mu \nu }\) to the LHS of the Einstein equations, with \(\Lambda \) a constant scalar. This is allowed since 

\begin{enumerate}
  \item it is tensorial (since it is a scalar multiple of the metric, which is a tensor);
  \item it is symmetric;
  \item it has zero covariant divergence, since \(\Lambda \) is constant and the metric is covariantly constant \(\nabla_{\mu } g^{\mu \nu } = 0\).
\end{enumerate}

Then, we can rewrite the EE in two equivalent ways: either 
%
\begin{align}
\widetilde{G}_{\mu \nu } &= 8 \pi G T_{\mu \nu } &\text{with}&& \widetilde{G}_{\mu \nu } &= G_{\mu \nu } - \Lambda g_{\mu \nu }  \\
G_{\mu \nu } &= 8\pi G \widetilde{T}_{\mu \nu } &\text{with}&& 
\widetilde{T}_{\mu \nu } &= T_{\mu \nu } + \frac{\Lambda g_{\mu \nu }}{8 \pi G} 
\,.
\end{align}
%

In the first interpretation, the cosmological constant is an intrinsic geometric property of spacetime; in the second interpretation cosmological constant is a particular kind of fluid, with the property of its contribution to the stress-energy tensor always being a constant multiple of the metric. 

In order to find out what the properties of this fluid are, w compare its stress-energy tensor to a generic ideal fluid tensor: 
%
\begin{equation}
T_{\mu \nu }^{\text{(generic)}} = \left[\begin{array}{cccc}
\rho & 0 & 0 & 0 \\ 
0 & P & 0 & 0 \\ 
0 & 0 & P & 0 \\ 
0 & 0 & 0 & P
\end{array}\right]
\qquad \qquad
T^{(\Lambda )}_{\mu \nu } = \frac{\Lambda g_{\mu \nu }}{8 \pi G} = \frac{\Lambda}{8 \pi G} \left[\begin{array}{cccc}
1 & 0 & 0 & 0 \\ 
0 & -1 & 0 & 0 \\ 
0 & 0 & -1 & 0 \\ 
0 & 0 & 0 & -1
\end{array}\right]
\,,
\end{equation}
%
so the corrections to the stress energy tensor must be \(\rho \rightarrow \rho + \Lambda/ 8 \pi G \) and \(P \rightarrow P - \Lambda  / 8 \pi G\), or, in other words, the density and pressure of the ``cosmological constant fluid'' are \(\rho_{\Lambda } = - P_{\Lambda } = \Lambda/8\pi G\).
This proves that the equation of state of the cosmological constant is \(w = -1\).

Inserting this into the Friedmann equations we get: 
%
\begin{align}
  \qty(\frac{\dot{a} }{a})^2 &= 
  \frac{8 \pi G}{3} \rho + \frac{\Lambda }{3}
  - \frac{k}{a^2} \label{eq:friedmann-1-cosmological-constant}\\
  \frac{\ddot{a}}{a}  &= - \frac{4 \pi G}{3} \rho + \Lambda \label{eq:friedmann-2-cosmological-constant}\\
 \dot{\rho} &= -3 \frac{\dot{a}}{a }\qty(\widetilde{\rho} + \widetilde{P}) = -3 \frac{\dot{a}}{a} \qty(\rho + P)\label{eq:friedmann-3-cosmological-constant}
\,,
\end{align}
%
and we can see that in the third equation, the effect of the source is encompassed in a term \(\widetilde{\rho }  + \widetilde{P} \): the two \(\Lambda \) terms cancel, since they are opposite.
For a cosmological constant-dominated universe --- that is, for a universe in which the only fluid behaves like the cosmological constant -- we have \(\dot{\rho} = \dot{P} = 0\). 

So, proceeding with the derivation by Einstein, we set \(\dot{a}= \ddot{a} =0 \): for the first Friedmann equation we get 
%
\begin{equation}
  \frac{8 \pi G}{3} \rho + \frac{\Lambda}{3} = \frac{k}{a^2}
\,,
\end{equation}
%
and for the second: 
%
\begin{equation}
  4 \pi G \rho = \Lambda 
\,.
\end{equation}

So, we substitute the expression for \(4 \pi G \rho \) into the first Friedmann equation: 
%
\begin{equation}
  \frac{2}{3} \qty(4 \pi G \rho ) + \frac{\Lambda}{3} =
  \Lambda \qty(\frac{1}{3}+ \frac{2}{3}) = \Lambda = \frac{k}{a^2}
\,.
\end{equation}

What are the physical conclusions to draw? 
Since we want matter in the universe we must have \(\rho >0\), which implies \(\Lambda >0\), which implies \(k =1\): so the universe must be closed. 

Friedmann studied perturbations around this solution and found it to be unstable: so, it is not suitable as a description of the universe.
This, combined with the observations by Hubble of an expanding universe, prompted the scientific community to discard the idea of a stationary universe in favor of an expanding one.

Einstein probably \cite{autInvestigatingLegendEinstein2018} called the introduction of the cosmological constant into the equation his ``greatest blunder''; however in modern cosmology the idea of a cosmological constant has gained new vigor: we observe the universe's expansion to be accelerated, that is \(\ddot{a} > 0\), and the only way for this to be the case if \(\rho >0\) is if \(\Lambda > 0\) as well. 
It is the only kind of fluid which has a repulsive gravitational effect.

As opposed to the approach by Einstein, in which the cosmological constant was inserted to stationarize the universe, we make it a measurable parameter of our theory.

A candidate for the cosmological constant term, which is a kind of intrinsic energy of space, is the vacuum energy in QFT: however the estimate we get when trying to make this quantitative is around \num{e120} times the measured value of \(\Lambda \).

% The next topic is a solution of the Friedmann equations.
% We will try to do it with \(p=0\): \(\rho \propto a^{-3}\).

\end{document}
