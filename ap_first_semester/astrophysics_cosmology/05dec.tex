\documentclass[main.tex]{subfiles}
\begin{document}

\marginpar{Thursday\\ 2019-12-05, \\ compiled \\ \today}
% \section*{Thu Dec 05 2019}

% There will be a section on gravitational waves in cosmology in January. 

% We found the relation: 
% %
% \begin{align}
%   \vec{r}  = a(\lambda ) \vec{x}
% \,,
% \end{align}
% %
% and 
% %
% \begin{align}
%   \vec{\dot{r}} = \vec{u} = \dot{a} \vec{x} + a \dot{\vec{x}} = H \vec{r} + \vec{v}
% \,,
% \end{align}
% %
% since \(H = \dot{a} / a\). 

% An issue we have is the \emph{redshift space distortion}, caused by \emph{peculiar velocities} which cause the light we see to have redshift beyond the one caused by cosmology alone. 

% In comoving coordinates, we can use the regular Newtonian fluid dynamics equations. 

% Since we know that the gravitational instability is relevant when the gravitational force overcomes the pressure forces, we set the pressure gradient to be equal to zero. We have 
% %
% \begin{subequations}
% \begin{align}
%   \pdv{\rho }{t} |_{\vec{r}} + \nabla_{\vec{r}} \qty(\rho \vec{u}) &= 0  \\
%   \dv{\vec{u}}{t} + \qty(\vec{u} \cdot \vec{\nabla_{\vec{r}}}) \vec{u} &=0  \\
%   \nabla_{\vec{r}}^2 \Phi &= 4 \pi G \rho 
% \,,
% \end{align}
% \end{subequations}
% %
% and we consider a background \(\Phi_b (\vec{r}, t)\), and \(\rho = \overline{\rho}(t)\): we have a peculiar gravitational field beyond the Robertson-Walker background: we define 
% %
% \begin{align}
%   \phi  = \Phi - \Phi_b
% \,,
% \end{align}
% %
% and analogously we define by 
% %
% \begin{align}
%   \delta \rho = \rho - \overline{\rho} = \overline{\rho} \qty(1 + \delta )
% \,,
% \end{align}
% %
% with \(-1 \leq \delta \leq \infty\). 

% The Laplace equation for \(\phi \) is given by 
% %
% \begin{align}
%   \nabla^2_{\vec{x}} \phi (\vec{x}, t) = 4 \pi G a^2(t) \overline{\rho} (t) \delta (\vec{x}, t)
% \,,
% \end{align}
% %
% and we actually \emph{can} have negative quantities on the RHS. Not repulsive gravity, but kind of. 

% The relation between the derivatives is 
% %
% \begin{align}
%   \pdv{}{r} = \frac{1}{a } \pdv{}{x}
% \,,
% \end{align}
% %
% therefore the continuity equation becomes:
% %
% \begin{align}
%   \pdv{\rho }{t} + 3H \rho + \frac{1}{a} \nabla_{x} \qty(\rho \vec{v}) = 0
% \,,
% \end{align}
% %
% we get that the density scales like \(a^{-3}\) if there is no velocity: then it becomes 
% %
% \begin{align}
%   \frac{\dot{\rho}}{\rho } + 3 \frac{\dot{a}}{a} =0 
% \,.
% \end{align}
% %

% For the momentum equation the calculation is longer, but in the end we find: 
% %
% \begin{align}
%   \pdv{\vec{v}}{t} + H \vec{v} +
%   \frac{1}{a} \qty(\vec{v} \cdot \nabla_{x}) \vec{v}
%   = - \frac{1}{a} \nabla_{x} \phi 
% \,,
% \end{align}
% %
% and once again we remark that we are not using GR because, in the absence of symmetry, that is definitely too difficult. 

% If we perturb, we will find that plane waves are not solutions anymore, because we will have time-dependent coefficients. 

% The perturbation of the density is: 
% %
% \begin{align}
%   \rho (\vec{x}, t) = \overline{\rho}(t) \qty(1 + \delta (\vec{x}, t))
% \,,
% \end{align}
% %
% and for the field: 
% %
% \begin{align}
%   \phi (\vec{x}, t) = \Phi - \Phi_b
% \,.
% \end{align}

\paragraph{Solving the equations}

The derivative of the density, which appears in the continuity equation, reads:
%
\begin{align}
  \pdv{\rho }{t} = \pdv{\rho_b}{t} \qty(1 + \delta ) + \rho_b \pdv{ \delta }{t}
\,.
\end{align}

The procedure we will then apply is to simplify the equation making use of the fact that it holds at zeroth order (with all the perturbations set to zero): we have \(\partial_{t} \rho _b + 3 H \rho_b = 0\).
% and we can discard higher-order terms: simplifying the zeroth-order equation we find for the  continuity equation: 

Also, we neglect second and higher order terms (recall that \(\vec{v}\) is already first order): the computation goes like
%
\begin{align}
  \pdv{\rho_b}{t} \qty(1 + \delta ) + \rho_b \pdv{ \delta }{t}
  + 3 H \rho_b (1 + \delta ) + \frac{1}{a } \vec{\nabla} \cdot (\rho_b (1 +\delta ) v) &= 0 \\
  (1 + \delta ) \qty[\pdv{\rho _b}{t} + 3 H \rho _b] 
  + \rho _b \pdv{ \delta }{t} + \frac{\rho _b}{a} \vec{\nabla} \cdot \vec{v} &= 0 \\
  \pdv{ \delta }{t} + \frac{1}{a} \vec{\nabla} \cdot \vec{v} &= 0
\,.
\end{align}

On the other hand, for the momentum equation we only need to neglect the \((\vec{v} \cdot \vec{\nabla}) \vec{v}\) term, which is second order, to find:
%
\begin{align}
  \pdv{\vec{v}}{t} + H \vec{v} = - \frac{1}{a} \vec{\nabla} \phi 
\,,
\end{align}
%
where \(\phi = \delta \Phi \) is the perturbation 

Finally, the Poisson equation is already in its simplest, linear form.

In order to solve these we expand in Fourier space: the density perturbation \(\delta \) is expressed in terms of its Fourier transform \(\widetilde{\delta}\) as 
%
\begin{align}
  \delta (\vec{x} , t) = 
  \frac{1}{(2 \pi )^3} \int \dd[3]{\vec{k}} \widetilde{\delta} (t) \exp(i \vec{k} \cdot \vec{x})
\,,
\end{align}
%
and for \(\vec{v}\) and \(\phi \). 
Note that we only expand in 3D space: plane waves do not propagate nicely in an expanding universe, so they are not a good Fourier base: we keep time derivatives, substituting spatial ones with multiplication by \(i \vec{k}\).

The actual quantities must be real: therefore we know that \(\widetilde{\delta}^{*}_{\vec{k}} (t) = \widetilde{\delta}_{-\vec{k}} (t)\).

Before we Fourier-transform the equations, we can make a \textbf{simplification}: as we mentioned before, any vector field \(\vec{v}\) can be decomposed by the Helmholtz theorem into the gradient of a scalar field and a divergenceless vector field: 
%
\begin{align}
  \vec{v} = \nabla \Psi + \vec{T}
\,,
\end{align}
%
where \(\nabla \cdot \vec{T} = 0\) (and, as is known from vector calculus, \(\nabla \times \qty(\nabla \Psi ) = 0\)).

% Using the convective time derivative, we have 
Before we considered only the first order, the Euler equation read:
%
\begin{align}
  \frac{\mathrm{D} \vec{v}}{\mathrm{D}t}  
  + H \vec{v} = - \frac{1}{a} \nabla \phi 
\,.
\end{align}

% Inserting here the Helmholtz decomposition applied to the velocity, we find 
If we substitute \(\vec{v}\) with its Helmholtz decomposition, we can split the equation into two, one for \(\vec{T}\) and one for \(\Psi \):
%
\begin{align}
  \frac{\mathrm{D} \vec{T}}{\mathrm{D}t} + H \vec{T} = 0
  \qquad \text{and} \qquad
  \frac{\mathrm{D} (\vec{\nabla} \Psi )}{\mathrm{D}t} + H \vec{\nabla} \Psi = - \frac{\vec{\nabla} \phi}{a}
\,.
\end{align}

We could determine which terms went on either side based on whether they had zero divergence or zero curl.

\todo[inline]{Does the convective derivative commute with \(\nabla \times \) and \(\nabla \cdot \), though? }

This has an important physical meaning: the divergenceless part of the velocity evolves by itself, unaffected by the gravitational field perturbation. The way it evolves is, roughly speaking, a decreasing exponential, so its magnitude will diminish over time. 
Therefore, any \(\vec{T}\) component which is part of the velocity field initially gets ever more diluted.
% which means that vorticity is conserved in fluid motion. 

% This means that any vorticity which might have been there at the beginning would have been diluted out over time.
Because of this, we neglect the divergenceless component of the velocity field, and only consider the \(\vec{v} = \nabla \Psi \) part. 
In Fourier space, this reads \(\vec{v} \propto \vec{k} \Psi \), so we can project the three equations along the unit vector \(\hat{k} = \vec{k} / \abs{\vec{k}}\), simplifying them to a single scalar one.
We will denote \(v = \vec{v} \cdot \hat{k}\) and \(k= \vec{k} \cdot \hat{k} = \abs{\vec{k}}\).

With this and denoting time derivatives with a dot, our equations will read:
%
\begin{align}
  \dot{\delta} + \frac{ik}{a} v &= 0 \\
  \dot{v} + H v &= - \frac{ik}{a} \phi \\
  - k^2 \phi &= 4 \pi G a^2 \rho_b \delta
\,.
\end{align}

% Now, by differentiating in one single step we will get a linear second order differential equation for \(\delta \) which is separated from the others. 
We can find an equation for \(\delta \) alone by differentiating the first equation with respect to time and substituting the Euler equation:
% Differentiating one more time and making the \(\vec{k}\) implicit we get: 
%
\begin{subequations}
\begin{align}
  \ddot{\delta} + \frac{ik}{a} \dot{v} - \frac{ik}{a^2}\dot{a} v &= 0  \\
  \ddot{\delta} + \frac{ik}{a} \qty(-Hv - \frac{ik}{a} \phi ) - \frac{ik}{a} Hv &=0  \\
  \ddot{\delta} - \frac{2ik}{a} Hv + \frac{k^2 \phi }{a^2} &=0
\,,
\end{align}
\end{subequations}
%
but, from the Poisson equation, the last term is equal to \(- 4 \pi G \rho_b \delta \), and from the continuity equation again the second term is equal to \(2H \dot{\delta}\): the equation then reads 
%
\begin{align}
  \ddot{\delta} + 2 H \dot{\delta} - 4 \pi G \rho_b \delta = 0 
\,.
\end{align}

This looks promising: if \(H\) and \(\rho _b\) were constant, it would be a simple second-order ODE. 
They are not constant, but they are functions corresponding to the background: we can use the solutions found earlier corresponding to a matter-dominated universe, 
% and we use the following solutions of the background equations: 
%
\begin{align}
  a(t ) \propto t^{2/3}\,,  &&
  H = \frac{2}{3t}\,,  &&
\rho_b = \qty(6 \pi G t^2)^{-1}
\,.
\end{align}

With these, the equation reads
%
\begin{align}
  \ddot{\delta} + \frac{4}{3t} \dot{\delta} - \frac{2}{3t^2} \delta = 0
\,,
\end{align}
%
% which will be a power of \(t\): plugging \(t^{\alpha }\) we find the equation 
which will have two independent solutions since it is of second order: it turns out that both can be recovered using a powerlaw ansatz, \(\delta \propto t^{\alpha }\): the equation for \(\alpha \) reads
%
\begin{align}
  \alpha (\alpha -1) + \frac{4}{3} \alpha - \frac{2}{3} =0 
\,,
\end{align}
%
whose solutions are \(\alpha = -1\) and \(\alpha =  2/3\). 

The solution with \(\alpha = 2/3\) is the \textbf{growing mode}, while the one with \(\alpha = -1\) (so, \(\delta \propto t^{-1} \propto H(t)\)) is the \textbf{decaying mode}. 

\begin{enumerate}
  \item The growing mode has \(\delta \propto t^{2/3} \propto a(t)\), \(v \propto t^{1/3}\), \(\phi = \const\);
  \item the decaying mode has \(\delta \propto t^{-1} \propto H(t)\), \(v \propto t^{-1/2}\), \(\phi \propto t^{-5/3}\). 
\end{enumerate}

Typically, we are more interested in the growing mode.  

\todo[inline]{"When the inflaton field becomes classical we lose a degree of freedom: this removes the decaying solutions. ": not really clear to me.}
% Another way to see it is to see that the decaying mode explodes as \(t \rightarrow 0\). 

% Thus, we usually remove the decaying solution. 
% 
% An interesting observation is the fact that the growing mode grows just as fast as the scale factor. 

% Inserting this into the other equations we find \(v \propto t^{\beta } \) with \(\beta = 1/3\) or \(-4/3\), while \(\phi \propto t^{\gamma }\) with \(\gamma = 0, -5/2\). 

% \todo[inline]{understand what the equation }
%
\section{Star formation}

% \begin{align}
%     \chi_{J} = c_s \sqrt{\frac{\pi}{4 \overline{\rho}}}
%   \,,
%   \end{align}
% %
% The Jeans density is given by 
% %
% \begin{align}
%   \rho_{J} = \frac{3}{4 \pi M^2} \qty(\frac{3k_B T}{2G \overline{m}})^{3}
% \,.
% \end{align}
% %

% %
% \begin{align}
%   \chi_J = c_s \qty(\frac{\pi}{4 \overline{\rho}})^{1/2}
% \,,
% \end{align}
% %

% %
% \begin{align}
%   M_J = \frac{4 \pi }{3} \overline{\rho} \qty(\frac{\lambda}{2})^{3}
% \,,
% \end{align}
% %

% %
% \begin{align}
%   c_s^2 \sim \frac{k_B T}{\overline{m}}
% \,,
% \end{align}
% %

% and now we are assuming that \(\rho > \rho _J\). 

% Stars are made of baryons. 
Let us now actually discuss stellar formation specifically.
The presence of molecular hydrogen, \ce{H2}, is correlated to stellar formation since it can absorb some of the kinetic energy of the collapsing cloud by dissociating, thus allowing for further collapse; it forms through the channels
%
\begin{subequations}
\begin{align}
  \ce{H} + e &\leftrightarrow \ce{H-} + \gamma &
  \ce{H-} + \ce{H} &\leftrightarrow \ce{H2} + e  \\
   \ce{H}+ p &\leftrightarrow \ce{H2+} + \gamma  &
  \ce{H2+} + \ce{H} &\leftrightarrow \ce{H2} + \ce{p}
\,,
\end{align}
\end{subequations}
%
and these processes, generally speaking, start to become efficient at redshifts of about \(z \sim 200\). 

As we have seen, if we fix the temperature then the Jeans critical density scales like \(\rho _J \propto M^{-2}\) \eqref{eq:jeans-critical-density-by-mass}:
so, it is easier to get above the Jeans density if the mass is high (and the temperature is low). 
This means that we expect the formation of stars to be a top-down process: larger structures form first. 

Let us now give some quantitative estimates of the densities at hand.

Approximately, the baryonic density today is \(\rho_{0b} \sim \SI{1e-28}{kg m^{-3}}\). 

Going backwards in time, the baryonic density scales like \(\rho _b (z) = (1+z)^{3} \rho_{0b}\): at \(z \sim 200 \), for example, we have \(\rho_{b} (z \sim 200) \sim \SI{e-22}{kg m^{-3}}\). 

Stars will not form anywhere, they will only do so in over-dense regions, which form preferentially in the centers of dark-matter halos, acting as gravitational ``traps''. 

Let us take, as an example, a molecular hydrogen (\(\overline{m} \approx 2 m_p\)) cloud with a mass of \(M = 1000 M_{\odot}\) and \(T \approx \SI{20}{K}\) (a typical temperature for matter at this redshift): we have a Jeans critical density of \(\rho _J \approx \SI{4e-22}{kg m^{-3}}\). 

\todo[inline]{Perhaps this is the place to put the scaling of the temperature of matter! The fact that the temperature of baryons decreases slower than the radiation's is crucial to reach a temperature low enough to reach the Jeans bound.}

For a solar-mass cloud, the critical density is much lower: \(\rho _J \approx \SI{4e-16}{kg m^{-3}}\). 

% \todo[inline]{Is \SI{20}{K} a typical temperature then?}

% The matter starts free-falling: then the kinetic energy increases. 
% However, it is used in chemical processes. That way matter can become opaque. 

\subsection{Collapsing a solar-mass cloud}

Let us suppose that, by virtue of being in a dark matter halo and after some large-scale collapse, we have indeed reached the conditions of \(\rho \sim \SI{4e-16}{kg m^{-3}}\) and \(T \sim \SI{20}{K}\).
We will try to understand how a Sun-like star might form.

Molecular hydrogen is present in the cloud, its dissociation energy is \(\epsilon _D \approx \SI{4.5}{eV}\) while, as we know, the ionization energy of hydrogen is \(\epsilon _H \approx \SI{13.6}{eV}\).

% We have the equation 
As the collapse starts, any kinetic energy which is developed is used, first to dissociate molecular hydrogen and then to ionize hydrogen, so for a while the evolution goes according to the free-fall equation of motion:
%
\begin{align}
  \frac{1}{2} \qty(\dv{r}{t})^2 = \frac{G m_0 }{r} - \frac{G m_0 }{r_0 }
\,,
\end{align}
%
which, as we have seen earlier, corresponds to a free-fall time given by \eqref{eq:free-fall-time}: we get \(t_{FF} \approx \SI{100}{kyr}\). 

The energy needed to dissociate all the \ce{H2} and then ionize the \ce{H} is given in terms of the mass of the cloud, \(M = M_{\odot}\), by
%
\begin{align}
  E = \frac{M}{2 m_H} \epsilon_{D} + \frac{M}{m_H} \epsilon_{I}
\,,
\end{align}
%
where \(m_H \approx m_p\) is the mass of hydrogen. The mass of \ce{H2} is slightly different from \(2 m_H\), but the difference is of the order of \(\epsilon _D / m_H \sim \num{e-8}\), completely negligible.
% where \(M\) is the mass of the star, \(m_H\) is the mass of a hydrogen atom, while \(\epsilon_{D} \approx \SI{4.5}{eV}\) and \(\epsilon_{U} \approx \SI{13.6}{eV}\). 

The cloud starts from a radius \(R_1 \sim \sqrt[3]{3 M / 4 \pi \rho _J} \approx \SI{1e15}{m} \approx \SI{0.1}{lyr}\): what radius \(R_2 \) does it reach by the time all the hydrogen is ionized?
% We are going from a radius \(R_1 \) to a radius \(R_2 \): we get , since that is the radius at which the collapse starts. 

% However, the \(R_1 \) term hardly contributes: the term \(R_2 \) is typically 4 orders of magnitude smaller. ˙
We can calculate it by equating the difference in potential energy to the ionization and dissociation energy: 
%
\begin{align}
GM_{\odot}^2 \qty( \frac{1}{R_2} - \frac{1}{R_1 }) = E 
&= \frac{M_{\odot}}{2 m_H} \epsilon_{D} + \frac{M_{\odot}}{m_H} \epsilon_{I}  \\
GM_{\odot} \qty(\frac{1}{R_2} - \frac{1}{R_1 }) &= \frac{\epsilon _D}{2 m_H} + \frac{\epsilon _I}{m_H} \approx \num{1.7e-8} c^2  \\
\qty( \frac{1}{R_2 } - \frac{1}{R_1 })^{-1} &\approx \frac{GM_{\odot}}{c^2} \times \num{6e7} \approx \SI{9e10}{m} 
\,,
\end{align}
%
% which yields 
so, since \(R_1 \gg \SI{e11}{m}\) the \(R_1^{-1}\) term is basically negligible, while \(R_2 \approx \SI{e11}{m}\). 
We have shrunk our cloud from \SI{0.1}{lyr} to about \SI{0.6}{AU}, just smaller than the radius of the orbit of Venus. This is still much larger than \(R_{\odot} \approx \SI{7e8}{m}\).

The rest of the collapse is much slower, and it happens under hydrostatic equilibrium; the proto-star will still shrink under its own gravity, getting ever hotter, until it reaches the temperature needed for the ignition of fusion, around \SI{e7}{K}. 

% The final equation we  get comes from the  virial theorem:
At this stage the virial theorem applies, since the plasma is optically thick, very little energy is lost to radiation: 
%
\begin{align}
  2E_{k} + E _{\text{gr}} =0
\,,
\end{align}
%
and we can recover the total kinetic energy from the equipartition theorem:
%
\begin{align}
  E_k = \frac{3}{2} N k_B T = \frac{3}{2} \frac{M}{\overline{m}} k_B T \approx \frac{3M}{m_H} k_B T
\,,
\end{align}
%
where we used the fact that \(\overline{m} = 0.5 m_H\): the hydrogen is ionized, so we have both free electrons and free protons, and basically all the mass is with the latter.
% so in the end, equating the energies, we have 

The gravitational binding energy can be estimated as what was lost in the dissociation and ionization:
%
\begin{align}
  2 \times \underbrace{3 k_B T \frac{M}{m_H }}_{E_k}
  &= - E _{\text{gr}} \approx \frac{M}{m_H} \qty(\frac{\epsilon_{D}}{2} + \epsilon_{I})  \\
  k_B T &\approx \frac{1}{6} \qty( \frac{\epsilon _D}{2} + \epsilon _I) \approx \SI{2.6}{eV} \approx \SI{30}{kK}
\,,
\end{align}
%
still very much lower than the temperature needed to ignite fusion, which is on the order of a \(\SI{}{keV} \sim \SI{10}{MK}\).
% which means that \(k_B T = \frac{1}{12} \qty(\epsilon_{D} + 2 \epsilon_{I}) \sim \SI{2.6}{eV}\), which is still very low compared to the temperature we need. 

\end{document}