\documentclass[main.tex]{subfiles}
\begin{document}

\section*{Thu Dec 05 2019}

There will be a section on gravitational waves in cosmology in January. 

We found the relation: 
%
\begin{align}
  \vec{r}  = a(\lambda ) \vec{x}
\,,
\end{align}
%
and 
%
\begin{align}
  \vec{\dot{r}} = \vec{u} = \dot{a} \vec{x} + a \dot{\vec{x}} = H \vec{r} + \vec{v}
\,,
\end{align}
%
since \(H = \dot{a} / a\). 

An issue we have is the \emph{redshift space distortion}, caused by \emph{peculiar velocities} which cause the light we see to have redshift beyond the one caused by cosmology alone. 

In comoving coordinates, we can use the regular Newtonian fluid dynamics equations. 

Since we know that the gravitational instability is relevant when the gravitational force overcomes the pressure forces, we set the pressure gradient to be equal to zero. We have 
%
\begin{subequations}
\begin{align}
  \pdv{\rho }{t} |_{\vec{r}} + \nabla_{\vec{r}} \qty(\rho \vec{u}) &= 0  \\
  \dv{\vec{u}}{t} + \qty(\vec{u} \cdot \vec{\nabla_{\vec{r}}}) \vec{u} &=0  \\
  \nabla_{\vec{r}}^2 \Phi &= 4 \pi G \rho 
\,,
\end{align}
\end{subequations}
%
and we consider a background \(\Phi_b (\vec{r}, t)\), and \(\rho = \overline{\rho}(t)\): we have a peculiar gravitational field beyond the Robertson-Walker background: we define 
%
\begin{align}
  \phi  = \Phi - \Phi_b
\,,
\end{align}
%
and analogously we define by 
%
\begin{align}
  \delta \rho = \rho - \overline{\rho} = \overline{\rho} \qty(1 + \delta )
\,,
\end{align}
%
with \(-1 \leq \delta \leq \infty\). 

The Laplace equation for \(\phi \) is given by 
%
\begin{align}
  \nabla^2_{\vec{x}} \phi (\vec{x}, t) = 4 \pi G a^2(t) \overline{\rho} (t) \delta (\vec{x}, t)
\,,
\end{align}
%
and we actually \emph{can} have negative quantities on the RHS. Not repulsive gravity, but kind of. 

The relation between the derivatives is 
%
\begin{align}
  \pdv{}{r} = \frac{1}{a } \pdv{}{x}
\,,
\end{align}
%
therefore the continuity equation becomes:
%
\begin{align}
  \pdv{\rho }{t} + 3H \rho + \frac{1}{a} \nabla_{x} \qty(\rho \vec{v}) = 0
\,,
\end{align}
%
we get that the density scales like \(a^{-3}\) if there is no velocity: then it becomes 
%
\begin{align}
  \frac{\dot{\rho}}{\rho } + 3 \frac{\dot{a}}{a} =0 
\,.
\end{align}
%

For the momentum equation the calculation is longer, but in the end we find: 
%
\begin{align}
  \pdv{\vec{v}}{t} + H \vec{v} +
  \frac{1}{a} \qty(\vec{v} \cdot \nabla_{x}) \vec{v}
  = - \frac{1}{a} \nabla_{x} \phi 
\,,
\end{align}
%
and once again we remark that we are not using GR because, in the absence of symmetry, that is definitely too difficult. 

If we perturb, we will find that plane waves are not solutions anymore, because we will have time-dependent coefficients. 

The perturbation of the density is: 
%
\begin{align}
  \rho (\vec{x}, t) = \overline{\rho}(t) \qty(1 + \delta (\vec{x}, t))
\,,
\end{align}
%
and for the field: 
%
\begin{align}
  \phi (\vec{x}, t) = \Phi - \Phi_b
\,.
\end{align}

The derivative is: 
%
\begin{align}
  \pdv{\rho }{t} = \pdv{\overline{\rho}}{t} \qty(1 + \delta ) + \overline{\rho} \pdv{ \delta }{t}
\,,
\end{align}
%
and we can discard higher-order terms: simplifying the zeroth-order equation we find for the  continuity equation: 
%
\begin{align}
  \pdv{ \delta }{t} + \frac{1}{a} \vec{\nabla} \cdot \vec{v} = 0
\,.
\end{align}

On the other hand, for the momentum equation we find: 
%
\begin{align}
  \pdv{\vec{v}}{t} + H \vec{v} = - \frac{1}{a} \nabla \phi 
\,,
\end{align}
%
and in order to solve these we can expand in Fourier space: 
%
\begin{align}
  \delta (\vec{x} , t) = 
  \frac{1}{(2 \pi )^3} \int \dd[3]{\vec{k}} \widetilde{\delta} (t) \exp(i \vec{k} \cdot \vec{x})
\,,
\end{align}
%
and similarly for \(\vec{v}\) and \(\phi \). 

The actual quantities must be real: therefore we know that \(\widetilde{\delta}^{*}_{\vec{k}} (t) = \widetilde{\delta}_{-\vec{k}} (t)\).

Any vector field \(\vec{V}\) can be decomposed by the Helmholtz theorem into a gradient and a divergence: 
%
\begin{align}
  \vec{V} = \nabla \Psi + \vec{T}
\,,
\end{align}
%
where \(\nabla \cdot \vec{T} = 0\).

Using the convective time derivative, we have 
%
\begin{align}
  \frac{\mathrm{D} \vec{v}}{\mathrm{d}t} 
  + H \vec{v} = - \frac{1}{a} \nabla \phi 
\,.
\end{align}

Inserting here the Helmholtz decomposition applied to the velocity, we find 
%
\begin{align}
  \frac{\mathrm{D} \vec{T}}{\mathrm{D}t} + H \vec{T} = 0
\,,
\end{align}
%
which means that vorticity is conserved in fluid motion. 

This means that any vorticity which might have been there at the beginning would have been diluted out over time.

We project the equations along the versor \(\vec{u}_{\vec{k}} = \vec{k} / \abs{\vec{k}}\). 

Then the equation becomes 
%
\begin{align}
  \dot{\delta}_{\vec{k}} + \frac{ik}{a} v_{\vec{k}} =0
\,,
\end{align}
%
and 
%
\begin{align}
  \dot{v}_{\vec{k}} + H v_{\vec{k}} = - \frac{\vec{ik}}{a} \phi_{\vec{k}}
  \,,
\end{align}
%
and finally 
%
\begin{align}
  - k^2 \phi_{\vec{k}} = 4 \pi G a^2 \overline{\rho} \delta_{\vec{k}}
\,.
\end{align}

Now, by differentiating in one single step we will get a linear second order differential equation for \(\delta \) which is separated from the others. 

Differentiating one more time and making the \(\vec{k}\) implicit we get: 
%
\begin{subequations}
\begin{align}
  \ddot{\delta} + \frac{ik}{a} \dot{v} - \frac{ik}{a^2}\dot{a} &= 0  \\
  \ddot{\delta} + \frac{ik}{a} \qty(-Hv - \frac{ik}{a} \phi ) - \frac{ik}{a} Hv &=0  \\
  \ddot{\delta} - \frac{2ik}{a} Hv + \frac{k^2 \phi }{a^2} &=0
\,,
\end{align}
\end{subequations}
%
but the last term is \(- 4 \pi G a^2 \overline{\rho} \delta \): then in the end we find 
%
\begin{align}
  \ddot{\delta} + 2 H \dot{\delta} - 4 \pi G \overline{\rho} \delta = 0 
\,,
\end{align}
%

and we use the following solutions of the background equations: 
%
\begin{subequations}
\begin{align}
  a(t ) &\propto t^{2/3}  \\
  H &= \frac{2}{3t}  \\
\overline{\rho} &= \qty(6 \pi G t^2)^{-1}
\,,
\end{align}
\end{subequations}
%
and plugging these in we find 
%
\begin{align}
  \ddot{\delta} + \frac{4}{3t} \dot{\delta} - \frac{2}{3t^2} \delta = 0
\,,
\end{align}
%
which will be a power of \(t\): plugging \(t^{\alpha }\) we find the equation 
%
\begin{align}
  \alpha (\alpha -1) + \frac{4}{3} \alpha - \frac{2}{3} =0 
\,,
\end{align}
%
which gives \(\alpha = -1, 2/3\). 

The solution with \(\alpha = 2/3\) is called the growing mode, while the one with \(\alpha = -1\) is called the decaying mode. 
When the inflaton field becomes classical we lose a degree of freedom: this removes the decaying solutions. 
Another way to see it is to see that the decaying mode explodes as \(t \rightarrow 0\). 

Thus, we usually remove the decaying solution. 

An interesting observation is the fact that the growing mode grows just as fast as the scale factor. 

Inserting this into the other equations we find \(v \propto t^{\beta } \) with \(\beta = 1/3\) or \(-4/3\), while \(\phi \propto t^{\gamma }\) with \(\gamma = 0, -5/2\). 

\todo[inline]{understand what the equation }
%
\begin{align}
    \chi_{J} = c_s \sqrt{\frac{\pi}{4 \overline{\rho}}}
  \,,
  \end{align}
%
The Jeans density is given by 
%
\begin{align}
  \rho_{J} = \frac{3}{4 \pi M^2} \qty(\frac{3k_B T}{2G \overline{m}})^{3}
\,.
\end{align}
%

%
\begin{align}
  \chi_J = c_s \qty(\frac{\pi}{4 \overline{\rho}})^{1/2}
\,,
\end{align}
%

%
\begin{align}
  M_J = \frac{4 \pi }{3} \overline{\rho} \qty(\frac{\lambda}{2})^{3}
\,,
\end{align}
%

%
\begin{align}
  c_s^2 \sim \frac{k_B T}{\overline{m}}
\,,
\end{align}
%

and now we are assuming that \(\rho > \rho _J\). 

Stars are made of baryons. 

Molecular hydrogen: 
%
\begin{subequations}
\begin{align}
  H + e &\leftrightarrow H^{-} + \gamma  \\
  H^{-} + H &\leftrightarrow H_2 + e  \\
   H+ p &\leftrightarrow H_2^{+} + \gamma  \\
  H_2^{+} + H &\leftrightarrow H_2 + p
\,,
\end{align}
\end{subequations}
%
and these processes are generally not very efficient: they start to be efficient at redshifts of about \(z \sim 200\). 

It is easier to get the Jeans density if the mass is high and the temperature is low. 

The formation of stars is somewhat top-down: larger structures form first. 

The critical energy density today is  \(\rho_{0c} \sim \SI{e-29}{g cm^{-3}}\), while for baryons we have \(\rho_{0b} \sim \SI{1e-31}{g cm^{-3}}\). 

On the other hand, at \(z \sim 200 \) we must multiply the density by a factor \((1+z)^3\) we get a density of \(\rho_{b} (z \sim 200) \sim \SI{e-22}{kg m^{-3}}\). 

If we take \(M = 1000 M_{\odot}\) and \(T \approx \SI{20}{K}\), we have a Jeans critical density of around \(\SI{e-20}{kg m^{-3}}\). 

\todo[inline]{Is \SI{20}{K} a typical temperature then?}

The matter starts free-falling: then the kinetic energy increases. 
However, it is used in chemical processes. That way matter can become opaque. 

We have the equation 
%
\begin{align}
  \frac{1}{2} \qty(\dv{r}{t})^2 = \frac{G m_0 }{r} - \frac{G m_0 }{r_0 }
\,,
\end{align}
%
which gives us a typical free-fall time of \SI{2e4}{yr}. 

The energy needed to ionize all the hydrogen is 
%
\begin{align}
  E = \frac{M}{2 m_H} \epsilon_{D} + \frac{M}{m_H} \epsilon_{I}
\,,
\end{align}
%
where \(M\) is the mass of the star, \(m_H\) is the mass of a hydrogen atom, while \(\epsilon_{D} \approx \SI{4.5}{eV}\) and \(\epsilon_{U} \approx \SI{13.6}{eV}\). 

We are going from a radius \(R_1 \) to a radius \(R_2 \): we get \(R_1 \sim \sqrt[3]{3 M / 4 \pi \rho _J}\), since that is the radius at which the collapse starts. 

However, the \(R_1 \) term hardly contributes: the term \(R_2 \) is typically 4 orders of magnitude smaller. 

Our objects must still become hotter and smaller in order to reach the \SI{e7}{K} needed for ignition. 

The final equation we  get comes from the  virial theorem:
%
\begin{align}
  2E_{k} + E _{\text{gr}} =0
\,,
\end{align}
%
and 
%
\begin{align}
  E_k = \frac{3}{2} N k_B T = \frac{3}{2} \frac{M}{\overline{m}} k_B T \approx \frac{3M}{m_H} k_B T
\,,
\end{align}
%
since \(\overline{m} = 0.5 m_H\): so in the end, equating the energies, we have 
%
\begin{align}
  2 \times 3 k_B T \frac{M}{m_H }
  = \frac{M}{m_H} \qty(\frac{\epsilon_{D}}{2} + \epsilon_{I})
\,,
\end{align}
%
which means that \(k_B T = \frac{1}{12} \qty(\epsilon_{D} + 2 \epsilon_{I}) \sim \SI{2.6}{eV}\), which is still very low compared to the temperature we need. 

\end{document}