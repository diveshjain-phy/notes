\documentclass[main.tex]{subfiles}
\begin{document}

% \section*{Thu Oct 31 2019}
\paragraph{Effective degrees of freedom}

\marginpar{Friday\\ 2019-10-31, \\ compiled \\ \today}

% Neutrinos have very low mass: therefore they become relativistic very quickly.

% We saw last time the case of either bosons' or fermions' number density, energy density and pressure. 

% In the nonrelativistic case instead we must consider the second order in \(q / \sqrt{m} \), and we get the Boltzmann suppression factor out of the integral: \(\exp(-m/T) \).

% In the nonrelativistic case there is no distinction between bosons and fermions.

% What happens in the part of the history of the universe where all the particles were in thermal equilibrium?

We have been discussing the behavior of a single particle species with \(g\) degrees of freedom; however we know that there were many types of particles in the early universe, so we need a way to generalize these results.
We do so by defining the number of effective degrees of freedom: 
%
\begin{align}
g_*(T) = 
\sum _{i \in \text{BE}} g_i \qty(\frac{T_i}{T})^{4}
+
\frac{7}{8}
\sum _{i \in \text{FD}} g_i \qty(\frac{T_i}{T})^{4}
\,,
\end{align}
%
where the index \(i\) labels all the particle species in our model, running over all of those which are relativistic at temperature \(T\) (that is, as a first approximation, we only count those with masses \(m_i < T\)). 
The equilibrium temperature is \(T\), while \(T_i\) are the temperatures of the various particle species, which we allow to be different from \(T\) --- we will elaborate on this point in a moment.
We distinguish two different terms in the sum, depending on whether the particle species obey Bose-Einstein or Fermi-Dirac statistics, since as we have seen the latter have a prefactor of \(7/8\) in the expression for the energy density. 

This definition is constructed so that we can write the compact relation 
%
\begin{align} \label{eq:radiation-energy-density-effective-dof}
\rho (T) = g_*(T) \frac{\pi^2}{30} T^{4}
\,.
\end{align}

Of course, considering particles completely when \(m_i < T\) and not at all when \(m_i > T\) is a simplification: in the region in which the temperature is of the order of the mass of the particle there will be a transition, which can be calculated properly by doing the integrals numerically. The results are shown in figure 3 of a paper by \textcite[]{husdalEffectiveDegreesFreedom2016}, which can also be referred to for many more details on effective degrees of freedom.
Figure 1 of the same paper shows how \(g_*\) decreases while the temperature of the universe decreases and more and more particle species become nonrelativistic. 

Why do we consider the possibility of the temperature of a particle species being different from the equilibrium temperature?
 
Each process involving particles, be it decay or scattering, is characterized by a certain timescale.
If the timescale of a certain interaction is larger than the cosmological timescale (the age of the universe), then that interaction statistically will not happen.
Particles which cannot reach thermal equilibrium because of this are called \emph{decoupled}, ones for which this is not the case are called \emph{coupled}.

Although they may not interact, as long as they are relativistic decoupled particles can still affect the energy density of the universe, so we need to count them. 

\paragraph{The time-temperature relation}

We want to find a relation between time and temperature in the early universe.
Let us consider ultrarelativistic particles which are coupled, in the early universe which is radiation dominated (here ``radiation'' refers to all kinds of ultrarelativistic particles). 

% In this case, from the ``conservation of the stress-energy tensor'' we know that \(\rho \propto a^{-4}\).

We start from the third Friedmann equation
%
\begin{equation}
  H^2= \frac{8 \pi G}{3} \rho - \frac{k}{a^2}
\,,
\end{equation}
%
neglect the curvature term\footnote{We can do so since we know that right now the contribution to the global \(\Omega \) of curvature is small (we have not been able to distinguish it from zero) and while this term scales as \(a^{-2}\) the matter term scales as \(a^{-3}\). Since the matter term is dominant over the curvature term now, it was even more so earlier.} and use the facts that for a radiation-dominated universe \(\rho \propto a^{-4}\) while \(a \propto t^{1/2}\), meaning that \(H = \dot{a} / a =  1/ (2t)\).

% The equation becomes 
% %
% \begin{subequations}
% \begin{align}
%   \qty(\frac{\dot{a} }{a})^2 &= \frac{8 \pi G }{3} \rho_{*} \qty(\frac{a}{a_{*}})^{-4} - \frac{k}{a^2} 
% \,,
% \end{align}
% \end{subequations}
% %
% but it is not enough to look at the slopes: we can, however, have information about the normalization as well from present-day observations.

% If the second term is much smaller than the first today, then it was even more so in the far past. We can rewrite the equation as
% %
% \begin{equation}
%   1 = \Omega_{\text{tot}} - \frac{k}{a^2 H^2} \equiv \Omega _{\text{tot}} + \Omega_{\text{curvature}}
% \,,
% \end{equation}
% %
% where \(\Omega _{\text{curvature}} \equiv -k / (a^2 H^2)\).

% So we neglect the second term: approximately we then have 
% %
% \begin{equation}
%   \qty(\frac{\dot{a } }{a})^2 = \frac{8 \pi  G }{3} \rho _{\text{rad}}
% \,.
% \end{equation}
% %
% We know that \(a(t) \propto t^{1/2}\), therefore \(\dot{a} / a = H = 1/2t\). 

% \todo[inline]{Wait, how does this work?}

Substituting these, as well as the expression we have found for the energy density in terms of the effective degrees of freedom, we get
%
\begin{equation}
  \frac{1}{4 t^2} = \frac{8 \pi G}{3} g_{*} (T) \frac{\pi^2}{30} T^{4}
\,.
\end{equation}
%
% where 
% %
% \begin{equation}
%   g_{*} = g_{*}(T) = \sum _{i \in BE } g_i \qty(\frac{T_{i}}{T})^{4} + \frac{7}{8} \sum _{i \in FD } g_i \qty(\frac{T_{i}}{T})^{4}
% \,,
% \end{equation}
% %
% where we insert a correction factor in order to not consider particles which are nonrelativistic (?).
% \(BE\) and \(FD\) denote the bosonic and fermionic degrees of freedom respectively.

% The \(i\) spans the different types of particles in the SM.
% The \(T_i\) are the thermalization temperatures of the various species.

% The plot of \(g_{*}\) in terms of the temperature goes ``down in steps'', as more and more species become nonrelativistic.

Then we have a formula for temperature in terms of time: 
%
\begin{align}
  \frac{1}{2t} &= \qty(\frac{8 \pi G}{3})^{1/2} g_{*}^{1/2} \qty(\frac{\pi^2}{30})^{1/2} T^2 \\
  t &\approx \underbrace{\frac{1}{2 \sqrt{\frac{8 \pi \pi^2}{3 \times 30}}}}_{\approx\num{.301}} g_{*}^{-1/2} \frac{m_P}{T^2} \approx \qty(\frac{T}{\SI{}{MeV}})^{-2} \SI{}{s}
  \label{eq:time-temperature-relation}
\,,
\end{align}
%
where \(m_P  = G^{-1/2} \approx \SI{1.2e19}{GeV}\) is the Planck mass. Beware: there are different conventions for this, sometimes the definition is chosen as \(m_P = (8 \pi G)^{-1/2}\), which simplifies the Friedmann and Einstein equations somewhat. 
This mass corresponds to the energy scale at which quantum gravitational effects cannot be neglected.

The last approximation in \eqref{eq:time-temperature-relation} is quite rough, as it neglects the variation of the effective number of degrees of freedom completely: however, the factor \(g_{*}^{-1/2}\) is of order 1 around \(T \approx \SI{1}{MeV}\), which is the region in which we will apply our formula, so this is fine for our purposes.

% When the universe is 1 second old, weak interactions decouple.

% Why are we allowing different thermalization temperatures?

% A hypothesis is that the particles exit the Planck epoch, \(t \approx m_P\), thermalized.
% When decoupling occurs, they can stop being thermalized.

% Another hypothesis is that they stop being thermalized at a certain temperature during inflation.

% Some time ago we discussed entropy conservation.

\paragraph{Entropy effective degrees of freedom}

Entropy density is defined as entropy per unit volume, \(s = S / V = (P+\rho ) / T\). Since the total entropy in a comoving region is conserved (if there is thermal equilibrium) the quantity \(s a^3\), proportional to \(S\), is conserved. 

If we only have relativistic particles (which satisfy \(P = \rho / 3\)), the entropy density can be expressed as 
%
\begin{equation}
s = (P + \rho ) / T = \frac{4}{3} \frac{\rho}{T} = (2 \pi^2 / 45) g_{* s} T^{3} 
\,;
\end{equation}
%
where we defined a new number of effective degrees of freedom, \(g_{*s}\), whose definition is slightly different from that of the one used for the energy, \(s \propto T^3\) as opposed to \(\rho \propto T^{4}\): 
%
\begin{equation}
  g_{* s} \equiv \sum _{i \in BE} g_i \qty(\frac{T_i}{T})^3
  + \frac{7}{8} \sum _{i \in FD} g_i \qty(\frac{T_i}{T})^3
\,.
\end{equation}

The expression for \(s \propto g_{*s} T^3\) is more general than simply \(s \propto T^3\), and in fact with this new one we can \textbf{update Tolman's law}: taking the cube root of the conserved quantity \(s a^3\) we find \(T a g_{* s }^{1/3} = \const\).

% When the neutrinos decouple, their temperature keeps scaling like \(1/a\): they keep being thermalized with the photons even though they do not react.

\subsection{Decoupling and radiation temperature}

The temperature \SI{1}{MeV} occurs when the age of the universe is approximately \SI{1}{s}, and this is the point at which the weak interactions involving neutrinos stop occurring.\footnote{This point is also relevant for another process: the weak interaction mediated processes are also what allows there to be an equilibrium between protons and neutrons, so when those reactions stop they become independent, and they evolve differently, since protons are stable while neutrons are not. This is crucial when discussing nucleosynthesis, the formation of nuclei.}

% When we reach \(T \approx \SI{0.5}{MeV}\) electrons and positrons become thermalized: they decay into photons: the photon temperature \(T\) increases with respect to what would happen without this process (it actually just decreases slower).

When this happens, neutrinos decouple, so they stop interacting: they start evolving as any relativistic particle species would (they are relativistic since their mass is much lower than the \SI{}{MeV}).

At this point, the temperatures of neutrinos and photons are ``disconnected'', there is no mechanism to equalize them.
However, while the neutrinos are freely floating by, their energy density scaling like \(\rho \propto a^{-4}\), the photons will be active for a few more seconds. 

The mass of electrons and positrons is around \SI{0.5}{MeV}, so until about \SI{4}{s} into the life of the universe, \SI{3}{s} from the decoupling of the neutrinos, the reaction \(e^{+} + e^{-} \leftrightarrow 2 \gamma \) is still in equilibrium. 
After this, the populations of electrons and positrons annihilate, and since they have less effective degrees of freedom in which to deposit their energy than before (since the neutrinos are not coupled anymore) they dump it all into photons, thus increasing their temperature. 

% The neutrinos are not updated when this process happens: their temperature then becomes consistently less than the one of the photons, but they keep scaling the same: one of their temperatures is a constant multiple of the other after this event.

After this occurs, the photons keep evolving like any other relativistic particle species, with \(\rho \propto a^{-4}\) but their temperature is higher than that of the neutrinos. Since they evolve in the same way, the ratio of the temperatures is constant.
Now we will calculate this ratio.

% First, at \SI{1}{MeV}, neutrinos decouple. Then, at \SI{0.5}{MeV}, electrons and positrons decouple.
We impose continuity of the entropy in a comoving volume across the transition which happens at \SI{0.5}{MeV} between the stage in which electrons, positrons and photons are in equilibrium and the stage in which they decouple, since the electrons and positrons are not relativistic anymore and thus annihilate.

This transition only affects the temperature of the photons, while the neutrinos decoupled three seconds earlier; so, before the transition the temperatures of photons and neutrinos are equal, after it the photons' temperature increases.

% We require continuity at this transition: \(T_{>}\) and \(T_{<}\) must both be \(T_{e}\).  
% %
% \begin{equation}
%   s a^3 = 
%   \frac{2 \pi^2}{45} g_{*s>}T^{3}_{>} a^{3}_{>} = 
%   \frac{2 \pi^2}{45} g_{*s<}T^{3}_{<} a^{3}_{<}
% \,,
% \end{equation}
% %
% but we can drop all the factors we know to be continuous on the boundary.
Let us denote with an index \(>\) the quantities pertaining to an earlier time, \(t _{\text{transition}} > t\), while an index \(<\) will denote the quantities pertaining to a later time, \(t _{\text{transition}} < t\). 

The ``updated'' version of Tolman's law reads \(T a g_{*s}^{1/3} = \const\),  and if the transition is fast enough the scale factor can be taken to be equal on both sides of it.
Therefore, we have 
%
\begin{equation}
  T_{<} = \qty(\frac{g_{*s>}}{g_{*s<}})^{1/3} T_{>}
\,,
\end{equation}
%
so we can compute the temperature after the transition, \(T_<\), if we find the effective degrees of freedom before and after: before the transition we have photons, electrons and positrons. Photons have two polarization, and so do both electrons and positrons; also, the latter are fermions, so we find
%
\begin{equation}
  g_{* s>} = 2 + \frac{7}{8}\qty(4) = \frac{11}{2}
\,,
\end{equation}
%
while after the transition only photons are relativistic, so we have 
%
\begin{equation}
  g_{*s<} = 2
\,.
\end{equation}

This means that the temperature of the photons increases by a factor
%
\begin{equation}
  T_{<} = \qty(\frac{11}{4})^{1/3} T_{>} \approx \num{1.4} T_>
\,,
\end{equation}
%
which allows us to compute the neutrino temperature at any time, since \(T_< / T_> = T_\gamma / T_\nu \), and because they scale in the same way:
%
\begin{equation}
  T_{\nu } = T_{\gamma } \qty(\frac{4}{11})^{1/3}
\,.
\end{equation}

Right now, the temperature will be around \(T_{0 \nu } \approx (4/11)^{1/3} T_{0 \gamma } \approx \SI{1.94}{K} \), where \(T_{0 \gamma }\) is the current CMB temperature. 

As an exercise, let us compute the number of effective degrees of freedom some time after the decoupling of electrons, say at \(T = \SI{0.1}{MeV}\).
The global temperature \(T\) we are referring to is the one of the photons: so, applying the definition we find
%
\begin{subequations}
\begin{align}
  g_{*} &=  \sum _{i \in BE} g_i \qty(\frac{T_i}{T})^{4} + 
  \frac{7}{8}\sum _{i \in FD} g_i \qty(\frac{T_i}{T})^{4}  \\
  &= 2 + \frac{7}{8} \qty(3 \times 2 \qty(\frac{T_{\nu }}{T_{\gamma }})^{4})  \\
  &= 2 + \frac{21}{4} \qty(\frac{4}{11})^{4/3} \approx 3.36
\,,
\end{align}
\end{subequations}
%
since we need to consider neutrinos (of which there are three flavors, each having two polarization states), which contribute to the total energy density, but not electrons which are not relativistic anymore.

With this result, we can find the energy density of radiation at that time according to \eqref{eq:radiation-energy-density-effective-dof}: we get 
%
\begin{align}
\rho_r (\SI{.1}{MeV}) \approx \SI{1.1e-4}{MeV^4} = \SI{25.7}{g cm^{-3}}
\marginnote{Multiplied by \(\hbar^{-3}c^{-5}\) to get the CGS units.}
\,.
\end{align}
%


% \medskip
% \begin{center}
%     ***
% \end{center}
% \medskip

% Last hour we discussed the Planck mass \(m_P \approx \SI{1.2e19}{GeV}\): it also defines a wavelength, \(\lambda_{P} \approx \SI{e-33}{cm}\) and a timescale \(t_{P} \approx \SI{e-43}{s}\) (both of these are \(1/m_P\) in natural units).

% When the age of the universe is of this order, our theories are not guaranteed to work. 

\section{Problems with the Hot Big Bang model, inflation}

Around the 1960s, cosmologies were trying to piece together a description of the early universe in terms of particle physics, as we discussed in this chapter up to here.
However, soon it became apparent that the standard cosmological model in use has some inconsistencies.
Let us now explore these. 

\subsection{The cosmological horizon problem}

This was noticed as early as 1956. 
Let us consider radial null geodesics in a universe described by a FLRW metric. These are the worldlines of photons we can detect with telescopes.
Imposing \(\dd{s}^2 = 0 \) we find: 
%
\begin{align}
  c^2 \dd{t^2} = a^2(t) \frac{1}{1-kr^2} \dd{r^2} 
\,,
\end{align}
%
which we can integrate (taking one of the two solutions for simplicity, choosing one over the other just amounts to parametrizing time in the opposite direction) to find
%
\begin{align}
  \int_0^{t} \frac{c \dd{t}}{a(t)} = \int_0^r \frac{ \dd{\widetilde{r}} }{\sqrt{1-k \widetilde{r}^2}} = f(r)
\,.
\end{align}

The function \(f(r)\) gives us the proper distance between emission and detection of a photon, but it does so in terms of the adimensional coordinate \(r\): in order to get something which has the dimensions of a length we need to multiply by \(a\) calculated at a certain time,\footnote{Note that this choice is arbitrary: we are computing the comoving distance \emph{as measured at the cosmic time of detection}.}
%
\begin{align}
  d _{\text{hor}} (t) = a(t) \int_0^{t} \frac{c \dd{\widetilde{t}}}{a(\widetilde{t})}
\,.
\end{align}

\textbf{If this integral is convergent, we should be worried}: let us see why.

If we integrate from the beginning of time to now, we get the spatial (current) comoving distance elapsed by a photon which started moving at the start of time.
This is the radius of the largest region we could in principle observe. It is of the order of \SI{3}{\giga\parsec}.
Since the integral is convergent this is finite, and it is increasing as time passes. So, ever-further regions are ``coming into view'' (at least in principle). 
Roughly speaking, the issue is that the regions which we start seeing at the edges should be causally disconnected from the ones already in view, so we would not expect them to exhibit the same properties --- but \emph{they do}.
This is the basic idea, let us formalize it slightly and connect it to observations.

We cannot actually see light coming from the very edge of the in-principle-observable universe, since for redshifts larger than \(z _{LS} \approx 1100\) the universe was opaque to electromagnetic radiation. The surface of points at this redshift is called the \emph{Last Scattering} surface. So, we refer our expectations to the CMB, which was emitted as the primordial plasma became transparent. 

% This is not the case, for example, in Minkowski spacetime.

% If there is an end of time, there is also a future horizon.

% The quantity \(d _{\text{Hor}} (t) \) is increasing in time: this causes the Cosmological Horizon problem.

% \todo[inline]{What is the problem?}

% An important moment is the \emph{recombination of hydrogen}: the formation of the first Hydrogen atoms, so the first moment at which Compton scattering can occur.
% This causes the decoupling of radiation and baryionic matter.
% This is the moment at which the radiation in the CMB was emitted.

The CMB was emitted at a cosmic time of \(t \approx \SI{3.8e5}{yr}\) after the Big Bang.
It is observed to be very close to being uniform, with \(\Delta T / T \sim \num{e-5}\) after correcting for the Doppler dipole modulation: 
it looks like a distribution emitted by matter in thermal equilibrium.
Crucially, this holds at any angular scale we choose: the equilibrium is there across the whole sphere.

Recall that the \emph{angular diameter distance} \(d_A\) is defined so that if an object with linear size at emission \(\Delta x\) spans an angle \(\Delta \theta \) then we have (in the small-angle approximation) 
%
\begin{align}
d_A = \frac{\Delta x}{\Delta \theta }
\,.
\end{align}

The angular diameter distance to the last scattering surface  is approximately \(d_A (z_{LS}) \approx \SI{12.8}{\mega\parsec}\).
On the other hand, the scale of the particle horizon at that redshift can be calculated by taking the difference of comoving distances to us, \( d_C(z_\infty) - d_C(z_{LS}) \approx \SI{281}{\mega\parsec}\) and multiplying it by the scale factor, \(a(z_{LS}) \approx \num{9.2e-4}\), which yields a horizon scale of approximately \(r_H \approx \SI{260}{\kilo\parsec}\) at that time.\cprotect\footnote{All the calculations were made automatically using the \texttt{astropy} package, using a flat \(\Lambda \)CDM model with parameters obtained from the Planck mission \cite[]{PlanckCollaboration:2016XIII}.
\begin{lstlisting}[language=Python]
from astropy.cosmology import Planck15 as cosmo
import numpy as np
import astropy.units as u
z_LS = 1089
dx = (cosmo.comoving_distance(np.inf) - cosmo.comoving_distance(z_LS)) * cosmo.scale_factor(z_LS)
dA = cosmo.angular_diameter_distance(z_LS)
(dx / dA).to(u.degree, equivalencies=u.dimensionless_angles())
\end{lstlisting}}

We can then say that \(\Delta x \sim r_H\), therefore the angular scale at which we expect to be able to observe correlations since there can be causal connections is around 
%
\begin{align}
\Delta \theta \approx \frac{ \Delta x}{d_A} \approx \SI{0.02}{rad} \approx \SI{1.2}{\degree}
\,.
\end{align}

\todo[inline]{A similar calculation \cite[eqs.\ 8--12]{tojeiroUnderstandingCosmicMicrowave} yields \(\Delta \theta \sim (1+z_{LS})^{-1/2} \approx \SI{1.7}{\degree}\), using the (reasonable) assumption of matter dominance in the epoch of recombination. Some steps there are not really clear to me, so I'm not sure whether my line of reasoning is equivalent (and valid) besides the assumption.}

This is in stark opposition with the scale of observed correlations, which span the whole sky! 

% Let us consider the past light cones from two points diametrically opposite with respect to us, at this point in time: they do not overlap, so the CMB cannot be causally correlated. However, it seems like it is!

% \todo[inline]{What is the angular scale at which the light cones overlap?}

% We have a classification by Bianchi of non-isotropic universes (in 9 classes).

\todo[inline]{Mention in the lecture of the Mixmaster Universe by Misner (and the Bianchi classification of Lie Algebras for context) as an alternative to inflation --- is this relevant here?}

% There is the Mixed Master Universe.

% Let us consider 
% %
% \begin{align}
%   R _{\text{Hor}} = \int_0^{t} \frac{c \dd{\widetilde{t}} }{a(\widetilde{t})}
% \,.
% \end{align}

\paragraph{Cosmic inflation}

Now, if the quantity \(d _{\text{Hor}}(t)\) were to diverge this would mean that we could have a causal connection with any point in the universe, provided we went far enough back in time.

We can approximate 
%
\begin{align}
  d _{\text{Hor}} (t) = a(t) \int_0^{t} \frac{c \dd{\widetilde{t}} }{a(\widetilde{t})} \approx ct \sim \frac{c}{H} \equiv
  d _{\text{H}}
\,,
\end{align}
%
where we defined the new \emph{Hubble distance}, \(d_H = c/H\).
This is a physical distance, but we can also define the corresponding
dimensionless comoving Hubble radius: \(r_H = c/(Ha) = c / \dot{a} \), which satisfies \(d_H = a r_H\).

We can hypothesize that there was a period in the early universe when the comoving radius \(r_H\) was decreasing with time: if this is the case, the regions we are observing today as ``coming into view'' could actually have been in causal contact in the early universe.

For the comoving radius to be decreasing (\(\dot{r}_H\) < 0), the condition is (neglecting factors of \(c\), or working in natural units):
%
\begin{align}
  \dot{r}_H = - \frac{ \ddot{a} }{\dot{a}^2} <0
\,,
\end{align}
%
therefore we need \(\ddot{a} >0\) for at least some time.

The second Friedmann equation (in natural units) tells us that
%
\begin{align}
  \ddot{a} = -\frac{4 \pi G }{3} \qty(\rho + 3P)
\,,
\end{align}
%
therefore the condition we need to have is \(\rho + 3P <0\).
So, since the energy density is positive, the condition is \(P < - \rho /3\).

\paragraph{Types of inflation}

Another way to express the parameter \(\ddot{a}\) is as the derivative of \(\dot{a} = Ha\):
%
\begin{align}
  \ddot{a} = \dot{a}H + a \dot{H} = a (H^2 + \dot{H})>0
\,,
\end{align}
%
so the condition can also be expressed as \(H^2+ \dot{H} >0\).

We have shown earlier that, neglecting curvature, the time-dependence of the scale factor looks like
%
\begin{align}
  a(t) = a_{*} \qty(1 + \frac{3}{2}\qty((1+w)H_{*}\qty(t-t_{*})))^{2/ (3(1+w))}
\,,
\end{align}
%
and it is reasonable to use this result since, as we will discuss in this section, inflation makes curvature negligible. 

We can characterize the solutions based on the sign of \(\dot{H}\), which determines whether the equation of state parameter \(P / \rho = w\) is larger or smaller than \(-1\): the possibilities are
\begin{enumerate}
    \item \(\dot{H} < 0\) while \(\dot{H} + H^2 >0\): this corresponds to \(-1 < w < -1/3\), which in General Relativity-speak is called a \emph{violation of the weak energy condition}, and inserting it in our general solution gives \(a(t) \propto t^{\alpha }\) for some \(\alpha > 0\), a \textbf{power-law inflation};
    \item \(\dot{H} =0\): this is a De Sitter, dark-energy dominated universe, whose scale factor evolves like \(a(t) = \exp(Ht) \), corresponding to \(w = -1\);
    \item \(\dot{H} >0\) (and so also \(\dot{H} + H^2 > 0\)),
    which corresponds to \(w<-1\) and which gives us \(a(t) \propto (t - t _{\text{bounce}})^{-\alpha }\) with \(\alpha > 0\), a singularity in the future. 
\end{enumerate}

% Big Rip singularity: with \(a(t) = \abs{t - t_{as}}^{- \alpha } \) with \(\alpha >0\). This is called poli-inflation.
The boundary at \(w = -1\) is called the \emph{phantom divide}.

\paragraph{An estimate of the inflation \(e\)-foldings.}

By how much does the early universe need to inflate? 
The condition we need to impose is that the comoving radius of the universe at some early time, \(r_H (t _0)\), should be larger than the current one, \(r_H (t_0 )\). Since \(r_H = d_H / a\), we can write this inequality as 
%
\begin{align}
\frac{d_H (t _{\text{in}})}{a(t _{\text{in}})} a(t_f)
\geq 
\frac{d_H (t _0)}{a(t _0)} a(t_f)
\,,
\end{align}
%
where we multiplied both sides by the scale factor calculated at \(t_f\), a time corresponding to the end of inflation, a minimum for the scale factor. Let us define \(Z _{\text{min}} = a(t_f) / a(t _{\text{in}})\).
This will be \(\gg 1\), and it will describe by how much the universe inflated. 

In our rough approximation \(d_H \sim H^{-1}\), so we can say that the boundary of the inequality, the minimum inflationary expansion, will be
%
\begin{align}
Z _{\text{min}} &= \frac{d_H (t_0 )}{d_H (t _{\text{in}})}
\frac{a(t_f)}{a(t_0 )}  \\
&= \frac{H(t _{\text{in}})}{H(t_0 )} \frac{a(t_f)}{a(t_0 )}  \\
&= \frac{H(t _{\text{in}})}{H(t_f )}
\frac{H(t_f)}{H(t_0 )} \frac{a(t_f)}{a(t_0 )}  \\
Z _{\text{min}} \frac{H_f}{H _{\text{in}}} &= \frac{H_f}{H_0} \frac{a_f}{a_0} \marginnote{Denoting \(H(t_i) \equiv H_i\) and similarly for \(a\).}
\,.
\end{align}

Now, we want to put some numbers into this expression: we know from the first Friedmann equation that (as long as there is no spatial curvature) \(H^2 \propto \rho \), while the third tells us that \(\rho \propto a^{-3 (1+ w)}\): therefore, \(H \propto a^{- 3 \frac{1 + w}{2}}\). Since we are working with ratios, proportionality is all we need. 
Now, what \(w\) should we use? At any stage in the evolution of the universe there are several fluids, but in order to simplify the calculation we will only consider the dominant one and neglect dark energy in the current phase of the evolution of the universe. 
In the inflationary phase we will have an undetermined \(w = w _{\text{inf}}\), so that 
%
\begin{align}
\frac{H_f}{H _{\text{in}}} = \qty(\frac{a_f}{a _{\text{in}}})^{- 3\frac{1+w}{2}}
\,,
\end{align}
%
so the left-hand side of the equation reads 
%
\begin{align}
Z _{\text{min}} \frac{H_f}{H _{\text{in}}} 
= Z _{\text{min}}^{1 - 3 \frac{1+w _{\text{inf}}}{2}} 
= Z _{\text{min}}^{\frac{-1 - 3w _{\text{inf}}}{2}} 
= Z _{\text{min}}^{\abs{\frac{1 + 3w _{\text{inf}}}{2}}} 
\,,
\end{align}
%
since \(w _{\text{inf}} > - 1/3 \) means \(1 + 3w _{\text{inf}} > 0\).
The right-hand side has precisely the same form, so we can express it as 
%
\begin{align}
\frac{H_f}{H_0 } \frac{a_f}{a_0 } = \qty(\frac{a_f}{a_0 })^{- \frac{1 + 3 w}{2}}
\,,
\end{align}
%
where \(w\) is that of the dominant fluid from the end of inflation to now. The issue is, there is not a single one! In the early stages radiation was dominant, then matter started dominating (now dark energy is dominant, but we shall not worry about it).
So, we split the term in two, with the radiation-matter equality being the breaking point. The earlier radiation-dominated phase is characterized by \(w = 1/3\), while the latter matter-dominated phase is characterized by \(w = 0\), so we can compactly write the term as 
%
\begin{align}
\frac{H_f}{H_0 } \frac{a_f}{a_0 } = \qty(\frac{a_f}{a _{\text{eq}}})^{- \frac{1 + 1}{2}} \qty(\frac{a _{\text{eq}}}{a_0 })^{- \frac{1 + 0}{2}} 
= \qty(\frac{a _{\text{eq}}}{a_f}) \qty(\frac{a_0}{a _{\text{eq}}})^{1/2}
\,.
\end{align}
%

With this result, we now have an almost explicit expression for \(Z _{\text{min}}\): 
%
\begin{align}
Z _{\text{min}}
= \qty[\qty(\frac{a _{\text{eq}}}{a_f}) \qty(\frac{a_0}{a _{\text{eq}}})^{1/2}]^{\abs{ \frac{2}{1 + 3 w _{\text{inf}}}}}
= \qty[\qty(\frac{a _0}{a_f}) \qty(\frac{a_0}{a _{\text{eq}}})^{-1/2}]^{\abs{ \frac{2}{1 + 3 w _{\text{inf}}}}}
\,.
\end{align}

The ratio \(a_0 / a _{\text{eq}}\) can be written as \(1 + z _{\text{eq}} \approx \num{2.3e4} \Omega h^2 \approx \num{e4}\) (very roughly).
The other rough estimate we make is to apply Tolman's law, so that 
%
\begin{align}
\frac{a_0 }{a_f} \approx \frac{T_f}{T_0 } = \frac{T_f}{m_p} \underbrace{\frac{m_p}{T_0 }}_{\sim \num{e32}}
\,;
\end{align}
%
properly speaking this only holds when there is radiation dominance so it does not apply for the whole range in which we are applying it (up to today), but we are estimating the order of magnitude \emph{of an exponent}, so even an order-of-magnitude error is not an issue.
We normalized by the Planck mass (or temperature, since we are using natural units) since the temperature \(T_f\) at the end of inflation probably was of that order of magnitude. 
The final estimate we get is 
%
\begin{align}
Z _{\text{min}} \approx \qty[\num{e30} \frac{T_f}{m_p}]^{\abs{ \frac{2}{1 + 3 w _{\text{inf}}}}}
\,.
\end{align}

We do not know what \(w _{\text{inf}}\) is besides it being smaller than \(- 1/3\); let us say that it is of the order of \(-1\) like the current dark-energy dominated phase.
If we further assume that \(T_f / m_p \sim 1\),\footnote{The constraints on \(T_f\) are the parameters of baryogenesis (we can observe it through the isotope distribution early galaxies) and the fact we have not detected gravitational waves from this early time. These tell us that \(T_f / m_p\) cannot be larger than \num{e-3}.} we find \(Z _{\text{min}} \sim \num{e30} \approx \exp(70)\). 

This is often written as ``70 \(e\)-foldings'', meaning 70 \(e\)-fold increases in size. 
 
% \todo[inline]{What is this about?}

% The third condition is very hard to achieve.

\subsection{The flatness problem}

Now, let us consider the \emph{flatness problem}, which was first proposed by Dicke and Peebles in 1986.

\todo[inline]{Cannot seem to find the paper or article\dots}

% The parameter \(\Omega = \frac{8 \pi G \rho(z) }{3 H^2(z)} \) diverges from 1 as time increases.
% Measurements of \(\Omega _{\text{tot}}\) gave approximately 0.1.

% The first Friedmann equation gives \(H^2\) in terms of \(\rho \) and \(k/a^2\); the first term scales like \(\rho \propto a^{-3(1+w)}\), while the second scales like \(a^{-2}\), so as long as \(w > - 1/3\) (which is the case since radiation dominance) the density term is dominant over the spatial curvature term. 
% Therefore, as \(a\) increases the 
The first Friedmann equation can be rearranged as 
%
\begin{align}
\underbrace{\frac{3 a^2 H^2}{8 \pi G}}_{= a^2 \rho_C} &= \rho a^2 - \frac{3k}{8 \pi G}  \\
(\rho_C - \rho ) a^2 &= 
\qty( \frac{1}{\Omega } - 1) \rho a^2
= - \frac{3k}{8 \pi G} = \const
\,,
\end{align}
%
where \(\rho _C = 3 H^2 / 8 \pi G\) is the critical density. The right-hand side only contains constants, therefore the left-hand side must be constant as well.
As the universe evolves \(\rho \) decreases while \(a\) increases; however \(\rho \propto a^{-3 (1+w)}\) by the third Friedmann equation, meaning that the term scales like \(\rho a^2 \propto a^{-1 - 3 w}\): as long as \(w > -1/3\), which is the case for most of the universe's evolution (with radiation and matter dominance) the term is decreasing, meaning that the other term must be increasing. 

If \(k = 0\) we also have \(\Omega \equiv 1\), so the point is moot, however as we have already mentioned there are reasons to think this is unlikely. 

So, let us consider the case \(k = \pm 1\): then, the term \(\Omega^{-1}- 1\) must scale like \(a^{1 + 3 w}\) in order to balance the other. 

% Let us assume that \(w\) is constant. Then, \(\rho (z) = \rho_{0} (1+z)^{3(1+w)}\). Recall that 
% %
% \begin{subequations}
% \begin{align}
%   H^2(z) &= \frac{8 \pi G }{3} \rho (z) - \frac{k}{a^2}  \\
%   H_0^2 &= \frac{8 \pi G}{3}\rho_0 - \frac{k}{a_0^2}
% \,,
% \end{align}
% \end{subequations}
% %
% the latter of which implies \(1 = \Omega - k / (a_0^2 H_0^2)\). So, 
% %
% \begin{subequations}
% \begin{align}
%   H^2(z) &= H_0^2 \qty(\frac{8 \pi G }{3 H_0^2} \rho (z) - \frac{k}{a^2 H_0^2}) \\
%   &= H_0^2 \qty(\frac{\rho (t)}{\rho_0 }\Omega_0 \frac{a_0^2}{a} (1-\Omega_0 ))  \\
%   &= H^2(z) = H_0^2 (1+z)^2 \qty(\Omega_0 (1+z)^{1+3w} + (1-\Omega_0 ))
% \,,
% \end{align}
% \end{subequations}
% %
% so in the end 
% %
% \begin{subequations}
% \begin{align}
%   \Omega (z) &= \frac{8 \pi G \rho_0 }{3 H_0^2 } \frac{\rho (z)}{\rho_0 } \frac{H_0^2}{H^2(z)} \\
%   &=\Omega_0 (1+z)^{1+3w} \qty(1-\Omega_0 + \Omega_0 (1+z)^{1+3w}) 
% \,,
% \end{align}
% \end{subequations}
% %
% and since \(\rho (z) = \rho_0 (1+z)^{3(1+w)}\) we get 
% %
% \begin{subequations}
% \begin{align}
%   \Omega^{-1} (z) -1 &= \frac{\Omega_0 \qty((1-\Omega_0 ) + \Omega_0 (1+z)^{1+3w}) - (1+z)^{1+3w} }{(1+z)^{1+3w}}  \\
%   &= (\Omega_0^{-1} -1) (1+z)^{-(1+3w)}
% \,.
% \end{align}
% \end{subequations}

If we assume \(w = 1/3\) for all times we get (using the fact that \(a \propto (1+z)^{-1}\) and Tolman's law) that the term calculated at a redshift \(z\) is given by:
%
\begin{align}
  \Omega ^{-1} (z) -1
  = (\Omega_0 ^{-1} -1) \qty(1+z)^{-2}
  = (\Omega_0 ^{-1} -1) \qty(\frac{T_0 }{T(z)})^{2}
\,.
\end{align}

\todo[inline]{The reasoning in \cite[]{paccianiAppuntiCorsoPhysical2018} looks way more complicated, but this way seems just as valid\dots}
The assumption \(w \equiv 1/3\) is not really correct, but the result would only change by a few orders of magnitude if we did the calculation properly, and as before we are giving only a rough estimate \emph{of an exponent}.

Let us extend this line of reasoning back to the Planck epoch, since beyond that our theory of gravity might behave differently so it is not justified to apply Friedmann's equations. 

If we compute \(T _{\text{Pl}} / T_0 \) we get approximately \num{e32}.
% Then when squaring we get \num{e-64}, without the approximation \(w=1/3\) we get \num{e-60}.
This means that 
%
\begin{align}
\Omega^{-1}(z_{\text{Pl}}) - 1 \approx (\Omega_0^{-1} - 1) \num{e-64}
\,.
\end{align}

The correct calculation, keeping track of the dominant fluid in each phase, gives \(\num{e-60}\) instead of \(\num{e-64}\): not that significant a difference when discussing these kinds of numbers.

The current estimate for \(\Omega_0 \), as discussed in section \ref{sec:global-omega-contributions}, is \emph{at most} of the order of \(\num{e-3}\) away from 1 --- therefore, in the Planck epoch the parameter would have needed to be different from 1 by a part in \num{e-63}. 

% Then, we see that there is something deeply unnatural in the Friedmann model.
% How is it possible that the universe is still so flat, even when the universe is so old? Oldness and flatness seem incompatible.
This is a type of \emph{fine-tuning} problem: the initial conditions seem to require an ``unnatural'' number like \(\Omega \sim 1 \pm \num{e-63}\).
Typically, a fine-tuning problem is interpreted as a signal that we should improve our theory.\footnote{This approach has also received criticism \cite[sec.\ 3.2]{hossenfelderScreamsExplanationFinetuning2019}: we do not know what distribution the initial conditions are drawn from, so how can we say whether a certain number is likely (or ``natural'') than another?}

\end{document}