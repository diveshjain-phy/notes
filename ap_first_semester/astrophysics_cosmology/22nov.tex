\documentclass[main.tex]{subfiles}
\begin{document}

\marginpar{Friday\\ 2019-11-22, \\ compiled \\ \today}
% \section*{Fri Nov 22 2019}

This means that the equation reads
%
\begin{align}
  \dot{n} + 3 \frac{\dot{a}}{a} n = \expval{\sigma_A v} \qty(n _{\text{eq}}^2 -  n^2)
\,.
\end{align}
%
% where the term \(\Psi \) is the source of new particles, while the next term accounts for the annihilation of particles. 

% Under decoupling, the equation reduces to 
% %
% \begin{align}
%   \dot{n} + 3 \frac{\dot{a}}{a} n = 0 \implies \dv{}{t} \qty(n a^3) = 0 
% \,.
% \end{align}
% %

% Under equilibrium, \(\Gamma > H\).

% Therefore \(\Psi = \expval{\sigma _A v} n^2 _{\text{eq}}\). 
% Recall that the \(\Gamma \) of annihilation is equal to \(\Gamma  = \expval{\sigma _A v} n\). 

% If the timescale of the collisions is longer than the age of the universe, \(\Gamma < H\), then once again we have \(a^3n = \const\). 

% We can see thermal \emph{relics}, thermal distributions of objects which are not coupled anymore. 

Since the number density \(n\) itself is not conserved, we define the \emph{comoving} number density, which is conserved if there is equilibrium
%
\begin{align}
  n_C = n \qty(\frac{a}{a_0 })^3
\,,
\end{align}
%
for some arbitrary initial scale factor \(a_0 \).
With this, we can simplify the left-hand side: 
%
\begin{align}
  \dot{n} + 3 \frac{\dot{a}}{a} n 
  &= \dv{}{t} \qty[n_C \frac{a_0^3}{a^3}] + 3 \frac{\dot{a}}{a} n_C \frac{a_0^3}{a^3} = \dot{n}_C \frac{a_0^3}{a^3} 
  + n_C \qty(-\frac{3 \dot{a}}{a^2}) \frac{a_0^3}{a^2}
  + 3 \frac{\dot{a}}{a} n_C \frac{a_0^3}{a^3}  \\
  &= \dot{n}_C \qty(\frac{a_0 }{a})^3
\,.
\end{align}

Similarly, we can express the right-hand side in terms of comoving densities: \(\expval{\sigma_A v}\) is the same, while 
%
\begin{align}
n _{\text{eq}}^2 - n^2 = \qty(n _{C, \text{eq}}^2 - n_C^2) \frac{a_0^{6}}{a^{6}}
= n _{C, \text{eq}}^2 \frac{a_0^{6}}{a^{6}} 
\qty[1 - \frac{n_C^2}{n _{C, \text{eq}}^2}]
\,,
\end{align}
%
so the equation will read:
% We want to factor our a parameter \(H\).
% We get: 
%
\begin{align}
  \dot{n}_C \frac{a_0^3}{a^3} 
  &=
  -\expval{\sigma _A v} 
  n _{C, \text{eq}}^2 \frac{a_0^{6}}{a^{6}} 
\bigg[\underbrace{\frac{n_C^2}{n _{C, \text{eq}}^2}}_{\mathclap{\text{The \(a^3\) factors simplify}}} - 1\bigg] 
\\
  \dot{n}_C &= - 
  \expval{\sigma _A v } \underbrace{\qty(\frac{a_0 }{a})^3 n _{C, \text{eq}}^2}_{ = n_{C, \text{eq}} n _{\text{eq}}} \qty[\frac{n^2}{n _{ \text{eq}}^2} - 1] 
\,.
\end{align}

We want to write this in a more intuitive way; the derivative with respect to time can be expressed in terms of the scale factor, as 
%
\begin{align}
\dv{}{t} = \dot{a}\dv{}{a} = H a \dv{}{a}
\,.
\end{align}

With it, we can express the equation as 
%
\begin{align}
\frac{a}{n _{\text{C, eq}}} \dv{n_C}{a} = - \frac{\expval{\sigma _A v }n _{\text{eq}}}{H} \qty[\qty(\frac{n}{n _{\text{eq}}})^2- 1]
\,.
\end{align}
%

The ratio before the parenthesis has an intuitive physical meaning:
the characteristic time of the collisions which can annihilate this particle species is \(\tau _{\text{coll}} = 1/ \Gamma =  1 / (\expval{\sigma _A v} n _{\text{eq}})\), while the timescale of the expansion of the universe is \(\tau _{\text{exp}} = 1/H\), we get 
\boxalign{
\begin{align}
   \frac{a}{n _{\text{C, eq}}} \dv{n_C}{a}
  = - \frac{\tau _{\text{exp}}}{\tau _{\text{coll}}} \qty[\qty(\frac{n}{n _{\text{eq}}})^2 -1 ]
\,.
\end{align}}

This allows us to characterize decoupling in a much more specific way than before.
\begin{enumerate}
  \item If \(\Gamma \gg H\), then \(\tau _{\text{exp}} \gg \tau _{\text{coll}} \), therefore \(n \approx n _{\text{eq}}\), which also implies \(n _{\text{C}} = n _{\text{C, eq}}\) (this quantity can vary!). This is the equilibrium case, the particles are \textbf{coupled}. 
  \item If \(\Gamma \ll H\), then \(\tau _{\text{exp}} \ll \tau _{\text{coll}} \) we have \textbf{decoupling}, and \(n_C = \const\), therefore \(n \propto a^{-3}\).
\end{enumerate}
\todo[inline]{\textcite[]{Pacciani:2018} writes that in the coupled case the comoving density is constant, but this will not be the case in general: allowed annihilation processes may change over time, as other particle species decouple!}

% As opposed to the qualitative description 
This approach is much more powerful than the qualitative one we gave earlier, since while the limiting cases are the same this equation allows us to also treat all the intermediate situations.

Let us apply this to both HDM and CDM.

\subsection{HDM density estimate}

Neutrinos are a candidate for HDM: we know that at temperatures below \(T_d = \SI{1}{MeV}\) they decoupled, after which --- as we have seen earlier --- their temperature\footnote{We call it ``temperature'' for clarity, but properly speaking it is not one, since ever since decoupling neutrinos are not thermal. It is better understood as the ``temperature parameter'' of the neutrinos' phase space distribution.} evolved like:
%
\begin{align}
  T_{\nu } = \qty(\frac{g_{* \text{after decoupling}}}{g_{* \text{before decoupling}}})^{1/3} T_{\gamma }
\,,
\end{align}
%
% and we want to do the same for dark matter. 
and so this scaled like \(T_\nu \propto a^{-1}\), while their number density scaled like \(n_\nu \propto a^{-3} \propto T_\nu^{3}\).

% Neutrinos are nonrelativistic today, but they became so at a relatively low redshift, a short time ago. 
% We can parametrize the number density of neutrinos by the temperature. 

% The formula for the temperature of neutrinos is a special case of the following formula: 
% %
% \begin{align}
%   T_{0 \nu } = \qty(\frac{g_{*0}}{g_{*d}})^{1/3} T_{0 \gamma }
% \,,
% \end{align}
%
% where \(0\) means \emph{now}, while  \(d\) means \emph{decoupling}. This can be applied to any species. 

There is nothing special about neutrinos: we can apply the same line of reasoning to a generic HDM species \(x\), whose number density today will be then given by
%
\begin{align}
  n_{0x} = B g_{x} \frac{\zeta (3)}{\pi^2} T_{0x}^3
\,,
\end{align}
%
where the factor \(B\) accounts for the statistics: it is 1 for bosons, \(3/4\) for fermions. The parameter \(g_x\) is the number of degrees of freedom \emph{of the particle species \(x\)}, 
% It is important to note that the temperature is a parameter, but these particles are \emph{not thermal} anymore! 

We can rescale this in terms of the photon number density, which is given by the same expression, with \(B = 1\) and \(g_x = 2\): using the 
% So, we get, using equation \eqref{eq:n-gamma}:
%
\begin{align}
\frac{n_{0x}}{n_{0\gamma }} &= \frac{B g_x \frac{\zeta (3)}{\pi^2} T^3_{0x}}{2 \frac{\zeta (3)}{\pi^2} T^3_{0 \gamma }} \\
n_{0x} &= \frac{B}{2} n_{0 \gamma } g_x \frac{g_{*0}}{g_{*dx}}
\,,
\end{align}
%
where \(g_{*0}\) is the current amount of effective degrees of freedom, while \(g_{*dx}\) is the same quantity, computed at the decoupling time of particle \(x\). 

\todo[inline]{It should not be \(g_{*0}\) though, right? we compute \(g_*\) on the two boundaries of the transition, the effective dof \emph{right now} are irrelevant, I'd think.}

The energy density, as long as today the particles in question are nonrelativistic,\footnote{Which we assume they are. This is not meant to be interpreted as an experimental fact --- for neutrinos, say, we do not actually know what their mass is, so we are not sure, although given their mass differences (which can be inferred from neutrino oscillations) at the current temperature \(T_{0 \nu } \sim \SI{200}{\micro eV}\) at least some neutrino species must be nonrelativistic. 

Rather, the point is that if the mass of one of these HDM particles were so low that they were relativistic today than their energy density would be very low, comparable to that of photons, which as we have seen earlier is negligible: they could not constitute the large fraction of the critical density \(\rho _C\) that we know DM constitutes.
In order for HDM to have a chance to be a significant constitutent of DM it must be nonrelativistic today.} is given by \(\rho_{0x} = m_x n_{0x}\), so we can write 
%
\begin{align}
  \rho_{0x} = \frac{B}{2} m_x n_{0 \gamma } g_x \frac{g_{*0x}}{g_{*dx}}
\,,
\end{align}
%
therefore the mass fraction of the HDM particle \(x\) today will be roughly
%
\begin{align}
\Omega_{0x} h^2 \approx \frac{\rho_{0x}}{\rho_{0c}} h^2 
\approx 2B g_x \frac{g_{*0}}{g_{*dx}} \frac{m_x}{\SI{e2}{eV}}
\,.
\end{align}

This, together with what we know the \(\Omega\) of dark matter to be, allows us to check whether a candidate for HDM is viable or not, based on its mass and on when it decouples.
We can already see that if the mass is lower than a few \SI{}{eV} the particle cannot make up most of the DM budget.

\subsection{CDM density estimate}

CDM is made of particles which were already nonrelativistic when they decoupled, so we can describe them using Boltzmann statistics: we keep referring to the DM candidate as \(x\), so at the temperature of decoupling \(T_{dx}\) we have
%
\begin{align}
  n_x (T_{dx}) = g_{x} \qty(\frac{m_x T_{dx}}{2 \pi })^{3/2} \exp(- \frac{m_x}{T_{dx}})
\,,
\end{align}
%
% we can assume that \(T_{dx} \ll m_x\). So, 
after which the density will scale like \(a^{-3}\), so\footnote{Applying the ``updated'' version of Tolman's law, \(T a g_{*s}^{1/3} = \const\).}
%
\begin{align}
  n_{0x} = n_x (T_{dx}) \qty(\frac{a(T_{dx})}{a_0 })^3
  = n(T_{dx}) \frac{g_{*0}}{g_{*x}} \qty(\frac{T_{0 \gamma }}{T_{dx}})^3
\,.
\end{align}

The difficulty lies in determining the decoupling temperature \(T_{dx}\), which is when the collision and expansion timescales are equal. 

The first Friedmann equation combined with the expression for the energy density (of radiation, but corrected according to the effective degrees of freedom at that time) tells us 
%
\begin{align}
  H^2 (T_{dx}) = \frac{8 \pi G}{3} g_{*dx} \frac{\pi^2}{30} T_{dx}^{4}
\,,
\end{align}
%
which we can use to estimate the expansion timescale \(\tau_{\text{exp}}= 1/H\):
%
\begin{align}
  \tau_{\text{exp}} \approx \num{.6} g_{*dx}^{-1/2} \frac{m _{\text{pl}}}{T_{dx}^2}
  \marginnote{\(\num{.6} \approx \qty(\frac{8\pi }{3} \frac{\pi^2}{30})^{-1/2}\).}
\,.
\end{align}

\todo[inline]{\textcite[]{Pacciani:2018} writes \num{.3}, am I getting the calculation wrong?}

Now, let us estimate the collision timescale: we know that its inverse is \(\Gamma = n \expval{\sigma _A v}\), and it is a fact from particle physics that the average cross section scales with the temperature like: 
%
\begin{align}
  \expval{\sigma _A v} = \sigma_0 \qty(\frac{T}{m_x})^{N}
\,,
\end{align}
%
where \(N = 0\) or \(1\), while \(\sigma_0 \) is a constant characteristic cross section of the process. So, the collision timescale is
%
\begin{align}
  \tau _{\text{coll}} (T_{dx}) = \qty(n(T_{dx}) \sigma _0 \qty(\frac{T_{dx}}{m_x})^{N})^{-1}
\,.
\end{align}

Equating the two timescales we get the following equation:
%
\begin{align}
\qty(n(T_{dx}) \sigma _0 \qty(\frac{T_{dx}}{m_x})^{N})^{-1} = \num{.6} g_{*}^{-1/2} \frac{m _{\text{pl}}}{T_{dx}^2}
\,,
\end{align}
%
which is transcendental in \(T\), since we have an exponential as well as a polynomial term in the expression for \(n_x(T_{dx})\).

We can solve it iteratively, in terms of the parameter \(x_{dx} = m_x / T_{dx}\), which are assuming to be much larger than one (in order for the procedure we have done so far to be valid, and in order for \(x\) to be CDM): this allows us to select the physical solution to the equation among the nonphysical ones. 

The solution, after the second iteration, is found to be something like: 
%
\begin{align}
  x_{dx} = \log \qty(\num{.038} \frac{g_{x}}{g_{*xd}^{1/2}} m _{\text{pl}}  m_x \sigma_0 ) - \qty(N - \frac{1}{2}) \log \log \qty(\num{.038} \frac{g_{x}}{g_{*xd}^{1/2}} m _{\text{pl}}  m_x \sigma_0 )
\,. 
\end{align}

From this we can determine the contribution of this CDM particle to the current energy density.

Qualitatively, we find that there is a very significant dependence on the mass of the particle (since we have an exponential suppression in its density) and on the way it interacts (\(\sigma_0 \)).

\chapter{Stellar Astrophysics}

\section{Stellar formation}

% Dark matter is collisionless, however it is not \emph{really}: the censorship theorem says that it cannot actually collapse into a naked singularity.
We will now discuss the formation of stars, which started happening at a redshift \(z \sim 20\).
We shall do so in a Newtonian approximation, neglecting the expansion of the universe --- later we will discuss how cosmology affects this.

A star is a gravitationally-bound sphere of plasma, inside which fusion occurs. Stars form from the gravitational collapse of instabilities, which is followed by an increase in temperature and pressure from the release of gravitational energy. 

We will model this through simple assumptions, since they already give a good picture of what this collapse looks like. Two forces are at play: gravity and pressure.\footnote{In this section we will neglect the Pauli exclusion principle, which does not allow fermionic matter to compress beyond a certain point. We will come back to this point when we discuss the Chandrasekhar mass.}
As we will see, the gravitational force initially dominates and compresses the material up to a certain point, at which the pressure prevents it from going further.

We start with a spherically symmetric region of baryonic matter, characterized by a density \(\rho (r)\): the mass enclosed in a radius \(r\) is 
%
\begin{align}
  m(r) = \int_{0}^{r} 4 \pi \widetilde{r}^2 \rho (\widetilde{r}) \dd{\widetilde{r}}
\,.
\end{align}

The modulus of the gravitational acceleration of the material at a radial coordinate \(r\) can be calculated from Gauss' theorem: 
%
\begin{align}
  g(r) = \frac{G m(r)}{r^2}
\,. 
\end{align}
%

Let us now consider a spherical shell at a radius \(r\), with its enclosed mass \(\Delta M = \rho (r) \Delta A \Delta r\), where \(\Delta A = 4 \pi r^2\).
% It can contrast the inward force due to gravity if there is a differential pressure. 
Let us denote \(P\) as the pressure at the inner surface, and \(P + \Delta P\) the pressure at the outer surface. Then, the net force on the surface is given by
%
\begin{align}
\qty(P + \Delta P) \Delta A - P \Delta A 
= \qty(P(r) + \dv{P}{r}\Delta r) \Delta A - P(r) \Delta A 
= \dv{P}{r} \Delta A \Delta r
= \dv{P}{r} \frac{\Delta M}{\rho (r)}
\,.
\end{align}

Note that this is an inward force if \(\Delta P\) is positive (and so \(\dv*{P}{r}\) also is), since then there is more pressure outside than inside. 

% The  minus sign in what follows is since the force is inward: 
The equation of motion of the spherical shell is given by \(ma = F\):
%
\begin{align}
  - \Delta M \ddot{r} &= \Delta M g(r) + \dv{P}{r} \frac{\Delta M}{\rho (r)} \\
  - \ddot{r} &= \frac{G m(r)}{r^2} + \frac{1}{\rho (r)} \dv{P}{r} 
\,,
\end{align}
%
where the minus sign comes from the two forces are positive if they are directed inward.
This means that, in order to achieve equilibrium (\(\ddot{r} = 0\)), the pressure gradient \(\dv*{P}{r} \) must be negative, since \(m(r)\) can never be.

% If the internal energy is used up to do internal (chemical, nuclear) work, then it cannot support the star anymore, and it then collapses. 
Now we shall give two estimates about stellar formation: the first is the \textbf{free-fall timescale}, in which we ignore pressure forces in order to ballpark the time taken for the matter to fall onto itself.

Then, we will study the equilibrium configuration of the star, in order to understand what are the conditions under which it is actually gravitationally bound (that is, with negative total energy).

% Usually, stars start from the Jeans phenomenon: dark matter and baryons collapse under their own weight. 

% Let us start at the decoupling of matter and radiation, something like \(z = 1100\).

% The clouds of matter collapse freely up to the moment at which their internal pressure starts to slow them down. Then, we get the necessary conditions for the fusion of Hydrogen. 

\subsection{The freefall timescale}

We are ignoring pressure (or, equivalently, assuming that it is constant), so the equation of motion reads
%
\begin{align}
  - \ddot{r} = g(r)
\,.
\end{align}

This will not generally be the case, but let us suppose that the collapse is \emph{orderly}: the ordering of the layers stays the same as they fall. 

We use the energy integral instead of directly solving the differential equation, that is, we impose energy conservation.
% We have to account for energy conservation: the total energy when our test shell is at a radius \(r\) is conserved as \(r\) changes. If the energy (per unit mass) is zero at infinity we have 
Computing the total energy (kinetic plus potential) at the initial radius of the cloud, \(r_0 \) (at which the gas is stationary), and at a radius \(r\) we get 
%
\begin{align}
  - \frac{G m_0 }{r_0 } &= \frac{1}{2} \qty(\dv{r}{t})^2 - \frac{Gm_0 }{r} \\
  \frac{1}{2} \qty(\dv{r}{t})^2 &= G m_0 \qty( \frac{1}{r} - \frac{1}{r_0 })
\,.
\end{align}

This way, we have found a first-order ODE instead of a second-order one.
Note that the potential energy at a radius \(r\) is computed using \(m_0 \) since as the layer falls the other layers below it are still below it, so the mass inside the layer is always equal to the initial one.

We can now directly compute the freefall time by integrating from \(r_0 \) to \(r\):
% What is the freefall time? we take \(\dv*{t}{r}\) from the equation an integrate it from \(r_0 \) to \(0\)
%
\begin{align}
  t _{\text{free fall}} =  \int_{r_0 }^{0} \dd{r} \dv{t}{r}
  = - \int_{r_0 }^{0} \dd{r} \frac{1}{\sqrt{2Gm_0 }}\qty(\frac{1}{r} - \frac{1}{r_0 })^{-1/2}
\,,
\end{align}
%
where we have a minus sign since, when simplifying the square of the derivative \(\dot{r}^2\) we must choose the negative sign: \(\dv*{r}{t}<0\), since the material is \emph{infalling}.

We can then change variables to \(x = r / r_0 \) (with \(\dd{r} = r_0 \dd{x}\)) and switch the bounds of integration, recovering the positive sign:
%
\begin{align}
  t _{\text{free fall}}
  &=  \frac{1}{ \sqrt{2Gm_0 }} \int_{0}^{1} r_0 \dd{x} \qty[ \frac{1}{r_0 } \qty( \frac{1}{x} - 1)]^{-1/2}  \\
  &= \sqrt{\frac{r_0^3}{2 G m_0 }} \int_{0}^{1} \dd{x} \sqrt{\frac{x}{1-x}}
  = \frac{\pi}{2} \sqrt{\frac{r_0^3}{2 G m_0 }}
\,.
\end{align}

Although the integrand diverges for \(x \to 1\) (meaning, at large \(r\)) the integral converges to \(\pi /2\).\footnote{The integral can be computed with the substitution \(x = \sin^2 \theta \) and then \(y = \cos \theta \): 
%
\begin{align}
\int_{0}^{1} \sqrt{ \frac{x}{1-x}} \dd{x} = \int_{0}^{\pi / 2} \sqrt{ \frac{\sin^2\theta}{\cos^2 \theta }} 2 \sin \theta \cos \theta \dd{\theta } = \int_{0}^{\pi /2} \sin^2\theta \dd{\theta } = \int_{0}^{1} \sqrt{1 - y^2} \dd{y} = \frac{\pi}{2}
\,.
\end{align}}

The average density is given by \(\bar{\rho} = m_0 / \qty(4 \pi r_0^3 / 3)\), which we can insert into our expression to get 
%
\begin{align} \label{eq:free-fall-time}
  t _{\text{free fall}} = \sqrt{\frac{3 \pi }{32 G \bar{\rho} }} \approx \num{.54} G^{-1/2} \overline{\rho}^{-1/2}
\,.
\end{align}

\paragraph{Comparison with the expansion timescale}

We might be tempted to ignore the expansion of the universe in these calculations: we know that
%
\begin{align}
  H^2 = \frac{8 \pi G}{3} \bar{\rho}
\,,
\end{align}
%
where we wrote \(\overline{\rho}\) to mean that this holds on the scales at which homogeneity applies; if we consider smaller scales we must take a spatial average of the density.
In the Newtonian (matter-dominated and flat) case we know that \(a(t) \propto t^{2/3}\) and \(H = 2/(3t)\). 

So, we can substitute this expression to find the expansion timescale  
%
\begin{align}
  \frac{4}{3 t^2} = \frac{8 \pi G}{3} \bar{\rho} 
  \implies 
  t^2 = \frac{4}{9} \frac{3}{8 \pi G} \bar{\rho}^{-1}
\,,
\end{align}
%
so 
%
\begin{align}
  t _{\text{exp}} \approx \num{.23} G^{-1/2} \bar{\rho}^{- 1/2}
\,.
\end{align}

If the universe was perfectly homogeneous we would then expect structure formation to be forbidden: however, if there are some over-dense regions to start with, their characteristic freefall time can become lower than the chracteristic expansion time of the universe. 

\todo[inline]{But still, shouldn't we consider expansion in the computation? By how much are we getting it wrong?}

% \todo[inline]{So, the expansion of the universe is not neglibigle and our Newtonian calculation is meaningless?
% Maybe an argument can b
% \todo[inline]{This would seem to suggest that stellar formation is not allowed under these conditions, but gravitational instability does exist}
% \todo[inline]{Add some considerations about the applicability of these calculations: we need some already over-dense regions in order to start stellar formation, right?}

% \todo[inline]{what is the relation between the universe's density and the star's?
% Is the idea: the matter in the universe does not disperse too fast, stars theoretically are allowed to form? }

\subsection{Hydrostatic equilibrium}

At equilibrium the stellar layers are static: \(\ddot{r} = 0\), so the equation of motion reads 
%
\begin{align}
  - \ddot{r} = 0 = g(r) + \frac{1}{\rho (r)} \dv{P}{r}
  \implies 
  \dv{P}{r} = - G \frac{m(r) \rho (r)}{r^2}
\,.
\end{align}

We multiply both sides by \(4 \pi r^3\) and integrate in \(\dd{r}\) from the core (\(r=0\)) to the surface of the star, \(r = R\): 
%
\begin{align}
  \int_0^{R} \dd{r} 4 \pi r^3 \dv{P}{r} = 
  - G \int_{0}^{R} \frac{m(r) \rho (r) 4 \pi r^2}{r} \dd{r}
\,,
\end{align}
%
and we can change variables: \(\rho (r) 4 \pi r^2 \dd{r} = \dd{m }\) (this is physically meaningful: it is the differential mass of the layer at a radius \(r\)), so we can identify the left-hand side with the total gravitational energy: 
%
\begin{align}
E _{\text{grav}} = -G \int \frac{m(r) \dd{m}}{r} 
\,. 
\end{align}

On the RHS, instead, we can integrate by parts: 
%
\begin{align}
\int_0^{R} \dd{r} 4 \pi r^3 \dv{P}{r} = 
  \qty[P(r) 4 \pi r^3]_0^{R} - 3 \int_{0}^{R} \dd{r} 4 \pi r^2 P(r)
\,,
\end{align}
%
where the boundary term vanishes: at the origin \(r =0\), at the surface (by definition of surface) \(P =0\).

We can better understand what this means if we divide and multiply by the volume \(V(R) \equiv 4 \pi R^3/3\): 
%
\begin{align}
  - 3 \int_{0}^{R} \underbrace{\dd{r} 4 \pi r^2}_{ \dd{V}} P(r) = - 3 V(R) \underbrace{\int_{0}^{R} \frac{ \dd{r} 4 \pi r^2   P(r)}{V(R)}}_{ \expval{P}} = - 3 V(R) \expval{P} 
\,,
\end{align}
%
where we interpret the integral as a weighted average, so we get 
%
\begin{align}
  E _{\text{grav}} = - 3 \expval{P} V 
  \qquad 
  \text{or} 
  \qquad 
  \expval{P} = -\frac{1}{3} \frac{E _{\text{grav}}}{V} = - \frac{1}{3} \rho_{GR}
\,.
\end{align}

This is the \textbf{Virial Theorem}. 

% This is Newtonian: above a certain limit, the relativistic corrections destabilize the system. 

\end{document}