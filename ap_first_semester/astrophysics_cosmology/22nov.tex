\documentclass[main.tex]{subfiles}
\begin{document}

\marginpar{Friday\\ 2019-11-22, \\ compiled \\ \today}
% \section*{Fri Nov 22 2019}

Yesterday we arrived at the following equation: 
%
\begin{align}
  \dot{n} + 3 \frac{\dot{a}}{a} n = \Psi - \expval{\sigma_A v} n^2
\,,
\end{align}
%
where the term \(\Psi \) is the source of new particles, while the next term accounts for the annihilation of particles. 

Under decoupling, the equation reduces to 
%
\begin{align}
  \dot{n} + 3 \frac{\dot{a}}{a} n = 0 \implies \dv{}{t} \qty(n a^3) = 0 
\,.
\end{align}
%

Under equilibrium, \(\Gamma > H\).

Therefore \(\Psi = \expval{\sigma _A v} n^2 _{\text{eq}}\). 
Recall that the \(\Gamma \) of annihilation is equal to \(\Gamma  = \expval{\sigma _A v} n\). 

If the timescale of the collisions is longer than the age of the universe, \(\Gamma < H\), then once again we have \(a^3n = \const\). 

We can see thermal \emph{relics}, thermal distributions of objects which are not coupled anymore. 

We define 
%
\begin{align}
  n_C = n \qty(\frac{a}{a_0 })^3
\,,
\end{align}
%
so we can simplify the expression: 
%
\begin{align}
  \dot{n} + 3 \frac{\dot{a}}{a} n = \dot{n} \qty(\frac{a_0 }{a})^3
\,.
\end{align}

We want to factor our a parameter \(H\). We get: 
%
\begin{align}
  \dot{n}_C = - \qty(\frac{a }{a_0 })^3
  \expval{\sigma _A v } \qty(\frac{a_0 }{a})^6 \qty(n_C^2-n^2 _{\text{eq}})
\,,
\end{align}
%
which can be written as 
%
\begin{align}
  \frac{a}{n _{\text{C, eq}}} \dv{n_C}{a} = - \frac{\expval{\sigma _A v }n _{\text{eq}}}{\dot{a}/a} \qty(\qty(\frac{n}{n _{\text{eq}}})^2- 1)
\,,
\end{align}
%
so, if we define \(\tau _{\text{coll}}\) as \(1 / (\expval{\sigma _A v} n _{\text{eq}})\) and \(\tau _{\text{exp}} = 1/H\), we get 
%
\begin{align}
  n _{\text{C, eq}} a \dv{n_C}{a}
  = - \frac{\tau _{\text{exp}}}{\tau _{\text{coll}}} \qty(\qty(\frac{n}{n _{\text{eq}}})^2 -1 )
\,,
\end{align}
%
so if \(\Gamma \gg H\), then \(\tau _{\text{exp}} / \tau _{\text{coll}} \gg 1\), therefore \(n = n _{\text{eq}}\), which also implies \(n _{\text{C}} = n _{\text{C, eq}}\). This is the equilibrium case. 

On the other hand, if \(\Gamma \ll H\), then \(\tau _{\text{exp}} / \tau _{\text{coll}} \ll 1\) we have decoupling, therefore \(n_C = \const\).

We know that at temperatures below \SI{1.5}{MeV} (after decoupling), the following holds: 
%
\begin{align}
  T_{\nu } = \qty(\frac{4}{11})^{1/3} T_{\gamma }
\,,
\end{align}
%
and we want to do the same for dark matter. 

Neutrinos are nonrelativistic today, but they became so at a relatively low redshift, a short time ago. 
We can parametrize the number density of neutrinos by the temperature. 

The formula for the temperature of neutrinos is a special case of the following formula: 
%
\begin{align}
  T_{0 \nu } = \qty(\frac{g_{*0}}{g_{*d}})^{1/3} T_{0 \gamma }
\,,
\end{align}
%
where \(0\) means \emph{now}, while  \(d\) means \emph{decoupling}. This can be applied to any species. 

Let us compute this for a generic species \(x\): we find 
%
\begin{align}
  n_{0x} = B g_{*} \frac{\zeta (3)}{\pi^2} T_{0x}^3
\,,
\end{align}
%
where the factor \(B\) accounts for the statistics: it is 1 for bosons, \(3/4\) for fermions. 
It is important to note that the temperature is a parameter, but these particles are \emph{not thermal} anymore! 

So, we get, using equation \eqref{eq:n-gamma}:
%
\begin{align}
  n_{0x} = \frac{B}{2} n_{0 \gamma } g_x \frac{g_{*0}}{g_{*dx}}
\,. 
\end{align}
%

The energy density in general is given by \(\rho_{0x} = m_x n_{0x}\). We get: 
%
\begin{align}
  \rho_{0x} = \frac{B}{2} m_x n_{0 \gamma } g_x \frac{g_{0x}}{g_{*gx}}
\,,
\end{align}
%
therefore 
%
\begin{align}
  \Omega_{0x} h^2 = \frac{m_x n_{0x}}{\rho_{0x}} h^2 
  = 2B g_x \frac{g_{*0}}{g_{*dx}} \frac{m_x}{\SI{e2}{eV}}
\,.
\end{align}

CDM is made of particles which were already nonrelativistic when they decoupled. 

Then, in the formula
%
\begin{align}
  n_x (T_{dx}) = g_{x} \qty(\frac{m_x T_{dx}}{2 \pi })^{3/2} \exp(- \frac{m_x}{T_{dx}})
\,,
\end{align}
%
we can assume that \(T_{dx} \ll m_x\). So, 
%
\begin{align}
  n_{0x} = n_x (T_{dx}) \qty(\frac{a(T_{dx})}{a_0 })^3
  = n(T_{dx}) \frac{g_{*0}}{g_{*x}} \qty(\frac{T_{0 \gamma }}{T_{dx}})^3
\,,
\end{align}
%
but it is difficult to compute \(T_{dx}\), which is the solution to the equation \(\Gamma (T_{dx}) = H (T_{dx})\). 

We know that 
%
\begin{align}
  H^2 (T_{dx}) = \frac{8 \pi G}{3} g_{*x} \frac{\pi^2}{30} T_{dx}^{4}
\,,
\end{align}
%
or, in terms of the quantity \(\tau_{\text{exp}}= 1/H\):
%
\begin{align}
  \tau_{\text{exp}} = 0.3 g_{*}^{-1/2} \frac{m _{\text{pl}}}{T_{dx}^2}
\,.
\end{align}
%
On the other hand, \(\Gamma = n \expval{\sigma _A v}\), and \(\tau _{\text{coll}} ( T_{dx})\), and in terms of \(\tau _{\text{coll}}\) we get 
%
\begin{align}
  \tau _{\text{coll}} (T_{dx}) = \qty(n(T_{dx}) \sigma _0 \qty(\frac{T_{dx}}{m_x})^{N})
\,,
\end{align}
%
where \(N = 0,1\): it is a fact from particle physics that the average has this kind of dependence: 
%
\begin{align}
  \expval{\sigma _A v} = \sigma_0 \qty(\frac{T}{m_x})^{N}
\,.
\end{align}

So, equaling the two \(\tau \) we get: 
%
\begin{align}
\qty(n(T_{dx}) \sigma _0 \qty(\frac{T_{dx}}{m_x})^{N})= 0.3 g_{*}^{-1/2} \frac{m _{\text{pl}}}{T_{dx}^2}
\,,
\end{align}
%
which can be solved iteratively. We solve it in terms of the parameter \(x_{dx} = m_x / T_{dx}\), which must be much larger than one: this allows us to select the physical solution to the equation among the nonphysical ones. 

The solution is found to be something like: 
%
\begin{align}
  x_{dx} = \log \qty(\num{.038} \frac{g_{x}}{g_{*xd}^{1/2}} m _{\text{pl}}  m_x \sigma_0 ) - \qty(N - \frac{1}{2}) \log \log \qty(\dots)
\,. 
\end{align}
%

\chapter{Stellar Astrophysics}

Dark matter is collisionless, however it is not \emph{really}: the censorship theorem says that it cannot actually collapse into a naked singularity. 

A star is a sphere of matter, usually baryonic matter, characterized by a density \(\rho (r)\): the mass enclosed in a radius \(r\) is 
%
\begin{align}
  m(r) = \int_{0}^{r} 4 \pi \widetilde{r}^2 \rho (\widetilde{r}) \dd{\widetilde{r}}
\,.
\end{align}

The gravitational acceleration can be calculated from Gauss' theorem: 
%
\begin{align}
  g(r) = \frac{G m(r)}{r^2}
\,. 
\end{align}
%

Let us consider a small volume, with its enclosed mass \(\Delta M = \rho (r) \Delta A \Delta r\). It can contrast the inward force due to gravity if there is a differential pressure. 

Let us denote \(P\) as the pressure at the inner surface, and \(P + \Delta P\) the pressure at the outer surface. Then,
%
\begin{align}
\qty(P + \Delta P) \Delta A - P \Delta A 
= \qty(P(r) + \dv{P}{r}\Delta r) \Delta A - P(r) \Delta A 
= \dv{P}{r} \Delta A \Delta r
\,.
\end{align}

The  minus sign in what follows is since the force is inward: 
%
\begin{align}
  - \Delta M \ddot{r} = \Delta M g(r) + \dv{P}{r} \Delta r \Delta A
\,.
\end{align}

The equation of motion is 
%
\begin{align}
  - \ddot{r} = g(r) + \frac{1}{\rho (r)} \dv{P}{r}
\,,
\end{align}
%
so we see that the pressure gradient must have a minus sign. 

If the internal energy is used up to do internal (chemical, nuclear) work, then it cannot support the star anymore, and it then collapses. 

Usually, stars start from the Jeans phenomenon: dark matter and baryons collapse under their own weight. 

Let us start at the decoupling of matter and radiation, something like \(z = 1100\).

The clouds of matter collapse freely up to the moment at which their internal pressure starts to slow them down. Then, we get the necessary conditions for the fusion of Hydrogen. 

During freefall, we do not have a pressure gradient, so we have 
%
\begin{align}
  - \ddot{r} = g(r)
\,.
\end{align}

It is not generally the case, but let us suppose that the collapse is \emph{orderly}, the interior collapses before the outer layers. 
We have to account for energy conservation: the total energy when our test shell is at a radius \(r\) is conserved as \(r\) changes. If the energy (per unit mass) is zero at infinity we have 
%
\begin{align}
  \cancelto{}{E _{\text{tot}} (r_0 )} - \frac{G m_0 }{r_0 } = \frac{1}{2} \qty(\dv{r}{t})^2 - \frac{Gm_0 }{r}
\,,
\end{align}
%
therefore 
%
\begin{align}
  \frac{1}{2} \qty(\dv{r}{t})^2= G m_0 \qty(r^{-1} - r_0^{-1})
\,. 
\end{align}
%
What is the freefall time? we take \(\dv*{t}{r}\) from the equation an integrate it from \(r_0 \) to \(0\)
%
\begin{align}
  t _{\text{free fall}} =  \int_{r_0 }^{0} \dd{r} \dv{t}{r}
  = - \int_{r_0 }^{0} \dd{r} \frac{1}{2GM}\qty(\frac{1}{r} - \frac{1}{r_0 })^{-1/2}
\,,
\end{align}
%
where we have a minus sign since \(\dv*{r}{t}<0\). We get 
%
\begin{align}
  t _{\text{free fall}} = \sqrt{\frac{r_0^3}{2 G m_0 }} \int_{0}^{1} \dd{x} \sqrt{\frac{x}{1-x}}
  = \frac{\pi}{2} \sqrt{\frac{r_0^3}{2 G m_0 }}
\,,
\end{align}
%
and if we define the average density \(\bar{\rho} = m_0 / \qty(4 \pi r_0^3 / 3)\) we get the simpler expression 
%
\begin{align}
  t _{\text{free fall}} = \sqrt{\frac{3 \pi }{32 G \bar{\rho} }}
\,.
\end{align}

We might be tempted to ignore the expansion of the universe in these calculations: we know that 
%
\begin{align}
  H^2 = \frac{8 \pi G}{3} \bar{\rho}
\,,
\end{align}
%
and in the purely Newtonian case we know that \(a(t) \propto t^{2/3}\) and \(H = 2/(3t)\). 

Therefore 
%
\begin{align}
  \frac{4}{3 t^2} = \frac{8 \pi G}{3} \bar{\rho} 
  \implies 
  t^2 = \frac{4}{9} \frac{3}{8 \pi G} \bar{\rho}^{-1}
\,,
\end{align}
%
so 
%
\begin{align}
  t _{\text{exp}} \propto \bar{\rho}^{- 1/2}
\,. 
\end{align}

\todo[inline]{what is the relation between the universe's density and the star's?

Is the idea: the matter in the universe does not disperse too fast, stars theoretically are allowed to form? }

At equilibrium, \(\ddot{r} = 0\): so 
%
\begin{align}
  - \ddot{r} = 0 = g(r) + \frac{1}{\rho (r)} \dv{P}{r}
\,,
\end{align}
%
so hydrostatic equilibrium implies that, locally, 
%
\begin{align}
  \dv{P}{r} = - G \frac{m(r) \rho (r)}{r^2}
\,.
\end{align}
%

We multiply both sides by \(4 \pi r^3\) and integrate in \(\dd{r}\): we get 
%
\begin{align}
  \int_0^{R} \dd{r} 4 \pi r^3 \dv{P}{r} = 
  - G \int_{0}^{R} \frac{m(r) \rho (r) 4 \pi r^2}{r} \dd{r}
\,,
\end{align}
%
and we can change variables: \(\rho (r) 4 \pi r^2 \dd{r} = \dd{m }\), so we can identify the LHS with the total gravitational energy: 
%
\begin{align}
E _{\text{grav}} = -G \int \frac{m(G) \dd{m}}{r} 
\,. 
\end{align}

On the RHS, instead, we can integrate by parts: 
%
\begin{align}
  \qty[P(r) 4 \pi r^3]_0^{R} - 3 \int_{0}^{R} \dd{r} 4 \pi r^2 P(r)
\,,
\end{align}
%
where the boundary term is zero: at the origin \(r =0\), at the surface (by definition) \(P =0\).

So, inserting the volume \(V(R) \equiv 4 \pi R^3/3\), we have 
%
\begin{align}
  E _{\text{grav}} = - 3 \int_{0}^{R} \frac{ \dd{r} 4 \pi r^2   P(r)}{V(R)} V(R)
\,,
\end{align}
%
which gives us the virial theorem: we can interpret the integral as a weighted average, so we get 
%
\begin{align}
  E _{\text{grav}} = - 3 \expval{P} V 
  \qquad 
  \text{or} 
  \qquad 
  \expval{P} = -\frac{1}{3} \frac{E _{\text{grav}}}{V}
\,.
\end{align}

This is Newtonian: above a certain limit, the relativistic corrections destabilize the system. 

\end{document}