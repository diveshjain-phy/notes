\documentclass[main.tex]{subfiles}
\begin{document}

\section*{Fri Oct 11 2019}

On Oct 31 Marco Peloso will not give his lecture, so Sabino will do both lectures.

Recalling last lecture: 
we consider a universe with constant density \(\rho\)\dots

We can also recover the third Friedmann equation 

\begin{equation}
  \dot{\rho} = -3 \frac{\dot{a}}{a} \qty(\rho + \frac{P}{c^2})\,,
\end{equation}

without the last term, which yet again will come from the relativistic consideration.

We will consider \emph{ideal fluids}. From thermodynamics we have:

\begin{equation}
  \dd{E} + p\dd{V} = 0\,.
\end{equation}

We can write \(E = \rho c^2 a^{3}\). Then, this becomes \(\dd{(\rho c^2 a^3)} + p \dd{a^3} = 0\); expanding:
\begin{equation}
  c^2 \dd \rho a^3 + c^2 \rho \dd{a^3} + p \dd{a^3} = 0\,.
\end{equation}

This amounts to the third Friedmann equation, exactly.

\begin{subequations}
    \begin{align}
        \dot{a}^2 &= \frac{8 \pi G }{3} \rho a^2 - kc^2  \\
        \ddot{a} &= - \frac{4 \pi G }{3} a  \qty(\rho  + \frac{3P}{c^2} )  \\
        \dot{\rho} &= -\frac{3\dot{a} }{a} \qty(\rho + \frac{P}{c^2} )
    \end{align}
\end{subequations}    

The equations are in terms of the three parameters \(a\), \(\rho\) and \(p\), which are all functions of time.

The third equation comes from the Bianchi identities \(\nabla_{\mu} G^{\mu \nu}=0\).

Rewriting the first equation:
\begin{equation}
  \ddot{a}^2 = \frac{8 \pi G}{3} \rho a^2 - k c^2
\end{equation}
and
\begin{equation}
  2 \dot{a} \ddot{a} = 
  \frac{8 \pi G}{3} \dot{\rho} a^2 + \frac{16 \pi G}{3} \rho \dot{a} a 
\end{equation}

We then substitute in the expression we have for \(\dot{\rho} \) from the third FE.
Everything is multiplied by \(\dot{a} \): if it is not zero we have
\begin{equation}
  \frac{\ddot{a}}{a} = - 4 \pi G \rho - \frac{4 \pi G p}{c^2} + \frac{8 \pi G}{3} \rho= 
  - 4 \pi G \qty(\frac{\rho}{3} + \frac{P}{c^2})\,.
\end{equation}

The equation system is underdetermined. We have to make an assumption: we will assume our fluid is a \emph{barotropic} perfect fluid: \(p \overset{!}{=} p(\rho)\).

Very often this equation of state will look like \(p = w \rho c^2\), for a constant \(w\) (not \(\omega\)!). This is related to the adiabatic constant \(\gamma\) by \(\gamma = w+1\).

We are going to assume homogeneity and isotropicity.

Some possible equations of state are \(p \equiv 0\), or \(w =0\): this means \(\gamma =1 \). This is a \emph{dust pressureless fluid}.

What is the speed of sound of our fluid? We only have the adiabatic speed of sound \(c_s^2 = \pdv*{p}{\rho}\), where the derivative is to be taken at constant entropy and is just a total derivative in the barotropic case.

Also in the baryionic case \(p=0\) is a good approximation.

If \(p=0\) we can simplify:
\begin{equation}
  \dot{\rho} = -3 \frac{\dot{a} }{a} \rho \implies 
  \rho \propto a^{-3}\,.
\end{equation}

More generally, not assuming \(w=0\) we get:
\begin{equation}
  \dot{\rho} = -3 (1+w) \rho \frac{\dot{a} }{a} \implies \rho \propto a^{-3 (1+w)} = a^{-3 -3\gamma}
\end{equation}

Another case is a \emph{gas of photons}: in that case \(p = \rho c^2 / 3\), so \(w=\frac[i]{1}{3} \), \(\gamma = \frac[i]{4}{3} \), \(c_{s}^2 = c^2 / 3 \): the speed of sound is \(c / \sqrt{3} \). These photons are thermal: perturbations can propagate (even without interactions with matter\dots). 

In this case we get \(\rho \propto p \propto a^{-4}\).

Stiff matter is \(p = \rho c^2\), \(w=1\), \(\gamma=2\) and \(c_s = c\). This is an incompressible fluids: it is so difficult to set this matter in motion that once one does it travels at the speed of light. Now, \(\rho \propto p \propto a^{-6}\).

A possible case is \(p = - \rho c^2\): \(w = -1\) and \(\gamma = 0\): we cannot compute a speed of sound. Now \(\rho\) and \(p\) are constants. This is the case of dark energy (?).

This can be interpreted as an interpretation of the cosmological constant \(\Lambda\).

Now we relace the last FE with \(w = \const\), \(\rho(t) = \rho_{*} \qty(a(t) / a_{*})^{-3(1+w)}  \).

Now, if we substitute into the second FE we get that gravity is attractive (\(\ddot{a} < 0 \)) iff \(w > -1/3\). 

[Plot: \(\rho\) vs \(a\): the cosmological constant is constant, matter is decreasing, radiation is decreasing faster].

In this plot, \(a\) can be interpreted as the time.
We can insert the spatial curvature in the plot: it decreases, but slower than matter. 
Now, the dark energy in the universe is more important than the curvature. 

Let us solve the first FE: inserting the third one we get
\begin{equation}
  \qty(\frac{\dot{a} }{a})^2 = 
  \frac{8 \pi G}{3} \rho_* \qty(\frac{a}{a_*}) ^{-3 (1+w)} - \frac{kc^2}{a^2}\,.
\end{equation}

We defined the parameter \(\Omega = \frac{8 \pi G \rho}{3 H^2} = \rho / \rho_C\).
Experimentally this is very close to 1.
The Einstein-de Sitter model is one where we take \(\Omega \equiv 1\): negligible spatial curvature. This amounts to making \(k=0\).
\begin{equation}
  \dot{a}^2 = \frac{8 \pi G}{3} \rho_* a_{*}^{3 (1+w)} a^{-(1+3w)}
\end{equation}

therefore \(\dot{a} = \pm A a^{\frac{1+3w}{2}}\), or \(a ^{\frac{1+3w}{2}}\dd{a} = A \dd{t}\). A solution is:
\begin{equation}
  a(t)= a_{*} \qty(
    1 + \frac{3}{2} (1+w) H_{*} (t - t_{*})
  )^{\frac{2}{3(1+w)}}
\end{equation}
where \(H_{*}^2 = \frac{8 \pi G}{3} \rho_{*}\), coupled to:
\begin{equation}
  \rho(t) = \rho_{*} \qty(1 + \frac{3}{2}(1+w H_* (t-t_*)))^{-2}
\end{equation}

There is a time where the bracket in \(a(t)\) is zero: we call it as \(t_{\text{BB}}\) define it by
\begin{equation}
  1 + \frac{3}{2} (1+w) H_{*} (t_{\text{BB}} - t_* ) = 0\,.
\end{equation}

Since the curvature scalar is \(R \propto H^2\), at \(t_{\text{BB}}\)  the curvature is diverges.

Hakwing \& Ellis proved that if \(w>-1/3\) we unavoidably must have a Big Bang.

We can define a new time variable by \(t_{\text{new}} \equiv (t - t_{*}) + 2 H_*^{-1} / (3 (1+w))\). Then, we can just write:
\begin{equation}
  a \propto t_{\text{new}}\,^{\frac{2}{3(1+w)}}
\end{equation}
and this allows us to get rid of \(t_{*}\).

Inserting this new time variable, we get
\begin{equation}
  \rho(t) = \frac{1}{6 (1+w)^2 \pi 4 t^2}
\end{equation}
and the Hubble parameter is:
\begin{equation}
  H(t) = \frac{2}{3(1+w) t}
\end{equation}
Some cases are:
\begin{equation}
  \begin{cases}
      w = 0 \implies a \propto t^{2/3}  \\
      w=1/3 \implies a \propto t^{1/2} \\  
      w=1 \implies a \propto t^{1/3}  
  \end{cases}
\end{equation}

The \emph{De Sitter} universe is one where \(w \rightarrow 1\): \(a(t) \propto \exp(Ht) \) and \(H = \const\). (CHECK) 

stuff

\end{document}