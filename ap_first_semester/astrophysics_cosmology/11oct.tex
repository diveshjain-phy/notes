\documentclass[main.tex]{subfiles}
\begin{document}

% \section*{Fri Oct 11 2019}
\marginpar{Friday \\ 2019-10-11}

% On Oct 31 Marco Peloso will not give his lecture, so Sabino will do both lectures.

% Recalling last lecture: 
% we consider a universe with constant density \(\rho\)\dots

We can also recover the third Friedmann equation 
\begin{equation}
\dot{\rho} = -3 \frac{\dot{a}}{a} \qty(\rho + \frac{P}{c^2}) \approx -3 \frac{\dot{a}}{a} \rho 
\,,
\end{equation}
%
where the approximation as before is the Newtonian one, \(P \ll \rho c^2\). In this case, however, we will be able to also recover the nonrelativistic pressure term if we account for conservation of energy and not just mass.
% without the last term, which yet again will come from the relativistic consideration.

When deriving these equations relativistically the third one comes from the conservation of the temporal component of the stress-energy tensor (\(\nabla_{\mu } T^{\mu 0} = 0\)), i.e. the ``conservation of energy'',\footnote{This is actually \emph{not} a conservation law as stated, it is not covariant: in order for it to be we would need to project it along a temporal Killing vector, which does not exist in cosmology. However, if we did have a temporal Killing vector \(\xi_{\mu }\) the projection \(\xi_{\nu } \nabla_{\mu } T^{\mu \nu } =0\) would indeed be the conservation of energy.

Indeed, it is more correct to say that this equation is not a conservation law at all, but is instead just an expression of the geometric Bianchi identities \(\nabla_{\mu } G^{\mu \nu }=0\), which are written in terms of the Einstein tensor \(G_{\mu \nu } = R_{\mu \nu } - \frac{1}{2} R g_{\mu \nu }\); if the Einstein equations hold then \(G_{\mu \nu } \) is proportional to \(T_{\mu \nu }\), therefore the two statements are equivalent.}
so to find it in our Newtonian calculation we will need to consider the nonrelativistic equivalent of that equation, which is the first law of thermodynamics.

We will consider \emph{ideal fluids}.
The first law, assuming adiabaticity --- which must be present, since net heat transfer in the universe would violate isotropy ---  states:
\begin{equation}
  \dd{E} + P\dd{V} = 0\,.
\end{equation}

We can write the total energy as the product of the energy density times the volume: \(E = \frac{4 \pi }{3} \rho c^2 a^{3}\), since the volume is \(V = \frac{4 \pi }{3} a^3\).

So, the first law reads: 
%
\begin{align}
0&=\frac{4\pi }{3}\qty[\dd{\qty(\rho c^2 a^3)} + P \dd{(a^3)}]   \\
&= c^2 \rho \dd{(a^3)} + c^2 a^3 \dd{\rho } + P \dd{(a^3)}   \marginnote{Divided through by \(4 \pi / 3\)}\\
&= 3 \qty(\rho + \frac{P}{c^2}) \frac{\dd{a}}{a} + \dd{\rho } \marginnote{Divided through by \(c^2a^3\), collected terms.}
\,,
\end{align}
%
which is the third Friedmann equation, \eqref{eq:friedmann-3}: the only manipulation left to do is to apply the differentials, which are covectors, to the temporal vector \(\dv*{}{t}\) in order to turn them into time derivatives.

Why were we able to recover the relativistic term this time? The completely non-relativistic approach to this would be to write \(M = \frac{4\pi }{3} \rho a^3 \), and to write down the equation for the conservation of mass alone. 
Indeed, this would yield the Friedmann equation without the \(P\) term.

\todo[inline]{Footnote number 5 at page 17 in \cite{Pacciani:2018} states: ``In generale si può porre \(\ell(t) = a(t)r\) con \(r\) coordinata comovente adimensionale per la validità del principio cosmologico; \(r\) può
essere sempre scelta piccola a piacere di modo che ad ogni istante valga la condizione di campo debole.''
This does not seem to make sense: the weak-field condition is not dependent on our coordinate choices, and in fact is written in terms of \(\ell\) and not \(r\).}

The three Friedman equations are not independent: for example, the second one \eqref{eq:friedmann-2} can be derived from the first and third. 

This means that we can derive the full relativistic equations in this Newtonian context, using the derivations we have shown for the first and third equation, and then combining these to find the second.

% \begin{subequations}
%     \begin{align}
%         \dot{a}^2 &= \frac{8 \pi G }{3} \rho a^2 - kc^2  \\
%         \ddot{a} &= - \frac{4 \pi G }{3} a  \qty(\rho  + \frac{3P}{c^2} )  \\
%         \dot{\rho} &= -\frac{3\dot{a} }{a} \qty(\rho + \frac{P}{c^2} )
%     \end{align}
% \end{subequations}    

% The equations are in terms of the three parameters \(a\), \(\rho\) and \(p\), which are all functions of time.

% The third equation comes from the Bianchi identities \(\nabla_{\mu} G^{\mu \nu}=0\).

Let us do this derivation explicitly:
we differentiate the first Friedmann equation
\begin{equation}
  \dot{a}^2 = \frac{8 \pi G}{3} \rho a^2 - k c^2
\end{equation}
with respect to time to find 
\begin{equation}
  2 \dot{a} \ddot{a} = 
  \frac{8 \pi G}{3} \dot{\rho} a^2 + \frac{16 \pi G}{3} \rho \dot{a} a 
\end{equation}

We then substitute in the expression we have for \(\dot{\rho} \) from the third equation: 
%
\begin{align}
\dot{\rho} = -3 \frac{\dot{a}}{a} \qty(\rho + \frac{P}{c^2})
\,,
\end{align}
%
which gives us  
%
\begin{align}
2 \dot{a} \ddot{a} &= 
\frac{8 \pi G}{3} \qty[- 3 \frac{\dot{a}}{a} \qty(\rho + \frac{P}{c^2})] a^2 + \frac{16 \pi G}{3} \rho  \dot{a} a  \\
\ddot{a} &= - 4 \pi G \qty(\rho + \frac{P}{c^2}) a
+ \frac{8 \pi G}{3} \rho a 
 \marginnote{Dividing through by \(2\dot{a}\)}  \\
\frac{\ddot{a}}{a} &= - \frac{8 \pi G}{3} \qty[\frac{3}{2} \qty(\rho + \frac{P}{c^2}) - \rho ]  
\marginnote{Dividing through by \(a\)}
\\
\frac{\ddot{a}}{a} &= -\frac{4 \pi G}{3} \qty(\rho + 3 \frac{P}{c^2})
\,,
\end{align}
%
which is precisely equation \eqref{eq:friedmann-2}.

\section{The equation of state}

So, the equation system is underdetermined: we do not in fact have three independent Friedmann equations, but just two. 
The variables we want to find, however, are three: \(P(t)\), \(\rho (t)\), \(a(t)\).

So, we have to make an assumption: we will assume our fluid is a \emph{barotropic} perfect fluid, that is, one for which the pressure only depends on the density: \(P \overset{!}{=} P(\rho)\).

Very often this equation of state will be linear: \(P = w \rho c^2\), with an adimensional constant \(w\).\footnote{This is a latin \(w\), not a greek \(\omega \): students historically call it ``omega'' for some reason.}
We will assume this relation to be true.

% This is related to the adiabatic constant \(\gamma\) by \(\gamma = w+1\).

% We are going to assume homogeneity and isotropicity.

\subsection{Common equations of state}

A thing we will compute for the different equations of state is the adiabatic speed of sound: 
%
\begin{align}
c^2_{s} = \pdv{P}{\rho } = \dv{P}{\rho } = w c^2
\,.
\end{align}


\begin{enumerate}
  \item \(w = 0\) is equivalent to \(P \equiv 0\): this is what we get in the nonrelativistic limit, for \(P \ll \rho c^2\), since there is no pressure this can be interpreted as a \emph{dust}. In this case \(\rho \propto a^{-3}\), and \(c_s^2 \ll c^2\). 
  \item \(w = 1/3\) is what we get if we seek the pressure of radiation.\footnote{This expression can be derived in different ways, one of which is to start from the fact that the stress energy tensor must be traceless since it is of the form \(T_{\mu \nu } \sim \sum _{i} \rho u^{(i)}_{\mu } u^{(i)}_{\nu }\), where \(u_{\mu }^{(i)}\) are the four-velocities of photons: their norm is zero, so we must have \(g^{\mu \nu} T_{\mu \nu } = 0\). Another, perhaps more illustrative derivation was given in the General Relativity course \cite[pag. 86-87]{tissinoGeneralRelativityNotes2020}.}
  In this case we have \(c_s = c/ \sqrt{3}\), while the energy density goes like \(\rho \propto a^{-4}\), since we get a factor \(a^{-3}\) from the volume expansion and another \(a^{-1}\) from the decrease of the energy of each photon due to redshift. So, for a radiation-dominated universe the total energy \(E \propto \rho a^{3} \propto a^{-1}\) is not conserved. 
  \item 
\end{enumerate}


% Some possible equations of state are \(P \equiv 0\), or \(w =0\): this means \(\gamma =1 \). This is a \emph{dust pressureless fluid}.

% What is the speed of sound of our fluid? We only have the adiabatic speed of sound \(c_s^2 = \pdv*{p}{\rho}\), where the derivative is to be taken at constant entropy and is just a total derivative in the barotropic case.

% Also in the baryionic case \(p=0\) is a good approximation.

If \(p=0\) we can simplify:
\begin{equation}
  \dot{\rho} = -3 \frac{\dot{a} }{a} \rho \implies 
  \rho \propto a^{-3}\,.
\end{equation}

More generally, not assuming \(w=0\) we get:
\begin{equation}
  \dot{\rho} = -3 (1+w) \rho \frac{\dot{a} }{a} \implies \rho \propto a^{-3 (1+w)} = a^{-3 -3\gamma}
\end{equation}

Another case is a \emph{gas of photons}: in that case \(p = \rho c^2 / 3\), so \(w=\frac{1}{3} \), \(\gamma = \frac{4}{3} \), \(c_{s}^2 = c^2 / 3 \): the speed of sound is \(c / \sqrt{3} \). These photons are thermal: perturbations can propagate (even without interactions with matter\dots). 

In this case we get \(\rho \propto p \propto a^{-4}\).

Stiff matter is \(p = \rho c^2\), \(w=1\), \(\gamma=2\) and \(c_s = c\). This is an incompressible fluids: it is so difficult to set this matter in motion that once one does it travels at the speed of light. Now, \(\rho \propto p \propto a^{-6}\).

A possible case is \(p = - \rho c^2\): \(w = -1\) and \(\gamma = 0\): we cannot compute a speed of sound. Now \(\rho\) and \(p\) are constants. This is the case of dark energy (?).

This can be interpreted as an interpretation of the cosmological constant \(\Lambda\).

Now we relace the last FE with \(w = \const\), \(\rho(t) = \rho_{*} \qty(a(t) / a_{*})^{-3(1+w)}  \).

Now, if we substitute into the second FE we get that gravity is attractive (\(\ddot{a} < 0 \)) iff \(w > -1/3\). 

[Plot: \(\rho\) vs \(a\): the cosmological constant is constant, matter is decreasing, radiation is decreasing faster].

In this plot, \(a\) can be interpreted as the time.
We can insert the spatial curvature in the plot: it decreases, but slower than matter. 
Now, the dark energy in the universe is more important than the curvature. 

Let us solve the first FE: inserting the third one we get
\begin{equation}
  \qty(\frac{\dot{a} }{a})^2 = 
  \frac{8 \pi G}{3} \rho_* \qty(\frac{a}{a_*}) ^{-3 (1+w)} - \frac{kc^2}{a^2}\,.
\end{equation}

We defined the parameter \(\Omega = \frac{8 \pi G \rho}{3 H^2} = \rho / \rho_C\).
Experimentally this is very close to 1.
The Einstein-de Sitter model is one where we take \(\Omega \equiv 1\): negligible spatial curvature. This amounts to making \(k=0\).
\begin{equation}
  \dot{a}^2 = \frac{8 \pi G}{3} \rho_* a_{*}^{3 (1+w)} a^{-(1+3w)}
\end{equation}

therefore \(\dot{a} = \pm A a^{\frac{1+3w}{2}}\), or \(a ^{\frac{1+3w}{2}}\dd{a} = A \dd{t}\). A solution is:
\begin{equation}
  a(t)= a_{*} \qty(
    1 + \frac{3}{2} (1+w) H_{*} (t - t_{*})
  )^{\frac{2}{3(1+w)}}
\end{equation}
where \(H_{*}^2 = \frac{8 \pi G}{3} \rho_{*}\), coupled to:
\begin{equation}
  \rho(t) = \rho_{*} \qty(1 + \frac{3}{2}(1+w H_* (t-t_*)))^{-2}
\end{equation}

There is a time where the bracket in \(a(t)\) is zero: we call it as \(t_{\text{BB}}\) define it by
\begin{equation}
  1 + \frac{3}{2} (1+w) H_{*} (t_{\text{BB}} - t_* ) = 0\,.
\end{equation}

Since the curvature scalar is \(R \propto H^2\), at \(t_{\text{BB}}\)  the curvature is diverges.

Hakwing \& Ellis proved that if \(w>-1/3\) we unavoidably must have a Big Bang.

We can define a new time variable by \(t_{\text{new}} \equiv (t - t_{*}) + 2 H_*^{-1} / (3 (1+w))\). Then, we can just write:
\begin{equation}
  a \propto t_{\text{new}}\,^{\frac{2}{3(1+w)}}
\end{equation}
and this allows us to get rid of \(t_{*}\).

Inserting this new time variable, we get
\begin{equation}
  \rho(t) = \frac{1}{6 (1+w)^2 \pi 4 t^2}
\end{equation}
and the Hubble parameter is:
\begin{equation}
  H(t) = \frac{2}{3(1+w) t}
\end{equation}
Some cases are:
\begin{equation}
  \begin{cases}
      w = 0 \implies a \propto t^{2/3}  \\
      w=1/3 \implies a \propto t^{1/2} \\  
      w=1 \implies a \propto t^{1/3}  
  \end{cases}
\end{equation}

The \emph{De Sitter} universe is one where \(w \rightarrow 1\): \(a(t) \propto \exp(Ht) \) and \(H = \const\). (CHECK) 

stuff

\end{document}