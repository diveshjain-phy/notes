\documentclass[main.tex]{subfiles}
\begin{document}

\section*{Fri Nov 15 2019}

We start again from where we left off, with hydrogen recombination.

We want to estimate the moment at which hydrogen first formed, which marks the point at which electrons and photons interact efficently: before, they interacted with Compton scattering which is very efficient; after they interact with hydrogen atoms in a way that is very inefficient.

After this, then, we say that photons and matter are \emph{decoupled}.

The scattering cross section (for Compton?) goes like the inverse square of the mass.

This means that the universe is not only \emph{globally},  but also \emph{locally} neutral.

This decoupling is what allows for star formation.
Also, this is when the CMB starts.
It is made of microwaves now, but it was higher earlier.

The phase space distribution of photons is scale-invariant since they have zero mass: so we can say that the photons' distribution \emph{looks} thermal, but it is actually not technically since there are no interactions anymore.

However the photons travel freely and are perceived as thermal, and they give an almost perfect blackbody! The errorbars in a plot for it must be magnified by \num{e4} in order to be seen.

We have \(\Gamma_{\gamma } = n_e \sigma_T\), where \(\sigma _T\) is the Thompson cross section. 
Neutrality implies \(n_e = n_p\).
In principle we should account for Helium: a couple minutes after the BB He-4 nuclei started to form, but they made up only something like \(25\%\) of the mass, which means \(6\%\) of the number density: so we say that the number density of baryons is 
%
\begin{align}
  n_b = n_p + n_H
\,.
\end{align}
%

Our ansatz for the Boltzmann equation is \(\mu_e + \mu _p = \mu _H\) since photons have no chemical potential.

\(i\) denotes a generic one in \(e\), \(p\) and \(H\).
Then 
%
\begin{align}
  n_i = g_i\qty(\frac{m_i T}{2 \pi })^{3/2} \exp(\frac{\mu _i - m_i}{T})
\,,
\end{align}
%
we need to account for the chemical potential since it is the driver of this process.
How much is \(n_e / n_b\)? the same as \(n_p / n_b\). We call this quantity \(X_e\), the ionization number.
We expect \(X_e = 1\) in the early universe, and at the end of the process it will diminish: naively we'd expect it to get to 0, but actually there remains some residual ionization, some free protons and electrons.
A proper calculation would account for the non-equilibrium contributions.
However, we estimate the process as being in equilibrium: this will underestimate the number of electrons.

Today, most of the hydrogen is ionized (there is a ***-Peterson effect which shows this): this means that matter and radiation interact again.

This is the second important time in the history of the universe.

Can we see the early stars? Not really, we see galaxies only up to \(z \sim 10\), these stars would be at something like \(z \sim 30\)\dots
There might be more to this.

Let us come back to the calculation: 
%
\begin{subequations}
\begin{align}
  n_H &=  g_H \qty(\frac{m_H T}{2 \pi })^{3/2} \exp(\frac{\mu _H - m_H}{T})  \\
  &= g_H \qty(\frac{m_H T}{2 \pi })^{3/2} \exp(\frac{\mu_e + \mu _p - m_e - m_p + B}{T}) \\
  &= \frac{g_H}{g_e g_p} \cancelto{}{\qty(\frac{m_H T}{2 \pi })^{3/2}}
  \qty(\frac{m_e T}{2 \pi })^{-3/2}
  \cancelto{}{\qty(\frac{m_p T}{2 \pi })^{-3/2}}
  n_e n_p
  \exp(\frac{B}{T}) \\
  \frac{n_H}{n_e n_p} &=  \qty(\frac{m_e T}{2 \pi })^{-3/2} \exp(\frac{B}{T})  \\
  \frac{n_b - n_p}{n_p^2} &= \qty(\frac{m_H T}{2 \pi })^{3/2} \exp(\frac{B}{T}) 
\,,
\end{align}
\end{subequations}
%
which we can manipulate, using the following identity: 
%
\begin{align}
  \frac{n_b - n_p}{n_p^2} = \frac{n_b \qty(1 - n_p / n_b)}{n_b^2 X_e^2} = \frac{1}{n_b} \frac{1 - X_e}{X_e}
\,,
\end{align}
%
where we use: \(n_e = n_p\), and the definition of \(X_e = n_p / n_b\). Then, we bring the \(n_b\) to the other side of the equation: we get
%
\begin{subequations}
\begin{align}
  \frac{1- X_e}{X_e^2} &= \underbrace{\frac{n_b}{n_\gamma }}_{\eta_p} \qty(\frac{m_e T}{2 \pi })^{-3/2} \exp(\frac{B}{T})  n_\gamma  \\
  &= \frac{4 \sqrt{2} \zeta (3)T^3}{\pi^2} \qty(\frac{m_e T}{2 \pi })^{-3/2} \exp(\frac{B}{T}) \eta _0  \\
  &= \frac{4 \sqrt{2} \zeta (3)}{\sqrt{\pi }} \eta_0 \qty(\frac{T}{m_e})^{3/2} \exp(\frac{B}{T})
\,,
\end{align}
\end{subequations}
%
which means \(T \sim \SI{0.3}{eV}\), much lower than the ionization energy of Hydrogen.

This does depend on the value we assign to \(\Omega_0\) and \(h\).

Approximately, it occurred somewhere around \(z \sim 1100\) (we conventionally say that recombination happened when \(X_e = 0.1\)).

After recombination, we have the \emph{last scattering}: the moment at which the CMB was formed.

One Nobel prize this year was awarded to Jim Peebles, a friend of Sabino's: together with his PhD supervisor Dicke, he was the first to calculate this stuff.

Peebles in 1964 (?) did this calculation both in GR and in Brahms-Dicke theory, a modified gravity theory.

Let us describe the early universe, before the first nucleosynthesis.
Important papers in this topic are by G. Gamow, and by Alpher, Bethe and Gamow.

Our hypotheses are:
%
\begin{enumerate}
    \item the universe passed through a very high temperature phase, with \(T > \SI{e12}{K}\);
    \item the universe is described by GR and SM;
    \item the chemical potentials for the neutrinos \(\mu_{\nu }\) have certain upper bounds; 
    \item there is no matter-antimatter separation (as in, ``bubbles'');
    \item there are no strong magnetic fields;
    \item the number of exotic particles has a certain upper bound.
\end{enumerate}

There are magnetic fields in the universe, but they are not homogeneous and relatively weak.
Exotic particles are predicted by certain unification theories, they are generically defined as ones which we have not observed yet.

We have to explain the fact that we observe an excess of He-4 in the early universe: we define the yield 
%
\begin{align}
  y \equiv \frac{m_{\ce{He-4}}}{m_b} > \num{.25}
\,.
\end{align}

In terms of particle number, the ratio is more like \num{.06}. 
We do not produce Carbon or anything higher than it: the process which forms it is inefficient at high temperature, low density like the early universe.
Higher \(Z\) elements are only produced in stars.

Helium-4 is produced but also destroyed by stars.

The main channels in the early universe are: 

\begin{enumerate}
    \item \(n+p \leftrightarrow d + \gamma \) (\(d\) denotes deuterium);
    \item \(d + d \leftrightarrow \ce{^{3}He} + n\);
    \item \(\ce{^{3}He} + d \leftrightarrow \ce{^{4}He} + p\).
\end{enumerate}

There is no weak interaction here, unlike what happens in stars.
In stars, there are no free neutrons so this process is not possible.

We cannot treat this properly, we only give a story.
The slowest process of the three is the first, since it is heavily affected by photons: they destroy deuterium.

The binding energy of deuterium is around \SI{2.2}{MeV}. After we have produced deuterium, Helium-4 is readily produced.

First of all, we need the neutron to proton ratio.
We are working at energies of around \SI{1}{MeV}, so protons and neutrons are not relativistic anymore.
This process takes place around three minutes after the beginning.
For \(i = n, p\): 
%
\begin{align}
  n_i = g_{i} \qty(\frac{m_i T}{2 \pi })^{3/2} \exp(\frac{\mu_i  m_i}{T})
\,,
\end{align}
%
so their number ratio is around: 
%
\begin{align}
  \frac{n}{p} \sim \exp(\frac{m_p- m_n}{T})
\,,
\end{align}
%
where \(m_p - m_n \approx \SI{1.3}{MeV} \approx \SI{1.5e10}{K}\).

We have the processes 
\begin{enumerate}
    \item \(n + \gamma _e \leftrightarrow p + e^{-}\);
    \item \(n + e^{+} \leftrightarrow p + \overline{\nu _e} \);
    \item \(n \rightarrow p + e^{-} + \overline{\nu _e} \).
\end{enumerate}

We can replace the temperature in the exponential by \(T \rightarrow T_{d_\nu }\), the decoupling temperature of the neutrinos, since that is the moment around which this happpens.

So, we get around \(\exp(-1.5)\).
We define: 
%
\begin{align}
  X_n (t) \equiv \frac{n}{n+p} \sim \num{.17}
\,.
\end{align}

Later it will change because of \(\beta \) decay, going like: 
%
\begin{align}
  X_n(t) = X_n(t_{d_\nu }) \exp( - \frac{t - t_{d_\nu }}{\tau _n})
\,,
\end{align}
%
where \(\tau _n = \log 2 \tau_{1/2}\), and this half-life is \(\tau_{1/2} \approx \num{10.5+-0.2}\).

The binding energy of deuterium is around \SI{2.2}{MeV}.
Then, we proceed exactly like we did with hydrogen, and finally get 
%
\begin{align}
  X_d &= \frac{3}{4} n_b X_n X_p \qty(\frac{m_d}{m_nm_p})^{3/2} \qty(\frac{T}{2 \pi })^{-3/2} \exp(\frac{B}{T}) \\
  &= \frac{3}{4} \eta_0 X_n X_p \qty(\frac{m_d}{m_nm_p})
  \frac{2 \zeta (3)}{\pi^2} (2 \pi )^{3/2} T^{3/2} \exp(\frac{B}{T})  \\
  &\approx \frac{3}{4} \eta_0 X_n (1-X_n) \qty(\frac{m_d}{m_nm_p})^{3/2} \frac{2}{\pi^2} (2 \pi )^{3/2} \zeta (3) T^{3/2} \exp(\frac{B_d}{T})
\,,
\end{align}
\todo[inline]{Check calculation.}
%
which describes the \emph{deuterium bottleneck}, which is what impedes this process until photons are very diluted.
As soon as photons are diluted enough, they stop bottlenecking.

\end{document}