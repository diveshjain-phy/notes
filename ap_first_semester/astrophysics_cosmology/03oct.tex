\documentclass[main.tex]{subfiles}
\begin{document}

\section*{3 October 2019}

Sabino Matarrese \(\heartsuit\).
There is a dropbox folder with notes by a student from the previous years.
Also, there are handwritten notes by Sabino.

Textbooks: there are many. One by Lucchini (?): ``Cosmology''.

On the astroparticle side, there is another book\dots

Exam: traditional oral exam, there are fixed dates but they do not matter: on an individual basis we should write an email to set a date and time.

In october the lessons of GR and this course on fridays are swapped.

\section{Cosmology}

The cosmological principle (or Copernican principle): \emph{we do not occupy a special, atypical position in the universe}. We will discuss the validity of this. A more formal statement is:
%
\begin{proposition}[Cosmological principle]
    Every \emph{comoving observer} observes the Universe around them at a \emph{fixed time} as being \emph{homogeneous} and \emph{isotropic}.
\end{proposition}

We are allowed to use this principle only on very large scales.
Comoving is a proxy for the absolute reference frame.
When we observe the CMB we see that we are surrounded by radiation of temperature \(\sim \SI{3}{K}\).
There is a \emph{dipole modulation} though: around a milliKelvin of difference. This is due to the Doppler effect: we are \emph{not} comoving with respect to the CMB.

We, however, ignore the anisotropies on the order of a \SI{}{\micro K}.

The velocity needed to expain the Doppler effect is of the order of \SI{630}{\kilo\metre\per\second}.
This is \emph{after} correcting for the motion of everything up to a galactic scale.

So, we must assume that to get a \emph{comoving} observer we should launch it in a certain direction at an extremely high velocity. So, we \emph{do} have a preferred frame.

\emph{Fixed time} refers to the proper time of the comoving observer.

This refers to scales on the order of \SI{100}{\mega\parsec}.

How can we talk about homogeneity if we can only look at the universe from a single point? We assume that any other observer would also see isotropy as we do.

Isotropy around every point is equivalent to homogeneity. We observe isotropy, we assume homogeneity.

What if we are not typical? a different assumption is that our observer status is \emph{random} according to some distribution\dots This discussion then starts to involve the anthropic principle.

We must however always keep in mind that these assumptions have to be made before any cosmological study starts.

In GR, our line element will generally be \(\dd{s^2} = g_{ab} \dd{x^a} \dd{x^b} \). The \emph{preferred} form of the metric is the one written in the comoving frame: the Robertson-Walker line element,
%
\begin{equation} \label{eq:robertson-walker}
  \dd{s^2} = c^2 \dd{t^2} - a^2(t) \qty(\frac{\dd{r^2}}{1 - k r^2} + r^2 \qty(\dd{\theta^2} + \sin^2(\theta) \dd{\varphi^2} ) )
\end{equation}
%
where \(a(t)\) is called the \emph{scale factor}. Sometimes it is convenient to write \(\dd{\theta^2} + \sin^2(\theta) \dd{\varphi^2} = \dd{\Omega^2}\).
The angular part is the same (just rescaled) as the Minkowski one.

The parameter \(k\) is a constant, which can be \(\pm 1\) or \(0\).

The parameter \(r\) is not a length: we choose our variables so that \(a(t)\)  is a length, while the stuff in the brackets in \eqref{eq:robertson-walker} is adimensional (and so is \(r\)).

\begin{itemize}
    \item \(k=1\) are called closed universes;
    \item \(k=-1\) are open universes;
    \item \(k=0\) are called flat universes.
\end{itemize}

The set of flat universes with \(k=0\) has zero measure.



\end{document}
