\documentclass[main.tex]{subfiles}
\begin{document}

\section*{Introduction and relevant material}

These are the
(yet to be)
revised notes for the course ``Fundamentals of Astrophysics and Cosmology'' held by professor Sabino Matarrese in fall 2019 at the university of Padua.

They are based on the notes I took during lectures, complemented with notes from the previous years.

They will be revised by the professor in the future, as of yet they have not.

The exam is a traditional oral exam, there are fixed dates but they do not matter: on an individual basis we should write an email to the professor to set a date and time.

\paragraph{Material}

There is a dropbox folder with notes by a student from the previous years \cite[]{Pacciani:2018} and handwritten notes by the professor.

There are many good textbooks, for example ``\citetitle{LucchinColes:2002}'' by \citeauthor{LucchinColes:2002} \cite[]{LucchinColes:2002}.

% Sabino Matarrese \(\heartsuit\).

% On the astroparticle side, there is another book: I did not catch its name.


% In october the lessons of GR and this course on fridays are swapped.

\chapter{Cosmology}

% \section*{3 October 2019}
\marginpar{Thursday\\ 2019-10-3}

The basis for the modern treatment of cosmology is the \textbf{cosmological principle} (or Copernican principle):
roughly stated, it is ``\emph{we do not occupy a special, atypical position in the universe}''.

\todo[inline]{We will discuss the validity of this.}

A more formal statement is:
%
\begin{proposition}[Cosmological principle]
    Every \emph{comoving observer} observes the Universe around them at a \emph{fixed time} in their reference frame as being \emph{homogeneous} and \emph{isotropic}.
\end{proposition}
This is expected to hold only on very large scales: the 

Comoving is a proxy for the absolute reference frame.
When we observe the CMB we see that we are surrounded by radiation of temperature \(\sim \SI{3}{K}\).
There is a \emph{dipole modulation} though: around a milliKelvin of difference. This is due to the Doppler effect: we are \emph{not} comoving with respect to the CMB.

We, however, ignore the anisotropies on the order of a \SI{}{\micro K}.

The velocity needed to explain the Doppler effect is of the order of \SI{630}{\kilo\metre\per\second}.
This is \emph{after} correcting for the motion of everything up to a galactic scale.

So, we must assume that to get a \emph{comoving} observer we should launch it in a certain direction at an extremely high velocity. So, we \emph{do} have a preferred frame.

\emph{Fixed time} refers to the proper time of the comoving observer.

This refers to scales on the order of \SI{100}{\mega\parsec}.

How can we talk about homogeneity if we can only look at the universe from a single point? We assume that any other observer would also see isotropy as we do.

Isotropy around every point is equivalent to homogeneity. We observe isotropy, we assume homogeneity.

What if we are not typical? a different assumption is that our observer status is \emph{random} according to some distribution\dots This discussion then starts to involve the anthropic principle.

We must however always keep in mind that these assumptions have to be made before any cosmological study starts.

In GR, our line element will generally be \(\dd{s^2} = g_{ab} \dd{x^a} \dd{x^b} \). The \emph{preferred} form of the metric is the one written in the comoving frame: the Robertson-Walker line element,
%
\boxalign{
\begin{align} \label{eq:robertson-walker}
  \dd{s^2} = c^2 \dd{t^2} - a^2(t) \qty(\frac{\dd{r^2}}{1 - k r^2} + r^2 \qty(\dd{\theta^2} + \sin^2(\theta) \dd{\varphi^2} ) )
\end{align}}
%
where \(a(t)\) is called the \emph{scale factor}. Sometimes it is convenient to write \(\dd{\theta^2} + \sin^2(\theta) \dd{\varphi^2} = \dd{\Omega^2}\).
The angular part is the same (just rescaled) as the Minkowski one.

The parameter \(k\) is a constant, which can be \(\pm 1\) or \(0\).

The parameter \(r\) is not a length: we choose our variables so that \(a(t)\)  is a length, while the stuff in the brackets in \eqref{eq:robertson-walker} is adimensional (and so is \(r\)).

\begin{itemize}
    \item \(k=1\) are called closed universes;
    \item \(k=-1\) are open universes;
    \item \(k=0\) are called flat universes.
\end{itemize}

The set of flat universes with \(k=0\) has zero measure.
In a 4D flat spacetime we have the 10 degrees of freedom of Lorentz transformations: Minkowski spacetime is \emph{maximally symmetric}.

Other geometries are De Sitter and Anti de Sitter spacetimes.

For the actual universe, however, not all of these symmetries hold: the universe seems not to be symmetric under time translation, and we have a preferential velocity.
4 symmetries are broken (time translation and Lorentz boosts), 6 hold (rotations and spatial translations): our assumption is that there is a 3D maximally symmetric space.

Under these assumptions, we can derive the RW line element.

Say we have a cartesian flat 2D plane: the flat metric for this is \(\dd{l^2} = a^2 (\dd{r^2} + r^2 \dd{\theta^2}) \).
The constant \(a\) is included since we want \(r\) to be adimensional.

The surface of a sphere has the following line element:
\(\dd[]{l^2} = a^2 \qty(\dd[]{\theta^2} + \sin^2 \theta \dd[]{\varphi^2}) \),
where \(a^2 = R^2\), the square radius of the sphere.

For a hyperboloid, we will have:
\(\dd[]{l^2} = a^2 \qty(\dd[]{\theta^2} + \sinh^2 \theta \dd[]{\varphi^2}) \),
therefore the only difference is that trigonometric functions become hyperbolic ones.

For both of these, let us define the variable: \(r = \sin\theta \) in the spherical case, and \(r = \sinh \theta \)  in the hyperbolic case.
Then, these line elements become respectively:

\begin{subequations}
\begin{align}
    \dd[]{l^2}_{\text{sphere}} &= a^2 \qty(\frac{\dd[]{r^2}}{1 - r^2}  + r^2 \dd{\varphi^2}  )\\
    \dd[]{l^2}_{\text{hyperboloid}} &= a^2 \qty(\frac{\dd[]{r^2}}{1 + r^2} + r^2 \dd{\varphi^2}  )\,.
\end{align}
\end{subequations}

We can rewrite the RW element as:

\begin{equation}
  \dd{l^2} = c^2 \dd{t^2} - a^2 \begin{cases}
      \dd{\chi^2} + \sin^2 \chi \dd{\Omega^2} \\
      \dd{\chi^2} + \chi^2\dd{\Omega^2} \\
      \dd{\chi^2} + \sinh^2 \chi \dd{\Omega^2}
\end{cases}
\end{equation}

where if \(k = +1\)  then \(r = \sin \chi\), if \(k=0\)  then \(r=\chi\), and if \(k = -1\) then \(r = \sinh \chi\).

If we wish to use cartesian coordinates we will have:

\begin{equation}
  \dd{s^2} = c^2 \dd{t^2} - a^2 (t) \qty(1 + \frac{k \abs{x}^2}{4} )^{-2} \qty(\dd{x^2} + \dd{y^2} + \dd{z^2}) \,.
\end{equation}

Universes in which \(a\) is a constant are called \emph{Einstein spaces}.
We can change time variable, defining \(\dd{t} = a(\eta) \dd{\eta} \), where \(a(\eta) \defeq a(t(\eta))\):  so, we will have

\begin{equation}
  \dd{s^2} = a^2 (\eta) \qty(c^2 \dd{\eta^2} - \qty(\frac{\dd{r^2}}{1 - kr^2} + r^2 \dd{\Omega})) \,.
\end{equation}

The parameter \(\eta\) is called \emph{conformal time}: RW is said to be \emph{conformal} to Minkowski.
Conformal geometry is particularly useful for systems which have no characteristic length.

Photons do not have a characteristic length: they do not perceive the expansion of spacetime.

The photons of the CMB look like they are thermal: they were thermal originally, and remained such despite the expansion of the universe.

For now we did not use any dynamics, but we will insert them later.

The Friedmann equations are:

\begin{subequations} \label{eq:friedmann-equations} 
\begin{align}
    \dot{a}^2 &= \frac{8 \pi G }{3} \rho a^2 - kc^2  \\
    \ddot{a} &= - \frac{4 \pi G }{3} a  \qty(\rho  + \frac{3P}{c^2} )  \\
    \dot{\rho} &= \frac{-3 \dot{a} }{a} \qty(\rho + \frac{P}{c^2} )
\end{align}
\end{subequations}
%
where dots denote differentiation with respect to the proper time of a cosmological observer, \(t\), which is called \emph{cosmic time}.
These imply that the energy density \(\rho = \rho( t)\) and the isotropic pressure \(P = P(t)\) only depend on \(t\).

An important parameter is \(H(t) \defeq \dot{a} / a\), the \emph{Hubble parameter}.
We can write an equation for it from the first Friedmann one:

\begin{equation}
  H^2 = \frac{8 \pi G}{3} \rho - \frac{kc^2}{a^2}
\end{equation}

If \(k=0\), then there we have a critical energy density \(\rho_C (t) = 3 H^2 (t) / (8 \pi G)\): we call \(\Omega(t) = \rho(t) / \rho_C(t)\).
Is \(\Omega\) larger or smaller than 1? This tells us about the sign of \(k\); while measuring \(k\) directly is \emph{very} hard. The former is a ``Newtonian'' measurement, while the latter is a ``GR'' measurement.

We define:
\begin{equation}
  H_0 = H(t_0) = 100 h \times \SI{}{\kilo\metre\per\second\per\mega\parsec}
\end{equation}
%
where \(h\) is a number: around \(0.7\), while  \(t_0\) just means \emph{now}.

\subsection{Energy density}

How do we measure it? We want the energy density \emph{today} of \emph{galaxies}: \(\rho_{0g}\).
This is \(\mathscr L_g \expval{M/L} \), where \(\mathscr L _g\) is the mean (intrinsic, bolometric) luminosity of galaxies per unit volume, while \(M/L\) is the mass to light ratio of galaxies.
It is measured in units of \(M_{\odot} / L_{\odot}\). Reference values for these are \(M_{\odot} \sim \SI{1.99e33}{g}\), while \(L_\odot \sim \SI{3.9e33}{\erg\per\second} \).

How do we measure this? We have a trick:

\begin{equation}
  \mathscr L _g = \int_0^\infty \dd{L} L \Phi(L)
\end{equation}

where \(\Phi(L)\) is the number of galaxies per unit volume and unit luminosity: the \emph{luminosity function}. With of our observations we estimate the shape of \(\Phi(L)\). We know that the integral must converge, so we can bound the shape of \(\Phi\) (?).

\end{document}
