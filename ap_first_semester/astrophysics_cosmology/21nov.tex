\documentclass[main.tex]{subfiles}
\begin{document}

\marginpar{Thursday\\ 2019-11-21, \\ compiled \\ \today}
% \section*{Thu Nov 21 2019}

We exponentiate the equation from before: 
we get 
%
\begin{align}
  X_d = X_n X_p
  \exp(-29.33 + \frac{25.82}{T_\rho  } - \frac{3}{2} \log T_\rho + \log ( \Omega_0 h ))
\,,
\end{align}
%
where \(T_\rho  = \) 
\todo[inline]{What is going on? What is \(T_{\rho }\)? }

We want to understand why there is so much He-4 in the universe, since it is destroyed in stars! 

This model fits observation as long as \(\num{.011} h^{-2}\leq \Omega_0 \leq \num{.25} h^{-2}\). 
Most people agree that we are around the upper bound. 

\todo[inline]{This is indirect evidence for dark energy: why?}

The lifetime of the neutron, \(\tau_{1/2}\) is something that is also relevant, since it affects the baryon ratios. 

The Gamow factor \(\Gamma \) is proportional to the Fermi coupling constant \(G_F^2\), which is connected to \(\tau_{1/2}\). 

Let us suppose we increase the lifetime of neutrons, \(\tau_{1/2}\). 
This changes the moment at which we reach equation \(\Gamma \sim H\). 

Increasing the lifetime of neutrons decreases the amount of He-4 in the universe: less is produced. 

We know that 
%
\begin{align}
  H^2 = \frac{8 \pi G}{3} \rho_r
\,,
\end{align}
%
where \(G = 1 / m_p^2\) and \(\rho = \frac{\pi^2}{30} g_{*}(T) T^4\).

If we fix the temperature, and change \(g_{*}\) (by adding degrees of freedom), then we get more He-4. 

This gives us observational constraints on the additions of exotic particles to our theory, since that would change \(g_{*}\). 
This bounds the number of neutrino families by \num{3.} something, so there cannot be more than \num{3} families of light neutrinos. 

If gravitons are thermal, then they also contribute to radiation. 

\section{Dark Matter dynamics}

There is also another parameter in the Friedmann equation; it is \(m_P\): modified gravity theories often predict variations of the gravitational constant with time. 

Dark matter has no relevant electromagnetic interactions: it only interacts gravitationally, and is able to cluster; dark energy, instead, is uniformly distributed. 

We divide it into Hot and Cold dark matter: HDM and CDM. 
There is also something called \emph{Warm} dark matter, which has intermediate properties.

In HDM, particles have very high thermal motion. They move fast, and tend to destroy gravitational potential wells in which they might settle by moving out of them, and thus decreasing the quantity of matter there. 

They do this on scales comparable to the maximum distance travelled by them: this is calculated as \(vt\), where \(v\) is their average thermal velocity, (and \(t\) is the age of the universe?). 

The structures formed by these are of scales similar  to or larger than \(\num{e15} M_{\odot}\), but we observe smaller structures also! They were formed later, by fragmentation: this is the top-down approach. 

We also have a bottom-up approach, which is compatible with CDM. 

Neutrinos were thought to be Dark Matter, and would have been hot. 

The top-down approach, however, is falsified by the observation of high-redshift quasars combined with the scale of the anisotropies of the CMB: in order to account fo high-redshift small-scale structures (we have seen stars at \(z \sim 20\)!) we would have to increase the amplitude of the anisotropies to a scale which is not compatible to the anisotropies we see in the CMB. 

Now, neutrinos are not useful for cosmology. 

We have \(\Gamma \sim T^{5}\), and \(\Gamma = H\): 
\(T^{5} / \tau_{1/2} \sim T^2\) implies that \(\tau_{1/2} \sim T^{3}\). 

The decouplig temperature for HDM is larger than the temperature at which they become nonrelativistic, which is of the order of the mass. 

For CDM, instead, the decoupling temperature is \emph{smaller} than the temperature at which they become relativistic. 

Now, we discuss the Boltzmann equation: there is an operator acting in phase space, the Liouville operator, which is equal to the collision operator.
%
\begin{align}
  \mathbb{L} [f] = \mathbb{C} [f]
\,,
\end{align}
%
where all the scattering operators live on the RHS. 

We start with a Newtonian description: the phase space has position, momentum and time as coordinates, and on it we define a density function \(f(\vec{q}, \vec{p}, t)\). 

The Einstein equations are blind to the momentum distribution, since \(T_{\mu \nu }\) does not depend on the momentum. We can say that all of the momentum has been marginalized. 

We define the operator 
%
\begin{subequations}
\begin{align}
  \hat{\mathbb{L}} = \frac{D}{Dt} &= \pdv{}{t} + \dv{\vec{x}}{t} \cdot \nabla_x + \dv{\vec{p}}{t} \cdot \nabla_p   \\
  &= \pdv{}{t} + \vec{v} \cdot \nabla_x + \frac{\vec{F}}{m} \cdot \nabla_p 
\,. 
\end{align}
\end{subequations}

In GR, we geometrize the gravitational force: it is included in the inertial motion of the particles. 

The relativistic version of this is 
%
\begin{align}
  \hat{\mathbb{L}} = p^{\alpha } \partial_{\alpha } - \Gamma^{\alpha }_{\beta \gamma } p^{\beta } p^{\gamma } \pdv{}{p^{\alpha }}
\,.
\end{align}
%

This is because \(p^{\alpha } = \dv*{x^{\alpha }}{\lambda }\) satisfies the geodesic equation: 
%
\begin{align}
  \dv{p^{\alpha }}{\lambda } + \Gamma^{\alpha }_{\beta \gamma } p^{\beta } p^{\gamma }=0
\,,
\end{align}
%
or \(\frac{Dp^{\alpha }}{Dt } =0\). We can then write down the Christoffel bit of the geodesic equation instead of the derivative \(\dv*{p^{\alpha }}{\lambda }\). This is because, \emph{on shell}: 
%
\begin{align}
  g^{\alpha \beta } p_{\alpha } p_{\beta } = m^2
\,. 
\end{align}

We want to describe the \emph{abundance} of dark matter, in phase space: the number of DM particle per EM-interacting particle. 

We then make the Liouville operator explicit, with the Christoffel symbols of the RW metric. Acting on \(f\), we get 
%
\begin{align}
  \hat{\mathbb{L}}  = p^{0} \partial_t f - \frac{\dot{a}}{a} \abs{\vec{p}}^2 \pdv{f}{E}
\,,
\end{align}
%
which we will not prove. 

The derivative with respect to the momentum must be \(\partial_{p^{0}}\),

\todo[inline]{because of isotropy?}

the Christoffel symbols of RW are 
%
\begin{align}
  \Gamma^{0}_{\beta \gamma } = \frac{\dot{a}}{a} \delta_{\beta \gamma }
\,.
\end{align}
%
Recall the definition of the number density 
%
\begin{align}
  n(t) = \frac{g}{(2\pi )^3} \int  \dd[3]{\vec{p}} f(\abs{\vec{p}}, t) 
\,,
\end{align}
%
so in the end the equation becomes: 
%
\begin{align}
  \hat{\mathbb{L}} = \dot{n} + 3 \frac{\dot{a}}{a} n 
\,. 
\end{align}
%

Conventionally we divide by \(p^{0}\): we get for the Boltzmann equation: 
%
\begin{align}
  \pdv{f}{t} - \frac{\dot{a}}{a } \frac{p^2}{E} \pdv{f}{E} = \frac{1}{E} \hat{\mathbb{C}} [f]
\,.
\end{align}

Now we integrate in \(\dd[3]{\vec{p}}\). We get 
%
\begin{align}
  \pdv{n}{t} - \frac{\dot{a}}{a } \int  \dd[3]{\vec{p}} \frac{p^2}{E} \dv{f}{E} = \frac{g }{(2\pi )^3} \int  \dd[3]{p} \frac{1}{E} \mathbb{C} [f]  
\,.
\end{align}
%
More properly, we are doing \(\mathbb{L} \expval{f}\). 

We manipulate: 
%
\begin{subequations}
\begin{align}
  \int \dd[3]{p} \frac{p^2}{E} \pdv{f}{E} 
  &= 2 \int \dd[3]{p} p^2 \pdv{f}{E^2} \\
  = 2 \int \dd[3]{p} p^2 \pdv{f}{p^2} 
  &= \int \dd[3]{p} p \pdv[]{f}{p} \\
  = 2 \pi \int_{0}^{ \infty } \dd{p} p^3 \pdv{f}{p} 
  &= -3 \int \dd[3]{p} f
\,.
\end{align}
\end{subequations}
%

Here, we integrated by parts in the second to last step, and set to zero the boudary term \(4 \pi p^3 f\), calculated from 0 to infinity, since at \(0\) we have \(p=0\), and at (momentum) infinity we have \(f=0\).

So, we can see that the LHS is equal to \(\dot{n} + 3 \dot{a}n/a \): so we find 
%
\begin{align}
  \dot{n} + 3 \frac{\dot{a}}{a} n = \frac{g}{(2 \pi )^3} \int \dd[3]{p} \frac{1}{E} \hat{\mathbb{C}} [f]
\,.
\end{align}

This is the cosmological version of the Boltzmann equation. 

We model the RHS as something like 
%
\begin{align}
  \hat{\mathbb{C}}[f] = \Psi - \expval{\sigma v } n^2
\,,
\end{align}
%
where we have \(\Gamma_A = \expval{\sigma v} n\) times \(n\), where \(\Gamma \) is the rate of annihilation. 

At equilibrium the LHS is 
equal to zero: so we can write \(\Psi = \expval{\sigma v} n^2 _{\text{eq}}\), and the RHS becomes \(\expval{\sigma v} \qty(n^2 _{\text{eq}} - n^2)\).

\end{document}