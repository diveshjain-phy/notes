\documentclass[main.tex]{subfiles}
\begin{document}

% \section*{Thu Oct 10 2019}
\marginpar{Thursday\\ 2019-10-10}

% Today at 17 there is a colloquium at Aula Rostagni.

% Next week, also at 17, there is a meeting at San Gaetano (for the general public).

% We just had a Nobel prize in cosmology: he predicted the existence of the CMB, and two in astrophysics, they found the first exoplanet.

% Send the professor an e-mail to get access to his Dropbox folder with many relevant texts.

% We come back to the RW metric:

% \begin{equation}
%   \dd{s^2} = c^{2} \dd{t^{2}} - a^2(t) \qty(\frac{\dd{r^2}}{1-kr^2} + r^2 \dd{\Omega^2})\,,
% \end{equation}
% %
% where \(k=0, \pm 1\) and \(r\) is the dimensionless comoving radius.

Another way of defining a distance is the one we get by directly integrating the radial part of the line element: this way, we are effectively using a space-like measuring stick, working at fixed cosmic time and fixing both of the angles \(\theta \) and \(\varphi \).

% The distance can be calculated as:
% %
% \begin{equation}
%   d_P (t) = a(t)\frac{r}{\sqrt{1-kr^2}}\,.
% \end{equation}

% We are allowed to consider radial trajectories, however this does not account for the fact that we cannot actually measure with a space-like measuring stick.

\begin{definition}[Proper distance]
The proper distance \(d_P\) at a fixed cosmic time \(t \) to an object at a comoving radial coordinate \(r\) is given by: 
%
\begin{equation}
d_P = a(t) \int_0^r \frac{\dd{\widetilde{r}}}{\sqrt{1-k \widetilde{r}^2}}\,.
\end{equation}
\end{definition}
  
% This is called the \emph{proper distance} (the subscript \(P\) is for ``proper''): the integration, at zero \(\dd{\Omega}\) and zero \(\dd{t}\), of the radial part of the metric.

% On the other hand (OTOH), the flux of photons can give us the \emph{luminosity distance}, which is actually measurable.

\subsubsection{A derivation of the Hubble law}

We want to derive the Hubble law (\(v = H_0 d\)) mathematically.
It can also be restated as \(cz = H_0 d\): the observed velocity of recession of the objects is measured through redshift, which is described (in  the nonrelativistic limit\footnote{The relativistic expression is 
%
\begin{align}
1 + z = \sqrt{\frac{1 + v/c}{1 - v/c}} \gtrsim 1+ \frac{v}{c}
\,.
\end{align}})
by the formula 
%
\begin{align}
\lambda _0 = \lambda _{e} \qty(1 \pm \frac{v}{c})
\,,
\end{align}
%
where the sign in the \(\pm\) is a plus if the object is receding from us.
We also defined \(z\) through \(\lambda_0 / \lambda_{e}  = 1+z\): this means that we can identify \(z = v/c\).

% For a fixed source at \(d = ar\), naïvely we would have:
% %
% \begin{equation}
%   \dot{d} = \dot{a} r = \frac{\dot{a} }{a} ar = H_0  d
% \end{equation}

Now, we are going to ``move away from the current epoch by Taylor expanding'':
the scale factor at a time \(t\), \(a(t)\), can be written as
%
\begin{subequations}
\begin{align}
    a(t) &= a_0 + \dot{a}_0 (t-t_0 ) + \frac{1}{2} \ddot{a}_{0}(t-t_0)^2 + \mathcal{O}\qty(\abs{t-t_0}^3)
    \marginnote{We drop the error term}\\
    &\approx a_0 \qty(1 + H_0 (t-t_0 ) - \frac{1}{2} q_0 H_0 ^2 (t-t_0 )^2)\,, \marginnote{Substituted \(H_0 = \dot{a}_{0} / a_0 \).} \label{eq:scale-factor-expansion-second-order}
\end{align}
\end{subequations}
%
where \(q_0 \defeq - \ddot{a}_0 a_0 / (\dot{a}_0 )^2\) is called the \emph{deceleration parameter} by historical reasons: people thought they would see deceleration \(\ddot{a}<0\) when first writing this, so a positive \(q_0 \), but
the deceleration parameter is instead measured to be negative.

% (the Hubble \emph{constant} is not constant wrt \emph{time} , but wrt \emph{direction}!)

Now, \(1 + z = a_0 / a\) can be expressed as:

\begin{equation}
    1+ z \simeq \qty(1 + H_0  (t-t_0 ) - \frac{1}{2} q_0 H_0 ^2 (t-t_0 )^2)^{-1}\,.
\end{equation}

Do note that this is derived starting with a formula which is correct up to second order in the time interval \(\Delta t = t - t_0 \): when expanding we cannot trust terms of order higher than second.
Expanding with this in mind we get:\footnote{We need to compute the first and second derivatives of 
%
\begin{align}
\qty(1 + H_0 \Delta t - \frac{q_0}{2} H_0^2 \Delta t)^{-1}
= \qty(1 + a \Delta t + b \Delta t)^{-1} = f(\Delta t)
\,:
\end{align}
%
they are 
%
\begin{align}
\dv{f}{\Delta t} &= - \qty(1 + a \Delta t + b \Delta t)^{-2} \qty(a + 2 b \Delta t)  \\
\dv[2]{f}{\Delta t} &= 2\qty(a + 2b \Delta t)^2 \qty(1 + a \Delta t + b \Delta t) - \qty(1 + a \Delta t + b \Delta t)^{-2} \qty(2 b)
\,,
\end{align}
%
so we have 
%
\begin{align}
\eval{\dv{f}{\Delta t} }_{\Delta t = 0} = x
\qquad \text{and} \qquad
\frac{1}{2!}\eval{\dv[2]{d}{\Delta t}}_{\Delta t = 0} = \frac{1}{2!} \qty(2 a^2- 2b) = a^2 - b
\,.
\end{align}}
%
\begin{equation}
  1+ z \simeq
  1 - H_0 \Delta t + \frac{q_0 }{2} H_0^2 \Delta t^2 + H_0^2 \Delta t^2\,,
\end{equation}
%
therefore
%
\begin{align}
  z &= H_0 (t_0 - t) + \qty(1 + \frac{q_0}{2}) H_0^2 (t_0 - t)^2 \marginnote{Changed the time intervals to \(t_0 -t = - \Delta t\), the square of which is the same as before.}  \\
  &= (t_0 - t) \qty[H_0 + \qty(1 + \frac{q_0}{2})H_0^2(t_0 - t) z]
  \,.
\end{align}

Bringing the bracket to the other side we get
%
\begin{align}
  t_0 - t &= z \qty[H_0 + \qty(1+\frac{q_0}{2})H_0^2 (t_0 -t)]^{-1}  \\
  &= z \qty[H_0 + \qty(1+\frac{q_0}{2})H_0 z]^{-1}  \\
  &= \frac{z}{H_0}  \qty[1+ \qty(1+\frac{q_0}{2}) z]^{-1}
  \,,
\end{align}
%
where we substituted the first order expression \(t_0 -t = z/ H_0 \): we are allowed to make this substitution since the expression is multiplied by \(z\), which has the same asymptotic order as \(t_0 -t\) (since \(H_0 \) is finite): working to first order inside the brackets is equivalent to working to second order in the global expression.

By the same reasoning, we can expand the inverse bracket to first order: 
%
\begin{align} \label{eq:time-difference-wrt-redshift}
t_0 - t = \frac{z}{H_0 } \qty[1 - \qty(1 + \frac{q_0}{2} z)]
= \frac{z}{H_0 } - \qty(1 + \frac{q_0}{2}) \frac{z^2}{H_0 }
\,,
\end{align}
%

% As before, when we expand we keep only the relevant orders:
% %
% \begin{equation}
%   t_0 - t = H_0 (t_0 - t) + \qty(1+ \frac{q_0}{2}H_0^2 (t_0 - t)^2)
% \end{equation}

We would like the time interval to disappear: we want a distance, not a time, so we should seek an expression for \(r\) instead of \(\Delta t\).
We are observing photons, for which \(\dd{s^2} =0\), which is equivalent \(c^2 \dd{t^2} = a^2 (t) \dd{r^2} / (1-kr^2) \).
Taking a square root and integrating we get:
\begin{equation}
\int_{t}^{t_0 } \frac{c \dd{t}}{a(t)}
= \pm \int_r^0 \frac{\dd{\widetilde{r}}}{\sqrt{1-k \widetilde{r}^2}}
\,,
\end{equation}
%
where we should select the negative sign since we want positive quantities on both sides. The other choice would correspond to the photon being emitted from the Earth and received at the comoving radius of the source. 

The integral on the right hand side can be solved analytically: it is 

\begin{equation} \label{eq:integral-radial-FLRW}
-\int_{r}^{0} \frac{ \dd{\widetilde{r}}}{\sqrt{1 - k \widetilde{r}^2}} =
\begin{cases}
  \arcsin r =r + \mathcal{O}(r^3) &k =1 \\
  r  &k=0 \\
  \operatorname{arcsinh} r =r + \mathcal{O}(r^3)\qquad\qquad  &k= -1
\end{cases}
\end{equation}
%
in all cases, it is just \(r\) up to \emph{second} order (since the next term in the expansion of an arcsine or hyperbolic arcsine is of third order).

On the left hand side, we can substitute in \(a(t)\) from equation \eqref{eq:scale-factor-expansion-second-order}:
\begin{align}
\int_{t}^{t_0} \frac{c\dd{t}}{a(t)} 
&= \frac{c}{a_0 }
\int_{t }^{t_0 } \dd{\widetilde{t}} \qty[1 + H_0 (\widetilde{t} - t_0) + \mathcal{O}(\Delta \widetilde{t}^2)]^{-1}  \\
&= \frac{c}{a_0 } \qty[t_0  - t + \frac{H_0}{2} \qty(t-t_0 )^2] + \mathcal{O}(\Delta t^3)
\end{align}
%
where we used the expression for the scale factor to first order only since the integration raised the order of the estimate by one.
we have:
\begin{align}
\frac{c}{a_0 } \qty((t_0 - t) + \frac{1}{2} H_0 (t_0 - t)^2 + \mathcal{O}(\Delta t ^3)) &= r  
\,,
\end{align}
%
since the term proportional to \(q_0 \) only gives a third order contribution.
We can now substitute the expression for the time difference with respect to the redshift \eqref{eq:time-difference-wrt-redshift}, only computed to second order: 
%
\begin{align}
r &= \frac{c}{a_0 } \qty[\frac{z}{H_0 } \qty(1 -\qty(1 + \frac{q_0}{2})z)+ \frac{H_0 }{2} \qty(\frac{z}{H_0 } \qty(1 -\qty(1 + \frac{q_0}{2})z))^2]  \\
&= \frac{c}{a_0 } \qty[\frac{z}{H_0 } - \qty(1+\frac{q_0}{2})\frac{z^2}{H_0 } + \frac{H_0}{2} \frac{z^2}{H_0^2}] \marginnote{Ignored the third and higher order terms in the square.}  \\
&= \frac{c}{a_0 H_0 } \qty[z - \frac{1}{2} \qty(1 + q_0 )z]
\,.
\end{align}
%

Now, we can insert this expression for \(r\) into the formula for the luminosity distance \eqref{eq:luminosity-distance}:
\begin{align}
d_L &= a_0^2 \frac{r}{a} = a_0 (1+z) r \\
&= a_0 (1+z) \frac{c}{a_0 H_0 } \qty(z - \frac{1}{2} (1+ q_0 )z^2)
\,.
\end{align}

As we expect, the term \(a_0 \) disappears: it is a bookkeeping parameter, the physical properties of a universe described by a FLRW metric are invariant under a global rescaling of the scale factor.

Our expression also contains cubic terms in \(z\): removing these to get back to second order we find
\begin{align}
d_L &= \frac{cz}{H_0} 
\qty(1+z) \qty(1- \frac{1}{2}\qty(1+q_0)z)  \\
&= \frac{cz}{H_0 } \qty(1 + \qty( 1 - \frac{1}{2} - \frac{1}{2}q_0 )z)  \\
&= \frac{cz}{H_0 } \qty(1 + \frac{1}{2}\qty(1 - q_0) z)
% \qty(z + \frac{1}{2} (1-q_0 ) z^2)
\,.
\end{align}

We can turn this into a relation for \(cz\) in terms of \(d_L\) by substituting in the first order expression \(d_L H_0 = cz\) into the second order term:
\begin{align}
cz &= d_L H_0 + \frac{q_0 -1}{2} cz^2 \\
cz &= H_0 \qty(d_L + \frac{1}{2} (q_0 -1 ) \frac{H_0}{c} d_L^2)\,,
\end{align}
%
and we can notice that the relation is approximately linear and independent of acceleration for low redshift, but we can detect the acceleration at higher redshift.
Typically we need to measure galaxies at least \SI{10}{\mega\parsec} away in order to detect these second order effects.
As we mentioned in the beginning, the data show the parameter \(q_0 \) to be negative.

This effect is similar to a Doppler effect, but the analogy is not perfect: the redshift is caused by the expansion of space itself, and the apparent velocities of the galaxies at high redshift would be superluminal. 

\todo[inline]{The reasoning in \cite{Pacciani:2018} is not completely convincing: extrapolating a superluminal velocity from \(cz \sim 10c\) is meaningless, since these formulas were derived in the nonrelativistic approximation. If we use the correct relativistic formula for the Doppler shift this issue does not arise. We might be able to show that the apparent speed of high-redshift galaxies is superluminal but we need another way to do it. (Decisamente hai ragione, aggiungi poi che la relazione vale per z piccolo e 10 non è piccolo solitamente. Ad ogni modo ha ragione a dire che il rate di espansione potrebbe essere maggiore di c senza creare problemi alla relatività)}

\begin{definition}[Angular diameter distance]
The angular diameter distance is defined as the ratio of the object's physical transverse size \(L\) to its angular size in radians \(\Delta \theta \): 
%
\begin{align}
d_A = \frac{L}{\Delta \theta } = a(t) r
\,,
\end{align}
%
which is peculiar in that it is not monotonic in \(z\) \cite{hoggDistanceMeasuresCosmology2000}: at \(z \gtrsim 1\) it starts decreasing.
\end{definition}

% We can do better than that if we go from \emph{cosmography} to \emph{cosmology}, by understanding what causes the acceleration of the expansion of the universe.

\subsubsection{Redshift-scale factor relation}

Let us prove the statement from before, \(\lambda_0 / \lambda  = a_0 / a\):
photons are emitted with a certain wavelength \(\lambda_{e}\), at a comoving radius \(r\) from us, and detected at \(\lambda_{o}\).

The line element for the photon is \(\dd{s^2} =0\), therefore \(c\dd{t}/ a(t) = \pm \dd{r} / \sqrt{1-kr^2} \).

As before, we can integrate this relation from the emission to the absorption: we call it \(f(r)\) (it can be any of the functions shown in equation \eqref{eq:integral-radial-FLRW}):
\begin{equation}
  \int_{t}^{t_0 } = \frac{c\dd{\widetilde{t}}}{a(\widetilde{t})} = f(r)
\end{equation}

If we map \(t \rightarrow t + \delta t\) and \(t_0 \rightarrow \delta t_0 \) in the integration limits, the integral must be constant since it only depends on \(r\) --- do note that all the expansion of the universe is accounted for by the increasing scale factor, objects are stationary with respect to the comoving radial coordinate.
We are computing the integral for two successive wavefront of the light.
Then, we equate the two: 
%
\begin{align}
f(r) &= \int_{t}^{t_0 } \frac{c \dd{\widetilde{t} }}{a(\widetilde{t})} = \int_{t+\delta t}^{t_0 + \delta t_0 } \frac{c \dd{\widetilde{t} }}{a(\widetilde{t})}
\,,
\end{align}
%
which we can split into: 
%
\begin{align}
\qty[\int_{t+ \delta t}^{t} + \int_{t}^{t_0} + \int_{t_0 }^{t_0 + \delta t_0 } - \int_{t}^{t_0 }] \frac{c \dd{\widetilde{t}}}{a(\widetilde{t})} = 0
\,,
\end{align}
%
where, since all the integrals have the same argument, we collect it at the end for clarity.
We simplify the original integral and swap the integration limits to get:
\begin{equation}
  \int_{t}^{t + \delta t} \frac{c\dd{\widetilde{t}}}{a(\widetilde{t})} = \int_{t_0 }^{t_0 + \delta t_0 } \frac{c\dd{\widetilde{t}}}{a(\widetilde{t})}\,,
\end{equation}
%
which can we approximate by
\begin{equation}
\frac{c \delta t}{a(t)} =   \frac{c \delta t_0 }{a(t_0 )}\,,
\end{equation}
%
since the periods of the photons we are considering are generally much smaller than the cosmic timescales.

Since the frequency of the emitted and observed photons must be proportional to the inverse of the time intervals \(\delta t\) or \(\delta t_0 \), we have
\begin{equation}
  \nu_{e} a(t_{e}) = \nu_{o} a(t_{o})\,,
\end{equation}
%
therefore 
%
\begin{equation}
  1 + z = \frac{\lambda_{o}}{\lambda_{e}}
  = \frac{a_0 }{a}\,.
\end{equation}

\chapter{Friedmann models}

\subsection{A Newtonian derivation of the Friedmann equations}

It is useful to do it first this way, pedagogically, although the proper derivation is done starting from the Einstein Field Equations.

Let us take a uniform spacetime with density \(\rho\). We consider a sphere, and take all the mass inside the sphere away.

Consider Birkoff's theorem: the gravitational field of a spherically symmetric body is always described by the Schwarzschild metric.
This can also be applied to a hole: in this case, it tells us that the metric inside the cavity is the Minkowski metric.

The mass taken away will be \(M(\ell) = \frac{4 \pi}{3} \rho \ell^{3}\), where \(\vec{l} = a(t) \vec{r}\) is the radius of the sphere.

We suppose that the gravitational field is \emph{weak}:
\begin{equation}
  \frac{GM(\ell)}{\ell c^2} \ll 1\,.
\end{equation}

We put a test mass on the surface of the sphere. What is the motion of the mass due to the gravitational field from the center? It will surely be radial, therefore
\begin{equation}
  \ddot{\ell} = - \frac{GM(\ell)}{l^2} = - \frac{4 \pi G}{3} \rho \ell\,.
\end{equation}

This seems to give us a net force even though by our hypotheses there should be none, this is actually not an issue since the unit vectors in our equations will go away in the end.

Then, we have
\begin{equation}
  \ddot{a} r = - \frac{4 \pi G}{3} \rho a r
  \to
  \ddot{a} = - \frac{4 \pi G}{3} \rho a\,.
\end{equation}

This is part of a Friedmann equation: the isotropic pressure term cannot be recovered, it's like the speed of light is infinite now.

Now consider 
\begin{equation}
  \dot{\ell} \ddot{\ell} = - \frac{GM}{\ell^2} \dot{\ell}\,,
\end{equation}
%
therefore 
%
\begin{equation}
  \frac{1}{2}\dot{\ell}^2 = \frac{4 \pi}{3} G \rho \ell^{3} + \const\,,
\end{equation}
%
and integrating we get
%
\begin{equation}
  \dot{a}^2 r^2 = \frac{8 \pi G}{3} \rho a^2 r^2 + \const
\end{equation}
%
or, removing the \(r^2\) term, which is a constant,
%
\begin{equation}
    \dot{a}^2 = \frac{8 \pi G}{3} \rho a^2 + \const\,.
\end{equation}

We know that this new constant is related to the energy per particle. 

A universe with negative \(k\) expands forever and so on.

The number \(k\) was badly defined to be only \(\pm 1\) or 0: its newtonian version is much better represented by \(k_N = kc^2\).

\end{document}