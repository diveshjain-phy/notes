\documentclass[main.tex]{subfiles}
\begin{document}

\section*{Thu Oct 10 2019}

Today at 17 there is a colloquium at Aula Rostagni.

Next week, also at 17, there is a meeting at San Gaetano (for the general public).

We just had a Nobel prize in cosmology: he predicted the existence of the CMB, and two in astrophysics, they found the first exoplanet.

Send the professor an e-mail to get access to his Dropbox folder with many relevant texts.

We come back to the RW metric:

\begin{equation}
  \dd{s^2} = c^{2} \dd{t^{2}} - a^2(t) \qty(\frac{\dd{r^2}}{1-kr^2} + r^2 \dd{\Omega^2})\,,
\end{equation}
%
where \(k=0, \pm 1\) and \(r\) is the dimensionless comoving radius.

The distance can be calculated as:
%
\begin{equation}
  d_P (t) = a(t)\frac{r}{\sqrt{1-kr^2}}\,.
\end{equation}

We are allowed to consider radial trajectories, however this does not account for the fact that we cannot actually measure with a space-like measuring stick.

A better definition is
%
\begin{equation}
  d_P = a(t) \int_0^r \frac{\dd{\widetilde{r}}}{\sqrt{1-k \widetilde{r}^2}}\,.
\end{equation}

This is called the \emph{proper distance} (the subscript \(P\) is for ``proper''): the integration, at zero \(\dd{\Omega}\) and zero \(\dd{t}\), of the radial part of the metric.

On the other hand (OTOH), the flux of photons can give us the \emph{luminosity distance}, which is actually measurable.

We want to derive the Hubble law (\(v = H_0 d\)) mathematically. It can also be restated as \(cz = H_0 d\). 

\begin{greenbox}
  why?
\end{greenbox}

For a fixed source at \(d = ar\), naïvely we would have:
%
\begin{equation}
  \dot{d} = \dot{a} r = \frac{\dot{a} }{a} ar = H_0  d
\end{equation}

We can ``move away from the current epoch by Taylor expanding''.

The scale factor \(a(t)\) can be written as

\begin{subequations}
\begin{align}
    a(t) &\simeq a_0 + \dot{a}_0 (t-t_0 ) + \frac[i]{1}{2} \ddot{a}(t-t_0)^2\\
    &= a_0 \qty(1 + \frac{\dot{a} }{a_0}|_{t_0 } (t-t_0 ) - \frac{1}{2} q_0 H_0 ^2 (t-t_0 ))\,,
\end{align}
\end{subequations}
%
where \(q_0 \defeq - \ddot{a_0 } a_0 / (\dot{a_0 } )^2\) is called the \emph{deceleration parameter} by historical reasons: people thought they would see deceleration when first writing this.

(the Hubble \emph{constant} is not constant wrt \emph{time} , but wrt \emph{direction}!)

The deceleration parameter is actually measured to be negative.

Now, \(1 + z = a_0 / a\) can be expressed as:

\begin{equation}
    1+ z \simeq \qty(1 + \frac{\dot{a} }{a_0}|_{t_0 } (t-t_0 ) - \frac{1}{2} q_0 H_0 ^2 (t-t_0 ))^{-1}\,.
\end{equation}

Do note that this is derived by using at most second order in the distance interval: when expanding we cannot trust terms of order higher than second. Expanding with this in mind we get:
%
\begin{equation}
  1+ z \simeq
  1 - H_0 \Delta t + \frac{q_0 }{2} H_0^2 \Delta t^2 + H_0^2 \Delta t^2\,,
\end{equation}
%
therefore
%
\begin{equation}
  z = H_0 (t_0 - t) + \qty(1 + \frac{q_0}{2}) H_0^2 (t_0 - t)^2\,.
\end{equation}

Rearranging the equation we get
%
\begin{equation}
  t_0 - t = z \qty(H_0 + \qty(1+\frac{q_0}{2})H_0 z)^{-1}\,,
\end{equation}
%
but as before we must expand to get only the relevant orders:
\begin{equation}
  t_0 - t = H_0 (t_0 - t) + \qty(1+ \frac{q_0}{2}H_0^2 (t_0 - t)^2)
\end{equation}

We would like the time interval to disappear: for photons \(\dd{s^2} =0\), therefore in that case \(c^2 \dd{t^2} = a^2 (t) \dd{r^2} / (1-kr^2) \). Taking a square root and integrating:
\begin{equation}
  \int_{t}^{t_0 } \frac{c \dd{t}}{a(t)}
  = \pm \int_r^0 \frac{\dd{\widetilde{r}}}{\sqrt{1-kr^2}}\,,
\end{equation}
%
where the plus or minus sign comes from\dots

The integral on the RHS can be solved analytically: it is 

\begin{equation}
  \begin{cases}
      \arcsin r \qquad k=1 \\
      r \qquad k=0 \\
      \operatorname{arcsinh} r \qquad k = -1
  \end{cases}
\end{equation}
%
in all cases, it is just \(r\) up to \emph{second} order (since the next term in the expansion of an arcsine or hyperbolic arcsine etc is of third order).

On the other side, we have:
\begin{equation}
  \frac{1}{a_0 }
  \int_{t_0 }^t c\dd{\widetilde{t}} \qty(1 + H_0 (\widetilde{t} - t_0) - \frac{q_0}{2}H_0^2 (\widetilde{t} - t_0)^2)^{-1}
\end{equation}
%
therefore, neglecting \emph{third} and higher order terms, we have:
\begin{equation}
  \frac{c}{a_0 } \qty((t_0 - t) + \frac{1}{2} H_0 (t_0 - t)^2 + o(\abs{t_0 - t} ^2)) = r \,,
\end{equation}
%
since the term proportional to \(q_0 \) only gives a third order contribution.

The explicit form of the luminosity distance was
\begin{equation}
  d_L = a_0^2 \frac{r}{a} = a_0 (1+z) r\,.
\end{equation}

The term \(a_0 \) should disappear at the end of every calculation: it is a bookkeeping parameter.
Moving on:
\begin{equation}
  d_L = a_0 (1+z) \frac{c}{a_0 H_0 } \qty(z - \frac{1}{2} (1+ q_0 )z^2)\,,
\end{equation}
%
but this also contains cubic terms:
\begin{equation}
  d_L \simeq \frac{c}{H_0} \qty(z + \frac{1}{2} (1-q_0 ) z^2 + o(z^2))\,.
\end{equation}

Therefore:
\begin{equation}
  cz = H_0 \qty(d_L + \frac{1}{2} (q_0 -1 ) \frac{H_0}{c} d_L^2)\,,
\end{equation}
%
and we can notice that the relation is approximately linear and independent of acceleration for low redshift, but we can detect the acceleration at higher redshift.
Typically we need to go at around \SI{10}{\mega\parsec}.

The parameter \(q_0 \) appears to be negative now.

We can do better than that if we go from \emph{cosmography} to \emph{cosmology}, by understanding what causes the acceleration of the expansion of the universe.

Let us expand on the concept of reshift:
photons are emitted with a certain wavelength \(\lambda_{e}\), at a comoving radius \(r\) from us, and detected at \(\lambda_{o}\).
The line element for the photon is \(\dd{s^2} =0\), therefore \(c\dd{t}/ a(t) = \pm \dd{r} \sqrt{1-kr^2} \).

As before, we can integrate this relation from the emission to the absorption: we call it \(f(r)\) (it can be any of the functions shown before).

\begin{equation}
  \int_{t}^{t_0 } = \frac{c\dd{\widetilde{t}}}{a(\widetilde{t})} = f(r)
\end{equation}

If we map \(t \rightarrow t + \delta t\) and \(t_0 \rightarrow \delta t_0 \) in the integration limits, the integral must be constant since it only depends on \(r\).

This can be written as:
\begin{equation}
  \int_{t}^{t + \delta t} \frac{c\dd{\widetilde{t}}}{a(\widetilde{t})} = \int_{t_0 }^{t_0 + \delta t_0 } \frac{c\dd{\widetilde{t}}}{a(\widetilde{t})}\,,
\end{equation}
%
which can we cast into
\begin{equation}
  \frac{c \delta t}{a(t)} =   \frac{c \delta t_0 }{a(t_0 )}\,.
\end{equation}

Since the frequency must be proportional to the inverse of the time intervals \(\delta t\) or \(\delta t_0 \), we have
\begin{equation}
  \nu_{e} a(t_{e}) = \nu_{o} a(t_{o})\,,
\end{equation}
%
therefore 
%
\begin{equation}
  1 + z = \frac{\lambda_{o}}{\lambda_{e}}
  = \frac{a_0 }{a}\,.
\end{equation}

\subsection{A Newtonian derivation of the Friedmann equations}

It is useful to do it first this way, pedagocically.

Let us take a uniform spacetime with density \(\rho\). We consider a sphere, and take all the mass inside the sphere away.

Consider Birkoff's theorem: the gravitational field of a spherically symmetric body is always described by the Schwarzschild metric.
This can also be applied to a hole: in this case, it tells us that the metric inside the cavity is the Minkowski metric.

The mass taken away will be \(M(\ell) = \frac[i]{4 \pi}{3} \rho \ell^{3}\).

Say that the gravitational field is \emph{weak}:
\begin{equation}
  \frac{GM(\ell)}{\ell c^2} \ll 1\,.
\end{equation}

We put a test mass on the surface of the sphere. What is the motion of the mass due to the gravitational field from the center? It will surely be radial, therefore
\begin{equation}
  \ddot{\ell} = - \frac{GM(\ell)}{l^2} = - \frac{4 \pi G}{3} \rho \ell\,.
\end{equation}

Then, we have
\begin{equation}
  \ddot{a} r = - \frac{4 \pi G}{3} \rho a r
  \to
  \ddot{a} = - \frac{4 \pi G}{3} \rho a\,.
\end{equation}

This is part of a Friedmann equation: the isotropic pressure term cannot be recovered, it's like the speed of light is infinite now.

Now consider 
\begin{equation}
  \dot{\ell} \ddot{\ell} = - \frac{GM}{\ell^2} \dot{\ell}\,,
\end{equation}
%
therefore 
%
\begin{equation}
  \frac{1}{2}\dot{\ell}^2 = \frac{4 \pi}{3} G \rho \ell^{3} + \const\,,
\end{equation}
%
and integrating we get
%
\begin{equation}
  \dot{a}^2 r^2 = \frac{8 \pi G}{3} \rho a^2 r^2 + \const
\end{equation}
%
or, removing the \(r^2\) term, which is a constant,
%
\begin{equation}
    \dot{a}^2 = \frac{8 \pi G}{3} \rho a^2 + \const\,.
\end{equation}

We know that this new constant is related to the energy per particle. 

A universe with negative \(k\) expands forever and so on.

The number \(k\) was badly defined to be only \(\pm 1\) or 0: its newtonian version is much better represented by \(k_N = kc^2\).

\end{document}