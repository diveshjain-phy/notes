\documentclass[main.tex]{subfiles}
\begin{document}

\marginpar{Thursday\\ 2019-12-12, \\ compiled \\ \today}
% \section*{Thu Dec 12 2019}

% \todo[inline]{Check room availability for the first week of January.}

% Tomorrow / next week we will discuss some black holes and neutron stars. 

% Last week we discussed the nuclear processes which occur in the center of the Sun when \(T \sim \SI{e7}{K}\) is reached. 

% The minimum mass for a star in order to ignite fusion, as we found, is around \(\num{.08} M_{\odot}\). 

% The most important difference between the processes outlined last week is in the first equation of each: there are very few free neutrons.

The weak interaction chain has a very low power density: \(P_{\odot} / V _{\odot} \approx \SI{.27}{W / m^3}\), a lower power density than a human (who generally has a volume less than a cubic meter, but can still output a few hundreds of Watts).

For each \ce{^{4}He} nucleus we get an energy \(E = (4 m_p - m_{\ce{^{4}He}}) c^2 \approx\SI{25}{MeV}\).
Then, we can calculate the number of protons per second the Sun uses in order to produce the power it does. 
% We will use the following relation: 
% %
% \begin{align}
%   \SI{1}{MeV} \approx \SI{1.78e-30}{kg} \approx \SI{1.6e-13}{J}
% \,,
% \end{align}
% %
% and then, in SI units: we get 

The rate \(r\) of proton usage is given by
%
\begin{align}
r = \frac{4 L_{\odot}}{E} 
 \approx \SI{4e38}{Hz}
  % \frac{\num{4e26}}{\num{2.6e-13} / 4} \approx \num{4e38} \frac{\text{protons}}{\SI{}{s}}
\,,
\end{align}
%
which corresponds to about \SI{6.5e11}{kg /s}.\footnote{Which, to use a gruesome comparison, corresponds to roughly the mass of the entire human population every second.}
% This is a large amount in human terms, but it corresponds to about 

For each process one electron neutrino is also emitted, and the first two steps of the process happen twice for each \ce{^{4}He} nucleus, so around \num{2e38} neutrinos per second are produced. 
These travel basically undisturbed through the Sun and out.

A proton's mass is around \(m_p \approx \SI{1}{GeV} \approx \SI{1.78e-27}{kg}\), so in the Sun there are around \num{e56} protons: this means that the total lifetime of the Sun will be around \num{e10} years. 
The Sun is approximately half-way through this lifetime.

% What is the final state of the Sun? After hydrogen runs out, the Sun should start the next process: helium burning. The core contracts and heats. However we also have the degeneracy pressure. 

\subsection{Stellar evolution}

This section is massively simplified, stellar evolution is complicated, not completely understood and there can be many confounding variables. 
We will only try to give a general overview.

Throughout the evolution of the star, the interior is near equilibrium between gravitation and the pressure gradient due to the energy emitted by fusion. If the fusion starts to produce more energy, then the star expands and reaches a new equilibrium. 

The series of processes which can happen in stellar fusion is shown in table \label{tab:solar-processes}.
%
\begin{figure}[ht]
    \centering
    \begin{tabular}{ccccc}
     Process & Fuel & Products & \(T_{\text{min}}\) & \(M _{\text{min}}\) \\
     \hline
     Hydrogen burning & Hydrogen & Helium & \SI{e7}{K} & \(\num{.08} M_{\odot}\) \\
     Helium burning & Helium & Carbon, Oxygen & \SI{e8}{K} & \(\num{.5} M_{\odot}\) \\
     Carbon burning & Carbon & Oxygen, Neon, Sodium & \SI{5e8}{K} & \(8 M_{\odot}\) \\
     Neon burning & Neon & Magnesium, Oxygen & \SI{e9}{K} & \(9 M_{\odot}\) \\
     Oxygen burning & Oxygen & Magnesium to Sulphur & \SI{2e9}{K} & \(10 M_{\odot}\) \\
     Silicon burning & Silicon  & Iron and nearby elements & \SI{3e9}{K} & \(11 M_{\odot}\) 
    \end{tabular}
    \caption{Solar processes.}
    \label{tab:solar-processes}
\end{figure}

As one type of fusion fuel starts to run out, the pressure from the inside diminishes, therefore the interior of the star starts contracting, which typically allows the interior to start fusing the next kind of fuel, while the exterior keeps expanding. 

The required temperature to fuse every next element is ever higher, after every cycle there is the possibility of reaching the maximum density allowed by electron degeneracy pressure, which yields a mass threshold for each burning stage, denoted as \(M _{\text{min}}\) in the table.
% The one we derived earlier is for the first stage of nuclear burning.

For example, the Sun will reach the Helium burning phase, but it will go no further.

A star with a mass of less than \(11 M_{\odot}\) will not reach the iron stage; as it starts burning helium it becomes a red giant, and it only keeps growing as it goes through the burning phases. As it runs out of its last fuel, it starts shrinking and becomes either a white dwarf or a neutron star.

There is a boundary, the Chandrasekhar mass \(M_C \approx \num{1.4} M_{\odot}\), between the final fate of the star being a white dwarf or a neutron star: this is the maximum mass a star made of protons and electron can have if it must resist its own gravitational collapse only through electron degeneracy pressure. 

Now, let us consider massive stars with \(M > 11M _{\odot}\). They are able to burn oxygen into iron in their cores; now, the thing to note here is that the iron nuclide \ce{^{56}Fe} has one of the highest binding energy per nucleon,\footnote{It is not the highest, since there are a couple nuclides like \ce{^{62}Ni} which slightly exceed its binding energy per nucleon. These are collectively known as the ``iron group'', and are also formed, albeit in smaller amounts, during stellar fusion. For more details, see section IV in \textcite[]{fewellAtomicNuclideHighest1995}, a very clear (and not so technical) paper.} so once it is reached any successive burning stages would absorb energy, instead of releasing it.

This region in the core will not have any way to provide pressure to counteract the gravity of all the rest of the star above it. 
When the iron core reaches the Chandrasekhar mass, it collapses onto itself, and the rebound from this creates a supernova, which will either leave a neutron star or a black hole as a remnant.

In the supernova the conditions allow for the formation of heavier elements, beyond iron.
The matter which is expelled can form a \emph{planetary nebula}.

\todo[inline]{Maybe add more details here? Not sure whether it makes sense, maybe refer to the advanced astrophysics notes.}

% A plot of the binding energy per nucleon \(B = E - M _{\text{nucleons}}\) shows that it is maximum for iron. 

% Something about the possibility to have \ce{^{8}Be} be stable in order for potassium to be formed. 


% From the point of view of the state of matter, brown dwarves and white dwarves are very similar. 

% So the core becomes denser and hotter and starts burning helium. The external parts, by the virial theorem, must then expand. 

% So the possibilities, in order of mass, are white dwarf, neutron star, black hole. 


% When the nucleus cannot reach the temperature needed for the next process, the collapse is stopped by electron degeneracy. 

% The final radius of the red giant phase of the Sun is around \(70\) times the radius of the Sun, while the white dwarf phase is \(70\) times smaller. \todo[inline]{Is this correct? I couldn't quite hear.}

\subsection{The Hertzsprung-Russel diagram}

In the Herzsprung Russel diagram we plot \(L / L_{\odot}\) versus \(T _{\text{eff}}\), the latter increasing right to left. 

The Main Sequence runs from the upper left to the lower right, we have Red Giants on the upper right and White Dwarves on the lower left. 
Most of the stars are on the Main Sequence: the hydrogen burning phase lasts a long time. 

\todo[inline]{Pacciani here has a more in-depth discussion of HR diagrams, absolute magnitudes and so on: this would be a useful thing to insert. Also, a figure would be useful.}

\subsection{The interior of a Main Sequence star}

We wish to describe the statics of the fluid which makes up a Main Sequence star; the goal we set is to calculate what is the maximum mass of a Main Sequence star --- the Main Sequence includes all the stars which are in the process of burning hydrogen. 

\paragraph{The density profile}

The first thing we will need is to calculate the density profile of the star, which is written as \(\rho (r)\) since as always we are assuming spherical symmetry. 

Our equation of hydrostatic equilibrium can be written as:
%
\begin{align}
  \dv{P}{r} &= - \frac{G m(r) \rho (r)}{r^2}  \\
  \frac{r^2}{\rho (r)} \dv{P}{r} &= - G m(r)
\,,
\end{align}
%
and we can relate the differential mass of a spherical shell with the differential radius through \(\dd{m} = 4 \pi r^2 \rho (r) \dd{r}\).

Differentiating the equation of hydrostatic equilibrium we find
% which can be restated as 
%
\begin{align}
  \dv{}{r} \qty(\frac{r^2}{\rho (r)} \dv{P}{r}) = - G \dv{m}{r} = - 4 \pi G \rho (r) r^2
\,,
\end{align}
%
more commonly stated as 
%
\begin{align}
  \frac{1}{r^2} \dv{}{r} \qty(\frac{r^2}{\rho (r)} \dv{P}{r}) = -4 \pi G \rho (r)
\,.
\end{align}
%
% which holds if we have hydrostatic equilibrium. 

\todo[inline]{This resembles the Laplacian in spherical coordinates, \(\nabla^2 f(r) = r^{-2} \partial_{r} \qty(r^2 \partial_{r} f)\)\dots can this be stated more precisely?}

We can solve this using an equation of state which gives us \(P = P (\rho )\); for stellar interiors the cosmological equations of state \(P \propto \rho \) do not in general work well, and instead we must generalize to a \emph{polytropic} equation:
% Commonly the equation of state used for this is called \emph{polytropic}: 
%
\begin{align}
  P = k \rho^{\frac{n+1}{n}} = k \rho^{\gamma }
\,,
\end{align}
%
where \(k\) and \(n\) are constants; also \(n = 1/ (\gamma -1)\).
This \(\gamma \) is the adiabatic index: for a monoatomic nonrelativistic gas \(\gamma = 5/3\) and \(n = 3/2\), for an ultrarelativistic gas \(\gamma = 4/3\) and \(n = 3\).
% If \(\gamma = 5/3\), which holds for a monoatomic gas, then \(n = 3/2\) (\(P \sim \rho^{5/3}\)); if \(\gamma = 4/3\), which holds for an ultrarelativistic gas, then \(n = 3\). 

% Using this law, we can write this equation in terms of either only the density or only the pressure. 
This will yield a second order differential equation for \(\rho \), which we must complement with two boundary conditions. 
We can set the value of the central density, \(\rho (r = 0) = \rho _c\); also, in order for the density to be a differentiable function of position inside the stare we must also have \(\partial_{r}\rho (r=0) = 0 \): this is because if we move in a straight line through the center of the star, as we pass \(r = 0\) we move from a certain value of the derivative to minus that value, since \(r\) goes from decreasing to increasing. The only way for this to be continuous is if the derivative is zero. 

% Let us kave only the density: 
This is confirmed by the fact that, if we substitute the polytropic equation of state, we find
%
% \begin{align}
%   \frac{1}{r^2} \dv{}{r} \qty(\frac{r^2}{\rho (r)} \dv{}{r} \qty(\rho^{\frac{n+1}{n}})) = - 4 \pi G \rho (r)
% \,,
% \end{align}
%
% and we need two boundary conditions: we can fix the central density \(\rho(r=0) = \rho_c\) and the derivative of the density at the center: 
% %
% \begin{align}
%   \dv{\rho_c}{r} (r=0) = 0
% \,.
% \end{align}
% This is because: 
%
\begin{align}
  \rho^{1/n} \dv{\rho }{r} \propto - \frac{Gm(r)}{r^2} \rho (r)
\,,
\end{align}
%
and the mass in a small region around the origin is approximately \(m(r) \sim \rho_c r^3\): therefore \(\rho^{1/n} \dv*{\rho }{r} \propto \rho _c r\). 

These equations can be solved numerically.
The radius of the star can be calculated as the one at which the density goes to 0: \(\rho (R) =0\), and the mass of the star is given by \(m(R) = M\). 

This model is unphysical in that it assumes that the star's interior can be described by a constant \(\gamma \) throughout; as we have seen there are several orders of magnitude of difference in pressure, temperature and density from the core to the surface, so it is a strong assumption to say that it behaves in the same way.

% Also, we can use an ansatz. 
\paragraph{The Clayton model}

Let us discuss a model proposed by Clayton in 1986, which uses an ansatz for the density profile in order to extract information about the star.
% \(P_c = \SI{2e16}{Pa}\), and since 

% %
% \begin{align}
%   \expval{P} = - \frac{1}{3} \frac{E _{\text{gr}}}{V_{\odot}}
% \,,
% \end{align}
% %
% where \(E _{\text{gr}} = G M_{\odot}^2 / R_{\odot}\), and \(V_{\odot} = \frac{4}{3} \pi R_{\odot}^3\): therefore the mean pressure is approximately \(1/200\) times the pressure at the center. 

Near the center of the star, we can estimate the mass contained within a spherical shell by assuming \(\rho = \rho _c\) throughout, so \(m(r) \sim \frac{4 \pi }{3} \rho _c r^3\), which we can substitute into the equation of hydrostatic equilibrium:
% We can write the equation of hydrostatic equilibrium as 
%
\begin{align}
  \dv{P}{r} = - \frac{Gm \rho }{r^2} \approx -\frac{4 \pi G}{3} \rho^2_c r 
\,,
\end{align}
%
so the pressure gradient goes to zero linearly in \(r\). 

% So, we know that 
% %
% \begin{align}
%   \dv{P}{r} (r=0) = 0
% \,,
% \end{align}
% %

Also, as \(r\) approaches the radius of the star, \(R\), we get \(\dv*{P}{r} \rightarrow 0\) as well, since the pressure gradient is proportional to \(\rho (r)\) in that region, while \(r\) approaches a constant. 

So, the pressure gradient approaches zero both at the core and at the surface, while in the interior it has a negative value.
% The pressure gradient around the center follows the density. Its variation is much larger near the center than on the outside. 

The ansatz by Clayton is a relatively simple expression which achieves these requirements:
%
\begin{align}
  \dv{P}{r} = -\frac{4 \pi }{3} G \rho _c^2 r \exp(- \frac{r^2}{a^2})
\,,
\end{align}
%
where the parameter \(a\) has the dimensions of a length, and we take it to be \(a \ll R \).
This model is quite accurate near the center, not so much near the surface! 

By integrating we can calculate the pressure profile:
%
\begin{align}
  P(r) = \frac{2 \pi }{3} G \rho_c^2 a^2 \qty(\exp(- \frac{r^2}{a^2}) - \exp(- \frac{R^2}{a^2}))
\,,
\end{align}
%
so that the pressure is exactly zero at the surface: \(P(R) = 0\).
% We will check \emph{a posteriori} that this model makes sense for a typical star.
% \todo[inline]{Do we set the pressure to stay at zero for \(r>R\)? Maybe does not matter\dots}

% We have the relation 
Substituting the relation \(\dd{m} = 4 \pi r^2\rho \dd{r}\) into the hydrostatic equilibrium equation we find
%
\begin{align}
  G m(r) \dd{m} = - 4 \pi r^{4} \dd{P}
\,,
\end{align}
%
which can be integrated in order to calculate the mass from the pressure profile: 
%
\begin{align}
  \frac{1}{2} G m^2(r) &= - 4 \pi \int_{0}^{r} \dd{\widetilde{r}} \widetilde{r}^{4} \dv{P}{\widetilde{r}} \\
  m^2(r) &= - \frac{8 \pi }{G} \qty(- \frac{4 \pi }{3}) G \rho_c^2 \int_{0}^{r} \dd{\widetilde{r}} \widetilde{r}^{5} \exp(- \frac{r^2}{a^2})\\
  m(r) &=  \frac{4 \pi a^3}{3} \rho_c \Phi (x)
\,,
\end{align}
% We recover \(m(r)\) by taking the square root: 
% %
% \begin{align}
%   m(r) =  \frac{4 \pi a^3}{3} \rho_c \Phi (x)
% \,,
% \end{align}
%
where we performed a change of variable to \(x = r/a\) (bringing out \(a^{6}\)) and defined \(\Phi (x)\) as the square root of the dimensionless integral: 
%
\begin{align}
  \Phi^2 (x) = 6 \int_{0}^{x} \dd{y} y^{5} e^{-y^2} 
  = 6 - 3 \qty(x^4 +2 x^2+2) e^{-x^2}
\,.
\end{align}

% and the second term is almost zero near the surface if \(a \ll R\): the power series starts from the sixth power (? what?). 
Near the surface \(x\) is very large (since, as we mentioned, \(a \ll R\)), so the exponential dominates: we find \(\Phi^2(x) \approx 6\) there.

The density profile can also be expressed in terms of \(\Phi (x)\):
%
\begin{align}
  \rho (r) &= \frac{1}{4 \pi r^2} \dv{m}{r}
  = \frac{1}{4 \pi a^2 x^2} \frac{1}{a} \frac{4 \pi a^3\rho _c}{3} \dv{\Phi }{x}  \\
  &=\frac{\rho _c}{3 x^2} \frac{1}{2 \Phi} \dv{}{x} \qty( 6\int_0^{x}  y^{5} \exp(-y^2) \dd{y})
  \\
  &= \rho_{c} \qty(\frac{x^3 e^{-x^2}}{\Phi (x)})
\,,
\end{align}
%
which, under the assumptions of the model, is a complete description of the density profile.
We can also recover the temperature profile from the ideal gas law (which holds as long as the gas inside the star is nonrelativistic): 
%
\begin{align}
  T(r) = \frac{\overline{m}}{k_B} \frac{P(r)}{\rho (r)}
\,.
\end{align}

% and, when \(x \ll 1\) we find 
Let us calculate this near the center of the star, meaning \(x \ll 1\); 
%
\begin{align}
  \Phi (x) &= \qty[6 - 3 (x^{4} + 2 x^2+ 2) \qty(1 - x^2  +\frac{x^{4}}{2} - \frac{x^{6}}{6})] \\
  &\approx \qty(x^{6} - \frac{3}{4} x^{8} + \frac{3}{10} x^{10} - \frac{1}{12} x^{12}+ \dots)^{1/2}
\,,
\end{align}
%
% and inserting this we find 
which we can insert into our expression for \(\rho (r)\), also let us expand \(P(r)\) to second order, so that we can also calculate \(T(r)\):
%
\begin{align}
  \rho (r) &\approx \rho _c \qty(1 - \frac{5}{8} \frac{r^2}{a^2} + \dots) \\
  P(r) &\approx \frac{2 \pi }{3} G \rho _c^2 a^2  \qty(1 - \frac{r^2}{a^2} + \dots) \\
  T(r) &\approx T_c \qty( 1 - \frac{3}{8} \frac{r^2}{a^2} + \dots)
\,,
\end{align}
%
where 
%
\begin{align}
T_c = \frac{\overline{m}}{k_B} \frac{2 \pi }{3} G \rho _c a^2
\,.
\end{align}

Moving to the surface, we can calculate the total mass 
%
\begin{align}
  M = m(R) = \frac{4 \pi \rho_c a^3}{3} \Phi (R/a) \approx
  \frac{4 \pi \rho _c a^3 \sqrt{6}}{3} 
\,.
\end{align}

Then, the average density is given by 
%
\begin{align}
\expval{\rho } = \frac{M}{\frac{4 \pi }{3} R^3} = \sqrt{6}  \qty(\frac{a}{R})^3 \rho _c
\,,
\end{align}
%
so if it is the case that \(a \ll R\) then we also have \(\rho _c \gg \expval{\rho }\). 

We can invert this relation to find \(a\) in terms of \(M\) and \(\rho _c\); for the Sun we find \(a \approx R_{\odot} / 5.4\), \(\rho (a) = 0.53 \rho _c\) and \(m(a) = \num{.28} M_{\odot}\). 

This means that, as we expected, the Sun is quite concentrated: over a quarter of its mass is contained within \((1/5.4)^3 \approx \SI{0.6}{\percent}\) of its volume.

% and lots of it are within a radius \(a\): 
% %
% \begin{align}
  
%   \qquad
%   \text{and}
%   \qquad
  
% \,.
% \end{align}
% %

% We also have the relation , and 
%
% \begin{align}
%   \frac{\expval{\rho }}{\rho _c } \sim \qty(\frac{a}{R})^3 
%   \sim 1
% \,.
% \end{align}
% %
% \todo[inline]{Where does this come from?}

% We have the relations 
% %
% \begin{align}
%   m(r) = \frac{4 \pi a^3}{3} \rho _c \Phi (x)
% \,,
% \end{align}
% %
% and 
% %
% \begin{align}
%   P_c = \frac{2 \pi }{3} G \rho_c^2 a^2
% \,.
% \end{align}

% Then we can combine these: 
A useful result we can derive from this model is the central pressure expressed in terms of the mass and central density:
%
\begin{align}
  P_c &= \frac{2 \pi }{3} G \rho _c^2a^2 = \frac{2 \pi }{3} G \rho _c^2 \qty(\frac{3M}{4 \pi \rho _c \sqrt{6}})^{2/3} \\
  &\approx \qty(\frac{\pi }{36})^{1/3} G M^{2/3} \rho _c^{4/3}
\,.
\end{align}

This model then predicts the prefactor \(q = (\pi / 36)^{1/3} \approx \num{.44}\). The powers of \(M\) and \(\rho _c\) are the same in more sophisticated models.

From simulations with varying \(\gamma \)  we get: for \(\gamma = 5/3 \) the factor is \(q \approx \num{.48}\), for \(\gamma = 4/3\) the factor is \(q \approx \num{.36}\). 
The results are quite close to ours!

\todo[inline]{Pacciani also states that \(q < (\pi / 6)^{1/3} \approx \num{.14}\)\dots not sure how that would make sense!}

\subsection{The maximum mass}

We can apply the ideal gas relation to the core of the star: then we find 
%
\begin{align}
  k_B T_c 
  = \overline{m} \frac{P_c}{\rho _c}
  \approx  \qty(\frac{\pi }{36 })^{1/3} G \overline{m} M^{1/3} \rho _c^{1/3}
\,.
\end{align}

% Then, we can figure out which processes actually can happen. 

\end{document}
