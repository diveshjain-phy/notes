\documentclass[main.tex]{subfiles}
\begin{document}

\marginpar{Thursday\\ 2019-12-12, \\ compiled \\ \today}
% \section*{Thu Dec 12 2019}

\todo[inline]{Check room availability for the first week of January.}

Tomorrow / next week we will discuss some black holes and neutron stars. 

Last week we discussed the nuclear processes which occur in the center of the Sun when \(T \sim \SI{e7}{K}\) is reached. 

The minimum mass for a star in order to ignite fusion, as we found, is around \(\num{.08} M_{\odot}\). 

The most important difference between the processes outlined last week is in the first equation of each: there are very few free neutrons.

So, the weak interaction process dominates: since it is so slow, it has a very low power density: \(P = \SI{4e26}{W}\), but we need to take its ratio to the volume of the Sun. This gives a density lower than that of a human. 

For each \ce{^{4}He} nucleus we get \SI{26}{MeV}, and we need 4 protons to make it. Then, this gives us the number of protons per second the Sun uses in order to produce the power it does. 
We will use the following relation: 
%
\begin{align}
  \SI{1}{MeV} \approx \SI{1.78e-30}{kg} \approx \SI{1.6e-13}{J}
\,,
\end{align}
%
and then, in SI units: we get 
%
\begin{align}
  \frac{\num{4e26}}{\num{2.6e-13} / 4} \approx \num{4e38} \frac{\text{protons}}{\SI{}{s}}
\,.
\end{align}

For each process we also emit one electron neutrino, and we need to do the first two steps of the process twice for each \ce{^{4}He} nucleus, so we are producing \num{2e38} neutrinos per second. 

A proton's mass is around \(m_p \approx \SI{1}{GeV} \approx \SI{1.78e-27}{kg}\), so in the Sun there are around \num{e56} of them: this means that the typical lifetime of the Sun is around \num{e10} years. 

What is the final state of the Sun? After hydrogen runs out, the Sun should start the next process: helium burning. The core contracts and heats. However we also have the degeneracy pressure. 

There is a boundary, the Chandrasekar mass \(M_C \approx \num{1.4} M_{\odot}\), between the final fate of the star being a white dwarf or a neutron star. 

From the point of view of the state of matter, brown dwarves and white dwarves are very similar. 

So the core becomes denser and hotter and starts burning helium. The external parts, by the virial theorem, must then expand. 

So the possibilities, in order of mass, are white dwarf, neutron star, black hole. 

The matter which is expelled can form a \emph{planetary nebula}. 


\begin{figure}[ht]
    \centering
    \begin{tabular}{ccccc}
     Process & Fuel & Products & \(T_{\text{min}}\) & \(M _{\text{min}}\) \\
     \hline
     Hydrogen burning & Hydrogen & Helium & \SI{e7}{K} & \(\num{.08} M_{\odot}\) \\
     Helium burning & Helium & Carbon, Oxygen & \SI{e8}{K} & \(\num{.5} M_{\odot}\) \\
     Carbon burning & Carbon & Oxygen, Neon, Sodium & \SI{5e8}{K} & \(8 M_{\odot}\) \\
     Neon burning & Neon & Magnesium, Oxygen & \SI{e9}{K} & \(9 M_{\odot}\) \\
     Oxygen burning & Oxygen & Magnesium to Sulphur & \SI{2e9}{K} & \(10 M_{\odot}\) \\
     Silicon burning & Silicon  & Iron and nearby elements & \SI{3e9}{K} & \(11 M_{\odot}\) 
    \end{tabular}
    \caption{Solar processes.}
    \label{tab:solar-processes}
\end{figure}    

A plot of the binding energy per nucleon \(B = E - M _{\text{nucleons}}\) shows that it is maximum for iron. 

Something about the possibility to have \ce{^{8}Be} be stable in order for potassium to be formed. 

When the nucleus cannot reach the temperature needed for the next process, the collapse is stopped by electron degeneracy. 

The final radius of the red giant phase of the Sun is around \(70\) times the radius of the Sun, while the white dwarf phase is \(70\) times smaller. \todo[inline]{Is this correct? I couldn't quite hear.}

Our equation of hydrostatic equilibrium is:
%
\begin{align}
  \dv{P}{r} = - \frac{G m(r) \rho (r)}{r^2}
\,,
\end{align}
%
and the equation giving the variation of the mass is \(\dv{m}{r} = 4 \pi r^2 \rho (r)\).
So we get 
%
\begin{align}
  \frac{r^2}{\rho (r)} \dv{P}{r} = - G m(r)
\,,
\end{align}
%
which can be restated as 
%
\begin{align}
  \dv{}{r} \qty(\frac{r^2}{\rho (r)} \dv{P}{r}) = - G \dv{m}{r} = - 4 \pi G \rho (r) r^2
\,,
\end{align}
%
more commonly stated as 
%
\begin{align}
  \frac{1}{r^2} \dv{}{r} \qty(\frac{r^2}{\rho (r)} \dv{P}{r}) = -4 \pi G \rho (r)
\,,
\end{align}
%
which holds if we have hydrostatic equilibrium. 

Commonly the equation of state used for this is called \emph{polytropic}: 
%
\begin{align}
  P = k \rho^{\frac{n+1}{n}}
\,,
\end{align}
%
where \(k = \const\) and \(n = 1/ (\gamma -1)\): so if \(\gamma = 5/3\), which holds for a monoatomic gas, then \(n = 3/2\) while if \(\gamma = 4/3\), which holds for an ultrarelativistic gas, then \(n = 3\). 

Using this law, we can write this equation in terms of either only the density or only the pressure. 

Let us kave only the density: 
%
\begin{align}
  \frac{1}{r^2} \dv{}{r} \qty(\frac{r^2}{\rho (r)} \dv{}{r} \qty(\rho^{\frac{n+1}{n}})) = - 4 \pi G \rho (r)
\,,
\end{align}
%
and we need two boundary conditions: we can fix the central density \(\rho(r=0) = \rho_c\) and the derivative of the density at the center: 
%
\begin{align}
  \dv{\rho_c}{r} (r=0) = 0
\,.
\end{align}

This is because: 
%
\begin{align}
  \rho^{1/n} \dv{\rho }{r} \propto - \frac{Gm(r)}{r^2} \rho (r)
\,,
\end{align}
%
and the mass is given by \(m(r) \sim \rho_c r^3\): therefore \(\rho^{1/n} \dv*{\rho }{r} \propto r\), so it makes sense to set the derivative to zero. 

The radius of the star is defined as the one at which the density goes to 0: \(\rho (R) =0\), and the mass of the star is given by \(m(R) = M\). 

These equations can be solved numerically. Also, we can use an ansatz. 

A model by Claytron 1986: \(P_c = \SI{2e16}{Pa}\), and since 
%
\begin{align}
  \expval{P} = - \frac{1}{3} \frac{E _{\text{gr}}}{V_{\odot}}
\,,
\end{align}
%
where \(E _{\text{gr}} = G M_{\odot}^2 / R_{\odot}\), and \(V_{\odot} = \frac[i]{4}{3} \pi R_{\odot}^3\): therefore the mean pressure is approximately \(1/200\) times the pressure at the center. 

We can write the equation of hydrostatic equilibrium as 
%
\begin{align}
  \dv{P}{r} = - \frac{Gm \rho }{r^2} \approx -\frac{4 \pi G}{3} \rho^2_c r 
\,,
\end{align}
%
so the pressure gradient goes to zero linearly in \(r\). 
So, we know that 
%
\begin{align}
  \dv{P}{r} (r=0) = 0
\,,
\end{align}
%
also, as \(r \rightarrow R\) we get \(\dv*{P}{r} \rightarrow 0\) as well, since it is proportional to \(\rho (r)\) in that region. 

The pressure gradient around the center follows the density. Its variation is much larger near the center than on the outside. 
The ansatz by Clayton is 
%
\begin{align}
  \dv{P}{r} = -\frac{4 \pi }{3} G \rho _c^2 r \exp(- \frac{r^2}{a^2})
\,,
\end{align}
%
where the parameter \(a\) has the dimensions of a length, and we take it to be \(a \ll R \). This model is quite accurate near the center, not so much near the surface! 

Integrating we find: 
%
\begin{align}
  P(r) = \frac{2 \pi }{3} G \rho_c^2 a^2 \qty(\exp(- \frac{r^2}{a^2}) - \exp(- \frac{R^2}{a^2}))
\,,
\end{align}
%
so that the pressure is exactly zero at the surface: \(P(R) = 0\).
We will check \emph{a posteriori} that this model makes sense for a typical star.
\todo[inline]{Do we set the pressure to stay at zero for \(r>R\)? Maybe does not matter\dots}

We have the relation 
%
\begin{align}
  G m(r) \dd{m} = - 4 \pi r^{4} \dd{P}
\,,
\end{align}
%
which can be integrated in order to give the following result: 
%
\begin{align}
  \frac{1}{2} G m^2(r) = - 4 \pi \int_{0}^{r} \dd{\widetilde{r}} \widetilde{r}^{4} \dv{P}{\widetilde{r}}
\,.
\end{align}

We recover \(m(r)\) by taking the square root: 
%
\begin{align}
  m(r) =  \frac{4 \pi a^3}{3} \rho_c \Phi (x)
\,,
\end{align}
%
where \(x = r/a\) and \(\Phi (x)\) is defined from the integral from before: 
%
\begin{align}
  \Phi^2 (x) = 6 \int_{0}^{x} \dd{y} y^{5} e^{-y^2} 
  = 6 - 3 \qty(x^4 +2 x^2+2) e^{-x^2}
\,,
\end{align}
%
and the second term is almost zero near the surface if \(a \ll R\): the power series starts from the sixth power (? what?). 

The density profile then is given by:
%
\begin{align}
  \rho (r) = \frac{1}{4 \pi r^2} \dv{m}{r}
  = \rho_{c} \qty(\frac{x^3 e^{-x}}{\Phi (x)})
\,,
\end{align}
%
and we can also recover the temperature profile from the ideal gas law: 
%
\begin{align}
  T(r) = \frac{\overline{m}}{k_B} \frac{P(r)}{\rho (r)}
\,,
\end{align}
%
and, when \(x \ll 1\) we find 
%
\begin{align}
  \Phi (x) \sim \qty(x^{6} - \frac{3}{4} x^{8} + \frac{3}{10} x^{10} - \frac{1}{12} x^{12}+ \dots)^{1/2}
\,,
\end{align}
%
and inserting this we find 
%
\begin{align}
  \rho (r) \approx \rho _c \qty(1 - \frac{5}{8} \frac{\overline{r}^2}{a^2} + \dots)
\,,
\end{align}
%
and 
%
\begin{align}
  T(r) \approx T_c \qty( 1 - \frac{3}{8} \frac{r^2}{a^2} + \dots)
\,,
\end{align}
%
which allows us to get the total mass 
%
\begin{align}
  M = m(R) = \frac{4 \pi \rho_c a^3}{3} \Phi (R/a) \approx
  \frac{4 \pi \rho _c a^3 \sqrt{6}}{3} 
\,,
\end{align}
%
and lots of it are within a radius \(a\): 
%
\begin{align}
  \rho (a) = 0.53 \rho _c
  \qquad
  \text{and}
  \qquad
  m(a) = \num{.28} M_{\odot}
\,.
\end{align}
%

We also have the relation \(a = R_{\odot} / 5.4\), and 
%
\begin{align}
  \frac{\expval{\rho }}{\rho _c } \sim \qty(\frac{a}{R})^3 
  \sim 1
\,.
\end{align}
%
\todo[inline]{Where does this come from?}

We have the relations 
%
\begin{align}
  m(r) = \frac{4 \pi a^3}{3} \rho _c \Phi (x)
\,,
\end{align}
%
and 
%
\begin{align}
  P_c = \frac{2 \pi }{3} G \rho_c^2 a^2
\,.
\end{align}

Then we can combine these: 
%
\begin{align}
  P_c = \qty(\frac{\pi }{36})^{1/3} G M^{2/3} \rho _c^{4/3}
\,.
\end{align}

The factor \((\pi / 36)^{1/3} \sim \num{.44}\). 
Changing \(\gamma \) we get: for \(\gamma = 5/4 \) the factor is \num{.48}.

We use the ideal gas relation: then we find 
%
\begin{align}
  k_B T_c = \qty(\frac{\pi }{36 })^{1/3} G \overline{m} M^{1/3} \rho _c^{1/3}
\,.
\end{align}

Then, we can figure out which processes actually can happen. 

\end{document}
