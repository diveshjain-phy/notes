\documentclass[main.tex]{subfiles}
\begin{document}

% \section*{Fri Oct 25 2019}
% \marginpar{Friday \\ 2019-10-25, \\ compiled \\ \today}

% Do we wish to have a part on stellar astrophysics right now, before going on with cosmology? Let him know.


% QFT must be dealt with not only at zero temperature, but also at finite temperatures.
% People started doing this in the seventies.

% The time variable is then periodic, with period \(2 \pi \beta \), where \(\beta  = 1/(k_B T)\).

% \todo[inline]{Is this connected to imaginary time?}

% This is ``in-in'' instead of ``in-out'': there are no equilibrium states before or after the interaction.

% We will simplify: we assume thermal equilibrium at any time in the evolution.
% A key point in cosmologi is the \emph{absence} of time translation invariance, therefore energy is not conserved.

% This part will come from Weinberg's book.

% We use units where \(c = k_B = \hbar = 1\). The number density of a certain species is 
%
% \begin{equation}
  % n(\dots ) = \frac{g}{(2\pi )^{3}} \int  \dd[3]{q} f(\vec{q})
% \,,
% \end{equation}
%
% where \(f(\vec{q})\) is the phase space distribution of the particles, while \(g\) is the number of helicity states of that species.
% The dots will be explained later: they are about the parametrization of the phase space distribution.

% For any species, \(E = \sqrt{m^2 + \abs{q}^2} \): this applies \emph{on shell}, for the classical equations of motion: when there are quantum fluctuations it does not hold.
% The energy density is 
%
% \begin{equation}
  % \rho (\dots) = \frac{g}{(2\pi )^3} \int  \dd[3]{q} E(\vec{q}) f(\vec{q}) 
% \,,
% \end{equation}
%
% while the pressure is 
%
% \begin{equation}
  % P (\dots) = \frac{g}{(2\pi )^3} \int  \dd[3]{q} f(\vec{q}) \frac{q^2}{3E}
% \,.
% \end{equation}
% 
% Because of isotropy, in the cases we need to consider, the dependence on \(\vec{q}\) is actually a dependence on \(\abs{\vec{q}} \equiv q \).

% What is our ansatz for the phase space distribution?  
%
% \begin{equation}
  % f(q) = \frac{1}{\exp(\frac{E-\mu }{T}) \mp 1 }
% \,,
% \end{equation}
%
% where the sign is \(-\) for bosons, \(+\) for fermions.

% The chemical potential deals with the flux of particles.
% We recover the Planck distribution when \(E = q\) and \(\mu = 0\): this tells us that photons do not have any chemical potential.
% If we measured \(\mu \neq 0 \) that would be called a \emph{spectral distortion}, but is seems like that is not the case.

% The rule for the sum of the chemical potentials only holds at equilibrium.

We can relate some chemical potentials by reactions, but not all of them: our system of equations will be degenerate, with degeneracy corresponding precisely to the globally conserved quantities (electric charge, lepton number, baryon number) which follow from the symmetry group of our theory. 
These can have any value and are conserved in any reaction,\footnote{This holds as long as the temperature is low enough: we are considering the reactions which are allowed by the Standard Model of interactions, with its symmetry group \(SU(3)_c \times SU(2)_L \times U(1)_Y\), but it is not currently known whether at higher temperatures (i.\ e.\ earlier times) this is the most general symmetry group which is spontaneously broken to the SM group. So, the statements we make only apply at relatively late times.} so they cannot be fixed by the system.

If there was a global electric charge, we'd expect global magnetic fields, but we only see them with magnitudes of the order of the \SI{1}{nT}, which gives an upper bound on the global charge of the universe.
So, any global electric charge would be quite small --- we will assume it is exactly zero. 

We can estimate the orders of magnitude for the abundances of the various particle species in the universe.
The baryon number is very small when compared to the number of photons in the universe, roughly speaking \(n_\gamma / n_b \sim \num{e10}\). 

The lepton number is harder to estimate, but it is reasonable to assume that it is quite small as well.
For slightly more detailed discussion, see the book by Weinberg \cite[before eq. 15.6.5]{weinbergGravitationCosmologyPrinciples1972}.

In the end, we can say that in the early universe \(\mu / T \ll 1\), so we can assume \(\mu \approx 0\).
This is just a reasonable simplification, which we make in order to get analytic results. 

Under this assumption the quantities characterizing the matter distribution in the universe only depend on the temperature: so, we will just write \(n(T)\), \(\rho (T)\) and \(P(T)\).

% \todo[inline]{There is a whole section before this in \cite[]{paccianiAppuntiCorsoPhysical2018}\dots}

In general, when dealing with thermodynamic problems in an expanding spacetime there is a complication: in Minkowski spacetime we have symmetry under time translations, thus it makes sense to talk about stationarity.
In an expanding universe, instead, we have no Killing vector with respect to time.
There is a competition between two evolutions, the thermodynamic evolution of the system and the expansion of the universe: we cannot truly have equilibrium!

The way to deal with this problem is: we assume that the first evolution is much faster than the other, that is, we reach thermal equilibrium on timescales that are short if compared to the expansion.
This way, we can neglect the expansion of the universe while our system reaches equilibrium.

So our problem is oversimplified: we assume thermodynamic equilibrium, which makes sense in certain periods of the life of the universe, and that allows us to embed a thermal situation into a universe which evolves in time.

% A proper treatment of this issue will include a term corresponding to the expansion in the equations of motion. 

\subsection{Entropy}

From the second principle of thermodynamics we know that the entropy in a certain volume \(V\) at temperature \(T\), denoted \(S(V, T)\) is given by: 
%
\begin{equation} \label{eq:entropy-differential-definition}
  \dd{S}  = \frac{1}{T} \qty(\underbrace{\dd{\qty(\rho(T) V)}}_{ \dd{E}}  + P(T) \dd{V})
  = \frac{1}{T} \qty(V \dd{\rho (T)} + (P(T) + \rho(T) ) \dd{V})
\,,
\end{equation}
%
since in order to get the total energy we must multiply the constant energy by the volume: \(E = \rho (T) V\). 
 
Then we can read off the partial derivatives of the entropy:
%
\begin{equation}
  \pdv{S}{V} = \frac{1}{T} \qty(\rho (T) + P(T))
  \qquad \text{and} \qquad
  \pdv{S}{T} = \frac{V}{T} \dv{\rho (T)}{T} 
\,.
\end{equation}

In order for the differential to be exact it needs to be closed, which means that the second partial derivatives need to commute (these are known as the \emph{Pfaff relations}):\footnote{They only hold in a simply connected space.}
%
\begin{align}
  \pdv[2]{S}{T}{V} &= \pdv[2]{S}{V}{T} \\ 
  \pdv{}{T}  \qty(\frac{1}{T}\qty(\rho (T) + P(T))) 
  &= \pdv{}{V} \qty(\frac{V}{T} \dv{\rho (T)}{T} )   \\
  - \frac{1}{T^2} \qty(\rho + P)
  + \frac{1}{T} \qty(\dv{\rho }{T} + \dv{P}{T})
  &= \frac{1}{T} \dv{\rho }{T}   \\
  \dv{P}{T} &= \frac{1}{T} \qty(\rho +P) 
  \marginnote{Simplified \(T^{-1}(\dv*{\rho }{T})\), multiplied by \(T\) and brought \(T^{-1} (\rho +P)\) to the other side}[-1cm]
  \label{eq:rewritten-pfaff-relations-entropy}
\,.
\end{align}

Cosmology has not entered into the picture yet, but it can by the third Friedmann equation, which can be rewritten as 
%
\begin{align}
\dot{\rho} &= - 3 \frac{\dot{a}}{a} (\rho + P)  \\
0 &= 3 \dot{a} a^2 (\rho + P) + a^3 \dot{\rho} 
\marginnote{Multiplied by \(- a^3\).}  \\
a^3 \dot{P} &= 3 \dot{a} a^2 (\rho + P) + a^3 \dot{\rho} + a^3 \dot{P} 
\marginnote{Added \(a^3 \dot{P}\) on both sides.}  \\
a^3 \dot{P} &= \dv{(a^3)}{t} (\rho + P) + a^3 \dv{(\rho + P)}{t}  \\
a^3 \dot{P} &= \dv{}{t} \qty(a^3 (\rho + P))
\,,
\end{align}
%
% %
% \begin{equation}
%   a^3 \dv{P}{t} =\dv{}{t} \qty(a^3 (P + \rho ))
% \,,
% \end{equation}
%
and these two, when put together, are equivalent to 
%
\begin{equation}
  \dv{}{t} \qty(\frac{a^3}{T} (\rho (T) + P (T) ))=0
\,,
\end{equation}
%
therefore this quantity is a constant of motion. 
Let us verify this statement: expanding the derivative we get 
%
\begin{align}
\dv{}{t} \qty(\frac{a^3}{T} (\rho + P )) 
&= \frac{1}{T} \dv{}{t} \qty(a^3 (\rho + P)) + a^3 (\rho + P) \dv{}{t} \qty( \frac{1}{T})  \\
&= \frac{1}{T} a^3 \dot{P} - a^3 (\rho + P) \frac{\dot{T}}{T^2}  \\
&= \frac{a^3}{T} \dv{P}{T} \dot{T} - a^3 (\rho + P) \frac{\dot{T}}{T}  \\
&= \frac{a^3}{T} \frac{(\rho + P)}{T} \dot{T} - a^3 (\rho + P) \frac{\dot{T}}{T} = 0
\marginnote{Used equation \eqref{eq:rewritten-pfaff-relations-entropy}.}
\,.
\end{align}

For the RW line element, the square root of the determinant is given by \(\sqrt{-g} =  a^3 \), so the conserved quantity can be written as
%
\begin{equation}
  \dv{}{t} \qty(\sqrt{-g }  \frac{\rho + P}{T}) = 0
\,.
\end{equation}

This is relevant because the volume of any given spatial region scales with \(\sqrt{-g}\) as the universe expands. 

So the quantity which is differentiated is constant. If we plug this back into the differential expression for the entropy, we get:\footnote{The procedure to prove this result is as follows: the expression we want to show is equal to \(\dd{S}\) can be written like 
%
\begin{align}
\dd{S} \overset{?}{=} \dd{\qty(\frac{(\rho + P)V}{T})} = \qty(- \frac{(\rho + P) V}{T^2} + \frac{V}{T} \qty(\dv{\rho }{T} + \dv{P}{T})) \dd{T} + \frac{\rho + P}{T} \dd{V} 
\,,
\end{align}
%
while the definition of \(\dd{S}\) \eqref{eq:entropy-differential-definition} can be written as 
%
\begin{align}
\dd{S} = \frac{V}{T} \dv{\rho }{T} \dd{T} + \frac{\rho + P}{T} \dd{V}
\,,
\end{align}
%
so we can see that, since the term proportional to \(\dd{V}\) is the same in both cases, we only need to show that the coefficients of \(\dd{T}\) are equal, so what we need to prove is 
%
\begin{align}
- \frac{(\rho + P) V}{T^2} + \frac{V}{T} \qty(\dv{\rho }{T} + \dv{P}{T})
&\overset{?}{=} \frac{V}{T} \dv{\rho }{T}  \\
- \frac{(\rho + P) V}{T^2} + \frac{V}{T} \dv{P}{T} &\overset{?}{=} 0  \\
\dv{P}{T} &= \frac{\rho + P}{T}
\,,
\end{align}
%
which is precisely the statement we found to be equivalent to the Pfaff relations \eqref{eq:rewritten-pfaff-relations-entropy}. 
}
%
\begin{equation}
  \dd{S} = \dd \qty(\frac{(\rho + P) V}{T})
\,,
\end{equation}
%
therefore the differentiated quantities are equal up to an additive constant; from the conserved quantity we found and the fact that \(V \propto a^3\) we now get that the \textbf{entropy is constant} in a comoving volume in thermal equilibrium:
%
\begin{equation}
  S \equiv S(a^3, T) 
  = \frac{a^3}{T} \qty(\rho + P)
  = \const
\,.
\end{equation}

Let us see what this entails: if we take photons, for example, we have \(\rho \propto P \propto a^{-4}\): if we substitute this in we find that \(a^{-4 + 3} / T = \frac{1}{aT}\) must be a constant, therefore \(T \propto a^{-1}\).
This is known as \textbf{Tolman's law}.  
% Since \(\rho \propto T^{4}\) and \(\rho \propto a^{-4}\), we expect Tolman's law: \(Ta \sim 1\).
% \todo[inline]{Did we derive it before?}

% Also, for photons \(P  = \rho / 3\): so we get \(S \propto \frac{4}{3} (a^3/T) T^{4} \propto T^3 a^3 = \const\).

We only consider photons since they have a much larger number density.

% \todo[inline]{Is this because the suppression is exponential while the ratio of energies is somehow polynomial?}

% \todo[inline]{Add consideration about rewriting the relation as \(g_{*S}^{1/3} T a = \const\), see \cite[pag.\ 33]{paccianiAppuntiCorsoPhysical2018}.}

\subsection{Explicit expressions for the thermodynamic quantities}

Let us give explicit expressions for the number density, energy density and pressure as a function of time. We are always assuming isotropy, so in all cases we will be able to simplify the angular part of the triple integral in \(\dd[3]{q}\) as
%
\begin{equation}
  \int  \dd[3]{\vec{q}} = 4 \pi \int_0^{\infty}  \dd{q} q^2 
\,,
\end{equation}
%
so the three expressions will read 
%
\begin{subequations}
\begin{align}
  n(T) &= \frac{g}{2\pi^2} \int  \dd{q} q^2 f(q) \\ 
  \rho(T) &= \frac{g}{2\pi^2} \int  \dd{q} q^2 f(q) E(q) \\ 
  P(T) &= \frac{g}{6\pi^2} \int  \dd{q} q^2 f(q) \frac{q^2}{E(q)} 
\,.
\end{align}
\end{subequations}

In general these do not have analytic solutions, however if we only consider the ultrarelativistic and nonrelativistic limiting cases we can do the calculation.

\paragraph{Ultrarelativistic limit}

A particle being ultrarelativistic means that its momentum is much greater than its rest energy, \(q \gg m\).

In our case we do not really care about any single particle being ultrarelativistic, rather, we ask that the temperature is high enough that the bulk of the particles is ultrarelativistic. 

The momentum of any single particles will not always be large --- in fact the distribution has its maximum at \(q = 0\) --- but the regions in which it is large give a much greater contribution than those in which it is small, as long as the temperature is large.

We define the rescaled momentum \(x = q/T\), so that then the term appearing in the exponential is \(E(q) / T = \sqrt{x^2+ m^2/T^2} \approx x = q/ T \) under the assumption that \(m / T \ll 1\). 


With this assumption we get:
%
\begin{align}
  n(T) &= \frac{g}{2 \pi^2} \int _{\mathbb{R}^{+}} \dd{q} q^2 \qty(\exp(q/T) \mp 1 )^{-1} \\
  \rho (T) &= \frac{g}{2 \pi^2} \int _{\mathbb{R}^{+}} \dd{q} q^3 \qty(\exp(q/T) \mp 1 )^{-1}  \\
  P (T) &= \frac{g}{6 \pi^2} \int _{\mathbb{R}^{+}} \dd{q} q^3 \qty(\exp(q/T) \mp 1 )^{-1} 
\,,
\end{align}
%
so we can see that in this approximation, which is equivalent to \(m \approx 0\), we get matter behaving like radiation: \(P = \rho /3\).  

The result of the integrals depends on the statistics of the particles (which determine the \(\pm\) sign in the distribution), and it is given by the following expressions:
%
\begin{equation}
  n(T) = \begin{cases}
      \displaystyle
      \frac{\xi (3)}{\pi^2} g T^3 & \text{Bose-Einstein}  \\[10pt]
      \displaystyle
      \frac{3}{4}\frac{\xi (3)}{\pi^2} g T^3 & \text{Fermi-Dirac}
  \,,
  \end{cases}
\end{equation}
%
where \(\zeta (3)\) is the Riemann zeta function calculated at 3, giving 
%
\begin{align}
\zeta (3) = \sum _{n=1}^{\infty } \frac{1}{n^3} \approx \num{1.202}
\,.
\end{align}
%

and we get the proportionality to \(T^3\).
For the energy density:
\begin{equation}
    \rho (T) = \begin{cases}
        \displaystyle
        \frac{\pi^2}{30} g T^4 & \text{Bose-Einstein}  \\[10pt]
        \displaystyle
        \frac{7}{8}\frac{\pi^2}{30} g T^4 & \text{Fermi-Dirac}  
    \,,
    \end{cases}
\end{equation}
while to get the result for the pressure \(P(T) = \rho (T) / 3\) we just divide by 3.

We note that in natural units the Stefan-Boltzmann constant is \(\sigma_{\text{SB}} = \pi^2 / 15\); the result we have found coincides with the Stefan-Boltzmann law for photons, which obey Bose-Einstein statistics and have two helicity states: \(g = 2\), therefore 
%
\begin{align}
\rho (T) = 2 \frac{\pi^2}{30 } T^{4} = \sigma_{\text{SB}} T^{4}
\,.
\end{align}

\paragraph{Nonrelativistic limit}

Now we work in the opposite limit, \(m \gg T\).
We can then expand the energy in powers of \(q/m\) (or \(T/m\): as before, the point is that the typical value of \(q\) is \(T\), so we can do it either way): 
%
\begin{align}
E = m \sqrt{1 + \frac{q^2}{m^2}} \approx m + \frac{q^2}{2m} + \mathcal{O}\qty( \frac{q^2}{m^2})
\,.
\end{align}

The first temptation one might have is to work at the lowest possible order, approximating \(E \approx m\). 
The exponential \(\exp(E / T)\) will be very large compared to \(1\), so we can neglect the \(\pm 1\) in the denominator (which also means that the difference between bosons and fermions becomes negligible).

% It would seem like this makes the exponential very large, so the difference between bosons and fermions becomes negligible.
So, to zeroth order in (\(q/m\)) we get 
%
\begin{equation}
  f \approx \exp(-\frac{m- \mu }{T}) 
\,,
\end{equation}
%
therefore the number density will be given by
%
\begin{equation}
  n = \frac{g}{2 \pi^2} \exp(-\frac{m- \mu }{T}) \int _{\mathbb{R}^{+}} \dd{q}q^2 
\,,
\end{equation}
%
which diverges.

This is called the ultraviolet catastrophe: it is due to the fact that, while we are assuming \(q\) is small, we are not enforcing this in any way, and approximating all states as having the same energy regardless of their momentum. If, instead, we go to first order in \(q/m\) then we find 
%
\begin{equation} \label{eq:boltzmann-statistics-number-density}
    n = \frac{g}{2 \pi^2} \exp(-\frac{m- \mu }{T}) \int _{\mathbb{R}^{+}} \dd{q}q^2 \exp(-\frac{q^2}{2mT}) 
    = g \qty(\frac{mT}{2 \pi })^{3/2} \exp(\frac{\mu - m}{T}) 
  \,,
\end{equation}
where we applied the identity 
%
\begin{equation}
  \int _{\mathbb{R}} \dd{x} x^2 \exp(-\alpha x^2) = \frac{\sqrt{\pi } }{2 \alpha^{3/2}} 
\,.
\end{equation}
%

The exponential factor \(\exp(- m /T)\) is known as the Boltzmann suppression factor, which tells us that as long as relativistic and nonrelativistic particles are in thermal equilibrium there will be a much smaller number of the latter.

The energy density can be easily recovered from the number density if neglect higher order terms: 
%
\begin{align}
\rho (T) = \frac{g}{2 \pi^2} \int \dd{q} q^2 E(q) f(q) \approx \frac{g}{2 \pi^2 } \int \dd{q} \qty(m + \frac{q^2}{2m}) q^2 f(q) \approx m \underbrace{\frac{g}{2 \pi^2}
\int \dd{q} q^2 f(q)}_{n(T)}
\,.
\end{align}

For the pressure, on the other hand, we have 
%
\begin{align}
P(T) &= \frac{g}{6 \pi^2} \int \dd{q} q^{4} \frac{f(q)}{m + \frac{q^2}{2m}} \approx \frac{g}{6 \pi^2} e^{-m / T} \int \dd{q} \frac{q^{4}}{m} \exp(- \frac{q^2}{2m T}) 
\,,
\end{align}
%
and now we can apply the Gaussian integral identity \cite[special case of eq.\ 10.1.11 (b)]{weberEssentialMathematicalMethods2003}:
%
\begin{align}
\int_{\mathbb{R}} \dd{x} x^{4} \exp(- \alpha x^2) = \frac{3}{8 \alpha^{2}} \sqrt{\frac{\pi }{\alpha}}
\,,
\end{align}
%
where for us \(\alpha = 1/2m T\), which gives us 
%
\begin{align}
P(T) &= \frac{g}{6 \pi^2} \frac{e^{-m/T}}{m}
\frac{3 (2 m T)^2}{8} \sqrt{2 m T \pi }
= g \frac{m^{3/2} T^{5/2} }{\pi^{3/2}2^{3/2}} e^{-m/T} = g \qty(\frac{mT}{2 \pi })^{3/2} e^{-m/T} \times T  \\
&= n(T) T
\,.
\end{align}

Therefore, \(P = T n = (T / m) \rho \), which tells us that the pressure of the nonrelativistic particles is much smaller than their energy density, since \(T /m \ll 1\): we characterize them as \emph{noninteracting dust}.
The result we found, \(P = nT\), is just the ideal gas law.

If we compare relativistic particles to nonrelativistic ones, the former dominate the latter in terms of all of these three quantities.

The physical context in which this becomes relevant, in the early universe, is that whenever the temperature drops below the mass of a certain particle, that particle starts to become nonrelativistic and its density drops exponentially, due to the Boltzmann suppression.

The main way for the particle to do so is generally to annihilate with its own antiparticle, thus producing radiation.

\end{document}