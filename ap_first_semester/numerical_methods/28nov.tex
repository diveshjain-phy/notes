\documentclass[main.tex]{subfiles}
\begin{document}

\section*{Thu Nov 28 2019}

We want to generate random numbers in order to draw samples from known distributions. 
We can only really generate \emph{pseudo}-random numbers: for example, we vary the integer\(x\) in the formula 
%
\begin{align}
  x' = \qty(ax+c) \mod m
\,,
\end{align}
%
with constant integer \(a\), \(c\) and \(m\). 

Importantly, the values of the constants and the starting value of \(x\) constitute a \emph{seed} which can be used to reproduce our results. 

If we want to produce numbers distributed according to an arbitrary pdf, we first produce uniformly distributed numbers, and then use the laws of probability. 
Say we have a desired pdf, \(p(x)\), and numbers uniformly distributed from 0 to 1, then 
%
\begin{align}
  \int_{ - \infty }^{y(x)} p(\widetilde{y}) \dd{\widetilde{y}} = \int_{0}^{x} \dd{\widetilde{x}} = x 
\,,
\end{align}
%
so if we are able to solve this and make \(y(x)\) explicit as a function of \(x\) we are done! The final expression is given in terms of the cumulative pdf: 
%
\begin{align}
  y = P^{-1} (x)
\,.
\end{align}
%
We want to generate numbers \(m\) between 0.1 and 150 distributed according to \(p(m) = m^{-\alpha }\). 

The cdf is 
%
\begin{align}
\frac{1}{N}\int_{0.1}^{m} m^{-\alpha } \dd{m} = \frac{1}{N}\qty(-\frac{0.1^{-\alpha }}{1-\alpha } +  \frac{1}{1-\alpha } m^{1-\alpha }  )
\,,
\end{align}
%
where the normalization is 
%
\begin{align}
  N = \int_{0.1}^{150} p(m) \dd{m} = \frac{1}{1-\alpha } \qty(150^{1-\alpha } - 0.1^{1-\alpha })
\,,
\end{align}
%
therefore we have 
%
\begin{align}
  Nx = \qty(-\frac{0.1^{-\alpha }}{1-\alpha } +  \frac{1}{1-\alpha } m^{1-\alpha }  )
\,,
\end{align}
%
so 
%
\begin{align}
  m = \qty(\qty(0.1^{1-\alpha } +  (1-\alpha ) Nx))^{1/(1-\alpha )}
\,,
\end{align}
%


\end{document}