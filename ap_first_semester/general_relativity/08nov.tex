\documentclass[main.tex]{subfiles}
\begin{document}

\section*{Fri Nov 08 2019}

It can be shown that the equation 
%
\begin{align}
  \dv[2]{x^{\mu }}{\tau } + \Gamma^{\mu }_{\alpha \beta } \dv{x^{\alpha }}{\tau } \dv{x^{\beta }}{\tau } = 0
\,,
\end{align}
%
in the low-field limit gives the regular acceleration in a gravitational field.

If we define \(u^{\mu }\) as \(\dv*{x^{\mu }}{\tau }\), we get that the equation is equivalent to 
%
\begin{align}
  u^{\alpha } \qty(\pdv{u^{\mu } }{x^{\alpha }} + \Gamma^{\mu }_{\alpha \beta }u^{\beta }) = u^{\alpha } \nabla_{\alpha } u^{\mu } = a^{\mu } = 0
\,.
\end{align}

The acceleration we feel corresponds to the difference between our motion and geodesic motion.

The four-velocity has constant square modulus: either 0, or \(\pm 1\), if we choose an appropriate parametrization. Therefore we can classify geodesics.

\paragraph{Timelike geodesics} have \(u^{\mu } u_{\mu } = -1\), and are related to the motion of a particle. They minimize the proper time \(\dd{\tau} = \sqrt{-\dd{s^2}}\), which is real in this case. We parametrize them by \(\tau \).

\paragraph{Spacelike geodesics} have \(u^{\mu } u_{\mu } = +1\) and can be seen as the shortest path between two points: for them, the integral of \(\dd{s}\) is stationary. We parametrize them by \(s\).

\paragraph{Null geodesics} have \(u^{\mu } u_{\mu } = 0\) are characterized by \(\dd{s}=0\). We parametrize them with some parameter of our choosing, \(\lambda \), which must be independent of proper time or space.

\subsection{Solutions of the geodesic equation}

We treat the problem in the case of a two-dimensional Euclidean plane, using polar coordinates: we know we should find straight lines, but in these coordinates the problem is nontrivial.

The metric is \(\dd{s^2} = \dd{r^2} + r^2 \dd{\theta^2}\), and the nonzero Christoffel symbols are (see exercise 3.3): 
%
\begin{align}
  \Gamma^{r}_{\theta \theta } = -r 
  \qquad \text{and} \qquad 
  \Gamma^{\theta }_{\theta r } = 
  \Gamma^{\theta }_{r \theta } = \frac{1}{r} 
\,.
\end{align}

Our equation for the \(r\) coordinate is then: 
%
\begin{align}
  \dv[2]{r}{s} - r \qty(\dv{\theta }{s})^2 = 0
\,,
\end{align}
%
while for the \(\theta \) coordinate by symmetry we can identify the two terms: 
%
\begin{align}
  \dv[2]{\theta }{s} + \frac{2}{r} \dv{r}{s} \dv{\theta }{s} =0
\,.
\end{align}

In general, in an \(n\) dimensional space, motion is defined by \(n\) scalar functions, which can be determined by our \(n\) differential equations with \(2n\) initial conditions.

It is in general useful to find \emph{first integrals}, quantities which are constant along the geodesic.

It can be shown that the second equation can be written as
%
\begin{align}
  \frac{1}{r^2} \dv[]{}{s} \qty(r^2 \dv{\theta }{s}) =0
\,,
\end{align}
%
which gives us the first integral \(A = r^2 \dv*{\theta }{s}\).

\begin{bluebox}
An easy way to see this: the equation can be written, denoting derivatives with respect to \(s\) with a dot, as 
%
\begin{align}
\ddot{\theta} + \frac{2 \dot{r} \dot{\theta}}{r} = 0
\,,
\end{align}
%
which we can rearrange as 
%
\begin{align}
\frac{ \ddot{\theta}}{\dot{\theta}}
+ 2 \frac{\dot{r}}{r} = 0
\,,
\end{align}
%
or 
%
\begin{align}
\dv{}{s} \qty( \log \dot{\theta} + 2 \log r) =
\dv{}{s} \log(\dot{\theta} r^2)
=0
\,,
\end{align}
%
so we have found our integral: the derivative of the logarithm of something is constant iff the thing is constant.
\end{bluebox} 

We can always also use the definition of the differential: 
%
\begin{align}
    \dd{s^2} = \dd{r^2} + r^2 \dd{\theta^2 } 
    \,,
\end{align}
%
so we can insert our integral:
%
\begin{align}
    \dd{s^2} = \dd{r^2} + r^2 \frac{A^2}{r^{4}} \dd{s^2}
\,,
\end{align}
%
so 
%
\begin{align}
  \dd{s^2} \qty(1- \frac{A^2}{r^2}) = \dd{r^2}
\,.
\end{align}

This has two solutions, but it can be shown that they give the same result in the end.

We want the trajectory: the locus of the points the geodesic passes through, \(r(\theta )\) or \(\theta (r)\).

We do: 
%
\begin{align}
  \dv[]{\theta }{r} = \dv[]{\theta }{s} \dv[]{s}{r} = \frac{A}{r^2} \frac{1}{\sqrt{1- \frac{A^2}{r^2}}}
\,,
\end{align}
%
so we can integrate this: 
%
\begin{align}
  \theta  = \int \dd{\theta } = \int
  \frac{A}{r^2} \qty(1 - \frac{A^2}{r^2})^{-1/2} \dd{r}
\,,
\end{align}
%
which comes out to be \(\Delta \theta  = \arccos (A/r)\), which can be inverted to find \(r\cos(\Delta \theta ) = A \): using the trigonometric relation 
%
\begin{align}
\cos(x-y) = \cos(x) \cos(y) + \sin(x) \sin(y)
\,,
\end{align}
%
we find that this is equivalent to 
%
\begin{align}
  r \cos(\theta ) \cos(\theta_0 ) + r \sin(\theta ) \sin(\theta_0 ) = A
\,,
\end{align}
%
therefore this can be written as \(y = \alpha x + \beta \).

\subsection{Euler-Lagrange equations}

The time interval can be written as 
%
\begin{subequations}
\begin{align}
  \tau_{AB} &= \int \dd{\tau } = \int \sqrt{-\dd{s^2}}  \\
  &= \int  \dd{\sigma } \mathscr{L} \qty(x^{\alpha }, \dv{x^{\alpha }}{\sigma }) 
\,,
\end{align}
\end{subequations}
%
so under a perturbation we get: 
%
\begin{subequations}
\begin{align}
  0 &= \delta \tau \\
  &= \int  \dd{\sigma } \qty(
      \pdv{\mathscr{L}}{x^{\alpha }} \delta x^{\alpha } 
      + \pdv{\mathscr{L}}{\dv{x^{\alpha }}{\sigma }} 
      \dv{ \delta x^{\alpha }}{\sigma }
  )  \\
  &= \int  \dd{\sigma } 
  \qty(
      \pdv{\mathscr{L}}{x^{\alpha }} - \dv{}{\sigma } \qty(\pdv[]{\mathscr{L}}{\dv{x^{\alpha }}{\sigma }})
  ) \delta x^{\alpha } =0
\,,
\end{align}
\end{subequations}
%
therefore the integrand must vanish identically: this gives us the Euler-Lagrange equations 
%
\begin{align}
  \pdv{\mathscr{L}}{x^{\alpha }} 
  - \dv{}{\sigma } 
  \pdv{\mathscr{L}}{\dv{x^{\alpha}}{\sigma }}=0
\end{align}

\subsection{Killing vectors}

Symmetries of the metric correspond to conserved quantities if our Lagrangian only depends on the metric.
If the metric does not depend on a coordinate, then the unit vector in that direction is called a Killing vector field. 

If we have a Killing vector field, then the momentum along the Killing vector is conserved.

If the Killing coordinate is \(x^{1}\), it can be expressed as 
%
\begin{align}
  \pdv{\mathscr L}{\dv{x^{1}}{\sigma }} = \frac{1}{2 \mathscr L} \qty(-2 g_{1 \beta } \dv{x^{\beta }}{\sigma }) = - g_{1 \beta } \dv{x^{\beta }}{\tau }
\,,
\end{align}
%
where we performed a change of variable from the derivation with respect to \(\sigma \) to one with respect to \(\tau \).

This is usually written as \(\xi^{\mu } u_{\mu } = \const\), which is actually more general.
We can write it with respect to the momentum: \(p^{\mu } \xi_{\mu }\) since we are considering a constant-mass particle.

We can apply this to our 2D example: the metric does not depend on \(\theta \), therefore \(\xi = (0,1)\) is a Killing vector, so \(g_{\theta \mu  } u^{\mu } = r^2 \dv*{\theta }{s} = \const\).



\end{document}