\documentclass[main.tex]{subfiles}
\begin{document}

\section*{Fri Oct 11 2019}

\subsection{Bases}

In Euclidean 2D geometry we can choose,  for example, the basis \(e_1 = (1,0)^\top\) and \(e_2 = (0,1)^\top\). This basis is  orthonormal with respect to the scalar product \(g_{\mu \nu} = \delta_{\mu \nu}\): \(e_{(\alpha)} \cdot e_{(\beta)} = e_{(\alpha)}^{\mu} e_{(\beta)}^{\nu} g_{\mu \nu} =  g_{(\alpha) (\beta)}\).


I use parentheses around indices to denote the fact that they are not tensorial indices, but instead denote which basis vector we are considering.
We express our vectors in components with respect to this basis.

In SR, we can do the same: our coordinate basis can be given by \(e_{(\alpha)}^{\mu} = \delta_{(\alpha)}^{\mu}\).
Now, the orthonormality \(e_{(\alpha)} \cdot e_{(\beta)} = g_{(\alpha) (\beta)}\) holds with respect to \(g_{\mu \nu} = \eta_{\mu \nu}\).

% what's your name? my name is $e_1$...

\subsection{Observers \& observations} 

Every observer will be characterized by their trajectory \(x^{\mu}(\tau)\).
We can associate a coordinate system with the observer: the one in which the observer's own  4-velocity \(u^{\mu}\) is the time-like unit vector (rescaled by a factor of \(c\): \(u^{\mu} = ce_{(0)}\)).

When the observer sees a particle with \(p^{\mu} = (E_p / c, p^i)\) they measure the energy of the particle to be \(p^{0}c\): in this frame this is \(E_{\text{measured}} = - e_{(0)}^\mu p_{\mu} c = - u^{\mu} p_{\mu} c\). 
Do note that this is a covariant expression, while \(p^{0}\) is not: the energy of a particle with 4-momentum \(p^{\mu}\) measured by an observer with 4-velocity \(u^{\mu}\) is an invariant. 

In the rest frame of the observer, their own 4-velocity is \((c, \vec{0}) = c (1, \vec{0})\).
In the rest frame of the particle, its own energy is measured to be \(mc^2\).
The measured energy by an observer such that the product of the 4-velocities of the particle and of the observer is \(-\gamma c^2\) is \(m \gamma c^2\).

The Earth moves with speed \(\num{e-4}  c\) around the Sun.

% wooooo

Now we can start using \(c=1\). We can put the \(c\) back whenever we want with dimensional analysis.

\section{Newtonian Gravity}

\subsection{The Equivalence Principle}

Just like Newton supposedly thought about universal gravity when, while looking at the sky, an apple fell on his head; Einstein supposedly thought up the equivalence principle when he saw a man falling from a rooftop.

\begin{proposition}[Equivalence principle]
Experiments in a small free falling system over a short amount of time give the same result as experiments in an inertial frame in empty space.
\end{proposition}

Why ``small''? The gravitational field is not really homogeneous.
The idea is that gravity can only be removed \emph{locally}.

If we were to see that objects fall differently even in the same neighbouhood then we would lose the EP.

\begin{definition}
The \emph{inertial} mass is an object's resistance to motion: \(m _{\text{inertial}} = F^{i}/ a^{i}\).
\end{definition}

\begin{definition}
The \emph{gravitational} mass is the one which defines the gravitational force on an object: \(m _{\text{gravitational}} = \abs{F} r^2 / (G M)\).
\end{definition}

These are \emph{a priori} different, but experimentally equal: in general the gravitational acceleration is given by
\begin{equation}
a^{i} = \frac{GM r^{i} }{r^3} \frac{m _{\text{gravitational}}}{m _{\text{inertial}}}  
\end{equation}

If the ratio of masses depended on the material, this could vary.

We can do a torsion pendulum experiment: the torsion applied by the Coriolis effect on a pendulum depends on the inertial mass, while its restoring force depends on the gravitational mass.
Experimentally we have measured them to be equal with an accuracy of \(\num{e-12} \).

A person on a rocket accelarating at \(g\) experiences the same acceleration as a person standing on Earth.

\subsection{Gravitational redshift}

We treat it now in a weak field approximation.

% now with gender roles!

Alice sends radiation to Bob from a higher altitude on Earth. Alice sends it with frequency \(f\), Bob receives it with \(f'\). They are at rest with respect to one another: there is no kinematic Doppler effect here.

We do this by applying the equivalence principle! We imagine A and B to be standing in a rocket which is accelerating at \(g\):
there is no more gravity.

Bob will receive a greater frequency: \(f'>f\). This can be seen by imagining two consecutive wavefronts as two particles. Alice sends them \(\Delta t_A = 1/f\) apart, Bob receives them as \(\Delta t_B=1/f'\) apart.

If the rocket is at rest, the time for the radiation to reach B is \(h/c\); if the rocket is moving then the time is \(<h/c\).

When the second wavefront starts moving the rocket is already going: the second wave starts later but it has less distance to travel. Therefore \(\Delta t_A > \Delta t_B\), which implies \(f'>f\).

\begin{claim}
    The first terms in the expansion are:
    \begin{equation}
        f' = f \qty(1+ \frac{gh}{c^2} + O \qty(\qty(\frac{gh}{c^2})^2))\,.
    \end{equation}
\end{claim}
    
\subsection{Potentials}

In electromagnetism, the potential energy between a charge \(Q\) and a test charge \(q\) is \(U = k Qq/r\): then we define the electromagnetic potential \(V = U/q\) which has the advantage of being test-charge independent.

Similarly, we define the gravitational potential \(\Phi = U/m = gh\).
Then, the second order term in the formula for the redshift becomes \(\Delta \Phi c^{-2}\): now we can properly say that this \emph{weak field} means \(\Delta \Phi c^{-2} \ll 1\).

This is surely the case for the cases we can treat concretely.
If two people are separated by \SI{1}{km} of difference in altitude, they have \(\Delta \Phi c^{-2} \approx \num{e-13} \): the difference they will experience is one second in a million years.

Our expression from \(\Phi \) in the newtonian approximation is \(\Phi = GM /r\).

We can say even now by dimensional analysis that \(GM / (rc^2)\) is the parameter which tells us how relevant the gravitational effects are: if it is similar to 1 we must consider GR.

This is very close to the expression for the Schwarzschild radius: it is \(r=M\) in natural units \(c=G=1\), while the correct expression is \(r= 2M\): that one can actually be recovered exactly if we calculate the radius at which the escape velocity is equal to \(c\).

\end{document}