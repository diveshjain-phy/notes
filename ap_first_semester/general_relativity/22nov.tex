\documentclass[main.tex]{subfiles}
\begin{document}

\section*{Fri Nov 22 2019}

We found the equation 
%
\begin{align}
  \frac{1}{l^2} \qty(\dv{r}{\lambda })^2 + W _{\text{eff}}(r) = \frac{1}{b^2}
\,,
\end{align}
%
where 
%
\begin{align}
  W _{\text{eff}} = \frac{1}{r^2} \qty(1 - \frac{2GM}{r})
\,
\end{align}
%
and 
%
\begin{align}
  b^2 = \frac{l^2}{e^2}
\,,
\end{align}
%
\(l\) and \(e\) being the integrals corresponding to the Killing vectors of time translations and azimuthal angle rotations. 

Now, we want to give the interpretation of \(b\) as the impact parameter. 
We consider a BH at the origin of the \(x, y\) axes, and a photon approaching parallel to the \(x\) axis with impact parameter \(d\). The impact parameter is the distance between the two lines: the trajectory of the photon far away from the BH and the line parallel to the trajectory and passing through the BH. 

The parameter 
%
\begin{align}
  \dv{\varphi }{t} = \underbrace{\dv{r}{t}}_{-1} \dv{\varphi }{r} = - \qty(-\frac{d}{r^2})
\,,
\end{align}
%
where we used a small angle approximation: \(\varphi \approx d/r\), which can be differentiated with respect to \(r\). 
So 
%
\begin{align}
  b = \frac{l}{e} = \frac{r^2 \dv*{\varphi }{\lambda }}{\dv*{t}{\lambda }} = r^2 \dv{\varphi }{t} = d 
\,,
\end{align}
%
which means that \(b=d\): the ratio of \(l\) to \(e\) is the impact parameter. 

So, the photon interacts with the BH and the angle of deflection of the path compared to a straight path is denote as \(\delta \varphi \). 
The total deflection angle is \(\Delta \varphi  = \pi + \delta \varphi \). 

The paramter \(l\) is 
%
\begin{align}
  l = r^2 \dv{\varphi }{\lambda } 
\,,
\end{align}
%
so 
%
\begin{align}
  \dv{}{\lambda } = \frac{l}{r^2} \dv{}{\varphi }
\,.
\end{align}

This allows us to change variables in our 1D equation, and we get 
%
\begin{align}
\frac{1}{l^2} \frac{l^2}{r^2} \qty(\dv{r}{\varphi })^2 + \frac{1}{r^2}\qty(1 - \frac{2GM}{r}) = \frac{1}{b^2}
\,,
\end{align}
%
so if we change variables to \(u = r^{-1}\), with 
%
\begin{align}
  \dv{r}{\varphi } = -\frac{1}{u^2} \dv{u }{\varphi }
\,,
\end{align}
%
we  get 
%
\begin{align}
  u^{4} u^{-4} \qty(\dv{u}{\varphi } )^2+ u^2 \qty(1-2GMu) = \frac{1}{b^2}
\,,
\end{align}
%
and then, differentiating, we find 
%
\begin{align}
  2 \dv{u}{\varphi } \dv[2]{u}{\varphi } + 2 u \dv{u}{\varphi } - 6GM u^2 \dv{u}{\varphi }= 0
\,,
\end{align}
%
and we can simplify as long as \(\dv*{u}{\varphi } \neq 0\), which only fails at one point. So in the end our equation is 
%
\begin{align}
  \dv[2]{u}{\varphi } + u = 3GMu^2
\,.
\end{align}
%

We solve it perturbatively. 
If \(GM=0\), there is no BH and we have a straight line: the impact parameter is constant and equal to  \(b = r \sin(\varphi )\), therefore 
%
\begin{align}
  u = \frac{1}{b} \sin(\varphi )
\,,
\end{align}
%
is the most general solution to the harmonic oscillator which satisfies the boundary conditions. 
So, we hypothesize that our solution satisfies 
%
\begin{align}
  u (\varphi ) = \frac{1}{b} \qty(\sin(\varphi ) + W(\varphi ))
\,,
\end{align}
%
where \(W\) is small. 

Then, we insert this: 
%
\begin{align}
  -\frac{1}{b} \sin(\varphi ) + \frac{1}{b} \dv[2]{W}{\varphi } + \frac{1}{b} W \approx 3GM \frac{\sin^2(\varphi )}{b}
\,,
\end{align}
%
but since the zeroth order equation is satisfied we find: 
%
\begin{align}
  \dv[2]{W}{\varphi } + W \approx \frac{3GM}{b} \sin^2(\varphi )
\,. 
\end{align}

Our ansatz is then: 
%
\begin{align}
  W = A + B \sin^2 \varphi 
\,,
\end{align}
%
which looks like it might solve the equation. Its second derivative is 
%
\begin{align}
  \dv[2]{W}{\varphi } = 2B \qty(\cos^2 \varphi  - \sin^2 \varphi ) = 2B - 4B \sin^2\varphi 
\,.
\end{align}

Inserting this we get:
%
\begin{align}
  2B - 4B \sin^2\varphi + A + B \sin^2\varphi 
  = \frac{3GM}{b} \sin^2\varphi 
\,,
\end{align}
%
which implies \(2B+A = 0\) and \(-3B = 3GM/b\). Then, 
%
\begin{align}
  W = \frac{2GM}{b} - \frac{GM}{b} \sin^2\varphi 
  = \frac{2GM}{b} \qty(1 - \frac{\sin^2\varphi}{2} )
\,
\end{align}
%
solves the perturbed equation. 
We can see that our condition of \(W \ll 1\) is actually the physically meaningful \(GM \ll b\), or the impact parameter being much larger than the Schwarzschild radius. 

Our complete solution is 
%
\begin{align}
  u (\varphi ) = \frac{1}{b} \qty(\sin \varphi + \frac{2GM}{b} \qty(1 - \frac{\sin^2\varphi}{2}))
\,,
\end{align}
%
and we are interested in the asymptotic past and future, which correspond to \(u =0\). 
Now, \(\varphi _{\text{in}}= 0\) and \(\varphi _{\text{out}} = \pi\) will not be a solution anymore. However, the deflection is small so we write the solution as \(\varphi _{\text{in}} = \epsilon _{\text{in}}\) and \(\varphi _{\text{out}} = \pi + \epsilon _{\text{out}}\).
We substitute these in to the equation \(u = 0\): 
%
\begin{align}
  0 = \sin(\epsilon _{\text{in}}) + \frac{2GM}{b}
\,,
\end{align}
%
or \(\epsilon _{\text{in}} \approx - 2GM/b\), since the deflection is small. 

For \(\epsilon _{\text{out}}\) we will have 
%
\begin{align}
  0= \sin(\pi + \epsilon _{\text{out}}) + \frac{2GM}{b} \approx - \epsilon _{\text{out}} + \frac{2GM}{b}
\,,
\end{align}
%
so \(\delta \varphi = 2 \varphi _{\text{in}} = 2 \varphi _{\text{out}} = 4GM/b\). 

This was one of the first tests of GR by Sir Eddington in 1919: during an eclipse he saw a shift in the apparent position of the stars. reinserting \(c\) we find that we must divide \(4GM/b\) by \(c^2\). 
Our \(b\) is approximately the radius of the Sun: the calculation is
\begin{lstlisting}[language=Python]
from scipy.constants import *
sun_mass = 2e30
sun_radius = 696e6
rad2arcsec = 3600 * 180 / pi
4*G*sun_mass/c**2 /sun_radius * rad2arcsec
\end{lstlisting}

For the rest of today, we will talk about the Schwarzschild horizon: recall the line element 
%
\begin{align}
  \dd{s^2} = - \qty(1- \frac{2GM}{r}) \dd{t^2}
  + \qty(1 - \frac{2GM}{r})^{-1} \dd{r^2} 
  + r^2 \dd{\Omega^2}
\,.
\end{align}
%

It is useful to plot light cones in order to understand the structure of the spacetime. 
We restrict ourselves to radial motion of light. So, we have 
%
\begin{align}
  0 = - \qty(1- \frac{2GM}{r}) \dd{t^2}
  + \qty(1 - \frac{2GM}{r})^{-1} \dd{r^2} 
\,,
\end{align}
%
or 
%
\begin{align}
  \dv{t}{r} = \pm \qty(1 - \frac{2GM}{r})^{-1}
\,.
\end{align}
%

The light cones become slimmer and slimmer as we approach the horizon, they are straight lines at \(r = 2GM\): the photon appears to cover less and less \(\dd{r}\) for a fixed \(\dd{t}\) as it approaches the horizon. 
Massive particles are even slower. 
Let us integrate this relation: 
%
\begin{align}
  \int_0^t \dd{t} = - \int_{r_0}^{r(t)} \frac{\dd{r}}{1 - \frac{2GM}{r}} 
\,,
\end{align}
%
which comes out, by a separation of fractions, to be 
%
\begin{align}
  t = r_0 - r(t) + 2GM \log \qty(r_0 - 2GM) - 2GM \log \qty(r(t) - 2GM)
\,,
\end{align}
%
which diverges as \(r(t)\) approaches \(2GM\). 

What do we really mean by \(t\)? 
An observer which is far away and at rest has \(g_{\mu \nu } = \eta_{\mu \nu }\), and \(t = \tau \): \(t\) is the proper time, as measured on the clock of an observer who is far away. 
They will measure the photon as going slower and slower, and becoming redder and redder. 

Let us neglect Doppler redshift, which is due to motion. 
The formula for gravitational redshift is 
%
\begin{align}
  f _{\text{obs}} = f _{\text{emit}} \sqrt{\frac{- g_{00} (\text{emit})}{-g_{00} (\text{obs})}}
\,.
\end{align}
%

In our case we have 
%
\begin{align}
  f _{\text{obs}} = f _{\text{emit}} \sqrt{1 - \frac{2GM}{r}}
\,,
\end{align}
%
which approaches zero as \(r \rightarrow 2GM\). 
We see the infalling observer becoming redder and ultimately freezing. 

We should use different coordinates to describe the infalling observer who passes through the event horizon. 

First of all, we discuss Minkowski spacetime as seen by an accelerating observer: Rindler spacetime and the Riddler horizon. 

Recall the exercise in sheet 2, about an accelerating observer: we now consider an observer moving with the position law 
%
\begin{align}
  x(t) = \frac{1}{\kappa } \sqrt{1 + \kappa^2t^2}
\,,
\end{align}
%
(as opposed to the homework, we remove the constant added to this position).
Like in the homework we compute the proper time for the observer: 
%
\begin{align}
  \dd{s^2} = - \dd{t^2} \qty(1 - \qty(\dv{x}{t})^2)
\,,
\end{align}
%
so 
%
\begin{align}
  \dd{\tau} = \frac{ \dd{t}}{\sqrt{1 + \kappa^2t^2}}
\,,
\end{align}
%
which means 
%
\begin{align}
t = \frac{1}{\kappa } \operatorname{sinh} (\kappa \tau )
    \qquad 
\tau  = \frac{1}{\kappa } \operatorname{arcsinh} (\kappa \tau )
\,,
\end{align}
%
therefore 
%
\begin{align}
  x = \frac{1}{\kappa } \cosh(\kappa \tau )
\,.
\end{align}
%

We want a coordinate system in which 
\begin{enumerate}
    \item the observer is at constant spatial position;
    \item where, up to a constant, the time is equal to the proper time.
\end{enumerate}

Our change of variable is 
%
\begin{align}
  \begin{cases}
      t = \rho \sinh \eta  \\
      x = \rho \cosh \eta 
  \end{cases}
\,.
\end{align}
%

the observer is at fixed spatial coordinate \(\rho_{*} = 1 / \kappa \), and the proper time measured is \(\tau = \eta / \kappa = \eta \rho_{*}\). 

Let us consider a family of observers at different spatial locations in the new frame: each has a constant acceleration, this means varying \(\kappa \) or \(\rho \).

If instead we vary \(\eta \) we have: 
%
\begin{align}
  \frac{t}{x} = \tanh \eta \implies 
  \eta = \tanh^{-1} \qty(\frac{t}{x})
\,,
\end{align}
%
and we can see that since \(\tanh 0 = 0 \) we have that the \(t=0\) axis has \(\eta = 0\), while the lightspeed observers are at \(\eta = \pm \infty\). 

These coordinates cover one quadrant of Minkowski spacetime. 
The line element in these new coordinates is 
%
\begin{align}
  \dd{s^2} &= - \dd{t^2} + \dd{x^2}   \\
  &= - \qty(\dd{\rho^2} \sinh \eta  + \rho \cosh \eta \dd{\eta })^2
  + \qty( \dd{\rho^2} \cosh \eta + \rho \sinh \eta \dd{\eta })^2  \\
  &= - \rho^2 \dd{\eta^2} + \dd{\rho^2}
\,.
\end{align}
%

This is \emph{Rindler geometry}. 

If we have an observer staying at \(x_0 >0\), and there is a Rindler observer, then after a time \(x_0\) the observer exits the quarter of the plane covered by the Rindler coordinates: if event \(A\) is at \((0, x_0 )\), and \(B\) is at \((x_0 , x_0 )\) then the \(\eta \) of event \(A\) is zero, while the \(\eta \) of event \(B\) is infinite. 

\end{document}