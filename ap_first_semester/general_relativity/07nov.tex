\documentclass[main.tex]{subfiles}
\begin{document}

\section*{Thu Nov 07 2019}

We are trying to fix the last coefficient in the Einstein equations: 
%
\begin{align}
  R_{\mu \nu } - \frac{R}{2} g_{ \mu \nu } = c_2 T_{\mu \nu }
\,.
\end{align}

The \(\frac[i]{1}{2} \) on the LHS is fixed by the conservation of the stress-energy tensor.

We look at a low-energy scenario: \(\rho \ll P\), \(u^{\mu } = (1, \vec{0})^\top\). Then the EFE become \(R_{00} = c_2 \rho /2\).

We consider a stationary metric \(g_{\mu \nu } (\vec{x})\), which does not depend on time. This is consistent with the stuff we usually see: planetary dynamics are quasi-static with respect to the speed of light.

In a LIF we have shown that the Riemann tensor is:
%
\begin{align}
  R_{\gamma \sigma \mu \nu } = \frac{1}{2} \qty(g_{\gamma \nu , \sigma \mu } - g_{\sigma \nu , \gamma \mu } - g_{\gamma \mu , \sigma \nu } + g_{\sigma \mu , \gamma \nu })
\,.
\end{align}

To get the Ricci tensor we need \(R^{\alpha }_{\sigma \mu \nu } = \eta^{\alpha \gamma }R_{\gamma \sigma \mu \nu }\) since we are in the LIF.

We get the Ricci tensor:
%
\begin{align}
  R_{\sigma \nu } &=  \eta^{\alpha \gamma } \qty(g_{\gamma \nu , \sigma \alpha } - g_{\sigma \nu , \gamma \alpha } - g_{\gamma \alpha , \sigma \nu } + g_{\sigma \alpha , \gamma \nu })
\,,
\end{align}
%
from which we can calculate \(R_{00}\):
%
\begin{subequations}
\begin{align}
    R_{0 0 } &= \frac{1}{2} \eta^{\alpha \gamma } \qty(g_{\gamma 0 , 0 \alpha } - g_{0 0 , \gamma \alpha } - g_{\gamma \alpha , 0 0 } + g_{0 \alpha , \gamma 0 })  \\
    &= -\frac{1}{2}\eta^{i j }g_{00, ij} 
    = - \frac{1}{2} \sum _{i} g_{00, ii} = -\frac{1}{2} \nabla^2 g_{00}
    \,,
\end{align}
\end{subequations}
%
where only the second term survives since the metric is time-independent, its time derivatives all vanish.

Therefore, our equation becomes \(\nabla^2 g_{00} = c_2 \rho \). We just need to find out what the meaning of \(g_{00}\) is, how it is related to the gravitational potential.
We do it with gravitational redshift: we found that 
%
\begin{align}
  \Delta \tau_{A} \approx \Delta \tau_{B} (1 - \Phi_{A} + \Phi_{B})
  \,,
\end{align}
%
if Alice, on the top of a building, is sending photons to Bob who is on the ground. To first order in the field, the expression is equivalent to 
%
\begin{align}
    \Delta \tau_{A} \approx \Delta \tau_{B} (1 - \Phi_{A}) (1+ \Phi_{B}) \approx \Delta \tau_{B} \frac{1 + \Phi_{B}}{1 + \Phi_{A}}
\,,
\end{align}
%
which is equivalent to the constancy of 
%
\begin{align}
  \frac{\Delta \tau}{1 + \Phi }
\,.
\end{align}

The \(\Delta \tau \) are proper times measured by the observers \(A\) and \(B\) at rest. 
For an observer at rest, the spacetime interval is 
%
\begin{align}
  -\dd{\tau }^2 = \dd{s}^2 = g_{00} \dd{t^2} 
\,,
\end{align}
%
since the \(\dd{x^{i}} \) are null. Therefore, \(\dd{\tau } = \sqrt{-g_{00}}  \dd{t} \).

We have 
%
\begin{align}
  \Delta \tau_{A} = \sqrt{-g_{00}(A)} \Delta t
    \qquad \text{and} \qquad
  \Delta \tau_{B} = \sqrt{-g_{00}(B)} \Delta t
\,,
\end{align}
%
but the \(\Delta t\) are the same, because the metric is constant with respect to time! so we get that \(\Delta t = \const\) and specifically it is equal to 
%
\begin{align}
  \frac{\Delta \tau }{\sqrt{-g_{00}}}
\,,
\end{align}
%
so we can finally identify \(\sqrt{-g_{00}} \approx 1 + \Phi \), or \(g_{00} = - (1+\Phi )^2 = - (1+ 2 \Phi )\).

We can now bring the Laplacian inside the \(g_{00}\): we get 
%
\begin{align}
  \nabla^2 \Phi = \frac{c_2}{2} \rho  
\,,
\end{align}
%
Poisson's equation for the gravitational potential.

We know the gravitational potential to be defined by \(- \nabla \Phi = \vec{F_G}\).

We can find Gauss' law for the gravitational field just like we did for the electromagnetic field, substituting \(Q \rightarrow M\) and \(1 / (4 \pi \epsilon_{0}) \rightarrow -G\).
I will also denote the electric and gravitational charges with subscripts \(E\) and \(G\).

The integral form of this equation is 
%
\begin{align}
  \int _{\partial V} \dd{\vec{A}} \cdot \vec{E} = \int_V \dd[3]{x} \rho_E
\,,
\end{align}
%
but it can be expressed differentially with the divergence theorem as 
%
\begin{align}
  \vec{\nabla} \cdot \vec{E} = \frac{\rho_E}{\epsilon_0 }
\,.
\end{align}

For the gravitational field \(\vec{F_G}\) then we can just substitute: 
%
\begin{align}
  \vec{\nabla} \cdot \vec{F_G} = - 4 \pi G \rho_G
\,,
\end{align}
%
and then we can substutite the gravitational potential: 
%
\begin{align}
  \vec{\nabla} \cdot ( - \vec{\nabla} \Phi ) = - \nabla^2 \Phi  = - 4 \pi G \rho_G
\,,
\end{align}
%
or \(\nabla^2 \Phi = 4 \pi G \rho_G\).
So this is what we found before, with \(c_2/2 = 4 \pi G \).
So, our constant is \(c_2 = 8 \pi G\). So we get Einstein's equations: 
%
\begin{align}
  R_{\mu \nu } - \frac{R}{2} g_{\mu \nu } = 8 \pi G T_{\mu \nu }
\,.
\end{align}

The LHS of these is called the Einstein tensor, \(G_{\mu \nu } \equiv R_{\mu \nu } - \frac[i]{R}{2} g_{\mu \nu }\).

Why do we not write an equation with more derivatives, more indices? We may, but we'd detect these only in the regime of very strong curvature: the relativistic effects are already hard to detect as is!
For now the vanilla EFE have always agreed with experiment.

\section{Geodesics}

In flat spacetime, we know that the curve with \(\dv*[2]{x^{\mu }}{\tau }=0\) stationarizes \(\tau = \int \dd{\tau }\).

Now we will do the exact same thing, except that \(\dd{\tau }\) will be calculated with \(g_{\mu \nu }\) instead of \(\eta_{\mu \nu } \).

We perturb our path \(x^{ \mu }(\lambda )\) as \(x^{\mu } + \delta x^{\mu }\), and we want to set to zero the first functional derivative \(\fdv*{\tau_{AB}}{x^{\mu }}\). This means that to first order 
%
\begin{align}
  \delta \qty(\int _{0}^{1} \dd{\sigma }\sqrt{-g_{\alpha \beta } \dv{x^{\alpha }}{\sigma }\dv{x^{\beta }}{\sigma }} )=0
\,,
\end{align}
%
where we performed a reparametrization of the curve. Computing away, with the notation \(u^{\mu } = \dv*{x^{\mu }}{\sigma }\): 
%
\begin{subequations}
\begin{align}
  \delta \tau_{AB} &= \int_0^{1} \dd{\sigma} \qty(
    \frac{{ -\delta g_{ \alpha \beta } u^{\alpha  } u^{\beta }}}{2 \sqrt{\dots}} +
      \frac{-g_{\alpha \beta } u^{\alpha } \dv{ \delta x^{\beta }}{\sigma } }{\sqrt{\dots}}
    )  \\
    &= - \frac{1}{2} \int_0^1  \dd{\sigma } \qty(
        \delta g_{\alpha \beta } \dv{x^{\alpha }}{\sigma }
        \dv{x^{\beta }}{\tau }
        + 2 g_{ \alpha \beta } \dv{x^{\alpha }}{\tau }
        \dv{ \delta x^{\beta }}{\sigma }
    )  \\
    &= - \frac{1}{2} \int_0^1 \dd{\sigma } \qty(
        \partial_{\gamma } g_{\alpha \beta } \delta x^{\gamma } \dv{x^{\alpha }}{\sigma }
        \dv{x^{\beta }}{\tau }
        -2 \dv{}{\sigma } \qty(g_{\alpha \beta } \dv{x^{\alpha }}{\tau }) \delta x^{\beta }
    )  \\
    &= - \frac{1}{2} \int_0^1 \dd{\tau } \qty(
        \partial_{\gamma } g_{\alpha \beta } \delta x^{\gamma } \dv{x^{\alpha }}{\tau } \dv{x^{\beta }}{\tau } +
        \dv{}{\tau } \qty(g_{ \alpha \beta } \dv{x^{\alpha }}{\tau } \delta x^{ \beta } )
    )  \\
    & \int \dd{\tau } \qty(
        -\frac{1}{2} \partial_{\gamma }g_{\alpha \beta } \dv{x^{\alpha }}{\tau } \dv{x^{\beta }}{\tau } + 
        \dv{}{\tau } \qty(g_{\alpha \gamma } \dv{x^{\alpha }}{\tau }) \delta x^{\gamma }
    )
\,,
\end{align}
\end{subequations}
%
where we applied the product rule, identified two symmetric terms in the velocity, used the identity 
%
\begin{align}
  \frac{1}{\sqrt{\dots}} \dv{}{ \sigma } = \dv{}{\tau }
\,,
\end{align}
%
expanded the metric using: 
%
\begin{align}
  \delta g_{\alpha \beta } = \partial_{\gamma } g_{ \alpha \beta } \delta x^{\gamma }
\,,
\end{align}
%
where we did not need to introduce a covariant derivative since we are just Taylor expanding.

Also, we integrated by parts (without boundary terms since the path variation vanishes at the path boundary), changed variables, and finally gotten an expression which must vanish for any path, therefore we get that 
%
\begin{align}
    -\frac{1}{2} \partial_{\gamma }g_{\alpha \beta } \dv{x^{\alpha }}{\tau } \dv{x^{\beta }}{\tau } 
    + \dv{}{\tau } \qty(g_{\alpha \gamma } \dv{x^{\alpha }}{\tau }) = 0
\,,
\end{align}
%
so we can expand the derivative: we get 
%
\begin{align}
  0= - \frac{1}{2} \partial_{\gamma }g_{\alpha \beta } \dv{x^{\alpha }}{\tau } \dv{x^{\beta }}{\tau } + 
  \partial_{\beta }g_{\alpha \gamma } \dv{x^{\beta }}{\tau } \dv{x^{\alpha }}{\tau } + g_{\alpha \gamma } \dv[2]{x^{\alpha }}{\tau }
\,,
\end{align}
%
where the second term can be symmetrized in \(\alpha \beta \): 
%
\begin{align}
  0= g_{\alpha \gamma } \dv[2]{x^{\alpha }}{\tau } +
  \qty(\frac{1}{2} \partial_{\beta }g_{\alpha \gamma } + \frac{1}{2}\partial_{\alpha } g_{\beta \gamma }) - \frac{1}{2}\partial_{\gamma }g_{\alpha \beta } \dv{x^{\alpha }}{\tau } \dv{x^{\beta }}{\tau }
\,,
\end{align}
%
so we get 
%
\begin{align}
  0= g_{\alpha \gamma } \dv[2]{x^{\alpha }}{\tau } +
  \frac{1}{2}\qty(g_{\alpha \gamma , \beta } + g_{\beta \gamma , \alpha } - g_{\alpha \beta , \gamma })
  \dv{x^{\alpha }}{\tau } \dv{x^{\beta }}{\tau }
\,,
\end{align}
%
which, raising an index and identifying the Christoffel symbols, gives us 
%
\begin{align}
  0 = \dv[2]{x^{\gamma }}{\tau } + \Gamma^{\gamma }_{\alpha \beta } \dv{x^{\alpha }}{\tau } \dv{x^{\beta }}{\tau }
\,,
\end{align}
%
the \emph{geodesic equation}.

It can be written also as \(u^{\mu } \nabla_{\mu }u^{\nu } = a^{\nu } = 0\).

\end{document}