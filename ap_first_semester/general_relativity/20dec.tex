\documentclass[main.tex]{subfiles}
\begin{document}

\section*{Fri Dec 20 2019}

\section{Gravitational waves}

They are solutions of the linearized EFE on a given fixed background. Today we only treat GW on a Minkowski background, but we could do it for any number of other (cosmological) background. 

So, we say: 
%
\begin{align}
g_{\mu \nu } = \eta_{\mu \nu } + h_{\mu \nu }
\,,
\end{align}
%
with \(h_{\mu \nu } \ll 1\), so we neglect anything which is \(\mathcal{O}(h^2)\). 

GW are very topical today: they could allow us to see beyond the last scattering surface, when the universe stopped being opaque to EM radiation: then, it had been transparent to gravitational radiation from a long time before. 

We can do statistics to the stochastic GW background, for now however we have only detected GW from localized event. 

We must start to write down the EFE from the Christoffel symbols: in general they are 
%
\begin{align}
\Gamma^{\alpha }_{\mu \nu } = \frac{1}{2} g^{\alpha \lambda } \qty(g_{\lambda \nu , \mu } + g_{\mu \lambda , \nu } - g_{\mu \nu , \lambda })
\,,
\end{align}
%
but in the derivatives of the metric we only have \(\mathcal{O}(h)\) terms, so in the inverse metric we neglect all the \(\mathcal{O}(h)\) terms since their global contribution would be \(\mathcal{O}(h^2)\): so we get 
%
\begin{align}
    \Gamma^{\alpha }_{\mu \nu } = \frac{1}{2} \eta ^{\alpha \lambda } \qty(h_{\lambda \nu , \mu } + h_{\mu \lambda , \nu } - h_{\mu \nu , \lambda })
\,,
\end{align}
%
and when we compute the Ricci tensor we will only keep the \(\partial \Gamma \) terms, since the \(\Gamma \Gamma \) terms are \(\mathcal{O}(h^2)\). So our expression becomes 
%
\begin{align}
R_{\mu \nu } &= \partial_{\alpha } \Gamma^{\alpha }_{\mu \nu } - \partial_{\nu } \Gamma^{\alpha }_{\mu \alpha }  \\
&= \frac{1}{2} \eta^{\alpha \lambda } \partial_{\alpha } \qty(h_{\lambda \nu , \mu } + \cancelto{}{h_{\mu \lambda , \nu }} - h_{\mu \nu , \lambda }) - \frac{1}{2} \eta^{\alpha \lambda } \partial_{\nu  }\qty(h_{\lambda \alpha , \mu } + \cancelto{}{h_{\mu \lambda , \alpha }} - h_{\mu \alpha , \lambda })  \\
&= \frac{1}{2} \eta^{\alpha \lambda } \qty(h_{\lambda \nu , \mu \alpha } - h_{\mu \nu , \alpha \lambda } - \frac{1}{2} h_{\lambda \alpha , \mu \nu } - \frac{1}{2} h_{\lambda \alpha , \mu \nu } + h_{\mu \alpha , \lambda \nu } )
\,,
\end{align}
%
where we split a term in two in order to collect, so we get 
%
\begin{align}
R_{\mu \nu } = \frac{1}{2} \eta^{\alpha \lambda } \partial_{\mu } \qty(h_{\lambda \nu , \alpha } - \frac{1}{2} h_{\lambda \alpha , \nu })
+ \frac{1}{2} \eta^{\alpha \lambda } \partial_{\nu } 
\qty(h_{\mu \alpha , \lambda } - \frac{1}{2} h_{\lambda \alpha , \mu })- \frac{1}{2} \eta^{\alpha \lambda } \partial_{\alpha } \partial_{\lambda } h_{\mu \nu }
\,,
\end{align}
%
where we recognize the Dalambertian \(\square = \eta^{\alpha \lambda  }  \partial_{\alpha } \partial_{\lambda }\) and the trace of \(h_{\mu \nu }\): \(h \equiv h^{\mu }_{\mu }\). So we get 
%
\begin{align}
R_{\mu \nu } = - \frac{1}{2} \square h_{\mu \nu } 
+ \frac{1}{2} \partial_{\mu } \qty(\partial_{\lambda } h^{\lambda }_{\nu } - \frac{1}{2} \partial_{\nu }h)
+ \frac{1}{2} \partial_{\nu }\qty(\partial_{\lambda}  h^{\lambda }_{\mu } - \frac{1}{2} \partial_{\mu} h)
\,,
\end{align}
%
so we need to solve \(R_{\mu \nu }  =0 \) with some initial conditions, since we need to solve the EFE \(G_{\mu \nu } = M_P^{-2} T_{\mu \nu } = 0\), which implies \(R_{\mu \nu }= 0 \). 
The problem is that the solution is not unique: we have the freedom of choosing a gauge, since different metrics connected to each other through changes of coordinates represent the same physical scenario: \(h_{\mu \nu } \) \emph{has no direct physical meaning}. 

In general we have a change of coordinates \(x \to \widetilde{x}\), and if we do this 
%
\begin{align}
g_{\mu \nu } \to \widetilde{g}_{\mu \nu } = \pdv{x^{\alpha }}{\widetilde{x}^{\mu  }} \pdv{x^{\beta }}{\widetilde{x}^{\nu }} g_{\alpha \beta }
\,,
\end{align}
%
but we can only do changes of coordinates which leave the Minkowski part of the metric invariant, that is \(\widetilde{g}_{\mu \nu } = \eta_{\mu \nu } + \widetilde{h}_{\mu \nu }\). We can actually restrict ourselves to infinitesimal changes of coordinates: \(x \to \widetilde{x} = x + \epsilon \). So we do 
%
\begin{align}
\pdv{\widetilde{x}^{\mu }}{x^{\alpha }} = \partial_{\alpha } \qty( x^{ \mu } + \epsilon^{\mu })
= \delta_{\alpha }^{\mu } + \pdv{\epsilon^{\mu }}{x^{\alpha }} = \delta_{\mu }^{\alpha } + \partial_{\alpha } \epsilon^{\mu }
\,.
\end{align}

Inserting this inside the inverse change of coordinates formula for the metric (the one for \(g_{\mu \nu }\) in terms of \(\widetilde{g}_{\mu \nu }\), the inverse of the one we wrote down before) we get 
%
\begin{align}
\qty(\delta^{\mu}_{\alpha } + \partial_{\alpha } \epsilon^{\mu }) \qty(\partial^{\nu }_{\beta } + \partial_{\beta } \epsilon^{\nu }) \qty(\eta_{\mu \nu } + \widetilde{h}_{\mu \nu }) &= \eta_{\alpha \beta } +h_{\alpha \beta }  \\
\qty(\delta^{\mu }_{\alpha } \delta^{\nu }_{\beta } + \delta^{\mu }_{\alpha } \partial_{\beta } \epsilon^{\nu } 
+ \delta^{\nu  }_{\beta  } \partial_{\alpha  } \epsilon^{\mu  } + \mathcal{O} (\epsilon^2))  \qty(\eta_{\mu \nu } + \widetilde{h}_{\mu \nu }) &= \eta_{\alpha \beta }  + h_{\alpha \beta }  \\
\widetilde{h}_{\alpha \beta }+
\partial_{\alpha } \epsilon_{\beta } + \partial_{\beta } \epsilon_{\alpha } &= h_{\alpha \beta } 
\,.
\end{align}

This freedom of changing \(h_{\mu \nu }\) is called gauge freedom: if \(h_{\alpha \beta }\) solves the linearized EFE, then \(\widetilde{h}_{\alpha \beta } = h_{\alpha \beta } + \partial_{(\alpha } \epsilon_{\beta )}\) also does (there is a factor 2 missing, but it does not matter since \(\epsilon \) is generic and still infinitesimal). 

We can solve \(R_{\mu \nu } \) together with some extra conditions: this means \emph{choosing a gauge}. 
We can do this as long as, starting from a generic \(h_{\mu \nu }\), we can find a transformation \(\epsilon_{\mu }\) which leaves \(\eta_{\mu \nu } \) invariant and for which \(\widetilde{h}_{\mu \nu } = h_{\mu \nu } - \partial_{(\mu } \epsilon_{\nu )}\). 

For example, a condition which cannot be imposed is \(h_{\mu \nu } =0\): if we could impose it, then it would be equivalent to have a wave or not to have it. 
This can be shown from degrees of freedom: we only have 4 degrees of gauge freedom. 

So when we set a condition we must prove that we \emph{can actually} do so. 

Then, any gauge we are actually allowed to impose will give the same results for any physical experiment. 

An electromagnetic analogy: since \(\vec{\nabla} \cdot \vec{B} = 0\), we know that we can locally choose a potential \(\vec{A} \) such that \(\vec{B} = \vec{\nabla} \times \vec{A}\). This has no intrinsic physical meaning, as any \(\vec{A}' = \vec{A} + \vec{\nabla} \lambda \) for a scalar function \(\lambda \) also gives the exact same magnetic field. 
Some gauge fixing choices are \(\vec{\nabla} \cdot \vec{A} = 0\) (Coulomb), \(A_{3} = 0\) (axial). 

We do, however, use potentials since they are convenient. 
Analogously, \(h_{\mu \nu }\) is convenient to use but it is not measurable nor unique. 

\todo[inline]{Is there a gauge-independent tensorial quantity we can define starting from \(h_{\mu \nu }\), a gravitational ``field strength''?}

The last parentheses in the expression for \(R_{\mu \nu } \) is annoying: so we choose the gauge 
%
\begin{align}
\partial_{\lambda } h^{\lambda  }_{\nu } - \frac{1}{2} \partial_{\nu } h = 0
\,,
\end{align}
%
and then in this gauge we have \(R_{\mu \nu } = - \frac{1}{2}  \square h_{\mu \nu }\). 
There are two questions we must ask: can we actually impose this condition, and why is it called harmonic? 
% The answers are yes, and because it gives a harmonic equation \(\square h_{\mu \nu } = 0\). 

Let us prove the first one: 
%
\begin{align}
0 \overset{?}{=} \partial_{\lambda } \widetilde{h}^{\lambda }_{\nu } - \frac{1}{2} \partial_{\nu }\widetilde{h}
&= \partial_{\lambda } \qty( h^{\lambda }_{\nu } - \partial^{\lambda } \epsilon_{\nu } - \cancelto{}{\partial^{\nu } \epsilon_{\lambda }} ) - \frac{1}{2} \partial_{\nu } \qty(h - 2 \cancelto{}{\partial_{\lambda } \epsilon^{\lambda }})  \\
&= \partial_{\lambda } h^{\lambda }_{\nu } - \frac{1}{2} \partial_{\nu } h  - \square \epsilon_{\nu }
\,,
\end{align}
%
which can be solved by setting an \(\epsilon_{\nu }\) such that:
%
\begin{align}
  \square  \epsilon_{\nu } = \partial_{\lambda } h^{\lambda }_{\nu } - \frac{1}{2} \partial_{\nu } h
\,.
\end{align}

Notice that in this gauge \(\square \widetilde{h}_{\mu \nu } = 0 \): the wave equation, for a wave which propagates with velocity 1. 

We have residual gauge freedom! As long as \(\square \widetilde{\epsilon}_{\mu} = 0\), we can still change \(h_{\mu \nu }\) .
So the solution is not unique. The gauge is called harmonic because we can do harmonic residual gauge fixing. 
We have 4 more degrees of freedom: we choose \(h_{0 \mu } \equiv 0\). Now we prove that this can be done. 

We do not prove this in general (although it can be done), because we want to move towards a more specific example. 

We then consider plane wave solutions: waves that only depend on one spatial coordinate, and are constant with respect to both other spatial coordinates. 

If the direction of motion is the \(z\) axis, then the solution looks like 
%
\begin{align}
h_{\mu \nu } (t, x, y, z) = C_{\mu \nu } \exp(-i (k t - k z ))
\,.
\end{align}

For a wave propagating in a generic direction we will have 
%
\begin{align}
h_{\mu \nu } (t, \vec{x}) = C_{\mu \nu } \exp(i k_{\alpha } x^{\alpha })
\,,
\end{align}
%
where \(k^{\alpha } k_{\alpha } = 0\) (which comes from the Fourier expression of the Dalambertian: \(\square = -k^{\alpha } k_{\alpha }\)) is a constant vector. The scalar product in the exponent must be computed with the Minkowski metric since we are working at  first order in \(h\) and \(C_{\mu \nu }\) is already first order.  
%
\begin{align}
\square h_{\mu \nu } &= \eta^{\alpha \beta } \partial_{\alpha } \partial_{\beta } \qty(C_{\mu \nu } \exp(i k_{\alpha }x^{\alpha }))  \\
&= \eta^{\alpha  \beta } C_{\mu \nu } \partial_{\alpha } \qty(\exp(i k_{\alpha } x^{\alpha }) \partial_{\beta } \qty(i k_{\alpha } x^{\alpha }))  \\
&= \eta^{\alpha  \beta } C_{\mu \nu } i k_{ \beta } \partial_{\alpha } \qty(\exp(i k \cdot x )) \\
&= \eta^{\alpha \beta } (i k_{ \beta }) (ik_{\alpha }) C_{\mu \nu } \exp(i k \cdot x) = - k^2 C_{\mu \nu } \exp(i k \cdot x)
\,,
\end{align}
%
then we see that this is always zero as long as \(k^2=0\), which is what we impsed for \(k^{\alpha }\). 

What is the particle analog of a gravitational wave? it is a so-called \emph{graviton}. 
For it we will have a momentum operator \(\vec{p} = - i \hbar \vec{\nabla}\) and an energy operator \(E = i \hbar \partial_{t}\). We can apply both of these to our wave to find that it is an eigenvector of both: 
%
\begin{align}
E h_{\mu \nu } = i \hbar \partial_{t} \qty(C_{\mu \nu } \exp(i k \cdot x))= \hbar k^{t} h_{\mu \nu } 
\,,
\end{align}
%
and we know that \(k^{t} = \abs{\vec{k}}\). 

If we compute the momentum, analogously we find \(\vec{p} h_{\mu \nu } = \hbar \vec{k}\). 

This means that \(p \cdot p  = 0\), which means that the mass of the graviton is equal to zero. 



\end{document}