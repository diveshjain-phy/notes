\documentclass[main.tex]{subfiles}
\begin{document}

\section*{Thu Oct 24 2019}

Last time we looked at the derivative along a curve.

Now, we are going to talk about the

\section{Einstein equations}

They look like 
%
\begin{equation}
  \text{curvature} \propto \text{energy}
\,,
\end{equation}
%
we will make this more formal in this lecture.

We take a vector and parallel transport it along a closed path on a curved manifold, such as a sphere [see homework sheet \#3]: the vector does \emph{not} come back to its original position.
In a flat manifold this is not the case.
Do note that flat manifolds can \emph{look} curved: an infinite cylinder's surface is flat.

Therefore we can \emph{define} a ``curved manifold'' by: 
%
\begin{equation}
  \nabla_{\mu } \nabla_{\nu } V^{\alpha } \neq \nabla_{\nu } \nabla_{\mu } V^{\alpha }
\,,
\end{equation}
%
for at least some directions. To quantify this noncommutativity, let us then look at the commutator: 
%
\begin{subequations}
\begin{align}
  [\nabla_{\mu }, \nabla_{\nu }]V^{\alpha } &= \nabla_{\mu }\qty(\nabla_{\nu } V^{\alpha }) - (\mu \leftrightarrow \nu )  \\
  &= \partial_{\mu } \qty(\nabla_{\nu } V^{\alpha })
  \cancelto{}{- \Gamma_{\mu \nu }^{\lambda } \qty(\nabla_{\lambda }V^{\alpha })}
  + \Gamma_{\mu \lambda }^{\alpha }\qty(\nabla_{\nu} V^{\lambda }) - (\mu \leftrightarrow \nu )  \marginnote{Symmetric in \(\mu \nu \)}\\
  &= \partial_{\mu } \qty(\cancelto{}{\partial_{\nu } V^{\alpha }} + \Gamma_{\nu \lambda }^{\alpha }V^{\lambda })
  + \Gamma_{\mu \lambda }^{\alpha }\qty(\partial_{\nu } V^{\lambda  } + \Gamma_{\nu \sigma  }^{\lambda }V^{\sigma  }) - (\mu \leftrightarrow \nu )  \\
  &= \partial_{\mu } \Gamma_{\nu \sigma  }^{\alpha }V^{\sigma  }
  + \Gamma_{\mu \lambda }^{ \alpha }\Gamma_{\nu \sigma  }^{ \lambda   } V^{\sigma }- (\mu \leftrightarrow \nu )  \\
  &= \qty(
    \partial_{\mu } \Gamma_{\nu \sigma  }^{\alpha }
       - \partial_{\nu } \Gamma_{\mu \sigma  }^{\alpha }
      + \Gamma_{\mu \lambda }^{ \alpha }\Gamma_{\nu \sigma  }^{ \lambda }
      - \Gamma_{\nu \lambda }^{ \alpha }\Gamma_{\mu \sigma  }^{ \lambda }
      )V^{\sigma }  \\
  &\overset{\text{def}}{=} R^{\alpha }_{\sigma  \mu \nu }V^{\sigma }
\,.
\end{align}
\end{subequations}
%

We have cancelled many terms which are symmetric with respect to \((\mu \leftrightarrow \nu )\).
The (3, 1) tensor we defined is called the \emph{Riemann tensor}.

\subsection{Local Inertial Frame}

\begin{claim}
    For any spacetime endowed with a metric and for any point \(P\) in it we can choose a (``prime'') coordinate system for which \(g^{\prime }_{\mu \nu } (x^{\prime }_{p}) = \eta_{\mu \nu }\) and \(\partial_{\lambda }^{\prime } g^{\prime }_{\mu \nu }(x^{\prime }_{p}) = 0\). 

    We cannot, however, set all the second coordinate derivatives to 0.
\end{claim}

We want to compute the Riemann tensor in the LIF. In the LIF, the Christoffel symbols are all zero since they are linear combinations of the coordinate derivatives of the metric. The \emph{derivatives} of the CS, however, are not zero!  
%
\begin{equation}
    \partial_\sigma\qty(\frac{1}{2} g^{\alpha \lambda } \qty(g_{\lambda \mu , \nu }+ g_{\lambda \nu , \mu } - g_{\mu \nu , \lambda })) = 
    \frac{g^{\alpha \lambda }}{2} \qty(g_{\lambda \mu , \nu \sigma } + g_{\lambda \nu , \mu \sigma } - g_{\mu \nu, \lambda \sigma })
\,.
\end{equation}
%

Therefore, we have: 
%
\begin{subequations}
\begin{align}
  R^{\alpha }_{\sigma \mu \nu } &= \frac{g^{\alpha \lambda }}{2} \qty(
      g_{\lambda \nu , \sigma \mu } 
      + \cancelto{0}{g_{\sigma \lambda , \mu \nu }} 
      - g_{\sigma \nu, \lambda \mu }
  ) - (\mu \leftrightarrow \nu)  \\
  &= \frac{g^{\alpha \lambda }}{2} \qty(
    g_{\lambda \nu , \sigma \mu } 
    -g_{\lambda \mu , \sigma \nu } 
    - g_{\sigma \nu, \lambda \mu }
    + g_{\sigma \mu, \lambda \nu }
)
\,.
\end{align}
\end{subequations}
%

We lower an index of the Riemann tensor with the metric and get: 
%
\begin{equation}
  R_{\gamma \sigma \mu \nu } = \frac{1}{2} \qty(
    g_{\gamma \nu , \sigma \mu } 
    -g_{\gamma \mu , \sigma \nu } 
    - g_{\sigma \nu, \gamma \mu }
    + g_{\sigma \mu, \gamma \nu }
  )
\,,
\end{equation}
%
which is a reasonably simple expression, however it is only true at a single point.

We can derive some symmetry properties: 
%
\begin{subequations}
\begin{align}
  R_{\gamma \sigma \mu \nu } &= - R_{\sigma \gamma \mu \nu }   \\
  R_{\gamma \sigma \mu \nu } &= - R_{\gamma \sigma \nu \mu }  \\
  R_{\gamma \sigma \mu \nu } &= R_{\mu \nu \gamma \sigma } \\
  R_{\gamma \sigma \mu \nu } + R_{\gamma \mu \nu \sigma } + R_{\gamma \nu \sigma \mu } &= 0
\,,
\end{align}
\end{subequations}
%
these can be checked in the LIF, and since they are tensorial expressions they will hold in any frame.

\begin{definition}[Ricci tensor and scalar]
The Ricci tensor is the trace of the Riemann tensor:
    \begin{equation}
      R_{\mu \nu } = R^{\alpha }_{\mu \alpha \nu }
    \,,
    \end{equation}
while the Ricci scalar, or scalar curvature, is the trace of the Ricci tensor: 
%
\begin{equation}
  R = g^{\mu \nu }R_{\mu \nu }
\,.
\end{equation}
\end{definition}

\subsection{The stress-energy tensor}

All of the objects we defined are defined locally in the tangent bundle of the manifold.

To discuss energy, we also need a local object: an energy \emph{density}.

This will not be a scalar: it is frame dependent, since it is energy over volume, but volume changes if we change frame.

The number of particles \(N\) is a scalar (not a scalar field!).
The number density is \emph{not} a scalar field: a moving observer with velocity \(v\) will se the volume as being \emph{smaller}. The average number density as measured in LIF will be 
%
\begin{equation}
  n_{*} = \frac{N}{\text{Vol}_{*}}
\,,
\end{equation}
%
where \(\text{Vol}_{*}\) is the spatial volume of the box in its own rest frame. For another observer moving at \(v = 1 - 1/ \gamma^2\), we will have 
%
\begin{equation}
  n = \frac{N}{\text{Vol}} = \gamma n_{*} \geq n_{*}
\,,
\end{equation}
since \(\text{Vol} = \text{Vol}_{*}/ \gamma \).

The 4-velocity of the box for the observer is \(u^{\mu }= (\gamma , \gamma \vec{v})\). Therefore, if we define \(n^{\alpha }= n_{*} u^{\alpha }\) we will have \(n = n^{0}\) for any observer.
It is not a scalar because it cannot be, but the whole density vector transforms as a proper vector.

What are the spatial components of this vector? \(n^{i}\)   is called the \emph{number current density}.

We imagine an area which is fixed with respect to the moving observer. How many particles cross the area \(\dd{A} \) in a time \(\dd{t}\), given that locally near the area the 3-velocity of the particles is \(\vec{v}\)?

It will be the density times the volume: 
%
\begin{equation}
  \frac{n_{*}}{\sqrt{1-v^2} } \dd{t} \vec{v} \cdot \dd{\vec{A}} = n^{i} \dd{A_{i}} 
\,.
\end{equation}
%

We have a scalar product because the area can be at an angle with respect to the velocity, and we need to compute the flux.

As an example, the electromagnetic current for electrons with number density \(n^{\mu }\) is simply \(j^{\mu } = -e n^{\mu }\).

Now, we will discuss the \emph{net flux} in or out of a certain region. If we imagine a certain region, the net flux will equal the variation of the particle number in the region: this gives us a conservation equation 
%
\begin{equation}
  \int _{\partial V} \dd{\vec{A}} \cdot \vec{n} + \partial_{t} \int _{V} \dd[3]{x} n 
  = \int _{V} \dd[3]{x} \qty(\nabla \cdot \vec{n} + \partial_t n) 
  = 0
\,,
\end{equation}
%
where \(V\) is our volume, and its boundary is \(\partial V\). If \(T\) is the time for which we consider the problem, this can be restated by integrating over time as well: 
%
\begin{equation}
    \int _{V \times T} \partial_{\mu } n^{\mu } \dd[4]{x} 
  = 0
\,.
\end{equation}
%

We used the divergence theorem, which states that the flux going out of the boundary of a volume is equal to the integral of the divergence over the volume (for any vector field).

This holds for any volume and for any time: therefore the integrand must be identically null, \(\partial_{\mu }n^{\mu }\equiv 0\).



\end{document}

