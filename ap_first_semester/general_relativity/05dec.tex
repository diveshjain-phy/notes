\documentclass[main.tex]{subfiles}
\begin{document}

\section*{Thu Dec 05 2019}

We have the spin vector \(s^{\mu }\): it is orthogonal to the 4-velocity and 
%
\begin{align}
  \dv{}{\tau } s^{\mu } = 0
\,.
\end{align}

By symmetry we also have \(s^{\theta } =0\). 

By orthogonality to the 4-velocity in the Schwarzschild frame we have 
%
\begin{align}
  s^{t} = \frac{R^2 \Omega }{1 - \frac{2GM}{r}} s^{\varphi }
\,,
\end{align}
%
and for geodesic circular orbit we have \(\Omega^2 R^3 = GM \). 

Let us also evolve \(s^{\varphi }\): 
%
\begin{align}
  \dv{s^{\varphi }}{\tau } + \Gamma^{\varphi }_{\beta \gamma } u^{\beta  } s^{\gamma } = 0
\,,
\end{align}
%
and the nonzero Christoffel symbols which can appear are the \(\Gamma^{3}_{\beta \gamma }\) with \(\beta = 0, 3\) and \(\gamma = 0, 1, 3\). 
The only one which is nonzero is \(\Gamma^{3}_{13}\): 
%
\begin{align}
  \Gamma^{3}_{13} = \frac{1}{2} g^{33} \qty(g_{33, 1})
 = \frac{r^{-2}}{2} \qty(2r) = \frac{1}{r} 
\,.
\end{align}

Then the evolution equation is 
%
\begin{align}
  \dv{s^{\varphi }}{\tau } + \frac{1}{r} \underbrace{u^{t} \Omega}_{ u^{\varphi }} s^{r} = 0  
\,,
\end{align}
%
and as before we can use the fact that \(u^{t} = \dv*{t}{\tau }\) in order to write the equation as 
%
\begin{align}
  \dv{s^{\varphi }}{t} + \frac{\Omega}{r} s^{r} =0 
\,,
\end{align}
%
which is coupled with the equation from before: 
%
\begin{align}
    \dv{s^{r}}{t} + \qty(3GM - r)\Omega s^{\varphi }= 0
\,,
\end{align}
%
which seems to look like a harmonic oscillator: we just need to differentiate one of them, to get 
%
\begin{align}
  \dv[2]{s^{r}}{t} + (3GM-r) \Omega \dv{s^{\varphi }}{t} = \dv[2]{s^{r}}{t} - \frac{3GM-r}{r} \Omega^2 s^{r} = 0
\,,
\end{align}
%
so we found a harmonic oscillator with angular velocity 
%
\begin{align}
  \overline{\Omega} = \Omega \sqrt{1 - \frac{3GM}{r}}
\,,
\end{align}
%
and since \(s^{\varphi }\) satisfies the same equation up to a constant it is also a harmonic oscillator with thee same frequency. 
Its solution will look like 
%
\begin{align}
s^{r} = A \cos(\overline{\Omega} t)
\,,
\end{align}
%
where \(A\) is an arbitrary constant,
we can relate \(s^{\varphi } \) to this solution by differentiating it:
%
\begin{align}
  \dv{s^{r}}{t} =
  - A  \overline{\Omega} \sin(\overline{\Omega} t) = R \Omega \qty(1 - \frac{3GM}{r}) s^{\varphi }
\,,
\end{align}
%
so in the end we have 
%
\begin{align}
  s^{\varphi } = - \frac{A}{r} \frac{\Omega}{\overline{\Omega}} \sin(\overline{\Omega} t)
\,.
\end{align}

Also, we can use 
%
\begin{align}
  s^{t} = r^2 \Omega \qty(1 - \frac{2GM}{r})^{-1} s^{\varphi }
\,.
\end{align}

By the normalization \(s^{\mu } s^{\nu } g_{\mu \nu } = 1\) we have \((s^{1})^2 g_{11} = A^2 \qty(1 - 3GM/r)^{-1}=s_{*}^2 = \const\) at \(t=0\): so 
%
\begin{align}
  A = s_{*} \sqrt{1 - \frac{3GM}{r}}
\,.
\end{align}

What does an observer see? What does somebody at infinity see? 

We call \(\Delta \varphi \) the angle between the spin vector and the radial direction: then 
%
\begin{align}
  \cos \Delta \varphi  = \frac{\text{radial component of the spin vector now}}{\text{radial component of the spin vector at }  t=0}
\,,
\end{align}
%
which means 
%
\begin{align}
  \cos(\Delta \varphi ) = \frac{e_{r} \cdot s (t)}{e_r \cdot s(0)}
\,,
\end{align}
%
where \(s\) is the spin vector and \(e_r\) is the radial unit vector: in Schwarzschild coordinates \(e_{r} = (0, 1/\sqrt{g_{11} } , 0, 0)\). 

This just simplifies to 
%
\begin{align}
  \cos(\Delta \varphi ) = \frac{s^{r}}{A} = \cos(\overline{\Omega}t)
\,,
\end{align}
%
and we are allowed to do this calculation in the Schwarzschild frame since the spatial velocity is always orthogonal to the radial unit vector. 
If we were to use a direction different from the radial one we'd need to boost along it.   

What does this solution mean? In the end our solution is 
%
\begin{subequations}
\begin{align}
  s^{t} &= s_{*} \sqrt{1 - \frac{2GM}{r}} \cos(\overline{\Omega} t)  \\
s^{\varphi } &= - s_{* } \sqrt{1 - \frac{2GM}{r}} \frac{\Omega}{\overline{\Omega} r} \sin(\overline{\Omega} t)  \\
 s^{t} &= r^{2} \Omega \qty(1 - \frac{2GM}{r})^{-1} s^{\varphi }
\,,
\end{align}
\end{subequations}
%
and the solution to \(\cos(\Delta \varphi ) = \cos(\overline{\Omega} t)\) is \(\Delta \varphi = \pm \overline{\Omega} t\), and we choose the solution for continuity with the case \(M=0\). 

If we are rotating around a BH counterclockwise (with \(M \rightarrow 0\)) then the spin must rotate clockwise in order to remain aligned with itself: therefore the right solution to choose is \(\Delta \varphi = - \overline{\Omega} t\). 

This is the reason why there is a minus sign in the expression for \(s^{\varphi } \). 

The basis is rotating, and the spin must rotate the other way to compensate and remain stationary. 

Someone at infinity sees the spin to be rotating with angular velocity \(\Omega - \overline{\Omega}\), where \(\Omega \) is the \(\dv*{\varphi }{t}\) of the orbit and \(\overline{\Omega}\) is the angular velocity of the spin in the coordinate system \(\varphi , r\). 

Over a turn, which takes a time \(t = 2 \pi / \Omega \), the total \(\Delta \varphi \) is given by 
%
\begin{align}
  \frac{2\pi}{\Omega } \qty( \Omega - \overline{\Omega})
  = 2 \pi \qty(1 - \sqrt{1 - \frac{3GM}{r}})
  \approx \frac{3GM\pi}{r} 
\,
\end{align}
%
in the same direction as the orbit. 

Let us look at the gyroscope in a slowly rotating geometry: we consider the metric 
%
\begin{align}
  \dd{s^2} = \dd{s^2}_{\text{schw}} - \frac{4 GJ}{r} \sin^2 \theta \dd{t} \dd{\varphi } + O(J^2)
\,,
\end{align}
%
which (see homework sheet 8) is a vacuum solution to the Einstein equations. We have a 4 since we have to account both for \(g_{03} \) and \(g_{30} \). 

Reinserting  \(c\), we get: 
%
\begin{align}
  \dd{s^2} = \dd{s^2}_{\text{schw}} - \frac{4GJ}{c^3r^2} \sin^2\theta \qty(r \dd{\varphi }) \qty(c \dd{t})
\,,
\end{align}
%
then the quantity \(4GJ / c^2 r^2\) must be adimensional: then the dimensions of \(J\) are those of \(c^3 r^2 / G\), which are 
%
\begin{align}
  \SI{}{m^3 s^{-3}} \times \SI{}{m^2} \times \frac{1}{\SI{}{kg m^3 s^{-2}}} = \SI{}{kg m^2 s^{-1}} 
\,,
\end{align}
%
the dimensions of an angular momentum, \(L = r \wedge p\). 

Essentially we are doing a boost along the \(\varphi \) direction. 

Let us consider geodetic motion of the gyroscope along the rotation axis: the gyroscope is falling into the rotating BH along that axis. 

The spin starts along the \(x\) axis, and we expect a change of the spin of the order 
%
\begin{align}
  \frac{GJ}{c^3r^2}
\,.
\end{align}

We might also have terms of order 
%
\begin{align}
  O \qty(\frac{GJ}{c^3r^2} \times \frac{GM}{rc^2 })
\,,
\end{align}
%
but there cannot be terms of just order \(O(GM/rc^2)\) since if there is no spin, even with positive BH mass the spin does not change. 

\begin{bluebox}
  This is because: the 4-velocity of the infalling observer looks like \(u^{\mu } = (u^{t}, u^{r}, 0,0,)\). The spin in the LIF of the infalling observer only ever has angular components, \(s^{\nu } = (0,0,s^{\theta }, 0)\) (let us neglect the singularity of the coordinates at the \(z\) axis, it is not relevant).
  
  When we perform a boost to go to the Schwarzschild frame from the LIF, the transformation will mix the temporal and radial components of vectors, but it will leave the angular ones unchanged. Therefore, the spin vector will look like \((0,0, s^{\theta }, 0 )\) in the Schwarzschild frame as well.
\end{bluebox}

The BH-mass contribution is small: then the mixed term is ``second order'', we can discard it and consider the \(M=0\) case. 

Therefore we can simply use the metric 
%
\begin{align}
  \dd{s^2} = \eta_{\mu \nu } \dd{x^{\mu }} \dd{x^{\nu }}
  - \frac{4GJ}{r} \sin^2 \theta \dd{t} \dd{\varphi }
\,,
\end{align}
%
and only keep the \(O(J)\) terms. 

\end{document}
