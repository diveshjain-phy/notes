\documentclass[main.tex]{subfiles}
\begin{document}

\section*{Thu Nov 14 2019}

\subsection{Riemann normal coordiates}

We want to actually build a LIF, in which \(g_{\mu \nu } (P ) = \eta_{\mu \nu }\) and \(g_{\mu \nu , \rho } (P)= 0\).

We can choose a set of vectors \(e_{(\mu )} (P)\) which are orthonormal: \(e_{(\mu )} \cdot e_{(\nu )} = \eta_{(\mu )(\nu )}\). 

If we choose this \emph{tetrad} as a basis, then the metric becomes the Minkowski one at that point.

Take a vector \(n^{\mu }\) at \(P\), and consider all possible geodesics which start from \(P\) with initial tangent vector \(n^{\mu }\).

Then, the coordinates of a point \(Q\) we get by moving for a time \(\tau \) along this geodesic are \(x^{\alpha } = \tau n^{\alpha }\) if \(n^{\alpha }\) is timelike, \(x^{\alpha } = s n^{\alpha }\) if it is spacelike, where 
%
\begin{align}
  \tau = \int _{P}^{Q} \dd{\tau } 
\,.
\end{align}

Since these lines are geodesics, they satisfy the geodesic equation: 
%
\begin{align}
  \dv[2]{x^{\alpha }}{\tau } + \Gamma^{\alpha }_{\beta \gamma } \dv{x^{\beta }}{\tau } \dv{x^{\gamma }}{\tau } = 0
\,,
\end{align}
%
but we can insert the relation \(x^{\alpha } = \tau n^{\alpha }\) here, to get \(\Gamma^{\alpha }_{\beta \gamma } n^{\beta } n^{\gamma }=0\): but this can be written for any \(n^{\alpha }\), therefore we immediately get that all the Christoffel symbols vanish: \(\Gamma^{ \alpha }_{\beta \gamma } \equiv 0\). 

Therefore, we have \(4^{3}\) equations of sums of derivatives of the metric which vanish: but the gradient of the metric only has \(4^{3}\) independent components, therefore the solution \(g_{\mu \nu , \alpha } \equiv 0\) is the only one, as long as \(\Gamma^{\alpha }_{\beta \gamma } (g_{\alpha \beta , \gamma })\) is an invertible system, which is nontrivial to show.
This invertibility is equivalent to the system \(2\Gamma_{\alpha \beta , \gamma } (g_{\alpha \beta , \gamma })\) being invertible. 
The following code shows this: 

\begin{lstlisting}[language=Python]
import numpy as np

# linear transformation between
# metric derivatives and Christoffel symbols
M = np.zeros((4**3, 4**3))

# three indices run from 0 to 3:
# we incorporate them into one
# from 0 to 4**3-1
d = lambda i,j,k: 4**2*i+4*j+k

# we populate the matrix with the relevant coefficients,
# starting from the formula for the metric
# in terms of the Christoffel symbols
for i in range(4):
    for j in range(4):
        for k in range(4):
            M[d(i, j, k), d(i, j ,k)] += 1
            M[d(i, j, k), d(i, k, j)] += 1
            M[d(i, j, k), d(k, j, i)] += -1

print(np.linalg.det(M))
\end{lstlisting}

Can the geodesics cross each other far from the starting point? \emph{Yes}, so the Riemann normal coordinates are only defined in a neighbourhood.

Let us consider an example: Riemann normal coordinates around the north pole of a sphere.

We can map every point on the sphere but the south pole by specifying the meridian and the distance to travel along the meridian.

So in our case, if \(R\) is the radius of the sphere, we get \(x^{\alpha } = (R \theta \cos(\phi ), R \theta \sin(\phi ))\), where then \(R \theta = \tau \) and \(n^{\alpha } = (\cos(\theta ), \sin(\theta ))\).
The angles \(\phi \) and \(\theta \) are the usual spherical coordinate angles: \(\phi \) specifies the meridian, while \(R\theta \) gives us the distance travelled away from the north pole (since \(\theta \) is in radians).

The second order expansion of the metric is 
%
\begin{align}
  g_{ij} = \left[\begin{array}{cc}
  1 - \frac{2y^2}{3R^2} & \frac{2xy}{3R^2} \\ 
  \frac{2xy}{3R^2} & 1-\frac{2x^2}{3R^2}
  \end{array}\right]
\,,
\end{align}
%
where \(x^{\alpha } = (x, y)\).

We can see that the metric is \(\delta_{ij}\) at the north pole, and its derivatives are \(g_{ij,k}= 0\) there.

\section{The Schwarzschild solution}

It descrives the geometry outside a stationary, spherically symmetric object which is not rotating and not electrically charged, such as a star, planet or BH.

In general \(\dd{s^2} = -A(r) \dd{t^2} + B(r) \dd{r^2} + C(r)^2 \qty(\dd{\theta^2} + \sin^2 \theta \dd{\phi })\) is our line element.

We can define \(\widetilde{r} = C(r)\), and then express \(A\), \(B\) with respect to to this, and recalling 
%
\begin{align}
    \dd{r^2} = 
  \frac{\dd{\widetilde{r}^2} }{(\dv*{C}{r})^2}
\,,
\end{align}
%
which is what multiplies \(B\), so we redefine \(B(\widetilde{r}) \) as \(B(\widetilde{r}) / (\dv*{C}{r})^2\).

So, we can just relabel \(B\) as this: the expression 
%
\begin{align}
    \dd{s^2} = -A(r) \dd{t^2} + B(r) \dd{r^2} + r^2\qty(\dd{\theta^2} + \sin^2 \theta \dd{\phi })
\,
\end{align}
%
is fully general. So far we have used the hypothesis of stationarity by writing everything only as a function of \(r\).

Recall the inverted Einstein equations: 
%
\begin{align}
  \frac{1}{8 \pi G} \qty(T_{\mu \nu } - \frac{T}{2} g_{\mu \nu }) = R_{\mu \nu }
\,.
\end{align}
%

We want to solve these outside of our source: we look for \emph{vacuum solutions}. Then the equations are just \(R_{\mu \nu } =0\).
A trivial solution to these is \(g_{\mu \nu } = \eta_{\mu \nu }\), but we will show that it is not the only one!
As a matter of fact, we know that the solution to a differential equation is determined by the boundary conditions: in our case, the mass of the object which sits at the origin.
The Minkowski metric respects these boundary conditions if \(M=0\). In general, it does not.

A homework exercise will be to show that, denoting \(\dv*{}{r}\) with a prime: 
%
\begin{subequations}
\begin{align}
    R_{00} &= \frac{A''}{2B} - \frac{A' B'}{4B^2} - \frac{A^{\prime 2}}{4AB} + \frac{A'}{rB}  \\
    R_{11} &= -\frac{A''}{2A} + \frac{A' B'}{4B^2} - \frac{A^{\prime 2}}{4A^2} + \frac{B'}{rB} \\
    R_{22} &= 1 - \frac{1}{B} - \frac{r A' }{2 AB} + \frac{rB'}{2 B^2}  \\
    R_{33} &= \sin^2 \theta  R_{22} \\
    R_{ij} &= 0 \qquad \text{ if } i \neq j
\,.
\end{align}
\end{subequations}

If we compute \(BR_{00} + A R_{11}\) we get many simplifications:
%
\begin{align}
  0 = A'B + AB' = (AB)'
\,,
\end{align}
%
therefore \(AB\) is a constant with respect to \(r\).

Also, we can write \(A' / A = - B' / B\): we substitute it into \(R_{22}\): we get 
%
\begin{align}
  0= 1 - \frac{1}{B} - \frac{r}{2B} \qty(\frac{B'}{B}) + \frac{rB'}{2B^2}
\,,
\end{align}
%
so 
%
\begin{align}
  0 = 1 - \frac{1}{B} + \frac{rB'}{2B^2}
\,,
\end{align}
%
which is first order in terms of \(B\). We can solve it by separating the variables.

We get 
%
\begin{align}
  \frac{\dd{r}}{r} = \frac{ \dd{B}}{B(1-B)}
\,,
\end{align}
%
from which we find that, if at \(r_{*}\) the variable \(B = B_{*}\), then 
%
\begin{align}
    \int _{r_{*}}^{r}\frac{\dd{r}}{r} = 
    \int _{B_{*}}^{B}  \frac{ \dd{B}}{B(1-B)}
\,,
\end{align}
%
where we can rewrite the term 
%
\begin{align}
  \frac{1}{B(1-B)} = \frac{1}{B} + \frac{1}{1-B}
\,.
\end{align}

So, 
%
\begin{align}
  \log \frac{r}{r_{*}} = \log \frac{B}{B_{*}} - \log \frac{1-B}{1-B_{*}}
\,,
\end{align}
%
or, exponentiating: 
%
\begin{align}
  \frac{r}{r_{*}} = \frac{B}{1-B} \frac{1-B_{*}}{B}
\,,
\end{align}
%
from which we can find \(B(r)\): first we write 
%
\begin{align}
  \frac{1-B}{B} = \frac{1-B_{*}}{B} \frac{r_{*}}{r} = \frac{\gamma }{r}
\,,
\end{align}
%
where we collected the integration constants into \(\gamma \).

So, 
%
\begin{align}
  \frac{1}{B} = \frac{\gamma}{r} + 1 \implies B = \qty(1 + \frac{ \gamma }{r})^{-1}
\,.
\end{align}
%
and because of \(AB = k\) we also get 
%
\begin{align}
  A = k \qty(1 - \frac{\gamma}{r})
\,.
\end{align}

We are almost at the full solution: by continuity with the large-\(r\) limit, for which we have the Minkowski metric with \(A = B = 1\), we have that \(k=1\). Now we have 
%
\begin{align}
  \dd{s^2} = - \qty(1 + \frac{\gamma }{r}) \dd{t^2} 
  + \qty(1 + \frac{\gamma}{r}) \dd{r^2}
  + r^2 \qty(\dd{\theta^2} + \sin^2 \theta \dd{\phi^2})
\,,
\end{align}
%
and \(\gamma \) can only be found by continuity with the weak-field approximation: else, every value for it solves the equations. 
We know that \(g_{00} = - (1+2 \Phi )\).
In the weak-field limit we have \(\Phi = -GM/r\), so we identify \(\gamma = - 2GM\).

\subsection{Gravitational redshift}

Now we can do the exact calculation for the gravitational redshift:
Alice, at \(r_A\), sends photons to Bob at \(r_B\). The motion of the photons need not be radial.
Alice and Bob are not following geodesics, we consider them to be stationary in this metric.

We know the metric to be independent of time: \(\xi^{\mu } = (1,\vec{0})\) is a Killing vector field.
Light has momentum \(p^{\mu }\) and moves along a geodesic.
We will use the relation \(p^{\mu } \xi_{\mu } = \const\) to solve our problem.

\end{document}