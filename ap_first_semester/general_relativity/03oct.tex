\documentclass[main.tex]{subfiles}
\begin{document}

\section*{3 October 2019}

Marco Peloso, \url{marco.peloso@pd.infn.it}

\section{Special relativity}

\begin{definition}
    An inertial frame is one in which Newton's laws hold: a free body moves with acceleration \(a^{i} = 0\).
\end{definition}

Newton's first law establishes the \emph{existence} of inertial frames.

\begin{proposition}
    The frames \(O\) and \(O'\) are both inertial frames iff \(O'\) moves with constant velocity wrt \(O\).
\end{proposition}

\begin{proposition}
    Coordinate transformations between inertial frames are Lorentz boosts, which in some coordinate frame can be written as
    %
    \begin{subequations}
    \begin{align}
      t' &= \gamma_v \qty(t - \frac{vx}{c^2})  \\
      x' &= \gamma_v \qty(x - vt)  \\
      y' &= y \\
      z' &= z\,,
    \end{align}
    \end{subequations}
    %
    where \(\gamma_v = 1/ \sqrt{1 - v^2 / c^2} \).
\end{proposition}

If \(v \ll c\), so \(v/c \sim 0\), they simplify to the identity for \(t\), \(y\), \(z\) and \(x' = x -vt\): these are Galilean transformations.

If we have two events, \(x^\mu\) and \(y^\mu\), they occur with some time and space separation \(\Delta x^\mu = x^\mu - y^\mu\). We can compute \(\Delta s^2 = \eta_{\mu\nu} \Delta x^\mu \Delta x^\nu \), where
%
\begin{equation}
  \eta_{\mu \nu} = \diag{-c^2, 1, 1, 1} \,.
\end{equation}

\begin{proposition}
Under Lorentz transformations \(\Delta s^2\) is invariant.
\end{proposition}

We can classify separations between events as
%
\begin{itemize}
    \item time-like when \(\Delta s^2 <0\);
    \item null-like when \(\Delta s^2 =0\);
    \item space-like when \(\Delta s^2 >0\).
\end{itemize}

We can draw spacetime diagrams. A light cone is the set of points which are null-like separated from a select point. Things can be only causally related to events inside the light-cone, with \(\Delta s^2 \leq 0\).

\subsection{Time dilation}

Take two events which occur at the same location for \(O'\). In the primed frame they will have coordinates \(x^{\mu} = (t_0, x_0)\) and \(y^\mu = (t_1, x_0)\).

\begin{definition}
    The \emph{proper time} between these two events is \(t_1 - t_0 \defeq \Delta \tau\).
\end{definition}

We now see that \(\Delta s'\,^2 = -c^2 \Delta \tau^2\). Then, any other observer will see the same  \(\Delta s^2 = - c^2 \Delta t^2 + \Delta x^2 = \Delta s'\,^2\).

This directly implies that \(\Delta \tau \leq \Delta t\) for any observer, since \(\Delta \tau^2 = \Delta t^2 - \Delta x^2 / c^2\). This effect is called \emph{time dilation}.

By how much exacly is time dilated? Of course \(\Delta x = v \Delta t\), therefore \(\Delta t = \gamma_v \Delta \tau\).

This effect explains a peculiar phenomenons: certain particles in the upper atmosphere decay into muons, which have a very short half-life. So short, in fact, that if we did not account for special relativity we'd expect to see next to none at the surface, since by the time they got here they would have alredy gone through several halving times. 
However, we must apply the rule of relativistic time dilation: the muons are travelling very fast towards the ground, therefore in the ground's frame of reference their time passes slower, allowing them to decay slower. 
So, a significant fraction of them arrives at the ground. 

Inverse Lorentz transformation have the same expression as direct ones, but with \(v \rightarrow -v\).
This can be proved both mathematically by solving the equations and phisically by reasoning about their meaning. There is no preferential inertial frame.

A Lorentz transformation can be written in matrix form in the \((ct, x)\) plane as:

\begin{equation}
  \Lambda = \begin{bmatrix}
    \gamma & -\gamma \beta \\
    -\gamma \beta & \gamma
  \end{bmatrix}
  = \begin{bmatrix}
  \cosh \theta & -\sinh \theta \\
    -\sinh \theta &  \cosh \theta
  \end{bmatrix}
\end{equation}
%
where we introduced the notation \(\beta = v / c\).

The second equation is justified by the fact that there is an angle \(\theta\) such that \(\gamma = \cosh \theta\) and \(\gamma \beta = \sinh \theta\): the angle \(\theta\) will be \(\theta = \tanh^{-1} \qty(v/c)\). 
This is true because \(\gamma^2 - \beta^2 \gamma^2 = 1\), which is the same law that the hyperbolic functions obey: \(\cosh^2 x - \sinh^2 x =1 \) holds for any \(x\).

After a boost the \(ct'\) and \(x'\) axes are rotated into, respectively, the lines \(ct=x/\beta\) and \(ct = \beta x\): this comes directly from the transformation law. 
The \(ct'\) axis is defined by the equation \(x' =0\), which in the transformed coordinates \(ct\) and \(x\) reads \(\gamma (x - \beta ct) = 0\), or \(ct  = x/ \beta \).

Similarly \(ct' = 0\) in the new coordinates reads \(\gamma (ct - \beta x)=0\), or  \(ct = \beta x\). 

The axes are rotated by an angle which can approach \(\pi /4\) but never reach it, since its tangent is defined by \(\beta \), which can never reach 1. 

\end{document}
