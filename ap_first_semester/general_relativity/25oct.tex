\documentclass[main.tex]{subfiles}
\begin{document}

\section*{Fri Oct 25 2019}

We define \(\Delta N = N^{\mu }n_{\mu } \Delta V\), where \(N^{\mu }\) is the current density from last time while \(n^{\mu } \) is the normal to the surface of a 3-volume.

Now, we define the energy-momentum-stress tensor \(T^{\mu \nu }\).

We define it by \(\Delta p^{\alpha } = T^{\alpha \beta }n_{\beta }\Delta V\), where \(\Delta p^{\mu}\) is the momentum in the volume.

Let us consider an inertial frame in which \(\Delta V\) is at rest, and take \(n_{\mu }\) to be its 4-velocity.

Then the energy density is given by: 
%
\begin{equation}
  \epsilon  = \frac{\Delta p^{0}}{\Delta V} = T^{00}
\,,
\end{equation}
%
while the momentum density is: 
%
\begin{equation}
  \Pi^{i} = \frac{\Delta p^{i}}{\Delta V} = T^{i0}
\,.
\end{equation}

Now let us consider a frame moving with velocity \(\vec{v}  \), in this frame each particle has energy \(m \gamma \) and momentum \(m \gamma \vec{v}\), and the energy density is \(n_{*} \gamma \).

The moving observer sees exacly the energy density \(\epsilon = m \gamma n_{*} \gamma  = T^{00}\), and the momentum density \(T^{i0} = m \gamma \vec{v}^{i} n_{*}\gamma \).

In general, for particles which do not interact with each other, we have 
%
\begin{equation}
  T^{\alpha \beta } = n_{*} m u^{\alpha }u^{\beta }
\,.
\end{equation}

The stress energy tensor is in general symmetric.

What is \(T^{0i}\)? We have \(\Delta p^{\alpha } = T^{\alpha 1} \Delta y \Delta z \Delta t\) (if we select the normal \(n_{\mu }\) parallel to the \(x\) axis).

For \(\alpha = 0\), this is the flux of energy in time (power) along the \(x\) direction.

For \(\alpha = i\), we can use the same relation: 
%
\begin{equation}
  T^{i1} = \frac{\Delta p^{i}}{\Delta y \Delta z \Delta t}
  = \frac{F^{i}}{\Delta y \Delta z} = \frac{F^{i}}{\text{Area}}
\,.
\end{equation}

Do note that \(i\) can be either \(x\), \(y\) or \(z\): we can consider the \emph{pressure}, which is the force along the \(x\) axis across the \(x\) axis, but also the \emph{deviatoric stresses} along the \(y\) or \(z\) axes but across the \(x\) axis.

Do also note that the stress tensor from fluid dynamics, \(\sigma^{ij}\), is not equal to \(T^{ij}\): it is its opposite.

The energy density measured by an observer with 4-velocity \(u^{\mu }\) is \(T_{\mu \nu } u^{\mu }u^{\nu }\).

\begin{claim}
Energy momentum is conserved.
\end{claim}

\begin{proof}
In the LIF \(\partial_{\beta }T^{\alpha \beta }= 0\) (in perfect analogy to the current number density) therefore in any frame \(\nabla_{\beta }T^{\alpha \beta }= 0\).
\end{proof}

\todo[inline]{Is this true even in the absence of translational symmetry?}

\begin{definition}[Perfect fluid]
A fluid is perfect if it has no dissipative effects: heat conduction, viscosity.
\end{definition}

\begin{claim}
The stress-energy tensor of a perfect fluid is 
%
\begin{equation}
  T^{\alpha \beta } = \diag{\rho , p, p, p }
\,.
\end{equation}
%
in its own rest frame; in any frame \(T^{\alpha \beta } = \rho u^{\alpha } u^{\beta } + P h^{\alpha \beta }\),  where \(h^{\mu \nu } = g^{ \mu \nu }+u^{\mu } u^{\nu }\).
\end{claim}

\begin{definition}[Equation of state]
It is \(P = \omega \rho \).
\end{definition}

\subsection{The Einstein equations}

We require them to be \emph{mathematically consistent} and \emph{physically correct}.

We have the principle of covariance: this tells us that the law should be a tensorial expression.

We know that energy is described by the tensor \(T_{\mu \nu }\), while for the curvature part we have the various contractions of the metric and the Riemann tensor.

Our ansatz is 
%
\begin{equation}
  R_{\mu \nu } + c_1 R g_{\mu \nu } = c_2 T_{\mu \nu }
\,.
\end{equation}

Symmetry is all right: \(R_{\mu \nu }\) is symmetric, and so is the metric.
The stress energy tensor is also conserved: therefore we impose 
%
\begin{equation}
  \nabla^{\mu } \qty(R_{\mu \nu } + c_1 R g_{\mu \nu })= 0
\,.
\end{equation}

By the contracted Bianchi identies, this implies \(c_1 = -1/2\).

To get \(c_2 \), we can look at the Newtonian limit: they should simplify to \(\square \phi \propto \rho \). This will imply \(c_2 = 8 \pi G \).

\subsection{The Newtonian limit}

We start by tracing the equations: we get \(R-2R = c_2 T \), where \(g^{\mu \nu }T_{\mu \nu }= T\).
Note that \(g^{ \mu \nu }g_{ \mu \nu }=4\): it is not the trace of the metric, but its Frobenius norm.

We can then rewrite the EFE as 
%
\begin{equation}
  R_{\mu \nu } = c_2 \qty(T_{\mu \nu } - \frac{T}{2}g_{\mu \nu })
\,.
\end{equation}
%

Let us consider a slowly moving perfect fluid: \(p \ll \rho \), and \(v \approx  0\), so \(u^{\mu } = (1, \vec{0})\).

We consider a weak gravitational field: \(g_{\mu \nu } \approx \eta_{\mu \nu } \).

Our stress energy tensor is approximately \(\rho u^{\mu } u^{\nu }\). We have \(T = - \rho \).

Then, we will get: 
%
\begin{equation}
  R_{00} \approx c_2 \qty(\rho - \frac{\rho}{2} (-)^2)
\,.
\end{equation}

Therefore we get \(R_{00} \approx c_2 \rho /2\).

\end{document}
