\documentclass[main.tex]{subfiles}
\begin{document}

\section*{Fri Nov 29 2019}

Yesterday we talked about Kruskal diagrams and coordinates. 

There is a 1-to-1 map between \(U^2 - V^2\) and \(R = r/2GM\): 
%
\begin{align}
  U^2 - V^2 = \qty(R-1) \exp(R)
\,,
\end{align}
%
and if we fix \(V = V_0 \), we can vary \(U, \theta \): each point in the manifold parametrized by \(U, \theta \) is a circle of radius \(r(U^2-V^2) \sin \theta \), parametrized by the coordinate we left out, \(\varphi \). 

What if we vary \(V_0 \)? there opens up a ``breach'', but it is only ever spacelike: no timelike curves can cross it. 

White holes would have existed ``before'', but they appear as solutions to a stationary problem: real black holes form from the collapse of a star, there is a time after which the BH exists and before which it does not. 

There is some more material at the end of the lecture notes on geodesics. 

An observer in geodesic motion feels no acceleration: \(a^{\mu } = 0\). 

What is the acceleration felt by an observer, \(a^{i}\), moving with \(4\)-acceleration \(a^{\mu }\) in general (non-geodesic) motion? 

A noncontroversial fact is \(-1 = \const\): this directly implies \(u^{\mu } a_{\mu } = 0\), since the metric is covariantly constant and \(u^{\mu } u_{\mu } = 0\). 

The acceleration felt by \(A\) is his acceleration measured in a LIF defined as having the same \(4\)-velocity as \(A\). There, 
%
\begin{align}
  a^{\mu } = u^{\nu } \nabla_{\nu } u^{\mu } = \dv{ u^{\mu }}{\tau } + \cancelto{}{\Gamma^{\mu }_{\alpha \beta } u^{\alpha } u^{\beta }}
\,,
\end{align}
%
but \(t = \tau \), and the 4-velocity in the LIF is \(u^{\mu } = (1, 0,0,0)\): this implies that 
%
\begin{align}
  a^{ \mu } = \dv{ u^{\mu }}{t} = \left[\begin{array}{cc}
  0 & a^{i}
  \end{array}\right]^{\top}
\,,
\end{align}
%
and the invariant quantity \(a^{\mu } a_{\mu } = \abs{\vec{a}}^{2}\) is the modulus of the 3-acceleration felt. 
Then, we are able to compute the norm of \(a^{i}\) in any frame using \(\sqrt{a^{\mu } a_{\mu }}\). 

Let us see this in some examples. 

The uniformly accelerating observer in Minkowski spacetime has a law of motion which looks like 
%
\begin{align}
  x(t) = \frac{\sqrt{1 + \kappa^2 t^2   }}{\kappa }
\,,
\end{align}
%
and we showed that \(t = \sinh (\kappa \tau ) / \kappa \) and \(x = \cosh( \kappa \tau ) / \kappa \). 

The three-velocity is 
%
\begin{subequations}
\begin{align}
  u^{\mu } = \dv{x^{\mu }}{\tau } = \left[\begin{array}{c}
  \cosh(\kappa t) \\ 
  \sinh(\kappa t)
  \end{array}\right]
\,,
\end{align}
\end{subequations}
%
with the metric \(\eta_{\mu \nu } = \left[\begin{array}{cc}
-1 & 0 \\ 
0 & 1
\end{array}\right]\), so we find \(u^{ \mu } u_{\mu } = -1\); also 
%
\begin{subequations}
\begin{align}
  a^{\mu } = \left[\begin{array}{c}
  \kappa \sinh(\kappa \tau ) \\ 
  \kappa \cosh(\kappa \tau )
  \end{array}\right]
\,,
\end{align}
\end{subequations}
%
which means \(a \cdot u = 0\). 
Also, we compute \(a \cdot a = \kappa^2\). 

We also consider an observer at rest in the Schwarzschild geometry: \(x^{\mu } = (t, r_0 , \theta_0 , \varphi_0 )\). The 4-velocity is \(u^{\mu } = \qty(1/\sqrt{-g_{00} }, 0, 0, 0)\). 

The 4-acceleration is given by 
%
\begin{align}
  a^{\mu } = u^{\nu } \qty( \partial_{\nu } u^{\mu } + \Gamma^{\mu }_{\nu \beta } u^{\nu } u^{\beta })
  = \Gamma^{\mu }_{tt} \qty(u^{0})^2
\,,
\end{align}
%
and also \(\Gamma^{\mu }_{tt} = 0\) for any \(\mu \neq r\), while 
%
\begin{align}
    \Gamma^{r}_{tt} =  \frac{1}{2} \qty(1 - \frac{2GM}{r}   ) (-1) \pdv{}{r} \qty(-1 + \frac{2GM}{r})
    = \frac{1}{2} \qty(1 - \frac{2GM}{r}) \frac{2GM}{r^2}
\,.
\end{align}

So, 
%
\begin{align}
  a^{r} = \qty(1 - \frac{2GM}{r}) \frac{GM}{r^2} \qty(1 - \frac{2GM}{r})^{-1} = \frac{GM}{r^2}
\,.
\end{align}

We have recovered the 4-acceleration felt by an observer: its modulus is 
%
\begin{align}
  \abs{\vec{a}} = \sqrt{g_{rr} a^{r} a^{r}} = \frac{GM}{r^2} \qty(1 - \frac{2GM}{r})^{-1/2} \overset{r \gg 2GM}{\rightarrow} \frac{GM}{r^2}
\,,
\end{align}
% 
while if we are at \(r = 2GM + \epsilon \), the acceleration looks like 
%
\begin{align}
  \abs{\vec{a}} \sim \frac{1}{\sqrt{ \epsilon }}
\,.
\end{align}

\section{Rotation and the Kerr solution}

The object \(u^{\nu } \nabla_{\nu } u^{\mu  } = a^{\mu } \) describes only the variation of the velocity \emph{along} the four-momentum. 

Let us consider a gyroscope: an object which has net angular momentum. 

In the rest frame of the object, we introduce the spin four-vector \(s^{\mu } = (0, \vec{s})\). 
In the rest frame \(s \cdot u = 0\), so this holds in any frame. 

A free object in Minkowski spacetime in its own rest frame has a constant \(\vec{s}\): so \(\dv{s^{\mu }}{t} = 0\). 
So in a LIF, for a moving object, 
%
\begin{align}
  \dv{s^{\mu }}{\tau } = u^{\nu } \partial_{\nu } s^{\mu } = 0
\,,
\end{align}
%
so in a general frame 
%
\begin{align}
  u^{\nu } \nabla_{\nu } s^{\mu } = 0
\,.
\end{align}

We can also do the following computation: \( u^{\nu } \nabla_{\nu } \qty(s \cdot s )= 2 s_{\mu } u^{\nu } \nabla_{\nu } s^{\mu } = 0\). 

Also, the product of two different spins is conserved by the same reasoning. 

Now we consider three specific examples: first, a gyroscope moving around a Schwarzschild mass. 
This effect is called \emph{geodetic precession}. 
The other effect we will treat is a gyroscope in a slowly rotating geometry: we will use the gyroscope as a probe for the rotating geometry.
This effect is called \emph{frame dragging} or \emph{Lense Thirring precession}. 
Then, we will discuss a generic rotating black hole: the \emph{Kerr} metric. 

\subsection{Geodesic precession}

We consider a slice of Schwarzschild geometry: we will have \(r , \varphi \) coordinates for space and \(t\) coordinate for time, while \(\theta = \pi / 2\). 

We will consider an observer moving along a geodesic orbit. 

The 4-velocity is given by 
%
\begin{align}
  u^{\mu } = \dv{t}{\tau } \qty(1, 0, 0, \dv{\varphi }{t})
\,,
\end{align}
%
and 
%
\begin{align}
  \dv{\varphi }{t} = \Omega = \sqrt{\frac{GM}{r^3}}
\,,
\end{align}
%
so we can write 
%
\begin{align}
-1 = u^{\mu } u_{\mu }  =  - \qty(u^{t})^2 \qty(g_{00} - r^2\Omega^2) 
 = - \qty(u^{t})^2 \qty(1 - \frac{2GM}{r} - \frac{GM}{r})
\,,
\end{align}
%
so we have 
%
\begin{align}
  u^{t} = \frac{1}{\sqrt{1 - \frac{3GM}{r}}}
\,.
\end{align}
%

Initially we have a vector \(s^{\mu } = \qty(s^{t}, s^{r }, s^{\theta }, s^{\varphi })\). \(s^{\theta }\) starts out zero in our coordinates, and it remains so by symmetry. 

We know that \(0= g_{\mu \nu } s^{\mu } u^{\nu }\), but the metric is diagonal so this is 
%
\begin{align}
  0= g_{00} s^{t} u^{t} + g_{33 } s^{\varphi } u^{\varphi }
\,,
\end{align}
%
which is 
%
\begin{align}
  0= u^{t} \qty(-\qty(1 - \frac{2GM}{r} ) s^{t} + r^2 s^{\varphi } \Omega )
\,,
\end{align}
%
so 
%
\begin{align}
  s^{t} = \qty(1 - \frac{2GM}{r})^{-1} r^2 \Omega s^{\varphi }
\,.
\end{align}

Consistently with our assumptions, as \(r \rightarrow \infty \) we have \(s^{t} \rightarrow 0\). 

The evolution of the spin is described by 
%
\begin{align}
  \dv{s^{\alpha }}{\tau } + \Gamma^{\alpha }_{\beta \gamma } u^{\beta } s^{\gamma } =0 
\,.
\end{align}

One can start by \(\alpha = r\): then we must look at the symbols \(\Gamma^{r}_{\beta \gamma }\): potentially, by the nonzero components of \(u\) and \(s\), we should compute the symbols with \(\beta = 0, 3 \) and \(\gamma = 0, 1, 3\). 

Since the metric is diagonal, the Christoffel symbols with three different indices are automatically zero.
Also, if two components are the same and one is different the different one has to be the one with respect to which one computes the derivatives. 

So our equation is 
%
\begin{align}
  \dv{s^{1}}{\tau } + \Gamma^{1}_{00} u^{0} s^{0} 
  + \Gamma^{1}_{33} u^{3} s^{3} = 0
\,,
\end{align}
%
and the relavant Christoffel symbols are: 
%
\begin{align}
  \Gamma^{1}_{00} = \qty(1 - \frac{2GM}{r}) \frac{GM}{r^2}
\,
\end{align}
%
and 
%
\begin{align}
  \Gamma^{1}_{33} = - \qty(1 - \frac{2GM}{r})r
\,.
\end{align}

We find that the equation is 
%
\begin{align}
  \dv{s^{1}}{\tau } + \Gamma^{1}_{00} \dv{t}{\tau } s^{t}
  + \Gamma^{1}_{33} \dv{t}{\tau } \Omega s^{\varphi } = 0
\,,
\end{align}
%
which becomes, changing variable in the derivative from \(\tau \) to \(t\) everywhere: 
%
\begin{align}
\dv{s^{1}}{t} + \Gamma^{1}_{00}  s^{t}
+ \Gamma^{1}_{33}  \Omega s^{\varphi } = 0
  \,,
\end{align}
%
which means 
%
\begin{align}
  \dv{s^{r}}{t} + \qty(3GM - r)\Omega s^{\varphi }= 0
\,.
\end{align}


\end{document}