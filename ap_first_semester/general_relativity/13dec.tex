\documentclass[main.tex]{subfiles}
\begin{document}

\section*{Fri Dec 13 2019}

There are two topics left: cosmology and gravitational waves. 
Today and Thursday we do cosmology. 

Now we introduce the Planck scale: the Compton wavelength is defined by  
%
\begin{align}
  \lambda = \frac{\hbar c }{E}
\,,
\end{align}
%
while the Schwarzschild radius is of the order of 
%
\begin{align}
  r_s \sim \frac{GM}{c^2} = \frac{GE}{c^{4}}
\,.
\end{align}

So, we have a quantum gravity regime when \(\lambda < r_S\): when the particle is so localized that it will form a BH by itself. We get: 
%
\begin{align}
  \frac{\hbar c}{E} \approx \frac{GE}{c^{4}} \implies
  E \approx \sqrt{\frac{\hbar c^{5}}{G}} \approx \SI{1.96e9}{J} \approx \SI{1.22e19}{GeV}
\,.
\end{align}

In \(c=1\) units, this is also the Planck mass.
It is also useful to define the reduced Planck mass:
%
\begin{align}
  M_p = \frac{E_p}{\sqrt{8 \pi }} = \SI{2.43e18}{GeV}
\,.
\end{align}

Natural units are ones in which 
\begin{enumerate}
    \item \(c=1\): velocities are dimensionless: then the unit of length is equal to the unit of time; 
    \item \(\hbar = 1\): then angular momenta are also dimensionless: then length and time have the dimensions of \(1/\text{mass}\) or \(1/\text{energy}\). 
\end{enumerate}

In natural units the Einstein equations look like: 
%
\begin{align}
  G_{\mu \nu } = 8 \pi G T_{\mu \nu }
\,,
\end{align}
%
but \(M_P = \frac{1}{\sqrt{8 \pi G}}\): therefore \(8 \pi G = 1 / M_P^2\), so 
%
\begin{align}
  G_{\mu \nu } = \frac{1}{M_P^2} T_{\mu \nu }
\,.
\end{align}

We are going to discuss the Friedmann-Lemaître-Robertson-Walker metric, which describes a homogeneous and isotropic universe. 

Homogeneous means symmetry with respect to translations, isotropic means symmetry with respect to rotations. 

Something which is \emph{homogeneous but not isotropic} is, for example, the inside of a capacitor. Also, the surface of a cylinder can be an example. 

We can have a space which is \emph{isotropic but not homogeneous} only around one point, \emph{global isotropy implies homogeneity}. 

We will also require the condition that the universe be \emph{spatially flat}. 
If a triangle has angles \(\alpha , \beta , \gamma \) then \(\sign ( \alpha + \beta +\gamma -\pi ) = k\) is a constant along the space and it measures the curvature of the space. 

The smaller the curvature (and the larger the length scale of the curvature) the more difficult it is to measure what \(k\) is. 

The line element in this kind of space is particularly simple, since we can : 
%
\begin{align}
  \dd{s^2} = - \dd{t^2} + a^2(t) \dd{\vec{x}^2}
\,.
\end{align}

Since the metric is diagonal, the only nonvanishing Christoffel symbols are \(\Gamma^{0}_{ij}\) and \(\Gamma^{i}_{0j}\), and both must be proportional to \(\delta_{ij}\). 
These symbols are
%
\begin{align}
  \Gamma^{0}_{ij} = \frac{1}{2} g^{00} \qty(-g_{ij,0}) = \delta_{ij} a \dot{a}
\,,
\end{align}
%
where \(\dot{a}\) denotes the derivative of \(a\) with respect to coordinate time, while 
%
\begin{align}
  \Gamma^{i}_{0j} = \frac{1}{2} g^{ik} \qty(g_{jk,0}) = \delta_{ij} a \dot{a} \times \frac{1}{a^2} = \delta_{ij} \frac{\dot{a}}{a}
\,.
\end{align}

The trace of the Christoffels are \(\Gamma^{0}_{ii} = 3a \dot{a}\), and \(\Gamma^{i}_{0i} = 3 \dot{a} / a\). We want to calculate \(R_{00} \), \(R_{0i}\) and \(R_{ij}\). 
We have \(R_{0i} = 0\) since nothing on the indices depends on space. We also have \(R_{ij} \propto \delta_{ij}\), since all the Christoffels depend only on \(\delta_{ij}\). 

We have 
%
\begin{align}
  R_{\mu \nu } = \partial_{\alpha} \Gamma^{\alpha }_{ \mu \nu } + \Gamma^{\lambda }_{\mu \nu } \Gamma^{\alpha }_{\lambda \alpha } - \partial_{\nu }\Gamma^{\alpha }_{\mu \alpha } - \Gamma^{\lambda }_{\mu \alpha } \Gamma^{\alpha }_{\nu \lambda }
\,.
\end{align}

Specializing to the \(R_{00} \) case: 
\begin{align}
    R_{0 0 } = \partial_{\alpha} \Gamma^{\alpha }_{ 0 0 } + \Gamma^{\lambda }_{0 0 } \Gamma^{\alpha }_{\lambda \alpha } - \partial_{0 }\Gamma^{\alpha }_{0 \alpha } - \Gamma^{\lambda }_{0 \alpha } \Gamma^{\alpha }_{0 \lambda }
  \,.
\end{align}

The \(\Gamma^{\alpha }_{00}\) must vanish. So we get 
%
\begin{subequations}
\begin{align}
  R_{00} &= - \partial_{0} \Gamma^{i}_{0i} - \Gamma^{i}_{0j} \Gamma^{j}_{0i}  \\
  &= - \partial_{0} \qty(\frac{3 \dot{a}}{a}) - \frac{\dot{a}}{a} \delta_{ij} \frac{\dot{a}}{a} \delta_{ij}  \\
  &= - \frac{3 \ddot{a}}{a} + \frac{3 \dot{a}^2}{a^2} 
  - 3 \frac{\dot{a}^2}{a^2} = - 3 \frac{\ddot{a}}{a}
\,.
\end{align}
\end{subequations}

On the other hand we have 
%
\begin{subequations}
\begin{align}
R_{i j } &= \partial_{\alpha} \Gamma^{\alpha }_{ i j } + \Gamma^{\lambda }_{i j } \Gamma^{\alpha }_{\lambda \alpha } - \cancelto{}{\partial_{j }\Gamma^{\alpha }_{i \alpha }} - \Gamma^{\lambda }_{i \alpha } \Gamma^{\alpha }_{j \lambda }  \\
&= \partial_{0} \Gamma^{0 }_{ i j } 
+ \Gamma^{\lambda }_{i j } \Gamma^{\alpha }_{\lambda \alpha }
- \Gamma^{\lambda }_{i \alpha } \Gamma^{\alpha }_{j \lambda }
\\
&= \qty(a \ddot{a} - \dot{a}^2)  \delta_{ij}
+ 3 a^2 \delta_{ij} - \dot{a}^2 \delta_{ij} - \dot{a}^2 \delta_{ij}  \\
&= a^2 \delta_{ij} \qty(\frac{\ddot{a}}{a} + 2 \frac{\dot{a}^2}{a^2})
\,.
\end{align}
\end{subequations}

So we have the whole of the Ricci tensor. 
The scalar curvature is given by 
%
\begin{subequations}
\begin{align}
  R &= g^{00} R_{00} + g^{ij} R_{ij}  \\
  &= + 3 \frac{\ddot{a}}{a} + \delta_{ij} \frac{a^2}{a^2} \delta_{ij}  \qty( 2 \frac{\dot{a}^2}{a^2} + \frac{\dot{a}}{a})  \\
  &= 6 \frac{\dot{a}^2}{a^2} + \frac{6 \ddot{a}}{a}
\,,
\end{align}
\end{subequations}
%
and we have 
%
\begin{align}
  G_{00 } = R_{00} - R g_{00} = - 3\frac{\ddot{a}}{a}
  + 3 \frac{\dot{a}^2}{a^2} + 3 \frac{\ddot{a}}{a} = \frac{3 \dot{a}^2}{a^2}
\,,
\end{align}
%
while 
%
\begin{subequations}
\begin{align}
  G_{ij} &= R_{ij} - Rg_{ij} \\
  &= a^2 \delta_{ij} \qty(\frac{\ddot{a}}{a} + 2 \frac{\dot{a}^2}{a^2}) - \frac{3 \dot{a}^2}{a^2} a^2  \\
  &= a^2 \delta_{ij} \qty(- \frac{\dot{a}^2}{a^2} - \frac{2 \ddot{a}}{a})
\,.
\end{align}
\end{subequations}

For the SEM tensor we choose a perfect fluid: \(T^{\mu \nu } = \rho u^{\mu } u^{\nu }  + p h^{ \mu \nu }\), where \(h^{\mu \nu } = u^{ \mu }  u^{\nu } + g^{ \mu \nu }\) is the projector on the space orthogonal to the velocity. 

We know that, in the expression of 
%
\begin{align}
  u^{\mu } = \qty(\dv{t}{\tau }, \dv{\vec{x}}{\tau })^{\top}
\,,
\end{align}
%
the compontent \(u^{0}\) must be positive. We have 
%
\begin{align}
  0 = G_{0i} = \frac{T_{0i} }{M_P^2}
\,,
\end{align}
%
but this means that \(u^{i}\) must be zero. \emph{The cosmic fluid is at rest}. This means that we are selecting a special frame: the rest frame of the cosmic fluid, the rest frame of the CMB. 

When we look at the CMB we see a large dipolar contribution, due to the motion of the Earth with respect to the cosmic fluid. 
The theory is globally Lorenz-invariant, however its realization is not. This means that \(u^{0} = 1\), since \(g_{00} = -1\). 

This means that \(h^{\mu \nu } = a^2 \delta^{ij}\) (informal, I mean that the only nonzero components are the spatial ones.)

Then \(T_{00} = \rho \), and \(T_{ij} = a^2 \delta_{ij} P\). 

So the EFE are: 
%
\begin{subequations}
\begin{align}
  \frac{3 \dot{a}^2}{a^2} &= \frac{\rho }{M_P^2}  & \text{00 equation}\\
  - 2 \frac{\ddot{a}}{a} - \frac{\dot{a}^2}{a^2} &=
  \frac{P}{M_P^2} & \text{ij equations}
\,,
\end{align}
\end{subequations}
%
where we factored out the \(a^2 \delta_{ij}\) in the \(ij\) equations. 
``Just for fun'' we discuss the Bianchi identities: \(\nabla_{\mu } G^{\mu \nu }=0\). 
If \(\nu =0\) we have 
%
\begin{align}
  \partial_{\mu } G^{\mu 0} + \Gamma^{\mu }_{\mu \lambda } G^{\lambda 0} + \Gamma^{0}_{\mu \lambda } G^{\mu \lambda } = 0
\,,
\end{align}
%
but simplifying the indices we get 
%
\begin{align}
  \partial_{0} G^{00} + \Gamma^{i}_{i0} G^{00} + \Gamma^{0}_{ij} G^{ij} =0
\,,
\end{align}
%
and substituting in the expressions we have we get 
%
\begin{subequations}
\begin{align}
  &\partial_{0} \qty(\frac{3 \dot{a}^2}{a^2})
  + \frac{3 \dot{a}}{a} \frac{3 \dot{a}^2}{a^2}
  + a \dot{a} \delta_{ij} \frac{1}{a^2} \delta_{ij}
  \qty(- \frac{2 \ddot{a}}{a} - \frac{\dot{a}^2}{a^2}) \\
  = & 6 \frac{\dot{a}}{a} \qty(\frac{\ddot{a}}{a} - \frac{\dot{a}^2}{a^2})
  + 9 \frac{\dot{a}^3}{a^3} - 6 \frac{\dot{a}}{a} \frac{\ddot{a}}{a} - 3 \frac{\ddot{a}^3}{a^3} = 0 
\,,
\end{align}
\end{subequations}
%
which confirms what we already knew. Verifying \(\nabla_{\mu} G^{\mu i } = 0\) is easier:  
%
\begin{align}
    \partial_{\mu } G^{\mu i} + \Gamma^{\mu }_{\mu \lambda } G^{\lambda i} + \Gamma^{i}_{\mu \lambda } G^{\mu \lambda } = 0
\,,
\end{align}
%
because all three of the terms vanish immediately. 

Correspongingly, we have \(\nabla_{\mu } T^{\mu \nu } =0 \). 
This is a local conservation law, not a global one generally since we do not have 4 Killing vectors. 
If \(\nu =0\) we have 
%
\begin{subequations}
\begin{align}
  &\partial_{\mu } T^{\mu 0} + \Gamma^{\mu }_{\lambda \lambda  } T^{\lambda 0} + \Gamma^{0}_{\mu \lambda } T^{\mu \lambda } = \\
  =& \partial_{0} T^{00} +
  \Gamma^{i}_{i0} T^{i0} + \Gamma^{0}_{ij} T^{ij} \\
  =& \partial_{0} \rho 
+ 3 \frac{\dot{a}}{a} \rho + \dot{a} a \delta_{ij} \frac{1}{a^2} \delta_{ij} P 
\,,
\end{align}
\end{subequations}
%
which means 
%
\begin{align}
  \dot{\rho} + 3 \frac{\dot{a}}{a} ( \rho +P) = 0
\,,
\end{align}
%
which we can add to the other equations. 

It is easier to study the case \(\nu = i\): we get 
%
\begin{align}
    &\partial_{\mu } T^{\mu i} + \Gamma^{\mu }_{\lambda \lambda  } T^{\lambda i} + \Gamma^{i}_{\mu \lambda } T^{\mu \lambda } = 0
\,,
\end{align}
%
since all three terms vanish immediately. Since the conservation  equation \(\nabla_{\mu } T^{\mu 0}\) comes from the Einstein equations, we should be able to derive it from the first two Friedmann equations: differentiating the 00 one we get 
%
\begin{align}
  6 \qty(\frac{\dot{a}}{a} - \frac{\dot{a}^2}{a^2}) = \dot{\rho} M_P^{-2} 
\,,
\end{align}
%
while for the second one, multiplying by \(3 \dot{a} / a\), we get 
%
\begin{align}
  - 6 \frac{\dot{a}}{a} \frac{\ddot{a}}{a}
  - 3 \frac{\dot{a}^3}{a^3} = \frac{3 \dot{a}}{a } P M_P^{-2}
\,.
\end{align}

Adding them together we find 
%
\begin{align}
  - 9 \frac{\dot{a}^3}{a^3} = M_P^{-2} \qty( \dot{\rho} + 3 \frac{\dot{a}}{a }P )
\,,
\end{align}
%
and from the first FE we have 
%
\begin{align}
  9 \frac{\dot{a}^3}{a^3} = 3 \frac{\dot{a}}{a} \rho M_P^{-2}
\,,
\end{align}
%
so we get 
%
\begin{align}
  \dot{\rho}
 + 3 \frac{\dot{a}}{a} \qty(\rho +P) = 0  
\,.
\end{align}

So, since the equations are not independent, we consider  only two of the three. It is convenient to use the first and the third since they have no second derivatives.  

What sources do we put for the equations? First of all, commonly we do \(w = P / \rho \), and sometimes we do \(w = -1\): this means \(\rho = \const\). 
This corresponds to \emph{vacuum energy}, associated with the space itself: it does not scale inversely with the volume. 

We can have vacuum energy: ``vacuum'' just means we are in a minimum of the potential. Mexican hats and stuff. 

In principle, the EFE can be modified in a simple way: 
%
\begin{align}
  G_{\mu \nu } + \Lambda g_{\mu  \nu } = \frac{T_{\mu \nu }}{M_P^2}
\,,
\end{align}
%
since the metric is covariantly constant. This can be interpreted as a \emph{constant negative energy density}: It would look like 
%
\begin{align}
  T_{\mu \nu } \rightarrow T_{\mu \nu } + \Lambda M_P^2 g_{\mu \nu }
\,.
\end{align}

Then we have 
%
\begin{align}
  G_{00} = \frac{1}{M_P^2} \qty(\rho + \Lambda M_P^2)
\,,
\end{align}
%
and 
%
\begin{align}
  G_{ij} = \frac{1}{M_P^2} \qty(a^2 \delta_{ij} p - \Lambda M_P^2a^2 \delta_{ij})
\,.
\end{align}

So, we have 
%
\begin{align}
  \frac{P_{\Lambda }}{\rho_{\Lambda }} = w_{\Lambda } = -1
\,.
\end{align}

This ratio between pressure and density is characteristic of a cosmological constant. 

\end{document}