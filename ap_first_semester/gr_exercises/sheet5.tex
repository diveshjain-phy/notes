\documentclass[main.tex]{subfiles}
\begin{document}

\section{Sheet 5}

\subsection{Alternative derivation of the contracted Bianchi identities}

The form of the Riemann tensor in a LIF was derived in section \ref{sec:LIF-form-Riemann-tensor}.

\subsubsection{Bianchi identities of the Riemann tensor}

In the LIF, given that \(  R_{\mu \nu \rho \sigma} = g_{\mu [\sigma |, \nu |\rho ]} - g_{\nu  [\sigma |, \mu | \rho ]} \), we want to show that \(R_{\mu \nu [\rho \sigma ; \alpha  ]} = R_{\mu \nu [\rho \sigma , \alpha ]} =0\).

This is equivalent to the formulation of the Bianchi identities given in the problem sheet, because of the antisymmetry of the Riemann tensor in its last two indices: there are six terms in the antisymmetrization of \(R_{\mu \nu [\rho \sigma , \alpha ]}\), but they are pairwise equal: the term \(R_{\mu \nu \rho \sigma , \alpha }\) is equal to \(- R_{\mu \nu \sigma \rho , \alpha }\) by antisymmetry, and these are exactly the pairs of terms which appear in the three-index antisymmetrization.

What we need to do is to take the derivative of the Riemann tensor in the LIF: 
%
\begin{align}
    R_{\mu \nu \rho \sigma , \alpha } = 
    g_{\mu [\sigma |, \nu |\rho ] \alpha } - g_{\nu  [\sigma |, \mu | \rho ] \alpha }
\,,
\end{align}
%
and permute the three indices \(\rho \sigma  \alpha \) cyclically. Writing all the terms out we get (up to a factor \(2\), which is irrelevant since we will find that all the terms cancel and everything is equal to 0):  
%
\begin{align}
    \begin{split}
        +&g_{\mu \sigma , \nu \rho \alpha } - g_{\nu  \sigma , \mu  \rho  \alpha }
        - g_{\mu \rho  , \nu \sigma \alpha } + g_{\nu  \rho  , \mu  \sigma  \alpha } + \\
        +&g_{\mu \alpha , \nu \sigma  \rho  } - g_{\nu  \alpha  , \mu  \sigma \rho  }
        - g_{\mu \sigma  , \nu \alpha \rho  } + g_{\nu  \sigma   , \mu  \alpha \rho  } + \\
        + &g_{\mu \rho  , \nu \alpha \sigma  } - g_{\nu  \rho  , \mu  \alpha \sigma  }
        - g_{\mu \alpha   , \nu \rho \sigma  } + g_{\nu  \alpha   , \mu \rho \sigma  } 
        \,,
    \end{split}
\end{align}
%
so we have \(6 \) terms with a + sign, and \(6\) with a \(-\) sign: they cancel pairwise, since the partial  derivatives commute.

\subsubsection{Contracting the identities}

We start by contracting \(2 R_{\mu \nu [ \rho \sigma ; \alpha ]}\) with \(g^{\mu \rho }\): we get 
%
\begin{align}
    0=
  g^{\mu \rho } \qty(R_{\mu \nu \rho \sigma ; \alpha } + R_{\mu \nu \alpha \rho ; \sigma } + R_{\mu \nu \sigma \alpha ; \rho }) = R_{\nu \sigma ; \alpha } - R_{\nu \alpha ; \sigma } + g^{\mu \rho } R_{\mu \nu \sigma \alpha ; \rho }
\,,
\end{align}
%
where, in the second term, we used the antisymmetry of the first two indices of the Riemann tensor in order to get the form which allowed us to use the definition of the Ricci tensor \(R_{\mu \nu } = g^{\rho \sigma } R_{\rho \mu \sigma \nu }\).
Also, we brought the metric inside the covariant derivatives since it is covariantly constant.
Then, we contract the expression we found with \(g^{\nu \sigma }\): 
%
\begin{align} 
  0=
  g^{\nu \sigma } \qty(R_{\nu \sigma ; \alpha } - R_{\nu \alpha ; \sigma } + g^{\mu \rho } R_{\mu \nu \sigma \alpha ; \rho })
  &= R_{;\alpha  } - \tensor{R}{^{\sigma }_{\alpha; \sigma  }} - g^{\mu \rho} R_{\mu \alpha ; \rho }  \\
  &= R_{;\alpha  } - \tensor{R}{^{\sigma }_{\alpha; \sigma  }} - \tensor{R}{^{\sigma }_{\alpha ; \sigma }}
\,,
\end{align}
%
where we used the same properties as before and the definition of the scalar curvature \(R = g^{\mu \nu }R_{\mu \nu }\). So, we have the contracted Bianchi identities \(0 = R_{;\alpha } - 2 \tensor{R}{^{\sigma }_{\alpha ; \sigma }}\).
Raising an index with the inverse metric \(g^{\alpha \beta }\) and relabeling \(\sigma \) to \(\alpha \) in the second term (after having raised the index), these can be written as 
%
\begin{align}
  \nabla_{\alpha } \qty(R g^{\alpha \beta } - 2R^{\alpha \beta })
\,.
\end{align}
%

\subsection{Weak-field geodesic equation}

This was already treated in section \ref{sec:weak-field-gravitational-potential}.

\subsection{Hyperbolic plane geodesics}

Our coordinates are \((x, y)\), and our metric is \(g_{ij } = y^{-2} \delta_{ij }\), with inverse \(g^{ij} = y^2 \delta^{ij}\).

So, we can calculate the Christoffel symbols as: 
%
\begin{align}
  \Gamma^{i}_{jk} = \frac{1}{2} g^{im} \qty(g_{mj,k} + g_{m k, j } - g_{jk,m})
\,,
\end{align}
%
this calculation is simplified by the fact that the only nonvanishing derivatives of the metric are \(g_{00,1}=g_{11,1} = -2 y^{-3} \). If the index \(i\) in \(\Gamma^{i}_{jk}\) is zero, then the last term in the sum vanishes since it corresponds to a derivative with respect to \(x\).
With these we get: 
%
\begin{subequations}
\begin{align}
  \Gamma^{0}_{00} &=  \frac{1}{2} y^2 \qty(2g_{00,0})= 0 \\
  \Gamma^{0}_{01} &=  \frac{1}{2} y^2 \qty(g_{00,1})= - \frac{1}{y}  \\
  \Gamma^{0}_{11} &=  \frac{1}{2} y^2 \qty(g_{01,1} + g_{01,1})= 0  \\
  \Gamma^{1}_{00} &=  \frac{1}{2} y^2 \qty(- g_{00,1})= \frac{1}{y}  \\
  \Gamma^{1}_{01} &=  \frac{1}{2} y^2 \qty(g_{10,0} + g_{11,0} - g_{01,1})= 0   \\
  \Gamma^{1}_{11} &=  \frac{1}{2} y^2 \qty(g_{11,1}+g_{11,1}-g_{11,1})= -\frac{1}{y}  
\,,
\end{align}
\end{subequations}
%
and the geodesic equation \(u^{\mu } \nabla_{\mu } u^{\nu }= 0 \) is written with respect to these.

\subsubsection{Vertical lines}

First of all we want to parametrize these vertical lines: we choose our parameter so that the length of the velocity vector is everywhere equal to one.

Since the lines are vertical, we want the position \(x^{i}\) with respect to the parameter \(s\) to look something like \(x^{i}(s) = (x_0, y(s))\).

We use the arclength parameter: 
%
\begin{align}
  s = \int \sqrt{g_{ij} u^{i} u^{j}} \dd{\lambda }
\,,
\end{align}
%
where \(u^{i} = \dv*{x^{i}}{\lambda }\) and \(\lambda \) is an arbitrary parameter.

We can rewrite this integral with respect to the Euclidean norm \(\norm{u}_E^2 = \delta_{ij} u^{i} u^{j}\): we get 
%
\begin{align}
  s = \int \frac{1}{y} \norm{u}_E \dd{\lambda }
\,,
\end{align}
%
so we can see that we get \(s = \int \dd{\lambda }\), or \(s = \lambda \), iff \(\norm{u}_E = y\): so, let us drop the distinction between \(s\) and \(\lambda \) and apply this condition.
The velocity vector is 
%
\begin{align}
  u^{i} = \dv{x^{i}}{s} = \qty(0, \dv{y}{s})
\,,
\end{align}
%
whose Euclidean norm is just (the absolute value of) \(\dv*{y}{s}\). 
So, we must impose the condition \(y = \dv*{y}{s}\), which can be solved by separation of variables to yield \(s = \log y\), or \(y = e^{s}\). 

So our parametrization for the curve is \(s \rightarrow x^{i} = (x_0, e^{s})\), the velocity is \(u^{i} = (0, e^{s})\) and the acceleration is again found by differentiating with zrespect to \(s\).

\end{document}