\documentclass[main.tex]{subfiles}
\begin{document}

\section{Sheet 5}

\subsection{Alternative derivation of the contracted Bianchi identities}

The form of the Riemann tensor in a LIF was derived in section \ref{sec:LIF-form-Riemann-tensor}.

\subsubsection{Bianchi identities of the Riemann tensor}

In the LIF, given that \(  R_{\mu \nu \rho \sigma} = g_{\mu [\sigma |, \nu |\rho ]} - g_{\nu  [\sigma |, \mu | \rho ]} \), we want to show that \(R_{\mu \nu [\rho \sigma ; \alpha  ]} = R_{\mu \nu [\rho \sigma , \alpha ]} =0\).

This is equivalent to the formulation of the Bianchi identities given in the problem sheet, because of the antisymmetry of the Riemann tensor in its last two indices: there are six terms in the antisymmetrization of \(R_{\mu \nu [\rho \sigma , \alpha ]}\), but they are pairwise equal: the term \(R_{\mu \nu \rho \sigma , \alpha }\) is equal to \(- R_{\mu \nu \sigma \rho , \alpha }\) by antisymmetry, and these are exactly the pairs of terms which appear in the three-index antisymmetrization.

What we need to do is to take the derivative of the Riemann tensor in the LIF: 
%
\begin{align}
    R_{\mu \nu \rho \sigma , \alpha } = 
    g_{\mu [\sigma |, \nu |\rho ] \alpha } - g_{\nu  [\sigma |, \mu | \rho ] \alpha }
\,,
\end{align}
%
and permute the three indices \(\rho \sigma  \alpha \) cyclically. Writing all the terms out we get (up to a factor \(2\), which is irrelevant since we will find that all the terms cancel and everything is equal to 0):  
%
\begin{align}
    \begin{split}
        +&g_{\mu \sigma , \nu \rho \alpha } - g_{\nu  \sigma , \mu  \rho  \alpha }
        - g_{\mu \rho  , \nu \sigma \alpha } + g_{\nu  \rho  , \mu  \sigma  \alpha } + \\
        +&g_{\mu \alpha , \nu \sigma  \rho  } - g_{\nu  \alpha  , \mu  \sigma \rho  }
        - g_{\mu \sigma  , \nu \alpha \rho  } + g_{\nu  \sigma   , \mu  \alpha \rho  } + \\
        + &g_{\mu \rho  , \nu \alpha \sigma  } - g_{\nu  \rho  , \mu  \alpha \sigma  }
        - g_{\mu \alpha   , \nu \rho \sigma  } + g_{\nu  \alpha   , \mu \rho \sigma  } 
        \,,
    \end{split}
\end{align}
%
so we have \(6 \) terms with a + sign, and \(6\) with a \(-\) sign: they cancel pairwise, since the partial  derivatives commute.

\subsubsection{Contracting the identities}

We start by contracting \(2 R_{\mu \nu [ \rho \sigma ; \alpha ]}\) with \(g^{\mu \rho }\): we get 
%
\begin{align}
    0=
  g^{\mu \rho } \qty(R_{\mu \nu \rho \sigma ; \alpha } + R_{\mu \nu \alpha \rho ; \sigma } + R_{\mu \nu \sigma \alpha ; \rho }) = R_{\nu \sigma ; \alpha } - R_{\nu \alpha ; \sigma } + g^{\mu \rho } R_{\mu \nu \sigma \alpha ; \rho }
\,,
\end{align}
%
where, in the second term, we used the antisymmetry of the first two indices of the Riemann tensor in order to get the form which allowed us to use the definition of the Ricci tensor \(R_{\mu \nu } = g^{\rho \sigma } R_{\rho \mu \sigma \nu }\).
Also, we brought the metric inside the covariant derivatives since it is covariantly constant.
Then, we contract the expression we found with \(g^{\nu \sigma }\): 
%
\begin{align} 
  0=
  g^{\nu \sigma } \qty(R_{\nu \sigma ; \alpha } - R_{\nu \alpha ; \sigma } + g^{\mu \rho } R_{\mu \nu \sigma \alpha ; \rho })
  &= R_{;\alpha  } - \tensor{R}{^{\sigma }_{\alpha; \sigma  }} - g^{\mu \rho} R_{\mu \alpha ; \rho }  \\
  &= R_{;\alpha  } - \tensor{R}{^{\sigma }_{\alpha; \sigma  }} - \tensor{R}{^{\sigma }_{\alpha ; \sigma }}
\,,
\end{align}
%
where we used the same properties as before and the definition of the scalar curvature \(R = g^{\mu \nu }R_{\mu \nu }\). So, we have the contracted Bianchi identities \(0 = R_{;\alpha } - 2 \tensor{R}{^{\sigma }_{\alpha ; \sigma }}\).
Raising an index, these can be written as 
%
\begin{align}
  \nabla_{\beta } \qty(R g^{\alpha \beta } - 2R^{\alpha \beta })
\,.
\end{align}
%


\end{document}