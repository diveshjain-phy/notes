\documentclass[main.tex]{subfiles}
\begin{document}

\section{Sheet 5}

\subsection{Alternative derivation of the contracted Bianchi identities}

The form of the Riemann tensor in a LIF was derived in section \ref{sec:LIF-form-Riemann-tensor}.

\subsubsection{Bianchi identities of the Riemann tensor}

In the LIF, given that \(  R_{\mu \nu \rho \sigma} = g_{\mu [\sigma |, \nu |\rho ]} - g_{\nu  [\sigma |, \mu | \rho ]} \), we want to show that \(R_{\mu \nu [\rho \sigma ; \alpha  ]} = R_{\mu \nu [\rho \sigma , \alpha ]} =0\).

This is equivalent to the formulation of the Bianchi identities given in the problem sheet, because of the antisymmetry of the Riemann tensor in its last two indices: there are six terms in the antisymmetrization of \(R_{\mu \nu [\rho \sigma , \alpha ]}\), but they are pairwise equal: the term \(R_{\mu \nu \rho \sigma , \alpha }\) is equal to \(- R_{\mu \nu \sigma \rho , \alpha }\) by antisymmetry, and these are exactly the pairs of terms which appear in the three-index antisymmetrization.

What we need to do is to take the derivative of the Riemann tensor in the LIF: 
%
\begin{align}
    R_{\mu \nu \rho \sigma , \alpha } = 
    g_{\mu [\sigma |, \nu |\rho ] \alpha } - g_{\nu  [\sigma |, \mu | \rho ] \alpha }
\,,
\end{align}
%
and permute the three indices \(\rho \sigma  \alpha \) cyclically. Writing all the terms out we get (up to a factor \(2\), which is irrelevant since we will find that all the terms cancel and everything is equal to 0):  
%
\begin{align}
    \begin{split}
        +&g_{\mu \sigma , \nu \rho \alpha } - g_{\nu  \sigma , \mu  \rho  \alpha }
        - g_{\mu \rho  , \nu \sigma \alpha } + g_{\nu  \rho  , \mu  \sigma  \alpha } + \\
        +&g_{\mu \alpha , \nu \sigma  \rho  } - g_{\nu  \alpha  , \mu  \sigma \rho  }
        - g_{\mu \sigma  , \nu \alpha \rho  } + g_{\nu  \sigma   , \mu  \alpha \rho  } + \\
        + &g_{\mu \rho  , \nu \alpha \sigma  } - g_{\nu  \rho  , \mu  \alpha \sigma  }
        - g_{\mu \alpha   , \nu \rho \sigma  } + g_{\nu  \alpha   , \mu \rho \sigma  } 
        \,,
    \end{split}
\end{align}
%
so we have \(6 \) terms with a + sign, and \(6\) with a \(-\) sign: they cancel pairwise, since the partial  derivatives commute.

\subsubsection{Contracting the identities}

We start by contracting \(2 R_{\mu \nu [ \rho \sigma ; \alpha ]}\) with \(g^{\mu \rho }\): we get 
%
\begin{align}
    0=
  g^{\mu \rho } \qty(R_{\mu \nu \rho \sigma ; \alpha } + R_{\mu \nu \alpha \rho ; \sigma } + R_{\mu \nu \sigma \alpha ; \rho }) = R_{\nu \sigma ; \alpha } - R_{\nu \alpha ; \sigma } + g^{\mu \rho } R_{\mu \nu \sigma \alpha ; \rho }
\,,
\end{align}
%
where, in the second term, we used the antisymmetry of the first two indices of the Riemann tensor in order to get the form which allowed us to use the definition of the Ricci tensor \(R_{\mu \nu } = g^{\rho \sigma } R_{\rho \mu \sigma \nu }\).
Also, we brought the metric inside the covariant derivatives since it is covariantly constant.
Then, we contract the expression we found with \(g^{\nu \sigma }\): 
%
\begin{align} 
  0=
  g^{\nu \sigma } \qty(R_{\nu \sigma ; \alpha } - R_{\nu \alpha ; \sigma } + g^{\mu \rho } R_{\mu \nu \sigma \alpha ; \rho })
  &= R_{;\alpha  } - \tensor{R}{^{\sigma }_{\alpha; \sigma  }} - g^{\mu \rho} R_{\mu \alpha ; \rho }  \\
  &= R_{;\alpha  } - \tensor{R}{^{\sigma }_{\alpha; \sigma  }} - \tensor{R}{^{\sigma }_{\alpha ; \sigma }}
\,,
\end{align}
%
where we used the same properties as before and the definition of the scalar curvature \(R = g^{\mu \nu }R_{\mu \nu }\). So, we have the contracted Bianchi identities \(0 = R_{;\alpha } - 2 \tensor{R}{^{\sigma }_{\alpha ; \sigma }}\).
Raising an index with the inverse metric \(g^{\alpha \beta }\) and relabeling \(\sigma \) to \(\alpha \) in the second term (after having raised the index), these can be written as 
%
\begin{align}
  \nabla_{\alpha } \qty(R g^{\alpha \beta } - 2R^{\alpha \beta })
\,.
\end{align}
%

\subsection{Weak-field geodesic equation}

This was already treated in section \ref{sec:weak-field-gravitational-potential}.

\subsection{Hyperbolic plane geodesics}

Our coordinates are \((x, y)\), and our metric is \(g_{ij } = y^{-2} \delta_{ij }\), with inverse \(g^{ij} = y^2 \delta^{ij}\).

So, we can calculate the Christoffel symbols as: 
%
\begin{align}
  \Gamma^{i}_{jk} = \frac{1}{2} g^{im} \qty(g_{mj,k} + g_{m k, j } - g_{jk,m})
\,,
\end{align}
%
this calculation is simplified by the fact that the only nonvanishing derivatives of the metric are \(g_{00,1}=g_{11,1} = -2 y^{-3} \). If the index \(i\) in \(\Gamma^{i}_{jk}\) is zero, then the last term in the sum vanishes since it corresponds to a derivative with respect to \(x\).
With these we get: 
%
\begin{subequations}
\begin{align}
  \Gamma^{0}_{00} &=  \frac{1}{2} y^2 \qty(2g_{00,0})= 0 \\
  \Gamma^{0}_{01} &=  \frac{1}{2} y^2 \qty(g_{00,1})= - \frac{1}{y}  \\
  \Gamma^{0}_{11} &=  \frac{1}{2} y^2 \qty(g_{01,1} + g_{01,1})= 0  \\
  \Gamma^{1}_{00} &=  \frac{1}{2} y^2 \qty(- g_{00,1})= \frac{1}{y}  \\
  \Gamma^{1}_{01} &=  \frac{1}{2} y^2 \qty(g_{10,0} + g_{11,0} - g_{01,1})= 0   \\
  \Gamma^{1}_{11} &=  \frac{1}{2} y^2 \qty(g_{11,1}+g_{11,1}-g_{11,1})= -\frac{1}{y}  
\,,
\end{align}
\end{subequations}
%
and the geodesic equation \(u^{\mu } \nabla_{\mu } u^{\nu }= 0 \) is written with respect to these.

\subsubsection{Vertical lines}

First of all we want to parametrize these vertical lines: we choose our parameter so that the length of the velocity vector is everywhere equal to one.

Since the lines are vertical, we want the position \(x^{i}\) with respect to the parameter \(s\) to look something like \(x^{i}(s) = (x_0, y(s))\).

We use the arclength parameter: 
%
\begin{align}
  s = \int \sqrt{g_{ij} u^{i} u^{j}} \dd{\lambda }
\,,
\end{align}
%
where \(u^{i} = \dv*{x^{i}}{\lambda }\) and \(\lambda \) is an arbitrary parameter.

We can rewrite this integral with respect to the Euclidean norm \(\norm{u}_E^2 = \delta_{ij} u^{i} u^{j}\): we get 
%
\begin{align}
  s = \int \frac{1}{y} \norm{u}_E \dd{\lambda }
\,,
\end{align}
%
so we can see that we get \(s = \int \dd{\lambda }\), or \(s = \lambda \), iff \(\norm{u}_E = y\): so, let us drop the distinction between \(s\) and \(\lambda \) and apply this condition.
The velocity vector is 
%
\begin{align}
  u^{i} = \dv{x^{i}}{s} = \qty(0, \dv{y}{s})
\,,
\end{align}
%
whose Euclidean norm is just (the absolute value of) \(\dv*{y}{s}\). 
So, we must impose the condition \(y = \dv*{y}{s}\), which can be solved by separation of variables to yield \(s = \log y\), or \(y = e^{s}\). 

So our parametrization for the curve is \(s \rightarrow x^{i} = (x_0, e^{s})\), the velocity is \(u^{i} = (0, e^{s})\) and the derivative of velocity with respect to \(s\) is again 
%
\begin{align}
  \dv[]{u^{i}}{s} = (0, e^{s})
\,.
\end{align}

Now we can plug these into our geodesic equation, which is simplified by the fact that \(u^{0} = 0\), therefore there is only one relevant term in the Christoffel sum: 
%
\begin{align}
  \dv[]{ u^{i}}{s} + \Gamma^{i}_{11} u^{1} u^{1} = 0
\,,
\end{align}
%
whose components are 
%
\begin{align}
  \underbrace{\dv{ u^{0}}{s}}_{0} + \underbrace{\Gamma^{0}_{11}}_{0} y^2 = 0
\,
\end{align}
%
and 
%
\begin{subequations}
  \begin{align}
    \dv{ u^{1}}{s} + \Gamma^{1}_{11} u^{1} u^{1} &= 0  \\
    y + \qty(- \frac{1}{y}) y^2 &=0
    \,,
  \end{align}
\end{subequations}
%
which are identities, therefore vertical lines are indeed geodesics in this hyperbolic plane.

\subsubsection{More solutions}

We have the Killing vector field \(\xi = (1,0)\) corresponding to the symmetry with respect to translations along the \(x\) axis: then, the quantity \(\vec{\xi} \cdot \vec{u}\) is conserved: so \(\xi^{\mu } u^{\nu } g_{\mu \nu }=\dot{x} g_{00} = \dot{x} y^{-2}\) is constant along the trajectory.

We also know that, if we parametrize with arclength, \(\norm{u}_{E} / y \equiv 1 \), which means 
%
\begin{align}
  y = \norm{u}_{E} = \sqrt{\dot{x}^2 + \dot{y}^2} = \dot{x} \sqrt{\qty(y^{\prime })^2 + 1}
\,,
\end{align}
%
where \(y^{\prime } = \dv*{y}{x} = \dot{y} / \dot{x}\).

Then, the conserved quantity can be rewritten as 
%
\begin{align}
  \dot{x} y^{-2} = y^{-2} \frac{y}{\sqrt{y^{\prime 2} +1}}
  = \frac{1}{y \sqrt{y^{\prime 2} + 1}}
\,,
\end{align}
%
which we can call \(1/R\), with \(R = y \sqrt{y^{\prime 2}+1}\). 

\subsubsection{Old, ugly solution}

Before finding the aforementioned derivation of the first integral, here is what I wrote. 

I find the solution which follows quite ugly and unjustified, I have yet to find a geometric justification for these manipulations, which were derived by reverse-engineering the first integral.
Anyhow, these manipulations work\dots

We insert the Christoffel symbols into the geodesic equations and multiplying through by \(y\) we get the following system, in which we denote differentiation with respect to \(s\) by a dot: 
%
\begin{subequations}
\begin{align}
  y \ddot{y} &= \dot{y}^2 - \dot{x}^2  \\
  y \ddot{x} &= 2 \dot{x} \dot{y} 
\,.
\end{align}
\end{subequations}

Now, we will denote the derivative of \(y \) with respect to \(x\) as 
%
\begin{align}
  y' = \dv[]{y}{x} = \frac{\dot{y}}{\dot{x}}
\,.
\end{align}

We want to find a first integral. We start with the definition of \(y'\): 
%
\begin{align}
  \dot{y} - y^{\prime } \dot{x} = 0
\,,
\end{align}
%
and add and subtract the quantity \(\dot{y} y^{\prime 2}\): 
%
\begin{align}
  \dot{y} + (\dot{y} y^{\prime } - \dot{x}) y^{\prime } - \dot{y} y^{\prime 2} =0
\,,
\end{align}
%
which can also be written as 
%
\begin{align} \label{eq:hyperbolic-plane-geodesics-integral-step1}
  \dot{y} + (\dot{y} y^{\prime } - \dot{x}) y^{\prime } + \dot{y} y^{\prime 2} - 2 \dot{y} y^{\prime 2} = 0
\,.
\end{align}

Now, we want to substitute in our equations of motion: we will need them in the form \(y \ddot{y} / \dot{x} = \dot{y} y^{\prime } - \dot{x}\) and \(y \ddot{x} / \dot{x} = 2 \dot{y}\).
We recognize the LHS of both of these in \eqref{eq:hyperbolic-plane-geodesics-integral-step1} and plug them in: it becomes
%
\begin{align} \label{eq:hyperbolic-plane-geodesics-integral-step2}
  \dot{y} + \frac{y \ddot{y}}{\dot{x}} y^{\prime } + \dot{y} y^{\prime 2} - \frac{y \ddot{x}}{\dot{x}} y^{\prime 2} = 0
\,.
\end{align}

We can recognize that the second derivative terms look similar to the derivative of \(y'\), which is:
%
\begin{align}
  \dv[]{}{s} y^{\prime } 
  = \frac{\ddot{y}}{\dot{x}} - \frac{\dot{y}}{\dot{x}
  } \ddot{x}
  = \frac{\ddot{y}}{\dot{x}} - y^{\prime } \ddot{x}
\,,
\end{align}
%
so equation \eqref{eq:hyperbolic-plane-geodesics-integral-step2} becomes:
%
\begin{align}
  \dot{y} + \dot{y} y^{\prime 2} + y y^{\prime } \dv{y'}{s}=0
\,,
\end{align}
%
which is \(1/(2y)\) times the derivative of \(y^2 (y^{\prime 2} + 1)\), which comes out to be: 
%
\begin{align}
  \frac{1}{2y}
  \dv{}{s} \qty(y^2 (y^{\prime 2} + 1))
  = \frac{1}{2y}\qty(2 y \dot{y} 
  (y^{\prime 2} + 1)
  + y^2 2 y^{\prime } \dv{}{s} y^{\prime })
\,,
\end{align}
%
exactly what we had before.

So, \(y^2 (y^{\prime 2} + 1) = \const\), and we can call this constant \(R^2\) (which is \(1/A^2\) in the homework notation: I find this notation to be more suggestive).

\subsubsection{Solving the equation}

The equation \(y^2 (y^{\prime 2}+1) = R^2\) can be rewritten as 
%
\begin{align}
  \dv{y}{x} = \pm \sqrt{\frac{R^2}{y^2} - 1}
\,,
\end{align}
%
which means that if we fix \(R\) the value of the derivative of the curve can only attain two opposite values. Do note that we can go from one branch to the other with the transformation \(x \rightarrow -x\), a mirror symmetry around some center.
Then, we can just make the gauge choice \(y^{\prime }>0\) and integrate by separation of varibles: 
%
\begin{align}
  \int \frac{y \dd{y}}{\sqrt{R^2-y^2}} = \int \dd{x}
\,,
\end{align}
%
which can be solved with the substitution \( y = R \sin(\theta )\), with \(\dd{y} = R \cos(\theta ) \dd{\theta }\). Inserting this, we find: 
%
\begin{align}
  x - x_0  = \int \frac{R \sin\theta R \cos\theta  \dd{\theta }}{R \sqrt{1 - \sin^2 \theta }} 
  = R \int \sin \theta  \dd{\theta } = - R \cos \theta 
\,,
\end{align}
%
which can be squared to find \((x- x_0 )^2 = R^2 (1 - \sin^2 \theta ) = R^2 - y^2\), or 
%
\begin{align}
  R^2 = (x-x_0  )^2 + y^2
\,,
\end{align}
%
the equation of a circle.
We can then confirm that this solution also holds in the other branch, up to a change of integration constant, by rewriting \((x- x_0 )^2 = ((-x) - x_1 )^2\): this is solved by \(x_0 = - x_1 \). 

Geometrically, we are looking at circles with origins on the \(x\) axis and radius \(R\). If \(y\) and \(R\) are fixed, then there are only two possible circles, which can be found by connecting a certain point at height \(y\) to the \(x\) axis with a segment of length \(R\).
One can then see that the right halves of the circles we found can be found from the left halves by symmetry.

\end{document}