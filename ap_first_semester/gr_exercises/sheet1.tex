\documentclass[main.tex]{subfiles}
\begin{document}

We set $c=1$.

\section{Sheet 1}

\subsection{Lorentz transformations }

\subsubsection{Inverses}

We can consider a Lorentz boost with velocity $v$ in the $x$ direction, and we look at its representation in the $(t, x)$ plane (since the $y$ and $z$ directions are unchanged). Its matrix expression looks like:
%
\begin{equation}\label{LorBoost}
    \Lambda = \begin{bmatrix}
        \gamma & -v \gamma \\
        -v \gamma & \gamma 
    \end{bmatrix}\,,
\end{equation}
%
where $\gamma = 1 / \sqrt{1 - v^2}$. The inverse of this matrix can be computed using the general formula for a 2x2 matrix:

\begin{equation}
    A^{-1}=
    \begin{bmatrix}
        a & b \\
        c & d
    \end{bmatrix}^{-1}
    =
    \frac{1}{\det(A)}
        \begin{bmatrix}
        d & -b \\
        -c & a
    \end{bmatrix}\,.
\end{equation}

The determinant of $\Lambda $ is equal to $\gamma^2 (1-v^2) = 1$, therefore the inverse matrix is:
%
\begin{equation}
    \Lambda = \begin{bmatrix}
        \gamma & v \gamma \\
        v \gamma & \gamma 
    \end{bmatrix}\,.
\end{equation}

\subsubsection{Invariance of the spacetime interval}

Our Lorentz transformation is 

\begin{subequations}
\begin{align}
    \dd{t}' &= \gamma (\dd{t} - v \dd{x}) \\
    \dd{x}' &= \gamma (-v\dd{t} + \dd{x}) \\
    \dd{y}' &= \dd{y} \\
    \dd{z}' &= \dd{z}
\end{align}
\end{subequations}
%
and we wish to prove that the spacetime interval, defined by $\dd{s^2} = \eta_{\mu\nu} \dd{x^\mu}\dd{x^\nu}$ is preserved: $\dd{s'\,^2} = \dd{s^2}$.
Let us write the claimed equality explicitly:
%
\begin{subequations}
\begin{align}
    -\dd{t^2} + \dd{x^2}+ \dd{y^2}+ \dd{z^2}
    &= \gamma (\dd{t} - v \dd{x})
\end{align}
\end{subequations}


\subsubsection{Tensor notation pseudo-orthogonality}

The invariance of the spacetime interval $\dd{s'\,^2} = \dd{s^2}$ can be also written as \(\eta_{\mu\nu} \dd{x^\mu} \dd{x^\nu} = \eta_{\mu\nu} \dd{x'\,^\mu} \dd{x'\,^\nu}\). By making the primed differentials explicit we have:
%
\begin{equation}
  \eta_{\mu\nu} \dd{x^\mu} \dd{x^\nu}
  =
  \eta_{\mu\nu} \tensor{\Lambda}{^\mu_\rho} \dd{x^\rho} \tensor{\Lambda}{^\nu_\sigma} \dd{x^\sigma} \,,
\end{equation}
%
but the dummy indices on the LHS can be changed to \(\rho\) and \(\sigma\), so that both sides are proportional to  \(\dd{x^\rho}\dd{x^\sigma}\). Doing this we get:
%
\begin{equation}
  \eta_{\rho\sigma}
  =
  \eta_{\mu\nu} \tensor{\Lambda}{^\mu_\rho} \tensor{\Lambda}{^\nu_\sigma}
  =\tensor{(\Lambda^\top)}{_\rho^\mu} \eta_{\mu\nu} \tensor{\Lambda}{^\nu_\sigma} \,,
\end{equation}
%
or, in matrix form, \(\eta = \Lambda^\top \eta \Lambda\).
\subsubsection{Pseudo orthogonality}
Defining $dx^\mu=(cdt,dx,dy,dz)^T$, and $dx_\mu=(cdt,dx,dy,dz)=\eta_{\mu\nu}dx^\nu$, we will have
\begin{align}
&ds^2=dx^\mu dx_\mu=dx^\mu\eta_{\mu\nu}dx^\nu\nonumber\\
&ds^2'=dx^\mu' dx_\mu'=dx^\mu'\eta_{\mu\nu}dx^\nu'=dx^\rho\Lambda_\rho^\mu\eta_{\mu\nu}dx^\sigma\Lambda_\sigma^\nu
\end{align} and $ds^2=ds^2'$ is equivalent to our thesis.

\subsubsection{Pseudo orthogonality by matrix product}

Let $\Lambda$ be the Lorentz transformation in \eqref{LorBoost}.bIn this case we have
\begin{align}
    \Lambda ^T\eta\Lambda=& \begin{bmatrix}
        \gamma & -v \gamma & 0 & 0\\
        -v \gamma & \gamma & 0 & 0\\
        0 & 0 & 1 & 0\\
        0 & 0 & 0 & 1\\
    \end{bmatrix}\cdot \begin{bmatrix}
        -1 & 0 & 0 & 0\\
        0 & 1 & 0 & 0\\
        0 & 0 & 1 & 0\\
        0 & 0 & 0 & 1\\
    \end{bmatrix}\cdot \begin{bmatrix}
        \gamma & -v \gamma & 0 & 0\\
        -v \gamma & \gamma & 0 & 0\\
        0 & 0 & 1 & 0\\
        0 & 0 & 0 & 1\\
    \end{bmatrix}=\nonumber\\
     =&\begin{bmatrix}
        -\gamma & -v \gamma & 0 & 0\\
        v \gamma & \gamma & 0 & 0\\
        0 & 0 & 1 & 0\\
        0 & 0 & 0 & 1\\
    \end{bmatrix}\cdot
    \begin{bmatrix}
        \gamma & -v \gamma & 0 & 0\\
        -v \gamma & \gamma & 0 & 0\\
        0 & 0 & 1 & 0\\
        0 & 0 & 0 & 1\\
    \end{bmatrix}=\nonumber\\
    =& \begin{bmatrix}
        -\gamma^2+v^2\gamma^2 & 0 & 0 & 0\\
        0 & -v^2\gamma^2+\gamma^2 & 0 & 0\\
        0 & 0 & 1 & 0\\
        0 & 0 & 0 & 1\\
    \end{bmatrix}=\begin{bmatrix}
        -1 & 0 & 0 & 0\\
        0 & 1 & 0 & 0\\
        0 & 0 & 1 & 0\\
        0 & 0 & 0 & 1\\
    \end{bmatrix}=\eta
\end{align}
Where we used the fact that $\gamma^2(1-v^2)=\frac{\gamma^2}{\gamma^2}=1$.
\subsubsection{Explicit pseudo-orthogonality}

For simplicity but WLOG we consider a boost in the \(x\) direction with velocity \(v\) and Lorentz factor \(\gamma\). The matrix expression to verify is:
%
\begin{subequations}
\begin{align}
  \begin{bmatrix}
  \gamma    & - v \gamma  \\
    -v \gamma & \gamma
  \end{bmatrix}
  \begin{bmatrix}
    -1 & 0 \\
    0 & 1
  \end{bmatrix}
  \begin{bmatrix}
    \gamma & -v \gamma \\
     -v \gamma&  \gamma
  \end{bmatrix}
  &\overset{?}{=}
  \begin{bmatrix}
  -1   & 0 \\
  0   & 1
\end{bmatrix} \\
  \begin{bmatrix}
  \gamma    & - v \gamma  \\
    -v \gamma & \gamma
  \end{bmatrix}
  \begin{bmatrix}
    -\gamma & v \gamma \\
    -v \gamma&  \gamma
  \end{bmatrix}
  &\overset{?}{=}
  \begin{bmatrix}
  -1   & 0 \\
  0   & 1
\end{bmatrix} \\
  \begin{bmatrix}
  -\gamma^2 + \gamma^2 v^2    & v \gamma^2 - v \gamma^2  \\
    v \gamma^2 -v \gamma^2 & -v\gamma^2 +\gamma^2
  \end{bmatrix}
  &=
  \begin{bmatrix}
  -1   & 0 \\
  0   & 1
\end{bmatrix}\,,
\end{align}
\end{subequations}
%
which by \(\gamma^2 = 1/ (1-v^2)\) confirms the validity of the expression.

\subsection{Muons}

\subsubsection{Nonrelativistic approximation}

The survival probability is given by \(\P (t) = \exp(-t/ \SI{2.2e-6}{s})\). If the ground is \( h =\SI{15}{km} \) away, then the muon will reach it in \(t = h/v = \SI{15}{km} / (0.995c) \approx \SI{5.03e-05}{s}\), therefore \(\P(t) \approx \num{1.2e-10} \).

\subsubsection{Relativistic effects: ground perspective}

The observer on the ground will see the muon having to traverse the whole \(h = \SI{15}{km} \), but the muon's time will be dilated for them by a factor \(\gamma_v \approx 10\): therefore the survival probability
will be \(\P(t) = \exp(- t / (\gamma_v \times \SI{2.2e-6}{s})) \approx 0.1\).


\subsubsection{Relativistic effects: muons perspective}

The muons in their system will observe lenght contraction, with respect to Lorentz boost, by a factor by a factor \(\gamma_v \approx 10\): therefore the survival probability
will be \(\P(t) = \exp(- t / (\gamma_v \times \SI{2.2e-6}{s})) \approx 0.1\). This result is the same of the one predicted by ground observer, with respect to relativity principle.
\subsection{Radiation}

\subsubsection{New angle}
Radiation travels at light speed for each inertial system. Now, $v_y=v_y'=\sin\theta'$, while $v_x=\frac{v_x'+v}{1+vv_x'}=\frac{\cos\theta'+v}{1+v\cos\theta'}$. So,
\begin{equation}
\theta=\arctan\frac{v_y}{v_x}=\arctan\left(\frac{\sin\theta'}{\frac{\cos\theta'+v}{1+v\cos\theta'}}\right).
\end{equation}
\subsubsection{Angle plot and relevant limits}
For $v=0$ we have $\theta'=\theta$ as we expected, while for $v=1$ we find the natural relation: $\tan\theta=\sin\theta'$.
\subsubsection{Radiation speed invariance}
It sounds like an hypothesis...
\subsubsection{Isotropic emission}

Since we had that angle emission vary when varying inertial system we obtain that for every system in relative motion respect to $O$ with $v\neq 0$ observs a nonisotropic emission. For $v\simeq 1$ we have that in the observer system there is no emission at $\theta=\frac{\pi}{2}$.
\end{document}