\documentclass[main.tex]{subfiles}
\begin{document}

We set $c=1$.

Here we will often use (anti)symmetrization of indices, which makes some calculations much easier. The idea of symmetrization is to sum over all permutation of the selected indices, with a minus sign for the odd permutation if the case of anti symmetrization. So, for instance, \(F_{\mu \nu }\) can be antisymmetrized into \(F_{[\mu \nu ]} = \frac[i]{1}{2} \qty(F_{\mu \nu }- F_{\nu \mu })\) and symmetrized into \(F_{(\mu \nu )}= \frac[i]{1}{2} \qty(F_{\mu \nu } + F_{\nu \mu })\).

The factor \(\frac[i]{1}{2} \) is in general \(1/n!\), where \(n\) is the number of antisymmetrized indices. This is included because in general we will be summing \(n!\) terms, and we want to write things like: ``\(F_{\mu \nu }\) is antisymmetric means \(F_{\mu \nu} = F_{[\mu \nu ]}\)'', so we need to rescale the sum to make it into an average.

The general formulas are then: 
%
\begin{subequations}
\begin{align}
  F_{[\mu_{1} \dots \mu_{n}]} &= \frac{1}{n!} \sum _{\sigma  \in \mathfrak{S}_n} \sign{\sigma} F_{\sigma (\mu_1)\dots \sigma(\mu_n)} \\
  F_{(\mu_{1} \dots \mu_{n})} &= \frac{1}{n!} \sum _{\sigma  \in \mathfrak{S}_n}F_{\sigma (\mu_1)\dots \sigma(\mu_n)}
\,,  
\end{align}
\end{subequations}
%
where \(\mathfrak{S}_n\) is the \emph{symmetric group} of permutations of \(n\) elements, and the sign of a permutation \(\sigma \in \mathfrak S_n\) is \(\pm 1\), depending on the parity of pair swaps that are needed to get that configuration (we fix (\(\sign{\mathbb{1}} = 1\))). 

If we want to symmetrize indices which are not next to each other, we will denote the end of the (anti)symmetrized indices by a vertical bar.

\section{Sheet 1}

\subsection{Lorentz transformations }

\subsubsection{Inverses}

We can consider a Lorentz boost with velocity $v$ in the $x$ direction, and we look at its representation in the $(t, x)$ plane (since the $y$ and $z$ directions are unchanged). Its matrix expression looks like:
%
\begin{equation}\label{LorBoost}
    \Lambda = \begin{bmatrix}
        \gamma & -v \gamma \\
        -v \gamma & \gamma
    \end{bmatrix}\,,
\end{equation}
%
where $\gamma = 1 / \sqrt{1 - v^2}$. The inverse of this matrix can be computed using the general formula for a 2x2 matrix:

\begin{equation}
    A^{-1}=
    \begin{bmatrix}
        a & b \\
        c & d
    \end{bmatrix}^{-1}
    =
    \frac{1}{\det(A)}
        \begin{bmatrix}
        d & -b \\
        -c & a
    \end{bmatrix}\,.
\end{equation}

The determinant of $\Lambda $ is equal to $\gamma^2 (1-v^2) = 1$, therefore the inverse matrix is:
%
\begin{equation}
    \Lambda = \begin{bmatrix}
        \gamma & v \gamma \\
        v \gamma & \gamma
    \end{bmatrix}\,.
\end{equation}

\subsubsection{Invariance of the spacetime interval}

Our Lorentz transformation is

\begin{subequations}
\begin{align}
    \dd{t}' &= \gamma (\dd{t} - v \dd{x}) \\
    \dd{x}' &= \gamma (-v\dd{t} + \dd{x}) \\
    \dd{y}' &= \dd{y} \\
    \dd{z}' &= \dd{z}
\end{align}
\end{subequations}
%
and we wish to prove that the spacetime interval, defined by $\dd{s^2} = \eta_{\mu\nu} \dd{x^\mu}\dd{x^\nu}$ is preserved: $\dd{s'\,^2} = \dd{s^2}$.
Let us write the claimed equality explicitly:
%
\begin{subequations}
\begin{align}
    -\dd{t^2} + \dd{x^2}+ \dd{y^2}+ \dd{z^2}
    &= \gamma (\dd{t} - v \dd{x})
\end{align}
\end{subequations}


\subsubsection{Tensor notation pseudo-orthogonality}

The invariance of the spacetime interval $\dd{s'\,^2} = \dd{s^2}$ can be also written as \(\eta_{\mu\nu} \dd{x^\mu} \dd{x^\nu} = \eta_{\mu\nu} \dd{x'\,^\mu} \dd{x'\,^\nu}\). By making the primed differentials explicit we have:
%
\begin{equation}
  \eta_{\mu\nu} \dd{x^\mu} \dd{x^\nu}
  =
  \eta_{\mu\nu} \tensor{\Lambda}{^\mu_\rho} \dd{x^\rho} \tensor{\Lambda}{^\nu_\sigma} \dd{x^\sigma} \,,
\end{equation}
%
but the dummy indices on the LHS can be changed to \(\rho\) and \(\sigma\), so that both sides are proportional to  \(\dd{x^\rho}\dd{x^\sigma}\). Doing this we get:
%
\begin{equation}
  \eta_{\rho\sigma}
  =
  \eta_{\mu\nu} \tensor{\Lambda}{^\mu_\rho} \tensor{\Lambda}{^\nu_\sigma}
  =\tensor{(\Lambda^\top)}{_\rho^\mu} \eta_{\mu\nu} \tensor{\Lambda}{^\nu_\sigma} \,,
\end{equation}
%
or, in matrix form, \(\eta = \Lambda^\top \eta \Lambda\).

% \subsubsection{Pseudo orthogonality}
% Defining $dx^\mu=(cdt,dx,dy,dz)^T$, and $dx_\mu=(cdt,dx,dy,dz)=\eta_{\mu\nu}dx^\nu$, we will have
% \begin{align}
% &ds^2=dx^\mu dx_\mu=dx^\mu\eta_{\mu\nu}dx^\nu\nonumber\\
% &\dd{s,^{2\prime}}=dx^{\mu\prime} dx_\mu'=dx^{\mu\prime}\eta_{\mu\nu}dx^{\nu\prime}=dx^\rho\Lambda_\rho^\mu\eta_{\mu\nu}dx^\sigma\Lambda_\sigma^\nu
% \end{align} and $\dd{s'\,^2} = \dd{s^2}$ is equivalent to our thesis.

% \subsubsection{Pseudo orthogonality by matrix product}

% Let $\Lambda$ be the Lorentz transformation in \eqref{LorBoost}.bIn this case we have
% \begin{align}
%     \Lambda ^T\eta\Lambda=& \begin{bmatrix}
%         \gamma & -v \gamma & 0 & 0\\
%         -v \gamma & \gamma & 0 & 0\\
%         0 & 0 & 1 & 0\\
%         0 & 0 & 0 & 1\\
%     \end{bmatrix}\cdot \begin{bmatrix}
%         -1 & 0 & 0 & 0\\
%         0 & 1 & 0 & 0\\
%         0 & 0 & 1 & 0\\
%         0 & 0 & 0 & 1\\
%     \end{bmatrix}\cdot \begin{bmatrix}
%         \gamma & -v \gamma & 0 & 0\\
%         -v \gamma & \gamma & 0 & 0\\
%         0 & 0 & 1 & 0\\
%         0 & 0 & 0 & 1\\
%     \end{bmatrix}=\nonumber\\
%      =&\begin{bmatrix}
%         -\gamma & -v \gamma & 0 & 0\\
%         v \gamma & \gamma & 0 & 0\\
%         0 & 0 & 1 & 0\\
%         0 & 0 & 0 & 1\\
%     \end{bmatrix}\cdot
%     \begin{bmatrix}
%         \gamma & -v \gamma & 0 & 0\\
%         -v \gamma & \gamma & 0 & 0\\
%         0 & 0 & 1 & 0\\
%         0 & 0 & 0 & 1\\
%     \end{bmatrix}=\nonumber\\
%     =& \begin{bmatrix}
%         -\gamma^2+v^2\gamma^2 & 0 & 0 & 0\\
%         0 & -v^2\gamma^2+\gamma^2 & 0 & 0\\
%         0 & 0 & 1 & 0\\
%         0 & 0 & 0 & 1\\
%     \end{bmatrix}=\begin{bmatrix}
%         -1 & 0 & 0 & 0\\
%         0 & 1 & 0 & 0\\
%         0 & 0 & 1 & 0\\
%         0 & 0 & 0 & 1\\
%     \end{bmatrix}=\eta
% \end{align}
% Where we used the fact that $\gamma^2(1-v^2)=\frac{\gamma^2}{\gamma^2}=1$.

\subsubsection{Explicit pseudo-orthogonality}

For simplicity but WLOG we consider a boost in the \(x\) direction with velocity \(v\) and Lorentz factor \(\gamma\). The matrix expression to verify is:
%
\begin{subequations}
\begin{align}
  \begin{bmatrix}
  \gamma    & - v \gamma  \\
    -v \gamma & \gamma
  \end{bmatrix}
  \begin{bmatrix}
    -1 & 0 \\
    0 & 1
  \end{bmatrix}
  \begin{bmatrix}
    \gamma & -v \gamma \\
     -v \gamma&  \gamma
  \end{bmatrix}
  &\overset{?}{=}
  \begin{bmatrix}
  -1   & 0 \\
  0   & 1
\end{bmatrix} \\
  \begin{bmatrix}
  \gamma    & - v \gamma  \\
    -v \gamma & \gamma
  \end{bmatrix}
  \begin{bmatrix}
    -\gamma & v \gamma \\
    -v \gamma&  \gamma
  \end{bmatrix}
  &\overset{?}{=}
  \begin{bmatrix}
  -1   & 0 \\
  0   & 1
\end{bmatrix} \\
  \begin{bmatrix}
  -\gamma^2 + \gamma^2 v^2    & v \gamma^2 - v \gamma^2  \\
    v \gamma^2 -v \gamma^2 & -v\gamma^2 +\gamma^2
  \end{bmatrix}
  &=
  \begin{bmatrix}
  -1   & 0 \\
  0   & 1
\end{bmatrix}\,,
\end{align}
\end{subequations}
%
which by \(\gamma^2 = 1/ (1-v^2)\) confirms the validity of the expression.

\subsection{Muons}

\subsubsection{Nonrelativistic approximation}

The survival probability is given by \(\P (t) = \exp(-t/ \SI{2.2e-6}{s})\). If the ground is \( h =\SI{15}{km} \) away, then the muon will reach it in \(t = h/v = \SI{15}{km} / (0.995c) \approx \SI{5.03e-05}{s}\), therefore \(\P(t) \approx \num{1.2e-10} \).

\subsubsection{Relativistic effects: ground perspective}

The observer on the ground will see the muon having to traverse the whole \(h = \SI{15}{km} \), but the muon's time will be dilated for them by a factor \(\gamma_v \approx 10\): therefore the survival probability
will be \(\P(t) = \exp(- t / (\gamma_v \times \SI{2.2e-6}{s})) \approx 0.1\).


\subsubsection{Relativistic effects: muons perspective}

The muons in their system will observe lenght contraction, with respect to Lorentz boost, by a factor \(\gamma_v \approx 10\): therefore the survival probability
will be \(\P(t) = \exp(- t / (\gamma_v \times \SI{2.2e-6}{s})) \approx 0.1\). This result is the same of the one predicted by ground observer, with respect to relativity principle.


\subsection{Radiation}

\subsubsection{New angle}
In the source frame the radiation velocity components are \( u_x' = \cos\theta', u_y' = \sin\theta' \). From the composition of velocities we obtain:

\begin{subequations}
\begin{align}
    u_y = \sin\theta &= \frac{\dd{y}}{\dd{t}} = \frac{\dd{y'}}{\gamma_v(\dd{t'} + v\dd{x'})} = \frac{\sin\theta'}{\gamma_v(1+v\cos\theta')} \\
    u_x = \cos\theta &= \frac{\dd{x}}{\dd{t}} = \frac{\gamma_v(\dd{x'} + v\dd{t'})}{\gamma_v(\dd{t'} + v\dd{x'})} = \frac{\cos\theta' + v}{1+v\cos\theta'}\,,
\end{align}
\end{subequations}
%
hence:
%
\begin{equation}
\frac{1}{\tan\theta} = \frac{\gamma_v}{\tan\theta'} + \frac{\gamma_vv}{\sin\theta'} \,.
\end{equation}

\subsubsection{Angle plot and relevant limits}
See the jupyter notebook in the \texttt{python} folder for plots.
For $v=0$ we have $\theta=\theta'$ as we expected, while for $v=1$, $\theta = 0$.

\subsubsection{Radiation speed invariance}

Are the components of the velocity, which we called \(\sin \theta\) and \(\cos \theta \), actually normalized? Let us check:

\begin{subequations}
\begin{align}
    \sin^2\theta + \cos^2\theta &= \frac{(\frac{\sin\theta'}{\gamma_v})^2 + (\cos\theta' + v)^2}{(1 + v\cos\theta')^2} \\ 
    &= \frac{(1-v^2)\sin^2\theta' + \cos^2\theta' + v^2 + 2v\cos\theta'}{(1 + v\cos\theta')^2} \\
    &= \frac{1+v^2(1-\sin\theta')+2v\cos\theta'}{(1 + v\cos\theta')^2} = 1 \,,
\end{align}
\end{subequations}
%
therefore the square modulus of the speed of the radiation is still \(c\), as we could have assumed earlier.

\subsubsection{Isotropic emission}

Since the angular distribution of emission varies when changing inertial reference, we might suppose that every system in relative motion respect to $O$ with $v\neq 0$ observes nonisotropic emission.

This can be seen by noticing that for $v\simeq 1$ we have that in the observer system there is almost only emission at an angle $\theta = 0$.
In general, since there is a Lorentz \(\gamma\) factor multiplying a function of the angle in the radiation emission frame \(O'\), the cotangent of the angle in the observation frame \(O\) must get larger and larger as the relative velocity \(v\) increases, therefore the radiation gets compressed towards angles with large cotangents: \(\theta\sim 0\).

See the jupyter notebook in the \texttt{python} folder for interactive plots :)

\end{document}

