\documentclass[main.tex]{subfiles}
\begin{document}

We set $c=1$.

\section{Sheet 1}

\subsection{Lorentz transformations }

\subsubsection{Inverses}

We can consider a Lorentz boost with velocity $v$ in the $x$ direction, and we look at its representation in the $(t, x)$ plane (since the $y$ and $z$ directions are unchanged). Its matrix expression looks like:
%
\begin{equation}
    \Lambda = \begin{bmatrix}
        \gamma & -v \gamma \\
        -v \gamma & \gamma
    \end{bmatrix}\,,
\end{equation}
%
where $\gamma = 1 / \sqrt{1 - v^2}$. The inverse of this matrix can be computed using the general formula for a 2x2 matrix:

\begin{equation}
    A^{-1}=
    \begin{bmatrix}
        a & b \\
        c & d
    \end{bmatrix}^{-1}
    =
    \frac{1}{\det(A)}
        \begin{bmatrix}
        d & -b \\
        -c & a
    \end{bmatrix}\,.
\end{equation}

The determinant of $\Lambda $ is equal to $\gamma^2 (1-v^2) = 1$, therefore the inverse matrix is:
%
\begin{equation}
    \Lambda = \begin{bmatrix}
        \gamma & v \gamma \\
        v \gamma & \gamma
    \end{bmatrix}\,.
\end{equation}

\subsubsection{Invariance of the spacetime interval}

Our Lorentz transformation is

\begin{subequations}
\begin{align}
    \dd{t}' &= \gamma (\dd{t} - v \dd{x}) \\
    \dd{x}' &= \gamma (-v\dd{t} + \dd{x}) \\
    \dd{y}' &= \dd{y} \\
    \dd{z}' &= \dd{z}
\end{align}
\end{subequations}
%
and we wish to prove that the spacetime interval, defined by $\dd{s^2} = \eta_{\mu\nu} \dd{x^\mu}\dd{x^\nu}$ is preserved: $\dd{s'\,^2} = \dd{s^2}$.
Let us write the claimed equality explicitly:
%
\begin{subequations}
\begin{align}
    -\dd{t^2} + \dd{x^2}+ \dd{y^2}+ \dd{z^2}
    &\overset{?}{=}  - \gamma^2 (\dd{t} - v \dd{x})^2
    + \gamma^2 (-v\dd{t} + \dd{x}) ^2
    +  \dd{y^2}+ \dd{z^2} \\
    (1-v^2) (-\dd{t^2} + \dd{x^2})
    &\overset{?}{=} - (\dd{t} - v \dd{x})^2
    + (-v\dd{t} + \dd{x}) ^2 \\
    -\dd{t^2} + \dd{x^2}
    +v^2\dd{t^2} -v^2 \dd{x^2}
    &\overset{?}{=}  - \dd{t}^2 - v^2 \dd{x}^2 +2v\dd{t} \dd{x}
    + v^2\dd{t}^2 + \dd{x}^2 -2v\dd{x}\dd{t} \\
    -\dd{t^2} + \dd{x^2}
    +v^2\dd{t^2} -v^2 \dd{x^2}
    &= - \dd{t}^2 - v^2 \dd{x}^2
    + v^2\dd{t}^2 + \dd{x}^2
\end{align}
\end{subequations}
%
where we simplified the \(y\) and \(z\) differentials, multiplied by \(1/\gamma^2 = 1-v^2\),  expanded the squares of the binomials and simplified the mixed terms.

\subsubsection{Tensor notation pseudo-orthogonality}

The invariance of the spacetime interval $\dd{s'\,^2} = \dd{s^2}$ can be also written as \(\eta_{\mu\nu} \dd{x^\mu} \dd{x^\nu} = \eta_{\mu\nu} \dd{x'\,^\mu} \dd{x'\,^\nu}\). By making the primed differentials explicit we have:
%
\begin{equation}
  \eta_{\mu\nu} \dd{x^\mu} \dd{x^\nu}
  =
  \eta_{\mu\nu} \tensor{\Lambda}{^\mu_\rho} \dd{x^\rho} \tensor{\Lambda}{^\nu_\sigma} \dd{x^\sigma} \,,
\end{equation}
%
but the dummy indices on the LHS can be changed to \(\rho\) and \(\sigma\), so that both sides are proportional to  \(\dd{x^\rho}\dd{x^\sigma}\). Doing this we get:
%
\begin{equation}
  \eta_{\rho\sigma}
  =
  \eta_{\mu\nu} \tensor{\Lambda}{^\mu_\rho} \tensor{\Lambda}{^\nu_\sigma}
  =\tensor{(\Lambda^\top)}{_\rho^\mu} \eta_{\mu\nu} \tensor{\Lambda}{^\nu_\sigma} \,,
\end{equation}
%
or, in matrix form, \(\eta = \Lambda^\top \eta \Lambda\).

\subsubsection{Explicit pseudo-orthogonality}

For simplicity but WLOG we consider a boost in the \(x\) direction with velocity \(v\) and Lorentz factor \(\gamma\). The matrix expression to verify is:
%
\begin{subequations}
\begin{align}
  \begin{bmatrix}
  \gamma    & - v \gamma  \\
    -v \gamma & \gamma
  \end{bmatrix}
  \begin{bmatrix}
    -1 & 0 \\
    0 & 1
  \end{bmatrix}
  \begin{bmatrix}
    \gamma & -v \gamma \\
     -v \gamma&  \gamma
  \end{bmatrix}
  &\overset{?}{=}
  \begin{bmatrix}
  -1   & 0 \\
  0   & 1
\end{bmatrix} \\
  \begin{bmatrix}
  \gamma    & - v \gamma  \\
    -v \gamma & \gamma
  \end{bmatrix}
  \begin{bmatrix}
    -\gamma & v \gamma \\
    -v \gamma&  \gamma
  \end{bmatrix}
  &\overset{?}{=}
  \begin{bmatrix}
  -1   & 0 \\
  0   & 1
\end{bmatrix} \\
  \begin{bmatrix}
  -\gamma^2 + \gamma^2 v^2    & v \gamma^2 - v \gamma^2  \\
    v \gamma^2 -v \gamma^2 & -v\gamma^2 +\gamma^2
  \end{bmatrix}
  &=
  \begin{bmatrix}
  -1   & 0 \\
  0   & 1
\end{bmatrix}\,,
\end{align}
\end{subequations}
%
which by \(\gamma^2 = 1/ (1-v^2)\) confirms the validity of the expression.

\subsection{Muons}

\subsubsection{Nonrelativistic approximation}

The survival probability is given by \(\P (t) = \exp(-t/ \SI{2.2e-6}{s})\). If the ground is \( h =\SI{15}{km} \) away, then the muon will reach it in \(t = h/v = \SI{15}{km} / (0.995c) \approx \SI{5.03e-05}{s}\), therefore \(\P(t) \approx \num{1.2e-10} \).

\subsubsection{Relativistic effects: ground perspective}

The observer on the ground will see the muon having to traverse the whole \(h = \SI{15}{km} \), but the muon's time will be dilated for them by a factor \(\gamma_v \approx 10\): therefore the survival probability
will be \(\P(t) = \exp(- t / (\gamma_v \times \SI{2.2e-6}{s})) \approx 0.1\).

\subsubsection{Relativistic effects: muon perspective}



\subsection{Radiation}

\end{document}
