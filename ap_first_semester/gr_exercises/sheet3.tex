\documentclass[main.tex]{subfiles}
\begin{document}

\section{Sheet 3}

\subsection{Changes of coordinate system}

We denote by \(x^{\mu } = (x, y)\) and \(x^{\prime \mu } = (r, \theta)\). Then, we have the following Jacobian matrices: 
%
\begin{subequations}
\begin{align}
  \pdv{x^{\mu }}{x^{\prime \nu }}  &= \left[\begin{array}{cc}
  \cos(\theta )  & -r \sin(\theta )  \\ 
  \sin(\theta )  & r \cos(\theta ) 
  \end{array}\right]  \\
  \pdv{x^{\prime \nu }}{x^{\rho }} &= \left[\begin{array}{cc}
  \displaystyle\frac{x}{\sqrt{x^2+y^2} } &\displaystyle \frac{y}{\sqrt{x^2+y^2} } \\ 
  \displaystyle -\frac{y}{x^2+y^2} &\displaystyle \frac{x}{x^2+y^2}
  \end{array}\right] 
  =
  \left[\begin{array}{cc}
  \cos(\theta )  & \sin(\theta )  \\ 
  \displaystyle- \frac{\sin( \theta ) }{r} & \displaystyle \frac{\cos(\theta ) }{r}
  \end{array}\right]
\,,
\end{align}
\end{subequations}
%
which can be found by plain differentiation of the change of coordinates, recalling \((\arctan x)' = 1/ (1+x^2)\). Then, we can compute the product of these two matrices: it comes out to be 
%
\begin{equation} \label{eq:inverse-jacobian-matrix-relation} 
  \pdv{x^{\mu }}{x^{\prime \nu }} \pdv{x^{\prime \nu }}{x^{\rho }} = \delta^{\mu }_{\nu }
\,,
\end{equation}
%
since on the diagonal we get \(\cos^2(\theta ) + \sin^2(\theta ) = 1\), while on the off-diagonal terms we get a multiple of \(\sin(\theta ) \cos(\theta ) - \sin(\theta ) \cos(\theta ) =0\).

Note that relation \eqref{eq:inverse-jacobian-matrix-relation} is just the chain rule written in more generality: substituting the explicit coordinates for \(x^{\mu }\) and \(x^{\prime \mu }\) we get the desired expression.

\subsection{Properties of covariant differentiation}

\subsubsection{Metric compatibility of the connection}

We wish to show that \(\nabla_{\alpha } g_{\mu \nu } = 0\).
A very simple way to prove this is by going in the LIF: there, the equation reads \(\partial_{\alpha }\eta_{\mu \nu } = 0\), which is immediately satisfied since the components of the Minkowski metric are constants.
Then, since the equation is tensorial, the result extends to any frame.

This was not the spirit of the exercise, however: let us prove it in a different generic frame.
To this end, we define the \emph{Christoffel symbols of the first kind} (while the regular ones are of the second kind): 
%
\begin{equation}
  \Gamma_{\mu \nu \rho } = g_{\mu \sigma }\Gamma^{\sigma }_{\nu \rho } = \frac{1}{2} \qty(g_{\mu \nu, \rho } + g_{\mu \rho , \nu } - g_{\nu \rho, \mu })
\,.
\end{equation}

These are useful since, as all their indices are down, it is easier to study their symmetry properties.

Note that if we symmetrize the first and last index, we get \(\Gamma_{[\mu | \nu | \rho]} = \frac[i]{1}{2} g_{\mu \rho, \nu }\) since the first and last terms in the sum cancel (in the latter we must invert the indices \(\nu \) and \(\rho \) in order to see this, but this can always be done by the symmetry of the metric).

Then, we write the expression for the covariant derivative of the metric: 
%
\begin{equation}
  \nabla_{\alpha }g_{\mu \nu }= \partial_\alpha g_{\mu \nu }
  - \Gamma^{\rho }_{\mu \alpha }g_{\rho \nu } 
  - \Gamma^{\rho }_{\nu \alpha  }g_{\mu \rho  } 
  = g_{\mu \nu , \alpha } 
  - \Gamma_{\nu \alpha \mu }
  - \Gamma_{\mu \alpha \nu }
\,,
\end{equation}
%
which is just \(g_{\mu \nu , \alpha } - 2 \Gamma_{[\nu | \alpha | \mu ]} = g_{\mu \nu , \alpha } - g_{\nu \mu , \alpha } = 0\), again by the symmetry of the metric.

\subsubsection{Leibniz rule}

As before, this can be proved in the LIF but we shall compute away in a general frame.

\end{document}