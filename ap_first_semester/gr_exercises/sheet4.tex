\documentclass[main.tex]{subfiles}
\begin{document}

\section{Sheet 3}

\subsection{Riemann tensor computations}

\subsubsection{LIF form}

In the LIF, we have \(g_{\mu \nu } = \eta_{\mu \nu }\) and \(= g_{\mu \nu , \alpha }=0\). Therefore, the Christoffel symbols are all zero. 
In the explicit expression of the Riemann tensor, which looks like \(R = \partial \Gamma  + \Gamma \Gamma\) we can drop the second term and keep only the derivatives of the Christoffel symbols, which are nonzero since they depend on the second derivatives of the metric.

If the Christoffel symbols are zero, the computation of the curvature tensor is significantly easier: 
%
\begin{subequations}
\begin{align}
  R^{\mu }_{\nu \rho \sigma }
  &=  2 \partial_{[\rho }\Gamma^{\mu }_{\sigma ] \nu}  \\
  &= 2 \partial_{[\rho } \qty(\frac{1}{2} g^{\mu \alpha } \qty(g_{\alpha |\sigma ], \nu } + g_{\alpha \nu  , |\sigma  ]} - g_{\nu  |\sigma], \alpha }))  \\
  &= g^{\mu \alpha }\qty(g_{\alpha [\sigma |, \nu | \rho      ]} + \cancelto{}{g_{\alpha \nu, [\sigma \rho]}}-g_{\nu [\sigma |, \alpha | \rho  ]})
\,.
\end{align}
\end{subequations}

So, if we lower the index of the (1, 3) Riemann tensor we get the desired expression for the all-lower (0, 4) Riemann tensor: 
%
\begin{equation}
  R_{\mu \nu \rho \sigma} = g_{\mu [\sigma |, \nu |\rho ]} - g_{\nu  [\sigma |, \mu | \rho ]} 
\,.
\end{equation}

\subsubsection{Ricci tensor and scalar}

The Ricci tensor is given by: 
%
\begin{equation}
  R_{\mu \nu } = R^{\alpha }_{\mu \alpha \nu } 
  = g^{\alpha \beta }R_{\alpha \mu \beta \nu }
  = g^{\alpha \beta }\qty(g_{\alpha [\nu|, \mu | \beta ]} - g_{\mu [\nu|, \alpha | \beta ]})
\,.
\end{equation}
%
% where we can simplify all the antisymmetric terms in \(\alpha \beta \), since they are contracted with a symmetric tensor.

Now, of the four terms in the sum one simplifies immediately: we know that the Frobenius norm of the metric is constant: \(g^{\alpha \beta }g_{\alpha \beta } \equiv 4\), so if we expand we get \(0=\frac[i]{1}{2} (g^{\alpha \beta }g_{\alpha \beta })_{,\nu \mu } = g^{\alpha \beta }g_{\alpha \beta , \nu \mu } + g^{\alpha \beta }\,_{,\nu }g_{\alpha \beta , \mu }\), but the first derivatives of the metric vanish in the LIF so we only keep the first term, therefore \(g^{\alpha \beta }g_{\alpha \beta , \nu \mu }=0\).

So we have 
%
\begin{subequations}
    \begin{align}
    R_{\mu \nu }
    &= g^{\alpha \beta }\qty(
      g_{\alpha \nu , \mu \beta }
      -\cancelto{}{g_{\alpha \beta , \mu \nu }}
      - g_{\mu \nu , \alpha \beta }
      + g_{\mu \beta , \alpha \nu }
    )  \\
    &=g^{\alpha \beta }\qty(
      2 g_{\alpha (\nu , \mu ) \beta }
      - g_{\mu \nu , \alpha \beta }
    )  \\
    &= 2 g_{\alpha (\nu , \mu )}\,\!^{\alpha } - \square g_{\mu \nu } \label{eq:Ricci-tensor-LIF}
\,,
\end{align}
\end{subequations}
%
where\footnote{All the indices after the comma are derivatives, even when they become upper.} the square \(\square\) denotes the D'alambert operator, \(\square = \partial_{\beta }\partial^{\beta }\).

The Ricci scalar, on the other hand, is given by 
%
\begin{equation} \label{eq:ricci-scalar-LIF} 
  R = g^{\mu \nu }R_{\mu \nu }
  = 2 g^{\mu \nu }g_{\alpha (\nu , \mu )}\,\!^{\alpha } - \cancelto{}{g^{\mu \nu }\square g_{\mu \nu }}
  = 2 \tensor{g}{_{\alpha \nu }^{, \alpha \nu }}
\,,
\end{equation}
%
where the last term vanished since \(\square \qty(g^{\mu \nu }g_{\mu \nu })=0\), and while expanding we get twice \(g^{\mu \nu }\square g_{\mu \nu }\) since the first derivatives are zero.

We neglected the symmetrization in the first term since when contracting with the metric the indices \(\mu \nu \) are automatically symmetrized.

\subsubsection{LIF identities}

%
\begin{subequations}
\begin{align}
  \nabla_{\alpha }\qty(R^{ \alpha }_{\beta } - \frac{R}{2} \delta^{\alpha }_{\beta }) &=  \partial_{\alpha }\qty(g^{\alpha \lambda }R_{\lambda \beta } - \frac{R}{2} \delta^{\alpha }_{\beta })  \\
  &= \eta^{\alpha \lambda }R_{\lambda \beta, \alpha} - \frac{1}{2} R_{,\beta } \label{eq:contracted-Bianchi-LIF}
\,,
\end{align}
\end{subequations}
%
since in the LIF the metric is the Minkowski one, and the Christoffel symbols are zero therefore \(\nabla_{\alpha }= \partial_{\alpha }\).

\subsubsection{Contracted Bianchi identities}

The first term is
%
\begin{equation}
  \eta^{\beta \mu }R_{\mu \nu, \beta } = 
  \eta^{\beta \mu }\qty(2 \tensor{g}{_{\alpha (\nu , \mu )}^{,\alpha }} - \square g_{\mu \nu })_{, \beta}
\,,
\end{equation}
%
which we can expand out: the Minkowski metric raises the derivative with respect to \(x^{\beta }\), and we can also expand the Dalambertian and the symmetrization: we get
%
\begin{equation}
  \tensor{g}{_{\alpha \nu , \mu }^{ \alpha \mu }}
  + \tensor{g}{_{\alpha \mu , \nu }^{ \alpha \mu }}
  - \tensor{g}{_{\mu \nu, \alpha }^{ \alpha \mu }}
  =   + \tensor{g}{_{\alpha \mu , \nu }^{ \alpha \mu }}
  = \partial_{\nu }\qty( \tensor{g}{_{\mu \alpha}^{,\mu \alpha }})
\,,
\end{equation}
%
since, up to a relabeling of indices, the first and last term are equal.

The derivative \(\partial_{\nu }R\), on the other hand, is twice that: \(\partial_{\nu }R = \partial_{\nu }\qty(2 \tensor{g}{_{\mu \alpha }^{, \mu \alpha }})\), as can be directly gathered from \eqref{eq:ricci-scalar-LIF}.

Therefore, the quantity in \eqref{eq:contracted-Bianchi-LIF} is zero.

\subsubsection{Alternative expression}

Since the metric is covariantly constant, we can just bring it inside the derivative:
%
\begin{equation}
  0 = g^{\beta \mu }\nabla_{\alpha }\qty(\tensor{R}{^{\alpha }_{\mu }} - \frac{1}{2} R \delta^{\alpha }_{\mu })
  = \nabla_{\alpha }\qty(R^{\alpha \beta } - \frac{1}{2} R g^{\alpha \beta })
\,.
\end{equation}
%

These are called the \emph{Contracted Bianchi identities}, and can be alternatively derived from the Bianchi identities of the Riemann tensor, \(R_{\mu \nu [\rho \sigma ; \alpha ]} = 0\).

\subsection{Properties of the Ricci tensor}

\subsubsection{Symmetry generality}

The fact that \(T^{\mu \nu } = T^{\nu \mu }\) is frame invariant could be derived plainly from the fact that it is a tensor equation; to be more explicit we can say that under a change of coordinates with Jacobian matrix \(\Lambda \) we have \(\widetilde{T}^{\mu \nu } = \tensor{\Lambda }{^{\mu }_{\alpha }} \tensor{\Lambda }{^{\nu }_{\beta }} T^{\alpha \beta }\); when written in components like this the Jacobians commute, therefore we can just interchange them for both \(\widetilde{T}^{\mu \nu }\) and \(\widetilde{T}^{\nu \mu }\) to recover their equality. 

\subsubsection{Symmetry of the Ricci tensor.}

The expression \eqref{eq:Ricci-tensor-LIF} is manifestly symmetric: part of it is explicitly symmetrized, part is proportional to the metric, which is symmetric.

Therefore, the Ricci tensor is symmetric in any frame.

\subsection{Weak field Einstein equations}

We consider the \emph{weak field} case, when \(g_{\mu \nu }= \eta_{\mu \nu } + h_{\mu \nu }\) with \(\abs{h_{\mu \nu }} \ll 1 \). We will then neglect second and higher order terms in \(h_{\mu \nu }\).

\subsubsection{The gravitational potential}

We want to work towards Newton's equation, which is \(\nabla^2 \Phi = 4 \pi G_N \rho \), where \(\Phi \) is the gravitational potential.
What defines the gravitational potential is its effect on test masses: they accelerate with \(\vec{a} = - \vec{\nabla} \Phi \).

What is the relativistic equivalent of this?
A particle in GR follows a geodesic, a curve without proper acceleration.
Proper acceleration is a four-vector defined by \(a^{\mu }=u^{\nu }\nabla_{\nu }^{\mu }\). If the four-velocity of a particle is \(u^{\mu }\), then for it the equation of geodesic motion reads: 
%
\begin{equation}
  0 = a^{\mu }= u^{\nu }\partial_{\nu }u^{\mu } + \Gamma^{\mu }_{\nu \alpha }u^{\nu} u^{\alpha }
\,,
\end{equation}
%
which can be written in terms of derivatives with respect to proper time: 
%
\begin{equation}
  0 = \dv[2]{x^{ \mu }}{s} + \Gamma^{\mu }_{\nu \alpha } u^{\mu } u^{\alpha }
\,.
\end{equation}

If the speeds are much less than that of light, this can be approximated: \(s \approx t\), and the only terms which contributes to order 0 in \(v\) in the Christoffel symbol sum is the one with \(u^{0} u^{0}\approx 1\): in the end then we get for the spatial components: 
%
\begin{equation}
  0 = \dv[2]{x^{i}}{t} + \Gamma^{i}_{00}
\,,
\end{equation}
%
so we can see that the main contribution in the low-speed limit to the (coordinate!) acceleration of the particle are the symbols \(\Gamma^{i}_{00}\).
The expression for these in terms of the perturbed metric is: 
%
\begin{equation} \label{eq:weak-field-Christoffel}
  \Gamma^{i}_{00} 
  = \frac{1}{2} \eta^{i \alpha } \qty(2 h_{\alpha 0, 0} - h_{00, \alpha }) 
  = -\frac{1}{2} \tensor{h}{_{00}^{,i}}= -\frac{1}{2} \vec{\nabla}^{i} h_{00}
\,,
\end{equation}
%
if we assume stationarity of the metric (which is justified if we are treating a problem such as the gravitational pull on a body on the surface of the Earth) at least up to first order in \(h\).

Putting everything together: we found that 
%
\begin{equation}
  \dv[2]{x^{i}}{t} = \frac{1}{2}\vec{\nabla}^{i}h_{00}
\,,
\end{equation}
%
but the equation which defines the gravitational field is 
%
\begin{equation}
  \dv[2]{x^{i}}{t} = - \vec{\nabla}^{i} \Phi 
\,,
\end{equation}
%
therefore, in the low-gravitational field and low-speed limit, we must identify \(h_{00} = -2 \Phi \).

\subsubsection{Reframing the EFE}

We can contract the EFE with the inverse metric, recalling that \(g^{\mu \nu }g_{\mu \nu }= 4\), to get \(-R = 8 \pi G_N T\). Therefore, they can be reframed by substituting the curvature scalar term with the trace of the stress-energy tensor: 
%
\begin{equation}
  R_{\mu \nu } = 8 \pi G_N \qty(T_{ \mu \nu }- \frac{1}{2} T g_{\mu \nu }) 
\,.
\end{equation}

This will aid us by reducing the number of difficult curvature tensors to compute.

In the low-speed, weak-field limit matter has negligible pressure and is mostly just travelling in the time direction. Therefore, we approximate the stress energy tensor as that of noninteracting dust: \(T^{\mu \nu } = \rho u^{\mu } u^{\nu }\). In our frame matter is almost stationary, therefore we have that \(T_{00} \approx T^{00} \eta_{00} \eta_{00} \approx \rho\).
The trace of the curvature tensor is instead approximately \(T \approx T^{00 } \eta_{00} \approx - \rho\). 

In the end, the 00 component of the equations reads: 
%
\begin{equation}
  R_{00 } = 8 \pi G_N \qty(\rho - \frac{1}{2} (-\rho) \eta_{00 } ) = 4 \pi G_N \rho 
\,.
\end{equation}
%

Now we only need to show that \(R_{00 } = \nabla^2 \Phi \).

\subsubsection{The derivatives in the curvature tensor}

The component we need to calculate is \(R_{00 } = R^{\mu }_{0 \mu 0} = R^{i}_{0i0}\) by antisymmmetry.

Let us recall here the general expression for the Riemann tensor: 
%
\begin{equation}
  R^{\mu }_{\nu \rho \sigma } =
  2 \Gamma^{\mu }_{[\nu| \sigma, |\rho]} + 2 \Gamma^{\alpha }_{\sigma [\nu} \Gamma^{\mu }_{\rho ] \alpha }
\,.
\end{equation}

In our case, this simplifies to 
%
\begin{equation}
  R^{i}_{0i0} = 
  \Gamma^{i}_{00, i} 
  - \Gamma^{i}_{i0, 0}
  + \Gamma^{\alpha }_{00} \Gamma^{i}_{i \alpha }
  - \Gamma^{\alpha }_{0i} \Gamma^{i}_{0 \alpha }
\,.
\end{equation}
%

The second term contains time derivatives, which we decided to neglect by dealing with the stationary case.
The third and fourth term are of second order in \(h\).

The first term, using the expression \eqref{eq:weak-field-Christoffel} and the identification \(h_{00} = -2 \Phi \), is
%
\begin{equation}
  \Gamma^{i}_{00, i} = \qty(-\frac{1}{2}\partial^{i}h_{00})_{,i} = \partial_{i}  \partial^{i} \Phi 
\,,
\end{equation}
%
which is what we wanted to prove.

\end{document}