\documentclass[main.tex]{subfiles}
\begin{document}

\section{Sheet 3}

\subsection{Riemann tensor computations}

\subsubsection{LIF form}

In the LIF, we have \(g_{\mu \nu } = \eta_{\mu \nu }\) and \(= g_{\mu \nu , \alpha }=0\). Therefore, the Christoffel symbols are all zero. 
In the explicit expression of the Riemann tensor, which looks like \(R = \partial \Gamma  + \Gamma \Gamma\) we can drop the second term and keep only the derivatives of the Christoffel symbols, which are nonzero since they depend on the second derivatives of the metric.

If the Christoffel symbols are zero, the computation of the curvature tensor is significantly easier: 
%
\begin{subequations}
\begin{align}
  R^{\mu }_{\nu \rho \sigma }
  &=  2 \partial_{[\nu }\Gamma^{\mu }_{\rho] \sigma }  \\
  &= 2 \partial_{[\nu} \qty(\frac{1}{2} g^{\mu \alpha } \qty(g_{\alpha |\rho], \sigma } + g_{\alpha \sigma , |\rho ]} - g_{\sigma |\rho], \alpha }))  \\
  &= g^{\mu \alpha }\qty(g_{\alpha [\rho|, \sigma | \nu     ]} + \cancelto{}{g_{\alpha \sigma, [\rho \nu ]}}-g_{\sigma [\rho |, \alpha | \nu ]})
\,.
\end{align}
\end{subequations}

So, if we lower the index of the Riemann tensor we get the desired expression for the all-lower tensor: 
%
\begin{equation}
  R_{\mu \nu \rho \sigma} = g_{\mu [\rho|, \sigma | \nu ]} - g_{\sigma [\rho|, \mu | \nu ]} 
\,.
\end{equation}

\subsubsection{Ricci tensor and scalar}

The Ricci tensor is given by: 
%
\begin{equation}
  R_{\mu \nu } = R^{\alpha }_{\mu \alpha \nu } 
  = g^{\alpha \beta }R_{\alpha \mu \beta \nu }
  = g^{\alpha \beta }\qty(g_{\alpha [\beta |, \nu | \mu ]} - g_{\nu  [\beta |, \alpha  | \mu ]})
\,.
\end{equation}
%
% where we can simplify all the antisymmetric terms in \(\alpha \beta \), since they are contracted with a symmetric tensor.

Now, of the four terms in the sum one simplifies immediately: we know that the Frobenius norm of the metric is constant: \(g^{\alpha \beta }g_{\alpha \beta } \equiv 4\), so \(g^{\alpha \beta }g_{\alpha \beta , \nu \mu }\) is one of the two equal terms we get by expanding \(\frac[i]{1}{2} (g^{\alpha \beta }g_{\alpha \beta })_{,\nu \mu } = 0\).

So we have 
%
\begin{subequations}
    \begin{align}
    R_{\mu \nu }
    &= g^{\alpha \beta }\qty(
        \cancelto{}{g_{\alpha \beta , \nu \mu }}
        - g_{\alpha \mu  , \nu \beta  }
        - g_{\nu  \beta , \alpha  \mu }
        + g_{\nu  \mu  , \alpha  \beta }      
        )  \\
    &= g^{\alpha \beta }g_{\mu \nu, \alpha \beta }
    - 2 g^{\alpha \beta } g_{\alpha (\mu, \nu) \beta }
\,,
\end{align}
\end{subequations}
%
but \(g^{\alpha \beta }g_{\alpha \mu } = \delta^{\beta }_{\mu }\), therefore the second term is proportional to \(\delta^{\beta }_{(\mu, \nu ) \beta }\), but the entries of the identity are constants, therefore this term is zero.
Also, recall that in the LIF the metric is just the Minkiwski one:
So in the end we get: 
%
\begin{equation}
  R_{\mu \nu } = \eta^{\alpha \beta }g_{\mu \nu, \alpha \beta   }
\,,
\end{equation}
%
which also proves the symmetry of the Ricci tensor by the symmetry of the metric.

The Ricci scalar is then just given by 
%
\begin{equation}
  R = \eta ^{\mu \nu }\eta ^{\alpha \beta }g_{\mu \nu , \alpha \beta }
\,.
\end{equation}

\subsubsection{Contracted Bianchi identities}



\end{document}