\documentclass[main.tex]{subfiles}
\begin{document}

\section{Sheet 6}

\subsection{Sphere pole Riemann coordinates}

Recall from section \ref{sec:spherical-surface-curvature} the spherical surface metric in the coordinates \((\theta,  \varphi )\): 
%
\begin{align}
  g_{\mu \nu } = \left[\begin{array}{cc}
  R^2 & 0 \\ 
  0 & R^2 \sin^2 \theta 
  \end{array}\right]
\,,
\end{align}
%
and the Christoffel symbols, which are all zero except for \(\Gamma^{0}_{11}= - \sin \theta \cos \theta \) and \(\Gamma^{1}_{01} = \Gamma^{1}_{10} = 1/ \tan \theta \).

We can plug these into the geodesic equation \(u^{\mu } \nabla_{\mu } u^{\nu }\), which comes out to be 
%
\begin{subequations}
\begin{align}
  \ddot{\theta } &= + \dot{\varphi }^2 \sin \theta \cos \theta  \\
  \ddot{\varphi } &= - 2 \frac{\dot{\varphi } \dot{\theta } }{\tan(\theta )} 
\,,
\end{align}
\end{subequations}
%
for a trajectory \((\theta (s), \varphi (s))\) with velocity \((\dot{\theta }, \dot{\varphi })\): dots denote derivatives with respect to \(s\).

Now, we want to check whether parallels and meridians are geodesics. First of all, we want to choose a parameter such that the velocity is of constant norm \(1\): the equation to satisfy is 
%
\begin{align}
  R^2 \left[\begin{array}{cc}
  \dot{\theta} & \dot{\varphi}
  \end{array}\right]
  \left[\begin{array}{cc}
  1 & 0 \\ 
  0 & \sin^2 \theta 
  \end{array}\right]
  \left[\begin{array}{c}
  \dot{\theta} \\ 
  \dot{\varphi}
  \end{array}\right]
  \equiv 1
\,,
\end{align}
%
or \(\dot{\theta}^2 + \dot{\varphi}^2 \sin^2 \theta = R^{-2}\).
Meridians have constant \(\varphi \): for them, then, \(\dot{\theta}^2 = R^{-2} \), so an appropriate parametrization is \((\theta(s), \varphi(s)) = (s/R, \varphi_0 )\).
Parallels have constant \(\theta \): by an analogous line of reasoning, we parametrize them as \((\theta (s), \varphi (s) )= (\theta_0, s / R \sin\theta  )\). 

One can readily check that for meridians both of the geodesic equations are identities, while for parallels the second one is an identity but the first reads \(0= \cos \theta  / \sin \theta \): it can only be satisfied if \(\theta = \pi /2\).
This makes sense: geodesics on a sphere are great circles, and the only parallel which is a great circle is the equator. 

\subsubsection{Riemann coordinates}

The coordinates we wish to use are 
%
\begin{subequations}
\begin{align}
  x &= R \theta \cos \varphi   \\
  y &= R \theta \sin \varphi 
\,.
\end{align}
\end{subequations}
%
% whose tangent vectors at the north pole (i.e. \(\theta = 0\)) are:
% %
% \begin{subequations}
% \begin{align}
%   \dv[]{}{x} &= R \cos \varphi \dv[]{}{\theta } 
%   - R \theta \sin \varphi \dv[]{}{\varphi } 
%   = R \cos \varphi \dv[]{}{\theta } \\ 
%   \dv[]{}{y} &= R \sin \varphi \dv[]{}{\theta } 
%   + R \theta \cos \varphi \dv[]{}{\varphi }
% \,,
% \end{align}
% \end{subequations}
% %

They are in the form \(x^{\alpha } = \theta n^{\alpha }\), for vectors \(n^{\alpha } = R (\cos \varphi, \sin \varphi )\).
An orthonormal basis for these vectors can be found by selecting \(\varphi = 0, \pi /2\). 

As we have shown before, the coordinates \(x^{\alpha }\) describe geodesics if we consider them for fixed \(\varphi \) and with parameter \(\theta \), since they are meridians.

\subsubsection{Metric computation}

The metric transforms as 
%
\begin{align}
  g_{\mu \nu }^{\prime } 
  = \pdv{x^{\alpha }}{x^{\prime \mu }}
  \pdv{x^{\beta }}{x^{\prime  \nu }}
  g_{\alpha \beta }
\,,
\end{align}
%
so we need the inverse Jacobian, which is expressed in terms of the new coordinates \(x^{\prime \mu } = (x, y)\) and the old ones \(x^{\mu } = (\theta , \varphi )\):
%
\begin{align}
  \pdv{x^{\alpha }}{x^{\prime \mu }}
  = \left[\begin{array}{cc}
  \frac{1}{R} \frac{x}{\sqrt{x^2 + y^2}} & 
  \frac{1}{R} \frac{y}{\sqrt{x^2 + y^2}} \\ 
  \frac{-y / x^2}{1 + (y/x)^2} & 
  \frac{1/x}{1 + (y/x)^2}
  \end{array}\right]
\,,
\end{align}
%
so we get, using the expansion \(\sin^2\theta \sim \theta^2 - \theta^{4} /3 + O (\theta^{6})\) and the identification \(R^2\theta^2 = x^2+y^2\):
%
\begin{subequations}
\begin{align}
  g_{00}^{\prime } &= R^2 \qty(\pdv{x^{0}}{x^{\prime 0}})^2
  + R^2 \sin^2 \theta \qty(\pdv{x^{1}}{x^{\prime 0}})^2  \\
  &= \frac{x^2}{x^2+y^2} + R^2 \sin^2 \theta \frac{y^2}{x^{4} \qty(1 + (y/x)^2)}  \\
  &=\frac{1}{x^2+y^2} \qty(x^2 + \frac{R^2 \sin^2\theta y^2}{x^2+y^2})  \\
  &= 1 - \frac{y^2}{3R^2} + O((x^2+y^2)y^2)
\,,
\end{align}
\end{subequations}
%
while for the off-diagonal elements \(g_{01}^{\prime } = g_{10}^{\prime }\): 
%
\begin{subequations}
\begin{align}
  g_{01}^{\prime } &= R^2 \pdv{x^{0}}{x^{\prime 0}} \pdv{x^{0}}{x^{\prime 1}} 
  + R^2 \sin^2 \theta \pdv{x^{1}}{x^{\prime 0}} \pdv{x^{1}}{x^{\prime 1}}  \\
  &= \frac{xy}{x^2+y^2} + R^2 \sin^2\theta \qty(- \frac{xy}{(x^2+y^2)^2})  \\
  &= \frac{xy}{x^2+y^2} \qty(1 - \frac{R^2 \sin^2\theta }{x^2+y^2})  \\
  &= \frac{xy}{x^2+y^2} \qty(\frac{(x^2+y^2)^2}{3R^2 (x^2+y^2)}) + O(x^2+y^2) \\
  &= \frac{xy}{3R^2} + O(x^2+y^2)
\,,
\end{align}
\end{subequations}
%
and lastly for the element \(g_{11}^{\prime }\): 
%
\begin{subequations}
\begin{align}
  g_{11}^{\prime } &=  R^2 \qty(\pdv{x^{0}}{x^{\prime 1}})^2
  + R^2 \sin^2 \theta \qty(\pdv{x^{1}}{x^{\prime 1}})^2  \\
  &= \frac{y^2}{x^2+y^2} + R^2 \sin^2 \theta \frac{1}{x^{2} \qty(1 + (y/x)^2)^2}  \\
  &= \frac{1}{x^2+y^2} \qty(y^2 +  R^2 \sin^2\theta \frac{x^2}{x^2+y^2})  \\
  &= 1 - \frac{x^2}{3R^2} + O ((x^2+y^2) x^2)
  \,.
\end{align}
\end{subequations}

At the north pole \(x=y=0\), so there \(g^{\prime }_{\mu \nu }= \delta_{\mu \nu }\), and all the first derivatives calculated there vanish since there are no first order terms.

\subsubsection{Scalar curvature calculation}

The expression we have for the scalar curvature in a LIF is given in equation \eqref{eq:ricci-scalar-LIF}.

We can evaluate it for \(g_{\mu \nu }^{\prime }\). Do note  that the non-differentiated metric can be identified with the identity, and derivatives with upper and lower indices are the same. So, we get:
%
\begin{align}
  R _{\text{Ric}} = \tensor{g}{_{\alpha \nu , }^{\alpha \nu }} - \delta^{\mu \nu } \square g_{\mu \nu }
\,,
\end{align}
%
where the only nonvanishing terms are: 
%
\begin{align}
  \tensor{g}{_{\alpha \nu,}^{\alpha \nu }} = 2 \pdv[2]{}{x}{y} \qty(\frac{xy}{3R^2}) = \frac{2}{3R^2}
\,
\end{align}
%
and what was denoted as the Dalambertian before is just the Laplacian: \(\square = \delta^{\mu \nu }\partial_{\mu } \partial_{\nu }= \partial^2_{xx} + \partial^2_{yy}\):
%
\begin{align}
  \delta^{\mu \nu } \square g_{\mu \nu }
  = \partial_{xx}^2 g_{11} + \partial_{yy}^2 g_{00}
  = - 2 \frac{2}{3R^2} 
\,,
\end{align}
%
so in the end we get 
%
\begin{align}
  R _{\text{Ric}} = \frac{2}{3R^2} + \frac{4}{3R^2} = \frac{2}{R^2}
\,,
\end{align}
%
which means that the curvature decreases as the radius increases, as we might expect.

\subsection{Schwarzschild metric curvature}

\subsubsection{Christoffel symbols}

The computation is tedious and not particularly enlightening: we start from the metric
\footnote{There is a typo in the homework assignment: the coefficients are written as functions of time.}
%
\begin{align}
  \dd{s^2} = - A(r) \dd{t^2}
  + B(r) \dd{r^2}
  + r^2 \qty(\dd{\theta^2} + \sin^2\theta \dd{\varphi^2})
\,,
\end{align}
%
and compute the Christoffel symbols with the usual formula: 
%
\begin{align}
  \Gamma^{\mu }_{\nu \rho } =
  \frac{1}{2} g^{\mu \alpha }\qty(g_{\alpha \nu , \rho }
  + g_{\alpha \rho, \nu } - g_{\nu \rho , \alpha })
\,,
\end{align}
%
where fortunately, since the metric is diagonal, we only need to compute one term in the sum (that is, \(\mu \equiv \alpha \) always), and the two indices in the metric must be equal in order for the term to not vanish. 

The metric only depends on \(\theta \) and \(r\), so any derivatives with respect to \(t\) and \(\varphi \) are to be discarded. 

With this out of the way, we start computing the \(40\) independent symbols and find that the nonzero ones are (denoting differentiation with respect to \(r\) with a prime): 
%
\begin{subequations} \label{eq:schwarzschild-christoffel}
\begin{align}
  \Gamma^{t}_{rt} &= \frac{A^{\prime }}{2A}  &
  \Gamma^{r}_{rr} &= \frac{B^{\prime }}{2B}  \\
  \Gamma^{r}_{tt} &= \frac{A^{\prime }}{2B}  &
  \Gamma^{\theta }_{r \theta } &= \frac{1}{r}  \\
  \Gamma^{r}_{\theta \theta } &= -\frac{r}{B}  &
  \Gamma^{\varphi }_{r \varphi } &= \frac{1}{r}  \\
  \Gamma^{r}_{\varphi \varphi } &= - \frac{r}{B} \sin^2 \theta  &
  \Gamma^{\theta }_{\varphi \varphi } &= - \sin \theta \cos \theta  \\
  \Gamma^{\varphi }_{\theta \varphi } &= \frac{\cos \theta }{\sin \theta }
\,.
\end{align}
\end{subequations}

\subsubsection{Ricci component}

We want to compute \(R^{\mu }_{t \mu t }\), and in order to do so we must find the three components \(R^{i}_{tit}\) with varying \(i\) and sum them (here \(i = r, \theta,\varphi \)), since \(R^{t}_{ttt}\) vanishes by antisymmetry. 

In general we have: 
%
\begin{subequations}
\begin{align}
  R^{i}_{tit} &= 2 \qty( \Gamma^{i}_{[t| t, |i]}  + \Gamma^{\alpha }_{t[t} \Gamma^{i}_{i] \alpha })  \\
  &= \Gamma^{i}_{tt, i} - \cancelto{}{\Gamma^{i}_{it,t}}
  + \Gamma^{\alpha }_{tt} \Gamma^{i}_{i\alpha }
  - \Gamma^{\alpha }_{ti} \Gamma^{i}_{t\alpha }
\,.
\end{align}
\end{subequations}

Note that the index \(i\) is consider not to be summed here, we are writing a formula for the components of the Riemann tensor; although the expression holds when summing over \(i\) as well. 

So, we can compute this for the specific values of \(i\): for \(i=r\) we have 
%
\begin{subequations}
\begin{align}
  R^{r}_{trt} &= \Gamma^{r}_{tt, r} + \Gamma^{\alpha}_{tt} 
  \Gamma^{r}_{r \alpha } - \Gamma^{\alpha }_{tr} \Gamma^{r}_{t \alpha }  \\
  &= \qty(\frac{A^{\prime }}{2B})^{\prime } + \frac{A^{\prime }}{2B} \frac{B^{\prime }}{2B} - \frac{A^{\prime }}{2A} \frac{A^{\prime }}{2B}  \\
  &= \frac{A^{\prime \prime}}{2B} - \frac{A^{\prime }}{2B^2} B^{\prime } + \frac{A^{\prime }}{4B} \qty(\frac{B^{\prime }}{B} - \frac{A^{\prime}}{A}) \\ 
  &= \frac{A^{\prime \prime}}{2B} - \frac{A^{\prime }}{4B} \qty(\frac{A^{\prime}}{A} + \frac{B^{\prime }}{B})
  \,,
\end{align}
\end{subequations}
%
for \(i= \theta \) instead 
%
\begin{subequations}
\begin{align}
  R^{\theta }_{t \theta  t} &= \cancelto{}{\Gamma^{\theta}_{tt, \theta}} + \Gamma^{\alpha}_{tt} 
  \Gamma^{\theta}_{\theta \alpha } - \cancelto{}{\Gamma^{\alpha }_{t\theta} \Gamma^{i}_{t \alpha }}  \\
  &= \frac{A^{\prime }}{2B} \frac{1}{r}
  \,,
\end{align}
\end{subequations}
%
and for \(i = \varphi \):  
%
\begin{subequations}
  \begin{align}
    R^{\varphi }_{t \varphi t} &= \cancelto{}{\Gamma^{\varphi}_{tt, \varphi}} + \Gamma^{\alpha}_{tt} 
    \Gamma^{\varphi}_{\varphi \alpha } - \cancelto{}{\Gamma^{\alpha }_{t\varphi} \Gamma^{i}_{t \alpha }}  \\
    &= \frac{A^{\prime }}{2B} \frac{1}{r}
    \,,
  \end{align}
\end{subequations}
%
so our final solution is 
%
\begin{align}
  R_{00} = \sum_i R^{i}_{0i0} = \frac{A^{\prime \prime}}{2B} 
  - \frac{A^{\prime }}{4B} \frac{(AB)^{\prime }}{AB} + \frac{A^{\prime }}{Br}
\,.
\end{align}
%

\subsection{Schwarzschild geometry orbits}

The derivation up to the equation for the perturbed orbit equation is documented in the lecture notes, I might copy it here later, but for now one can find it there.

During the lecture we got up to the first order equation for the perturbation \(w\) for the orbit \(u\), written in the form \(u(\varphi ) = u_{c} \qty(1+w(\varphi ))\): 
%
\begin{align}
    \dv[2]{w}{\varphi }  = (6GMu_c-1) w
\,,
\end{align}
%
which is in the form \(\ddot{w} + \omega^2 w = 0\), for \(\omega^2 = 1- 6GMu_c\). Now, we know that the first order equation must be complemented by the zeroth order one: 
%
\begin{align}
    u_c = \frac{GM}{l^2} + 3GM u_c^{2}
\,,
\end{align}
%
which can be solved for \(u_c\) to yield: 
%
\begin{align}
  u_c = \frac{1 \pm \sqrt{1 - 3 \times 4 \frac{G^2 M^2}{l^2}}}{6GM}
\,,
\end{align}
%
therefore the square angular velocity of the perturbation's evolution is: 
%
\begin{align}
  \omega^2 = 1 - \cancelto{}{6GM} \qty(\frac{1 \pm \sqrt{1 - 12 \frac{G^2 M^2}{l^2}}}{ \cancelto{}{6GM}})
  = \pm \sqrt{1 - 12 \frac{G^2 M^2}{l^2}}
\,.
\end{align}

The solution with the minus sign has no meaning for us, since the solution we want to consider must be stable, with positive \(\omega^2\). So, the angular velocity is 
%
\begin{align}
  \omega = \qty(1 - 12 \frac{G^2M^2}{l^2})^{1/4}
\,,
\end{align}
%
and we know that angular velocity and period are related by \(T = 2 \pi / \omega \): therefore we get 
%
\begin{align}
  T = 2 \pi \qty(1 - 12 \frac{G^2M^2}{l^2})^{-1/4}
\,,
\end{align}
%
which we can Taylor expand: at \(x=0\) we have 
%
\begin{align}
    (1-12x)^{-1/4} = 1 - \frac{1}{4} (1-12 \times 0)^{-5/4} (-12x) + O(x^2) = 1 + 3x + O(x^2 )
\,.
\end{align}
%

Therefore: 
%
\begin{align}
  T = 2 \pi \qty(1 + 3 \qty(\frac{GM}{l})^2) + O\qty(\qty(\frac{GM}{l})^{4})
\,,
\end{align}
%
which is approximately \(2 \pi \) as we should expect: the Newtonian approximation is \(l \gg GM\), and Newtonian orbits have a period of exactly \(2 \pi \). Then we can read off the first-order correction directly from the first term in the expansion: it is 
%
\begin{align}
  \delta \varphi = 6 \pi \qty(\frac{GM}{l})^2
\,.
\end{align}
%

The \(M\) here is the mass of the central object, while \(l\) is the angular momentum of the orbit. 



\end{document}