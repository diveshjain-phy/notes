\documentclass[main.tex]{subfiles}
\begin{document}

\section{Sheet 6}

\subsection{Sphere pole Riemann coordinates}

Recall from section \ref{sec:spherical-surface-curvature} the spherical surface metric in the coordinates \((\theta,  \varphi )\): 
%
\begin{align}
  g_{\mu \nu } = \left[\begin{array}{cc}
  R^2 & 0 \\ 
  0 & R^2 \sin^2 \theta 
  \end{array}\right]
\,,
\end{align}
%
and the Christoffel symbols, which are all zero except for \(\Gamma^{0}_{11}= - \sin \theta \cos \theta \) and \(\Gamma^{1}_{01} = \Gamma^{1}_{10} = 1/ \tan \theta \).

We can plug these into the geodesic equation \(u^{\mu } \nabla_{\mu } u^{\nu }\), which comes out to be 
%
\begin{subequations}
\begin{align}
  \ddot{\theta } &= + \dot{\varphi }^2 \sin \theta \cos \theta  \\
  \ddot{\varphi } &= - 2 \frac{\dot{\varphi } \dot{\theta } }{\tan(\theta )} 
\,,
\end{align}
\end{subequations}
%
for a trajectory \((\theta (s), \varphi (s))\) with velocity \((\dot{\theta }, \dot{\varphi })\): dots denote derivatives with respect to \(s\).

Now, we want to check whether parallels and meridians are geodesics. First of all, we want to choose a parameter such that the velocity is of constant norm \(1\): the equation to satisfy is 
%
\begin{align}
  R^2 \left[\begin{array}{cc}
  \dot{\theta} & \dot{\varphi}
  \end{array}\right]
  \left[\begin{array}{cc}
  1 & 0 \\ 
  0 & \sin^2 \theta 
  \end{array}\right]
  \left[\begin{array}{c}
  \dot{\theta} \\ 
  \dot{\varphi}
  \end{array}\right]
  \equiv 1
\,,
\end{align}
%
or \(\dot{\theta}^2 + \dot{\varphi}^2 \sin^2 \theta = R^{-2}\).
Meridians have constant \(\varphi \): for them, then, \(\dot{\theta}^2 = R^{-2} \), so an appropriate parametrization is \((\theta(s), \varphi(s)) = (s/R, \varphi_0 )\).
Parallels have constant \(\theta \): by an analogous line of reasoning, we parametrize them as \((\theta (s), \varphi (s) )= (\theta_0, s / R \sin\theta  )\). 

One can readily check that for meridians both of the geodesic equations are identities, while for parallels the second one is an identity but the first reads \(0= \cos \theta  / \sin \theta \): it can only be satisfied if \(\theta = \pi /2\).
This makes sense: geodesics on a sphere are great circles, and the only parallel which is a great circle is the equator. 

\subsubsection{Riemann coordinates}

The coordinates we wish to use are 
%
\begin{subequations}
\begin{align}
  x &= R \theta \cos \varphi   \\
  y &= R \theta \sin \varphi 
\,.
\end{align}
\end{subequations}
%
% whose tangent vectors at the north pole (i.e. \(\theta = 0\)) are:
% %
% \begin{subequations}
% \begin{align}
%   \dv[]{}{x} &= R \cos \varphi \dv[]{}{\theta } 
%   - R \theta \sin \varphi \dv[]{}{\varphi } 
%   = R \cos \varphi \dv[]{}{\theta } \\ 
%   \dv[]{}{y} &= R \sin \varphi \dv[]{}{\theta } 
%   + R \theta \cos \varphi \dv[]{}{\varphi }
% \,,
% \end{align}
% \end{subequations}
% %

They are in the form \(x^{\alpha } = \theta n^{\alpha }\), for vectors \(n^{\alpha } = R (\cos \varphi, \sin \varphi )\).
An orthonormal basis for these vectors can be found by selecting \(\varphi = 0, \pi /2\). 

As we have shown before, the coordinates \(x^{\alpha }\) describe geodesics if we consider them for fixed \(\varphi \) and with parameter \(\theta \), since they are meridians.

\subsubsection{Metric computation}

The metric transforms as 
%
\begin{align}
  g_{\mu \nu }^{\prime } 
  = \pdv{x^{\alpha }}{x^{\prime \mu }}
  \pdv{x^{\beta }}{x^{\prime  \nu }}
  g_{\alpha \beta }
\,,
\end{align}
%
so we need the inverse Jacobian, which is expressed in terms of the coordinates \(x^{\prime \mu } = \)
%
\begin{align}
  \pdv{x^{\alpha }}{x^{\prime \mu }}
  = \left[\begin{array}{cc}
  \frac{1}{R} \frac{x}{\sqrt{x^2 + y^2}} & 
  \frac{1}{R} \frac{y}{\sqrt{x^2 + y^2}} \\ 
  \frac{-y / x^2}{1 + (y/x)^2} & 
  \frac{1/x}{1 + (y/x)^2}
  \end{array}\right]
\,,
\end{align}
%


\subsection{Schwarzschild metric curvature}

\subsection{Schwarzschild geometry orbits}

The derivation up to the equation for the perturbed orbit equation is documented in the lecture notes, I might copy it here later, but for now one can find it there.

During the lecture we got up to the first order equation for the perturbation \(w\) for the orbit \(u\), written in the form \(u(\varphi ) = u_{c} \qty(1+w(\varphi ))\): 
%
\begin{align}
    \dv[2]{w}{\varphi }  = (6GMu_c-1) w
\,,
\end{align}
%
which is in the form \(\ddot{w} + \omega^2 w = 0\), for \(\omega^2 = 1- 6GMu_c\). Now, we know that the first order equation must be complemented by the zeroth order one: 
%
\begin{align}
    u_c = \frac{GM}{l^2} + 3GM u_c^{2}
\,,
\end{align}
%
which can be solved for \(u_c\) to yield: 
%
\begin{align}
  u_c = \frac{1 \pm \sqrt{1 - 3 \times 4 \frac{G^2 M^2}{l^2}}}{6GM}
\,,
\end{align}
%
therefore the square angular velocity of the perturbation's evolution is: 
%
\begin{align}
  \omega^2 = 1 - \cancelto{}{6GM} \qty(\frac{1 \pm \sqrt{1 - 12 \frac{G^2 M^2}{l^2}}}{ \cancelto{}{6GM}})
  = \pm \sqrt{1 - 12 \frac{G^2 M^2}{l^2}}
\,.
\end{align}

The solution with the minus sign has no meaning for us, since the solution we want to consider must be stable, with positive \(\omega^2\). So, the angular velocity is 
%
\begin{align}
  \omega = \qty(1 - 12 \frac{G^2M^2}{l^2})^{1/4}
\,,
\end{align}
%
and we know that angular velocity and period are related by \(T = 2 \pi / \omega \): therefore we get 
%
\begin{align}
  T = 2 \pi \qty(1 - 12 \frac{G^2M^2}{l^2})^{-1/4}
\,,
\end{align}
%
which we can Taylor expand: at \(x=0\) we have 
%
\begin{align}
    (1-12x)^{-1/4} = 1 - \frac[]{1}{4} (1-12 \times 0)^{-5/4} (-12x) + O(x^2) = 1 + 3x + O(x^2 )
\,.
\end{align}
%

Therefore: 
%
\begin{align}
  T = 2 \pi \qty(1 + 3 \qty(\frac{GM}{l})^2) + O\qty(\qty(\frac{GM}{l})^{4})
\,,
\end{align}
%
which is approximately \(2 \pi \) as we should expect: the Newtonian approximation is \(l \gg GM\), and Newtonian orbits have a period of exactly \(2 \pi \). Then we can read off the first-order correction directly from the first term in the expansion: it is 
%
\begin{align}
  \delta \varphi = 6 \pi \qty(\frac{GM}{l})^2
\,.
\end{align}
%


\end{document}