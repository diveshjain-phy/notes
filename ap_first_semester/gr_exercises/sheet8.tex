\documentclass[main.tex]{subfiles}
\begin{document}

\section{Sheet 8}

\subsection{Acceleration in Rindler coordinates}

Rindler coordinates for the Minkowski (1, 1) space, originally parametrized with \((t, x)\), are defined by 
%
\begin{align}
  \begin{cases}
      t = \rho \sinh \eta \\
      x = \rho \cosh \eta 
  \end{cases}
\,.
\end{align}

The metric is given by 
%
\begin{align}
  \dd{s^2} = - \dd{t^2} + \dd{x^2} =
  - \rho^2 \dd{\eta^2} + \dd{\rho^2}
\,.
\end{align}

We defined Rindler coordinates so that our uniformly accelerated observer would have velocity only along \(\eta \), so we can find the general expression of the 4-velocity by computing the normalization: since it will look like \(u^{\alpha } = (N, 0)\) for some \(N\) and we want \(u^{\alpha } u_{\alpha } = -1\), we get that \(N^2(-\rho^2) = -1 \), which means \(N = 1/ \rho \). 

The only nonzero Christoffel symbols are \(\Gamma^{\eta }_{\eta \rho } = 1/\rho \) and \(\Gamma^{\rho }_{\eta \eta }= \rho \). 

The computation of the acceleration then can be started: 
%
\begin{align}
  \dv{}{\tau } u^{\mu } &= u^{\nu } \nabla_{\nu } u^{\mu }  \\
  &= u^{\eta } \qty(\cancelto{}{\partial_{\eta } u^{\mu }} + \Gamma^{\mu }_{\eta \nu } u^{\nu })  \\
  &= \frac{1}{\rho^2} \Gamma^{\mu }_{\eta \eta }
\,,
\end{align}
%
so the only nonzero component is the one where \(\mu = \rho \), which is \(a^{\rho } = 1/\rho \), or in other words
%
\begin{align}
  a^{\mu } = \left[\begin{array}{c}
  0 \\ 
  \kappa 
  \end{array}\right]
\,,
\end{align}
%
where \(\kappa = 1/ \rho \) is the modulus of the 4-acceleration. Indeed 
%
\begin{align}
  a^{\mu } a^{\nu } g_{\mu \nu } = \kappa^2 g_{\rho \rho } = \kappa^2
\,.
\end{align}
%

\subsection{Acceleration in Schwarzschild motion}

\subsubsection{Acceleration computation}

The acceleration is given in general by 
%
\begin{align}
  a^{\mu } = u^{\nu } \qty(\partial_{\nu } u^{\mu } + \Gamma^{\mu  }_{\nu \rho } u^{\rho })
\,,
\end{align}
%
and we can see that in this formula we can only have a nonzero result if \(\mu = t, r, \varphi \), \(\nu = t, \varphi \) and \(\rho = t, \varphi \). 

Now we use the following facts: the velocity is both stationary (\(\partial_{t} u^{\mu } = 0\)), rotationally symmetric (\(\partial_{\varphi } u^{\mu } = 0\)), and the the Christoffel symbols with the upper index different from \(r\) must have at least one \(r\) between the lower ones in order to be nonzero, therefore the terms containing them must be zero since they must then be contracted with \(u^{r} =0 \).
This gives us that the only nonzero component of the acceleration is 
%
\begin{align}
  a^{r } = u^{\nu } u^{\nu } \Gamma^{r}_{\nu \nu }
  = \qty(u^{t})^2 \Gamma^{r}_{tt} + \qty(u^{\varphi })^2 \Gamma^{r}_{\varphi \varphi }
\,.
\end{align}

This already confirms that \(u^{\mu } a_{\mu } =0\), since the Schwarzschild metric is diagonal.

The symbols which appear are 
%
\begin{align}
  \Gamma^{r}_{tt} = \frac{1}{2} g^{rr} \qty(- g_{tt, r})
  = \qty(1 - \frac{2GM}{r_{*}}) \frac{GM}{r^2_{*}}
\,
\end{align}
%
and 
%
\begin{align}
  \Gamma^{r}_{\varphi  \varphi } = 
  \frac{1}{2} g^{rr} \qty(-g_{\varphi \varphi , r})
  = - \qty(1 - \frac{2GM}{r_{*}}) r_{*} 
\,.
\end{align}

What are the components of the \(4-\)velocity? We can use the normalization \(u^{\mu } u_{\mu } = -1\); inserting the variable name \(u^{\varphi } = \omega \) this gives us 
%
\begin{align}
  u^{t} u^{t} g_{tt} + \omega^2 g_{\varphi \varphi } = -1
\,,
\end{align}
%
which can be written as 
%
\begin{align}
  u^{t} u^{t} = \frac{1 + \omega^2 r_{*}^2}{1 - \frac{2GM}{r_*}}
\,.
\end{align}
%
Let us plug this expression into the acceleration one: 
%
\begin{align}
  a^{r} = \frac{1 + \omega^2 r_{*}^2}{1 - \frac{2GM}{r_*}}
  \qty(1 - \frac{2GM}{r_{*}}) \frac{GM}{r_{*}^2} - \omega^2 \qty(1 - \frac{2GM}{r_{*}}) r_{*}
\,,
\end{align}
%
which simplifies to 
%
\begin{align}
  a^{r} &= \frac{GM}{r_{*}^2} + \omega^2 GM - \omega^2 r_{*} + 2GM \omega^2  \\
  &= \frac{GM}{r_{*}^2} + 3GM \omega^2 - \omega^2 r_{*}
\,.
\end{align}
%
If we set it to zero we find a relation which we can express in terms of \(l = g_{\varphi \varphi } u^{\varphi } = r^2_{*} \omega \). We get 
%
\begin{align}
  0 &= \frac{GM}{r_{*}^2} + 3GM \frac{l^2}{r^{4}} - \frac{l^2}{r_{*}^{3}}  \\
  &= GMr_{*}^2 +3GMl^2 - l^2 r_{*}
\,,
\end{align}
%
which  is the relation we know for the angular momentum in circular orbits. 

\subsubsection{Acceleration modulus limits}

The modulus of the acceleration is given by \(\abs{\vec{a}}^2 = a^{r} a^{r} g_{rr}\), so we will have 
%
\begin{align}
  \abs{\vec{a}}^2 = \frac{1}{1- \frac{2GM}{r_{*}}} \qty(
  \frac{GM}{r_{*}^2} + 3GM \omega^2 - \omega^2 r_{*})^2
\,,
\end{align}
%
which can be written in terms of the adimensionalized variables \(R = r / 2GM\) and \(\widetilde{\omega} = 2GM \omega \): 
%
\begin{align}
  \abs{\vec{a}}^2 = \frac{1}{(4GM)^2} \frac{R_{*}}{R_{*} - 1}
  \qty(\frac{1}{R_{*}^2} + \widetilde{\omega}^2 \qty(3 - 2R_{*}))^2
\,.
\end{align}
%
This diverges as \(R_{*} \rightarrow 1\), as we might expect: a \emph{stationary} observer with respect to the Schwarzschild radial coordinate must have ever more acceleration in order to stay on their non-geodesic path. 

Of we set \(\omega = 0\) it becomes 
%
\begin{align}
  \abs{\vec{a}}^2 = \frac{1}{(4GM)^2} \frac{R_{*}}{R_{*}- 1} \frac{1}{R_{*}^4}
\,,
\end{align}
%
while for \(R_{*} \gg 1\) it becomes 
%
\begin{align}
  \abs{\vec{a}}^2 \sim \frac{1}{(4GM)^2} \qty(\frac{1}{R_{*}^2}
   - 2\widetilde{\omega}^2 R_{*})^2
\,,
\end{align}
%
which can become 0 (that is, we have an orbit, a geodesic) if we set \(2\widetilde{\omega}^2 R_{*}^3= 1\), or \(\omega^2 r_{*}^3 = GM\), (the spherical orbit formulation of) Kepler's third law. 

We neglected only the \(3 \widetilde{\omega}^2 \) term since it is the smallest one: in terms of powers of \((GM)\) and of \(c\) we have: \(R_{*}^{-2} \sim (GM)^{-2}\), \(\widetilde{\omega}^2\sim (GM)^{-2} c^{-2}\) while \(\widetilde{\omega}^2 R_{*}^2 \sim (GM) c^{-2}\).

\subsubsection{Some comments on orbits (complement)}

Another interesting thing to note is the fact that at \(R_{*} = 3/2\) the acceleration becomes independent of \(\omega \), and at \(R_{*} < 3/2\) the effect of the rotation is to \emph{increase} the acceleration instead of decreasing it; this corresponds to the fact that there are no orbits (not even unstable ones) for \(R_{*}<3/2\). 

An interesting fact which was not mentioned in class: there exist orbits for any \(R_{*}\) between \(3/2\) and \(3\), although they are unstable. 

We found that the radius of a circular orbit is given by  
%
\begin{align}
  R = L^2 \qty(1 \pm \qty(\sqrt{1 - \frac{3}{L^2}}))
\,,
\end{align}
%
where \(L = l / 2GM\) and \(R = r / 2GM\). 

The two branches of this expression correspond to stable and unstable orbits: in both cases we have \(\sqrt{3} < L < \infty \), and for the stable (plus sign) branch we find \(R > 3\) and \(\dv*{R}{L} > 0\) everywhere, while for the unstable branch we have \(3/2 < R < 3\) and \(\dv*{R}{L} < 0 \) everywhere. 

\subsection{Perturbed rotating metrics}

We want to prove that, to first order in \(\delta g_{\mu \nu }\), the inverse of \(g_{\mu \nu } = \overline{g}_{\mu \nu } + \delta g_{\mu \nu }\) is 
%
\begin{align}
  g^{\mu \nu } = 
  \overline{g}^{\mu \nu } + \overline{g}^{\mu \alpha } \overline{g}^{\nu \beta } \delta g_{\alpha \beta } + O((\delta g_{\mu \nu })^2)
\,.
\end{align}

This can be proved directly by verifying \(g_{ \mu \nu } g^{\nu \rho } = \delta_{\mu }^{\rho } + O((\delta g)^2)\): it comes out to be 
%
\begin{align}
  &\qty(\overline{g}_{\mu \nu } + \delta g_{\mu \nu })
  \qty(\overline{g}^{\nu \rho } - \overline{g}^{\nu  \alpha } \overline{g}^{\rho  \beta } \delta g_{\alpha \beta } + O((\delta g)^2)) \\
  &= \delta_{\mu }^{\rho } + \delta g_{\mu \nu } \overline{g}^{\nu \rho } - \overline{g}_{\mu \nu }\overline{g}^{\nu \alpha } \overline{g}^{\rho \beta } \delta g_{\alpha \beta } + O((\delta g)^2)  \\
  &=  \delta_{\mu }^{\rho } + \delta g_{\mu \nu } \overline{g}^{\nu \rho } - \delta_{\mu}^{\alpha } \delta g_{\alpha \beta } \overline{g}^{\rho \beta }  + O((\delta g)^2)
\,,
\end{align}
%
so we can notice that the first order terms cancel: we have the inverse, up to first order. 

\subsubsection{Inverse perturbed Schwarzschild}

The Schwarzschild metric is 
%
\begin{align}
  \overline{g}_{\mu \nu } = \left[\begin{array}{cccc}
  -\qty(1 - \frac{2GM}{r}) & 0 & 0 & 0 \\ 
  0 & \qty(1 - \frac{2GM}{r})^{-1} & 0 & 0 \\ 
  0 & 0 & r^2 & 0 \\ 
  0 & 0 & 0 & r^2 \sin^2 \theta 
  \end{array}\right]
\,,
\end{align}
%
and its inverse is 
%
\begin{align}
  \overline{g}^{\mu \nu } = \left[\begin{array}{cccc}
  -\qty(1 -\frac{2GM}{r})^{-1}  & 0 & 0 & 0 \\ 
  0 & \qty(1- \frac{2GM}{r}) & 0 & 0 \\ 
  0 & 0 & r^{-2} & 0 \\ 
  0 & 0 & 0 & r^{-2} \sin^{-2} \theta 
  \end{array}\right]
\,.
\end{align}

We want to compute \(\overline{g}^{\mu \alpha } \overline{g}^{\nu \beta } \delta g_{\alpha \beta }\), and we know that \(\delta g_{\alpha \beta } \) has only one independent component: 
%
\begin{align}
  \delta g_{t \varphi } = - \frac{2 GJ  \sin^2\theta }{r}
\,.
\end{align}

This, combined with the fact that the background metric is diagonal, gives us the result that we only have one entry in the sum: 
%
\begin{align}
  \overline{g}^{\mu \alpha } \overline{g}^{\nu \beta } \delta g_{\alpha \beta }
  = \overline{g}^{tt} \overline{g}^{\varphi \varphi }
  \delta g_{\varphi  t}
  =(-)^2 \frac{2GJ \sin^2 \theta /r }{(1 - \frac{2GM}{r}) r^2 \sin^2 \theta }
  = \frac{2GJ}{(r - 2GM) r^2}
  \,,
\end{align}
%
so the  full inverse metric to first order in \(J\) is given by subtracting this off of the regular Schwarzschild inverse's \(t \varphi \) components: 
%
\begin{align}
  g^{\mu \nu } = \left[\begin{array}{cccc}
    -\qty(1 -\frac{2GM}{r})^{-1}  & 0 & 0 & -\frac{2GJ}{(r - 2GM)r^2} \\ 
    0 & \qty(1- \frac{2GM}{r}) & 0 & 0 \\ 
    0 & 0 & r^{-2} & 0 \\ 
    -\frac{2GJ}{(r - 2GM)r^2} & 0 & 0 & r^{-2} \sin^{-2} \theta 
    \end{array}\right]  
\,.
\end{align}

\subsubsection{Ricci component computation}

We want to compute the 00 component of the Ricci tensor, \(R_{00} = R^{\alpha }_{0 \alpha 0}\). It is given by 
%
\begin{align}
  R_{00} = \Gamma^{\alpha }_{00, \alpha }
  - \Gamma^{\alpha }_{0 \alpha, 0}
  + \Gamma^{\alpha }_{\alpha \lambda }
  \Gamma^{\lambda }_{00}
  - \Gamma^{\alpha }_{0\lambda } 
  \Gamma^{\lambda }_{0 \alpha }
\,,
\end{align}
%
and we will show that each of these 4 terms is either constant with respect to \(J\) or quadratic in \(J\), when computed with respect to the metric 
%
\begin{align}
  g_{\mu \nu } = 
  \left[\begin{array}{cccc}
  -\qty(1- \frac{2GM}{r}) & 0 & 0 & - \frac{2GJ}{r} \sin^2\theta  \\ 
  0 & \qty(1 - \frac{2GM}{r})^{-1} & 0 & 0 \\ 
  0 & 0 & r^2 & 0 \\ 
  - \frac{2GJ}{r} \sin^2\theta  & 0 & 0 & r^2 \sin^2 \theta 
  \end{array}\right]
\,.
\end{align}

First of all, note that the only derivatives of the metric which can give \(J\)-dependent contributions are \(g_{03,1}\), \(g_{03, 2}\) or these terms with \(0\) and \(3\) exchanged. 

The sum \(\Gamma^{\alpha }_{00, \alpha } \) is independent of \(J\): the only terms which can contribute in the sum are \(\alpha = r, \theta \) and then the three indices in the Christoffel symbol are either \(001\) or \(002\): in either case we cannot form the \(J\)-dependent metric component \(g_{03}\).

More explicitly: the expression is 
%
\begin{align}
  \Gamma^{\alpha }_{00} = \frac{1}{2} g^{\alpha \beta } \qty(2g_{\beta 0,0}- g_{00, \beta } )
\,,
\end{align}
%
and the time derivatives vanish, terms with \(\beta = 1, 2\) could contribute but in neither case would we have the possibility to form the \(g^{03}\) component. 

The term \(\partial_{0} \Gamma^{\alpha }_{0\alpha }\) is zero by stationarity: the metric is \(t\)-independent. 

In the term \(\Gamma^{\alpha }_{\alpha \lambda } \Gamma^{\lambda }_{00}\) we ask ourselves: where can the metric component \(g_{03}\) or \(g^{03}\) appear? As we saw above it cannot appear in the second symbol, while for it to appear in the first symbol we would need to have \(\alpha, \lambda = 0, 3\) (in either order), but then the third index in that symbol would also be \(0\) or \(3\), so the metric component \(g_{03}\) would be differentiated with respect to \(t\) or \(\varphi \), and it is independent of both. So, in order to have the symbol \(\Gamma^{\lambda }_{00} \) not be zero we need to have \(\lambda = 1, 2\). 
Then, in the expression 
%
\begin{align}
  \Gamma^{\alpha }_{\alpha \lambda }  = \frac{1}{2}
  g^{\alpha \beta } \qty(g_{\beta \alpha , \lambda }
  + g_{\beta \lambda , \alpha }
  - g_{ \alpha \lambda , \beta })
\,
\end{align}
%
the last two terms of the sum cannot depend on \(J\) since they have an index which is neither 0 nor 3. 
Terms such as those with \(\alpha , \beta = 0,3\) or the inverse can contribute: however in these terms we have to multiply a \(g^{03}\), which is linear in \(J\), with a \(g_{03}\), which also is. So in the end the term is quadratic in \(J\). 

The term \(\Gamma^{\alpha }_{0 \lambda } \Gamma^{\lambda}_{0 \alpha }\) is also \(O(J^2)\); to see this, let us write a symbol \(\Gamma^{\alpha }_{0 \lambda }\): it is 
%
\begin{align}
  \Gamma^{\alpha }_{0 \lambda } = 
  \frac{1}{2} g^{\alpha \beta } \qty(g_{\beta 0, \lambda } +  \cancelto{}{g_{\beta \lambda, 0 } }
  - g_{0 \lambda , \beta })
\,,
\end{align}
%
and we ask ourselves: how can this depend on \(J\)?
We could have \(\alpha , \beta = 0, 3\): then the term \(g_{0 \lambda , \beta }\) vanishes and we have \(g^{03} g_{30, \lambda }\) which is already quadratic in \(J\). 

We could have \(\alpha , \beta = 3, 0\): then \(g_{0 \lambda , \beta }\) vanishes. We could have a nonzero term \(g^{30} g_{00, \lambda }\) if \(\lambda = 1, 2\): in this case the whole term would look like \(\Gamma^{3}_{0 1} \Gamma^{1}_{03}\) or  \(\Gamma^{3}_{0 2} \Gamma^{2}_{03}\). In either case, in both symbols the sum of derivatives of the metric the only nonvanishing term will be necessarily linear in \(J\) since it will be differentiated with respect to the 1 or 2 index. Therefore, the product of the two Christoffels will be at least quadratic in \(J\). 

The final case for the single Christoffel is \(\alpha = \beta \): in that case, to have \(J\)-dependence we need to have either \(\beta =3\) which means \(\alpha = 3\) or \(\lambda =3\). When we set one of the two indices \(\alpha , \lambda \) to \(3\) the other one must necessarily be \(1\) or \(2\), since otherwise the derivatives would vanish.
Then we get back to the case in which only one term in the sum of the three derivatives of the metric survives, which means that the symbol is linear in \(J\) as a whole, but we have two symbols for which this holds multiplied together, so on the whole the term is quadratic in \(J\). 

In the end then the sum, when expressed as a function of \(J\), looks like: 
%
\begin{align}
  R_{00} = \const(J)+  O(J^2)
\,,
\end{align}
%
and we know that if \(J=0\) then \(R_{00} = 0\), so we are done: \(R_{00} = O(J^2)\).

\end{document}