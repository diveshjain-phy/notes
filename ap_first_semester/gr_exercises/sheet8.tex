\documentclass[main.tex]{subfiles}
\begin{document}

\section{Sheet 8}

\subsection{Acceleration in Rindler coordinates}

Rindler coordinates for the Minkowski (1, 1) space, originally parametrized with \((t, x)\), are defined by 
%
\begin{align}
  \begin{cases}
      t = \rho \sinh \eta \\
      x = \rho \cosh \eta 
  \end{cases}
\,.
\end{align}

The metric is given by 
%
\begin{align}
  \dd{s^2} = - \dd{t^2} + \dd{x^2} =
  - \rho^2 \dd{\eta^2} + \dd{\rho^2}
\,.
\end{align}

We defined Rindler coordinates so that our uniformly accelerated observer would have velocity only along \(\eta \), so we can find the general expression of the 4-velocity by computing the normalization: since it will look like \(u^{\alpha } = (N, 0)\) for some \(N\) and we want \(u^{\alpha } u_{\alpha } = -1\), we get that \(N^2(-\rho^2) = -1 \), which means \(N = 1/ \rho \). 

The only nonzero Christoffel symbols are \(\Gamma^{\eta }_{\eta \rho } = 1/\rho \) and \(\Gamma^{\rho }_{\eta \eta }= \rho \). 

The computation of the acceleration then can be started: 
%
\begin{align}
  \dv{}{\tau } u^{\mu } &= u^{\nu } \nabla_{\nu } u^{\mu }  \\
  &= u^{\eta } \qty(\cancelto{}{\partial_{\eta } u^{\mu }} + \Gamma^{\mu }_{\eta \nu } u^{\nu })  \\
  &= \frac{1}{\rho^2} \Gamma^{\mu }_{\eta \eta }
\,,
\end{align}
%
so the only nonzero component is the one where \(\mu = \rho \), which is \(a^{\rho } = 1/\rho \), or in other words
%
\begin{align}
  a^{\mu } = \left[\begin{array}{c}
  0 \\ 
  \kappa 
  \end{array}\right]
\,,
\end{align}
%
where \(\kappa = 1/ \rho \) is the modulus of the 4-acceleration. Indeed 
%
\begin{align}
  a^{\mu } a^{\nu } g_{\mu \nu } = \kappa^2 g_{\rho \rho } = \kappa^2
\,.
\end{align}
%

\subsection{Acceleration in Schwarzschild motion}

The acceleration is given by 
%
\begin{align}
  a^{\mu } = u^{\nu } \qty(\partial_{\nu } + \Gamma^{\mu  }_{\nu \rho } u^{\rho })
\,,
\end{align}
%
and we can see that in this formula we can only have a nonzero result if \(\mu = t, r, \varphi \), \(\nu = t, \varphi \) and \(\rho = t, \varphi \). 
This, combined with the fact that velocity is both stationary and rotationally symmetric, gives us that the only nonzero component of the acceleration is 
%
\begin{align}
  a^{r } = (u^{\nu })^2 \Gamma^{r}_{\nu \nu }
\,.
\end{align}
%
 
\end{document}