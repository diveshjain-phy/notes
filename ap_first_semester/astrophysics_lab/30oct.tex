\documentclass[main.tex]{subfiles}
\begin{document}

\section*{Wed Oct 30 2019}

Line emission: the light from the corona illuminates the accretion disk, and we see the spectral lines from the various elements in it.

When looking at a disk edge-on, we have several effects: we see
\begin{enumerate}
    \item Doppler shift (Newtonian): the outer annulus has a faster tangential velocity since the accretion disk rotates approximately rigidly;
    \item Beaming (special relativistic);
    \item Transverse Doppler (special relativistic);
    \item Gravitational redshift (general relativistic).
\end{enumerate}

We can simulate the spectra due to a line emission at different angles with respect to the accretion plane, and with respect to either a Schwarzschild or Kerr BH.

Exercise.

We assume that the matter in the accretion disk basically orbits in circular paths, regulated by 
%
\begin{equation}
  \frac{v^2}{r} = \frac{GM}{r^2}
\,,
\end{equation}
%
with energy given by 
%
\begin{equation}
  E = \frac{1}{2} m v^2 - \frac{GmM}{r}
\,.
\end{equation}

We get the luminosity of the disk by assuming that the heat is dissipated through viscosity, 
%
\begin{equation}
  L = \int _{r_{*}}^{\infty} - \dv{E}{T} 2 \pi r \dd{r} = \frac{G \dot{m} M}{2 r_{*}}
\,,
\end{equation}
%
where \(T\) is time, and \(\dot{m} = \dv*{m}{T}\).

The temperature is increasing as \(r\) decreases. The total spectrum is the sum of several blackbodies, at each temperature.

Instead, we see a blackbody component, and a powerlaw coming from the reflection of the accretion disk, plus an iron line.



\end{document}