\documentclass[main.tex]{subfiles}
\begin{document}

\section*{30 september 2019}

Teacher: Roberto Ragazzoni, Gabriele Umbriaco et al.

\url{roberto.ragazzoni@inaf.it}

\begin{equation}
    \text{Lab Course} = \text{Course} (\lambda)
\end{equation}

Objective of this course: \emph{how to build an astronomical instrument} which will push the limits of astrophysical knowledge and technology.

\subsection{Orders of magnitude}

Diameters: in the future \SI{24}{m}, \SI{37}{m}.

Resolution: \SI{0.04}{\arcsec} for the HST.

Collected photons: \SI{2700}{Hz} for an eye at \(V=6\).

For a galaxy at redshift \(z=10\), the Lyman (?) break at \SI{91.4}{nm} is shifted.

In the background: magnitude 18 (with moon), 21 (no moon) per arcsec square.

10 meters, ``seeing limited'': \SI{1}{arcsec}; 1 meter, ``diffraction limited'': \SI{0.1}{arcsec}. Collection ratio: 100X, but the size of the background in which the unresolved source is located is also 100X.

We have a great school of \emph{adaptive optics}.



\end{document}
