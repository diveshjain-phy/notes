\documentclass[main.tex]{subfiles}
\begin{document}

\section*{Tue Nov 05 2019}

We start by analyzing the data from CHANDRA. 

Then, we start to talk about \emph{grazing angles}:
they depend on \(\delta \), where 
%
\begin{align}
  \delta \sim \frac{1}{2 \pi } N_e r_e \lambda^2
\,,
\end{align}
%
which is of the order of one degree, and allows us to
build detectors with more effective area.

The effective area is: 
%
\begin{align}
  A _{\text{eff}} = A _{\text{geom}} R(E) V(E, x, y) Q(E, x, y)
\,.
\end{align}
%

The vignetting \(V\) accounts for the fact that the outermost pixels are less bright than the innermost ones; 
the quantum efficiency \(Q\) is\dots



\end{document}