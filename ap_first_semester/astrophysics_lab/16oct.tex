\documentclass[main.tex]{subfiles}
\begin{document}

\section*{Wed Oct 16 2019}

Or galaxy might have had a jet perpendicular to it in the past.

Seeing galaxies at very high redshift is difficult: \(\gamma \) ray astronomy is useful that way, since it can tell very low FOV (optical?) telescopes where to look.

We look at different wavelengths detected for 3C 279: a blazar.

AGNs are Active Galactic Nuclei. They are best seen through X-rays, which can pierce the dusty accretion disc.

Iron is the final state of fusion: we expect to see iron lines near SMBHs.
When looking at a BH accretion disc at a high inclination, we see the region \emph{behind} the BH, since light is curved.

Mrk 876 iron line: it seems \emph{too} reshifted, cosmological redshift alone does not explain it. We need to account for gravitational reshift from the SMBH, and tell what the distance from the center is.

\begin{greenbox}
    Also, we can use Doppler redshift to measure the speed of rotation around the SMBH: how does this work? It seems like the two effects should combine\dots
\end{greenbox}  

About INTEGRAL: it has 4 instruments, a \(\gamma \) ray telescope, an X ray telescope, an optical monitoring camera (and...?).

It has a huge FOV (\SI{20}{\degree} by \SI{20}{\degree}), but the angular resolution is only 12'. 

We talk about the \SI{}{MeV} gap.


\end{document}