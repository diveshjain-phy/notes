\documentclass[main.tex]{subfiles}
\begin{document}

\section*{Wed Nov 06 2019}

\section{Non-isothermal winds}

Now, we consider the possibility that our winds are \emph{not isothermal}.
This will change the structure of the wind, by the introduction of an additional pressure gradient.

It will change the speed of sound and thus the Mach number.

It is useful to define the energy per unit mass \(e\): 
%
\begin{align}
  e(r) = \frac{v^2(r)}{2} - \frac{GM}{r} + \frac{\gamma }{\gamma -1} \frac{ \mathcal{R} T}{\mu }
\,,
\end{align}
%
where \(\gamma / (\gamma -1) = 5/2\) for a monoatomic gas, which has \(\gamma = 3/2\).

In the lower boundary of the wind the velocity is much less than the escape velocity (\(v \ll v _{\text{esc}}\)), and also the thermal velocity of the particles is not enough for them to escape the gravitational well: \(\mathcal{R} T / \mu \ll v _{\text{esc}}\) at the surface of the star, while far from the star we have \(v \gg v _{\text{esc}}\).

If we have an isothemal wind, some energy must be added in order to prevent the adiabatic cooling of the gas, lift it from the potential well, and to increase its kinetic energy.

If a force is applied it increases the momentum, but the heat transmission \(q\) also appears in the momentum equation, which is \(\Delta e = \int f + q \dd{r} \).
Heat transmission changes the pressure profile, which affects the momentum, even though \(q\) does not appear explicitly in the momentum equation.

We define the total heat deposition \(Q\) and the total work done by the force \(W\):
%
\begin{align}
Q(r) = \int_{r_0 }^{r} q(\widetilde{r}) \dd{\widetilde{r}}
\qquad \text{and} \qquad
W(r) = \int_{r_0 }^{r} f(\widetilde{r}) \dd{\widetilde{r}}
\,,
\end{align}
%
and we will have \(e( \infty ) - e(r_0 ) = Q( \infty ) + W( \infty )\). 

The most general momentum equation is given by: 
%
\begin{align}
  \frac{1}{v} \dv{v}{r} = \qty(2 \frac{c_s^2}{r} - \frac{GM}{r} + f -(\gamma - 1) q) / \qty(v^2- c_s^2)
\,,
\end{align}
%
where we introduce the adiabatic speed of sound \(c_s = \sqrt{\gamma a^2} \), \(a\) being the \emph{isothermal} speed of sound.
In general \(- (\gamma - 1)<0\), therefore if we add heat this is equivalent to pushing \emph{inward}.

If either \(f\) or \(q\) depend on the velocity gradient \(\dv*{v}{r}\) then the sonic point can \emph{decouple} from the critical point.

There are cases in which we have multiple critical points (specifically, multiple zeros of the denominator).

The momentum equation plus the energy equation 
%
\begin{align}
  \dv{}{r} \qty(\frac{v^2}{2} + \frac{5}{2} \frac{RT}{\mu } - \frac{GM}{r}) = f(r) + q(r)
\,,
\end{align}
%
can be solved numerically, and if we impose smooth passage through the critical point this yields the mass loss rate.

Qualitatively, the results are the same as in the isothermal case.
Adding either momentum or energy to the subsonic region of the wind increases the bottom-of-the-corona velocity and the mass loss rate. Doing it in the supersonic region has no effect.

This ends our general introduction to stellar winds.

Now we will do a couple of exercises to get familiar with the theory. 

\begin{greenbox} \textbf{Exercise}

The wind is isothermal.
The solar wind has a mean coronal temperature of \SI{1.5e6}{K} and a mass loss rate of \num{2e-14} solar masses per year.
The bottom of the corona is at \(r_0 \approx 1.003 R_{\odot}\), where the density is \(\rho (r_0 ) = \SI{e-14}{g\per\centi\metre\cubed}\).

Calculate the potential energy, the kinetic energy and the enthalpy of the gas at \(r_0 \).

Calculate the same quantities at the critical point. Which of these energies has absorbed the largest fraction of the energy input?
\end{greenbox}

\begin{bluebox}
We can use the continuity equation \(\dot{M} = 4 \pi \rho_0 r_0^2 v_0 \) to get 
%
\begin{align}
  v_0 = \frac{\dot{M}}{4 \pi r_0^2 \rho_0 }
  \approx \SI{21}{m/s}
\,,
\end{align}
%
and with this we can calculate 
%
\begin{align}
  e(r_0 ) = - \frac{GM}{r_0 } + \frac{1}{2} v_0^2 + \frac{5}{2} \frac{RT}{\mu }
\,.
\end{align}

We get: 
%
\begin{align}
  E _{\text{kin, 0}} = \frac{v_0^2}{2} \approx \SI{212}{J/kg}
\,,
\end{align}
%
while for the gravitational energy we'd need the mass of the star. Assuming it is equal to the solar mass, we find 
%
\begin{align}
  E _{\text{grav, 0}} = -\frac{GM}{r_0 } \approx \SI{-1.9e11}{J/kg}
\,.
\end{align}
%

The mean molecular weight of the gas for the Sun is something like \(\mu = 0.62\) (we count electrons in it).
Then we get 
%
\begin{align}
  E _{\text{chem, 0}} = E _{\text{\text{chem, crit}}} = \frac{5}{2} \frac{RT}{\mu } \approx \SI{5.0e7}{J/kg}  
\,.
\end{align}
%

The enthalpy is the same everywhere in the flow, since the flow is isothermal.

The values at the critical radius are calculated with the same formula. The velocity will be the speed of sound \(a = \sqrt{RT / \mu } \approx \SI{4.5e3}{m/s} \).

Then we find: 
%
\begin{align}
  E _{\text{kin, crit}} = \frac{a^2}{2} \approx \SI{1.0e7}{J/kg} \approx \num{5e4} \times E _{\text{kin, 0}} 
\,.
\end{align}
%

The critical radius is given by \(r_c = GM / (2 a^2)\): so, we get 
%
\begin{align}
  E _{\text{grav, crit}} = - \frac{GM}{r_{c}} = -2 a^2 \approx - \SI{4.0e7}{J/kg} \approx \num{2.1e-4} \times E _{\text{grav, 0}}
\,.
\end{align}
%

Qualitatively, at the corona we have 
%
\begin{align}
   \abs{ E _{\text{grav, 0}}} \gg E _{\text{chem, 0}} \gg E _{\text{kin, 0}}
\,,
\end{align}
%
while at the critical point they are similar, and specifically 
%
\begin{align}
  E _{\text{chem, crit}} \gtrsim \abs{E _{\text{grav, crit}}} \gtrsim E _{\text{kin, crit}}
\,.
\end{align}
%

\end{bluebox}

\begin{greenbox}
    \textbf{Exercise}

  A star with \(T _{\text{eff}} = \SI{3200}{K}\), \(R_{*} = 30 R_{\odot}\), \(L_{*} = 85 L_{\odot}\) and \(M_{*} = 6 M_{\odot}\) has an isothermal corona of \(T = \SI{e6}{K}\) with a density at the lower boundary of \SI{e-13}{g \per\centi\metre\cubed}.

  Calculate the energy per unit mass at the bottom of the corona at \(r_0 = R_{*}\).

  Calculate the location of the critical point, \(r_c\), and the mass loss rate.

  Calculate the energy per gram gained by the wind between \(r_0 \) and \(r\). What fraction of the stellar luminosity is used to drive the wind up to the critical point?
\end{greenbox}

\begin{bluebox}
  The energy per unit mass at the bottom of the corona is given by 
  %
  \begin{align}
    e(r) = - \frac{GM}{r_0 } + \frac{1}{2} v_0^2 + \frac{5}{2} \frac{RT}{\mu }
  \,,
  \end{align}
  %
but we cannot use this formula since we do not have \(v_0 \) nor \(\dot{M}\). However, we can approximate the density profile as an exponential, applying the formula 
%
\begin{subequations}
\begin{align}
  \dot{M} &= 4 \pi r_c^2 a \rho_{c}  \\
  &= 4 \pi r_c^2 a \rho_0 \exp(- \frac{r_c - r_0 }{H_0 } \frac{r_0 }{r_c})  
\,,
\end{align}
\end{subequations}
%
where we have: the length scale \(H_0 = RT r_0^2 / (GM \mu )\), the critical velocity \(a = \sqrt{RT/\mu }\), and the critical radius \(r_c = GM/(2a^2)\).

Plugging these in, we find: 
%
\begin{subequations}
\begin{align}
  \dot{M} &= 4 \pi \qty(\frac{GM \mu }{2 RT})^2 \sqrt{\frac{RT}{\mu }} \rho_0 
  \exp(\frac{\mu GM}{RT} \qty(\frac{2RT}{GM \mu } - \frac{1}{r_0 }))  \\
  &= \pi (GM)^2 \qty(\frac{\mu }{RT})^{3/2} \rho_0 
  \exp(2 - \frac{\mu GM}{RT r_0 })
\,.
\end{align}
\end{subequations}
%
\todo[inline]{One temperature should probably be the effective temperature, but I do not really know what that means. }

Then, the mass loss rate can be calculated, since we have all of these quantities. It comes out to be barely anything, since \(\mu GM / RT r_0 \) is very large and we have a negative exponential of it\dots

\end{bluebox}

Lamers-Cassinelli: chap 4, section 3: multiple critical points.

Next Tuesday, the 12th, after the lecture, we will likely move to the DFA in Via Marzolo until 14.30.

\section{Wind types}

\paragraph{Coronal winds}
These are driven by gas pressure due to high temperature.
Stars with a convection zone right under the photosphere can have coronas of a few million Kelvin degrees. 
These are rather well described as isothermal, however the temperature decreases slowly because of conduction.

\paragraph{Dust driven winds}
For these winds, the driving mechanism is the radiation pressure on the dust grains, which are continuum absorbers. The gas component is dragged along by momentum transfer. 
This happens for cool stars, whose envelopes have temperatures of less than a thousand degrees Kelvin, in which the dust grains can actually be formed. 
These can be well modelled as winds with a force like \(f = [r \geq r_d] A / r^2\), where the radius \(r_d\) is the one after which the temperature is low enough for the grains to form.

\paragraph{Line-driven winds}
These are winds of hot stars, driven by radiation pressure on spectral lines of abundant ions which have many in the UV and far UV.
The pressure depends on the Doppler effect: the ions can absorb different wavelengths of light at different radii because of it.
The force depends on the velocity gradient: it cannot be modelled with what we discussed so far.

\paragraph{Pulsation driven winds}
Stars such as Miras and those in the Asymptotic Giant Branch may pulsate: the atmosphere is tossed up and then falls back. Because of the low gravity, they fall slowly and are hit by an outmoving layer before they have completed their fall. 
This means that for each pulsation cycle they get a ``kick''.  

This can be a very efficient driving method if we account for dust formation. The shockfront from the pulsation cycle basically travels outwards to infinity.

\paragraph{Sound wave driven winds}
They are modelled in a way that is similar to coronal winds. 
The resulting amplitudes are usually small. 

\paragraph{Alfvén Wave Driven winds}
Due to magnetic fields: if the points at which the field lines make contact with the surface move, than a magnetic wave travels forward with a speed \(v_A = B / \sqrt{4 \pi \rho } \gg a\). 
This can be very efficient, and result in high wind speeds. 
It is relevant for stars which do not have strong radiation pressure (ie not more than a thousand times more luminous than the Sun). 

\paragraph{Magnetic rotating winds}


\section{Coronal winds}

Now, we introduce the next topic.
We deal with hot, luminous stars: at the top left of the HR diagram.

[Picture from the slides]

We talk of line-driven winds for hot stars.
These are winds driven by \emph{spectral lines}. 
The blackbody emission for these stars is mainly at high frequencies, like the UV.

One may ask: hydrogen is much more abundant than heavier elements like carbon, nitrogen\dots why do we see the spectral lines for these heavier elements?

This is because hydrogen is completely ionized at these temperatures, and helium is also.

The strongest lines are far in the UV, where the stellar flux is low: only few atoms are hit, the rest of the gas is dragged along.

A peculiar characteristic of the spectrum is the so-called P-Cygni profile.

Now, an overview of the formation of spectral lines:
there are 5 processes \begin{enumerate}
    \item Line scattering: if it comes from the ground state of the atom, it is called \emph{resonance scattering}, which is the main phenomenon.
    \item Emission by recombination: the ion recombines to an excited state.
    \item Emissional from collisional or photo-excitation. A photon is absorbed, it excites the atom which then descends to a lower energy level.
    \item Pure absorption and then de-excitation of an already excited atom.
    \item Masering by stimulated emission: it can happen in a very narrow set of circumstances: an excited atom is hit by a photon which has exactly the same energy as the one between the atom's state and the ground state, so the atom is deexcited and now there are two photons.
    This happens when there are many excited atoms, and when the velocity gradient is very small --- otherwise, the photons are Doppler-shifted out of the right frequency.
\end{enumerate}


\end{document}