\documentclass[main.tex]{subfiles}
\begin{document}

\section*{Tue Dec 10 2019}

We continue with the evolution of a \(15 M_{\odot}\) star. 
Due to the mechanism of neutrino production described last lecture, after a few tens of thousands of years after the start of carbon burning we have an iron core, since the burning was sped up.
The evolution of the envelope is decoupled from that of the core: the process is too fast to affect it. Therefore, the position of the star in the HR diagram is unchanged.

We use Kippenhahn diagrams: in these, we have age on the \(x\)-axis and the mass coordinate on the \(y\)-axis. We can then show how the placement of the different regions of the star's interior evolves.
We can, for example, have the negative logarithm of the time until the core's collapse of the star on the \(x\)-axis.
There can be different burning stages simultaneously, in different shells. 

\subsection{Stellar winds in massive stars}

Stellar winds are of general application in the study of stellar evolution. 

We have \textbf{radiation driven winds} for OB-type stars and Blue Supergiants: these are very hot massive stars.
So, we have lots of UV absorption from resonance transitions in the spectral lines of Fe, O, Si, C.

This leads to high asymptotic velocities \(v_{ \infty } \sim \SI{4e3}{km/s}\) and high mass loss rates \(\dot{M} \sim \SI{5e-5}{} M_{\odot} / \SI{}{yr}\). 
There is a dependence on metallicity like \(\dot{M} \sim Z^{\num{.6}}\).
CAK theory deals well with these.

For red supergiants, instead, we have 
\textbf{dust driven winds}: the mass loss rates can be very high, as much as \(\num{e-4} M_{\odot} / \SI{}{yr}\).
Stars with masses lower than \(40 M_{\odot}\) spend a long time in the core-Helium burning phase as red supergiants: they have time to shed their outer layers; they are then observed as Wolf-Rayet stars, which are the very hottest.

\subsection{Wolf-Rayet stars}

\subsubsection{The Humphrey-Davidson limit}

As discussed in the last section, very luminous stars shed their outer layers: this is a plausible explanation of the empirical \emph{Humphrey-Davidson Limit}, an upper limit on the luminosity of red supergiant stars: it is found at around \(M _{\text{bol}} \approx -9.5 \divisionsymbol -10 \) for stars cooler than \SI{15}{kK}.\footnote{\url{http://articles.adsabs.harvard.edu/pdf/1979ApJ...232..409H}}

The relation between bolometric magnitude and visual magnitude is: 
%
\begin{align}
  M _{\text{bol}} = -2.5 \log L + 4.73
\,,
\end{align}
%
where the luminosity is measured in solar luminosities. 
Then, the limit can be stated as saying that there are no red supergiants with \emph{visual} luminosities larger than \(\SI{e5.8}{}L_{\odot}\).

This limit depends on the effective temperature of the star, and it is a sharper bound than the Eddington luminosity.

The Eddington luminosity is calculated by setting
%
\begin{align}
  \Gamma_{\text{Edd}} = \frac{a _{\text{rad}}}{g} = \frac{\kappa L _{\text{Edd}}}{4 \pi R^2 c} \frac{R^2}{GM} =1
\,,
\end{align}
%
which implies 
%
\begin{align}
  L _{\text{Edd}} = \frac{cGM 4 \pi }{\kappa_e}
\,.
\end{align}

It can be interpreted as a generalized Eddington limit: if the luminosity is higher the outer layers of the star are shed.
Stars near the limit are unstable: they periodically undergo intense mass loss events, and are known as Luminous Blue Variables.

\subsubsection{Wolf-Rayet stars}

Wolf-Rayet stars have high effective temperatures (\(T > \SI{e4.8}{K}\)) and high luminosities, they expel a great quantity of mass through stellar winds. 

The spectra of WR stars exhibit emission lines: we see little or no \ce{H}, and an abundance of either \ce{He + N} or \ce{C + O}.

We distinguigh them into 
\begin{enumerate}
    \item WNL stars have some hydrogen on the surface (but its mass fraction is lower than \num{.4}) and lots of \ce{H} and \ce{Ne};
    \item WNE are similar but with no hydrogen;
    \item WC stars have no hydrogen, little \ce{N} and more \ce{He}, \ce{C} and \ce{O};
    \item WO stars are similar but with more oxygen.
\end{enumerate}

These are actually evolutionary stages: the outer layers are successively stripped by winds, and the inner ones are exposed.

We observe lots of \ce{^{14}N} in WNE stars, since the Nitrogen burning phase is the slowest process. 

% We have these stars when \(\log T _{\text{eff}} > 4\), and \(M > 30 M_{\odot}\).

\subsection{Final fates by mass}

Depending on the mass of the progenitor red supergiant, we have different behaviours: below \(30 M_{\odot} \) the star explodes as a red supergiant; more massive stars will experience more mass loss and explode as blue supergiants, further left on the HR diagram. 

Here is the general progress of the evolution of stars with \(M > 30 M_{\odot}\):
\begin{enumerate}
  \item \(30 M_{\odot} \lesssim M \lesssim 35 M_{\odot}\) stars go from Red SuperGiants to WNL stars;
  \item \(35 M_{\odot} \lesssim M \lesssim 40 M_{\odot}\) stars go from Red SuperGiants to WNL stars to WNE stars;
  \item \(40 M_{\odot} \lesssim M \lesssim 60 M_{\odot}\) stars go from Red SuperGiants to WNL stars to WNE stars to WCO stars;
  \item \(60 M_{\odot} \lesssim M \) stars go from WNL stars to WNE stars to WCO stars;
\end{enumerate}

We see a simulation: a \(15 M_{\odot}  \) star starts off on the high part of the Main Sequence, then quickly moves right becoming a red supergiant when its core collapses.

A \(3 M_{\odot}\) star instead moves away from the MS more slowly, and then after having moved right it quickly moves far left and cools, as a white dwarf. 

\section{The collapse of the iron core}

Now, let us discuss the engine of the explosion of these massive stars. The explosion is triggered by an implosion: we are at the end of the Silicon-burning phase. 
The pressure of the iron core is mantained by the degenerate electrons. 

The degenerate iron core starts off with density \(\rho \approx
\SI{e9}{g cm^{-3}}\), temperature \(T \approx \SI{e10}{K}\), mass \(M _{\text{Fe}} \approx \num{1.5} M_{\odot}\) and radius \(R \approx \SI{8000}{km}\). 

What triggers the collapse of the iron core? 

In order to study its dynamical instability, we nee to look at the adiabatic exponent:
%
\begin{align}
  \gamma _{\text{ad}} = \qty(\pdv{\log P}{\log \rho })_{\text{ad}}
\,,
\end{align}
%
we have a dynamic instability when  \(\gamma _{\text{ad}}\) falls below the critical value of \(4/3\). 
It is a local quantity, it can be defined for each layer; we can use a global parameter to measure the instability: 
%
\begin{align}
  \int \qty(\gamma _{\text{ad}} - \frac{4}{3}) \frac{P}{\rho } \dd{m}
\,,
\end{align}
%
which will tell us whether the system as a whole is stable. 
This is effecively an integral over the work of the pulsation: \(\rho^{-1} \dd{m} = \dd{V}\), so \(P/\rho \dd{m} = \dd{W}\).

We have two processes which are opposed: the first is \textbf{photodisintegration}, a process like 
%
\begin{align}
\ce{^{56}_{26}Fe} + \gamma \rightarrow 
13 \ce{^{4}_{2}He} + 4n
\,,
\end{align}
%
in which a heavy nucleus is shattered into \(\alpha \) particles. This is an \emph{endothermic} process.

The other process is \textbf{electron capture}, or \emph{neutronization}: \(\ce{p+ + e- \rightarrow n +} \nu_{e}\).
This is a weak-interaction process.
This increases the mean molecular weight per electron \(\mu_{e}\), so the Chandrasekar mass decreases since it is inversely dependent on \(\mu_{e}\): it becomes
%
\begin{align}
M _{\text{ch}} = \frac{5.83}{\mu_{e}^2} \sim 1.26 M_{\odot}
\,
\end{align}
%
for the iron core.
A huge amount of neutrinos are produced. 

\todo[inline]{Are these actually opposed? it seems like they both act to speed up the collapse; this is what is written in Janka et al: ``Electron capture, \(\beta \) decay, and partial photodisintegration of iron-group nuclei to alpha particles
cost the core energy and reduce its electron density. As a consequence, the collapse is accelerated.''}

\subsection{The steps of the collapse}

Now we follow a work by Janka et al (2007) at the Max Planck, which outlines the steps of the collapse. 

The first step in the collapse happens after the iron core, which is only supported by electron degeneracy pressure since it cannot fuse iron esothermally, grows past its Chandrasekar mass, which as we have discussed can decrease by several mechanisms. Then, gravity overcomes this pressure and forces \emph{electron capture} to happen. This 

As the density of the core increases, it becomes opaque to neutrinos: their scattering cross section is
%
\begin{align}
  \sigma_{\nu } \approx \num{e-49} A^2 \qty(\frac{\rho }{\mu_{e}})^{2/3} \SI{}{cm^2}
\,,
\end{align}
%
so when their diffusion time becomes longer than the time of the collapse they do not have time to escape, on average.
Here \(A\) is the mass number of the relevant nucleus, \(\rho \) is the density while \(\mu_{e}\) is the mean molecular weight per electron.
We have the formation of a shell below which neutrinos are trapped and above which they escape.

\todo[inline]{The formula as given is dimensionally inconsistent, I posit that it is to be interpreted ignoring the dimensionality of \(\rho \).}

In the meantime the collapse is proceding \emph{homologously} (the different layers are falling simultaneously), until it reaches a density of around \(\rho \sim \SI{e14}{g cm^{-3}}\): this is comparable to the density of the nucleus, which is around \SI{2e14}{g cm^{-3}}: at this point the material becomes incompressible, since nuclear matter is much less compressible than the gas of electrons, protons and neutrons we had before.

A proto-neutron star is formed, and a shock front travels back towards the outer layers of the infalling iron core. 

% \todo[inline]{What triggers the initial infall and compression in the first place? }

It is unlikely that the energy of the shock is already enough to blow off the envelope. This would be called the \textbf{prompt mechanism}: all modern simulations show it is not actually effective.
This is because there are several endothermic or effectively endothermic processes taking place: 
\begin{enumerate}
  \item the dissociation of heavy nuclei into nucleons: the binding energy of iron is around \(B \sim \SI{492}{MeV}\), if we had a \(1.4 M_{\odot}\) iron core then the energy needed to break up all of the iron nuclei would be 
  %
  \begin{align}
  E = B \frac{1.4 M_{\odot}}{m _{\text{Fe}}} \approx \SI{24}{foe}
  \,,
  \end{align}
  %
  where \(m _{\text{Fe}}\) is the mass of an iron nuclide, and \(\SI{1}{foe} = \SI{e51}{erg}\) (ten to the Fifty-One Erg);
  \item neutrino prodution: the energy emitted in the form of neutrino radiation is essentially lost to the star, although
  the energy of the delayed neutrinos which are still trapped in the nucleus will contribute.
\end{enumerate}

An alternative, more plausible mechanism driving the explosion is the  \textbf{delayed neutrino-heating mechanism}: the shockwave travelling outward after the infalling matter hits nuclear density would stall under the inward pressure of the infalling material, but it is \emph{revived} by the neutrino flux: these neutrinos carry most of the energy which was converted from gravitational potential energy.

These neutrinos keep streaming; when they are still inside they are degenerate and thus energetic, as they stream outward they are down-scattered in energy space.

As they are heated by neutrinos, the layers between the forming remnant and the shock front expand, creating a region of high temperature and low density.
This process then keeps driving the full explosion.

Do note that the neutrinos need not convert all their energy to heat in this process: in simulations they convert to thermal energy around \SI{10}{\%} to \num{20}{\%} of the radiated neutrino energy.
The supernova does not need a large fraction of the gravitational binding energy of the iron nucleus in order to explode.

The time needed to reach this stage from the start of the collapse is on the order of \SI{e-1}{s}.

\subsection{Advanced stellar burning phases and mass of the iron core}

Now we start a new part: the advanced burning stages, after the \ce{He}-burning phases. 

There are four major burning phases: 
%
\begin{enumerate}
    \item Carbon;
    \item Neon;
    \item Oxygen;
    \item Silicon.
\end{enumerate}

The central burning region forms a convective core.
Burning in external shells forms convective shells.
After a certain type of fuel is locally exhausted, the burning shifts outward.
This gives rise to a so-called ``onion-skin'' model.

As the initial mass of the star increases, the fraction of mass in the CO core decreases from around \num{.35} to \num{.15} as \(M\) goes from 10 to 120 solar masses.  
So, the mass of the CO core scales with the initial mass, but sublinearly so.

% What about the density structure in the pre-supernova phase? 
The mass of the iron core right before the SN explosion also increases with the stellar mass: as \(M\) goes from 10 to \(120 M_{\odot}\) it goes from 1.2 to 1.8 solar masses. 
So, the higher the mass of the CO core was, the more compact is the structure of the presupernova stage.

\todo[inline]{Hold on: is the mass of the iron core right before the explosion just the Chandrasekar mass? I'd think that, as long as there is silicon left to burn, iron would keep on being formed, until it reaches the Chandrasekar mass\dots} 

As the total mass increases, at a fixed radius we have more internal mass: the density \emph{increases} with mass. 

\subsubsection{Explosion and fallback}

Some of the material near the core, when the shock occcurs, falls back on it if it is inside a certain critical radius, if it is outside that radius it is thrown out. 
How much of it falls back depends on the physical properties of the core.
For an average supernova, the energy of the matter thrown into the interstellar medium is of the order of \SI{1}{foe}.

So, the questions we ask are these: 
\begin{enumerate}
  \item Do all massive stars actually explode?
  \item What are their remnants?
  \item What is the efficiency of the fall back?
\end{enumerate}

The remnant can either be a black hole or a neutron star.
If the fallback is very efficient we have an \emph{implosion}, if it is inefficient we have an \emph{explosion}.
The possibilities are 
\begin{enumerate}
    \item Explosion and Neutron Star;
    \item Implosion and Black Hole;
    \item (rarely) Explosion and Black Hole.
\end{enumerate}

There is no clear-cut law or mass thereshold. One thing to look at is the \textbf{bounce-compactness parameter}: 
%
\begin{align}
  \xi_{M^*} = \eval{\frac{M^{*} / M_{\odot}}{R(M^{*}) / \SI{e7}{m}}}_{t = t _{\text{bounce}}}
\,:
\end{align}
%
it seems like this parameter's value being high is correlated to a black hole being formed, while it being low is correlated with succesful explosions.
We usually look at \(\xi_{2.5}\), which means we are basically considering the inverse of the radius of the region encompassing \(2.5 M_{\odot}\).

\(\xi_{2.5} \) is a measure of compactness: the time-scale of black hole formation scales like \(t _{\text{BH}} \propto \xi_{2.5}^{-3/2}\), so if \(\xi_{2.5}\) is high the time for black hole formation is short: if the shock revitalization time is held constant then we can have implosion into a BH if \(t _{\text{BH}} < t _{\text{shock}}\), and explosion into a SN plus a neutron star remnant otherwise.

\todo[inline]{In the slides it is stated: ``Successful SN: the shock is revived on a time longer than the time-scale for a BH formation.'', but it seems to me like it is the other way around.}

We can plot \(\xi_{2.5}\) as a function of the Zero-Age Main Sequence (so, initial) or ZAMS mass of the star: it has peaks around \(23 M_{\odot}\) and \(40 M_{\odot}\).

\begin{bluebox}
This variable is effectivey equal to 
%
\begin{align}
\xi_{2.5} = 6772.2 \times \frac{G (2.5 M_{\odot})}{c^2 R}
\,,
\end{align}
%
so it measures how relativistic the core is. Generally, \(\xi \sim \num{.5}\), so it is not very relativistic.
\end{bluebox}

We have also the parameter 
%
\begin{align}
  \mu_{4}  = \abs{\frac{ \dd{m} / M_{\odot}}{ \dd{\tau } / \SI{e7}{m}}}_{s=4}
\,,
\end{align}
%
the normalized mass inside a dimensionless entripy per nucleon of \(s=4\). 
We can make a plot of the various final fates of the parts of the initial mass: some of it is expelled by winds, some of it is expelled through the supernova explosion, some of it ends up in the final remnant.

The mass of the iron core is never very much larger than a couple of solar masses, but BHs can become more massive through fallback. 

Then it seems that WR stars of more than \(30 M_{\odot}\) will not explode: they will not eject material as supernovas. 

\todo[inline]{Simulations show that there is not a clear-cut boundary between neutron star and BH formation, but then we go on as if there were\dots}

The behaviour of LA89 massive stars is quite different, NS are more favoured towards higher masses. 

\end{document}