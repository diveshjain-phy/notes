\documentclass[main.tex]{subfiles}
\begin{document}

\section*{Mon Dec 16 2019}

In a binary system things are more complicated. 

It seems that stars more massive than \(35 M_{\odot}\) have large fallback: they do not contribute to the heavy element population. 

There are still many uncertain areas: convection is not well understood, especially when it is heavily coupled to nuclear burning. 

Rotation can cause convective overshooting (what is it?). 

We do not have a self-consistent hydrodinamical model which includes neutrino transport. 

We can plot the mass of the various mass components of the star with respect to \(\log \qty(t _{\text{collapse}} - t)\). 

We can also plot the mass fractions of \ce{^{12}C} and \ce{^{16}O} with respect to the fraction of \(X(\ce{^{4}He})\), which decreases with time. At some point the helium burning phase stops, and then we are left with a carbon/oxygen fraction which can be either higher or lower than one. 

Depending on whether we are using a LA89 or an LN00 model we can have different behaviours. 

The final kinetic energy after the fallback is a ``foe'': \SI{e51}{\erg}. 

If we have a higher mass-loss rate then we will have a lower mass of the CO core, which will then be less compact. The fallback will be less efficient.

The state of the art is that at higher initial masses BHs are more favoured. 
It seems that BHs are more often produced by WR stars.

According to a certain model, BHs are mainly produced by \emph{failed} supernovae: there is a discontinuity around \(M \sim 35 M_{\odot}\), above which we get a collapse and a failed supernova. 
As the metallicity increases, the final masses of the failed supernovae (so, the resulting BHs) decrease. 

Now, we discuss the evolution and final fate of very massive stars. 
We consider stars with \(100 M_{\odot} < M _{\text{in}} < \num{5e4}M_{\odot}\). 
For these, we do not have a core collapse, but a thermonuclear explosion instead. 

There is some evidence for the existence of these stars, for example in the cluster R136a1  and R136a2. 

These stars seem to be very young (\(\sim \SI{2}{Myr}\)). 

We have Super Luminous supernovae (SLSNe) which can have luminosities brighter by a factor of 10 than the regular supernovae. 

We will need to understand under which metallicity regime we can have these SLSNe: there seems to be a metallicity threshold.  

These are ``pair creation supernovae'' because the radiation they emit is more energetic than \SI{1}{MeV}? 

These might be the first contributors to the presence of metal in the universe, the Black Sabbath of the cosmos. 

Pair Creation SNe and Pair Instability SNe are the same thing. 

The relevant phase in stellar evolution is when a large amount of thermal energy goes into pair production instead of producing pressure. 

An important parameter is the mass of the helium core. 

We'd expect more and more massive stars to lose more and more mass loss through winds. 
We need a Helium core with mass such that \(40 M_{\odot} < M _{\text{He}} < 133 M_{\odot}\). If \(M _{\text{He}} < 65 M_{\odot}\) then we have ``Pulsation pair instability supernovae'', above this we have ``Pair creation supernovae''. 

The pair creation triggers the collapse. Then we have a runaway thermonuclear reaction. If this exceeds the binding energy of the star, then we have complete disintegration. 

% The threshold for this turns out to be give by around \(65 M_{\odot}\). 

The pair production makes \(\gamma _{\text{ad}} < 4/3\): this causes a gravitational collapse. 

For \(M _{\text{He}} > 133 M_{\odot}\) the infall produces a BH.

This occurs before oxygen burning. 

We have the relation \(\SI{1}{MeV} \sim \SI{e10}{K}\). 

The star is able to eject most of the material in a series of pulses even if the nuclear energy is smaller than the binding energy. 

The pulses are separated by time intervals on the order of \SI{e5}{s} to \SI{e9}{s}. 

\end{document}