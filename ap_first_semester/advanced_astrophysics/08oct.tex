\documentclass[main.tex]{subfiles}
\begin{document}

\section*{8 October 2019}

Last lecture we started talking about linearization \& perturbation theory.

We use a complex spinning ansatz, which is by itself not physical, however its conjugate is also a solution so after we have found a solution we just need to take the real part.

Substituting \(\zeta (m, t) = \eta (m) \exp(i \sigma t)\) into the LAWE we find that the exponentials simplify: 
%
\begin{align}
-r \sigma^2 \eta  =
4 \pi r^2 \eta \pdv{}{m} \qty((3 \Gamma_1 - 4)P)+
\frac{1}{r} \pdv{}{m} \qty(16 \pi^2 \Gamma_1 P \rho r^6 \pdv{\eta }{m} )  
\,.
\end{align}

We can rewrite this in the Eulerial formalism, using the continuity equation: we get 
%
\begin{align}
- r \sigma^2 \eta = 
\frac{\eta }{\rho } \pdv{}{r} \qty((3\Gamma_1 - 4) P)
+ 4 \dots 
\,,
\end{align}
%


We will have analytic solutions for the adiabatic case, with the additional hypotheses of either \(\Gamma_1 = 4/3\) or \(\Gamma_1 > 4/3\) and homogeneity.

The justification of the adiabatic approx might be asked at the exam.

After plugging our ansatz in the LAWE we should take the real part.

What are the boundary conditions we should set?

\begin{enumerate}
    \item \(\delta r = 0\) at \(r = 0\);
    \item \(\pdv*{\eta}{r} = 0 \) at \(r=0\), which allows us to fix many divergences at the center;
    % \item \(\pdv*{\zeta}{r} < + \infty \) or, equivalently (from the continuity equation) \(3 \zeta + \delta \rho / \rho = 0\);
    \item \((4 + R^3 \sigma^2 / (GM)) \eta + \delta P / P = 0\) at \(r =R\);
    \item \(\eta = \delta r / r = 1\) at \(r = R\).
\end{enumerate}

\begin{greenbox}
  What is the \(\pdv[2]{\eta}{r} \) stuff about?
\end{greenbox}

We use the Eulerian form of the LAWE to figure out the surface boundary conditions.
We assume that all perturbation are in phase, and write them all as proportional to \(\exp(i \sigma t)\).
The pressure scale height is defined as:

\begin{equation}
  H_P = - \qty(\pdv{\log P }{r} )^{-1}
\end{equation}

and represents ``how far we should move in the star for the pressure to change \(e\)-fold''.
By inserting this in the equation we see that \(H_P \rightarrow 0\) when \(r \rightarrow R\): the pressure changes very quickly in the photosphere of the star. Therefore, the term multiplying it must also go to zero.

The last condition comes from the fact that we want our study to give us \emph{periods}, not \emph{amplitudes}: we cannot find those out, so we normalize. The LAWE is 1-homogeneous!

The LAWE can be written compactly with a linear operator \(\mathcal L\):

\begin{equation}
  \mathcal L (\eta) = \sigma^2 \eta\,,
\end{equation}

therefore the eigenvalue is the square of the pulsation.
There are infinitely many solutions to the LAWE, only (finitely many?) fulfill the boundary condition.
The eigenvalues are real [\(\mathcal L\) is Hermitian], have a wavefunction associated: \(\eta_m (r)\) corresponding to \(\sigma_m ^2\).

If \(\sigma^2 > 0\) we have an oscillating solution, if \(\sigma^2 <0 \) we have an exponential collapse or explosion since the solution is proportional to \(\exp(i \sigma t)\).

We label solutions by \emph{radial order} \(m \in \mathbb{N}\): \(m=0\) has the lowest frequency, and then we have overtones.
We choose the labels so that \(\sigma_{m_1} < \sigma_{m_2} \iff m_1 < m_2\).

The radial order \(m\) is also the number of nodes.

The eigenfunctions are orthogonal wrt the scalar product

\begin{equation}
  \braket{\eta_m}{\eta_n} = \int _{0}   ^{R} \eta_m \eta_n \rho r^4 \dd{r}
\end{equation}

\begin{greenbox}
  Possibly there is a \(4 \pi\) missing in order for this to be consistent with the following?
\end{greenbox}

The functions \(\zeta_m\) are orthogonal wrt the same product. The system is linear: we can write a general solution as a superposition.

We can define the moment of inertia:

\begin{equation}
  J_m = \int _{0}   ^{M} \abs{\zeta_m}^2 r^2 \dd{m}
\end{equation}

and the following holds:

\begin{equation}
  \sigma_m^2 = \frac{1}{J_m} \int_0^M \zeta_m ^* \mathcal L \zeta_m r^2 \dd{m}
\end{equation}

\subsection{Simplifications}

\subsubsection{Period-mean density relation}

If \(\eta = \const\), and \(\rho \) and \(\Gamma_1\) are also constanst, we immediately get:

\begin{equation}
  \sigma^2 = (3 \Gamma_1 - 4) \frac{GM}{r^3}
\end{equation}

and by inserting the mean density formula we get the period-mean density relation: this is consistent with our previous assumptions.

\subsubsection{Polytropic model}

It is a gas sphere with the following constitutive equaition:

\begin{equation}
  P = K_n \rho^{1 + 1/n} = K_n \rho^{\frac{n+1}{n}}
\end{equation}
%
with varying \emph{polytropic} index \(n\).
It models spheres with different mass distributions:
\(n=0\) is constant density, \(n=5\) is infinite central density, \(n=3\) is the Eddington standard model, which is reasonable for the Sun and stars on the main sequence.

With these assumptions we can explicitly solve for the wavefunctions, and make predictions of the fractional modulus of the oscillations at a certain radius wrt the modulus at the surface (which can be found only experimentally).

The overtones die out toward the center even faster than the fundamental: these oscillations are very much a \emph{surface phenomenon}.

Beyond the \(\eta\)s we can also plot the pressure perturbations: these will not be normalized.

\subsubsection{A concrete example}

This is done looking at an RR Lyrae variable.
We integrate the stellar structure equations numerically.
We can see that \(\sigma_m - \sigma_{m-1} \approx \const\) when \(m\) gets large.

The wavefunctions die out faster than the polytropic model when \(r \rightarrow 0\).

There are ``bumps'' in the pressure plot: these are the partial ionization regions of \ce{H} and \ce{He}.

These appear because we start from a solution of the stellar structure equations, where all the properties of stellar matter were used, to start off with the LAWE.

\section{Non-adiabatic oscillations}

How can we tell, theoretically, how stable and how wide the various modes are?
We expect to see the stable modes, and not to see the unstable ones.

Let us start from the Lagrangian momentum conservation:

\begin{equation}
    \pdv[2]{r}{t} = - 4 \pi r^2 \pdv{P}{m} - \frac{Gm}{r}
\end{equation}

and apply to it the identity: \(\frac[i]{1}{2} \pdv{}{t} v^2 = \pdv{r}{t} \pdv[2]{r}{t}\), by multiplying everything by \(\pdv*{r}{t} \).

We then integrate everything with respect to \(m\) and apply some manipulations [see slides].

\begin{equation}
  \pdv{}{t} \int_M \frac{v^2}{2} \dd{m} =
  - \dv{\Omega}{t} + \int_M P \pdv{}{t} \frac{1}{\rho} \dd{m}
\end{equation}

We integrate a pulsation period, which cancels out the gravitational potential term which is conservative.

\begin{equation}
  \expval{\dv{W}{t}}_\Pi = \frac{1}{\Pi} \int_\Pi \int_M   P \pdv{}{t} \qty(\frac{1}{\rho}) \dd{m} \dd{t}
\end{equation}

Some layers will provide energy to the oscillation motion (\emph{drive} it), some others will \emph{damp} it.
These are characterized by the sign of the RHS of this equation.

If it is positive, we have instability; if it is negative the pulsations will tend to die out, giving stability.

The average time scale of change of the perturbations is

\begin{equation}
  \kappa \defeq \frac{1}{\tau} = -\frac{1}{2} \frac{\expval{\dv*{W}{t}}_\Pi}{\expval{ \delta \psi}_\Pi }
\end{equation}

The term \(\expval{\dv*{W}{t}}_\Pi \) can also be interpreted as the net heat gain fed into mechanical work during a pulsation cycle.

In the adiabatic case, we had \(\pdv{}{t}\qty( \frac{\delta P}{P} - \Gamma_1 \frac{\delta \rho}{\rho}) =0 \): the perturbations were in phase.

Now we add a term to the time derivatives: the pressure and density perturbations will stop being in phase. The sign of the heat variation term gives us the difference between \emph{driving} heat transfer and \emph{damping} heat transfer.

In a PV diagram, we can see that these correspond to right and left oriented loops (as opposed to the loops with zero total signed area we had in the adiabatic case).

\begin{bluebox}
    The star is effectively a themal engine converting heat into work; this will result in an increased overall entropy of the star, and a smoothing of its temperature gradient, however:
    %
    \begin{enumerate}
        \item the timescales on which this process occurs are much larger than the timescales on which oscillating motions are created and destroyed;
        \item then energies of the oscillations are much smaller than the global thermal energy of the star.
    \end{enumerate}
    %
    therefore this process is typically not relevant.
\end{bluebox}

\end{document}
