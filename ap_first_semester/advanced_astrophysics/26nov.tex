\documentclass[main.tex]{subfiles}
\begin{document}

\marginpar{Tuesday \\ 2019-11-26}

% (Based on few Giorgio notes-BOFGN hereafter)

\paragraph{Pulsation assisted winds}

The density \(\rho_{c}\) at the condensation radius \(r_C\) is critical: it determines the mass loss rate \(\dot{M}\). 
If it is too low no wind can be generated.
Therefore, there is a need for a mechanism to increase the density scale height: these winds are \emph{pulsation assisted}, the driving mechanism are shocks occurring in pulsating stars.
So, \(\dot{M}\) is very sensitive to these pulsations, which ``launch'' the gas to a height sufficient for it to form dust grains.

The wind starts to be accelerated at the critical radius \(r_C\), and its acceleration is so fast that it is reasonable to assume that this critical radius is the same as the sonic radius at which the gas reaches \(v=a\). 
This is the dust formation radius and also the condensation radius.

The momentum equation of the dust driven wind reads: 
%
\begin{align}
v \dv{v}{r} + \frac{GM}{r^2} + \frac{1}{\rho }\dv{P}{r} = g _{\text{rad}} = \frac{GM}{r^2} \Gamma_{d}
\,,
\end{align}
%
with the \(\Gamma_{d}\) defined in \eqref{eq:definition-gamma-dust}, proportional to the opacity \(\kappa_{rp}\).

\paragraph{Maximum mass loss rate}

In this case we also have a maximum mass loss rate given by 
%
\begin{align}
\dot{M} _{\text{max}} v_{ \infty } = \frac{L}{c}
\,,
\end{align}
%
which is rather low since the asymptotic velocity is low.

As was the case with line driven winds, multiple scattering can enhance this maximum mass loss rate by a factor depending on the optical depth \(\tau_{w} = \int_{r_s}^{ \infty } \kappa \rho \dd{r}\), however now we also have a dependence on \(\Gamma_{d}\): 
%
\begin{align}
\dot{M} _{\text{max}} = \frac{L}{v_{ \infty } c } \qty(\frac{\Gamma_{d} -1}{\Gamma_{d}}) \tau_{w}
\,.
\end{align}

\begin{bluebox}
Here is the derivation of this expression, from Lamers:
we integrate the momentum equation multiplied by \(4 \pi r^2 \rho \) from the star's radius \(R_{*}\) to infinity, but we split certain integrals into the contributions from \(R_{*}\) to \(r_s\) and from \(r_s\) to infinity. We get: 
%
\begin{subequations}
\begin{align}
0 &= \int_{R_*}^{ \infty } \qty[ 4 \pi r^2 \rho  v \dv{v}{r} + 4 \pi GM \rho    
+ 4 \pi r^2 \dv{P}{r} - 4 \pi GM \Gamma_{d}] \dd{r}  \\
\begin{split}
0 &= \int_{R_*}^{ \infty } 4 \pi r^2 \rho  v \dv{v}{r} \dd{r}
+ \int_{R_*}^{r_s}  \qty[\frac{1}{\rho } \dv{P}{r} + \frac{GM}{r^2} ] 4 \pi r^2 \rho \dd{r} + \\
&\phantom{=}\ 
+ \int_{r_s}^{ \infty } \frac{1}{\rho } \dv{P}{r} 4 \pi r^2 \rho \dd{r} 
+ \int_{r_s}^{ \infty } \frac{GM}{r^2} (1 - \Gamma_{d})
\rho 4 \pi r^2 \dd{r} 
\,,
\end{split}
\end{align}
\end{subequations}
%
but \(4 \pi r^2 \rho \dd{r} = \dd{m}\), so the second integral can be written as 
%
\begin{align}
\int_{R_*}^{r_s} \frac{1}{\rho } \dv{P}{r}  + \frac{GM}{r^2}
 \dd{m}
\,,
\end{align}
%
which is precisely 0 if we have hydrostatic balance, and we do under \(r_s\). So this term can be neglected. 
Similarly, the third integral is 
%
\begin{align}
\int_{r_s}^{ \infty } \frac{1}{\rho } \dv{P}{r} \dd{m}
\,,
\end{align}
%
and it is also negligible since the pressure gradient is very low beyond the sonic point.

The first integral can be written as 
%
\begin{align}
\int_{R_*}^{ \infty } v \dv{v}{r} \dd{m}
\,,
\end{align}
%
and we can recognize the expression of the convective derivative: \(\dot{v}  = v \dv*{v}{r}\) if the wind is stationary. The contribution to this integral at \(R_*\) is negligible: the term which remains is \(\dot{M} v_{ \infty }\). So, we are left with 
%
\begin{align}
\dot{M} v_{ \infty } \approx - 4 \pi  GM (1- \Gamma_{d}) \int_{r_s}^{ \infty }
\rho \dd{r}
\,,
\end{align}
%
and we define \(\tau_{w} = \int_{r_s}^{ \infty } \rho \kappa_{rp} \dd{r}\) we can express this as 
%
\begin{align}
\dot{M} v_{ \infty } \approx 4 \pi GM \frac{\Gamma_{d} -1 }{\kappa_{rf}} \tau_{w}
\,,
\end{align}
%
and now we can insert the expression for \(\kappa_{rf}\) which is found by inverting the definition of \(\Gamma_{d}\): we get 
%
\begin{align}
\kappa_{rp} = \frac{4 \pi c GM }{L} \Gamma_{d}
\,,
\end{align}
%
so 
%
\begin{subequations}
\begin{align}
\dot{M} v_{ \infty } &\approx 4 \pi  GM \frac{L}{4 \pi c GM} \frac{\Gamma_{d} - 1}{\Gamma_{d}} \tau_{w}  \\
&\approx \frac{L}{c} \frac{\Gamma_{d} - 1}{\Gamma_{d}} \tau_{w}
\,,
\end{align}
\end{subequations}
%
and far from the star \(\Gamma_{d} \gg 1\), so the expression simplifies to 
%
\begin{align}
\dot{M} \approx \frac{L}{c v_{ \infty }}\tau_{w}
\,.
\end{align}

It would seem that \(\tau \) can be arbitrarily large, but this is not the case: by conservation of energy, we can see that 
%
\begin{subequations}
\begin{align}
\frac{1}{2} \dot{M} v_{ \infty }^2 &< L  \\
\dot{M} &< \frac{2L}{v_{ \infty }^2}
\,,
\end{align}
\end{subequations}
%
so, comparing the formulas, we see that \(\tau_{w} < 2 c / v_{ \infty }\). 
More precise considerations allow us to drop the factor \(2\). 
\end{bluebox}

\paragraph{Dust grain opacities}

In this lecture the attention was focused on the importance of C/O ratio in the formation of dust grains in dust driven winds. 

In AGB stars the core, mainly made of heavier elements such as carbon and oxygen, is sourrounded by a helium envelope and, more externally, a hydrogen envelope. 
Between these areas there exist some convective flows that can bring oxygen and carbon from the core to the more external parts of the star. 
This process is called the \emph{third dredge-up}.

Reaching thanks to convection these external layers, which lay at a lower temperature, the heavy elements can form molecular bounds.
In particular the strongest bounded molecule that can be formed is carbon monoxyde (\ce{CO}).
This implies that, if the star has a C/O ratio $>1$ (i.e.\ it has more abundance of carbon over oxygen) almost all the oxygen is used to form carbon monoxyde, while the rest of the carbon can form other molecules: for this reason, in these stars we observe carbon grains.

Viceversa, in oxygen-rich stars we have with the specular mechanism, which ends in the formation of silicate grains.
At this point we started a discussion that will be continued in the following lecture about dust grains' opacity, focusing first on cross section.

We must distinguish two different kinds of opacities: 
\begin{enumerate}
  \item the absorptive grain opacity \(\kappa_{d}\): this accounts for pure absorption \emph{without} scattering, it determines the pressure, so it defines the condensation radius;
  \item the radiation pressure mean opacity \(k_{rp}\):\footnote{Do note that this is a latin letter \(k\), while the other opacity is a greek \(\kappa \).} this accounts for both absorption and scattering. Scattering does not affect the temperature of the gas, but it \emph{does} affect the radiative acceleration.
\end{enumerate}

Let $a$ be the radius of the grain, and $Q$ the efficency of the cross section, which depends both on \(a\) and on the wavelength \(\lambda \) of the light.
The distribution of the grain radii is as such: 
%
\begin{align}
n(a) \dd{a} \approx K a^{-3.5} [a _{\text{min}} < a < a _{\text{max}} ] \dd{a}
\,,
\end{align}
%
where \(a _{\text{min}} \approx \SI{5}{nm}\) while \(a _{\text{max}} \approx \SI{250}{nm}\), while \(k\) is a normalization factor.


In this case we have a cross section $c=\pi a^2$ and, considering both absorption and scattering cross sections $c_a$, $c_s$, we have
%
\begin{equation}
c_{\text{tot}}
=c_a+c_s
=\pi a^2(Q_a(a,\lambda)+Q_c(a,\lambda))
\end{equation}
%
where $Q_a \ll Q_s$ for IR wavelengths: typically the absorptive efficiency \(Q_a\) can be modelled by \(Q_a \propto a \lambda^{-p}\), where the exponent \(p\) is around 1 in the infrared --- it then approaches 2 for longer wavelengths;\footnote{Actually the exponent is very much dependent on the chemical composition of the dust, this is an average.} on the other hand the scattering efficiency is in the Rayleigh regime, \(Q_s \propto \lambda^{-4}\).

Now we can define the global absorption opacity
%
\begin{equation}
k_\lambda=\frac{1}{\rho } \int_{a_{\text{min}}}^{a_{\text{max}}}Q_a\pi a^2 n \dd{a}
\,,
\end{equation}
%
the scattering opacity 
%
\begin{align}
\sigma_{\lambda } = \frac{1}{\rho } \int_{a_{\text{min}}}^{a_{\text{max}}} Q_s \pi a^2 n \dd{a}
\,,
\end{align}
%
and the mean opacity
%
\begin{equation}
k=\int \dd{\lambda} k_\lambda
\end{equation}
%
which is the relevant quantity for the momentum equation.
\end{document}