\documentclass[main.tex]{subfiles}
\begin{document}

\section*{Tue Oct 29 2019}

Recall the equations from last time: continuity, momentum and energy conservation in the isothermal case:
%
\begin{subequations}
\begin{align}
  \dot{M}  &= 4 \pi r^2 \rho v  \\
  v \dv{v}{r} &= - \frac{1}{\rho } \dv{P}{r} - \frac{GM}{r^2} \\
  T(r) &\equiv T 
\,.
\end{align}
\end{subequations}

The term \(\rho^{-1} \dv*{P}{r} \) is equal to 
%
\begin{equation}
  \frac{RT}{\mu } \frac{1}{\rho } \dv{\rho }{r} 
\,,
\end{equation}
%
so we manipulate the equation into 
%
\begin{align}
  v \dv{v}{r}  &= - \frac{1}{\rho } \frac{RT}{\mu } \dv{\rho }{r} - \frac{GM}{r^2} 
\,,
\end{align}
%
but we know by the continuity equation that 
%
\begin{equation}
  - \frac{1}{\rho } \dv{\rho }{r} = \frac{1}{v} \dv{v}{r} + \frac{2}{r}     
\,,
\end{equation}
%
which we can substitute into the equation: we get
%
\begin{align}
    v \dv{v}{r}  &=  \frac{RT}{\mu } \qty(\frac{1}{v} \dv{v}{r} + \frac{2}{r}) - \frac{GM}{r^2} 
  \,.
\end{align}

The isothermal speed of sound is given by 
%
\begin{equation}
  a^2 = \pdv{P}{\rho } =\pdv{}{\rho } \qty(\frac{R \rho T}{\mu }) = \frac{RT }{\mu }
\,,
\end{equation}
%
so 
%
\begin{align}
    v \dv{v}{r}  &=  a^2 \qty(\frac{1}{v} \dv{v}{r} + \frac{2}{r}) - \frac{GM}{r^2} 
  \,.
\end{align}

Now we put all the terms which are proportional to the velocity gradient on the LHS: 
%
\begin{equation}
  \dv{v}{r} \qty(v - \frac{a^2}{v}) = \frac{2a^2 }{r }-\frac{GM}{r^2}
\,,
\end{equation}
%
which is just 
%
\begin{equation}
    \frac{1}{v}\dv{v}{r} \qty(v^2 - a^2) = \frac{2a^2 }{r }-\frac{GM}{r^2}
  \,,
\end{equation}
%
so the Jacobian of the differential equation is zero is singular in \(v=a\): if \(v=a\) we must have \(2a^2r = GM\), which fixes the radius to the Parker radius: \(r_P = GM / 2 a^2\).
Close to the star, the speed is subsonic and the numerator is negative in 
%
\begin{equation}
  \frac{1}{v} \dv{v}{r} = \frac{N}{D}
\,,
\end{equation}
%
which is consistent with our assumption \(\dv*{v}{r} > 0 \), which we make since we are considering winds, as opposed to accretion.

Far from the star the numerator is positive, so the speed must still be supersonic.

The critical velocity \emph{must} be attained at the Parker radius in order to have a phyisically meaningful solution.

The velocity gradient at the critical point can be found by de l'Hôpital's rule to be 
%
\begin{equation}
  \dv{v}{r} \bigg|_{r_P} = \pm \frac{a^3}{GM}
\,.
\end{equation}

So, the only physical solution is transsonic.

\begin{claim}[Exercise]
The speed of sound at the critical point equals half of the escape velocity at that radius.
\end{claim}

The boundary condition is the velocity at some \(r_0 \).

There are other solutions, but if we trace a cross in the \(r, v\) plane centered on the critical point and speed of sound we see that all solutions meet it perpendicular to it.

Accretion solutions are also found in this diagram: what is plotted is the \emph{absolute value} of the velocity. They have always-negative absolute velocity gradient.

Always-supersonic solutions and always-subsonic ones are also found, but they do not obey the monotonicity of the velocity.

The choice we make for \(v_0  = v(r_0)\) is key, and nontrivial.

If we have the density, velocity and radius of the lower boundary we have fixed the accretion rate: \(\dot{M}  = 4 \pi r_0^2 \rho_0 v_0 \).
We can fix these by fixing the constant temperature \(T\) and the velocity \(v_0 \) at the critical radius \(r_0\) and assuming our solution is transsonic.

\todo[inline]{What is stated in the slides seems different. What are we fixing?}

We can solve the momentum equation analytically: we get 
%
\begin{equation}
  \frac{v}{v_0 } \exp( \frac{-v^2}{2 a^2}) = \qty(\frac{r_0}{r})^2 \exp(\frac{GM}{a^2} \qty(\frac{1}{r_0 } - \frac{1}{r}))  
\,.
\end{equation}
%

At large distances, we get \(v \rightarrow 2 a \ln(r/r_0)\).

Now we look at the structure of the wind in the subcritical region.

In the corresponding slide: the dashed line in the density profile is the density we'd expect in a hydrostatic atmosphere, while the solid one is the solution.

The hydrostatic density structure is given by 
%
\begin{subequations}
\begin{align}
  \frac{1}{\rho } \dv{P}{r}  +\frac{GM }{r^2}  &= 0 
\,,
\end{align}
\end{subequations}
%
manipulating this we get 
%
\begin{equation}
  \frac{r^2}{\rho } \dv{\rho }{r} = - \frac{GM}{a^2}
\,,
\end{equation}
%
So the density profile follows a decreasing exponential law: 
%
\begin{equation}
  \frac{\rho (r)}{\rho_0 } = \exp(\frac{-(r-r_0 )}{H_0 } \frac{r_0}{r} ) 
\,,
\end{equation}
%
where \(H_0 \) is the scale height, \(H_0 = RT / \qty(\mu g_0 )\) with \(g_0 = GM / r_0^2\).
The length scale at which this decreases is defined by \(H_0 \). 

The density profile in the subsonic region is very well approximated by this hydrostatic profile.

The mass loss rate is our main prediction: we have \(\dot{M } = 4 \pi r^2_0 \rho_0 v_0 = 4 \pi r_c^2\rho_c a\).

Then, we can use the density profile equation: 
%
\begin{equation}
  \dot{M} = 4 \pi r_c^2 a \rho_0 \exp(\frac{-(r_c - r_0 )}{H_0 } \frac{r_0 }{r_c}) 
\,.
\end{equation}
%

If we consider this numerically, we find that the exponential is the dominant part.
The mass loss rate is lower when the critical point moves outward.
We can specify it by fixing 
%
\begin{enumerate}
    \item the temperature at the corona, \(T_C\);
    \item the radius at the bottom of the corona, \(r_0 \);
    \item the stellar mass \(M\);
    \item the density at the bottom of the corona \(\rho_0 \).
\end{enumerate}

In the slides there are numerical estimates. 
As \(H_0 \) increases, the density profile is less steep: the density remains high at larger radii.

\end{document}
