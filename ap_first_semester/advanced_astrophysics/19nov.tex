\documentclass[main.tex]{subfiles}
\begin{document}

\section*{Tue Nov 19 2019}

\subsubsection{Which lines drive the winds?}

We classify the driving lines according to which series of the spectral lines of hydrogen they are closest to, even though hydrogen is not the element which is ionized and responsible for line driven winds.
Recall: the wavenumbers of the lines are given by 
%
\begin{align}
\lambda^{-1} = R_H \qty(\frac{1}{n^2}- \frac{1}{m^2})
\,,
\end{align}
%
so, the series which are relevant for us are:
\begin{enumerate}
  \item \(n=1\) gives the Lyman lines, which are around \(\lambda \sim \SI{1000}{\angstrom}\) --- these drive the winds of the very hottest stars;
  \item \(n=2\) gives the Balmer lines, which are around \(\lambda \sim \SI{4000}{\angstrom} \divisionsymbol \SI{6000}{\angstrom}\) --- these drive the winds of stars between \SI{20}{\kilo\kelvin} and \SI{30}{\kilo\kelvin};
  \item \(n=3\) gives the Paschen lines, but actually even the winds of stars below \SI{10}{\kilo\kelvin} are driven by lines in the Balmer continuum.
\end{enumerate}

As the effective temperature decreases, the Planckian shifts to longer wavelengths, and since the gas is on average less ionized, the absorption lines move to longer wavelengths as well.

\paragraph{Which elements contribute most?}

The dependence on temperature is significant. A synthesis is as follows: 
\begin{enumerate}
  \item below \SI{25}{\kilo\kelvin} the iron-group elements are most important;
  \item above \SI{25}{\kilo\kelvin} Carbon, Nitrogen, Oxygen, Neon and Calcium are more important.
\end{enumerate}

Hydrogen and helium hardly matter at all.
 
\subsubsection{The theory of line driven winds}

Yesterday we introduced the CAK formalism. 

We consider a spherically symmetric, stationary problem, and we assume that the star looks pointlike to us. 
Also, we assume that the process is isothermal: \(\forall r: T = \const \). 

The momentum conservation equation is 
%
\begin{align}
v \dv{v}{r} = - \frac{GM}{r^2} - \frac{1}{\rho} \dv{P}{r} 
+ g_{c} + g_{L} 
\,,
\end{align}
%
where we distinguished the continuum acceleration \(g_{c}\) and the line acceleration \(g_{L}\), which we assume is expressible in the form described by CAK: \(g_L = g_e k t^{-\alpha } s^{ \delta } \). 

We assume that the gas follows the ideal gas law: 
%
\begin{align}
  P = \frac{R \rho T}{\mu }
\,,
\end{align}
%
which implies 
%
\begin{align}
  - \frac{1}{\rho } \dv{P}{r} = - \frac{1}{\rho }
  \dv{}{r} \qty(\frac{R T \rho }{\mu })
  = - \frac{1}{\rho } \frac{RT}{\mu } \dv{\rho }{r}
\,,
\end{align}
%
where we can recognize the square speed of sound \(a^2 = RT / \mu \). Using the continuity equation then we get 
%
\begin{align}
  - \frac{1}{\rho } \dv{P}{r} = a^2 \qty(\frac{1}{v} \dv{v}{r} + \frac{2}{r})
\,.
\end{align}

For the continuum and line acceleration we use the CAK formalism: 
%
\begin{align}
  g_{c} = \frac{GM}{r^2} \Gamma_{e}
\,,
\end{align}
%
and 
%
\begin{align}
  g_{L} = \frac{\kappa_{e}}{c} \frac{L}{4 \pi r^2} k t^{-\alpha } s^{ \delta }
\,,
\end{align}
%
so in the end we get: 
%
\begin{align}
    v \dv{v}{r} = - \frac{GM}{r^2} + a^2 \qty(\frac{1}{v} \dv{v}{r} + \frac{2}{r}) 
    + \frac{GM}{r^2} \Gamma_{c} + \frac{\kappa_{e}}{c} \frac{L}{4 \pi r^2} k t^{-\alpha } s^{ \delta }
\,.
\end{align}

Now recall 
%
\begin{align}
  t = C_T \rho \dv{r}{v} = \frac{C_T \dot{M}}{4 \pi } \qty(r^2 v \dv{v}{r})^{-1}
\,,
\end{align}
%
where we defined the constant \(C_T = \kappa_{e} \sqrt{2 k_B T _{\text{eff}} / m_H}\).

We plug this into the momentum conservation equation and multiply by \(r^2\): 
%
\begin{align}
    v r^2 \dv{v}{r} =
    - GM (1 - \Gamma_{e})
    + a^2 r^2 \qty(\frac{1}{v} \dv{v}{r} + \frac{2}{r}) 
    + C \qty(r^2 v \dv{v}{r})^{\alpha } 
\,,
\end{align}
%
where 
%
\begin{align}
  C = \frac{\kappa_{e}}{c} \frac{L}{4 \pi} k \qty(C_T \frac{\dot{M}}{4 \pi })^{-\alpha } \qty(\frac{\num{e11} \rho }{m_H W })^{ \delta }
\,. 
\end{align}

Also, we can bring a term to the RHS: we get 
%
\begin{align}
    \qty(1 - \frac{a^2}{v^2}) v r^2 \dv{v}{r} =
    - GM (1 - \Gamma_{e})
    + 2a^2 r 
    + C \qty(r^2 v \dv{v}{r})^{\alpha } 
\,.
\end{align}
%

This equation has a critical point corresponding to the speed of sound, similarly to the ones we encountered before, but this time it is nonlinear.

Now, we make a simplifying assumption: we neglect the gas pressure. This is reasonable at large distances from the star.
It also can be shown that the main physical characteristics we will find are the same we would find if we did consider the gas pressure. 

Since the speed of sound is \(a^2 = \pdv*{P}{\rho }\) neglecting the pressure gradient means \(a=0\): removing  the terms with \(a\) we get 
%
\begin{align}
  r^2 v \dv{v}{r} - C \qty(r^2 v \dv{v}{r})^{\alpha } 
  = - GM (1 - \Gamma_{e}) = \const
\,,
\end{align}
%
and if we denote \(r^2  v \dv*{v}{r} = D\), the equation can be written as 
%
\begin{align}
  D - C D^{\alpha } = - GM \qty(1 - \Gamma_e)
\,.
\end{align}

Our parameters here are \(D\) and \(C \propto \dot{M}^{-\alpha }\).

The system is equivalent to \(CD^{\alpha } = D + GM(1-\Gamma_{e})\): so the solutions, when plotted with respect to \(D\), are the intersections between a slope-1 straight line and a powerlaw (with index \(0 < \alpha <1\)). 

There are values of \(C\) such that there is no solution. 

We select the value of \(C\) such that the system has a unique solution --- we do this because, inspecting the equation more closely, we find that it is also a critical-point equation, and if we want a monotonic velocity gradient we must have the solution passing through the critical point.

\todo[inline]{This is not really justified: the critical point condition should be set by setting the numerator of the full equation, with \(a\), to zero when \(v=a\)\dots I guess it works this way.}

So we differentiate \(D - C D^{\alpha } + GM (1-\Gamma_{e})\) with respect to \(D\), set it equal to \(0\) and find \(C = D^{1-\alpha } / \alpha \).

Plugging this expression for \(C\) in we find a differential equation: 
%
\begin{align}
  r^2 v \dv{v}{r} = D = \frac{\alpha }{1 - \alpha } GM_{*} (1-\Gamma_{e})
\,,
\end{align}
%
which can be solved by direct integration if we express it like: 
%
\begin{align}
\frac{1}{2} \dv{}{r} \qty(v^2)
&= \frac{\alpha }{1-\alpha } GM_{*} (1 - \Gamma_{e}) \\
v^2 &= \int  \frac{\alpha }{1-\alpha } 2GM_{*} (1 - \Gamma_{e}) \dd{r} 
\,,
\end{align}
%
so the solution is 
%
\begin{align}
  v(r) = \qty(\frac{\alpha }{1-\alpha } 2GM_{*} \qty(1-\Gamma_{e} ) \qty(\frac{1}{R_{*}} - \frac{1}{r}))^{1/2}
  = v_{ \infty } \sqrt{1 - \frac{R_{*}}{r}}
\,. 
\end{align}
%
this is a \(\beta \)-type law, with \(\beta = \num{.5}\) and 
%
\begin{align}
  v_{ \infty } = \sqrt{\frac{\alpha }{1-\alpha } 2GM_{*} \qty(\frac{1-\Gamma_{e} }{R_{*}})}
  = v _{\text{esc}} \sqrt{\frac{\alpha }{1 - \alpha }}
\,,
\end{align}
%
since the escape velocity for electrons in the photosphere is 
%
\begin{align}
  v _{\text{esc}} = \sqrt{\frac{2 G M_{*} (1 - \Gamma_{e })}{R_{*}}}
\,,
\end{align}
%
and we include \(\Gamma_{e} \) in the escape velocity since there is electron scattering there.

This gives us a rather complicated but well-defined expression for the mass loss rate, which appears in \(C\), for which we have an expression in terms of \(D\): we find
%
\begin{align}
\dot{M} = \frac{4 \pi }{C} \qty(\frac{\kappa_{e}}{4 \pi})^{1/\alpha } \qty(\frac{1-\alpha }{\alpha })^{\frac{1-\alpha }{\alpha }} \qty(k \alpha )^{1/\alpha }
s^{ \delta / \alpha } \qty(\frac{L_{*}}{c})^{1/\alpha }
\qty(GM_{*} (1-\Gamma_{e}))^{(\alpha -1) / \alpha }
\,,
\end{align}
in which the main dependences are \(\dot{M} \propto L_{*}^{ 1/ \alpha } M_{*}^{(\alpha - 1 ) / \alpha }\).

For Main Sequence stars \(L_{*} \sim M_{*}^{2.5}\), so we can turn the luminosity dependence into a mass dependence: \(\dot{M} \sim M^{(3.5 \alpha  - 1 ) / \alpha }\), or similarly into a luminosity dependence: 
%
\begin{align}
\dot{M} \sim L^{\frac{2}{5} + \frac{3}{5 \alpha }}
\,.
\end{align}
%

For \(\alpha = 0.52\) we get approximately \(\dot{M} \sim M^{1.58}\) and \(\dot{M} \sim L^{1.55}\).
This is the same result we would see if we did consider the gas pressure.

\paragraph{The terminal velocity does not depend on the mass loss rate}
An important result we found is the fact that \(D\) is a constant, regardless of the value of \(\alpha \) (so, for any value of the optical thickness of the line). 

The asymptotic velocity does not depend on the mass loss rate, even though the radiative acceleration does (\(g _{\text{line}} \sim \dot{M}^{\alpha }\)).
This is because we are fixing the passage of the wind velocity law through the critical point.

\paragraph{Corrections due to the finite size of the star}

Now, we talk about the corrections due to the finite size of a star. Near the star, we cannot approximate it as a point source: the direction of the momentum provided by the radiation is not all radial anymore; a significant part of the momentum is in a non-radial direction.

Accounting for this is in general hard to do, the result is: we get a lower mass loss rate, and a higher terminal velocity. 

The line acceleration formula picks up a correction factor \(D_f\), which is derived geometrically: 
%
\begin{align}
D_f = \frac{(1+\sigma )^{\alpha +1 } - (1+\sigma \mu_{*}^2)^{\alpha +1}}{ (1 + \mu_{*}^2) (\alpha +1) \sigma (1+\sigma )^{\alpha }}
\,,
\end{align}
%
where \(\mu_{*} = \cos(\theta ) = \sqrt{1 - (R_{*} /r)^2}\),
with \(\theta \) being half of the angle subtended by the star from the point of view we are considering. \(\alpha \) is the same as before, while \(\sigma \) is defined as 
%
\begin{align}
\sigma = \frac{r}{v} \dv{v}{r}  -1
\,.
\end{align}
%
\todo[inline]{The definition of \(\mu_{*}\) is inconsistent in the slides, I suspect that the square needed to make it into a cosine of the angle was forgotten in Lamers' book.}

So now our radiative acceleration will look like 
%
\begin{align}
g _{\text{line}} = \frac{\kappa_{e}}{c}\frac{L_{*}}{4 \pi r^2} k t^{-\alpha } s^{ \delta } D_f
\,,
\end{align}
%


The correction factor can be both below and above unity, depending on the distance from the star and on \(\beta \). 
As we'd expect in the \(r \rightarrow \infty \) limit we have \(D_f \rightarrow 1\). 
For most realistic values of \(\beta \) it starts off below 1 near the star, then rises above it (by not much, it becomes of the order \num{1.1}) and then tends towards 1.

\begin{bluebox}
At this point, Lamer states that a decrease in the effective optical thickness of the line causes a decrease of the line acceleration. 
This is counterintuitive: after all, if the line absorbs more it should also accelerate more, right?
Actually, the effect is the opposite, as I will now show: we prove that if \(\beta > .5\) then
%
\begin{align}
\dv{g _{\text{line}}}{\alpha } < 0
\,,
\end{align}
%
which means that as the optical thickness increases (\(\alpha \) increases) the radiative acceleration decreases.

We know that \(g _{\text{line}} \sim D^{\alpha }\), and let us use a \(\beta \)-type velocity law. Then, we have (setting \(R_{*} =1\) here):
%
\begin{align}
\dv{v}{r} = \frac{1}{r^2} \qty(1 - \frac{1}{r})^{\beta -1}
\,,
\end{align}
%
therefore 
%
\begin{align}
D^{\alpha } = \qty(v r^2 \dv{v}{r})^{\alpha}
= \qty(1 - \frac{1}{r})^{\alpha (2\beta -1)}
\,,
\end{align}
%
so 
%
\begin{align}
\dv{}{\alpha } D^{\alpha }
&= \dv{}{\alpha } \exp(\alpha \log D)
= D^{\alpha } \log D  \\
&= \qty(1 - \frac{1}{r})^{\alpha (2 \beta -1)}
(2 \beta -1 ) \log \qty(1 - \frac{1}{r})
\,,
\end{align}
%
where, as long as \(2 \beta -1 >0\), all the terms are positive except for \(\log (1 - 1/r)\), which is negative outside the star.
This proves that an increase in optical depth decreases the radiative acceleration.
\end{bluebox}

Typically, the mass loss rate variation due to this correction is similar to multiplying it by \(1/2\), while the escape velocity is approximately multiplied by 2 (although the multiplier varies with effective temperature).

\paragraph{Line driven winds' instability}

Phenomena which are not accounted for by this model are: 
\begin{enumerate}
    \item X-ray emission;
    \item superionization (beyond the sixth line);
    \item discrete absorption components \& variability.
\end{enumerate}

In general these winds are unstable to small-amplitude stellar oscillations: they will feel shocks.

These can be simulated: the time-averaged \(v_{ \infty }\) and \(\dot{M}\) are similar to those found in stationary models. 
The fact that the line acceleration has a positive dependence on the velocity gradient means that the velocity profile is unstable to perturbations. However, they are generally more or less sinusoidal so their effects cancel.

The takeaways are: 
\begin{enumerate}
  \item \(v_{ \infty }\) and \(\dot{M}\) in perturbed models are similar to those found in unperturbed ones;
  \item the amplitude of the velocity perturbation is comparable to the velocity itself and the perturbations affect a large part of the wind: so, the P-Cygni profiles are broadened;
  \item if the density is high the temperatures in the perturbed regions rise significantly: we then see X-ray emission and ionization of very energetic lines.
\end{enumerate}

\end{document}