\documentclass[main.tex]{subfiles}
\begin{document}

\section*{7 October 2019}

% The exam for this part of the course: it might be around the end of october.

% The subscripts on last lecture \emph{were} inverted after all.

We introduce the \emph{mirror principle}: when the core contracts or expands, the envelope does the opposite.

The shell must remain at around the same temperature to maintain equilibrium: contracting the core would increase the temperature, therefore the envelope exapands.
This heuristic argument is actually derived from simulations.

The relevant time scale for oscillations is the free-fall, dynamical time scale.

We come back to the energy equation

\begin{equation}
  \pdv{L}{m} = \varepsilon - \varepsilon_\nu - \varepsilon_g
\end{equation}

we incorporate the nuclear energy generation rate and the energy lost as neutrino production into an effective energy generation rate per unit mass \(\varepsilon - \varepsilon_\nu = \varepsilon_{\text{eff}} \) and express the energy absorbed by the stellar layer as 
%
\begin{align} \label{eq:heat-transfer-effective}
 \varepsilon_g = \dv{Q}{t} = \varepsilon_{\text{eff}} - \pdv{L}{m}
\,.
\end{align}
%

This makes the meaning of this transfer equation clearer.
Using the first and second laws of thermodynamics, and recalling some thermodynamical values: the specific heat at constant volume \(c_V = \qty(\pdv{Q}{T})_V \), the equation of state exponents \(\chi_T \) and \(\chi_\rho
\) which satisfy: \(P = T^{ \chi_{T}}\) and \(P = \rho^{\chi_{\rho }}\) and the adiabatic exponents \(\Gamma_{1,2,3}\), which are defined by 
%
\begin{subequations}
\begin{align}
\Gamma_1 = \gamma_{\text{ad}} &= \qty(\pdv{\log P}{\log \rho })_{s}  \\
\frac{\Gamma_2 }{\Gamma_2 -1} = \frac{1}{\nabla _{\text{ad}}} &= \qty(\pdv{\log P}{\log T})_{s}  \\
\Gamma_3 -1 &= \qty(\pdv{\log T}{\log \rho })_{s}  \\
\,
\end{align}
\end{subequations}
%
and satisfy 
%
\begin{align}
\frac{\Gamma_1 }{\Gamma_3 -1} = \frac{\Gamma_2 }{\Gamma_2 -1}
\,.
\end{align}

These are all \emph{exponents} in some power law.
We use log values since our variables change by orders of magnitude.

We start with  the definition of the entropy differential: \(\dd{Q}\) is not an exact differential but \(\dd{Q} / T = \dd{S}\) is. So, we express it using the first law of thermodynamics: \(\dd{Q} = \dd{E} + P \dd{V}\). 
We assume the internal energy \(E\) to be a function of the volume \(V\) and of the temperature \(T\), so we will have: 
%
\begin{align}
\dd{E} = \pdv{E}{V} \dd{V} + \pdv{E}{T} \dd{T}
\,,
\end{align}
%
which we can substitute into the expression for the entropy differential: 
%
\begin{subequations}
\begin{align}
\dd{S} &= \frac{1}{T} \qty(\pdv{E}{V} \dd{V} + \pdv{E}{T} \dd{T} + P \dd{V})  \\
&= \qty(\frac{1}{T} \pdv{E}{V} + \frac{P}{T}) \dd{V} 
+ \frac{1}{T} \pdv{E}{T} \dd{T}  \\
&= \pdv{S}{V} \dd{V} + \pdv{S}{T} \dd{T}
\,,
\end{align}
\end{subequations}
%
so we have identified the partial derivatives of the entropy. By Schwarz's lemma, we then have the equality: 
%
\begin{subequations}
\begin{align}
\pdv[2]{S}{T}{V} &= \pdv[2]{S}{V}{T}  \\
\pdv{}{T} \qty(\frac{1}{T} \pdv{E}{V} + \frac{P}{T})&=
\pdv{}{V} \qty(\frac{1}{T} \pdv{E}{T})  \\
\frac{1}{T} \qty(\cancelto{}{\pdv[2]{E}{T}{V}} + \pdv{P}{T} ) 
- \frac{1}{T^2} \qty(\pdv{E}{V} + P) &=
\cancelto{}{\frac{1}{T} \pdv[2]{E}{V}{T}}  \\
 \pdv{E}{V} &= T \pdv{P}{T} - P 
\,,
\end{align}
\end{subequations}
%
but we can turn the derivatives with respect to \(V\) with ones with respect to \(\rho \propto V^{-1}\), by 
%
\begin{align}
\pdv{}{V} = \pdv{\rho }{V} \pdv{}{\rho } = - \rho^2 \pdv{}{\rho }
\,,
\end{align}
%
so after dividing through by \(\rho \) we find: 
%
\begin{align}
\rho \pdv{E}{\rho } &= -\frac{T}{\rho } \pdv{P}{T} + \frac{P}{\rho }  \\
&= - \frac{P}{\rho } \pdv{\log P }{\log T} + \frac{P}{\rho }  \\
\rho \pdv{E}{\rho }&= - \frac{P}{\rho } \qty(\pdv{\log P }{\log T} -1 )  \\
\pdv{E}{\log \rho }&= -\frac{P}{\rho } \qty(\chi_{T} - 1)
\,.
\end{align}
%

We used the fact that 
%
\begin{align}
\pdv{}{\log x} = \pdv{x}{\log x} \pdv{}{x}= x \pdv{}{x} 
\,.
\end{align}

Now, we can write the first law of thermodynamics for the specific energy density \(E(\rho, T)\): 
%
\begin{align}
\dd{Q} &= \dd{E} - \frac{P}{\rho^2} \dd{\rho }  \label{eq:heat-differential}\\
&= \pdv{E}{\log \rho } \dd{\log \rho } 
+ \pdv{E }{\log T} \dd{\log T} - \frac{P}{\rho}  \dd{\log \rho }  \\
&= \qty(- \frac{P}{\rho } \qty(\chi_{T} -1) - \frac{P}{\rho }) \dd{\log \rho } + \pdv{E}{\log T} \dd{\log T}  \\
&= - \frac{P \chi_T }{\rho } \dd{\log \rho } + \pdv{E}{\log T} \dd{\log T} 
\,,
\end{align}
%
which in the adiabatic (\(\dd{Q} = 0\)) case reduces to 
%
\begin{align}
\Gamma_3 - 1 \overset{\text{def}}{=} \pdv{\log T }{\log \rho } &= - \frac{P \chi_{T}  }{\rho } \pdv{\log T}{E} = - \frac{P \chi_T }{\rho T c_V}  \\
&=- \frac{1}{\rho } \times \underbrace{\frac{P \chi_T}{T}}_{\pdv{P}{T}} \times \underbrace{\frac{1}{c_V}}_{\pdv{T}{E}}  \\
&= - \frac{1}{\rho } \pdv{P}{E}
\,,
\end{align}
%
where we used the fact that \(c_V = \pdv*{E}{T}\). This can also be written as 
%
\begin{align}
\pdv{E}{P} = \frac{1}{\rho (\Gamma_3 -1)}
\,.
\end{align}

If we drop the hypothesis of adiabaticity, we can study the variation with respect to time of \(Q\), both when writing \(E = E(\rho , T)\) and when writing \(E = E(\rho , P)\). 
In the first case we can use equation \eqref{eq:heat-differential}, ``dividing through by \(\dd{t}\)'' (more formally, applying the differential covector equation to the vector \(\partial_{t}\)), after some manipulation we can bring out a factor \(T \pdv*{E}{T}\) to get: 
%
\begin{align} \label{eq:heat-variation-temperature-dependence}
\dv{Q}{t} = T \pdv{E}{T} \qty(\dv{\log T}{t} + \frac{\rho \pdv{E}{\rho } - \frac{P}{\rho }}{T \pdv{E}{T}} \dv{\log \rho }{t})
\,,
\end{align}
%
and similarly if we express \(E = E(\rho, P)\) we find 
%
\begin{align}\label{eq:heat-variation-pressure-dependence}
\dv{Q}{t} = P \pdv{E}{P} \qty(\dv{\log P}{t} + \frac{\rho \pdv{E}{\rho } - \frac{P}{\rho }}{P \pdv{E}{P}} \dv{\log \rho }{t})
\,.
\end{align}

We can simplify these two expressions by recalling some results from before, plus two more expression we now derive for the coefficients \(\Gamma_1\) and \(\Gamma_3 -1\): we assume adiabaticity, and get
%
\begin{align}
0=\dd{S} &= \frac{1}{T} \qty(\pdv{E}{\rho } \dd{\rho } + \pdv{E}{P } \dd{P} - \frac{P}{\rho^2} \dd{\rho })  \\
0&= \qty(\rho \pdv{E}{\rho } - \frac{P}{\rho } ) \dd{\log \rho }   + P \pdv{E}{P} \dd{\log P}  \\
\Gamma_1 = \left. \dv{\log P}{\log \rho } \right\vert_{\text{ad}} &= \frac{\frac{P}{\rho } - \rho \pdv{E}{\rho }}{P \pdv{E}{P}}
\,
\end{align}
%
and 
%
\begin{align}
0 = \dd{S} &= \frac{1}{T} \qty(\pdv{E}{\rho } \dd{\rho } + \pdv{E}{T} \dd{T} - \frac{P}{\rho^2} \dd{\rho })  \\
0&= \qty(\rho \pdv{E}{ \rho } - \frac{P}{\rho }) \dd{\log \rho } + T \pdv{E}{T} \dd{\log T}  \\
\Gamma_3 -1 = \left. \dv{\log T}{\log \rho } \right\vert_{\text{ad}} &= \frac{\frac{P}{\rho } - \rho \pdv{E}{\rho }}{T \pdv{E}{T}}
\,,
\end{align}
%


The identifications are: 
%
\begin{align}
\dv{Q}{t} = T \underbrace{\pdv{E}{T}}_{c_V} \qty(\dv{\log T}{t} + \underbrace{\frac{\rho \pdv{E}{\rho } - \frac{P}{\rho }}{T \pdv{E}{T}}}_{- (\Gamma_3 - 1) } \dv{\log \rho }{t})
\,,
\end{align}
%
and
%
%
\begin{align}
\dv{Q}{t} = P \underbrace{\pdv{E}{P}}_{1/ \rho (\Gamma_3 -1)} \qty(\dv{\log P}{t} + \underbrace{\frac{\rho \pdv{E}{\rho } - \frac{P}{\rho }}{P \pdv{E}{P}}}_{- \Gamma_1 } \dv{\log \rho }{t})
\,.
\end{align}

Finally, we substitute in the equation of energy conservation: 
%
\begin{align}
\dv{Q}{t} = \varepsilon_{\text{eff}} - \pdv{L}{m}
\,,
\end{align}
%
to find the equations for the evolution of the energy and pressure:
%
\boxalign{
\begin{subequations}
\begin{align}
  \pdv{\log P }{t} &= \Gamma_1 \pdv{\log \rho}{t} + \frac{\rho}{P} (\Gamma_3 - 1) \qty(\varepsilon_{\text{eff}} - \pdv{L}{m})  \label{eq:log-pressure-energy-conservation}\\
  \pdv{\log T }{t} &= (\Gamma_3 - 1) \pdv{\log \rho}{t} + \frac{1}{c_V T} \qty(\varepsilon_{\text{eff}} - \pdv{L}{m}) \label{eq:log-temperature-energy-conservation}
\end{align}
\end{subequations}}

\subsection{Linear perturbation theory}

Say we have a solution for these equations, we look at linear perturbations of them.
This makes sense: the main solution is basically static on the pulsation time-scales.

The perturbed model is \(f = f(m)\),  the unperturbed one is \(f_0(m)\).
The Lagrangian perturbation is \(\delta f (m, t) = f(m, t) - f_0(m, t)\).

Let us consider specific cases for \(f\): the radial displacement is \(\delta r (m, t)\). The position of the layer at time \(t\) is \(r = r_0 + \delta r\).

We can write:
%
\begin{equation}
  r = r_0 \qty(1 + \frac{\delta r}{r_0} ) = r_0 (1+ \zeta)
  \,,
\end{equation}
%
where we define \(\zeta = \delta r / r_0 \).

In general the fractional perturbation \(\delta f / f_0\) is assumed to be \(\ll 1 \). So, \(\delta f / f_0 \sim \delta_f / f\). Formally, we only consider terms which are of first order in either perturbed function.
We will insert expressions which are functions of perturbations of all our variables, and thus get linear differential equations.

\subsubsection{Properties of Lagrangian perturbations}

In general for a Lagrangian perturbation we have the following useful properties: 
\begin{enumerate}
  \item we can use the properties of derivatives: \(\delta (f^n) = n f_0^{n-1} \delta f\);
  \item we can use the properties of logarithmic derivatives:
  %
  \begin{align} \label{eq:lagrangian-perturbation-logarithmic}
  \frac{ \delta \qty( \prod f_i)}{\prod f_i} = \sum \frac{ \delta f_i}{f_i}
  \,;
  \end{align}
  %
  \item \(\delta \) commutes with partial derivation. 
\end{enumerate}

\subsubsection{Continuity equation}

Let us try the continuity equation, substituting in \(r = r_0 (1+\zeta)\) and \(\rho = \rho_0 (1 + \delta \rho / \rho_0)\).

\begin{align}
  \pdv{r}{m} &= \frac{1}{4 \pi r^2 \rho }  \\
  \pdv{}{m} \qty(r_0 (1+\zeta)) &=
  \frac{1}{4 \pi r_0^2} \qty(1+\zeta)^{-2} \qty(1 + \frac{\delta \rho}{\rho_0})^{-1}
\end{align}

and we use \((1+x)^n \approx 1 + nx\) plus the zeroth order equation: \(\pdv*{r_0 }{m} = 1 / 4 \pi r_0^2 \rho_0\). 
With these, we find:
%
\begin{equation}
  4 \pi \rho_0^2 \qty(\pdv{r_0}{m} (1+\zeta) + r_0 \pdv{\zeta}{m})
  = (1-2 \zeta) -\frac{\delta \rho}{\rho_0}
\end{equation}

We can collapse the equation into:
\boxalign{
\begin{align} \label{eq:linearized-continuity}
  \frac{\delta \rho}{\rho_0} =
  - 3 \zeta - 4 \pi r_0^3 \rho_0 \pdv{\zeta}{m}
\end{align}}

or, the density perturbation is proportional with a negative constant to the radial perturbation, plus a term proportional to \(\pdv*{\zeta}{m}\).
If there is a positive gradient of radial perturbation, the corresponding layer expands.

\subsubsection{Momentum conservation}

Let us also perturb the momentum conservation equation; the unperturbed solution will be at hydrostatic equilibrium, so \(\pdv*[2]{r_0}{t} = \pdv{r_0 }{t} =0 \), which means 
%
\begin{align} \label{eq:hydrostatic-equilibrium}
\pdv{P_0 }{m} = - \frac{Gm}{4 \pi r_0^{4}}
\,.
\end{align}

Substituting in we find that, to linear order: 
%
\begin{align}
\pdv[2]{r}{t} &= - 4 \pi r^2 \pdv{P}{m} - \frac{Gm}{r^2} \\
\pdv[2]{}{t} \qty(r_0 \qty(1 + \zeta )) 
&= 
- 4 \pi \qty(r_0 (1+z))^2 \pdv{}{m} \qty(P_0 \qty(1 + \frac{ \delta P }{P_0 })) - \frac{Gm}{r_0^2 (1 + \zeta^2)} 
\\
r_0 \pdv[2]{\zeta }{t} 
&=
- 4 \pi r_0^2 \qty(1 + 2 \zeta )
\qty(\pdv{P_0 }{m} \qty(1 + \frac{ \delta P}{P_0 }) + P_0 \pdv{}{m} \qty(\frac{ \delta P}{P_0 }))
- \frac{GM}{r_0^2} \qty(1 - 2z) 
\\
\frac{r_0}{4 \pi r_0^2} \pdv[2]{\zeta }{t} &= - \qty(1 + 2 \zeta )
\qty(\pdv{P_0 }{m} \qty(1 + \frac{ \delta P}{P_0 }) + P_0 \pdv{}{m} \qty(\frac{ \delta P}{P_0 }))
+ \qty(1 - 2z) \pdv{P_0 }{m} \marginnote{Used equation \eqref{eq:hydrostatic-equilibrium}.} \\ 
\begin{split}
\frac{1}{4 \pi r_0} \pdv[2]{\zeta }{t} &= 
- \qty(\pdv{P_0 }{m} \qty(1 + \frac{ \delta P}{P_0 }) + P_0 \pdv{}{m} \qty(\frac{ \delta P}{P_0 })) + \pdv{P_0 }{m} + \\
&\phantom{=}\ - 2\zeta \qty(\pdv{P_0 }{m} \qty(1 + \frac{ \delta P}{P_0 }) + P_0 \pdv{}{m} \qty(\frac{ \delta P}{P_0 }))
- 2 \zeta \pdv{P_0 }{m} 
\end{split} \marginnote{Split the terms into different orders of \(\zeta \)} \\
\begin{split}
\frac{1}{4 \pi r_0} \pdv[2]{\zeta }{t} &= 
- \pdv{}{m} \qty(\delta P ) + \\
&\phantom{=}\ 
- 4 \zeta \pdv{P_0 }{m}
- 2\zeta \pdv[]{}{m} \qty(\delta P )
\end{split} \\
r_0 \dv[2]{\zeta }{t} &= -4 \pi r_0^2 \qty( \pdv{}{m} (\delta P) + 4 \zeta \pdv{P_0 }{m}) \marginnote{Neglected a second order term} 
\,.
\end{align}

The final equation then looks like 
%
\boxalign{
\begin{align} \label{eq:linearized-momentum-conservation}
r_0 \pdv[2]{\zeta}{t} = - 4 \pi r_0^2 \qty(
  \underbrace{P_0 \pdv{}{m} \qty( \frac{ \delta P}{P_0 }) + 
  \frac{ \delta P}{P_0 } \pdv{P_0 }{m}}_{ \pdv{ \delta P}{m} } + 4 \zeta \pdv{P_0 }{m}
)
\,,
\end{align}}
%
and we can now give a physical interpretation of the various terms.  

We have a term \(-16 \pi r_0^2 \zeta \pdv*{P_0}{m} = 4 \zeta Gm /r_0^2\).
This, by itself, is a force moving the system away from equilibrium: the equation with only that term on the RHS is precisely like a harmonic repulsor, \(\ddot{\zeta} = \omega^2 z\) with 
%
\begin{align}
\omega^2 = \frac{4Gm}{r_0^3}
\,.
\end{align}

This term is of geometric origin: as the layer moves outwards it expands, and the expansion is favoured by the decrease in the gravitational potential and the corresponding increase in pressure due to the increase of the area of the layer. 

The equation with only the other term looks like 
%
\begin{align}
\ddot{\zeta} = - 4 \pi r_0 \pdv{ \delta P }{m}
= - 4 \pi r_0  \qty( P_0 \pdv{}{m} \qty(\frac{ \delta P }{P_0 }) + \frac{ \delta P}{P_0 } \pdv{P_0 }{m})
\,.
\end{align}

\todo[inline]{According to the slides, the action of this term is to be split in two: for the second, a restoring force towards equilibrium, since if the pressure decreases as the layer expands then the force is inward, and a more vague interpretation for the first bit, stating that ``a non uniform variation of \(\delta P\) from a layer to the next generates a change in the pressure gradient'': but it \emph{is} a change in the pressure field\dots}

\subsubsection{Energy conservation}

Let us also consider the expression for the time derivative of \(\log P\) coming from the energy conservation equation: equation \eqref{eq:log-pressure-energy-conservation}; the perturbed equation for the time derivative of \(\log T\) (equation \eqref{eq:log-temperature-energy-conservation}) is analogously derived.  
Besides \(\rho \) and \(P\), we also perturb the adiabatic exponents and the \(\dv*{Q}{t} = \epsilon_{\text{eff}} - \pdv*{L}{m}\) term.

After difficult manipulations we get back an equation which relates the changes in density and pressure to the change in energy: we manipulate until we get something which is similar to the original equation: 
%
\boxalign{
\begin{align}
\pdv{}{t} \qty(\frac{ \delta P }{P_0 }) &= 
\Gamma_{1,0} \pdv{}{t} \qty(\frac{ \delta \rho }{\rho_0 })
+ \frac{\rho_0 }{P_0 } \qty(\Gamma_{3, 0} -1) \delta \qty(\epsilon_{\text{eff}} - \pdv{L}{m}) \label{eq:perturbed-linearized-energy-log-pressure}\\ 
\pdv{}{t} \qty(\frac{ \delta T }{T_0 }) &= 
(\Gamma_{3,0} - 1) \pdv{}{t} \qty(\frac{ \delta \rho }{\rho_0 })
+ \frac{1}{c_V T} \delta \qty(\epsilon_{\text{eff}} - \pdv{L}{m}) \label{eq:perturbed-linearized-energy-log-temperature}
\,,
\end{align}}
%
where we note that the first index of the adiabatic exponents denotes \emph{which exponent it is}, while the second indicates that it is the unperturbed value.

\todo[inline]{The last \(T\) in the temperature equation is not unperturbed nor perturbed...?}

\subsubsection{Luminosity equation}

In the radiative case with the diffusion approximation we can perturb the luminosity equation, \eqref{eq:luminosity-equation}. It is much more convenient not to calculate \(\delta L\) but \(\delta L  / L_0 \) instead: this allows us to use the logarithmic derivative properties of the perturbation; also, since 
%
\begin{align}
T^{3} \pdv{T}{m} = T^{4} \pdv{\log T}{m}
\,
\end{align}
%
we use the latter expression, which is more convenient. 
So we have \(L = \prod f_i\), with 
%
\begin{align}
f_{i} = \qty{- \frac{64 \pi^2 ac}{3}, r^{4}, \kappa_{R}^{-1}, T^{4}, \pdv{\log T}{m}}
\,.
\end{align}

Now we can apply the rule given in equation \eqref{eq:lagrangian-perturbation-logarithmic}: we find
%
\begin{equation}
  \frac{\delta L}{L_0} = 4 \zeta + 4 \frac{\delta T}{T_0} - \frac{\delta \kappa_R}{\kappa_{R,0}} + \qty(\pdv{\log T}{m})^{-1}_{0} \pdv{}{m} \qty( \frac{\delta T}{T_0 })
  \,,
\end{equation}
%
where we used the simplification 
%
\begin{align}
\delta \qty(\pdv{\log T}{m}) &= \delta \qty(\frac{1}{T} \pdv{T}{m})  \\
&= \delta \qty(\frac{1}{T}) \pdv{T_0 }{m} + \frac{1}{T_0 }  \delta \qty(\pdv{T}{m})  \\
&= - \frac{ \delta T}{ T_{0}} \pdv{T_0 }{m} + \frac{1}{T_0 } \pdv{ \delta T }{m }  \\
&= \pdv{}{m} \qty(\frac{ \delta T}{T_0 }) \marginnote{Inverse application of the product rule}
\,.
\end{align}

The last step is to assume a certain dependence of the Rosseland mean opacity on the temperature and density: specifically, it is a ``Kramers-like'' expression, given by 
%
\begin{align}
\kappa_{R} \propto \rho^{n} T^{-s}
\,,
\end{align}
%
which can be substituted into our expression: the proportionality factor does not matter, and we get additional temperature and density terms: 
%
\boxalign{
\begin{align}
\frac{ \delta L}{L_0 } 
= 4 \zeta + (4+s) \frac{\delta T}{T_0} - n \frac{ \delta \rho }{\rho_0 } + \qty(\pdv{\log T}{m})^{-1}_{0} \pdv{}{m} \qty( \frac{\delta T}{T_0 })
\,,
\end{align}}
%


In the end, we have a set of four linear PDE equations (written as a 5-equation system).

These describe implicitly how the properties of the star change over time.

Pulsation usually affects mostly the outer layers of a star.

% Moving on to

\section{Adiabatic oscillations}

Exploiting the adiabatic approximation we will get the  Linear Adiabatic Wave Equation (LAWE): a single equation which summarizes the 4 and can be solved explicitly.

Recall the heat transfer equation \eqref{eq:heat-transfer-effective}: 
if we suppose that each layer does not lose nor gain heat, \(\dv*{Q}{t} =0\), this implies that \(\delta (\varepsilon_{\text{eff}} - \pdv*{L}{m} )=0\).

Is this approximation justified? The term multiplying \(\delta (\varepsilon_{\text{eff}} - \pdv*{L}{m} )\) in the perturbed energy equation \eqref{eq:perturbed-linearized-energy-log-pressure} is \(\rho (\Gamma_3-1)/P  = \chi_T / (c_V T)\).
Usually \(\chi_T \sim 1\), while the density perturbation term is multiplied by \(\Gamma_1 \sim 1\).

This term,  \( \chi_T / (c_V T) \delta (\varepsilon_{\text{eff}} - \pdv*{L}{m} )\), is of the order \(1/\tau_{\text{th}}\), the thermal time scale of this layer, while the term before, \(\Gamma_1 \pdv*{}{t} (\delta \rho / \rho)\), is of the order \(1/\tau_{\text{dyn}}\), the dynamical time scale.

\todo[inline]{Is this just because the first term contains a time derivative while the second one does not?}

Therefore, we neglect the second part.
This only works for the star as a whole, not for single layers.
There are stellar layers which are \emph{strongly} non-adiabatic (driving layers). We will need some non-adiabatic theory to explain how pulsations \emph{start}.

So, the energy conservation equations become 
%
\begin{align}
\frac{ \delta P}{P} = \Gamma_1 \frac{ \delta \rho }{\rho }
\qquad \text{and} \qquad
\frac{ \delta T }{T} = \qty(\Gamma_3 -1) \frac{ \delta \rho }{\rho }
\,,
\end{align}
%
which we can substitute into the momentum conservation equation \eqref{eq:linearized-momentum-conservation} to find 
%
\begin{align}
r_0 \pdv[2]{\zeta }{t} = 
- 4 \pi r_0^2 \qty(
P_0 \pdv{}{m} \qty(\Gamma_1  \frac{ \delta \rho}{\rho })
+ \Gamma_1 \frac{ \delta \rho }{\rho } \pdv{P_0}{m}
+ 4 \zeta \pdv{P_0 }{m}
)
\,,
\end{align}
%
and now we can use the continuity equation \eqref{eq:linearized-continuity} which gives us an expression for \(\delta \rho  / \rho \): inserting it we get 
%
\begin{align}
\begin{split}
r_0 \pdv[2]{\zeta }{t} &= 
- 4 \pi r_0^2 \bigg[
P_0 \pdv{}{m} \qty(\Gamma_1  \qty(-3\zeta - 4 \pi r_0^3 \rho_0 \pdv{\zeta }{m})) + \\
&\phantom{=}\ + \Gamma_1 \qty(- 3\zeta - 4 \pi r_0^3 \rho_0 \pdv{\zeta }{m}) \pdv{P_0}{m}
+ 4 \zeta \pdv{P_0 }{m}
\bigg]
\end{split}  \\
\label{eq:step0-LAWE-derivation-substituted}
\begin{split}
r_0 \pdv[2]{\zeta }{t} &= 4 \pi r_0^2 \bigg(
\underbrace{(3 \Gamma_1 - 4)\zeta \pdv{P}{m}}_{\Circled{1}} 
+ 
\underbrace{4 \pi r_0^3 \Gamma_1 \rho \pdv{\zeta }{m} \pdv{P}{m} }_{\Circled{2}}
+ \\
&\phantom{=}\ 
+ \underbrace{3 P \pdv{}{m} \qty(\Gamma_1 \zeta )}_{\Circled{3}} 
+ \underbrace{4 \pi P \pdv{}{m} \qty(\Gamma_1 r^3 \rho \pdv{\zeta }{m})}_{\Circled{4}}
\bigg)
\,,
\end{split}
\end{align}
%
and now we can can manipulate the terms inside the parentheses; we start to drop the zero indices, any quantity not being \(\delta \)'d is meant to be unperturbed.
The first and third terms are respectively given by  
%
\begin{align}
\Circled{1} = 
\pdv{P}{m} \zeta \qty(3 \Gamma_1 - 4) =
\hlc{pink}{\zeta \pdv{}{m} \qty((3 \Gamma_1 - 4) P)} \hlc{teal}{-3 \zeta P \pdv{\Gamma_1}{m}} \marginnote{Backwards derivative of a product}
\,
\end{align}
%
and 
%
\begin{align} \label{eq:step1-LAWE-derivation}
\Circled{3} = 
3 P \pdv{}{m} \qty(\Gamma_1 \zeta ) = 3P \Gamma_1 \pdv{\zeta }{m} \hlc{teal}{+ 3 P \zeta \pdv{\Gamma_1 }{m}}
\,,
\end{align}
%
so we can see that the highlighted terms cancel.
Also, the other term in equation \eqref{eq:step1-LAWE-derivation} can be rewritten using the continuity equation \eqref{eq:continuity}: 
%
\begin{align}
3P \Gamma_1 \pdv{\zeta }{m} 
= \hlc{lightgray}{12 \pi r^2 \rho  \Gamma_1 P \pdv{\zeta }{m} \pdv{r}{m}}
\,,
\end{align}
%
so we can see that if we expand the fourth term in \eqref{eq:step0-LAWE-derivation-substituted} we find a thing that is equal to it: 
%
\begin{align}
\Circled{4} = 
4 \pi P \pdv{}{m} \qty(r^3 \times  \Gamma_1  \rho \pdv{\zeta }{m})
= \hlc{lightgray}{4 \pi P \times 3 r^2 \pdv{r}{m} \Gamma_1 \rho \pdv{\zeta }{m}}
+  \hlc{lime}{4 \pi P r^3 \pdv{}{m} \qty(\Gamma_1 \rho \pdv{\zeta }{m})}
\,,
\end{align}
%
so we will have twice that contribution in the final result.

For now then, we have shown that 
%
\begin{align}
\Circled{1} + \Circled{3} + \Circled{4} = 
\hlc{pink}{\zeta \pdv{}{m} \qty((3 \Gamma_1 - 4) P)} +
2 \times \hlc{lightgray}{12 \pi P r^2 \pdv{r}{m} \Gamma_1 \rho \pdv{\zeta }{m}}
+ \hlc{lime}{4 \pi P r^3 \pdv{}{m} \qty(\Gamma_1 \rho \pdv{\zeta }{m})}
\,.
\end{align}

Now then, the full equation reads 
%
\begin{align}
\begin{split}
r_0 \pdv[2]{\zeta }{t} &= 
4 \pi r^2 \zeta \pdv{}{m} \qty((3\Gamma_1 -4) P) 
+ \hlc{red}{16 \pi^2 r_0^5 \Gamma_1 \rho \pdv{\zeta }{m} \pdv{P}{m}} + \\
&\phantom{=}\ 
\hlc{red}{+ 96 \pi^2 P r^4 \pdv{r}{m} \Gamma_1 \rho \pdv{\zeta }{m}
+ 16 \pi^2 P r^5 \pdv{}{m} \qty(\Gamma_1 \rho \pdv{\zeta }{m})}
\end{split}
\,,
\end{align}
%
and we can notice a certain similarity between the highlighted terms: consider the expression 
%
\begin{align}
\pdv{}{m} \qty(16 \pi^2 \Gamma_1 P \rho r^{6} \pdv{\zeta }{m})
\,,
\end{align}
to which we can apply the general expression, which holds for nonzero differentiable functions of a certain variable \(x\) (and if it is interpreted as a limit, even if the functions go to 0): 
%
\begin{align}
\pdv{}{x} \qty(\prod_i f_i) 
= \qty(\prod_i f_i) \sum_i \frac{1}{f_i} \pdv{f_i}{x}
\,,
\end{align}
%
so 
%
\begin{align}
\frac{\displaystyle\pdv{}{m} \qty(16 \pi^2 \Gamma_1 P \rho r^{6} \pdv{\zeta }{m})}{\displaystyle 16 \pi^2 \Gamma_1 P \rho r^{6} \pdv{\zeta }{m}} &= 
\frac{1}{P} \pdv{P}{m} 
+ \frac{1}{\Gamma_1 \rho \pdv*{\zeta }{m}} \pdv{}{m} \qty(\Gamma_1 \rho \pdv{\zeta }{m})
+ \frac{1}{r^{6}} \pdv{r^{6}}{m}  \\
&= \frac{1}{P} \pdv{P}{m} 
+ \frac{1}{\Gamma_1 \rho \pdv*{\zeta }{m}} \pdv{}{m} \qty(\Gamma_1 \rho \pdv{\zeta }{m})
+ \frac{6}{r} \pdv{r}{m} 
\,,
\end{align}
%
so 
%
\begin{align}
\begin{split}
\frac{1}{r} \pdv{}{m} \qty(16 \pi^2 \Gamma_1 P \rho r^{6} \pdv{\zeta }{m}) &= 
\hlc{red}{16 \pi^2 r^{5} \Gamma_1 \rho \pdv{\zeta }{m} \pdv{P}{m} +} \\
&\phantom{=}\ 
\hlc{red}{+ 16 \pi^2 r^{5} P \pdv{}{m} \qty(\Gamma_1 \rho \pdv{\zeta }{m}) + 96 \pi^2r^{4} \Gamma_1 \rho \pdv{\zeta }{m} P \pdv{r}{m}}
\,.
\end{split}
\end{align}

Now, we can finally write the simplest form of the LAWE:
%
\boxalign{
\begin{align}
r \pdv[2]{\zeta}{t} =
4 \pi r^2 \zeta \pdv{}{m} \qty((3 \Gamma_1 - 4)P)+
\frac{1}{r} \pdv{}{m} \qty(16 \pi^2 \Gamma_1 P \rho r^6 \pdv{\zeta}{m} )  
\,,
\end{align}}
%

Next time, we will decompose: \(\zeta(m, t) = \eta(m) e^{i \sigma t}\) with a constant \(\sigma\): putting this into the LAWE we simplify the exponentials and get the space dependent form of the LAWE.

The LAWE is a Storm-Liouville equation.

\end{document}
