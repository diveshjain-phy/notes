\documentclass[main.tex]{subfiles}
\begin{document}

\section*{Tue Dec 03 2019}

% (BOFGN)

In the treatment of dust driven winds, we must consider grains of different sizes as different fluids, which are dynamically coupled to the gas.
To do this, we have to implement some simplifying assumptions:
\begin{enumerate}
    \item a fixed radius $a$ of dust grains
    \item a radial profile of flow velocity for the grains $u(r)$
    \item the existence of a drag force $f_{\text{drag}}(r)$: this models the effective drag on the dust due to its coupling to the gas.
\end{enumerate}

In this case the equation for the momentum can be written as

\begin{equation}
u \dv{u}{r} = -\frac{GM}{r^2}+Q_{rp}\pi a^2 \frac{L}{c m_d 4\pi r^2}-\frac{f_{\text{drag}}}{m_d}
\,,
\end{equation}
%
where \(m_d\) is the mass of a single grain. 

The acceleration of a dust grain is given by 
%
\begin{align}
g _{\text{rad}} = \frac{\kappa \mathcal{F}}{c}
\,,
\end{align}
%
where \(\mathcal{F} = L / 4 \pi r^2\) is the radiative flux, \(\sigma = \pi a^2 Q_{rp}\) is the cross section for the absorption of radiation and \(\kappa = \sigma / m_d\) is the opacity per unit mass.

The thermal speed of sound is given by
\begin{equation}
\frac{m a_{th}^2}{2} = \frac{1}{2} \mu m_H a^2_{th} =k_b T
\,,
\end{equation}
we want to study flows for which the drift velocity $w_{\text{drift}}=u-v$ is much higher than this $a_{th}$; in this definition
\(u\) is the velocity of the dust, \(v\) the one of the gas.

In this case the drag force, postulated to be
\begin{equation}
    f_{drag}=\pi a^2\rho w \sqrt{w_{drift}^2+a_{th}^2}
\end{equation}
reduces to $f_{drag}=\pi a^2\rho w_{drift}^2$.
Imposing now that at infinity the velocity gradient is null, we have
\begin{equation}
    0=Q_{RP}\pi a^2 \frac{L}{c m_d 4\pi r^2}-\frac{f_{drag}}{m_d}
\end{equation}
that leads to an expression for the drift velocity
\begin{equation}
    w_{drift}=\sqrt{Q_{RP}\frac{L}{4\pi r^2\rho c}}
\end{equation}
and remembering that $\rho$ scales as $\dot M$ in the mass loss equation,
\begin{equation}
    w_{drift}=\sqrt{Q_{RP}\frac{L v}{\dot M c}}
\end{equation}
We know from momentum equation
\begin{equation}
    v \frac{\partial v}{\partial r}=-\frac{1}{P}\frac{d P}{d r}-\frac{GM}{r^2}+n_d\frac{f_{drag}}{\rho}= -\frac{1}{P}\frac{d P}{d r}-\frac{GM}{r^2}+n_d\frac{Q_{RP}\pi a^2 v}{\rho 4\pi c}=-\frac{1}{P}\frac{d P}{d r}-\frac{GM}{r^2}(1-\Gamma_d)
\end{equation}
where we defined the corrective factor, $\Gamma_d$ that encodes the pull against gravity driven by the drag force. The physical interpretation of this solution is that radiation transfers momentum to dust, that can transfer it to the gas, reaching a terminal velocity.
We know from previous lectures that mass loss rate is lower for static atmosphere, because the scale factor $H_r$ is very small in this cases, implying an higher density than the hydrostatic case.
From mass loss equation we have
\begin{subequations}
\begin{align}
    \dot M&= 4\pi r^2 \rho v\\
    \dot M_d&=4\pi r^2 n_d m_d u_{dust}
\end{align}
\end{subequations}
This implies a lower limit on the luminosity $L$ in the case $\Gamma_d>1$
\end{document}