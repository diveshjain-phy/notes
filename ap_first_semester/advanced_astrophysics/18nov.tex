\documentclass[main.tex]{subfiles}
\begin{document}

\section*{Mon Nov 18 2019}

Let us come back to 
%
\begin{align}
  g_l = \frac{F_\nu }{c} \frac{n_i}{\rho } \frac{\pi e^2}{m_e c} f \sim \frac{L_{\nu }}{r^2} \frac{n_i}{\rho }
\,.
\end{align}

We can reduce drastically the number of spectral lines we have to account for if we assume that the density of the wind is very low: only lines from the ground state have to be accounted for.

How do we deal with the radiation pressure for an ensemble of lines?
We have the CAK formalism: it gives the estimate 
%
\begin{align}
  g_{L} \sim \qty(\rho^{-1} \dv{v}{r})^{\alpha }
  \sim \qty(\frac{vr^2}{\dot{M}} \dv{v}{r})^{\alpha }
\,,
\end{align}
%
where \(\alpha \) is a parameter quantifying the optical thickness of the line: it goes from \(0\) for an optically thin line to \(1\) for an optically thick line. 
In general we have from the continuity equation \(\dot{M} \propto r^2 \rho v\), which justifies the expression here.

The expression proposed by CAK was \(g_L = g_e M(t)\), where \(g_e\) is the radiative acceleration from continuum phenomena (mostly electron scattering), while the \emph{force multiplier} \(M\) is in the form: 
%
\begin{align}
  M(t)  = k t^{-\alpha } s^{ \delta }
\,,
\end{align}
%
where \(\kappa , \alpha , \delta \) are called \emph{force multiplier parameters}.

The radiative acceleration due to electron scattering is given by 
%
\begin{align}
  g_L (e) = \frac{\kappa _e}{c} \frac{L_{*}}{4 \pi r^2} = \Gamma _e \frac{GM_{*}}{r^2}
\,,
\end{align}
%
where \(\Gamma _e \) is the so-called \emph{Eddington factor}: 
%
\begin{align}
  \Gamma _e = \frac{\kappa _e}{4 \pi c G} \frac{L_{*}}{M_{*}}
\,,
\end{align}
%
which is the luminosity divided by the Eddington luminosity.
The Compton opacity \(\kappa _e\) is given by 
%
\begin{align}
  \kappa _e = \sigma _e \frac{m_e}{\rho }
\,,
\end{align}
%
which is measured in \SI{}{cm^2g^{-1}}. The value \(\sigma _e = \SI{6.65e-25}{cm^2}\) is the Thomson scattering cross section for electrons.

The \(t\) in \(M(t)\) is defined by 
%
\begin{align}
  t \equiv \kappa _e v _{\text{thermal}} \rho \dv{r}{v}
  = \kappa_e \sqrt{\frac{2 k_B T}{m_H}} \rho \dv{r}{v}
\,,
\end{align}
%
and is inversely proportional to the velocity gradient.

The \(s\) in the definition of \(M(t)\) contains information about the degree of ionization: it is 
%
\begin{align}
  s \equiv \frac{\num{e-11} \rho }{m_H W}
\,,
\end{align}
%
where \(W\) is the \emph{geometrical dilution factor}: 
%
\begin{align}
  W(r) = 0.5 \qty(1 - \sqrt{1 - \qty(\frac{R_{*}}{r})^2})
  \sim \qty(\frac{R_{*}}{2r})^2
\,,
\end{align}
%
and now we present a proof for the fact that this encodes the radial dependence of the observed intensity. 
It is a ratio of solid angles: 
%
\begin{align}
  W(r) = \frac{\int _{0}^{\Omega } \dd{\Omega } }{4 \pi }
\,,
\end{align}
%
where the integration limit \(\Omega \) encodes the solid angle subtended by the star.
Assuming simmetry with respect to the azimuthal angle, we can rewrite it as: 
%
\begin{align}
  W (r) = \frac{2 \pi }{4 \pi } \int _{0}^{\theta_1 } \sin \theta  \dd{\theta } 
  = \frac{1}{2} \int _1^{\cos(\theta_1 )} (-\dd{x}) 
  = \frac{1}{2} \qty(1 - \sqrt{1 - (R/r)^2})
\,,
\end{align}
%
so \(W(r)\) is the solid angle fraction subtended by a star with radius \(R\) at a distance \(r\). 

The complete expression for an ensemble of lines is given by 
%
\begin{align}
  g_L = \frac{\kappa _e}{ c} \frac{L_{*}}{4 \pi r^2} k t^{- \alpha } s^{ \delta }
\,,
\end{align}
%

Simulations show that \(M(t)\) decreases with \(t\) with some kind of power law: the approximation \(\log M \sim - \alpha \log t\) is justified. 

Numerical simulations show that \(\alpha \sim 0.5\), independent of temperature.
The \(\delta\) parameter is almost always of the order \(\delta \sim 0.1\).

We also expect a dependence on the metallicity: \(M_n(t_n) = M_n (t_n)_{\odot} (Z/Z_{\odot})\).

A typical velocity gradient is something like 
%
\begin{align}
  v(r) = v_{ \infty } \qty(1 - \frac{r_0 }{r})^{\beta }
\,,
\end{align}
%
with \(\beta \sim 0.7\).
Then, the profile of \(g_L\) can be computed (using the continuity equation): it is in the form 
%
\begin{align}
  g_L \sim r^{-2} \qty(\rho \dv{r}{v})^{-\alpha }
  \sim r^{2(\alpha -1)} \qty(v \dv{v}{r})^{\alpha }
\,,
\end{align}
%
where \(\alpha \sim 0.6\). The \(r\) dependence is something like 
%
\begin{align}
  g_L \sim r^{-0.8} \qty(1 - \frac{r_0 }{r})^{0.21}
\,,
\end{align}
%
which we can plot. 

\end{document}