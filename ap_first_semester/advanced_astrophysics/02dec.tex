\documentclass[main.tex]{subfiles}
\begin{document}

% \section*{Mon Dec 02 2019}

Coming back to concepts from last time: the cross section is given by the product of the geometrical cross section times an adimensional efficiency; we define it both for absorption and scattering. 

\begin{align}
  C^{\text{A, S}} (a, \lambda )
  = \pi a^2 Q^{\text{A, S}} (a, \lambda )
\,,
\end{align}
%
while for the total cross section we have 
%
\begin{align}
  C^{\text{TOT}} (a, \lambda )
  = \pi a^2 \qty(Q^{\text{A}}(a, \lambda ) + Q^{\text{S}}(a, \lambda ))
\,.
\end{align}

We are speaking about dust grains. 
If \(\lambda \gg a\), then the following holds: 
%
\begin{align}
  Q^{A} \propto a \lambda^{-p}   
\,,
\end{align}
%
where \(p\) depends on the grain composition, and approaches \(2\) for increasing \(\lambda \) for any composition.

At infrared wavelengths, we have \(Q^{S} \sim \lambda^{-4}\), so \(Q^{S}\ll Q^{A}\) there. 

These are the wavelengths we should consider: stars with dust driven winds are cool, with \(T _{\text{eff}} \sim \SI{3}{\kilo\kelvin}\), so their Planckian peaks are around \SI{1}{\micro m}.

% The opacity per unit mass is given by 
% %
% \begin{align}
%   \kappa_{l}(a) = \rho^{-1} \int_{a _{\text{min}}}^{a _{\text{max}}}
%    \pi a^2 Q(\lambda , a) \dd{a}
% \,,
% \end{align}
% %
%  \todo[inline]{is this true?}

The values of \(p\) are not the result of any theoretical model, they are observed. 

% Today we will speak more about mean opacities. 

\subsection{Planck mean opacity}

We define 
%
\begin{align}
  Q^{A}_{p} (a, T) = \frac{\int_{0}^{ \infty } Q^{A} (a, \lambda ) B_{\lambda } (T) \dd{\lambda }}{\int_{0}^{\infty} B_{\lambda } (T) \dd{\lambda }}
\,,
\end{align}
%
where we use the Planck function \(B_{\lambda }(T)\); with it we define 
%
\begin{align}
  \kappa_{p} (T) = \frac{1}{\rho } \int_{a _{\text{min}}}^{a _{\text{max}}} Q^{A}_{p} (a, T) \pi a^2 n(a) \dd{a}
\,,
\end{align}
%
which is called the Planck mean absorption coefficient. 
This quantity will enter into the balance between cooling and heating of the single grain. We are effectively doing a mean over energy, since 
%
\begin{align}
 B_{\lambda } \dd{\lambda } = \dd{E}
\,
\end{align}
%
for a blackbody. 

For hydrodynamical calculations we need to account both for scattering and absorption: then we need some geometric considerations, since there is a strong dependence on the scattering angle.
This is the definition of the radiation pressure mean efficiency: 
%
\begin{align}
Q _{\text{rp}} (a ) = \frac{\int_{0}^{ \infty  } \qty(Q^{A} (a, \lambda) + (1- g_{\lambda }) Q^{S} (a, \lambda ) ) F_{\lambda } \dd{\lambda }}{\int_{0}^{ \infty } F_{\lambda } \dd{\lambda }}
\,.
\end{align}
%
Here we are accounting for both the absorption efficiency and the scattering efficiency. Similarly, we can define the radiation pressure mean opacity \(\kappa _{\text{rp}}\): 
%
\begin{align}
  \kappa _{\text{rp}} = \frac{\int_{0}^{ \infty  } \qty(\kappa_{\lambda } + (1-g_{\lambda }) \sigma_{\lambda }) F_{\lambda } \dd{\lambda }}{\int_{0}^{ \infty } F_{\lambda } \dd{\lambda }}
\,,
\end{align}
%
where \(F_{\lambda }\) is the monochromatic flux: if we have spherical symmetry it is given by
%
\begin{align}
  F_{\lambda } = \frac{L_{\lambda }}{4 \pi r^2}
  \,.
\end{align}
%

Do note that the \(F_{\lambda }\) is not generally known \emph{a priori}: it is a solution to the radiative transfer equation in the presence of dust.

The parameter \(g_{\lambda }\) is the \emph{mean cosine} of the scattering angle: if it is equal to 1 we have forward scattering, if it is equal to $-1$ we have backward scattering, while if it is equal to 0 we have isotropic scattering. 

We are shown a plot of \(g_{\lambda } \) in terms of the wavelength: it decreases from \num{.8} to 0 as \(\lambda \) goes from \SI{0.1}{\micro\metre} to \SI{10}{\micro\metre}: less energetic photons have little momentum, so they are less likely to keep moving in the same direction they were before, the isotropic thermal motion of the gas can easily sway them.

The temperaure of the grain is determined by the balance of the heating and cooling rates: heating occurs because of collisions with fast-moving gas particles or because of the absorption of radiation. 

Cooling, on the other hand, occurs because of collisional energy transfer or by emission of thermal radiation. 

We make an approximation: we only consider radiative processes, and estimate the temperature of the dust grain with the radiative equilibrium temperature. 

The radiative equilibrium equation is 
%
\begin{align}
  \int_{0}^{ \infty } \kappa_{\lambda } B_{\lambda }(T_d) \dd{\lambda }
  = \int_{0}^{ \infty } \kappa_{\lambda } J_{\lambda } \dd{\lambda }
\,,
\end{align}
%
where \(\kappa_{\lambda } B_{\lambda } (T_d)\) is the radiative cooling per unit wavelength of the grains, while \(\kappa_{\lambda } J_{\lambda }\) models the radiative heating:
%
\begin{align}
  J_{\lambda } = \frac{1}{4 \pi } \int_{0}^{4 \pi } I_{\lambda } \dd{\Omega } = W(r) B( T _{\text{eff}})
\,,
\end{align}
%
where \(W(r)\) is the geometrical dilution factor: 
%
\begin{align}
  W(r ) = \frac{1}{2} \qty(1 - \sqrt{1 - \qty(\frac{R}{r})^2}) \sim \qty(\frac{R}{2r})^2
\qquad \text{ as } r \gg R
\,,
\end{align}
%
where \(R\) is the radius of the star, \(r\) is our considered radial position.

This allows us to fix \(r\) and compute \(T_d (r)\). 

% The LHS of the balance equation corresponds to the radiative cooling, while the RHS corresponds to the radiative heating. 

In both terms we make the dependence on \(Q_p^{A}\) explicit: 
%
\begin{align}
  \int_{0}^{ \infty } \pi a^2 Q_p^{A} (a, T_d) B_\lambda (T_d) \dd{\lambda } =
  \int_{0}^{ \infty } \pi a^2 Q_p^{A} (a, T _{\text{eff}}) B_\lambda (T _{\text{eff}}) W(r) \dd{\lambda }
\,.
\end{align}

We are assuming that diffusion of heat between gas grains is negligible: we simplify some terms. 
%
\begin{align}
  Q_p^{A} (a, T_d ) \int_{0}^{ \infty } B_\lambda(T_d) \dd{\lambda }
  = W(r) Q_p^{A} (a, T _{\text{eff}}) \int_{0}^{ \infty }
  B_\lambda (T _{\text{eff}}) \dd{\lambda }
\,,
\end{align}
%
and we know that the integral of the Planck function is given by \(\sigma _{\text{SB}} T^{4}\), where \(\sigma \) is the Stefan-Boltzmann constant. In the end then our expression is 
%
\begin{align}
  Q_p^{A} (a, T_d) T_d^{4} = T^{4} _{\text{eff}} Q^{A}(a, T _{\text{eff}}) W(r)
\,,
\end{align}
%
so, far from the star,
%
\begin{align}
  T_d(r) \sim T _{\text{eff}} \qty(\frac{R}{2r})^{1/2} \qty(\frac{Q^{A}_{p} (a, T _{\text{eff}})}{Q^{A}_{p} (a,T_{d})})^{1/4}
\,
\end{align}
%
or 
%
\begin{align}
T_d^{4} Q_p^{A} = T _{\text{eff}}^{4} Q^{A} (a, T _{\text{eff}}) W(r)
\,.
\end{align}
%

\subsection{Condensation radius and dust temperature}

An immediate application is to find the condensation radius: what is the radius at which the temperature becomes low enough so that the grains are not broken up by thermal motion? 

We calculate this by substituting the condensation temperature, which is \(T_c \sim 1 \divisionsymbol  \SI{1.5e3}{K} \), in place of the dust temperature \(T_d\).

Then, the only dependence on the radius is in \(W(r)\): we are still in the far-from-the-star approximation, so we find:
%
\begin{align}
r_c \sim \frac{R}{2} \qty(\frac{T _{\text{eff}}}{T_c})^2
\sqrt{\frac{Q^{A}_p (a, T _{\text{eff}})}{Q^{A}_p (a,  T_c)}}
\,,
\end{align}
%
and this allows us to see that typically the condensation radius is around \(2 \divisionsymbol 4\) star radii, for stars at a few thousand Kelvin.

In order to do this estimate we also need to assume the functional dependence of \(Q^{A} \propto \lambda^{-p} \propto T^{p}\): then we get 
%
\begin{align}
T_d = T _{\text{eff}} W(r)^{\frac{1}{4+p}}
\,,
\end{align}
%
so the condensation radius becomes 
%
\begin{align}
r_c \approx \frac{R_*}{2} \qty(\frac{T_d}{T _{\text{eff}}})^{- \frac{4 + p}{2}}
\,.
\end{align}

Tomorrow we will speak of the combined dust \& wind equation. 

\end{document}