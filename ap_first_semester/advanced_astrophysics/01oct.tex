\documentclass[main.tex]{subfiles}
\begin{document}

\section*{1 October 2019}

Figure 22.8 in some PDF: run of adiabatic, radiation gradients vs \(\log T \).

We compute \(\nabla_{\text{ad}}\) and \(\nabla_{\text{rad}}\) and see whether the region is convective or radiative.

We can move from the Eulerian and Lagrangian formalisms using the continuity equation. In the Eulerian formalism:

\begin{equation}
  m(r) = \int_0^\infty 4 \pi r^2 \rho(x) \dd{x} \,.
\end{equation}

We need several \emph{constitutive equations} for the parameters \(\rho\), \(c_P\) (heat capacity of stellar matter), the opacity \(\kappa\), the nuclear transformation rate \(r_{ij}\) and the rate of generation of nuclear energy \(\varepsilon\). These can all be considered as functions of \(P\), \(T\), \(\mu\).

We define:
%
\begin{equation}
  \mu^{-1} = \sum _{i}  (1 + \nu_i) \frac{X_i}{A_i}
\end{equation}

(CHECK)

The variables \(X\), \(Y\) and \(Z\) represent the abundances of \(\ce{H}\), \(\ce{He}\) and metals, and satisfy \(X+Y+Z=1\).

We may need to know the metal mixture inside \(Z\), but often we can approximate it as the Sun's distribution.

\begin{equation}
  \pdv{X_i}{t} = \frac{m_i}{\rho} \sum _j \qty( r_{ji} - r_{ij})
\end{equation}

\subsection{Classification of stars}

Low mass stars have between \(0.8\) and \(2\) solar masses.
Intermediate mass stars have masses between \(2\) and \(8\) \(M_{\odot}\).
Massive stars have masses of over \(8 M_{\odot}\).
(Add characteristics of these).

\subsubsection{Low-mass star evolution}

See slides for figures. What are Hayashi lines?

\subsubsection{Intermediate-mass star evolution}

\end{document}
