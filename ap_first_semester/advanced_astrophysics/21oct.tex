\documentclass[main.tex]{subfiles}
\begin{document}

\section*{Mon Oct 21 2019}

We were talking about RR Lyrae varibles, classifying them.
In a Bailey diagram we plot variables as points with the coordinates amplitude and period.

We can tell whether a star is a RR Lyr variable, and then we can use it as a standard candle.

We have nonlinear models which accurately predict the light curves.
There are HUMPs, BUMPs, JUMPs and LUMPs in the light curves:
they tell us about the propagation of shock waves in the star.

We also have the Blazhko Effect: when it is not present, the light curve is the same at every cycle; when it is present the light curve changes, it is modulated every cycle.
This implies that the phased light curve scatter plot is very spread, no matter how well we fix the period. 
This is a poorly understood effect, it might have something to do with magnetic fields.

About half of RRab have this effect, very few RRc have it.

In a given Globular Cluster, we look at the distribution of RR Lyr with respect to period: we can see two distinct clusters, corresponding to the fundamental and first harmonic (RRab and RRc respectively).
We can characterize them with respect to the Oosterhoff group (the one of the fundamental): if it is big with respect to the first harmonic it is a type I, if it is comparable it is a type II.

We can distinguish these by making a histogram of the average period of GCs: OoII are clustered around \SI{.65}{d}, while OoI around \SI{.55}{d}. There is a distinct gap between the two. 

Metal-rich Globular Clusters tend to few RR Lyrae, metal poor-ones tend to have more, unless their horizontal branch is very extended to the blue, in which case they may have few RR Lyr. 

The Oosterhoff gap can be seen in metallicity as well. 

The period, theoretically, should only change on evolutionary (i.\ e.\ very long) time scales, however we observe much faster period changes, especially in stars which exhibit the Blazhko effect. 

So far, RR Lyrae have been found in local dwarf galaxies, in the Magellanic Clouds, in M31 (Andromeda
Galaxy), in M33 (Triangulum Galaxy).

RR Lyr beyond the Local Group have not
yet been observed.

\subsection{Classical Cepheids}

They massive stars (\(4 \divisionsymbol 9 M_{\odot}\)) are younger than RR Lyr, typically around \(\num{e7} \divisionsymbol \SI{e8}{yr}\).
Their absolute visual magnitude \(M_V\) can change in a range of typically \num{-2} to \num{-6}. 
Their periods are typically beteen \(\num{.5} \divisionsymbol \SI{135}{d}\). 

They are very important for the \emph{cosmic distance ladder} determination, because of the Period-Luminosity relation, which can be used to measure distances: period is not affected by reddening.

The light curves of Cepheids are characterized by a sharp rize and a shallow fall in magnitude if the star is vibrating on the fundamental, a more symmetric light curve if it is vibrating on the first overtone. 
A second overtone pulsation is rarely found, and it generally has a much lower amplitude. 
In classical Cepheids we also find multimodal pulsators. 

As with RR Lyrae, the amplitude of the pulsation is larger at higher light frequencies such as the UV, and decreases through visible and IR.

Cepheids with periods between \num{6} and \SI{16}{d} we often have a \emph{bump}, which could be caused by resonances between the fundamental and the second overtone or by echoes from the core.

Some Cepheids, like some RR Lyrae, also show the Blazhko effect (very long-lived modulations in the light curve).

\subsubsection{The Period-Luminosity relation}

The fundamental relation we start from is the Stefan-Boltzmann law:
%
\begin{equation}
  L = 4 \pi R^2 \sigma T^4 _{\text{eff}}
\,,
\end{equation}
%
which can be converted into a relation involving the bolometric magnitude: we first of all divide through by the solar values for the parameters, and then take the log: 
%
\begin{align}
\log \qty(\frac{L}{L_{\odot}})= 2 \log \qty(\frac{R}{R_{\odot}}) + 4 \log \qty(\frac{T _{\text{eff}}}{T _{\text{eff}, \odot}})
\,,
\end{align}
%
and we use the definition of bolometric magnitude: 
%
\begin{align}
M_{\text{bol}} - M _{\text{bol}, \odot} = \num{-2.5} \log \frac{L}{L_{\odot}}
\,,
\end{align}
%
so we substitute it in: 
%
\begin{align}
M _{\text{bol}, \odot} -  M_{\text{bol}}  &= \num{2.5} \qty( 2 \log \qty(\frac{R}{R_{\odot}}) + 4 \log \qty(\frac{T _{\text{eff}}}{T _{\text{eff}, \odot}}))  \\
M _{\text{bol}} &= -5 \log \qty(\frac{R}{R_{\odot}})
- 10 \log \qty(\frac{T _{\text{eff}}}{T _{\text{eff}, \odot}}) + \const
\,,
\end{align}
%
and now we can use the period-mean density relation: 
%
\begin{align}
\Pi^2 \overline{\rho} = \mathcal{Q} = \const
\implies 
2 \log \Pi + \log \overline{\rho} = \log \mathcal{Q}
\,,
\end{align}
%
but \(\overline{\rho} \sim M / R^3\), so this becomes 
%
\begin{align}
2 \log \Pi + \log M - 3 \log R - \log \mathcal{Q} &= \const \\
\log R &= \frac{2}{3} \log \Pi  + \frac{1}{3} \log M - \frac{2}{3} \log \mathcal{Q} + \const
\,.
\end{align}
%
We can now substitute this into the other equation; we get 
%
\begin{align}
M _{\text{bol}} + 5 \qty(\frac{2}{3} \log \Pi  + \frac{1}{3} \log M - \frac{2}{3} \log \mathcal{Q} + \const)
+ 10 \log T _{\text{eff}} &= 0 \marginnote{Multiply by \(3/10\)}  \\
\frac{3}{10} M _{\text{bol}} + \log \Pi + \frac{1}{2} \log M - \log \mathcal{Q} + 3 \log T _{\text{eff}} &= \const
\,,
\end{align}
%
where we started dropping the adimensionalizing divisions by the solar values of the parameters, since they are ``constants''.

Now, we make the assumption of a mass-luminosity relation like that of Main Sequence stars: \(\log M _{\text{bol}} = - 8 \log M + \const\), so we substitute \(\log M\) with \(- M _{\text{bol}} / 8\).
The number multiplying \(M _{\text{bol}}\) is then \(3/10 - 1/16 \approx \num{.24}\). So, in the end, we get
%
\begin{equation}
  \log(\Pi ) = \num{-0.24}M _{\text{bol}} - 3 \log(T _{\text{eff}}) + \log(Q) + \const
\,,
\end{equation}
%
which is really useful because the period can be measured precisely, while the \emph{absolute} bolometric magnitude is really difficult.

Since the stars which satisfy the P-L relations have different temperatures we have a certain spread of the P-L relation scatter plot. 
\todo[inline]{Is \(T _{\text{eff}}\) not very much correlated with \(M _{\text{bol}}\)?}

The derivation of this might be asked at the exam.

\section{Non-radial oscillations and astroseismiology}

This is just an introduction, a full course might be given at the PhD level.

We use similar assumptions as before, except we lose the spherical symmetry. Specifically: 
\begin{enumerate}
  \item we consider perturbations to a spherically symmetric equilibrium model;
  \item we make the Cowling approximation: we consider the unperturbed gravitational potential.
\end{enumerate}

The equations describing the oscillations are 
%
\begin{align}
- \frac{1}{r^2} \pdv{}{r} \qty(r^2 \zeta_{r}) - \frac{g}{v_s^2} \zeta_{r} + \qty(1 - \frac{L_\ell^2}{\sigma^2}) \frac{P'}{v_s^2 \rho } &= 0  \\
\frac{1}{\rho } \pdv{P'}{r} + \frac{g}{v_s^2\rho } P' + \qty(N^2-\sigma^2)\zeta_{r} &=0
\,,
\end{align}
%
where: 
\begin{enumerate}
  \item the displacement vector is separated into radial and horizontal components: \(\vec{\zeta} = \zeta_{r} \hat{e}_{r} + \vec{\zeta}_{h}\); 
  \item any perturbed quantity changes in two ways: we write 
  %
  \begin{align}
  \delta f( \vec{r_0} + \delta \vec{r}) = f' (\vec{r_0} ) + \vec{ \delta  r} \cdot \vec{\nabla} f_0 
  \,,
  \end{align}
  %
  \todo[inline]{Does the prime denote a derivative? how are these things defined?} 
  \item \(L_{\ell}\) is the Lamb frequency, or acoustic frequency: \(L_{\ell}^2 = \ell (\ell+1) (v_s / r)^2\)
  \item \(N\) is the Brunt-Väisälä frequency, or bouyancy frequency: 
  %
  \begin{align}
    N^2= -g \qty(\dv{\ln \rho }{r} - \frac{1}{\Gamma_1} \dv{\ln P }{r} )
  \,,
  \end{align}
  %
  where the term inside brackets is called the \emph{Schwarzschild discriminant} \(A(r)\), and it measures the non-adiabaticity of the system, since it is proportional (with a positive constant) to \(\nabla - \nabla_{\text{ad}}\). 
  Recall Schwarzschild's criterion: we have convectively unstable regions if \(A(r)\) is positive. So, \(N^2\) is \emph{negative} in regions which are convectively unstable according to Schwarzschild's criterion.
\end{enumerate}

Our ansatz for the solution is similar to what is done in quantum mechanics: we have a radial part with a quantum number \(n\), and spherical harmonics indexed by the angular order \(\ell \) and the azimuthal order \(m\), with \(\abs{m} \leq \ell\). The temporal periodicity will still look like an oscillating complex exponential, but its frequency will depend on the quantum numbers: so, in the end we will have 
%
\begin{align}
\zeta (\vec{r}) = R_{n} (r) Y_{\ell m} (\Omega) \exp(i \sigma_{n \ell m} t)
\,,
\end{align}
%
where \(R\) and \(Y\) are the radial and angular components of the perturbation, \(\Omega = (\theta , \varphi )\) and \(Y_{\ell m}\) are the spherical harmonics, the solutions \(Y \colon S^{2} \rightarrow \mathbb{C}\) to \(\nabla^2 Y = - \ell (\ell +1) Y \), satisfying \(\partial_{\varphi }Y = im\): these are restrictions on the unit sphere of \(\nabla^2Y = 0\). 

We can approximate the differential equations locally as a single differential equation (by assuming that all the parameters are constant: we do the following change of variables: 
%
\begin{align}
\pdv{\widetilde{\zeta}}{r} &= h \frac{r^2}{v_s^2} \qty(\frac{L_{\ell}^2}{\sigma^2}  -1 )\widetilde{\eta}  \\
\pdv{\widetilde{\eta}}{r} &= \frac{1}{r^2h} \qty(\sigma^2-N^2) \widetilde{\zeta}
\,,
\end{align}
%
which can be combined into 
%
\begin{align}
\pdv[2]{\widetilde{\zeta}}{r} = - k_r^2 \widetilde{\zeta}
\,,
\end{align}
%
with a wavenumber given by
%
\begin{equation}
  k^2_r = - \frac{1}{v_s^2 \sigma^2 } \qty(L_\ell^2- \sigma^2   ) \qty(\sigma^2 - N^2)
  \,.
\end{equation}

We have several families of solutions: 

\begin{enumerate}
    \item \(p\)-modes, mostly radial, pressure waves, high frequency: \(\sigma^2> L^2_{\ell}, N^2\);
    \item \(g\)-modes, mostly horizontal, buoyancy, low frequency, \(\sigma^2 < L^2_{\ell}, N^2\);
    \item \(f\)-modes; without nodes.
\end{enumerate}

We can derive equations for this case similarly to the LAWE.


This is because \(k_r^2\) is negative iff \(\sigma^2\) is either smaller or larger than both \(N^2\) and \(L_\ell^2\); these are respectively \(g\) and \(p\)-modes. If \(\sigma^2\) is between them, we get a real exponential as a solution, which decays quickly.
This is called the \emph{evanescent regime}.

\(p\)-modes are characterized by a \emph{turning point}: below a certain radius, the mode is in the evanescent regime.
For \(g\)-modes we have the opposite behaviour: they only exist in the core, and are evanescent for large radii. 

We can plot \(N\) and \(L_\ell\) as a function of radius.

Mixed modes arise when the evanescent region is thin, therefore the mode can tunnel through.

The frequencies are degenerate in \(m\), but the degeneracy is split when the star is rotating.

In the slide for ``asymptotic behaviour'', the \(y\) axis is frequency.

We analyze the power spectra of stars.

\end{document}