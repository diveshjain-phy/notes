\documentclass[main.tex]{subfiles}
\begin{document}

\section*{Mon Oct 21 2019}

We were talking about RR Lyrae varibles, classifying them.
In a Bailey diagram we plot variables as points with the coordinates amplitude and period.

We can tell whether a star is a RR Lyr variable, and then we can use it as a standard candle.

We have nonlinear models which accurately predict the light curves.
There are HUMPs, BUMPs, JUMPs and LUMPs in the light curves:
they tell us about the propagation of shock waves in the star.

We also have the Blazhko Effect: when it is not present, the light curve is the same at every cycle; when it is present the light curve changes, it is modulated every cycle.
This implies that the phased light curve scatter plot is very spread, no matter how well we fix the period. 
This is a poorly understood effect, it might have something to do with magnetic fields.

About half of RRab have this effect, very few RRc have it.

In a given Globular Cluster, we look at the distribution of RR Lyr with respect to period: we can see two distinct clusters, corresponding to the fundamental and first harmonic (RRab and RRc respectively).
We can characterize them with respect to the Oosterhoff group (the one of the fundamental): if it is big with respect to the first harmonic it is a type I, if it is comparable it is a type II.

We can distinguish these by making a histogram of the average period of GCs: OoII are clustered around \SI{.65}{d}, while OoI around \SI{.55}{d}. There is a distinct gap between the two. 

Metal-rich Globular Clusters tend to few RR Lyrae, metal poor-ones tend to have more, unless their horizontal branch is very extended to the blue, in which case they may have few RR Lyr. 

The Oosterhoff gap can be seen in metallicity as well. 

The period, theoretically, should only change on evolutionary (i.\ e.\ very long) time scales, however we observe much faster period changes, especially in stars which exhibit the Blazhko effect. 

So far, RR Lyrae have been found in local dwarf galaxies, in the Magellanic Clouds, in M31 (Andromeda
Galaxy), in M33 (Triangulum Galaxy).

RR Lyr beyond the Local Group have not
yet been observed.

\subsection{Classical Cepheids}

They are younger than RR Lyr.
They are very important if the \emph{cosmic distance ladder}.

We have a very important Period-Luminosity relation: this can be used to measure distances.

Period is not affected by reddening.

The fundamental relation is 
%
\begin{equation}
  L = 4 \pi R^2 \sigma T^4 _{\text{eff}}
\,,
\end{equation}
%
which can be converted into a relation involving the bolometric magnitude.
In the end,  using the mass luminosity relation of main sequence stars, we have: 
%
\begin{equation}
  \log(\Pi ) = \num{-0.24}M _{\text{bol}} - 3 \log(T _{\text{eff}}) + \log(Q) + \const
\,.
\end{equation}
%

The derivation of this might be asked at the exam.

\section{Non-radial oscillations and astroseismiology}

This is just an introduction, a full course might be given at the PhD level.

We use the same assumptions as before, except we lose the spherical symmetry.

We have several families of solutions: 

\begin{enumerate}
    \item \(p\)-modes, mostly radial, pressure;
    \item \(g\)-modes, mostly horizontal, buoyancy;
    \item \(f\)-modes; without nodes.
\end{enumerate}

We can derive equations for this case similarly to the LAWE.

[Equations]

We get two different velocities of propagation: \(L_{\ell} = \ell (\ell + 1) (v_s / r)^2\) is the Lamb frequency, while \(N^2= -g \qty(\dv*{\ln \rho }{r} - 1/\Gamma_1 \dv*{\ln P }{r} )\) is the Brunt-Väisälä frequency.

We can approximate the differential equations locally as a single differential equation, which is a harmonic oscillator for the perturbation, with wavenumber
%
\begin{equation}
  k^2_r = - \frac{1}{v_s^2 \sigma^2 } \qty(L_\ell^2- \sigma^2   ) \qty(\sigma^2 - N^2)
\,.
\end{equation}

This is negative iff \(\sigma^2\) is either smaller or larger than both \(N^2\) and \(L_\ell^2\); these are respectively \(g\) and \(p\)-modes. If \(\sigma^2\) is between them, we get a real exponential as a solution.
This is called the \emph{evanescent regime}.

\(p\)-modes are characterized by a \emph{turning point}: below a certain radius, the mode is in the evanescent regime.
For \(g\)-modes we have the opposite behaviour: they only exist in the core, and are evanescent for large radii. 

We can plot \(N\) and \(L_\ell\) as a function of radius.

Mixed modes arise when the evanescent region is thin, therefore the mode can tunnel through.

The frequencies are degenerate in \(m\), but the degeneracy is split when the star is rotating.

In the slide for ``asymptotic behaviour'', the \(y\) axis is frequency.

We analyze the power spectra of stars.

\end{document}