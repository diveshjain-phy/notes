\documentclass[main.tex]{subfiles}
\begin{document}

\section*{Fri Nov 08 2019}

One can time a bash command using

\begin{lstlisting}[language=bash]
time [Command]
\end{lstlisting}

\section{Linear systems}

We want to solve a linear system, in the form \(A \vec{x} = \vec{b}\), with unknown \(\vec{x}\). 
How do we solve this numerically?
We can transform our system to an equivalent one, by
\begin{enumerate}
    \item exchanging two rows;
    \item multiplying an equation by a nonzero constant;
    \item adding an equation to another.
\end{enumerate}

These allow us to do Gaussian elimination, and LU decomposition. These are \emph{direct methods}.

Another class is that of \emph{indirect methods}: we start with an \emph{ansatz} and refine it. These are easier to implement, more generally applicable, more efficient if the matrix is sparse. 
They, however, do not always converge.

An example is the Gauss-Seidel method.

\end{document}