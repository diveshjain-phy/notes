\documentclass[main.tex]{subfiles}
\begin{document}

\section*{Thu Nov 14 2019}

\subsection{Gauss-Seidel}

We can rewrite \(\sum _{j} A_{ij} x_{j} = b_i\) as 
%
\begin{align}
  x_{i} = \frac{1}{A_{ii}} \qty(b_{i} - \sum _{j \neq i} A_{ij}x_{j}) 
\,,
\end{align}
%
and the algorithm works by starting with and \emph{ansatz}, updating it with this formula, and iterating.
The update can be written more generally as 
%
\begin{align}
    x_{i}^{n+1} = \frac{\omega }{A_{ii}} \qty(b_{i} - \sum _{j \neq i} A_{ij}x_{j}^{n}) + (1-\omega) x_{i}^{n} 
  \,,
  \end{align}
%
with the \emph{relaxation parameter} \(\omega \).
Do note that \(n\) is not an exponent but an iteration number.

A good choice for \(\omega \) after the \(5\)th iteration: 
%
\begin{align}
  \omega _{\text{opt}} = \frac{2}{1 + \sqrt{1- (\Delta x^{k+p} / \Delta x^{k})^{1/p}}}
\,,
\end{align}
%


\end{document}