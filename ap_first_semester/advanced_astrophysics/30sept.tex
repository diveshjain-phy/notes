\documentclass[main.tex]{subfiles}
\begin{document}

\section*{30 september 2019}
\section{Introduction}

Teachers: Paola Marigo, Michele Trabucchi.

\url{paola.marigo@unipd.it} and
\url{michele.trabucchi@unipd.it}

\subsection{Topics}

They are selected topics in stellar physics.
%
\begin{enumerate}
    \item Stellar pulsations and Astroseismiology (dr. Michele Trabucchi);
    \item stellar winds (dr. Paola Marigo);
    \item final fates of massive \& very massive stars (dr. Paola Marigo).
\end{enumerate}
%
Basics in Stellar Physics: ``astrophysics II'' inside the bachelor's degree in astronomy (second semester). It can be taken as an optional course.

Material:

\begin{enumerate}
    \item \emph{Introduction to stellar winds} by Lamers, Cassinelli.
    \item \emph{Stellar Atmospheres: Theory and observations} (lecture notes from 1996).
\end{enumerate}
%
and more on Paola Marigo's site.

\subsubsection{Stellar Winds}

Moving flows of materials ejected by stars. 20 to \SI{2e3}{km/s}.

See \emph{Bubble Nebula} in Cassiopea, there is a \(45 M_{\odot}\) star ejecting stellar wind at \SI{1700}{km/s}.

Diagram: luminosity vs effective temperature. We see the \emph{main sequence}.
We can also plot the \emph{mass loss rate}, \(\dot{M}>0 \) in solar masses/year.
An other important parameter is \(v_\infty\).

Diagram: mass loss (or gain) rate vs age of star.

Stellar winds affect stellar evolution, the dynamics of the interstellar medium, the chemical evolution of galaxies.

Momentum is approximately injected with \(\dot{M} v\), kinetic energy with \(\frac[i]{1}{2} \dot{M} v^2\).
Within \SI{1e8}{yr} around half of the infalling matter is reemitted.

\subsubsection{Contents}

We will start with the basic theory of stellar winds, and then: \emph{coronal}, \emph{line-driven} and \emph{dust-driven} winds.

\subsection{Final fates of massive \& very massive stars}

Masses over \(10 M_{\odot}\).

\subsection{Stellar oscillations}

\dots see slides.

Material: slides on moodle or Marigo's page.

\begin{enumerate}
    \item \emph{Pulsating stars} by Catelan \& Smith (introductory);
    \item \emph{Theory of stellar pulsation} by Cox (harder).
\end{enumerate}

Written exam, partial exam on stellar pulsation.

\section{Variability in Astronomy}

First observations of variable stars: \(\sim\) 1600, omicron-Ceti. It changes in magnitude by 6 orders of magnitude.

Others are found from the 1600 onwards, but since the XX century the reason is unknown. Is it \emph{rotation}, \emph{eclipses}?

For some it sure are eclipses, but the Cepheids are different. See \(\delta\)-Cephei, asymmetric continuous curve.
What if stars \emph{pulsate}?

The \emph{light curve} is the luminosity curve over time.

We can also look at the \emph{phased} light curve. Of course we need the period:
the phase is

\begin{equation}
    \varphi = \frac{(t-t_0) \operatorname{mod} \Pi}{\Pi}
\end{equation}
%
where \(\Pi\) is the period.
\(E(t) = \lfloor (t-t_0 )/ \Pi \rfloor\) is the epoch.

We can then measure the period, but if the light curve is multiperiodic we can subtract the model from the curve to see if there are additional periods: this is \emph{prewhitening}.

We can also look at the luminosity in Fourier space.

Of course there are issues with observational gaps (day-night, full moon): aliases; accuracy, duration of observations\dots

Also, the period can change in time.

Things have improved a great deal with large-scale surveys and space suveys.

\subsection{Classification}

By variability type: regular, semi-regular or irregular.

By intrinsic variability: extrinsic, external to the star: eclipses, transits, microlensing, rotation; intrinsic: rotation, eclipses (self-occultation), eruptive and explosive variables, oscillations, secular variations (?).

Whether rotation is to be considered intrinsic or extrinsic is a matter of taste.

Oscillations can be classified by several criteria.

The geometry can be  radial (classical pulsators) or non-radial.

The restoring force can be the pressure gradient or the gravitational force (bouyancy, not gravitational waves).

The excitation mechanisms can be different.

The evolutionary phase and mass of the star can also be different.

\section{Summary of stellar structure \& evolution}

Eulerian: properties of a gas are fields, the position is the position of an observer.
To differentiate position with respect to time is meaningless: position is an independent variable. \(f = f(r^i, t)\).

Lagrangian: we follow an element of fluid: \(\dv*{r^i}{t} = v^i \). We can identify univocally these fluid elements.

When treating stellar structure \& evolution, we look at mass layers \(\dd{m}\).  \(f = f(m, t)\).

In the lagrangian case, we use the convective derivative \(\dv*{}{t} = \partial_t + v^i \partial_i\) where \(v^i\) is the velocity defined before.

\subsection{Equations of stellar structure}

We write these in the spherically symmetric case.

The continuity equation is:
%
\begin{equation}
    \pdv{r}{m} = \frac{1}{4 \pi r^2 \rho} \,.
\end{equation}

Momentum conservation is given by:
%
\begin{equation}
    \pdv{P}{m} = - \frac{Gm}{4 \pi r^4}\,,
\end{equation}
%
where

Energy conservation is given by:
%
\begin{equation}
    \dv{L}{m} = \varepsilon - \varepsilon_\nu - \varepsilon_g \,,
\end{equation}
%
where \(L\) is the luminosity, \(\varepsilon\) is the rate of nuclear energy generation per unit mass, while \(\varepsilon_\nu\) is the rate of energy loss due to neutrino emission per unit mass, and \(\varepsilon_g \) is the work done by the gas per unit mass \& time.

The energy transfer equation is:
%
\begin{equation}
    \pdv{T}{m} = - \frac{GmT}{4 \pi r^4 P} \nabla \,,
\end{equation}

where \(\nabla = \pdv*{\log T}{\log P} \) is the temperature gradient, which has contributions from radiation, conduction, convection\dots

With the diffusion approximation, we can write the gradient as
%
\begin{equation} \label{eq:diffusion-approx-gradient}
    \nabla = \nabla_{\text{rad}} = \frac{3}{16 \pi a c G} \frac{\kappa_R L P}{mT^4} \,,
\end{equation}

\begin{greenbox}
  I assume \(c\) is the speed of light, what is \(a\) though?
\end{greenbox}

%
where \(\kappa_R\) is the Rosseland mean opacity, given by
%
\begin{equation}
  \frac{1}{\kappa_{R}} =
  \frac{ \displaystyle\int _{0}   ^{\infty}  \dv{B_\nu}{T} \frac{1}{\kappa_\nu} \dd{\nu}}
  {  \displaystyle \int _{0}   ^{\infty} \dv{B_\nu}{T} \dd{\nu}}
\end{equation}

Substituting in the result in \eqref{eq:diffusion-approx-gradient} we get:
%
\begin{equation}
    L = - \frac{64 \pi^2 a c }{3} r^4 \frac{T^3}{\kappa} \pdv{T}{m}
\end{equation}
%
where \(\kappa\) is a generalized opacity, the harmonic mean of the Rosseland opacity \(\kappa _R \) and the convective opacity \(\kappa_c = 4acT^3 / (3 \rho \lambda_c )\).

\begin{greenbox}
  Where \(\lambda_c\) is\dots  ?
\end{greenbox}

If we need to deal with convection, this defies any simple modeling. There are instability criteria: where is it relevant? This is given by
Ledoux's criterion,
%
\begin{equation}
    \nabla_{\text{rad}} > \nabla_{\text{ad}} - \frac{\chi_\mu}{\chi_T} \nabla_\mu\,,
\end{equation}
%
where:
%
\begin{subequations}
\begin{align}
  \nabla_\mu  &= \dv{\log \mu }{\log P} \\
  \nabla_{\text{ad}}  &= \qty(\pdv{\log T }{\log P} )_{\text{ad}} \\
  \chi_{\mu}  &= \qty(\pdv{\log P }{\log \mu} )_{\rho, T} \\
  \chi_{T}  &= \qty(\pdv{\log P }{\log T} )_{\rho, \mu}
\end{align}
\end{subequations}
%
which are thermodynamic parameters.

\begin{greenbox}
  how are these called? What do they mean?
\end{greenbox}

In the convective core, \(\nabla \approx \nabla_{\text{ad}}\), but outside of it we need something else.

Mixed-length theory model convection with ``bubbles'':

\end{document}
