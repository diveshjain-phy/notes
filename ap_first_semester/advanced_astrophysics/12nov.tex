\documentclass[main.tex]{subfiles}
\begin{document}

\section*{Tue Nov 12 2019}

Last time we estimated the radiation pressure due to one line.

If we know \(\dot{M} \) and the luminosity \(L\), then we can calculate the effective number of  lines: \(N _{\text{eff}} = \dot{M} c^2 / L\).

We define the \emph{efficiency parameter}: 
%
\begin{align}
  \eta = \frac{\dot{M} v_{\infty}}{L / c}
\,.
\end{align}

This can be split into various components due to several factors: we always divide by the total luminosity, and have 
%
\begin{subequations}
\begin{align}
  \eta _{\text{pot}} &= \frac{\dot{E}_{\text{pot}}}{L}  = \frac{\dot{M} G M_{*}}{R_{*}L}  \\
  \eta _{\text{kin}} &= \frac{\dot{E}_k}{L} = \frac{\dot{M} v_{\infty}^2}{2L}  \\
  \eta _{\text{th}} &= \frac{5 \dot{M} R T_w}{2 \mu L} 
  \,,
\end{align}
\end{subequations}
%
and we will need 
%
\begin{align}
    \eta _{\text{moment}} &= \frac{\dot{M} v_{\infty} c}{L}
\,.
\end{align}

This is basically the energy emission budget, \(L\), divided into its contributions.

The efficiency for non-momentum components is very low as compared to momentum efficency: what matters, again, is the transfer of momentum, not the transfer of energy.

We'd expect \(\eta \leq 1\), but actually sometimes the momentum efficiency can be larger: in one case we have \(\eta _{\text{momentum}} \approx 59\).

If across all the spectrum of the star we have completely optically thick absorption, then the momentum of the wind is equal to the momentum of radiation: 
%
\begin{align} \label{eq:completely-optically-thick}
  \dot{M} _{\text{max}} v_{\infty} = \frac{L}{c} 
  \implies \dot{M} _{\text{max}} = \frac{L}{c v_{\infty}}
\,,
\end{align}
%
and typically \(v_{\infty}\) is of the order of 2, 3 times the escape velocity at the photosphere: \(v_{\infty} \approx 3 \sqrt{2 GM / R}\).
For the luminosity we consider \(L = \num{e5} L_{\odot}\).

Then, the maximum mass loss rate estimated by \eqref{eq:completely-optically-thick} is similar to the observed one. The ratio \(\eta _{\text{momentum}} = \dot{M} / \dot{M} _{\text{max}}\) is typically \(0.5 \divisionsymbol 1 \), but for some stars it can be of the order \num{e1} to \num{e2}.

This is called the \emph{single scattering upper limit}: we assume that scattering is isotropic, therefore we'd expect that after the first scattering the photon does not contribute anymore.
This is not actually the case: \emph{multiple scattering} can contribute, enhancing the maximum mass loss rate by a factor \(\tau_{w}\), equal to 
%
\begin{align}
  \tau_{w} = \int _{r_c}^{\infty} \kappa \rho \dd{r} 
\,,
\end{align}
%
so \(\dot{M} _{\text{max, multiple scattering}} =\tau_{w} \dot{M} _{\text{max, single scattering}}\).

This typically gives us enhancement factors of the order \(2\) to \(6\).
The quantity \(\tau _w\) is adimensional, since the units of \(\kappa \) are \SI{}{cm^2/g}.

Even though we have this corrective factor, the efficiency is always expressed with respect to the single scattering cross section.
Multiple scattering theory accounts for all of the increased mass loss rate, even up to \(\tau \sim \num{e2}\).

Now, we derive the expression for the radiative acceleration privided by one line in a moving atmosphere.

We have a unit volume, of \SI{1}{cm^3}, it has a velocity gradient inside it from \(v\) to \(v + \Delta v\), and it will be heated by a monochromatic flux given by \(F_{\nu } = I_{\nu } / (4 \pi r^2)\).

What is the acceleration \(g _{\text{line}}\)?
First we need to know the absorption properties of the medium. 
The absorption coefficient per cubic centimeter of gas is 
%
\begin{align}
  \kappa_{\nu } = \frac{\pi e^2}{m_e c} f n_i \phi (\nu )
\,,
\end{align}
%
where \(\pi e^2/ (m_e c) \approx \SI{2.654e-2}{cm^2/s}\) is the frequency integrated cross section of the classical oscillator (for the electron): 
%
\begin{align}
  \frac{\pi e^2}{m_e c} = \sigma = \int_0^{\infty} \sigma (\nu ) \dd{\nu }
\,;
\end{align}
%
\(f\) is the oscillator strength, which depends on the line (it is a probability);
\(n_i\) is the number density of atoms which can absorb the line, and \(\phi (\nu )\) is the Gaussian profile of the absorption coefficient, centered at the rest frequency \(\nu_0 \), and normalized so that \(\int \phi (\nu ) \dd{\nu } = 1\): therefore the units of \(\phi\) are \SI{}{1/Hz}.

The typical profile function is a \emph{Doppler profile}: its width is of an order based on the thermal velocity of the atoms, which can sometimes be approximated as much less than the wind velocity: so, we apply the \emph{Sobolev approximation}, and estimate \(\phi (\nu ) \sim \delta (\nu - \nu_0 )\). 

So we need to consider lines which do not overlap. 

We get 
%
\begin{align}
  F _{\text{rad}} = \dv{P _{\text{rad}}}{t} 
  = \frac{1}{c} \dv{E _{\text{rad}}}{t}
\,,
\end{align}
%
where \(\dv*{E _{\text{rad}}}{t}\) is the radiative energy absorbed per unit time and volume by the line.

Then we get 
%
\begin{align}
  g _{\text{line}} = \frac{F _{\text{rad}}}{\rho }
  = \frac{1}{c \rho } \dv{E _{\text{rad}}}{t}
\,.
\end{align}
 
We still need to calculate the radiative energy absorbed: for a single optically thick line is \(\dv*{E _{\text{rad}}}{t} = F_{\nu } \kappa _{\text{line}}\).

A more general expression, which however still applies the Sobolev approximation, is 
%
\begin{align}
  g _{\text{rad}} = \frac{F_{\nu_0} \nu_0 }{c} \qty(1 - \exp(- \tau_{\nu_0 } (\mu=1)) ) \dv{v}{r} \frac{1}{c \rho }
\,;
\end{align}
\todo[inline]{What is \(\mu = 1\) about?}
%
here \(F_{\nu_0}\) is the monochromatic flux from the star emitted at the line rest frequency \(\nu_0 \), the quantity 
%
\begin{align}
  \frac{\nu_0}{c} \dv{v}{r} 
\,
\end{align}
%
is the width of the frequency band that can be absorbed, while \(1 - \exp(\dots) \) is the probability that the absorption occurs in our cubic centimeter: there \(\tau \) is the optical depth (integrated up to infinity).

The product of these three terms gives an energy absorption rate \(\dv*{E _{\text{rad}}}{t}\), if we multiply by \(1/c\) we get absorbed momentum, if we multiply by \(1/ \rho \) we get acceleration.

Let us start from optically thin line: they absorb only part of the radiative flux.

Then we get \(\exp(-\tau_{\nu_0 }) \sim 1- \tau_{\nu_0 } \), therefore \(1- \exp(-\tau_{\nu_0 }) = \tau_{\nu_0 } \).
This is proportional to \(\rho \kappa \propto n_i\).

In the end, we get 
%
\begin{align}
  g_{\text{line}} = \frac{F_{\nu }}{c} \frac{n_i}{\rho } \qty(f \frac{\pi e^2}{m_e c}) \sim \frac{L_{\nu }}{r^2} \frac{n_i}{\rho }
\,,
\end{align}
%
where \(L_{\nu }\) is the monochromatic luminosity of the star: we have \(F_{\nu } \sim L_{\nu }/ r^2\).
We can approximate \(n_i / \rho \) as a constant with respect to the radius: therefore, we have \(g _{\text{line}} \propto L_{\nu } / r^2\).

\end{document}