\documentclass[main.tex]{subfiles}
\begin{document}

\section*{Mon Oct 28 2019}

\section{Stellar Winds}

With Paola Marigo, now.

One should be able to follow this part even without a strong background in stellar astrophysics.

Bubble Nebula in Cassiopeia: a 45 \(M_{\odot}\) star is ejecting mass at \SI{1.7e6}{m\per\second}.

Some important quantities: we introduce 

\begin{enumerate}
    \item \(\dot{M}\) is the mass loss rate: 
    %
    \begin{equation}
      \dot{M} = - \dv{M}{t} >0
    \,;
    \end{equation}
    %
    \item \(v_{\infty}\) is the terminal wind velocity, in the limit of radial infinity.
\end{enumerate}

The gas initially escapes from the star at low (subsonic, \SI{1}{km/s}) velocity; then it is accelerated.
It is accelerated, and in the far field when no more forces are acting on it it approaches \(v_{\infty}\).
We describe it with a \emph{velocity law}: \(v(r)\), and physically since the force is always radially outward we have 
%
\begin{equation}
  \dv{v}{r} > 0
\,
\end{equation}
%
for any \(r\). A typical law is something like: 
%
\begin{equation}
  v(r) = v_0 + \qty(v_{\infty}-v_{0} ) \qty(1 - \frac{R_{*}}{r})^{\beta }
\,,
\end{equation}
%
where \(\beta \approx 0.8\), and \(R_{*}\) is the radius of the photosphere (where, from infinity, we have an opacity of \(\tau = 1\)).

In an H-R diagram we can plot the mass loss rate.

\todo[inline]{The relation \(\dot{M} (M)\) seems to be something like a power law: what is it?}

We have Massive, Cool Luminous and Solar-type stars. 

The momentum input can come either from a force, like radiation pressure; or from heating.

We have:
\begin{enumerate}
    \item coronal winds: driven by gas pressure;
    \item line driven winds: driven by radiation pressure on highly ionized atoms, O and B stars;
    \item dust driven winds: due to radiation pressure on dust grains, solid particles, which are very opaque.
\end{enumerate}

\todo[inline]{Are ionized atoms more opaque to radiation?}

We will not consider winds driven by pulsation, sound waves and Alfvèn waves (magnetic winds).

We will assume:
\begin{enumerate}
    \item spherical symmetry;
    \item stationarity;
    \item no magnetic fields.
\end{enumerate}

The equation of continuity with the hypothesis of stationarity is given by \(\dot{M} = 4 \pi r^2 \rho (r) v(r) \equiv \const\).

If we differentiate \(\dot{M} \) with respect to \(t\) we get 0 on the LHS, and on the RHS: 
%
\begin{equation}
  0 = 2 \rho v + rv \dv{\rho }{r} + r \rho \dv{v}{r} 
\,,
\end{equation}
%
where we simplified the \(4 \pi \). If we divide by \(\rho v r \) we get: 
%
\begin{equation}
  \frac{2}{r} + \frac{1}{\rho }\dv{\rho }{r} + \frac{1}{v} \dv{v}{r} =0     
\,,
\end{equation}
%
and then this gives us 
%
\begin{equation}
  2\log(r)^{\prime } + \log(v)^{\prime } + \log(\rho)^{\prime } =0
\,,
\end{equation}
%
where the prime denotes derivatives with respect to \(r\).
The gradients of the velocity and density are related.

The force per unit volume is given by 
%
\begin{equation}
  F = \rho \dv{v}{t} 
\,,
\end{equation}
%
and if we divide by \(\rho \) we get the force per unit mass: 
%
\begin{equation}
  f = \frac{F}{\rho } = \dv{v}{t}  
\,.
\end{equation}

Under stationarity (\(\partial_t = 0\)), we have: 
%
\begin{equation}
  \dv{v}{t} = v(r) \dv{v}{r} 
\,.
\end{equation}
%

The conservation of momentum gives us the Euler equation: 
%
\begin{equation}
  v \dv{v}{t} = - \frac{1}{\rho } \dv{P}{r} - \frac{GM}{r^2} + f(r)
\,,
\end{equation}
%
where \(f(r)\) is a generic unspecified external force, which we assume to be \emph{outward} (no dissipative effects!).

\begin{bluebox}
    The projection of \(\nabla_{\mu }T^{\mu \nu }= 0\) along the flow of the fluid \(u^{\nu }\), which is a timelike Killing vector field, gives us the conservation of the energy: the 1st law of thermodynamics. 
\end{bluebox}

%
\begin{equation}
  \dv{Q}{t} = \dv{u}{t} + P \dv{\rho^{-1}}{t} 
\,,
\end{equation}
%
where \(Q\) is the specific heat, \(u \) is the specific internal energy.

The internal energy, for an ideal gas, scales linearly with the temperature: 
%
\begin{equation}
  u = \frac{3}{2} \frac{k_B T}{\mu m_u} = \frac{3}{2} \frac{R T}{\mu }
\,,
\end{equation}
%
where \(\mu \) is the mean molecular weight, while \(m_u\) is the atomic unit of mass: \(R = k_B / m_u\).

We will also assume that the gas pressure follows the ideal gas law: 
%
\begin{equation}
  P = \frac{k_B T \rho }{\mu m_u} = \frac{RT \rho }{\mu }
\,.
\end{equation}
%

All time derivatives can be written as \(\dv{}{t} = v \dv{}{r} \).

We define 
%
\begin{equation}
  q(r) = \dv{Q}{r} 
\,,
\end{equation}
%
the heat input or loss per unit mass per unit distance in the wind. Inserting in the previous equation, we get: 
%
\begin{equation}
  q = \frac{3}{2} \frac{R}{\mu } \dv{T}{r} + P \dv{\rho^{-1}}{r} 
\,.
\end{equation}

These can be incorporated in the global energy equation: 
%
\begin{equation} \label{eq:energy-conservation-differential} 
  \dv{}{r} \qty(\frac{v^2}{2} + \frac{5}{2} \frac{RT}{\mu }- \frac{GM}{r}) = f(r) + q(r)
\,,
\end{equation}
%
where the expression inside the brackets is the total internal energy. The terms are called A, B, C.
This clarifies the claim from before: we can change the internal energy by applying a force.

\todo[inline]{Was the claim from before not the reverse?}

Equation \eqref{eq:energy-conservation-differential} can be written in integral form, by integrating from a generic radius \(r_0 \), we could choose the photospheric radius. 

Near the photosphere \(\abs{C} \gg A+B\), so  \(e(r_0 ) \approx GM / r_0 < 0 \).

At large radii \(B, C \to 0\): \(e(r_{\infty}) \approx v_{\infty}^2 /2\).

If we integrate from \(r_0 \) at the photosphere to infinity then we get: 
%
\begin{equation}
  \frac{v^2_{\infty}}{2} = - \frac{GM}{r_0 } + \int _{r_0 }^{\infty} f(r) + q(r) \dd{r} 
\,,
\end{equation}
%
so the sum of the two integrals must be large enough for the gas to escape the gravitational well.

\todo[inline]{Can't this be made stronger by simply swapping the equality with a less than sign, and integrating not only from \(r_0 \) but from any radius?}

Now we will treat isothermal winds. 
These calculations were first done by Parker.

The assumption of constant temperature already gives us \(T(r) \equiv T\), and then we do not need the energy equation: there must be an energy input.

\todo[inline]{So we do not consider the energy equation because energy \emph{is not conserved?}}

This is a good model for low \(\dot{M} \), which does not affect the total mass of the star.

Tomorrow we will discuss full solutions in this case.

Now, let us talk about the corona.
It is very hot: it goes to \(\num{e6}\) degrees,  but with very low density.
We see this experimentally by seeing highly ionized 
elements like \(\ce{Fe}^{+5 \divisionsymbol 13}\).

Why is this the case? We \emph{do not know}. Magnetism?

\end{document}