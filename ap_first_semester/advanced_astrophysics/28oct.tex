\documentclass[main.tex]{subfiles}
\begin{document}

% \section*{Mon Oct 28 2019}

\section{Introduction}

% With Paola Marigo, now.

One should be able to follow this part even without a strong background in stellar astrophysics.

Bubble Nebula in Cassiopeia: a 45 \(M_{\odot}\) star is ejecting mass at \SI{1.7e6}{m\per\second}.

Some important quantities: we introduce 

\begin{enumerate}
    \item \(\dot{M}\) is the mass loss rate: 
    %
    \begin{equation}
      \dot{M} = - \dv{M}{t} >0
    \,;
    \end{equation}
    %
    \item \(v_{\infty}\) is the terminal wind velocity, in the limit of radial infinity.
\end{enumerate}

The gas initially escapes from the star at low (subsonic, \(\sim\)\SI{1}{km/s}) velocity; then it is accelerated.
It is accelerated, and in the far field when no more forces are acting on it it approaches \(v_{\infty}\).
We describe it with a \emph{velocity law}: \(v(r)\), and physically since the force is always radially outward we have 
%
\begin{equation}
  \dv{v}{r} > 0
\,
\end{equation}
%
for any \(r\). A typical law is something like: 
%
\begin{equation}
  v(r) = v_0 + \qty(v_{\infty}-v_{0} ) \qty(1 - \frac{R_{*}}{r})^{\beta }
\,,
\end{equation}
%
where \(\beta \approx 0.8\), and \(R_{*}\) is the radius of the photosphere (where, from infinity, we have an opacity of \(\tau = 1\)).

In an H-R diagram we can plot the mass loss rate using color: we see that it increases when going up the main sequence, and is also high in the RGB.
\todo[inline]{Probably the thing is that the mass loss rate increases through the later stages of stellar evolution\dots}


\todo[inline]{The relation \(\dot{M} (M)\) seems to be something like a power law: what is it?}

The momentum input can come either from a force, like radiation pressure (line driven winds and dust driven winds) or from heating.

Stellar winds can be characterized by their temperature, measured with respect to the effective temperature of the star; velocity, measured with respect to the escape velocity, and density: 
\begin{enumerate}
  \item late type\footnote{``Late type'' means cooler than the Sun.} supergiant stars have \emph{cold},  \emph{slow} and \emph{dense} winds;
  \item luminous hot stars have \emph{cold}, \emph{fast} and \emph{dense} winds;
  \item cool dwarfs and giants have \emph{hot}, \emph{fast} and \emph{tenuous} winds.
\end{enumerate}

We can also characterized them by their driving mechanisms:
\begin{enumerate}
    \item coronal winds: driven by gas pressure;
    \item line driven winds: driven by radiation pressure on highly ionized atoms, O and B stars;
    \item dust driven winds: due to radiation pressure on dust grains, solid particles, which are very opaque.
\end{enumerate}

% \todo[inline]{Are ionized atoms more opaque to radiation?}
For line-driven winds we have a relations between luminosity and wind momentum: it is a powerlaw. 
Specifically, what is measured is \(\dot{M}_{ \infty } \sqrt{R / R_{\odot}}\) versus the visual magnitude. 

We will not consider winds driven by pulsation, sound waves and Alfvèn waves (magnetic winds).

\section{Wind structure equations}

We will assume:
\begin{enumerate}
    \item spherical symmetry;
    \item stationarity;
    \item no magnetic fields.
\end{enumerate}

The equation of continuity with the hypothesis of stationarity is given by \(\dot{M} = 4 \pi r^2 \rho (r) v(r) \equiv \const\).

If we differentiate \(\dot{M} \) with respect to \(r\) we get 0 on the LHS (since, by continuity, the mass loss rate across any layer is constant), and on the RHS: 
%
\begin{equation}
  0 = 2 \rho v + rv \dv{\rho }{r} + r \rho \dv{v}{r} 
\,,
\end{equation}
%
where we simplified the \(4 \pi \). If we divide by \(\rho v r \) we get: 
%
\begin{equation}
  \frac{2}{r} + \frac{1}{\rho }\dv{\rho }{r} + \frac{1}{v} \dv{v}{r} =0     
\,,
\end{equation}
%
and then this gives us 
%
\begin{equation}
  2\log(r)^{\prime } + \log(v)^{\prime } + \log(\rho)^{\prime } =0
\,,
\end{equation}
%
where the prime denotes derivatives with respect to \(r\).
So, the gradients of the velocity and density are related.

The force per unit volume is given by 
%
\begin{equation}
  F = \rho \dv{v}{t} 
\,,
\end{equation}
%
and if we divide by \(\rho \) we get the force per unit mass: 
%
\begin{equation}
  f = \frac{F}{\rho } = \dv{v}{t}  
\,.
\end{equation}

Under stationarity (\(\partial_t = 0\)), we have: 
%
\begin{equation}
  \dv{v}{t} = v(r) \dv{v}{r} 
\,.
\end{equation}
%

The conservation of momentum gives us the Euler equation: 
%
\begin{equation} \label{eq:momentum-conservation}
  v \dv{v}{r} = \underbrace{- \frac{1}{\rho } \dv{P}{r}}_{f_P} \underbrace{- \frac{GM}{r^2}}_{f_g} + f(r)
\,,
\end{equation}
%
where \(f(r)\) is a generic unspecified external force, which we assume to be \emph{outward} (no dissipative effects!), while \(f_P\) and \(f_g\) are respectively the force due to the pressure gradient and to gravitation.

% \begin{bluebox}
%     The projection of \(\nabla_{\mu }T^{\mu \nu }= 0\) along the flow of the fluid \(u^{\nu }\), which is a timelike Killing vector field, gives us the conservation of the energy: the 1st law of thermodynamics. 
% \end{bluebox}
Do note that while \(f_P\) has a minus sign, \(\dv*{P}{r}<0\) so the force due to the pressure gradient is outward. On the other hand, \(f_g<0\). 
The first principle of thermodynamics gives us:
%
\begin{equation}
  \dv{Q}{t} = \dv{u}{t} + P \dv{\rho^{-1}}{t} 
\,,
\end{equation}
%
where \(Q\) is the specific heat, \(u \) is the specific internal energy.

The internal energy, for an ideal gas, scales linearly with the temperature: 
%
\begin{equation}
  u = \frac{3}{2} \frac{k_B T}{\mu m_u} = \frac{3}{2} \frac{\mathcal{R} T}{\mu }
\,,
\end{equation}
%
where \(\mu \) is the mean molecular weight, while \(m_u \approx m_H\) is the atomic unit of mass. Using these, we define the specific gas constant \(\mathcal{R} = k_B / m_u\).

\begin{bluebox}
The mean molecular weight is defined as the average weight of a molecule in atomic mass units: \(\mu = \overline{m} / m_H\), and for a neutral gas it can be calculated as 
%
\begin{align}
\frac{1}{\mu m_H} = \frac{\sum _{j} N_j}{M _{\text{tot}}} =
\sum _{j} \frac{N_j}{N_j m_j} \frac{N_j m_j}{M _{\text{tot}}}
= \sum _{j} \frac{N_j}{N_j A_j m_H} X_j 
\,,
\end{align}
%
so 
%
\begin{align}
\frac{1}{\mu } = \sum _{j} \frac{N_j}{A_j}
\,,
\end{align}
%
where we have used the facts that \(m_j = A_j m_H\), we defined the mass fraction \(X_j = N_j m_j / M _{\text{tot}}\): the fraction of the total mass which is of the species \(j\). 
For an ionized gas the calculation is the same except we need to include a factor \((1+z_j)\), at the numerator in the sum, where \(z_j\) is the atomic number of the element, since for every ionized atom we have an additional particle (the electron).

Using the mass fractions for the Sun we get approximately \(\mu \approx 1.3\) for neutral gas and \(\mu \approx \num{.62}\) for fully ionized gas.
\end{bluebox}

We will also assume that the gas pressure follows the ideal gas law: 
%
\begin{equation}
  P = \frac{k_B T \rho }{\mu m_u} = \frac{\mathcal{R} T \rho }{\mu }
\,,
\end{equation}
%
where we used the fact that \(N = M / \overline{m}\), and \(\overline{m} = \mu m_H\). 

By stationarity, all time derivatives can be written as \(\dv{}{t} = v \dv{}{r} \).

We define 
%
\begin{equation}
  q(r) = \dv{Q}{r} 
\,,
\end{equation}
%
the heat input or loss per unit mass per unit distance in the wind (since \(Q\) is already a specific heat). Inserting this in the previous equation, we get the following expression for the energy equation:
%
\boxalign{
\begin{align}
  q = \frac{3}{2} \frac{\mathcal{R}}{\mu } \dv{T}{r} + P \dv{\rho^{-1}}{r} 
\,,
\end{align}}
%

This can be incorporated, using the momentum conservation as well, into the global energy equation: 
%
\boxalign{
\begin{align} \label{eq:energy-conservation-differential}
  \dv{}{r} \qty(\underbrace{\frac{v^2}{2}}_{\Circled{A}} + \underbrace{\frac{5}{2} \frac{\mathcal{R} T}{\mu }}_{\Circled{B}}\underbrace{- \frac{GM}{r}}_{\Circled{C}}) = f(r) + q(r)
\,,
\end{align}}
%
where the expression inside the brackets is the total internal energy.

\begin{bluebox}
Here is the derivation of the formula: the momentum conservation equation \eqref{eq:momentum-conservation} can be written as: 
%
\begin{align}
\frac{1}{2} \dv{}{r} \qty(v^2) = - \dv{}{r} \qty(\frac{P}{\rho }) + P \dv{}{r} \qty(\frac{1}{\rho }) + \dv{}{r} \qty(\frac{GM}{r}) + f \marginnote{Inverse Leibniz rule on the \(P,\rho \) term.}
\,,
\end{align}
%
so we can substitute in our expression from the energy equation for \(P \dv*{\rho^{-1}}{r}\): we get 
%
\begin{align}
\dv{}{r} \qty(\frac{v^2}{2} + \frac{P}{\rho } - \frac{GM}{r}) = f + q - \frac{3}{2} \frac{\mathcal{R}}{\mu} \dv{T}{r}
\,,
\end{align}
%
but \(P / \rho \) is precisely equal to \(\mathcal{R} T / \mu \), and the molecular weight and the gas constant are constants, so we can bring that term inside the derivative on the LHS in order to get the \(5/2\) factor in the final result.
\end{bluebox}

The terms \(\Circled{A}\), \(\Circled{B}\) and \(\Circled{C}\) describe the different types of energy we can have: \(\Circled{A}\) is kinetic energy, \(\Circled{B}\) is chemical potential energy (enthalpy) while \(\Circled{C}\) is gravitational potential energy.

We can change the total energy by either adding heat (increasing \(q\)) or by adding momentum (increasing \(f\)).

% This clarifies the claim from before: we can change the internal energy by applying a force.

% \todo[inline]{Was the claim from before not the reverse?}

Equation \eqref{eq:energy-conservation-differential} can be written in integral form, by integrating from a generic radius \(r_0 \): a useful choice is often the photospheric radius. 
We introduce the notation \(e (r) = \Circled{A} + \Circled{B}+ \Circled{C}\) for the total energy. 

Near the photosphere \(\abs{\Circled{C}} \gg \Circled{A}+\Circled{B}\), so  \(e(r_0 ) \approx GM / r_0 < 0 \) (this is not obvious theoretically: it is an experimental fact that the gas is slow-moving near the surface).

At large radii \(\Circled{B}, \Circled{C} \to 0\): \(e(r_{\infty}) \approx v_{\infty}^2 /2 > 0\).

If we integrate from \(r_0 \), the radius of the photosphere, to infinity then we get: 
%
\begin{equation}
  \frac{v^2_{\infty}}{2} = - \frac{GM}{r_0 } + \int _{r_0 }^{\infty} f(r) + q(r) \dd{r} 
\,,
\end{equation}
%
so the sum of the two integrals must be large enough for the gas to escape the gravitational well.

% \todo[inline]{Can't this be made stronger by simply swapping the equality with a less than sign, and integrating not only from \(r_0 \) but from any radius?}

\section{Coronal winds}

The corona is very hot: it goes to \(\SI{e6}{K}\),  but with very low density.
We know this experimentally by seeing the emission lines of highly ionized 
elements like \(\ce{Fe}^{+5 \divisionsymbol 13}\).

Why is this the case? We \emph{do not know}. Magnetism?

If the gas particles in the corona have enough kinetic energy to reach terminal velocity and escape the gravitational well, they form what is known as the coronal wind.
This is a good model for the Sun: its wind's asymptotic velocity is around \SI{500}{km/s}, so it reaches the Earth in around \SI{3}{d}. 
For the sun \(\dot{M} \sim \SI{1e9}{kg/s}\):
this means that it loses \SI{.016}{\percent} of its total mass per \SI{10}{Gyr}, which is approximately the lifespan of the Sun. 

\todo[inline]{Calculating \(\dot{M} \times \SI{10}{Gyr} / M_{\odot}\) gives this instead of \SI{.1}{\percent}, as the slides say!}

\subsection{Isothermal winds}

Now we will treat isothermal winds. 
These calculations were first done by Parker.

The assumption of constant temperature already gives us \(T(r) \equiv T\), which acts as our energy equation: there must be an energy input exactly equal to 
%
\begin{align}
q = P \dv{}{r} \frac{1}{\rho }
\,:
\end{align}
%
this is a type of wind which is driven by gas pressure only. 

\todo[inline]{Is this actually the case? it is not explicitly mentioned in the slides, but it seems to be a natural consequence of the hypothesis.}

% \todo[inline]{So we do not consider the energy equation because energy \emph{is not conserved?}}

This is a good model for low \(\dot{M} \), which does not affect the total mass of the star.

Tomorrow we will discuss full solutions in this case.

\end{document}