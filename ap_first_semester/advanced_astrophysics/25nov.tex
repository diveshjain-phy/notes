\documentclass[main.tex]{subfiles}
\begin{document}

% \section*{Mon Nov 25 2019}

\paragraph{The bistability jump}

Now, let us treat the terminal vs escape velocities for O-B stars: we seem to have a linear relation between these,  for each group of stars, where we group them by effective temperature. 

If we plot \(v_{ \infty }/ v _{\text{esc}}\) in terms of \(T _{\text{eff}}\), we get three horizontal regions: 
\begin{enumerate}
  \item under \SI{10}{\kilo\kelvin} the ratio is around \num{.7}: this corresponds to an effective \(\alpha \approx \num{.33}\);
  \item between \SI{10}{\kilo\kelvin} and \SI{21}{\kilo\kelvin} the ratio is around \num{1.3}: this corresponds to an effective \(\alpha \approx \num{.63}\);
  \item over \SI{21}{\kilo\kelvin} the ratio is around \num{2.6}: this corresponds to an effective \(\alpha \approx
  \num{.87}\). 
\end{enumerate}

This is called the \emph{bistability jump}, and it seems to be due to the fact that the strength of different lines, such as those in carbon, is heavily dependent on the ionization level of the gas.

We cannot reproduce all the data with a single \(\alpha \) value, instead we need at least three. Simple CAK theory cannot fully explain the data.

We have the maximum mass loss rate when the equation 
%
\begin{align}
  \dot{M} v_{ \infty } = \tau_{w} \frac{L}{c}   
\,
\end{align}
%
with \(\tau_{w} = 1\) is satisfied. 

We can plot both sides of this equation for \(O\)-stars\footnote{Hey now, you're an O star, get your game on, go play.} and Wolf-Rayet stars of type WNL.
\(O\)-stars' observations agree with the model (in that their mass loss rate is slightly lower than the luminosity), WNL observations seem not to (in that their \(\dot{M} v_{ \infty }\) is higher than \(L/c\)). 

We can make a plot of initial mass vs metallicity and distinghish regions fo the final fates of stars. 
Metallicity characterizes the amount of heavier-than-Helium elements in the star. 

The definition is 
%
\begin{align}
  \text{Metallicity} = \frac{m_{\text{metals}}}{m_{\text{tot}}} = Z
\,,
\end{align}
%
where \(m\) denotes the mass of element of a certain species. \(1 = X + Y + Z\) are the fractional masses of hydrogen, helium and metals respectively, so \(X = m _{\text{H}}/ m _{\text{tot}}\) and so on. 

\subsubsection{Achievements of line driven wind theory}

It correctly predicts \(v_{ \infty } \propto v _{\text{esc}}\) and \(\dot{M} \sim L^{1.5}\).

Unexplained issues are the bistability jump, the momentum problem and the metallicity dependence.

\section{Dust driven winds}

\subsection{General considerations}

Now, we will speak of dust driven winds: we move to the cool and luminous part of the H-R diagram, which is populated by red giants and so on. 

The main characteristics of these stars are: 
\begin{enumerate}
  \item mass between \num{.7} and \(2 M_{\odot}\);
  \item luminosity between \num{e3} and \(\num{e5}L_{\odot}\);
  \item temperature between \num{2} and \SI{3}{\kilo\kelvin};
  \item mass loss rate between \num{e-7} and \(\num{e-4} M_{\odot} / \SI{}{yr}\);
  \item asymptotic velocity between \num{20} and \SI{30}{km/s};
  \item mass of the shells between \num{.1} and \(\num{.6}M_{\odot}\);
  \item pulsation period between \num{e2} and \(\SI{e3}{d}\).
\end{enumerate}


Qualitatively, the driving mechanism is the radiation pressure on dust grains; dust can exist in low temperature regions of the atmosphere, with \(T \sim \SI{e3}{K}\). 
Dust absorbs momentum and is accelerated outwards: it is very opaque. 

The bulk of the outward travelling matter is gas: the dust is a small part, but it is dynamically coupled to the gas (it ``drags it along'', they share momentum). 

A good approximation is a wind with a force \(f \sim r^{-2}\). 
This mechanism critically depends on the dust formation radius: the distance at which the temperature becomes low enough for dust to form. 

Dust grains can absorb photons in the continuum, they are then said to be ``continuum driven'' as opposed to ``line driven''. 
Then, this very much diminishes the role of Doppler shift in making it possible for the radiation to pass through much of the gas uninterrupted. 
Besides, the actual shift is low since the wind is not accelerated to very high velocities.

The wind speeds of cool stars become comparable to the escape speed near the transsonic points, as opposed to what happens for hot stars, for which this happens right after the wind has left the stellar surface.

The typical photons in this process are infrared, on the order of \(\lambda \sim \SI{1}{\micro\metre}\). 

The dust causes significant reddening. We have low \(v _{ \infty }\) and high \(\dot{M}\).

Most of the momentum must be transferred to the wind in the subsonic region to account for the high \(\dot{M}\), while little energy must be transferred in the supersonic region in order for the exit velocity to be low. 

The formation of dust grains must be described with molecular chemistry, and it heavily depends on the ratios of chemical species. 

An example: the Helix Nebula. 

We define the Eddington factor: the ratio of radiative acceleration to gravity. 
In the case of radiation forces on dust, it is 
%
\begin{align} \label{eq:definition-gamma-dust}
  \Gamma_{d} = \frac{\kappa _{\text{rp}}L_{*}}{4 \pi c G M_{*}}
\,,
\end{align}
%
where the \(\kappa _{\text{rp}} \) is called the ``radiation pressure mean opacity'': it measures the capacity of the dust to absorb photons. 

In order to have a wind, we must have \(\Gamma_{d}\) greater than unity after some radius, otherwise the net force on a single particle is inward.

To a good approximation, the transsonic transition corresponds to the point at which \(\Gamma_{d}\) becomes greater than one. 

The plot of our model of \(\Gamma_{d}\) looks something like \(\Gamma_{d} (r) = \Theta (r - r _{\text{sonic}}) \times 1.4\), where \(r _{\text{sonic}} \sim (3 \divisionsymbol 4) r_{*}\), where \(\Theta \) is the Heaviside theta.

\subsection{Necessary conditions}

What are the conditions in order to have a dust driven wind? We first need to form the dust grains: they need to survive against sublimation. 
Then, the dust must be coupled enough to the gas in order to drive it. 

\subsubsection{Condensation radius}

There are two interesting temperatures. The first is the radiative equilibrium temperature \(T _{\text{rad}}\), which is the temperature of the grains: it is the result of the balance between the radiative heating and cooling of the grains. 

The second one is the condensations temperature \(T _{\text{cond}}\), below it the grain becomes a solid. 
If \(T _{\text{rad}} >  T _{\text{cond}}\) we have  immediate sublimation.

\(T _{\text{cond}} \sim \const\) is not a bad approximation; the radiative equilibrium temperature instead decreases moving outwards from the star. 
There is a radius at which they are equal, and for larger radii we have \(T _{\text{rad}} < T _{\text{cond}}\). 
This critical radius is called the dust condensation radius \(r_{d}\). 

Let us suppose that we are at \(r > r_d\). 
The grain then can form, and it can then gain momentum \(h \nu /c\) and energy \(h \nu \) from the Sun's photons, and transfer it to the gas molecules. 
The energy is not really relevant: it is absorbed and then reemitted isotropically. The momentum, on the other hand, is radial.

We can get a lower limit to the mass loss rate \(\dot{M} \gtrsim \num{e-7} M_{\odot} / \SI{}{yr}\) because of the gas-dust coupling condition.
Deriving this is complicated, but qualitatively it is since if there is less mass loss rate there is also less gas density, so there is less transfer. 

\subsubsection{Drift speed and gas-dust coupling}

% Also, we can have an upper limit on the speed of the dust driven wind.
Also, we have an upper limit on the drift speed of a grain through the gas, which is defined as the difference in velocity between dust and gas: \(w _{\text{drift}}= u_{\text{grain}} - v _{\text{gas}} \). If it is too high, collisions with the gas particles will destroy the gas grains. 

The drift speed decreases if there is more dynamical coupling between gas and dust, and the dynamical coupling depends on the density of the gas: the precise relation is
\(w _{\text{drift}} \propto \rho _{\text{gas}}^{-1/2}\). 
The density decreases moving outward, so there is a radius beyond which the collisions exceed the limiting energy at which they are able to destroy the grains, so there is no further increase in the wind speed, since the dust and gas get progressively more decoupled as the density decreases. 

\subsection{Dust grain composition}

The properties of the grains are very important, but studying the processes in grain formation and growth is hard. 

There are both processes of accretion of the gas onto the dust particle and erosion by sputtering: collisions of the gas onto the grain. 

There will be a distribution of grain sizes, typically from \(\num{.05} \divisionsymbol \SI{.1}{\micro\metre}\): we can sometimes make the small grain approximation. 

The composition of the grains depends on the materials in the gas, particularly on the \ce{C/O} (Carbon to Oxygen) ratio (defined as a ratio of number of atoms).

\begin{enumerate}
  \item If \(\ce{C/O} <1\) typically we form silicate grains. The stars with this ratio are called \(M\)-type stars, they are oxygen-rich.
  \item If \(\ce{C/O} >1\) typically we form carbonaceous grains. The stars with this ratio are called \(C\)-type stars. 
  \item If \(\ce{C/O} \sim 1\) the star is called an \(S\)-type star.
\end{enumerate}

Stars in the AGB evolve through these: they start off as \(M\)-type, move through \(S\) type into \(C\)-type.

If we make an infrared color-magnitude diagram, we can see that oxygen-rich stars are redder, while carbon-rich ones are bluer.

The phase we are interested in is called TP-AGB: Thermally Pulsating Asymptotic Giant Branch.

\end{document}
