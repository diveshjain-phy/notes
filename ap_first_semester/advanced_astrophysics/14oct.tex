\documentclass[main.tex]{subfiles}
\begin{document}

\section*{Mon Oct 14 2019}

% Summary of past lectures: we discussed:

% \begin{enumerate}
%     \item the different time scales of stellar evolution, and their impact on the adiabatic approximation;
%     \item easy approximations for stellar pulsation;
%     \item acoustic approximations;
%     \item Ritter's relation;
%     \item linearized structure equations: the adiabatic approximation (arguments for it!), deriving the LAWE;
%     \item examples of solutions to the LAWE and stability conditions;
%     \item the LNAWE: today.
% \end{enumerate}

\subsection{The LNAWE}

First of all, let us recall the linearized equations of stellar structure: 
%
\boxalign{
\begin{align}
\frac{ \delta \rho }{\rho } &= - 3 \zeta - 4 \pi r^3 \rho \pdv{\zeta }{m}  \label{eq:linearized-cont}\\
r \ddot{\zeta} &= - 4 \pi r^2 \qty[\qty(4 \zeta + \frac{ \delta P}{P})\pdv{P}{m} + P \pdv{}{m} \qty(\frac{ \delta P}{P})]  \label{eq:linearized-mom}\\
\pdv{}{t} \qty(\frac{ \delta P}{P}) &= \Gamma_1 \pdv{}{t} \qty(\frac{ \delta \rho }{\rho }) + \frac{\rho }{P} \qty(\Gamma_3 - 1) \delta \qty(\epsilon_{\text{eff}} - \pdv{L}{m})  \label{eq:linearized-E1}\\
\pdv{}{t} \qty(\frac{ \delta T}{T}) &= \qty(\Gamma_3-1) \pdv{}{t} \qty(\frac{ \delta \rho }{\rho }) + \frac{1}{c_V T} \delta \qty(\epsilon_{\text{eff}} - \pdv{L}{m})  \label{eq:linearized-E2}\\
\frac{ \delta L}{L} &= 4 \zeta - n \frac{ \delta \rho }{\rho } + (s+4) \frac{ \delta T}{T} + \qty(\pdv{\log T}{m})^{-1} \pdv{}{m} \qty(\frac{ \delta T}{T}) \label{eq:linearized-RT}
\,,
\end{align}}
%
using which we can derive after long manipulations the Linear Non-Adiabatic Wave Equation, which in its Lagrangian formulation reads: 
%
\begin{align}
\dot{\ddot{\zeta}}
=
4 \pi r \qty(\dot{z} \pdv{}{m} \qty((3 \Gamma_1 - 4)P) 
-  \pdv{}{m} \qty(\rho (\Gamma_3 - 1) \delta \qty(\dv{Q}{t})))
+ \frac{1}{r^2} \pdv{}{m} \qty(16 \pi^2 \Gamma_1 P \rho r^{6} \pdv{\dot{\zeta}}{m})
\,,
\end{align}
%
where we used the fact that \(\dv*{Q}{t} = \epsilon _{\text{eff}} - \pdv*{L}{m}\). On the other hand, the Eulerian formulation is 
%
\begin{align}
\dot{\ddot{\zeta}} = \frac{1}{r \rho } \qty(\dot{\zeta}\pdv{}{r} \qty((3\Gamma_1 -4) P ) - \pdv{}{r} \qty(\rho (\Gamma_3 -1) \delta \qty(\dv{Q}{t})))
+ \frac{1}{r^4 \rho}\pdv{}{r}\qty(r^{4} \Gamma_1 P \pdv{\dot{\zeta}}{r}) 
\,.
\end{align}

\begin{bluebox}
The procedure to derive the LNAWE is as follows: 
\begin{enumerate}
  \item substitute the continuity equation \eqref{eq:linearized-cont} into the \(P, \rho \) form of the energy conservation equation \eqref{eq:linearized-E1}; 
  \item substitute the \(P, \rho \) form of the energy conservation equation \eqref{eq:linearized-E1} into the time derivative of the Eulerian form of the momentum conservation equation \eqref{eq:linearized-mom};
  \item simplify.
\end{enumerate}
\end{bluebox}

\subsection{Solving the LNAWE}

Our ansatz for the LNAWE will still be of the form \(\zeta (r, t) = \eta (r) e^{i \sigma t}\), but now we insert \(\sigma = \omega + i \kappa \): this means we also consider \emph{damped} exponential solutions and \emph{diverging} exponential solutions. 
We want to simplify the exponentials, so we must assume that 
%
\begin{align}
\delta \qty(\dv{Q}{t}) = \delta \qty(\dv{Q}{t}) _{\text{sp}} e^{i \sigma t}
\,.
\end{align}

\todo[inline]{Why do we assume that the heat derivative perturbation is \emph{in phase} with the displacement? Maybe we do not, and the \(_{\text{sp}}\) heat variation is complex?}

With this substitution we get: 
%
\begin{align}
- i \sigma^3 \eta = \frac{i \sigma \eta }{r \rho }
\pdv{}{r} \qty((3 \Gamma_1 -4)P) - \frac{1}{r \rho }
\pdv{}{r} \qty(\rho (\Gamma_3 -1) \delta \qty(\pdv{Q}{t})_{\text{sp}}) 
+ \frac{i \sigma}{r^{4} \rho } \pdv{}{r} \qty(r^{4} \Gamma_1 P \pdv{\eta }{r})
\,.
\end{align}



The time scales for these parameters are \(\omega \sim \omega _{\text{ad}} \sim \tau _{\text{dyn}}\), while \(\kappa \sim 1/ \tau _{\text{th}}\): therefore \(\omega \gg \abs{\kappa } \).

Using this result, we can make some useful \emph{quasi-adiabatic} approximations: in the LNAWE we identify the LAWE operator \(\mathcal{L}\), and replace its application to the wavefunction with the corrisponding eigenvalue: 
%
\begin{align}
\begin{split}
- i \sigma^3 \eta  &= 
-i \sigma \qty(- \frac{1}{r \rho } \pdv{}{r} \qty((3\Gamma_1 -4) P) - \frac{1}{r^{4} \rho } \pdv{}{r} r^{4} \Gamma_1 P \pdv{\eta }{r}) + \\
&\phantom{=}\ 
 - \frac{1}{r \rho } \pdv{}{r} 
\qty(\rho (\Gamma_3 -1) \delta \qty(\pdv{Q}{t})_{\text{sp}})
\end{split} 
\\
- i \sigma^3 \eta &= 
- i \sigma \mathcal{L}(\eta ) 
- \frac{1}{r \rho } \pdv{}{r} 
\qty(\rho (\Gamma_3 -1) \delta \qty(\pdv{Q}{t})_{\text{sp}}) \\
\mathcal{L}(\eta ) - \sigma^2 \eta &= 
\frac{i}{r \sigma  \rho } \pdv{}{r} 
\qty(\rho (\Gamma_3 -1) \delta \qty(\pdv{Q}{t})_{\text{sp}})
\,,
\end{align}
%
so we can see that the eigenvalue of the LAWE operator cannot be \(\sigma^2\) now. We take this equation, multiply it by \(\eta r^2\) and integrate it over the whole star in \(\dd{m}\): we get 
%
\begin{align}
i \sigma^3 \int \eta^2r^2 \dd{m} 
- i \sigma \int \eta \mathcal{L}(\eta) r^2 \dd{m}
= \int \frac{r}{\rho } \pdv{}{r} \qty(\rho (\Gamma_3 -1) \delta \qty(\pdv{Q}{r}) _{\text{sp}}) \eta \dd{m}
\overset{\text{def}}{=} C
\,,
\end{align}
%
where we defined the \emph{work integral} \(C\). 
We only look at the first order terms in \(\kappa \): so we make the approximation \(i \sigma^3 \approx \omega^2 \qty(i \omega - 3 \kappa )\), while (to first order, but also exactly) \(i \sigma = i \omega - \kappa \). We substitute these two, and then make the key manipulation: we substitute \(\mathcal{L}(\eta )\) with \(\omega^2 \eta \). 
The only thing missing is the definition: \(J \overset{\text{def}}{=} \int \eta^2 r^2 \dd{m}\).
So we get 
%
\begin{align}
\omega^2 \qty(i \omega - 3 \kappa ) J 
- (i \omega - \kappa ) \omega^2 J &= C \\
- 3 \kappa + \kappa &= \frac{C}{J \omega^2}
\,,
\end{align}
%


%
\begin{equation}
  \kappa = - \frac{C}{2 \omega^2 J }
\,.
\end{equation}

We make some considerations on the expression of \(C\), integrate by parts, getting:
%
\begin{equation}
  C = \int_M \qty(\frac{\delta T}{T}) \delta \qty(\epsilon _{\text{eff}}- \pdv{L}{m} ) \dd{m}
\,,
\end{equation}

which allows us to study the mechanisms which create perturbations.

The energy of the vibrations comes from the internal thermal energy of the star, which ultimately comes from thermonuclear reactions.

The two terms in \(C\) come respectively from energy generation and transfer.

The \(\epsilon \)-mechanism is about energy generation, which is assumed to be due to nuclear reactions, without considering neutrino processes:
this can happen if the magnitude of the temperature and density perturbations are large enough.

The \(\kappa \)-\(\gamma \)-mechanism: we look at regions where the luminosity gradient and temperature gradient are discordant. This means that the considered stellar layer is absorbing or emitting; this is usually assumed to be happening through free-free interactions (bremsstrahlung and inverse bremsstrahlung).

\end{document}
