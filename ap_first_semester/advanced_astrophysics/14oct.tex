\documentclass[main.tex]{subfiles}
\begin{document}

\section*{Mon Oct 14 2019}

Summary of past lectures: we discussed:

\begin{enumerate}
    \item the different time scales of stellar evolution, and their impact on the adiabatic approximation;
    \item easy approximations for stellar pulsation;
    \item acoustic approximations;
    \item Ritter's relation;
    \item linearized structure equations: the adiabatic approximation (arguments for it!), deriving the LAWE;
    \item examples of solutions to the LAWE and stability conditions;
    \item the LNAWE: today;
\end{enumerate}

We consider the LNAWE: inside of \(\zeta (r, t) = \eta (r) e^{i \sigma t}\) we insert \(\sigma = \omega + i \kappa \): this means we also consider \emph{damped} exponential solutions and \emph{diverging} exponential solutions.

The time scales for these parameters are \(\omega \sim \omega _{\text{ad}}\), while \(\kappa \sim 1/ \theta _{\text{th}}\): therefore \(\omega \gg \abs{\kappa } \).

Using this result, we can make some useful \emph{quasi-adiabatic} approximations: in the LNAWE we identify the LAWE operator \(\mathcal{L}\), and replace its application to the wavefunction with the corrisponding eigenvalue.

We only look at the first order terms in \(\kappa \).
We define the \emph{work integral} \(C\), and then we derive 
%
\begin{equation}
  \kappa = - \frac{C}{2 \omega^2 J }
\,.
\end{equation}

We make some considerations on the expression of \(C\), integrate by parts, getting:
%
\begin{equation}
  C = \int_M \qty(\frac{\delta T}{T}) \delta \qty(\epsilon _{\text{eff}}- \pdv{L}{m} ) \dd{m}
\,,
\end{equation}

which allows us to study the mechanisms which create perturbations.

The energy of the vibrations comes from the internal thermal energy of the star, which ultimately comes from thermonuclear reactions.

The two terms in \(C\) come respectively from energy generation and transfer.

The \(\epsilon \)-mechanism is about energy generation, which is assumed to be due to nuclear reactions, without considering neutrino processes:
this can happen if the magnitude of the temperature and density perturbations are large enough.

The \(\kappa \)-\(\gamma \)-mechanism: we look at regions where the luminosity gradient and temperature gradient are discordant. This means that the considered stellar layer is absorbing or emitting; this is usually assumed to be happening through free-free interactions (bremsstrahlung and inverse bremsstrahlung).

\end{document}
