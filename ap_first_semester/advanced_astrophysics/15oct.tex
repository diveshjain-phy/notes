\documentclass[main.tex]{subfiles}
\begin{document}

\section*{Tue Oct 15 2019}

Today we look at the \(\kappa \)-\(\gamma \)-mechanism.

This deals with the term 
%
\begin{equation}
  \int_M \frac{\delta T}{T} \pdv{\delta L}{m} \dd{m} 
\,,
\end{equation}
%
which keeps account of the regions in which the two terms multiplied in the integrand are discordant.

The mean Rosseland opacity is approximated by a law in the form: 
%
\begin{equation}
  \kappa _R \simeq \overline{\kappa }_R \rho^{n} T^{-s}      
\,,
\end{equation}
%
with \(n \approx 1\), \(s \approx 7/2\) in the case of free-free absorption (inverse \& direct bremsstrahlung) in a non-degenerate, totally ionized gas.

This allows us to relate the perturbations in \(\rho \) to those in \(T\). However, in this law \(\delta T\) is positive iff \(\delta \rho \) is positive: \emph{damping layers}.

Driving layers, however, are present in certain regions: where there is ionization, it provides an additional channel for energy stocking: it is like a \emph{dam} for energy.

The ionization energy is released through mechanical work.

We look at the linearized equations of continuity and radiative transfer for an expression for the gradient of \(\delta L\): in the outer regions of the star we have \(\pdv*{L}{m} \approx 0\). 
In these regions \(\kappa _R \propto \rho^{n}T^{-s}\) with negative \(s\).

There is an opacity bump at \(5<\log(T)< 6.5 \), which was found in the eighties by Simon.
Do note that these are temperatures reached when looking a bit inside the star, not at the surface (although close to it).

There are very luminous \emph{strange modes}, very dim convectively driven modes.

Convective driving is called the \(\delta \)-mechanism.

There also are \emph{stochastically driven} oscillations: they are intrinsically stable.

The pulsation region has boundaries:
for the \(\kappa \)-mechanism:
\begin{itemize}
    \item if the star is too hot, the regions of partial ionization get too close to the surface;
    \item if the star is too cold, they are too far in: in the outer region the pulsation is damped.
\end{itemize}

What is the optimal region? We will only look  at the fundamental mode. We define
%
\begin{equation}
  \phi (m) \equiv \frac{1}{L(\Pi / 2 \pi )}\int_m^M c_V T \dd{m} 
\,,
\end{equation}
%
which represents the thermal balance of a single oscillation.

We are in the helium ionization region: there are no nuclear reactions, so we set energy generation to zero. 
%
\begin{equation}
  \frac{\delta T}{T} = (\Gamma_3 -1 ) \frac{\delta \rho }{\rho } - i \qty(\pdv{\phi }{x} )^{-1} \pdv{}{x} \qty(\frac{\delta L}{L})
\,,
\end{equation}
%
where \(\delta T\) is actually complex, since the perturbations are out of phase.

On the surface, the imaginary term is negligible. In the interior, it is relevant.

The pulsation region is the one in which the ionization region can build up an energy excess. 

Now for some star zoology: RR Lyr\ae.
It is a stage, which lasts no more than \SI{e8}{yr}.

They are classified by a, b, or c according to the shape of the light curve, its amplitude, its period:

\begin{enumerate}
  \item RRa: sharp rise, large amplitude: the fundamental;
  \item RRb: similar to RRa with smaller amplitude, longer period: the fundamental;
  \item RRc: more symmetric light curve, short periods, low amplitudes: they pulsate in the first overtone.
\end{enumerate}

We have also RRd: bimodal, RRe: second overtone.

[Argument for the different amplitudes at different wavelengths: to understand]

[Qualitative part of the lecture.]

\end{document}